\chapter{\label{introduction}Introduction}



\minitoc


\section{Abstract}


\section{First night in Beijing, August 2015}

Three has-been rugby players---Adrian, Kai, and I---waited for Mr Shi to arrive in the upstairs area of the Korean BBQ restaurant in a quiet Peking willow lined street just inside Beijing's East 4th Ring Road.  Adrian, the host of the dinner, was an elder of Chinese rugby.  He was captain of the second graduating class of rugby players the Chinese Agricultural University (CAU), the home of China's first official rugby union program, established in 1990.  I first met Adrian two years earlier through my good friend Kai---the second rugby player at the dinner.  Kai was a more recent graduate of CAU (2007), a former Chinese National rugby team representative, and since graduation, a lawyer in Beijing.  I was the third rugby player at the dinner.  For the nine years prior to that night, I had been coming and going from China for study and work.  Each stint in China involved rugby in some way, and over time I had built a network of friends and colleagues in the Chinese rugby community.

Mr Shi, the guest of honour for whom the three of us were waiting, was a technical producer for Chinese Central Television's Sport Channel, CCTV5.  World Rugby, the international governing body of rugby, had recently given CCTV5 the broadcasting rights to the 2015 Rugby World Cup, to be held imminently in England during September and October 2015.  Mr Shi had been charged with the technical production of the 48-match Tournament, but was completely new to rugby, and so needed help making rugby accessible and understandable for a Chinese audience.\footnote{The Rugby World Cup is the third largest sporting event in world sport, behind the Summer Olympics and the Football World Cup. The broadcast of the 2015 World Cup was the first time international rugby was televised on Chinese national television}  Mr Shi reached out through his network in Beijing and soon tracked down Adrian; Adrian, in turn, tracked down Kai.  Both were well connected to the Chinese rugby community and fluent in English, and so were well placed to assist Mr Shi in the tasks of translating relevant rugby materials and organising expert commentators for the broadcast.

My arrival in Beijing was timely for this project, and Kai was quick to recruit me to join them at dinner.  I was eager to catch up with old friends as well as begin my fieldwork. And so, despite my jet lag, I accepted the invitation and set off to the Korean BBQ restaurant on that first Saturday evening with my notebook and audio recorder (i.e., my mobile phone) in hand.

Adrian naturally held the floor while the three of us waited for Mr Shi to arrive.  He reminisced fondly about his time playing rugby at CAU, as well as his time after graduation playing with the Beijing Devils, a rugby club in Beijing whose members were predominantly expats.

\begin{quotation}
      Rugby was so much fun in those days, not like today (in the professional era of Chinese rugby).  Everyone was just scraping together the money to go on tour, we all payed our own way, sometimes you'd get a bit of help from someone or whatever. It was for the love of the game, not for any other reason.
\end{quotation}

When Mr Shi finally arrived, Adrian continued the nostalgic story telling mode, but naturally shifted his target audience from Kai and me to Mr Shi.  When Adrian began to describe in rich detail the experience of camaraderie between he and his Beijing Devils team mates when they participated in overseas rugby tour, he interrupted his own story with an explanatory aside directed at Mr Shi, accommodating for the fact that Mr Shi was relatively unacquainted with the sport: ``This sport of rugby union, it's actually very mysterious. If you haven't played it yourself you might not know this type of feeling,'' Adrian respectfully suggested to Mr Shi.  ``Because rugby, you know, you're all on the field together, there's body contact between...'' he paused to find the right phrasing,  ``...its a very ``carnal'' type of feeling. Everyone is very close.''  His attempts to enrich his communication by gesticulating had led him to have both of his hands clenched as fists in front of him like they were cradling a rugby ball, a lit cigarette smouldering between the index and middle finger of his right hand.  Adrian concluded by reiterating: ``Its very mysterious,'' and shaking his head as if baffled.  He released his clenched fists to dab the ash from his cigarette into the ashtray in front of him. He took another drag and finally added,``So it means this rugby circle here in China is very tight...'' (a short pause for another dab of his cigarette) ``...but it doesn't mean that this circle is not also not also completely chaotic!''\footnote{Circle (\textit{quanzi} 圈子) is a common way to refer to a social group or community of people}.  The wisdom of Adrians's final punchline was confirmed with a knowing chuckle from all of us, including Mr Shi. Adrian concluded silently, with a long, satisfying drag of his cigarette.

%英式橄榄球这个项目其实特别神秘,没玩过的话您可能不知道这种感觉,因为英式橄榄球么,大家在场上有身体接触,是一种``肉''的感觉,大家互相都特别亲,特别神秘。
%所以在橄榄球这个圈子特别亲, 但这不是说这个圈儿也不乱!

I was both captivated and somewhat shocked by Adrian's monologue. I was not expecting, so early into my fieldwork, to happen upon a declaration in which the link between visceral (i.e., carnal \textit{rou}) sensations associated with on-field joint action and more abstract social processes of interpersonal emotional affiliation (\textit{qin}) and group cohesion of the rugby community (\textit{quanzi}) was so explicitly and spontaneously emphasised.  It was clear that, some fifteen years after he had finished playing, rugby's carnal dimension continued to capture Adrian emotionally---demonstrated by clenched fists and shaking head.  I was also intrigued that the source of this emotional capture was to him at once both very specific (derived from playing together with others on the field) and at the same time ultimately ``mysterious.''

I did not fully realise it at the time, but Adrian's closing caveat concerning the messiness of the Chinese rugby circle also helped guide my observations during the ensuing months of fieldwork.  I came to experience first hand the complexity beneath Adrian's sarcasm.  In the case of the group of professional athletes of the Beijing men's rugby team, with whom I conducted extended ethnographic research, the experience of on-field joint action in rugby was used by athletes as a source for explaining social discord as much as it was cited as a source of cohesion, and social connection and affiliation appeared to co-exist at times with the ``chaos'' of interpersonal competition, bickering, and hierarchy.  In reality, the on-field demands of rugby, while central to their experience, were only one part of the larger social game that the athletes, coaches, and officials I observed were constantly playing.  And such was the improbability of success in rugby's joint action scenarios that the exhilarating carnal sensation described by Adrian was experienced by athletes only occasionally and fleetingly, if at all.  By contrast, off-field social interactions were structured by institutionalised incentives and chronic cultural dispositions that have existed in China well before (and well beyond) the introduction of modern interactional team sports such as rugby union.

Nonetheless, as I peeled back these layers of context in the research setting, I found evidence that on-field experience of joint action in rugby was indeed highly relevant to processes of social affiliation and group membership.  In this dissertation I present ethnographic and experimental evidence that a relationship between joint action and social bonding was resilient to the chaos of rugby in China.

Needless to say, I left that first dinner eager to investigate the sources of Adrian's experience of mysterious carnality and social connection in rugby's joint action.  My next stop was the Temple of God of Agriculture Sports Institute, on Monday morning, where I had organised to conduct ethnographic research with the Beijing Provincial rugby program.

                            \begin{center}
                              * * *
                            \end{center}

\section{Scientific explanations of group exercise}

Competitive team sports, whirling Sufi dervishes, late-night electronic music raves, Maasi ceremonial dances, or the fitness cults of Cross-fit and Soul Cycle---endless examples can be plucked from across cultures and throughout time to exemplify the human compulsion to come together and move together.  How is it possible to explain the prevalence of physiologically exertive and socially coordinated movement in the human record?  In this dissertation I contribute to a scientific understanding of physiologically exertive and socially coordinated movement (hereafter simply ``group exercise'') by way of a focussed study of the social cognition of joint action between professional Chinese rugby players.


%ATQ: something about greater attention to mechanisms and dynamics of movement regulation in multi-agent human cognitive systems.

%\myparagraph{Definition of Group Exercise}

Because physical movement is a metabolically expensive endeavour for all biological organisms, it is justifiable, in an evolutionary sense, only if the benefits somehow outweigh the costs.  It is easy to imagine how physiologically exertive and coordinated group activities would have served important survival functions in our ancestral past, such as hunting, travel, communication, and defence \citep{Sands2010}.  In more recent domains of human history, however, the task of explaining the persistent recurrence of group exercise is  more complicated.  At least since the late Pleistocene era (approx. 500ka), and particularly since the Holocene transition (approx. 11ka) from hunter-gatherer to agricultural, and later industrial and post-industrial societies, group exercise is identifiable in shared cultural practices as varied as religion, organised warfare, music, dance, play, and sport.  Unlike group hunting or defence, the fitness-relevant benefits of these activities are not always as immediate or obvious.

Instead, the prevalence of group exercise in a diverse array of cultural practices must be understood in the context of humans' species-unique evolutionary trajectory, defined by increasingly complex cognitive and cultural capacities, including technical manipulation of extra-somatic materials and ecologies; advanced theory of mind; and information-rich, malleable, and scaleable communication systems \citep{Fuentes2016}.  A theory capable of satisfactorily explaining group exercise within these distinctive evolutionary parameters is yet to be fully formulated \citep{Cohen2017}.

In this dissertation I draw attention to the social function of behavioural coordination in group exercise.  Group exercise contexts invariably involve multiple actors who coordinate behaviours in time and space, usually in pursuit of a shared goal.  I concentrate on the cognitive mechanisms and system dynamics that enable and constrain joint action in group exercise, and draw attention to the social and psychological effects that occur when joint action functions optimally---i.e., when joint action ``clicks.''
The interactive team sport of rugby union serves as a highly suitable case study for studying team click in joint action.  (Contrived environment of high uncertainty). The cultural setting in China adds an extra dimension to my theoretical and empirical contributions.


The core prediction of this dissertation is thus that the widely identifiable phenomenon of ``team click'' mediates a relationship between joint action and social bonding. I present ethnographic and field-experimental evidence that tests and confirms this core prediction, and I evaluate these results in terms of their implications for understanding the proximate cognitive mechanisms, ecological system dynamics, and ultimate evolutionary processes relevant to the anthropology of group exercise.

%2) introduce the research domain of the social cognition of joint action, and draw attention to ways in which a more sophisticated theoretical appreciation of coordination dynamics of human movement could prove useful in formulating testable research hypotheses, and 3) formulate the specific predictions of my thesis and introduce the empirical research designed to test them.  This chapter concludes with an outline of the empirical contributions of this dissertation and suggestions of the theoretical and methodological considerations for the cognitive and evolutionary anthropology of group exercise, which are developed fully in the General Discussion (HYPERLINK Chapter 8).




\subsubsection{The ``social high'' theory of group exercise and social bonding}

In this section I outline existing cognitive and evolutionary accounts of group exercise.  Existing research detailing the immediate physiological, cognitive, and social mechanisms associated with group exercise suggest that group exercise can be responsible for generating a psychophysiological environment conducive to social affiliation and trust.  This evidence lends support to the hypothesis, long speculated by social scientists \citep[see, for example][]{Durkheim1965}, that cultural activities in which group exercise feature foster social cohesion.  It is not yet clear, however, how or whether group exercise uniquely generates social cohesion, or in what ways particular mechanisms vary by activity and culture.

\myparagraph{Proximate: group exercise generates a psychophysiological environment conducive to social affiliation and trust}

The proximate mechanisms of group exercise tend to relate to one of two dimensions: 1) the level of physiological exertion and
2) the level of coordination of movement between co-actors.  It is now understood that strenuous and prolonged physical exercise is modulated by the same neuropharmacological systems (namely, the opioidergic and endocannabinoid systems) responsible for regulating pain, fatigue, and reward \citep{Boecker2008,Raichlen2013}.  Exercise-specific activity of these systems offers a plausible neurobiological explanation for commonly reported sensations of positive affect, anxiety reduction, and improved subjective well-being during and following exercise---extremes of which are popularly referred to as the ``runner's high'' \citep{(Dietrich2004,Boecker2008,Raichlen2012}.  Meanwhile, research in social psychology focusing on the relationship between time-locked behavioural synchrony and processes of self-other merging, social alignment, and affiliation have made a strong case for the social and affective significance of interactive and coordinated movement typical of many group exercise contexts \cite{Wiltermuth2009,Kirschner2010,Reddish2013,Tuncgenc2016}.

Experimental evidence suggests that time-locked coordination of behaviour between two or more individuals in the stable attractor/equilibrium states of either in-phase or anti-phase synchrony is conducive to psychological processes of self-other merging, liking, trust, and psychological affiliation. In a recent review of the synchrony literature, Mogan and colleagues
identify three candidate mediators of the relationship between behavioural synchrony and social bonding: 1) lower cognitive affective mechanisms implicating neuropharmacological reward systems (such as endorphins), 2) neurocognitive action perception responsible for  self-other merging, and 3) and processes of group-centred cognition responsible for perception and reinforcement of cooperation.An assessment of current evidence suggests that affective physiological mechanisms may be more relevant to joint action involving larger group sizes in which generalised feelings of euphoria and pro-sociality are common, whereas neurocognitive mechanisms linking joint action and social bonding may be more applicable to smaller group sizes in which individuals can share intentions through ostensive signals and implicit (emotional) cues \citep{Semin2008,Frith2010}.

Studies linking synchrony with social bonding and cooperation are supported by a literature than connects nonconscious mimicry with liking and affiliation\citep{VanBaaren2009}.  The experimental studies above predominantly refer to dyadic synchronisation of behaviour.   There is also experimental evidence to suggest that exertive and social or coordinated dimensions of group exercise interact to produce social effects.  Recent experimental evidence suggests that social features of the exercise environment (for example, perceived social support, level and quality of behavioural synchrony, etc.) modulate exercise-induced mechanisms of pain, and reward \citep{Cohen2009,Sullivan2014,Tarr2015,Davis2015,Weinstein2016}. This work is bolstered by existing literature on the social modulation of pain \citep{Eisenberger2012a} and links between pain and prosociality \citep{Bastian2014a}.


\myparagraph{Ultimate explanation: Grooming at a distance:}

It is possible to infer from this evidence that an adaptive value of group exercise pertains to the way in which it fosters social cohesion.  In so far as tightly bonded and well coordinated groups face better survival odds than those which are less so, bonding activities which foster social cohesion and trust can be considered collectively advantageous and adaptive \citep{Dunbar2010}.  It is plausible to suggest that group exercise has been subject to processes of cultural group selection, whereby the shared cultural activities in which group exercise commonly feature (music, dance, ritual, sport) have proliferated and fixated in populations, owing to the assemblage of adaptive benefits contained within \citep{Dunbar2010,Whitehouse2004,Atkinson2011a}.

The social high theory of group exercise and social bonding has its roots in studies of social grooming in non-human primates. Dunbar and colleagues propose a neuropharmacologically mediated affective mechanism linking dyadic grooming practices with group-size maintenance \citep{Machin2011}.  The capacity for social cohesion is thought to have arisen in primates as an adaptive response to the pressures of group living.  Aggregating in groups serves to reduce threat from predation.  At the same time, it can be individually costly due to stress arising from interactions at close proximity and conflict over resources among genetically unrelated individuals.  These pressures are hypothesised to have led to selection for social bonding (e.g., via dyadic grooming).  Resulting coalitional alliances among close partners allow for the maintenance of the group by buffering the stresses of group living.  Primate social grooming, for example, is associated with the release of endorphins, presumably leading to sustained rewarding and relaxing effects.  While other neurotransmitters such as dopamine, oxytocin, or vasopressin may also be important in facilitating social interaction, endorphins are argued to reinforce individuals (who are not related or mating) to interact with each other long enough to build ``cognitive relationships of trust and obligation'' \citep[1839]{Dunbar2012}.  It is thought that, as the homo genus evolved more complex collaborative capacities for survival in interdependent group contexts, grooming-like behaviours sustained social bonding in larger groups where dyadic grooming would cumulatively take too much time \citep{Dunbar2012}.

Experimental studies suggest that neurophysiological mechanisms activated by activities that involve physical exertion and coordinated movement, such as group laughter, dance and music-making, exercise, and group ritual can bring groups closer together, mediated by the psychological effects of endogenous opioid and endocannabinoid release \citep{Cohen2009,Fischer2014a,Fischer2014,Sullivan2014,Tarr2016,Tarr2015}. Group exercise contexts can in this sense be understood as practices that resemble ``grooming at a distance.''



\subsubsection{Knowledge Gaps in evolutionary approaches to group exercise}

Current evolutionary accounts of group exercise emphasise the role of proximate cognitive and neuropharmacological mechanisms capable of inducing a psychophysiological environment conducive to forging affiliative social bonds \citep{Dunbar2010,Cohen2017}. Under the assumption that social bonding contributes to processes of social cohesion, and that social cohesion is an adaptive group-level behaviour, researchers propose the evolutionary story that group exercise has become implicated in processes of cultural group selection.

This evolutionary account of group exercise is coherent and testable, but it is also quite preliminary.  When compared to the full range of observable phenomena in group exercise, it is clear that group exercise entails more than simply the experience of a feel good ``social high.''

cost
Meaning


fails to integrate the unmistakably ``embodied'' dimensions of joint action and movement coordination in group exercise contexts.  In this dissertation, I argue that this embodied dimension can be quantified via the science of coordination dynamics, and should be incorporated into existing accounts of group exercise.


%BT: Specifically I will discuss this with regards to (insert sub-points here: flow, X, Y Z) I.e. provide some signposting for this section…what is the purpose of this section?
% But how does this relate to what you have just reviewed, specifically? You need to highlight how the sense of fusion etc described in above section is different to what you mean here… otherwise this just seems like a jump from previous section.
%MEANING & FLOW


\myparagraph{Proximate: cost, meaning, quality}


<Cost>
Relatedly, recent empirical studies of extreme ritual practice present evidence of a positive relationship between pain experienced during high ordeal rituals (e.g., walking on hot coals) and subsequent expressions of parochial prosociality. Interestingly, in contrast to the prevailing adherence to multilevel selection and cultural-group selection theories of social cohesion, these researchers speculate that the selective advantages available in extreme group ritual are afforded primarily to the individual (e.g., moral cleansing, improvements in social standing), with group-level benefits arising only secondarily (Fischer and Xygalatas 2014). In so doing, Fischer and Xygalatas, for example, centralise the explanatory role of proximate psychological mechanisms (of arousal, pain, and reward) in establishing and maintaining social cohesion.
<Signpost here: where are we, where are you going with this, what’s next in the review?>

<Meaning>
It is clear that many people do not engage in exercise just for health, enjoyment, or a ``social high''; rather, in some contexts it forms part of a life of meaning, purpose, and self-discovery (see for example Jackson, 1995; Jones, 2004; White \& Murphy, 2011). Modern sport has always been much more than ``just a game'', and instead offers an arena in which virtues and vices are learned, and the ``morality plays''—--of community, national, and globe—--thus performed (Elias \& Dunning, 1986; McNamee, 2008).  Ethnographic perspectives on group exercise contexts emphasise the ethical and moral dimensions of athletes’ experiences, and contextualise these experiences within political processes relating to the construction of the self, community, and nation-state (Alter, 1993; Brownell, 1995; Downey, 2005b; Wacquant, 2004).


<Quality>
Activities such as music-making, dance, and sport depend upon highly complex coordination of behaviours between individuals, in which the movements of one individual must align in time and space with the movements of another. This is a highly complex movement coordination problem involving high levels of uncertainty and ``free energy'' (degrees of freedom). Highly skilled practitioners who develop a fine-grained sensitivity concerning the perceived outcome of joint action, often report that the ecstasy of group activity is contingent not just on participation, but on the extent to which joint action with co-participants ``clicks.'' To date, however, very little research has dealt directly with the relationship between perceptions of \textit{quality} joint action and processes of social bonding and group formation \citep[but see][]{Marsh2009}.
In order to explain the social and evolutionary significance of the observable phenomenology of team click, this dissertation is grounded in the current state of the art of social cognition of joint action.

<Flow/Quality>
Evidence that there is more to group exercise than existing theories currently explain lies in observation and anecdote relating to the profound meaning derived from participation in group exercise contexts.  In addition to reports of exercise-induced euphoria and positive affect, adherents to (group) exercise and other activities---particularly highly skilled practitioners---also commonly report experiencing states of ``optimal'' or ``peak'' performance, which include feelings of heightened focus, personal transcendence, time-warp (the experience of time either speeding up or slowing down), spontaneity, creativity, and effortlessness \citep{Jackson1995a}.  ``Flow,'' as this particular cluster of states has commonly been referred to, is a powerful, autotelic and embodied experience, which combines components of both ``hedonic'' (sensation-centred, see \citep{Huta2010}) and ``eudaimonic'' (meaning-centred, see \cite{Ryff1989,Ryff2015}) dimensions of subjective well-being, and is theorised to emerge when activity strikes a balance for the individual between challenge and skill requirements \citep{Csikszentmihalyi1990,Abuhamdeh2012}.  At the level of the group, the ``team click'' and ``group flow'' are highly elusive possibilities, coveted by athletes, coaches, and fans alike \citep{Novak1993,Sawyer2006}.  While the experience of flow associated with prolonged exercise may be in part neuropharmacologically mediated by the opioidergic and endocannabinoidergic systems, phenomenological accounts suggest that there is something distinct about the experience of flow in exercise that requires a more complete cognitive and social explanation \citep{Dietrich2006,Dietrich2011}.


\myparagraph{Ultimate questions?}


1. Whitehouse modes of ritual and identity fusion



2. Cultural attraction of cultural representations
















\subsubsection{Team Click and the social cognition of joint action}

Describe phenomenon:
tacit understanding
ATMOSPHERE
flow

 I draw upon emerging evidence from the social cognition of joint action, to analyse the psychological phenomenon of what I term ``team click''---a subjective perception of the tacit quality of coordination in joint action among athletes.  Similar in many respects to psychological states associated with ``flow'' and peak performance \citep{Csikszentmihalyi1992}, team click specifically delineates perceptions of joint action from individual action, and therefore implicates physiological, cognitive, and social mechanisms unique to joint action, as well as nonlinear systems dynamics associated with participating in a socially-coordinated, multi-agent system of physical movement \citep{Kelso2009}. As I will discuss in Chapter 2, team click is anecdotally present in a wide range of joint action contexts, and is often associated in these contexts with psychological processes of positive affect and wellbeing, as well as personal agency, social affiliation, and group membership \citep{Jackson1995,Marsh2009,Wheatley2012,Slingerland2014}.

While the contextual antecedents\citep{Fong2015}, neuropharmacological \citep{Boecker2008} and neurocognitive mechanisms \citep{Dietrich2004,Dietrich2011,Cheron2016}, and psychological consequences \citep{Wheatley2012} of flow and related states have been well researched, very few direct attempts have been made to incorporate non-linear systems dynamics of joint action into these accounts \citep[but see][]{Marsh2009}.  Further still, very few researchers have attempted to situate flow and related group-level psychological phenomena within a broader evolutionary framework \citep[but, for a general theoretical proposal, see][]{Slingerland2014}).  Reasons for these knowledge gaps within cognitive and evolutionary approaches to flow and joint action can be explained in part by the theoretical occlusion of dynamical properties associated with established theoretical paradigms, and in part due to methodological difficulty in quantifying non-linear dynamics of human movement \citep{Kelso2009}.

Recent advances in neuroimaging technologies \citep{Frith2007}, emerging neurocomputational theories of brain function \citep{Friston2010,Frith2010,Clark2013}, and constructive attempts to extend the theoretical paradigm of human social cognition to account for inter-individual processes of interaction and coordination \citep{Sebanz2006,Dale2014}, now provide an opportunity to address these knowledge gaps within the research domain of the social cognition of joint action.  Owing to innovative research within this domain, it is now more clearly understood that basal human capacities for physical movement regulation and coordination set the foundation for social cognitive systems whose resources are distributed between brains, bodies, and physical features of task-specific environments \citep{Hutchins2000,Kirsh2006,Semin2008,Semin2012,Coey2012}.
Furthermore, it has been shown that the quality of coordinated movement within these cognitive systems has implications for psychophysiological health and subjective well-being \citep{Wheatley2012}, and is relevant to the effective function of more complex goal oriented social activities, including the large-scale reproduction and transmission of shared cultural practices \citep{Dunbar2012,Roepstorff2010,Claidiere2014,Launay2016}. Thus, there is evidence to suggest that the ``embodied'' dimension of group exercise, traditionally occluded by theoretical convention and methodological challenges, can now be accessed by  interrogating the ways in which cognitive mechanisms and system dynamics of joint action are responsible for processes of social cohesion.


  Recent experimental evidence suggests that social features of the exercise environment (for example, perceived social support, level and quality of behavioural synchrony, etc.) modulate exercise-induced mechanisms of pain, and reward \citep{Cohen2009,Sullivan2014,Tarr2015,Davis2015,Weinstein2016}. This work is bolstered by existing literature on the social modulation of pain \citep{Eisenberger2012a} and links between pain and prosociality \citep{Bastian2014a}.

%The social and psychological effects of group level synchronisation have been harder to induce and measure in experimental settings. However, in addition to in- and anti-phase behavioural matching, group synchronisation may be subject to more complex and dynamical processes of coupling, which could entail specific psychological consequences. This also appears to be true in cases of joint---but not necessarily explicitly synchronised---action, whereby implicit processes of movement regulation link two or more individuals in a complex and dynamic coupling. The variation and stabilisation of such dynamic couplings could have psychological effects (see \citep{Schmidt2008,Marsh2009a}).











\subsubsection{A theory of social bonding through joint action}










A combination of advances in neuroimaging technologies \citep{Frith2007}, emerging neurocomputational theories of brain function \citep{Friston2010,Frith2010,Clark2013}, and constructive attempts to extend the theoretical paradigm of human social cognition to account for inter-individual processes of interaction and coordination \citep{Sebanz2006,Dale2014}, has created an opportunity to empirically examine the relationship between coordinated and exertive group activities and social cohesion.  It is now more clearly understood that basal human capacities for physical movement regulation and coordination set the foundation for social cognitive systems whose resources are distributed between brains, bodies, and physical features of task-specific environments \citep{Hutchins2000,Kirsh2006,Semin2008,Semin2012,Coey2012}.  Furthermore, it has been shown that the quality of coordinated movement within these cognitive systems has implications for psychophysiological health and subjective well-being \citep{Wheatley2012}, and is relevant to the effective function of more complex goal oriented social activities, including the large-scale reproduction and transmission of shared cultural practices \citep{Dunbar2012,Roepstorff2010,Claidiere2014,Launay2016}.
Establishing functional interpersonal synergies is an adaptive response to the ``degrees-of-freedom problem'' encountered by the nervous system in social interactions involving many moving parts.  Synergies have been shown to act as an extra-neural basis for prediction error minimisation, aid reciprocal information sharing throughout system nodes, and increase individual cognitive performance \citep{Schmidt2016}.  The link between interpersonal coordination and social bonding has been addressed in the behavioural mimicry and synchrony literatures \citep[e.g.,][]{Wheatley2012,Launay2016,Mogan2017}, but there is less substantive evidence in relation to dynamic interpersonal coordination in natural joint action settings such as those found in group exercise contexts \citep{Marsh2009,Miles2009,Lumsden2012}.


\myparagraph{Specialised Mechanisms of Joint Action}

Literature suggests that successful joint action in humans is contingent on the ability to share functionally equivalent task representations. Considering the cognitive principles of ``active inference'' referenced above, shared task representation amounts to minimising prediction error in social cognitive systems involving two or more co-actors \citep{Semin2008,Frith2010}.  Humans appear to employ an array of explicit and implicit behavioural strategies in order to achieve this.  The ways in which co-actors ``close the loop'' \citep{Frith2007} on joint action through deliberate ostensive communication has been the traditional focus of developmental, comparative \cite{Tomasello2005a}, and social psychologists \citep{Sebanz2006}.
More recently, however, analysis of dynamic coupling of co-actors in joint action scenarios reveals that synchronised movement implicates an array of implicit and pre-perceptual cognitive processes of alignment and prediction error minimisation \citep{Schmidt2011}, which, in addition to more explicit forms of communication, could be central to the generation of feelings of self-other merging, self-other distinction, and perceived reliability and trust associated with social bonding \citep{Marsh2009}.




\myparagraph{Predictive Coding & Thermodynamics of cognition}




A key facet of the paradigm shift in social cognition outlined above is the overhaul of traditional individual-centred computational models of information processing (originally inspired by the mechanics of the electronic computer), which tend to render movement as the final product of a linear sequence of sensory perception, amodal mental representation, and action selection \citep{Lewis2005}.  By contrast, the prevailing neurocomputational paradigm of ``predictive processing,'' also known as ``predictive coding,'' \citep[see][]{Frith2007,Kilner2009,Clark2013} offers a model of brain function in which perception, representation, emotion, and action are functionally and temporally integrated in the service of informational processes.  Perception, cognition, and action work closely together to minimise sensory prediction errors by selectively sampling, and actively sculpting, the stimulus array. <link to next section>






\myparagraph{Functional Interpersonal Synergies}


For a brain in the business of prediction error minimisation, interpersonal interactions with other agents present a computational challenge due to the moving parts.  The large number of independently controllable movement system degrees of freedom of multiple agents places computational burden to the central nervous system, dubbed by Bernstein \textcite{Bernstein1967} as the ``degrees-of-freedom problem'' \citep[see also][]{Turvey1982,Turvey1990}.  From a system dynamics perspective, a solution to this problem is to work in such a way that the movement system degrees of freedom residing in different actors and environmental features are coupled to form low-dimensional, reciprocally compensating synergies, known as ``functional interpersonal synergies'' \citep{Riley2011}.

Much like intrapersonal coordination, functional interpersonal synergies are temporarily assembled, task-specific, functional couplings between a system's componential degrees of freedom, such that one component of a synergy reacts to changes in the others \citep{Kelso2009}.  Once coordinated to behave as a functional unit, the individual degrees-of-freedom do not need to be controlled independently of one another, and perturbations applied to a component are automatically compensated by the coupled components \citep{Kelso1984,Latash2002,Riley2011}, making them adaptive responses to environments of high computational uncertainty.

Experimental evidence has shown that functional interpersonal synergies, for example in-phase or anti-phase coordination of hand movements, facilitates memory recall of incidental information concerning co-actors \citep{Miles2010}. It has also been shown that interpersonal synergies more generally facilitate performance of social cognitive or linguistic tasks, such as gaze coordination and turn taking in conversation \citep{Richardson2005,Shockley2009}.  Richardson and Dale (2005), for example, show that more tightly coupled gaze between conversing dyads leads to higher discourse comprehension.  Conversely, being psychologically distanced from another individual can inhibit the emergence of interpersonal synergies \citep{Miles2010}.  The ways in which functional interpersonal synergies facilitate adaptive information transfer between individuals and within groups suggests that psychological mechanisms and cultural practices responsible for generating these synergies could have been subject to cultural evolutionary forces of selection and attraction \citep{Claidiere2014,Mesoudi2016a}.

Considered form within the PP and Free Energy principle paradigms, interpersonal synergies represent one way in which error minimisation can be achieved via extra-neural mechanisms.  In addition to the behavioural mimicry and synchrony literatures mentioned above, research into the coordination dynamics of natural joint actions has shown evidence of dynamic coupling (synchronisation) in joint-action tasks, such as dancing, martial arts, moving objects like furniture, etc. In these studies, specific component degrees of freedom are modelled as coupled oscillators (using the HKB model \citep{Haken1985,Kelso1986}, which describes the change in the relative phase between two oscillatory components). Models are analysed for non-random fluctuations in relative phase over multiple time scales.  This type of synchronisation is said to be of a fractal or semi-fractal organisation, also known as 1/f scaling or ``pink noise'' \citep{Caron2017}. According to Anderson and colleagues \citep{Anderson2012}, 1⁄f scaling is ubiquitous in smooth cognitive activity, and indicates a self-similar structure in the fluctuations that occur over time (within a time series of measurements). 1⁄f scaling indicates that the connections among the cognitive system's components are highly nonlinear \citep{Ding2002,Holden2013,Kello2010,Riley2011,VanOrden2003,VanOrden2005}. Pink noise has been measured beyond dyadic synchronisation, in the analysis of sub-phases of team sports \citep{Passos2014,Duarte2012} and group dancing \citep{Chauvigne2017}.\footnote{1⁄f scaling is temporal long-range dependencies in the fluctuations of a repeatedly measured behaviour or activity. Analogous to spatial fractals, 1⁄ f scaling denotes a fractal or self-similar structure in the fluctuations that occur over time. That is, higher frequency, lower amplitude fluctuations are nested within lower frequency, higher amplitude fluctuations as one moves from finer to courser grains of analysis \cites(for a more detailed description see, for example)(){Holden2005}{Kello2009}}
% Do you need all of this? It doesn’t seem to directly inform what comes next, so I’d say, CUT!

In studies involving skilled versus non-skilled practitioners in dyadic interactions, it has been shown that more skilled practitioners create stronger dynamical coupling through flexibly modulating their actions with others \citep{Schmidt2011, Caron2017}. These findings are corroborated by other studies that find that professional footballers (versus novice controls) are able to more accurately predict the direction of a kick from another player's body kinematics (\cite{Tomeo2012}, see also \cite{Aglioti2008,Mulligan2016} for similar results with basketball and dart players). Interestingly, when analysing co-regulation between members of basketball teams, it was shown by Bourbousson \textcite{Bourbousson2015} that more expert teams made fewer mutual adjustments (at the level of the activity that was meaningful for co-actors), suggesting an enhanced capability of expert social systems to achieve and maintain an optimal level of awareness during the unfolding activity, potentially implicating down-regulation of prediction error management processes.





\myparagraph{Expectations violation: a potential affective mechanism}










\subsection{Joint action in interactional team sport}












\section{The present study}

      \subsection{Research Setting}
      The empirical content of this dissertation draws from one contemporary instance of group exercise, in one geographic region.  Rugby union football and the People's Republic of China are subjects that do not commonly appear in the same sentence, but nonetheless, the Olympic status of rugby union, and the deep Olympic logic of the state-sponsored Chinese sports system, means that today hundreds of professional Chinese rugby players are meaningfully engaged in one of the world's most physiologically strenuous interactive team sports.  During a two year period between August 2015 and September 2017, I spent three separate periods in China during which I conducted a total of 10 months of ethnographic research with the Beijing Men's Provincial Rugby Team. I then extended this ethnographic analysis by conducting as two field studies, for which I sampled from a broader population of professional Chinese rugby players from 9 different provinces.

      Between August 2015 --- March 2016, I spent seven months in Beijing engaged in participant observation and conducting unstructured and semi-structured interviews, and informal surveys with the Beijing Men's Rugby Team. Between July --- August 2016, I returned to China for a further two months, during which time I continued ethnographic observations of the Beijing team, while also conducting two pseudo-experimental field studies spanning two other locations, Hebei province and Shandong province. Finally, I spent one month in Beijing and Tianjin between August --- September 2017 during which time I conducted follow-up structured interviews the with 10 athletes who participated in the Chinese National Games, as well as follow up informal interviews with athletes from the Beijing Men's Rugby Team.

      \subsection{Study Predictions}



      The overarching prediction of this thesis is that the psychological phenomenon of team click mediates a relationship between joint action and social bonding.

      Within this main hypothesis, I also formulate the following sub-hypotheses:
      1)	Athletes who perceive greater success in joint action will experience higher levels of felt ``team click.'' I predict that relevant perceptions of joint action success will relate to athlete perceptions of:
      1.a) a combination of specific technical components; or
      1.b) an overall perception of team performance relative to prior expectations; or
      1.c) an interaction between these two dimensions of team performance.
      2)	Athletes who experience higher levels of team click will report higher levels of social bonding.
      3)	More positive perceptions of joint action success will predict higher levels of social bonding, driven by more positive:
      3.a) perceptions of components of team performance; or
      3.b) violation of team performance expectations; or
      3.c) an interaction between the two predictors.




      \subsection{Outline}



\section{Thesis Contributions}

\section{Chapter Summary}

% A brief chapter summary of what we have just read would be useful. And a bridge to the next chapter.



























%%%%%%%%%%%%%%%%%%%%%%%%%%%%%%%%%%%%%%%%%%%%%%%%%%%%%%%%%%%%%%



\section{Group exercise and social cohesion}
% 527 words
There is something unmistakably ``embodied'' about coordinated and exertive group activity.
%This is the idea that the dynamical vitality of movement is important to its evolutionary story...
Indeed, perhaps this visceral quality has helped ground the traditional intuition within anthropology that coordinated group activity (broadly construed) is somehow causally relevant to broader social processes, despite the lack of precise theoretical frameworks within which to test such speculations (see for example, \citep{Durkheim1965,Mauss1935,Radcliffe-Brown1952,Turner1974,Merleau-Ponty1956,Bourdieu1990}).

%MS to human evolution -- what are the approaches?
The modern evolutionary synthesis and cognitive revolution have since helped spawn rigorous scientific inquiry into the

sociality

cultural practices
cultural norms can rarely be explained by genetic differences...

 role of human cultural activities in

 relationship between shared cultural practices (including those in which group exercise commonly features) and human social cohesion, and great strides have been made in identifying proximate mechanisms and ultimate evolutionary explanations for the transmission of, and widespread adherence to, cultural practices throughout populations \citep{Dawkins1976,Boyd1988,Sperber1996,Barrett2002,Whitehouse2004,Whitehouse2014,Henrich2007}.

 However, still in their relative infancy, these theories are necessarily limited in their range and scope by start-up assumptions and idealisations, particularly in regards to the details concerning proximate cognitive mechanisms and psychophysiological, social, and contextual constraints on the transmission of cultural information transfer \citep{Sperber1996,Dunbar2012,Claidiere2014}. A theory capable of satisfactorily explaining the visceral and social dimensions of group exercise so strongly substantiated by observation, anecdote, and intuition is yet to be fully formulated \citep{Cohen2017}.

A combination of advances in neuroimaging technologies \citep{Frith2007}, emerging neurocomputational theories of brain function \citep{Friston2010,Frith2010,Clark2013}, and constructive attempts to extend the theoretical paradigm of human social cognition to account for inter-individual processes of interaction and coordination \citep{Sebanz2006,Dale2014}, has created an opportunity to empirically examine the relationship between coordinated and exertive group activities and social cohesion.  It is now more clearly understood that basal human capacities for physical movement regulation and coordination set the foundation for social cognitive systems whose resources are distributed between brains, bodies, and physical features of task-specific environments \citep{Hutchins2000,Kirsh2006,Semin2008,Semin2012,Coey2012}.  Furthermore, it has been shown that the quality of coordinated movement within these cognitive systems has implications for psychophysiological health and subjective well-being \citep{Wheatley2012}, and is relevant to the effective function of more complex goal oriented social activities, including the large-scale reproduction and transmission of shared cultural practices \citep{Dunbar2012,Roepstorff2010,Claidiere2014,Launay2016}.

% This is the evolutionary system human niche section:
A combination of advances in neuroimaging technologies \citep{Frith2007}, emerging neurocomputational theories of brain function \citep{Friston2010,Frith2010,Clark2013}, and constructive attempts to extend the theoretical paradigm of human social cognition to account for inter-individual processes of interaction and coordination \citep{Sebanz2006,Dale2014}, has created an opportunity to empirically examine the relationship between coordinated and exertive group activities and social cohesion.  It is now more clearly understood that basal human capacities for physical movement regulation and coordination set the foundation for social cognitive systems whose resources are distributed between brains, bodies, and physical features of task-specific environments \citep{Hutchins2000,Kirsh2006,Semin2008,Semin2012,Coey2012}.  Furthermore, it has been shown that the quality of coordinated movement within these cognitive systems has implications for psychophysiological health and subjective well-being \citep{Wheatley2012}, and is relevant to the effective function of more complex goal oriented social activities, including the large-scale reproduction and transmission of shared cultural practices \citep{Dunbar2012,Roepstorff2010,Claidiere2014,Launay2016}. Thus, the somewhat nagging visceral intuition associated with the observable human compulsion to come together and move together could in fact prove useful as a source of insight for progressing the science of human evolution.  By interrogating the ways in which component mechanisms and system dynamics of joint action generate social bonding, this dissertation seeks to offer a novel contribution to cognitive and evolutionary anthropology.















%%%%%%%%%%%%%%%%%%%%%%%%%%%%%%%%%%%%%%%%%%%%%%



explanation of ritual practices: Whitehouse, Fisher & Xygalatas
%Unifying theories: WHITEHOUSE ritual practices
Recently, anthropologists have attempted to integrate analyses of ethnographic, archaeological, and phylogenetic information in order to develop broader theories of social cohesion. Drawing initially from ethnographic observations of ritual practice in the Papua New Guinean Highlands, Harvey Whitehouse developed a general theory of human social cohesion based on two divergent modes of ritual practice and their associated psychological and sociopolitical effects \citep{Whitehouse1996}. High-frequency, low-arousal religious rituals (weekly attendance at church sermon, praying, etc.) are associated with identification with the prototypical features of the group (“group identification”) whereas low-frequency, high-arousal rituals (tribal initiation rituals, dysphoric experiences such as frontline combat) generate “identity fusion”—--a visceral feeling of oneness with the group.

Whereas group identification can be understood as a psychological adaptation deriving from a norm coalitional cooperative mechanisms, identity fusion arises from the generalisation of kin-detection mechanisms, whereby individuals recognise others with whom they have shared core self-defining experiences as ``fictive kin.'' Whitehouse and colleagues (2014) argue that these two distinct psychological states, and the ritual practices that reliably generate such states, represent ``attractor positions'' (Sperber 1996) in the cultural evolution of religion and human sociality. In a return to Durkheim’s original outlay for the study of social cohesion, the modes theory incorporates the two key underpinning mechanisms of social cohesion (i.e., cumulative culture and its interaction with human capacities for social bonding) by accounting for variance both in the modes of shared cultural practices and in the emotional quality of group-level commitment.

In his study of religion, prosociality, and extreme altruistic behaviour, Scott Atran similarly insists on the need to carefully consider the interaction between cultural, cognitive, and affective mechanisms, in particular the role of communal rituals in ``rhythmically coordinat[ing] emotional validation of, and commitment to, moral truths'' (Atran and Norenzayan 2004, 714). Atran suggests that extreme altruism can be explained only by considering the codependent relationship between the affective motivational processes (of arousal, pain, and reward) and the cultural representations (''sacred values'' of the group) with which these psychophysiological mechanisms interact and become associated.



Most encouraging is evidence that manages to integrate the social and neurophysiological dimensions of group exercise.





%Current evolutionary accounts of group exercise infer adaptive value  from the effects of group exercise recorded in laboratory or field settings.  In addition to the obvious health and wellbeing benefits of physical exercise more generally, which appear to suggest the adaptive function of physical activity on the level of the individual, evolutionary accounts of group exercise emphasise evidence of positive emotional and social effects such as social bonding and affiliation.

%In other words, it is understood that at some point in human evolutionary history, more cohesive groups outperformed less cohesive groups, and that group exercise, as a promoter of social bonding, may have been an important dimension to social cohesion.


%As I will develop in the sections that follow, this prevailing evolutionary account of group exercise---while convincing and coherent---threatens to occlude key dimensions of group exercise context, particularly those relating to the nonlinear self-organsing dynamics of human movement systems.  The current evolutionary account, in which group exercise is tethered to processes of social cohesion via evolved cognitive and neuropharmacological mechanisms is currently unable to account for subjective and distributed cognitive processes of joint action common in interactive team sport scenarios. I explain that constraints upon theories of group exercise are part theoretical, part methodological.  This dissertation makes contributions that work towards relieving both constraints.

%The social and psychological effects of group level behavioural synchrony have been harder to induce and measure in experimental settings. There is some evidence to suggest that, in addition to in- and anti-phase behavioural matching, group synchronisation may be subject to more complex and dynamical processes of coupling, which could entail specific psychological consequences. This also appears to be true in cases of joint---but not necessarily explicitly synchronised---action, whereby implicit processes of movement regulation link two or more individuals in a dynamic coupling. The variation and stabilisation of such dynamic couplings could have psychological effects (see \citep{Schmidt2008,Marsh2009a}).



It is believed that lower cognitive processes of joint attention mediate the link between synchrony and social bonding, with synchronised activity (common in music, dance, and some sports) providing a shared spatio-temporal (and often haptic) referent around which to coordinate attention and behaviour \cite{Launay2016,Wolf2015}.

% BT: Before you launch into grooming story, include here (e.g. at end of this sentence) something that indicates where you are going with this (e.g. the ‘grooming at a distance’ concept).
