
\begin{savequote}[8cm]

  It takes two to know one.

  \qauthor{--- Gregory Bateson}

\end{savequote}









\chapter{\label{chap:intro}Introduction}



\minitoc





                                          \begin{CJK}{UTF8}{gbsn}

\section{My first night in Beijing \label{sect:adrian}}

Adrian, Kai, and I waited for Mr Shi to arrive in the upstairs area of the Korean BBQ restaurant in a quiet willow lined street just inside Beijing's East 4th Ring Road.  Adrian, the host of the dinner, was a veritable elder of Chinese rugby.  He was captain of the second class of rugby players to graduate from the Chinese Agricultural University (CAU), the home of China's first official rugby union program, established in 1990.  I first met Adrian two years earlier through Kai. Kai---a close friend of mine of many years---was a more recent graduate of CAU (2007), a former Chinese National rugby team representative, and since graduating from CAU, a lawyer in Beijing.  Mr Shi, the guest of honour for whom the three of us were waiting, was a technical producer for Chinese Central Television's Sport Channel, ``CCTV5.''

The backstory was that CCTV5 needed help producing the commentary for the Rugby World Cup, which they were planning to broadcast for the first time in October 2015.  Mr Shi reached out through his network of relationships in Beijing and soon tracked down Adrian; Adrian tracked down Kai; Kai, in turn, tracked down me.  I was eager to catch up with old friends as well as begin my fieldwork, and so, despite my jet lag, I accepted the invitation. Early that first Saturday evening I set off to the Korean BBQ restaurant with my notebook and audio recorder (i.e., my mobile phone) in hand.

Adrian naturally held the floor in conversation while the three of us waited for Mr Shi to arrive.  He reminisced fondly about his time playing rugby at CAU, as well as his time after graduation playing with the Beijing Devils, a rugby club in Beijing whose members were predominantly expats.  He assured us that rugby in China was, in those days, fun and free-spirited.  Not like today, now that Chinese rugby has become a professional program in the state sponsored sport system, (owing to its Olympic status in the modified form of the game, rugby sevens, see Chapter ~\ref{chap:researchSetting}).  Adrian talked about going on a rugby tour to the UK with the Beijing Devils :  ``Everyone only just scraped together the money to go on tour, we all payed our own way, sometimes you'd get a bit of help from someone or whatever. We did it because we loved the game, not for any other reason,'' he insisted.  Kai and I listened intently.  All of a sudden I realised that this conversation could be relevant, so I started taking notes.

When Mr Shi finally arrived, Adrian continued the nostalgic story telling mode, but naturally shifted his target audience from Kai and me to Mr Shi.  Adrian began to describe in rich detail the experience of camaraderie between he and his Beijing Devils team mates when they participated in overseas rugby tour. But he then interrupted his own story to make an explanatory aside directed at Mr Shi, accommodating for the fact that Mr Shi was relatively unacquainted with the sport: ``This sport, rugby: it's actually very mysterious. If you haven't played it yourself you might not know this type of feeling,'' Adrian respectfully suggested to Mr Shi.  ``Because rugby, you know, you're all on the field together, there's body contact...'' he paused to find the right phrasing,  ``...its a very \textit{carnal} type of feeling.''  His attempts to enrich his communication by gesticulating had led him to have both of his hands clenched as fists in front of him like they were cradling a rugby ball or gripping the steering wheel of a car---a lit cigarette smouldering between the index and middle finger of his right hand.  Adrian concluded by repeating: ``Its very mysterious.'' He shook his head as if baffled and finally released his clenched fists to dab the ash from his cigarette into the ashtray in front of him.  After taking another drag from his cigarette he finally added: ``So it means this rugby circle here in China is very tight...''---a short pause for another dab of his cigarette--- ``...but it doesn't mean that this circle is not also not also complete chaos!''
  \footnote{Circle (\textit{quanzi} 圈子) is a common colloquial way to refer to a social group or community of people in modern standard Chinese.}
The wisdom of Adrians's final punchline was confirmed with a knowing chuckle from all of us, including Mr Shi. Adrian concluded his performance silently, by taking a long, satisfying drag of his cigarette.

%英式橄榄球这个项目其实特别神秘,没玩过的话您可能不知道这种感觉,因为英式橄榄球么,大家在场上有身体接触,是一种``肉''的感觉,大家互相都特别亲,特别神秘。
%所以在橄榄球这个圈子特别亲, 但这不是说这个圈儿也不乱!

I was captivated---but also somewhat surprised---by Adrian's monologue.  I was not expecting, so early into my fieldwork, to happen upon a declaration in which a link between the carnal (\textit{rou} 肉) sensations associated with on-field joint action, and social processes of interpersonal emotional affiliation (\textit{qin} 亲) and group cohesion of the rugby community (\textit{quanzi} 圈子) was so explicitly and spontaneously emphasised.  It was clear that rugby's visceral dimension continued to capture Adrian emotionally; some fifteen years after he had finished playing his fists still clenched with energy, and his head still shook with amazement.

I was also intrigued that Adrian cited the source of his emotional capture as at once both very specific (derived from playing together with others on the field) and, at the same time, ultimately ``mysterious.''  The aim of my research, which by that time I had broadly formulated, was to explain the human behavioural phenomenon of group exercise in terms of its social, evolutionary, cognitive, and physiological causal processes and dynamics.  In essence, the aim was to somehow move from mystery to scientific mechanism in explaining group exercise.  At this first dinner in Beijing, Adrian's comments both captured the phenomenological mystery of group exercise, and pointed me in the direction of the underlying scientific mechanisms.

Needless to say, I left that first dinner eager to investigate the sources of Adrian's experience of mysterious carnality and social connection in rugby's joint action.  My next stop, on Monday morning, was the Temple of God of Agriculture Sports Institute, where I had organised to conduct ethnographic research with the Beijing Provincial rugby program.


                            \begin{center}
                              * * *
                            \end{center}





\section{Scientific explanations of group exercise}
Competitive team sports, whirling Sufi dervishes, late-night electronic music raves, Maasi ceremonial dances, or the fitness cults of Cross-fit and Soul Cycle---endless examples can be plucked from across cultures and throughout time to exemplify the human compulsion to come together and move together.  How is it possible to explain the prevalence of these activities in the human record?  In this dissertation I contribute to a scientific understanding of physiologically exertive and socially coordinated movement (hereafter simply ``group exercise'') by way of a focussed study of the social cognition of joint action among professional Chinese rugby players.

Because physical movement is a metabolically expensive task for all biological organisms, it is justifiable from an evolutionary standpoint only if the benefits somehow outweigh the costs.  Using this basic calculus, it is easy to imagine how group exercise would have served important survival functions in our ancestral past.  Activities involving group exercise such as hunting, travel, communication, and defence all appear to confer immediate and obvious benefits to individuals and groups \citep{Sands2010}.

In more recent domains of human history, however, the task of explaining the persistent recurrence of group exercise is more complicated.  At least since the late Pleistocene era (approx. 500ka), and particularly since the Holocene transition from hunter-gatherer to agricultural (approx. 11ka), and later industrial and post-industrial societies, group exercise can be identified in shared cultural practices as varied as religion, organised warfare, music, dance, play, and sport.  Unlike group hunting or defence, however, the fitness-relevant benefits of group exercise in cultural practices such as sport, music, or dance are not always as obvious.  On the contrary, many of these activities appear on the face of things to entail extreme time and energy costs for very little immediate reward.

The prevalence of group exercise in a diverse array of shared cultural practices in the more recent human record thus presents an evolutionary puzzle.  A solution to which requires a more nuanced calculus that incorporates an appreciation of humans' species-unique evolutionary trajectory, defined by increasingly complex cognitive and cultural capacities, including technical manipulation of extra-somatic materials and ecologies; advanced theory of mind; and information-rich, malleable, and scaleable communication systems \citep{Roepstorff2010,Clark2015,Fuentes2016}.  A theory capable of satisfactorily explaining group exercise within humans' distinctive evolutionary parameters is yet to be fully formulated \citep{Cohen2017}.

In this dissertation, I develop existing cognitive and evolutionary understandings of group exercise by formulating and testing a novel theoretical relationship between joint action and social bonding.  Existing theories of group exercise and social bonding fail to satisfactorily account for the variation in, and complexity of, interpersonal movement coordination common to many real world settings of human behaviour.  I concentrate on the cognitive mechanisms and system dynamics that enable and constrain joint action in group exercise, and draw attention to the social and psychological effects known to occur when joint action functions successfully---i.e., when joint action ``clicks'' between co-actors.  I propose the phenomenon of ``team click'' as a candidate construct that can help explain a hitherto under-examined link between joint action and social bonding.

In this chapter, I review existing theories of group exercise and social cohesion (Section ~\ref{sect:existingGESoCo}), and point to key knowledge gaps that require attention (Section ~\ref{sect:knowGaps}).
I then introduce the important features of the present study (Section ~\ref{sect:presentStudy}). In order to anchor the theory of social bonding through joint action, I first introduce the widely identifiable phenomenon of ``team click'' in joint action, and propose that mechanisms relating to team click could mediate a relationship between joint action and social bonding (Section ~\ref{sect:teamClickIntro}).  I then preview the main contributions of this dissertation (Section ~\ref{sect:contributions}), which include 1) a novel theory of social bonding through joint action, 2) an ethnographic study of the Beijing men's rugby team (n = 26), 3) an \textit{in situ} survey study of professional rugby players during a National Rugby Tournament (n = 174), and 4) a controlled field experiment with a sample of Chinese rugby players across two locations (n = 58).  I conclude with an overview of the dissertation structure (Section ~\ref{sect:chapters}).

\subsection{Existing theories of group exercise and social cohesion\label{sect:existingGESoCo}}
It has been long speculated by social scientists \citep[see, for example][]{Mauss1935,Durkheim1965} that cultural activities in which group exercise feature foster social cohesion \citep{Dunbar2010,Whitehouse2004}.  Social cohesion implies proximity, coordination, and stability of relationships between members of a group, which serve some benefit to the group as a whole \citep{Taylor2018}.  In so far as more cohesive groups face better survival odds than those which are less so, activities that foster social cohesion can be considered collectively advantageous and adaptive \citep{Dunbar2010}.  It is not yet clear, however, precisely how or whether group exercise uniquely generates social cohesion, or in what ways particular mechanisms vary by activity and cultural context.

\subsubsection{The social high theory of group exercise and social bonding \label{sect:socialHigh}}
Existing research detailing the proximate physiological, cognitive, and social mechanisms associated with group exercise suggests that group exercise is responsible for generating a psychophysiological environment conducive to social affiliation and trust.  Anthropologist Emma Cohen has recently proposed a ``social high'' theory of group exercise and social bonding \citep[hereafter ``the social high theory,'' see][]{Cohen2017}, in which she identifies links between the two essential ingredients of group exercise---1) physiological exertion and 2) interpersonal movement coordination---and their common psychophysiological effects, including increased pain tolerance, athletic performance, positive affect, wellbeing, pro-sociality, and cooperation \citep{Davis2015}. I outline the key principles of the social high theory below, before identifying the knowledge gaps that require further attention.

\myparagraph{Physiological exertion}
Group exercise necessarily entails rigorous physiological exertion.
The health and wellbeing benefits associated with regular physical exercise, including reduced risk of cardiovascular disease, autonomic dysfunction, and early mortality; as well as enhanced neurogenesis, cognitive ability, and mood, are becoming increasingly well-known \citep{Blair1994,Nagamatsu2014}. It is now also understood that strenuous and prolonged physical exercise is modulated by the same neuropharmacological systems responsible for regulating pain, fatigue, and reward \citep{Boecker2008,Raichlen2013}.  Neurobiological rewards in exercise are associated with both central effects (improved affect, sense of well-being, anxiety reduction, post-exercise calm) and peripheral effects (analgesia), and appear to be dependent for their activation on exercise type, intensity, and duration \citep{Dietrich2004}.  Exercise-specific activity of neurobiological reward systems offers a plausible explanation for commonly reported sensations of positive affect, anxiety reduction, and improved subjective well-being during and following exercise---extremes of which are popularly referred to as the ``runner's high'' \citep{Dietrich2004,Boecker2008,Raichlen2012}.  This neurobiological evidence maps on to more extensive literature concerning the psychological effects of exercise, which indicates a duration and intensity ``sweet spot'' for exercise and positive affect, whereby moderate intensity exercise for durations of $\sim45$ minutes appears most optimal \citep{Reed2006}.

It is possible that the function of exercise-induced positive affect extends to the realm of social bonding, particularly when achieved in group exercise contexts \citep{Cohen2009,Machin2011}.  Endocannabinoids and opioids have been implicated in mammalian social bonding \citep{Fattore2010,Keverne1989}, and in humans specifically, there is evidence that endorphins (a particular class of endogenous opioids) mediate social bonding \citep{Dunbar2012,Shultz2010}.

\myparagraph{Interpersonal movement coordination}
Meanwhile, experimental evidence (predominantly from social psychology) suggests that time-locked coordination of behaviour between two or more individuals is conducive to psychological processes of self-other merging, liking, trust, and psychological affiliation.  In these contexts, interpersonal coordination is primarily operationalised as behavioural synchrony---i.e., stable time- and phase-locked movement of two or more independent components (limbs, bodies, fingers, etc.) \citep{Pikovsky2007}. It is believed that synchrony enables a tight attentional union between individuals who match the timing and content of their actions, leading to the enhancement of interpersonal similarity and the blurring of self-other boundaries in cognitive processing and recall \citep{Cohen2017}.  Relative to non-synchronous group activities, synchrony increases social bonding and pro-social behaviour---an evolutionarily important outcome of bonded relationships \citep{Reddish2013,Reddish2013a,Wiltermuth2009}.  Recent studies have also found that, compared to solo and non-synchronous group exercise, synchronous group exercise leads to significantly greater post-workout pain threshold \citep{Cohen2009,Sullivan2014,Sullivan2013a, Sullivan2013b}.

A recent meta analysis of the behavioural synchrony literature in social psychology suggests three candidate mediators of the relationship between behavioural synchrony and social bonding: 1) lower cognitive affective mechanisms implicating neuropharmacological reward systems (e.g., opioidergic and dopaminergic systems), 2) neurocognitive action-perception networks responsible for the experience of self-other merging, and 3) processes of group-centred cognition responsible for perception and reinforcement of cooperation \citep{Mogan2017}.  The current balance of existing evidence suggests that affective physiological mechanisms may be more relevant to joint action involving larger group sizes in which generalised feelings of euphoria and pro-sociality are common \citep[e.g., mass religious rituals or music festivals, see][]{Weinstein2016}, whereas neurocognitive mechanisms linking joint action and social bonding may be more applicable to smaller group sizes in which individuals can share intentions through ostensive communicative signals and implicit movement regulation cues \citep{Semin2008,Frith2010}.  Studies linking synchrony with social bonding and cooperation are supported by a literature than connects nonconscious mimicry with liking and affiliation \citep{VanBaaren2009}.

\myparagraph{The social high $=$ synchrony $\times$ exertion}
In addition to recorded independent effects of exertion synchrony There is also some preliminary evidence to suggest that exertion and coordination in group exercise interact to produce additive social effects.  Social features of the exercise environment (for example, perceived social support, level and quality of behavioural synchrony, etc.) modulate exercise-induced mechanisms of pain, and reward \citep{Cohen2009,Sullivan2014,Tarr2015,Davis2015,Weinstein2016}, and this work is bolstered by existing literature on the social modulation of pain \citep{Eisenberger2012a} and links between pain and prosociality \citep{Bastian2014a}.  The social high theory thus combines these two bodies of literature to tell a story in which positive affect---associated with neuropharmacologically-mediated pain analgesia and reward—--is extended to the social group via synchrony-activated cognitive mechanisms of self-other merging, and the perception and reinforcement of in-group cooperation.


\subsection{Knowledge gaps in the relationship between group exercise and social cohesion\label{sect:knowGaps}}

Recently, researchers interested in the cognitive and evolutionary significance of group exercise have (perhaps unwittingly) made scholarly ceremony out of invoking one particular passage from Durkheim (1965, pg. 217).  ``Once the individuals are gathered together,'' reads the beginning of the passage, ``a sort of electricity is generated from their closeness and quickly launches them to an extraordinary height of exaltation...'' \citep[see ][]{McNeill1995,Konvalinka2011,Fischer2014,Mogan2017}.  Indeed, this passage lends itself neatly a foray into physiological and cognitive mechanisms thought to facilitate the observed ``collective effervescence'' of group exercise.

However,


 only a cursory survey of the spectrum of group exercise contexts identifiable in human sociality is needed to reveal that many group exercise scenarios deviate markedly from the narrow profile of group exercise set out by the prevailing social high theory (and for which Durkheim's passage is enlisted to support).  Anecdotal and observational evidence suggests that group exercise contexts often entail extreme (and not just moderate) levels of psychophysiological exertion.  High-stakes professional competitive sporting contexts (international-level sports, rowing or ultra-marathon running, for example), extreme adventure sports (big wave surfing, free-diving), or high-intensity contact (rugby union, American football, ice hockey) or combat sports (MMA, boxing, wrestling) are known to involve extreme physiological demands, often including high levels of pain.  Although it is expected that extremely physiologically costly exercise contexts will involve activation of neurobiological reward mechanisms outlined above, some contexts may on average exceed (or alternatively never reach) the intensity and duration sweet-spot for optimal activation of neurobiological reward \citep{Raichlen2013} or positive affect \citep{Ekkekakis2011,Reed2006}.

Similarly, while some group exercise contexts do indeed appear to contain high levels of behavioural synchrony (rowing, synchronised swimming, diving, mass calisthenics, and dance such as ballet), exact in-phase synchrony is in fact atypical of most instances of group exercise \citep{Fusaroli2014}.  Interpersonal coordination is more often achieved through flexible, function-specific assemblages of complimentary and contrasting behaviours (for example, coordination in an interactional team sport, a dyadic conversation, or an ensemble music performance).  Real world instances of joint action in group exercise are usually composed of various distinct elements, often organised hierarchically within a sequence \citep{Schmidt1975,Rosenbaum2009}.  Successful execution of the complex structure of real world joint action requires temporal and spatial precision and flexibility of movement across multiple timescales and sensorial modalities \citep{Sebanz2006}.

It is also apparent that physical exercise offers to its participants and observers an opportunity for profound meaning.  Many people do not engage in exercise \textit{just} for health or enjoyment; rather, in some contexts sport forms part of a life of purpose and self-discovery \citep[see, for example][]{Jackson1995,Jones2004,White2011}.  Modern sport has always been much more than ``just a game,'' and instead offers an arena in which virtues and vices are learned, and the ``morality plays''—--of community, nation, or globe—--thus performed \citep{Elias1986,McNamee2008}.  Athletes at the elite apex of their sports commonly report the autotelic experience of ``flow''--—described as full immersion in the ``here and now,'' effortlessness, or optimal experience \citep{Csikszentmihalyi1992,Dietrich2004}.  Whereas the social high theory predicts motivation for exercise based on ``hedonic'' enjoyment, anecdotal and ethnographic perspectives emphasise instead the ethical and moral dimensions of athletes’ experiences, and contextualise these experiences within political processes relating to the construction of the self, community, and nation-state \citep{Alter1993,Brownell1995,Downey2005b,Wacquant2004}.
In many instances, it may be that the primary psychological motivation for exercise is not immediate, reward-induced hedonic wellbeing, but instead \textit{eudaimonic} wellbeing, or the psychological awareness of a process through which life becomes ``well-lived'' \citep{Fave2009,Huta2013}.


Participants in group exercise scrutinise the quality of coordination in joint action.  For their success, technically demanding group exercise contexts such as competitive interactional team sports or music-making and dance, depend upon highly complex coordination of behaviours between individuals.  In these activities, the movements and goals of one individual must align precisely in time and space with the movements and goals of another.  For highly skilled expert practitioners, who develop a fine-grained sensitivity concerning the perceived outcome of joint action, often the ecstasy of group activity is contingent not just on participation, or on strict synchrony or equivalence or similarity of behaviours, but on the extent to which joint action with co-participants ``clicks'' \citep{Jackson1992}.  The psychological literature of optimal human experience (also known as ``flow'' \citep{Csikszentmihalyi1992}) offers extensive documentation of the positive psychological and social effects of successful performance of technically complex movement in individual and, to a lesser extent, joint action.

Nowhere in Durkheim's particular passage is there mention or detail concerning performers' subjective experience of group exercise. We learn nothing from Durkheim about the perceived difficulties associated with learning, or the triumphs associated with successfully performing prescribed schemas for joint action. Nor are we encouraged to empathise wit the anxiety or excitement experienced by first time participants, or the evaluation of the quality performance administered by experienced onlookers.  All of these factors, or at least factors like these, could conceivably impact the social outcomes of the group activity.  More importantly, an explanation of these subjective factors associated with group exercise call for a consideration of proximate causal mechanisms beyond those that are currently considered by the social high theory.  As Adrian attempted to articulate to Mr Shi that first night in Beijing, there is an unmistakably ``visceral'' dimension to the experience of group exercise that is yet to be fully comprehended by current scientific accounts.

Clear variation in the types, intensities, and durations of group exercise, and the complex structure and subjective experience of, and motivation for group exercise presents an opportunity for further research into explanatory cognitive, evolutionary, and social mechanisms underlying these observable phenomena.



% Furthermore, many group exercise contexts appear, on the surface at least, to be more fundamentally defined by pain, coercion, discipline, and even violence, instead of hedonic enjoyment.


\subsection{Joint action in group exercise, beyond strict synchrony}
How is it possible to address these clear gaps in the social high theory?  Foundational to this dissertation is the thesis that existing cognitive and evolutionary understandings of group exercise can be most productively addressed through a closer and more thorough scientific examination of physical movement.  Physical movement is central to the adaptive success of biological life, particularly life for which movement can intentionally directed in order to bring about change in the environment.  The ability of humans to coordinate movement with one another vastly increases the range of their potential actions \citep{Tomasello2009}.  Humans display a rich capacity for complex forms of autonomic and preconceived physical movement, which is employed both when coordinating behaviours with the environment and socially with conspecifics. Scientific understandings of the precise details concerning how humans move together, and why joint action may be crucial to the human evolutionary niche, remain rudimentary.



the need to move beyond synchrony as a stand in for coordination in group exercise.  I concentrate on the cognitive mechanisms and system dynamics that enable and constrain joint action in group exercise, and draw attention to the social and psychological effects known to occur when joint action functions successfully---i.e., when joint action ``clicks.''


In this dissertation I aim to extend the social high theory to include a greater appreciation of the cognitive mechanisms and psychological and social effects of joint action in group exercise that exist beyond moderate intensity physiological exertion, exact behavioural synchrony, and a feel-good social high.



\section{The present study\label{sect:presentStudy}}



\subsection{Introducing the phenomenon of team click\label{sect:teamClickIntro}}

The construct of ``team click'' that I develop in this thesis acts as an empirical anchor to ground a novel theory of social bonding through joint action (outlined in Chapter ~\ref{chap:theory}).  More importantly, I propose the novel construct of ``team click'' as candidate for explaining a hitherto under-examined link between joint action and social bonding.

Team click provides a widely recognised real-world phenomenon that captures aspects of experience in joint action beyond synchrony.  One of the big mysteries of competitive team sport, particularly at the elite level, is the elusiveness of peak team performance.  While each individual athlete may exhibit expert level competence in sport specific skills, the much sought after aggregation of these components, i.e., a team that consistently performs ``in the zone,'' and ``firing on all cylinders,'' in reality often proves frustratingly difficult to achieve and sustain.  As King and De Rond \textcite[568]{King2011} note in their ethnography of the 2008 Cambridge University rowing crew who participated (and who were eventually victorious) in the famous annual Boat Race against Oxford University, the search for collective rhythm is a universal in human social interaction, but  the physiological and psychological complexity of finding that rhythm ``...is extremely difficult to attain; collective performance is a possibility not a certainty.''   The moment in which everything ``clicks'' into place in team sport can, for various reasons, disappear as abruptly as it arrives, if indeed it arrives at all.

But, when team click is somehow cultivated, and even sustained, it is celebrated as the ultimate, albeit often inexplicable magic of sporting feats. Consider Leicester City Football Club's unbelievable outhouse-to-penthouse title run in the 2015 English Premier League, the recent dominance of the Golden State Warriors in the American National Basketball League, or the astonishingly consistent performance of the New Zealand men's national rugby union team (the ``All Blacks'').
  \footnote{The All Blacks are arguably the most successful sporting team ever, with a winning percentage of 77\% in the last 150 years (and 88\% in the last 6 years) \citep{Kerr2013}}.
All of these successful teams carry with them a powerful ``aura'' associated with their capacity to effectively coordinate their behaviours on the field over extended time scales: individual games, seasons, and, in the case of the All Blacks, entire generations.  The aura associated with such rare instances of collective performance can seduce fascination and elaborate exegesis.

%In this thesis I commit to the more banal task of naturalising the phenomenon of team click, by locating it within a novel theory of social bonding through joint action.

For the purposes of this dissertation, I use the term team click to describe the phenomenology of peak performance within a team of athlete engaged in joint action, but the term has potentially broader applications beyond group exercise.  For athletes, coaches, and spectators alike, team click can be a hugely powerful sensation.  As theologian Michael Novak explains, ``[f]or those who have participated on a team that has known the click of communality, the experience is unforgettable, like that of having attained, for a while at least, a higher level of existence'' \citep[11]{White2011}.  It has been extensively documented in the psychological literature of flow and optimal human performance in sport that athletes engaged in team coordination often report total absorption in and complete focus on the task at hand, a transformation of the experience of time (either speeding up or slowing down), and a blurring or transcendence of individual agency, or a ``loss of self''   \citep{Csikszentmihalyi1992,Jackson1995,Jackson1999,McNeill1995}.  Research suggests that flow often occurs in scenarios in which there are clear goals inherent in the activity, as well as unambiguous feedback concerning extent to which goals are either being achieved or not.  In addition, scenarios most conducive to the experience of flow are those in which the technical requirements are challenging but achievable if practitioners are able to extend slightly beyond their normal capabilities\citep{Fong2015}.
The coalescence of these factors is intrinsically rewarding and autotelic\citep{Csikszentmihalyi1975}, activating both ``hedonic'' and ``eudaimonic'' dimensions of subjective well-being \citep{Huta2010,Fave2009}.


%It was just one of those programs that clicked. I mean everything went right, everything felt good . . . it’s just such a rush, like you feel it could go on and on and on, like you don’t want it to stop because it’s going so well. It’s almost as though you don’t have to think, it’s like everything goes automatically without thinking . . . it’s like you’re in automatic pilot, so you don‘t have any thoughts. You hear the music but you’re not aware that you’re hearing it, because it’s a part of it alI.

The experience of flow has by now been extensively studied by psychologists and neuroscientists, from which a series of neuropharmacological \citep{Boecker2008}, neurocognitive \citep{Dietrich2006,Dietrich2011,Labelle2013}, and psychological \citep{Csikszentmihalyi1992} theories for its emergence have been tabled.  The vast majority of flow research has focussed on the experience of the individual---the athlete, musician, or performer.  Some attempts have been made to extended an analysis of flow and its antecedents to the level of the group and dynamics of interpersonal coordination---a phenomenon termed ``group flow'' \citep{Sawyer2006}---but these attempts lack coherence and development.

Team click shares many similarities with the psychological states associated with flow, but is distinct in that it specifically delineates perceptions of joint action from individual action, and therefore implicates physiological, cognitive, and social mechanisms unique to joint action \citep{Vesper2010}, including the complexity systems dynamics associated with participating in a socially-coordinated, multi-agent system of physical movement \citep{Kelso2009}.  Team click is anecdotally present in a wide range of joint action contexts, and is often associated in these contexts with psychological processes of positive affect and wellbeing, as well as personal agency, social affiliation, and group membership \citep{Jackson1995,Marsh2009,Wheatley2012,Slingerland2014}.

In the 1990s, psychologist Susan Jackson conducted a number of qualitative studies documenting athletes' (in this case elite figure skaters) experience of flow in solo and joint action:

\begin{quotation}
  What was really so special about this performance was that there was nothing between (partner's name) and I but flow.  You know, through the whole three minutes.  That meant that her mind and my mind were clear and in the same...in a partnership. It’s always adjustments for ``Am I?'' and ``Are You?'' and ``Where Are We?'' and stuff like that.  That day was really a marriage of (partner) and (self) and the ice.  So I think it’s a different kind of flow experience than a single skater who has much more control over themselves. \citep[173-4]{Jackson1992}.
\end{quotation}

This passage, taken from an interview with a professional figure skater, demonstrates the key elements of distinction between the experience of ``click'' in solo and joint action.  As explained in Section ~\ref{sect:JAbeyondSync}, joint action requires constant finessing (adjustments) of behaviours and intentions between co-participants in which individuals are consulting internal models of self, other, and joint action \citep{Pesquita2017}. This passage thus highlights the cognitive complexity of joint over individual action, suggesting the cost and uncertainty associated with attending to and attempting to control the movement df of others in addition to one's own.  ``Click'' in joint action, by contrast, may represent jitter-less ``co-confidence'' in which fixation on the contributions of you and I are momentarily abandoned or down regulated, and the agency of ``we'' is realised \citep{Gallotti2013,Noy2017}.

Importantly, team click appears to have important flow-on consequences relevant to social bonding and affiliation. Tightly synchronised activity in particular, found in team sports such as rowing, can help dissolve the boundaries between individual and social agency: ``In rowing...it feels like you have at your command the power of everybody else in the boat. You are exponentially magnified. What was a strain before becomes easier. It is absolutely the ultimate team sport'' \citep{Brown2016}.
The blurring of agency between self and team may be responsible for facilitating affiliation and trust between teammates in competitive athletic environments such as professional rugby, which often involves high physiological stress and uncertainty.  As Jeremy Paul, a famous Australian Rugby player commented of his late teammate Dan Vickerman ``...you always wanted a guy who would go into the trenches with you and he always played consistently well...he could really play and was just one of the good lads that you enjoyed his company'' \citep{Fox-Sports2017}.  In this sense, the experience of team click may act as a social diagnostic tool, a powerful signal of commitment to joint action and willingness to cooperate \citep{Reddish2013a}. \\
\\
\\

\noindent\fbox{%
    \parbox{\textwidth}{%
Team click refers to the perception of peak team performance.  Based on anecdote and evidence from psychology and anthropology, an individual's perception of team click should contain some or all of the following components:
    \begin{enumerate}
      \item Flow or coherence of joint action
      \item Tacit understanding between co-actors
      \item Atmosphere or aura around team performance
      \item Teammates are responsible for extending individual ability
      \item Teammates are reliable co-actors
      \item The individual is a reliable co-actor for teammates
    \end{enumerate}
    }%
}

\\
\\
\\

%SURPRISING:
%Oh, it’s awesome! Everything’s at peace . . . the crowd was involved, we were involved as a pair team, and we were just aware of everything, it just kept snowballing, like we weren’t going to miss, and it was great, it was an awesome experience.


As summarised above, team click is a phenomenon distinct to joint action.  Team click involves an experience of pleasurable flow or coherence of joint action, tacit (rather than explicit) understanding between co-participants, a degree of aura or positivity around joint action, a blurring of agency between self, other and group (or ``we''), and perception of reliability of others and self in performing their roles. Importantly, team click derives from instances of joint action defined more broadly than exact behavioural synchrony. Team click thus offers a powerful empirical anchor for exploring a broader relationship between joint action and social bonding in group exercise contexts.



\subsection{Contributions of the present study\label{sect:contributions}}



% Please add the following required packages to your document preamble:
% \usepackage{booktabs}
\begin{table}[]
\centering
\begin{tabular}{@{}llcl@{}}
\toprule
\textbf{} & \textbf{Components}                                                                               & \multicolumn{1}{l}{\textbf{Chapters}} & \textbf{Description}                                                                                                                                         \\ \midrule
          &                                                                                                   & \multicolumn{1}{l}{}                  &                                                                                                                                                              \\
1         & \begin{tabular}[c]{@{}l@{}}A novel theory of social\\ bonding through joint\\ action\end{tabular} & Ch. 2-3                               & \begin{tabular}[c]{@{}l@{}}Supported by an active inference framework; \\ proposes team click as a mediating construct\end{tabular}                          \\
          &                                                                                                   &                                       &                                                                                                                                                              \\
2         & Ethnography                                                                                       & Ch. 5-7                               & \begin{tabular}[c]{@{}l@{}}Beijing men's rugby team (n = 26); \\ Data: participant observation, \\ informal surveys, semi-structured interviews\end{tabular} \\
          &                                                                                                   &                                       &                                                                                                                                                              \\
3         & \begin{tabular}[c]{@{}l@{}}\textit{In situ} survey \\ study\end{tabular}                                   & Ch. 8                                 & \begin{tabular}[c]{@{}l@{}}Conducted during the The National 7s Rugby\\ Championship (n = 174, male = 93)\end{tabular}                                       \\
          &                                                                                                   &                                       &                                                                                                                                                              \\
4         & \begin{tabular}[c]{@{}l@{}}Controlled field\\  experiment\end{tabular}                            & Ch. 9                                 & \begin{tabular}[c]{@{}l@{}}Conducted with athletes from Beijing \\ and Shandong provincial rugby \\ programs, (n = 58, male = 30)\end{tabular}               \\ \bottomrule
\end{tabular}
\caption{Core components of this dissertation}
\label{tab:thesisComponents}
\end{table}






\subsubsection{A novel theory of joint action through social bonding}

These predictions set the foundation for a particular study focussed on the relationship between joint action and social bonding in the case of professional Chinese rugby players. The specific ethnographic context (sport in China) demands a careful consideration of the predictions formulated above.  Culturally specific processes of self-construal and social group formation challenge some of the assumptions built in to the literatures mentioned above.  However, I argue that the predictions outlined above are robust to these cultural specificities, due to the fact that they are grounded in an agent neutral distributed social cognition framework. Indeed, despite distinct cultural variation processes of team membership, ethnographic analysis reveals that the experience of team click is strongly identifiable.

In order to test these predictions, I conduct a series of three interrelated studies.

\subsubsection{Ethnography of the Beijing men's rugby team}
The first study consists of extended ethnographic research with one Chinese professional rugby team, the Beijing Provincial men's rugby team (see Chapters ~\ref{4partAIntroMethod}\nobreakdash~\ref{6ethnographicResults}).

\subsubsection{\textit{In situ} survey study of a Chinese National
 Rugby Tournament}
 From this ethnographic starting point, I then broadened the scope of analysis to include all available Chinese professional provincial rugby players. Participants for the second study ($n = 174, men = 93$) were athletes in a Chinese national tournament, in which fifteen different teams from nine different provinces competed over two days for the 2016 Championship (Chapter ~\ref{5ethnographicField}).  The tournament offered the opportunity to investigate hypotheses concerning the relationship between joint action, team click, and social bonding \textit{in situ}, in a real-world instance of high intensity, high stakes joint action.

\subsubsection{Controlled field experiment}
  Subsequently, in order to more definitively assesses the causal mechanisms identified in the predictions of this dissertation, I conducted a controlled field experiment ($n = 58, male = 29$). In a between-subjects design, I manipulated the level of predicted difficulty prior to athletes participating in ostensibly different (but in fact identical) training drills.

%\subsection{Contributions}




\section{Overview of chapters\label{sect:chapters}}
In this introductory chapter, I outline the overarching research question of this dissertation, which is a scientific explanation for the ubiquity of group exercise in human sociality.  I explain that, while the social high theory of group exercise and social bonding has shed light on important causal mechanisms involved in many group exercise contexts, it contains obvious knowledge gaps, owing in part to a bifurcation in scientific approaches to sport and exercise.  I hone in on team click as an observable phenomenon in group exercise worthy of further theorisation and empirical investigation, due to the way in which it appears to involve a relationship between interpersonal movement coordination and social bonding.  I identify the social cognition of joint action as a field of research in which novel theoretical predictions regarding the relationship between group exercise and social bonding can be formulated.  I preview this theoretical formulation and outline this dissertation's three main empirical studies and their knowledge contribution to cognitive and evolutionary anthropology of group exercise.  In the following chapter, I outline in detail a novel theory of social bonding through joint action, in particular the mediating role of the phenomenon of team click.




                                              \end{CJK}{UTF8}{gbsn}
