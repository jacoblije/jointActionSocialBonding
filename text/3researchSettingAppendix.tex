\chapter{\label{app3:researchSetting}Appendix: Research Setting}

\begin{CJK}{UTF8}{gbsn}






  \section{Qualifications and position of the researcher\label{app3:qualPositionResearch}}

  Before arriving in Beijing in 2015 to begin my doctoral research, the last time I was in China was two years earlier in 2013, when I spent eight months coaching the Chinese men's youth rugby 7s team in the lead up to the Nanjing Asian Youth Olympics.  Before that, I had spent one year studying on Exchange at Beijing University in 2008, and another year before that on an intensive Chinese language course at Liaoning University, Shenyang, in 2006---my first trip to China.  Rugby featured heavily in both instances.  In 2006, an Australian classmate and friend Ed had caught wind of the fact that there was a rugby program down the road from Liaoning University at the Shenyang Sports College (SSC).  Despite the fact that we had both been diligently attending class and courageously deploying our elementary Chinese to order food at restaurants and befriend local taxi drivers, Ed and I were, nonetheless, three months into our intensive language exchange and feeling that our Chinese skills were floundering.  We suspected that this was in large part due to the fact that we had met very few local Chinese people our age.  So one afternoon we rode our bikes over to the Shenyang Sports College in time for the rugby team's afternoon training session.  Less than six months later, we were boarding an overnight train from Shenyang to Shanghai with the SSC rugby team to compete in the annual Shanghai Rugby 7s Tournament.  We had become closely integrated into the community of rugby athletes at SSC, due in part to the common language of rugby that we all shared, and perhaps mostly due to the overwhelming hospitality of the SSC athletes and coaches.  The decision to find the rugby team may have also helped us improve our Chinese. Ed and I were the only two in our cohort to finish the year in Shenyang with a Level 6 in the Chinese Proficiency Exam, which qualified us to study alongside Chinese local students at an undergraduate level.

  Buoyed by this experience with the SSC rugby team in 2006, I followed a similar template two years later when I arrived at Beijing University on exchange from Sydney University to study sociology at Beijing University.  I had just finished working at the 2008 Beijing Olympics. At that time in Beijing, the only Chinese rugby program was based at the Chinese Agricultural University, a forty minute cycle north of Beijing University.  The CAU rugby program was the strongest in the country: CAU consistently outperformed its rivals at the time (Shanghai Sports Institute, the People's Liberation Army, and SSC) and it was awarded with the responsibility of hosting the Chinese national team.  When the International Olympic Committee announced in late 2009 that rugby would be played in the 2016 Rio De Janeiro Olympics, it was subsequently decided in 2010 that rugby would be inducted into the state sponsored sports system and played in the next Chinese National Games in 2013.  Following this announcement, many of CAU's athletes and coaches dispersed to various professional provincial rugby programs, the main ones being Beijing and Shandong.

  %It was during my time training and generally ``hanging out'' at CAU that I met and developed a strong friendship with Kai, who was at the time playing for CAU and China, while also finishing a Master's degree in Labour Law.  I also met and developed relationships with many rugby players, coaches, and general fans of the Beijing rugby community.

  Between 2009 and 2013 I returned to Australia to finish my undergraduate degree, during which time my own rugby career also rapidly developed.  After a successful season in the Sydney Premiership competition in 2009, I was selected to play for the Australian Rugby Sevens Team. I represented Australia from 2009 through to the end of 2012.  In 2013, during the 9 month gap between my Australian rugby contract ending and the start of my graduate studies at Oxford University, I returned to China to coach the Chinese Youth Men's 7s program in their lead-up to the 2013 Nanjing Asian Youth Olympics.  Along with a small team of Chinese coaches and management, I coached a core group of roughly 25 athletes aged between 15 and 18 years old. We trained 6 days a week for approximately 6 months, with only occasional breaks for National holidays, or for athletes to return to their home provinces to complete compulsory exams.  The program was based predominantly in Anhui province, and we travelled from Anhui to other provinces further afield to find suitable practice opportunities against provincial programs.

  This combination of experience in China and expertise with rugby enabled me privileged access to this particular research setting.

  %Soon after the completion of the Asian Youth Olympics in Nanjing in 2013, the Chinese National Games were held in Shenyang. Rugby was played---for the first time in National Games history---with dramatic consequences, explained above.














































































































\end{CJK}
