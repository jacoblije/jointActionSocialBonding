

Semi-structured Interview Script

Introduction:
- Brief explanation of my research
- When was the first time you came into contact with rugby?
- Where are you from?
- History with sport before rugby?

Family:
What does your family think about rugby?
Do they support you playing rugby?
Whose decision was it to start playing rugby?
Do they worry about you getting injured?

Perceived Costs Of Rugby:
Opportunity Cost: what would you be doing if you weren’t here playing rugby?
How do you feel about the fitness requirements of rugby?
What is the feeling like when you’re out on your feet and can’t go on?
Injury: Have you had any major injuries?
What do you think is the hardest thing about rugby?

Perceived Benefits of Rugby:
Do the following motivate you to play rugby? (counter-balanced order)
- Education
- Your parents / Family (i.e., to make them proud/content)
- Beijing Residency
- Future Employment
- Teammates
- Earn Respect from (from society, family, friends)
- Find a girlfriend
- Fun
- Represent Beijing

What is something new that you have learnt through rugby?

Team Membership:
What role do you play in the team?
What is the most important thing for you to do in your current role in the team?

Dissonance/Personal Shortcomings?
Have you ever felt like you have let the team down?  (neijiu 内疚: failure to uphold obligation to another)
In what situations do you feel like that?
How do you react to those feelings?

Flow/team click:
Have you ever experienced the team playing extremely well together; everything clicking, like everyone on the field has a “tacit understanding” of each other (moqi 默契)?
When was this experience?
What did it feel like?
How do you think that “team click” can be achieved?

Final Activities:
Sorting Task: rank your motivations for playing rugby
Social Network: write down:
1.	The three most competent rugby players in the team
2.	The three people in the team most willing to help others
3.	Your three closest friends in the team

 
