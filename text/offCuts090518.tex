
  %HXL: return to his period of absence from the team after final tournament.
  %\myparagraph{The ``team'' in hierarchical relationalism}
  %Jing's speech, official
  %Example: Last fitness session ``一团一团!''
  %Performed for me, for the cameras...


  %\footnote{This assertion was probably true for XG; LJX's story was a bit more complex. LJS did appear He had a tough run with injury, and had persisted for some time through a number of painful shoulder injuries, noticeably stressed by the situation).

  %E: Any sanctity around the team appears secondary to more immediate strategic goals

  %P: Power of regulation:
  %F: BYH, SHL, and WCY privileged position: contrast with HXL and LP (and Zhu).
  %E:Setting up regulations in institutions to incentivise and constrain, but not something that generates trust in the category of the institution itself.

  %F: Zhu --> WCY transition:
     %WWX criticism of Zhu; subsequent alignment with WCY & ZJ


  %Other Examples:- 不要站错队 -- Zhu Jing trying to organise relational factions within the team F: HXL with contract and residency permits (HXL's period of withdrawal) F: Zhu when SHW came in above his control.


  %F: motivations for personal strategic life-course opportunities, more explicit than teammates
      %Rugby/Employment: Old heads (1st team) and undergrads (BYH, MXK etc.)
      %junior: Young bucks (Chaoyang crew) and freshers (SHW etc..)
        %family strong, team weaker but still present; coach??



             %Wang Zhengfeng
             %“ You know, compared to before I think the team atmosphere and culture is much more unified. Before, there still a few players who played really well and whatever, but I think its like we’re unified, I feel like our thinking is all the same, there is no gap, I feel like we’re all very open” (3) HANG OUT TOGETHER, not small units like before (pre 2013)	:我进入橄榄球的时候其实没有这种感觉。打篮球什么的没有这种感觉。就是我们又一批人在一块儿,玩的比较容下,也比较合得来, 师哥也比较好,不像那些体育队的一些师哥对你比较不好啊,师哥对师弟都特别照顾,环境这个团队项目跟其他的团队项目还不一样。也可能别的球队不一样,但是北京队这个省队,橄榄球气氛、文化比较好。  (2)  :比以前应该是更团结了吧。之前还是有一些人玩的特别好啊什么的,但是觉得现在就是跟统一的一样,感觉思想是一样的,没有什么隔阂,感觉玩的特别开,(3) :也有可能是老跟老的有一些玩的特别好, 但是打球的时候都会一条心,但是平时会有时候是,我跟谁特别好我们是天天玩。现在都是我们差不多大的了,玩就是一起玩,都关系特别好.

             %3) Even COMPETITION is a PASSIVE MOTIVATION, a more active stance would be “I love rugby, I like it, so I want to do it.”	: 然后这个造成你说的那个主动的状态?:竞争力和动力是比较被动的,更主动的是,“我热爱橄榄球,我喜欢,我想去做好。”

             %E: a cognitive strategy in reorganising information in a way that maintains coherence (Nowak 2017)


             %MHT returning it to what he does:中国人就属于那个破车的,就没事修理修理。我看见你,好好做,你一回头,他们就不做了。还有一点就是,我为什么老去做力量,自己认真的,因为我受伤了,现在我特别注重,我觉得自己不练,就是加力量什么的,做完五六次自己切,我会自己去凑空做我要做的(exercises),因为受过伤知道你身体是本钱,我身体不好,能力不行,提不上来,我会在球队慢慢没有位置了。我真么说你就知道了,他们(年轻的)没有危机感,





3researchSetting:

%In the case of group exercise contexts, therefore, it is essential to thoroughly consider both 1) the local parameters of joint action associated with the group exercise context, as well as 2) the global ecological and cultural frames in which the group exercise context is situated.  In this dissertation I concentrate on the empirical case of rugby union in contemporary China.  In order to provide a sufficient contextual grounding for the three empirical studies that follow, in this Chapter I introduce 1) the parameters of joint action associated with rugby union, 2) important psychological and cultural factors relevant to social cognition of joint action in China, and 3) a history of physical activity and sport in China, from which rugby in China has emerged and in which it remains situated.
