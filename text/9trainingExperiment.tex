

\begin{savequote}[8cm]

  \qauthor{}
\end{savequote}



\chapter{\label{chap:trainingExperiment}A controlled study of uncertainty and expectation violation in an Invasion Drill}



                                          \begin{CJK}{UTF8}{gbsn}







\minitoc


\section{Introduction}
The perception of click in social interaction is commonplace and not limited to group exercise contexts such as sport, music, or dance.  Even interactions in sedentary work place contexts can be the site of click, chemistry, and connection between co-workers.   But do these day-to-day activities reliably generate the visceral and socially agentic perceptions of joint action that are characteristic of team click?

Findings from the the National Tournament study (Chapter~\ref{chap:chap:tournamentSurvey}) found that team click mediated a positive relationship between perceptions of team performance and social bonding in a real-world joint action setting in which uncertainty was assumed to be high.  This study sought to manipulate uncertainty in a rugby training drill to examine the effects of uncertainty on athlete expectation violation in team performance, perceptions of team click, and social bonding.

It was hypothesised that higher levels of uncertainty in joint action would generate lower expectations for success in team performance (Hypothesis 2.a), and that, in turn, more positive perceptions of team performance relative to prior expectations would lead to higher levels of team click, assuming that levels of success in the training drill would be more or less constant (Hypothesis 2.b).  In addition, it was expected that this more positive violations of expectations concerning team performance would set the foundation for higher levels of social bonding.

\myparagraph{The Invasion Drill}
A between-subjects experimental design was used, in which athletes participated in a typically encountered rugby training drill known as the ``Invasion Drill’’ \citep{Passos2011}.  The Invasion Drill emulated a sub-phase of attack and defence in rugby.  The training drill was preceded by one of two experimental primes designed to manipulate athlete expectations of uncertainty concerning joint action.  In the ``high difficulty'' condition, athletes were primed to believe that they would be assessed on their ability to perform a technically challenging training drill; athletes in the ``low difficulty'' condition were primed to believe that they would be assessed on their ability to perform a technically straightforward rugby training drill.  In fact, all athletes performed the exact same drill, only the prime differed between groups.

The experimental manipulation was designed to encourage athletes in the high difficulty condition to generate lower expectations for successful team performance in joint action than athletes in the low difficulty condition. Assuming that actual performance in the subsequent training drill would be constant (the drill was pre-rated by another group of athletes as a moderate difficulty drill), it was predicted that athletes in the high difficulty drill should on average formulate lower expectations concerning success in team performance, as well as experience more positive violations of expectations than athletes in the low difficulty drill.   As with the previous study, this study also sought to confirm the role of positive expectation violation in generating team click (Hypothesis 2.b).  In addition, it was predicted that higher perceptions of success in team performance relative to expectations would generate higher levels of social bonding, to the immediate training group, as well as the provincial team as a whole

Design features of this study aimed to encourage athletes to consult their personal perceptions of team performance.  Overt feedback concerning performance was restricted to athletes' perceptions of each phase of the Invasion Drill.   there were no explicit winners or losers, or immediately obvious benefits or costs to incur through participation.  The drill required only moderate levels of physiological exertion, and physical contact between athletes in the Invasion Drill was limited to ``grab'' as opposed to full-contact.  Thus, in contrast to the   National Tournament study, this study sought to control many extraneous factors that could impact on athletes' perceptions of joint action (see Chapter~\ref{}).

\myparagraph{Survey measures}
Athletes were surveyed at 3 time points: once prior to the test day to collect baseline measures, once immediately before, and once immediately after the training drill.  Survey measures were designed to assess athletes’ perceptions of uncertainty in joint action (measured by confidence in team and self to meet the technical challenges in the impending training drill.

To check the effectiveness of the experimental manipulation, athletes were asked to report their confidence their own and their group's ability to meet the technical challenges of the impending training drill.  It was expected that athletes in the high difficulty condition would be less confident about their own and their group's ability to meet the technical challenges of the training session.

Employing a similar rationale to the survey study (see Chapter~\ref{sect:introSurvey}), team and individual performance were delineated as separate variables in order to isolate athlete perceptions of team performance relative to expectation violations.  In the pre- and post-drill surveys, athlete were asked to report perceptions of team performance with the specific training group with whom they completed the drill.

Team click was measured using the same collection of items described in the previous chapter (see~\ref{}).  In the pre- and post-drill surveys, athlete were asked to report perceptions of team click with the specific training group with whom they completed the drill.  Social bonding was also measured using the same items described in the previous chapter (see Chapter~\ref{}).  In addition to perceptions of social bonding to the specific training group measured pre- and post-drill, athletes were also asked—at baseline and post-drill about their perceptions of social bonding to the provincial team as a whole.  The inclusion of this measure offered the opportunity to test whether team click experienced with a small group of teammates had implications for more generalised perceptions of social bonding to the group as a whole \citep[see, for example][]{Reddish2013a}.  In essence, the proposal advanced in this thesis---that team click serves as a template for identification with higher-level representations such as group identity---could be more directly tested.

Together, these items were designed to operationalise hypothesised relationships between uncertainty in joint action, expectations and violations of expectations concerning perceptions of team performance (relative to prior expectations), team click, and social bonding.  In addition to these items, measures of technical competence and personality type were collected at baseline, and arousal and injury status were recorded before and after the drill.  Objective measures of performance in joint action were derived from video footage.  Two subsets of the collected data (Post-Tournament, PrePost Tournament) were analysed for relationships between variables of interest within individuals and over time.  Clustering of responses according to groupings such as experimental session, provincial team, and was assessed and controlled for statistically.

Specifically, this study tested the predictions outlined in Table~\ref{}.  In essence, this study tested the claim that when higher levels of uncertainty in joint action (e.g., when athletes are ``under invasion’’), they will feel the click of social connection when performance pans out better than originally expected.


\input{images/trainingExperimentPredictions}




\clearpage
\section{Method}

\subsection{Participants}
64 professional Chinese rugby players were recruited for the study ($M(age) = 21.33, SD = 3.33, range = 16-29, men = 32$,).  Athletes were recruited from two provincial rugby programs, 32
participants were athletes from Shandong province ($M(age) = 22.1, SD = 2.32, men = 15$) and the remaining 32 athletes were from Beijing province ($M(age) = 20.4 SD = 3.11, men = 16$).  This study was approved by the University of Oxford's Central University Research Ethics Committee (SAME/CUREC1A/15-059).


\subsection{Materials}


\begin{figure}[htbp]
  \centering
      \includegraphics[width=0.9\linewidth,keepaspectratio] {images/invasionDrill}
      \caption{The Invasion Drill: a common training drill for rugby union in which a group of 4 attackers attempt to penetrate two consecutive lines of defence. Adapted from Passos (2011).}
      \label{fig:invasionDrill}
  \end{figure}


\subsubsection{Experimental Paradigm: The Invasion Drill}
This study required athletes to participate in a rugby training drill---known as ``Invasion Drill'' \citep{Biscombe1998}---which was preceded by one of two experimental primes—--a ``high difficulty'' condition, or a ``low difficulty'' condition.  This training drill was selected as it was representative of a typical subphase of joint action in rugby union and it has been successfully used to measure dynamic coupling between rugby players in joint action \citep[see][]{Passos2011}.  In this drill, a group of 4 rugby players form an attacking group and two pairs of opponents form a first and second defensive line.  The drill had two performance aims: (a) attackers were asked to carry the ball into the try area, and (b) defenders were asked to stop the progression of attack toward the try line.
One trial consisted of one attempt by the attacking sub-group to penetrate two defending sub-groups, and carry the ball over the try line. The drill was conducted on a regulation multipurpose 110m x 70m grass or artificial turf training field within a 22m x 15m rectangle area marked by plastic cones (see Figure ~\ref{fig:invasionDrill}). The ball used was size 5, as recommended by World Rugby for this age group of athletes.\footnote{During my time conducting research and coaching in China, I did not commonly witness the Invasion Drill as described herein in training scenarios. As such, I judged that the Invasion Drill would be suitable for this experiment because the technical requirements of the drill were be suitably familiar to athletes (i.e., a typical subphase of joint action in rugby), but the drill itself would be suitably novel.}


%The training drill, known as ``Invasion Drill,'' requires 8 athletes, one sub-grouof 4 attackers and two sub-groups of two defenders.  The primary aim of the drill is for the sub-group of 4 attacking players to successfully penetrate two consecutive lines of two defenders (also commonly known as a ``4 on 2 + 2'' drill). The aim of the two defending sub-groups is to stop the attacking team from achieving the primary goal, by interfering with their coordination by halting the ball-carrier, or the ball in flight between attacking players.


\subsubsection{Experimental Conditions: high and low difficulty\label{sect:expPrimes}}
Two different experiment primes were developed in order to manipulate athletes' expectations around the technical difficulty of the impending joint-action task.  Athletes in the ``high difficulty'' condition learned that the overall average difficulty rating for the training drill, provided by World Rugby coaches and athletes, was 77.5/100, or approximately 8 out of 10.  Athletes were told that the drill would require an extension of their abilities as individuals and as a training group.  By contrast, athletes in the ``low difficulty'' condition learned that they would be participating in an ostensibly different drill with an overall average difficulty rating of 22.5/100, or approximately 2 out of 10 (see Appendix~\ref{app9:conditionPrime} for full script).  In reality, the training drill was exactly the same for all experimental sessions.  Based on a pilot study with 20 athletes from the Beijing men's and women's rugby programs who did not participate in the study, the actual difficulty of this drill was estimated to be approximately 5/10 (see Appendix~\ref{app9:difficultyPilot} for full explanation of pilot study).  The experiment primes were designed to encourage athletes to either over or under rate the difficulty of the impending training drill. If successful, the prime would alter athletes' expectations around certainty of performance, such that athletes in the high difficulty condition would enter the training drill with expectations for performance that were on average lower than athletes in the low difficulty prime. It was predicted that athletes in the high difficulty condition would experience more positive perceptions of team performance relative to prior expectations, as compared to athletes in the low difficulty condition, who would on average experience experience less positive violations of expectations around team performance.


\subsubsection{Measures}


\myparagraph{Surveys}
Surveys were administered at 3 time points: Baseline ( approximately 24 hours before the experiment), Pre-Drill (immediately before the training drill and after receiving the experiment prime), and Post-Drill (immediately following the completion of the drill; see ~\ref{tab:surveyMeasureSummaryTable}). Consistent with the previous study (Chapter~\ref{chap:tournamentSurvey}), athletes responded to questions about 1) \textit{performance} (including perceptions of individual and team performance relative to prior expectations, and perceptions of components of individual and team performance), 2) \textit{team click} (feelings associated with team click in joint action), and (3) \textit{social bonding} (perceptions of social bonding and group membership).  Additionally, athletes responded to questions designed to measure \textit{moderator variables} such as technical competence, personality type, injury status, arousal, and fatigue. Measures of athlete confidence in group and self to meet the technical challenges of the Drill would be used to check the experimental manipulation (see Section~\ref{sec:manChecksDrill}).


    % Please add the following required packages to your document preamble:
% \usepackage{booktabs}
\begin{table}[]
\centering
\begin{tabular}{@{}rlcc@{}}
\toprule
\multicolumn{1}{c}{Items} & \multicolumn{1}{r}{Baseline} & Pre-Drill & Post-Drill \\ \midrule
\multicolumn{1}{c}{\textbf{Training Group}} & \multicolumn{1}{r}{} &  &  \\ \midrule
\begin{tabular}[c]{@{}r@{}}Confidence in group \\ to meet technical \\ challenges\end{tabular} &  & \cmark & \multicolumn{1}{l}{} \\
\multicolumn{1}{l}{} &  & \multicolumn{1}{l}{} & \multicolumn{1}{l}{} \\
\begin{tabular}[c]{@{}r@{}}Group performance \\ vs prior expectations\end{tabular} & \multicolumn{1}{r}{} &  & \cmark \\
\multicolumn{1}{l}{} &  & \multicolumn{1}{l}{} & \multicolumn{1}{l}{} \\
\begin{tabular}[c]{@{}r@{}}Components of \\ performance\end{tabular} & \multicolumn{1}{r}{} & \cmark & \cmark \\
\multicolumn{1}{l}{} &  & \multicolumn{1}{l}{} & \multicolumn{1}{l}{} \\
Team Click &  & \cmark & \cmark \\
\multicolumn{1}{l}{} &  & \multicolumn{1}{l}{} & \multicolumn{1}{l}{} \\
Social Bonding &  & \cmark & \cmark \\ \midrule
\multicolumn{1}{c}{\textbf{Individual}} &  & \multicolumn{1}{l}{} & \multicolumn{1}{l}{} \\ \midrule
\begin{tabular}[c]{@{}r@{}}Confidence in self\\ to meet technical \\ challenges\end{tabular} &  & \cmark & \multicolumn{1}{l}{} \\
\multicolumn{1}{l}{} &  & \multicolumn{1}{l}{} & \multicolumn{1}{l}{} \\
\begin{tabular}[c]{@{}r@{}}Individual performance\\  vs prior expectations\end{tabular} &  & \multicolumn{1}{l}{} & \cmark \\
\multicolumn{1}{l}{} &  & \multicolumn{1}{l}{} & \multicolumn{1}{l}{} \\
\begin{tabular}[c]{@{}r@{}}Components of \\ individual performance\end{tabular} & \multicolumn{1}{r}{\cmark} & \cmark & \cmark \\ \midrule
\multicolumn{1}{c}{\textbf{Team (Province)}} & \multicolumn{1}{r}{} &  &  \\ \midrule
Team Performance & \multicolumn{1}{r}{\cmark} &  & \cmark \\
\multicolumn{1}{l}{} &  & \multicolumn{1}{l}{} & \multicolumn{1}{l}{} \\
Team Click & \multicolumn{1}{r}{\cmark} &  & \cmark \\
\multicolumn{1}{l}{} &  & \multicolumn{1}{l}{} & \multicolumn{1}{l}{} \\
Social Bonding & \multicolumn{1}{r}{\cmark} &  & \cmark \\ \midrule
\multicolumn{1}{c}{\textbf{Moderators}} &  & \multicolumn{1}{l}{} & \multicolumn{1}{l}{} \\ \midrule
Arousal & \multicolumn{1}{r}{} & \cmark & \cmark \\
Exertion & \multicolumn{1}{r}{} &  & \cmark \\
Injury & \multicolumn{1}{r}{} & \cmark & \cmark \\
Personality & \multicolumn{1}{r}{\cmark} &  &  \\
Technical competence & \multicolumn{1}{r}{\cmark} &  & 
\end{tabular}
\caption{Survey items measured at Baseline, Pre-Drill, and Post-Drill.}
\label{tab:surveyMeasureSummaryTable}
\end{table}


\myparagraph{Baseline}
Baseline measures were obtained 24 hours before the start of the experiment by asking athletes about their impressions of recent individual performance and the performance of their provincial squad as a whole.  Survey items included questions relating to specific components of team and individual performance (e.g., team components: attack, defence, on-field communication, and support play; individual components: passing technique, one-on-one tackling, effectiveness in contact areas), as well as items relating to overall impressions of performance, for example: ``How well do you feel your team has been performing in training and competition over the past month?'' (100 point scale, 0 = ``Extremely bad'', 100 = ``Extremely good'').  Athletes also responded to questions about perceptions of team click and social bonding with their team as a whole.  In addition, the Baseline survey also included items concerning basic personal information, current injury status, subjective and objective measures of technical competence, and personality type.  For a full explanation of the Baseline survey, see Appendix ~\ref{app9:trainingExperiment} Section ~\ref{app9:surveyItemsBaseline}.


\myparagraph{Pre-Drill}
Immediately before each experiment session, athletes were asked questions about feelings and expectations concerning their individual performance and the performance of their specific training group in the impending training drill.  For example, athletes were asked ``How do you feel \textit{right now} about your individual performance'' (100 point scale, 0 = ``Extremely bad'', 100 = ``Extremely good'').  Athletes also responded to items asking about confidence in self and the training group to meet the technical challenges of the drill, i.e. ``How confidence are you in your own / your training group's ability to meet the technical challenges of the upcoming drill'' (100 point scale, 0 = ``Not at all confident'', 100 = ``Extremely confident'').

Athletes were also asked a series of questions about perceptions of team click and social bonding to their specific training group. For team click, for example, athletes were asked questions like: ``How do you feel the tacit understanding is between the training group today?'' (100 point scale, 0 = ``Extremely bad'', 100 = ``Extremely good'').  For social bonding, athletes were asked questions like: ``How emotionally supportive does your training group feel right now?'' (100 point scale, 0 = ``Extremely weak'', 100 = ``Extremely strong'').

%See Appendix~\ref{app9:surveyItemsPre} for an outline of the Pre-Drill survey.

\myparagraph{Post-Drill}
Immediately upon completion of the drill, athletes responded to questions concerning their impression of their individual performance and the performance of their training group in the training drill. A  single item measure asked about individual and team performance relative to prior expectations: ``Overall, how do you feel about your training group's performance in training today?'' (100 point scale, 0 = ``Much worse than expected'', 100 = ``Much better than expected'').  Athletes were also asked about their feelings about team click and social bonding to the training group.

In addition, athletes were asked to answer survey items designed to measure social bonding to the provincial team as a whole (these team-focussed questions replicated the survey items that were asked one day earlier at Baseline).  See Appendix ~\ref{app9:trainingExperiment} Section ~\ref{app9:surveyItemsPost} for an outline of the Post-Drill survey.
%~\ref{} Section ~\ref{}.

Surveys were designed and administered using Qualtrics software (Qualtrics version 9, Provo, UT). The surveys (initially designed in English) were translated into modern Chinese by the researcher and back translated by two independent native Chinese speaking translators from Beijing Sports University to verify accuracy.  Athletes completed the modern Chinese version of each survey using WeChat on their personal mobile devices connected remotely to the internet. Survey data were processed and analysed using the R environment (R Development Core Team, 2006).

\subsubsection{Video analysis of training team performance in training drill\label{sec:videoAnalysis}}
In order to produce quantitative measures of interpersonal coordination between co-actors, athletes' motion was captured by a single digital video camera (Sony FDR-AX700 4K HDR Camcorder) mounted on a 1.2m high tripod.

The digital camera and tripod were positioned on a platform 2m above the level of the playing field, approximately 10-15m from the bottom try-line corner of the Invasion Drill perimeter (see Figure ~\ref{fig:invasionDrill}). The experimenter began video recording before the athletes arrived, and ceased recording after all athletes had left the training field following the experiment. Digital video images of action were acquired by a computer and files were saved on an encrypted external hard disk in .AVI format.

%For image treatment, TACTO 8.0 software was used for digitising at 25 frames per second.

Video footage from all 8 experiment sessions was analysed by the researcher as well as an hypothesis blind research assistant familiar with the sport of rugby.  Each experimental trial of the Invasion Drill (16 in total) was coded according to success of the group in the primary (attacking) goal of the drill, i.e., scoring a try by carrying the ball over the try line.  Two measures were constructed: a binary try or no try (``Try''; 1 = ``try'', 0 = ``no try'') measure, and a measure designed to reflect the quality of the outcome of each trial (``Trial Outcome''). For this second and more fine-grained measure, a maximum value of 6 was awarded to a ``clear try'' (trials in which a clear unobstructed try was scored by the attacking sub-unit); 5 to a ``rough try'' (the ball-carrier crossed the try-line following some minimal level of physical contact or obstruction from the defence that did not halt the momentum of the attacking phase); 4 to an ``Obstructed try'' (a trial in which a try is scored, but the momentum of the attacking sub-unit was clearly interrupted); 3 to a ``no try - defence'' (a trial in which a ball carrier is completely obstructed by a defender, and the trial is unsuccessful largely due to the effectiveness of the defence); 2 to a ``no try - defence forced error'' (unsuccessful trial due to a combination of the defence and error in attack); 1 to a ``no try - unforced error'' (a failed trial due to an unforced error in attack, such as an athlete passing the ball forward, or dropping the ball without obvious pressure from the defence).  For each experiment session of 16 trials, the total number of successful trials was calculated (``Tries Total''), as well as the average quality of performance outcome (``Trial Outcome Average'').

%In addition, the the standard deviation of this mean outcome (``Trial Outcome SD'') was calculated in order to capture the variance of performance for each training group.

The researcher briefed the hypothesis blind research assistant using 10 trials randomly selected from the warm-up trials of each experiment session. Owing to this detailed briefing and the research assistant's familiarity, inter-rater reliability, measured using Cohen's (1960) kappa \citep[suitable for two coders, see][]{DiEugenio2004} was high for both the Try ($\kappa = .98, Z(128) = 12.60, p < .001$) and Trial Outcome ($\kappa = .96, Z(128) = 19.96, p < .001$), suggesting almost perfect agreement.

\subsection{Procedure}
Permission to run the study was sought from the head coach of each of the 4 teams (Beijing men's, Beijing women's, Shandong men's, Shandong women's).  These coaches nominated athletes who were fit and able to complete the session without compromising their existing training schedules.  Athletes were randomly assigned to one of two conditions, and then to a particular experimental group, with some subsequent adjustments to assignments to ensure that each condition was matched as much as possible according to average training age.  Once athletes were assigned to an experiment group, they were then added to a WeChat group populated by other training group members and the researcher.

\myparagraph{Cover Story}
Athletes were notified (first via WeChat and then in a team meeting) that they were due to participate in a trial of a various rugby training drills selected from a recent World Rugby report on training methods for rugby sevens.  Athletes were told that the training drills had been previously rated by a selection of international level coaches and players from all over the world (including Asia and China).  Athletes were informed that the purpose of the exercise was to assess the ratings provided by World Rugby (the international governing body of rugby union) by replicating these drills with more rugby athletes in China.  Athletes were told that survey measures and video footage would be collected, which would be used to analyse individual performance of athletes in each training drill.

%%It was also explained to athletes that there would be a second round of drills also requiring groups of 8 athletes, but the makeup of training groups for these groups may reshuffle depending on athletes preferences following the first round of drills.  This detail allowed for the inclusion of a Post-Drill bonding measure, in which athletes were asked to what extent they wished to continue to train with the same 8-athlete training group in a subsequent round of drills.

\myparagraph{Baseline survey}
Approximately 24 hours before the experiment session, athletes were instructed to complete the baseline survey by opening a link provided in the WeChat group.  This survey included formal consent for the study.  Approximately 1 hour before the experiment was due to take place, athletes received a notification in WeChat including information about the projected difficulty of their impending drill session (the experimental prime, for either the ``high difficulty'' or ``low difficulty'' conditions (see Section ~\ref{sect:expPrimes} for detailed descriptions).   All athletes were lead to believe that athletes in the other experiment groups were performing different drills to their assigned drill.  In reality, the training drill for each condition was identical, the only thing that varied was the prime administered before the drill.

\myparagraph{Pre-Drill survey}
Once the 8 athletes participating in the experiment session were all assembled at the designated training field, the researcher verbally re-administered the same prime that had been sent earlier to the athletes via WeChat, and athletes were told in more detail about the requirements of the Invasion Drill.  For example, athletes learned that they would be assessed based on their performance in attack, defence, and their ability to coordinate attack and defence with others (to be assessed through subsequent video analysis).  Athletes were provided with no other explicit information regarding performance goals, besides completing the drill to the best of their ability within the rules of rugby.

%Specifically, athletes were told that their performance in the experiment would be assessed based on subsequent video analysis.

Athletes were then given the Pre-Drill survey, which took approximately 4 minutes to complete.  After all the athletes finished the survey, they participated in a standard warm up routine lasting approximately 10 minutes, including slow jogging and dynamic stretching.  Athletes were instructed not to use the rugby ball during this period, which served to reduce the amount of incidental coordination and interaction between participants prior to the start of the experiment trials.  Following the warm up, the group assembled in the training drill area.  One athlete was randomly assigned to stand at each of 8 available plastic markers.  In this position, athletes were told once more about the structure and procedure for the training drill, in particular the way in which athletes were expected to rotate clockwise after every trial of the drill so that athletes did not habituate to occupying certain positions or coordinating within certain sub-units in the drill.

\myparagraph{The Invasion Drill}
To begin the drill, the ball-carrier at the front and centre of the attack sub-group was instructed to tap the ball with his/her foot, before initiating the attacking sub-phase by advancing forward towards the defence sub-units.  In the case that the ball was immediately fumbled during the initiation of the trial, the training group was instructed to restart that trial and the trial in which the mistake was made was not counted.  Following a block of 4 practice trials, athletes were told by the researcher that the formal test was beginning.  The group of athletes then completed 16 trials of the drill, which allowed each athlete to complete 4 trials of attack and 4 trials of defence, in different positions.

\myparagraph{Post-Drill survey}
Following completion of all 16 test trials, athletes assembled and were thanked for their participation.  Athletes were then instructed to go directly to the side of the field to complete the final Post-Drill survey using their mobile devices.

%Following the completion of this final survey, athletes were told that they would be informed within two days about the next drill (in fact, no more drills were taking place).

%\myparagraph{Video Analysis Procedure}
%The digital camera and tripod were positioned on a platform 2m above the level of the playing field, approximately 10-15m from the bottom try-line corner of the Invasion Drill perimeter (see figure ~\ref{fig:invasionDrill}).  The experimenter began video recording before the athletes arrived, and ceased recording after all athletes had left the training field following the completion of the drill.  Digital video images of action were acquired by a computer, using a USB2.0 cable, and files were saved on an encrypted external hard disk in .AVI format.

%For image treatment, TACTO 8.0 software was used for digitizing at 25 frames per second.
%The procedure is how the study was carried out. It often works well to describe the procedure in terms of what the participants did rather than what the researchers did. For example, the participants gave their informed consent, read a set of instructions, completed a block of 4 practice trials, completed a block of 20 test trials, completed two questionnaires, and were debriefed and excused.

\clearpage
\section{Data analysis}


\subsection{Road map for analysis of study predictions}

The experimental manipulation was checked by analysing condition-wise effects on athlete confidence in ability of group and self to meet the technical challenges of the training drill.  Subsequently, study predictions were tested by analysing two data sets: 1) Post-Drill survey data, and 2) Pre- and Post-Drill survey data.  Post-Tournament data were primarily analysed for a relationship between experiment condition, perceptions of team click, and social bonding.  Pre- to Post-Drill data were analysed for a relationship between experiment condition and a \textit{change} in team click and social bonding as a result of the training drill.




\subsection{Data structure}
The \textit{in situ} and field experimental nature of this study meant that the collected data contained multiple levels of dependency: observations were nested within individuals (collected at Baseline, Pre-, and Post-Drill), location (Beijing or Shandong), team (Shandong men, Shandong women, Beijing men or Beijing women), sex, and experiment session.   To test relationships between variables of interest, linear mixed effects regression (LMER) models (package \textit{lme4} in the R environment) were used for their ability to model the random structure of the data (see Chapter~\ref{tournamentSurvey} Section ~\ref{survey:survey:dataStructureModelSelection} for full explanation).  Intra-class correlation (ICC) estimates were calculated to account for clustering of model residuals according to groupings of athlete sex, team, and specific experiment session.  ICC values of $>.15$ were considered meaningful \citep{Field2012}, and informed which random structure to specify in LMER models.  Unless otherwise stated, all models controlled for the average performance outcome of each trial (Trial Outcome Average), subjective and objective measures of athlete technical competence, personality (extraversion), and arousal measured Post-Drill.

\myparagraph{Data manipulation for Pre- to Post-Tournament analysis}
In order to account for variation over time between Pre- and Post-Drill surveys, variables of interest (individual and team performance relative to prior expectations, team click, social bonding, and fatigue) were calculated by subtracting pre- from post-Drill scores for each athlete. The calculation of change scores reduced the complexity of the data structure down to two levels of analysis, and meant that relationships between these change variables could be modelled using a linear mixed effects regression. Change scores of relevant factors were introduced to the model as fixed effects, so their intercepts were allowed to vary according to the random structure of each model.

 The item measuring athlete perceptions of overall team performance in the Pre-Drill survey was not framed in terms of expectation violation (but instead in terms of more or less ``good'' or ``poor'').  As such, it was not possible to directly compare the Pre- and Post-Drill measures of Team Performance Vs Expectations.  Instead, the Post-Drill measure of Team Performance Vs Expectation was used to predict Team Click Change between Pre- and Post-Tournament measurements.


%Results from models of each data set are reported according to study predictions in Section ~\ref{sect:resultsStudyPredictions}.


\subsection{Data reduction}
Exploratory factor analysis (EFA) was used to reduce multicolinearity between variables while retaining as much variance as possible in the observed data \citep[see Appendix~\ref{app8:EFA}]{Yong2013}. EFA followed the procedure outlined in the previous empirical chapter (see Chapter~\ref{Ch5:dataReduction}).
EFAs were performed on clusters of survey items pertaining to performance, team click, and social bonding, directed at both the training group and the athlete's provincial team as a whole.  Moderator variables---technical competence (objective and subjective), components of individual and training team performance, arousal, fatigue, and athlete perception of team discipline---were also reduced to factors. Due to the equivalence of measures between Pre- and Post-Drill surveys, EFAs were conducted on the entire available data (rather than separate EFAs for the Post-Drill and Pre- to Post-Drill subsets).

%The only exception to this procedure was


%(see Appendix~\ref{app9:dataReduction} for full description)
%In addition, items relating to team click and social bonding directed at the provincial team as a whole (and not just the specific training group) were reduced to factors, in order to assess pre- to post-experiment variation in generalised bonding to the team.

\myparagraph{Perceptions of group and individual performance relative to prior expectations}
The main predictor variable of interest, perceptions of training team performance relative to prior expectations (Team Performance Vs Expectations), was a single item measure, and did not require data reduction.  Given that most outcome variables of interest were factors standardised as z-scores (with $mean \approx 0, SD \approx 1$), perceptions of team performance relative to prior expectations was also transformed to a standardised z-score so as to accord with explanatory variables in subsequent subsequent linear mixed effects modelling \citep[for an explanation, see][1058]{Beckmann2003}.


\myparagraph{Mediation analyses}
Mediation analyses were conducted using linear mixed effects regressions in the Causal Mediation Analysis package in R (Version 4.4.5).  To make inferences concerning the average indirect and total effects, quasi-Bayesian Markov Chain Monte Carlo (MCMC) method based on normal approximation and 1000 simulations was used to estimate the 95\% Confidence Intervals \citep{Tofighi2016a,Imai2010}. MCMC estimation is a form of non-parametric bootstrapping whereby the sampling distribution for the effect of interest is not assumed to be normal but is instead simulated from the model estimates and their asymptotic variances and covariances \citep{Preacher2008}.

%Moderated mediation occurs when either path a (team performance to team click) or path b (from team click to bonding), or both are moderated \citep{Edwards2007,Hayes2017}


\subsection{Manipulation Checks: Confidence in team and self \label{sec:manChecksDrill}}
Following the researcher's in-person delivery of the experimental prime and immediately before the training drill, two survey items were administered to test the effectiveness of the experimental manipulation.  Athletes were asked to report:

\begin{enumerate}
  \item Team Confidence: \textit{How confident are you that you will meet the technical challenges of the training drill?}
  \item Self Confidence: \textit{How confident are you that your team will meet the technical challenges of the training drill?}
%  \item Physiological arousal (see Appendix ~\ref{app5:tournamentSurvey} Section ~\ref{app5:exertionMid} for full details)
\end{enumerate}

Both items were measured using a 100 point scale (0 = ``Not at all confident'', 100 = ``Extremely confident'').  Athlete responses to these items were compared according to experiment condition using LMER to model the random structure of the data, and controlling for technical competence.


































\clearpage
\section{Results}


\subsection{Descriptive Statistics \label{sec:descriptives}}

\subsubsection{Participants}
58 athletes ($men = 31, M(age) = 21.33 SD = 3.33, range = 16-29$) participated in the experiment in groups of 8 (or on two occasions, groups of 6).  A total of 8 experimental sessions were conducted, with participants each taking part in a session only once.  31 participants were from Shandong province ($men = 15, M(age) = 22.1$) and the remaining 27 athletes were from Beijing province ($men = 16, M(age) = 20.4$).  The experimental groups consisted of athletes from the same province. On 3 separate occasions, a dummy participant stood in for an athlete who failed to attend due to injury or illness.  Dummy participants were competent in rugby athletes who were naive to the predictions of the study, and did not participate in the pre- or post-drill surveys. When conducting experimental trials with the Beijing women's team, only 11 (of a total of 16 recruited) athletes were available to participate in the two sessions.  As a compromise, both drills (high and low difficulty) were modified to become a ``$3+2+1$'' version of the Invasion Drill.\footnote{Three attackers advanced towards a first line of two defenders and a second line of defence containing only one defender.} In the experiment session in which only 5 fit athletes participated, a dummy participant made up the 6th athlete.  Survey data of the remaining 58 participants (and video data of all 64 athletes, including dummy participants) were analysed.

\input{images/athleteDescriptivesTrainingOverallOffline}

Basic information concerning participants is displayed in Table ~\ref{tab:athleteDescriptivesTrainingOverall}.  Athletes' average training age (years dedicated to full-time rugby training) was 4.22 years ($SD = 2.11$), and athletes had spent an average of 5.64 years ($SD = 2.08$) in the team.  50\% of the sample (29) were either full-time employees of their provincial team (21), or were otherwise employed on a full-time (but fixed term) contract (8).  The remaining 29 athletes were employed either on a ``student contract'' (8 or 14.5\%), or on a short term training contract (16 or 29.1\%), or on a short-term trial basis (2 or 3.6\%).  18 athletes (31\%) declared that there were in the starting team of their respective teams.

%\footnote{Being a member of the starting team indicated that athletes were among the most (if not the most) competent athletes in their position in their team.}
All athlete attributes mentioned above appeared to be evenly matched between conditions, except for key variables of athlete age and training age, which appeared to be higher in the low difficulty condition than in the high difficulty condition. Independent-sample t-tests were conducted to compare key variables of athlete training age and athlete age in the high and low difficulty conditions. Athlete age differed significantly between the high ($M= 20.48, SD =3.11$) and low ($M= 22.14, SD =3.39$) difficulty conditions ($t(52.09) = -3.33, df = 162.59, p-value = .001$).  The difference in training age between high ($M= 3.74, SD =1.90$) and low ($M= 20.48, SD =2.25$) difficulty conditions, was not significant ($t(52.09)= -1.67, df = 162.59, p-value = 0.10$).

%(see Table ~\ref{tab:athleteDescriptivesTrainingOverall}).



\subsubsection{Key variables of interest \label{sect:surveyResponses}}
Tables~\ref{tab:rawPerformance} and~\ref{tab:rawClickBond} display main variables of interest by condition (high/low) and time point (Baseline, Pre-Drill, Post-Drill).  In general, the central tendency of all survey items was between .5 to 1.5 standard deviations above the mid-point of the scale (50 for 100 points scales such as Tacit Understanding or Emotional Support, or 2.5 for the Fusion and Group Identity scales.   For instance, average athlete perceptions of individual performance ranged from a low of 58.31 ($SD = 17.71$) for perceptions of individual performance relative to prior expectations (measured Post-Drill) to a high of 78.62 ($SD = 12.75$) for athlete confidence in group ability to meet the technical challenges of the drill, measured Pre-Drill.  Central tendencies of variables measuring athletes' perceptions of team click and social bonding ranged from a low of 68.34 ($SD = 14.69$) for perceptions of tacit understanding measured Post-Drill in the low difficulty condition, to a high of 90.21 ($SD = 9.33$) for perceptions of shared goal with the training group measured Pre-Drill in the low difficulty condition.

Athletes were on average more critical in regards to perceptions of individual performance than they were of group and team performance, and were generally more critical of individual and team performance in the Post-Drill survey than they were in the Baseline and Pre-Drill surveys.  Measures relating to team click and social bonding appeared to decrease in the Post-Drill survey relative to the Pre-Drill and Baseline surveys. Confidence in self and the training group to meet the challenges of the training drill did not show obvious signs of variation according to condition.

Objective performance outcome did not noticeably vary across experiment sessions, with the low difficulty experiment sessions averaging 11.83 ($SD = .85$) tries and the high condition only 8 ($SD = 2.20$) tries (the most successful tries scored in an experiment session was 13, while the lowest amount of tries scored in an experiment session was 4).  For the more fine grained performance score, ``Trial Outcome Average'' the average for the low difficulty condition was 4.68 ($SD = .25$), while the average for the high difficulty condition was 3.52 ($SD = .49$). A more detailed report of performance outcome by experiment session is shown in Table ~\ref{tab:trainingObjPerformanceSession}.
%For a description of additional variables, including moderator variables, see Appendix~\ref{app9:descriptives}.

\input{images/trainingDescriptivesRawPerformanceGROUP}
\input{images/trainingDescriptivesRawPerformanceIND}
% Please add the following required packages to your document preamble:
% \usepackage{booktabs}
\begin{table}[]
  \centering
  \begin{tabular}{@{}rcccccc@{}}
  \toprule
  \multicolumn{1}{c}{Item} & \multicolumn{2}{c}{Baseline} & \multicolumn{2}{c}{Pre} & \multicolumn{2}{c}{Post} \\ \midrule
  \multicolumn{1}{l}{} & High & Low & High & Low & High & Low \\
  \multicolumn{1}{c}{\textbf{Training Group}} &  &  &  &  &  &  \\
  Team Click: &  &  &  &  &  &  \\
  Tacit understanding & - & - & \begin{tabular}[c]{@{}c@{}}72.90 \\ (15.77)\end{tabular} & \begin{tabular}[c]{@{}c@{}}72.45 \\ (19.40)\end{tabular} & \begin{tabular}[c]{@{}c@{}}64.64 \\ (20.99)\end{tabular} & \begin{tabular}[c]{@{}c@{}}68.34 \\ (14.69)\end{tabular} \\
  Team aura & - & - & \begin{tabular}[c]{@{}c@{}}78.03 \\ (17.48)\end{tabular} & \begin{tabular}[c]{@{}c@{}}81.79 \\ (11.60)\end{tabular} & \begin{tabular}[c]{@{}c@{}}74.50 \\ (20.54)\end{tabular} & \begin{tabular}[c]{@{}c@{}}76.79 \\ (13.73)\end{tabular} \\
  Click pictorial & - & - & \begin{tabular}[c]{@{}c@{}}4.00 \\ (.89)\end{tabular} & \begin{tabular}[c]{@{}c@{}}3.76 \\ (1.18)\end{tabular} & \begin{tabular}[c]{@{}c@{}}3.75 \\ (.93)\end{tabular} & \begin{tabular}[c]{@{}c@{}}3.45 \\ (1.24)\end{tabular} \\
  Reliability of others & - & - & - & - & \begin{tabular}[c]{@{}c@{}}72.29 \\ (17.79)\end{tabular} & \begin{tabular}[c]{@{}c@{}}68.55 \\ (23.47)\end{tabular} \\
  Reliability for others & - & - & - & - & \begin{tabular}[c]{@{}c@{}}63.46 \\ (14.56)\end{tabular} & \begin{tabular}[c]{@{}c@{}}61.03 \\ (21.33)\end{tabular} \\
  Ability extended & - & - & \begin{tabular}[c]{@{}c@{}}72.28 \\ (16.22)\end{tabular} & \begin{tabular}[c]{@{}c@{}}77.66 \\ (14.33)\end{tabular} & \begin{tabular}[c]{@{}c@{}}61.86 \\ (28.68)\end{tabular} & \begin{tabular}[c]{@{}c@{}}65.86 \\ (18.09)\end{tabular} \\
  \multicolumn{1}{l}{} &  &  &  &  &  &  \\
  Social Bonding: &  &  &  &  &  &  \\
  Emotional support & - & - & \begin{tabular}[c]{@{}c@{}}78.55 \\ (18.90)\end{tabular} & \begin{tabular}[c]{@{}c@{}}78.69 \\ (24.33)\end{tabular} & \begin{tabular}[c]{@{}c@{}}73.86 \\ (19.61)\end{tabular} & \begin{tabular}[c]{@{}c@{}}76.52 \\ (14.92)\end{tabular} \\
  Common goal & - & - & \begin{tabular}[c]{@{}c@{}}80.79 \\ (24.98)\end{tabular} & \begin{tabular}[c]{@{}c@{}}90.21 \\ (9.33)\end{tabular} & \begin{tabular}[c]{@{}c@{}}79.04 \\ (20.80)\end{tabular} & \begin{tabular}[c]{@{}c@{}}83.62 \\ (15.34)\end{tabular} \\
  Fusion (Pictorial) & - & - & \begin{tabular}[c]{@{}c@{}}4.14 \\ (0.64)\end{tabular} & \begin{tabular}[c]{@{}c@{}}4.31 \\ (.71)\end{tabular} & \begin{tabular}[c]{@{}c@{}}4.00 \\ (1.09)\end{tabular} & \begin{tabular}[c]{@{}c@{}}3.79 \\ (1.52)\end{tabular} \\
  \multicolumn{1}{l}{} &  &  &  &  &  &  \\
  \multicolumn{1}{c}{\textbf{Provincial Team}} &  &  &  &  &  &  \\
  Social Bonding: &  &  &  &  &  &  \\
  Emotional support & \begin{tabular}[c]{@{}c@{}}71.89 \\ (24.66)\end{tabular} & \begin{tabular}[c]{@{}c@{}}77.11 \\ (17.76)\end{tabular} & - & - & \begin{tabular}[c]{@{}c@{}}71.71 \\ (17.46)\end{tabular} & \begin{tabular}[c]{@{}c@{}}76.90 \\ (11.97)\end{tabular} \\
  Common goal & \begin{tabular}[c]{@{}c@{}}81.59 \\ (17.66)\end{tabular} & \begin{tabular}[c]{@{}c@{}}84.96 \\ (16.93)\end{tabular} & - & - & \begin{tabular}[c]{@{}c@{}}82.64\\ (17.18)\end{tabular} & \begin{tabular}[c]{@{}c@{}}84.28 \\ (14.13)\end{tabular} \\
  Fusion (Pictorial) & \begin{tabular}[c]{@{}c@{}}4.33 \\ (.73)\end{tabular} & \begin{tabular}[c]{@{}c@{}}4.67 \\ (.48)\end{tabular} & - & - & \begin{tabular}[c]{@{}c@{}}4.25 \\ (.75)\end{tabular} & \begin{tabular}[c]{@{}c@{}}4.45 \\ (.63)\end{tabular} \\
  Fusion (Verbal) & \begin{tabular}[c]{@{}c@{}}4.00 \\ (.81)\end{tabular} & \begin{tabular}[c]{@{}c@{}}4.15 \\ (.65)\end{tabular} & - & - & \begin{tabular}[c]{@{}c@{}}3.88 \\ (.76)\end{tabular} & \begin{tabular}[c]{@{}c@{}}4.06 \\ (.61)\end{tabular} \\
  Group Identification & \begin{tabular}[c]{@{}c@{}}4.20 \\ (.78)\end{tabular} & \begin{tabular}[c]{@{}c@{}}4.49 \\ (.57)\end{tabular} & - & - & \begin{tabular}[c]{@{}c@{}}4.16 \\ (.71)\end{tabular} & \begin{tabular}[c]{@{}c@{}}4.40 \\ (.64)\end{tabular} \\ \bottomrule
  \end{tabular}
\caption{Mean (SD) of key Team Click and Social Bonding variables of interest measured by condition at Baseline, Pre-Drill, and Post-Drill (all $n's = 58$).}
\label{tab:rawClickBond}
\end{table}

% Please add the following required packages to your document preamble:
% \usepackage{booktabs}
\begin{table}[]
\centering


\begin{tabular}{@{}rcc@{}}
\toprule
\multicolumn{1}{c}{\textbf{Session}} & \textbf{Tries Total} & \textbf{\begin{tabular}[c]{@{}c@{}}Trial Outcome \\ (Mean (SD))\end{tabular}} \\ \midrule
\multicolumn{1}{c}{\textbf{High difficulty}} &  &  \\
Beijing men & 13 & 4.39 (1.82) \\
\multicolumn{1}{l}{} & \multicolumn{1}{l}{} & \multicolumn{1}{l}{} \\
Beijing women & 11 & 4.5 (1.9) \\
\multicolumn{1}{l}{} & \multicolumn{1}{l}{} & \multicolumn{1}{l}{} \\
Shandong men & 9 & 3.81 (2.04) \\
\multicolumn{1}{l}{} & \multicolumn{1}{l}{} & \multicolumn{1}{l}{} \\
Shandong women & 10 & 4.01 (2.01) \\
\multicolumn{1}{c}{} &  &  \\
\multicolumn{1}{c}{\textbf{Low difficulty}} &  &  \\
Beijing men & 8 & 3.31 (1.85) \\
\multicolumn{1}{l}{} & \multicolumn{1}{l}{} & \multicolumn{1}{l}{} \\
Beijing women & 4 & 2.75 (1.81) \\
\multicolumn{1}{l}{} & \multicolumn{1}{l}{} & \multicolumn{1}{l}{} \\
Shandong men & 12 & 5 (1.71) \\
\multicolumn{1}{l}{} & \multicolumn{1}{l}{} & \multicolumn{1}{l}{} \\
Shandong women & 11 & 4.75 (1.34) \\
\multicolumn{1}{l}{} & \multicolumn{1}{l}{} & \multicolumn{1}{l}{} \\ \bottomrule
\end{tabular}



\caption{Objective performance outcomes according to experiment session.  Objective performance was measured using a count of total tries scored over the 16 trials,  and an average of the quality of performance in each trial (Tries Total and Trial Outcome).}
\label{tab:trainingObjPerformanceSession}
\end{table}


 %(see Tables ~\ref{tab:indPerfTimeLowTraining}\nobreakdash~\ref{tab:teamPerfTimeBaselineTraining} in Appendix ~\ref{app9:trainingExperiment} Section ~\ref{app9:descriptives}).

 %Objective measures of performance in the training Drill (derived from video footage) appeared to be consistent across experiment conditions (see Table ~\ref{tab:objectiveOutcomeCondition} in Appendix ~\ref{app9:trainingExperiment} Section ~\ref{app9:descriptives}).



 %(see Tables ~\ref{tab:groupClickTimeHighTraining}\nobreakdash--\ref{tab:teamBondingTimeHighTraining} in ~\ref{app9:trainingExperiment} Section ~\ref{app9:descriptives}).



\subsection{Data Reduction}

\subsubsection{Components of team performance}
Measures of components of team performance in the Invasion Drill (defence, attack, support play, and on field communication) were subjected to EFA.  Correlations between group component performance items was very high (all $r's > 0.67$), which suggested that one factor was appropriate (see Table ~\ref{tab:jointActionSuccessCorrTable}).  The KMO index and Bartlett's test both suggested high sampling adequacy, ($KMO =  0.83, \chi^2(6, N = 116) = 379.28, p = <.001$).  One factor, labelled ``Team Performance Components'' was imposed on the data, which explained 77\% of the overall variance ($SS Loading = 3.08$).  $Guttman's \lambda = 0.92$ and $Cronbach's \alpha = 0.93$ indicated that the data reduction was appropriate and reliable.

\subsubsection{Components pf individual performance\label{app9:dataReductionPerformance}}
Components of individual components of performance in the Invasion Drill (1-on-1 defence, passing technique, support play in attack, decision making in attack, and effectiveness in contact) were subjected to EFA.  The variable ``Effectiveness in contact'' was removed from analysis as the Invasion Drill was predominantly a non-contact training drill, and so this item was not relevant to athletes' performance.  Correlations between individual component performance items was very high (all $r's > .61$), which suggested that one factor was appropriate (see Table ~\ref{tab:indComponentPerfCorrTable}).  The KMO index and Bartlett's test both suggested high sampling adequacy, ($KMO = .83, \chi^2(6, N = 116) = 280.70, p < .01$). One factor, labelled ``Individual Performance Components'' was imposed on the data, which explained 69.1\% of the overall variance ($SS Loading  2.77$).  $Guttman's \lambda = .87$  and  $Cronbach's \alpha = .90$ indicated that the data reduction was appropriate and reliable.

\subsubsection{Perceptions of team click within the training group (Team Click)}
An EFA was performed on items relating to training group team click (tacit understanding, general atmosphere, click pictorial, and ability extended by the group).  Correlations between the remaining items were moderate to strong (all $r's > .29$), suggesting that one factor was appropriate (see Table ~\ref{tab:groupClickCorrTable} in Appendix ~\ref{app9:trainingTournament} Section ~\ref{app9:dataReduction}). The KMO index and Bartlett's test both suggested high sampling adequacy, ($KMO =  .72, \chi^2(6, N = 116) = 127.86, p < .001$).
One factor, labelled ``Team Click'' was imposed on the data, which explained 47.7\% of the overall variance ($SS Loading = 2$). $Guttmans \lambda = .73$ and $Cronbachs \alpha = .77$ indicated that the data reduction was appropriate and reliable.

%In addition to the 4 survey items common to both Pre- and Post-Drill surveys, athletes were also asked in the Post-Drill survey about their own reliability and the reliability of other athletes to perform their role on the field specifically. These two variables, ``Reliability for others'' and ``Reliability of others'' were included in an EFA involving only the Post-Drill measures (for details, see Appendix ~\ref{} Section ~\ref{}).  The variable ``reliability for others'' was removed from the matrix because it did not significantly correlate with other variables (all $r's <= .16$).

\subsubsection{Social bonding to the training group (Social Bonding)}
Items concerning bonding to the training group (emotional support, shared goal, fusion pictorial) were subjected to EFA.  Correlations between items were high (all $r's > .37$), which suggested that one factor would be appropriate (see Table ~\ref{tab:groupBondingCorrTable} in Appendix ~\ref{app9:trainingTournament} Section ~\ref{app9:dataReduction}). The KMO index and Bartlett's test both suggested high sampling adequacy, ($KMO =  .65, \chi^2(3, N = 116) = 54.23, p < .001$).
One factor, labelled ``Social Bonding'' was imposed on the data, which explained 42\% of the overall variance ($SS Loading = 1.26$). $Guttman's \lambda = .59$ and $Cronbach's \alpha = .68$ indicated that the data reduction was appropriate and reliable.

\subsubsection{Social bonding to the provincial team (Social Bonding (Province)) \label{app9:teamBondingEFA}}
Measures of bonding to the team (emotional support, shared goal, fusion pictorial, fusion verbal, and group identification) were subjected to EFA.  Correlations between items were moderate-to-high (all $r's > .18$), which suggested that one factor would be appropriate (see Table ~\ref{tab:teamBondingCorrTable}).
The KMO index and Bartlett's test both suggested high sampling adequacy, ($KMO = .69, \chi^2(10, N = 116) = 172.59, p < .001$).
One factor, labelled ``Social Bonding (Province)'' was imposed on the data, which explained 39.2\% of the overall variance ($SS Loading = 1.96$, $Guttmans \lambda = .77$ and $Cronbach's \alpha = .76$ indicated that the data reduction was appropriate and reliable).

\myparagraph{Technical Competence (objective and subjective measures) \label{app9:competenceEFA}}
All 8 items relevant to technical competence were analysed in a correlation matrix to assess relatedness (see Table ~\ref{tab:technicalCompetenceCorrTable}). All measures of objective competence were highly correlated all $r's >= .64$),  and among measures of subjective competence (all items except for the team competence measure correlated at $r > .52$) suggested that the data could be explained by two underlying factors. Team Ability Chinese Provinces was dropped from analysis due to low correlation with other competence variables, possibly because the item did not ask about an individual athlete's competence (it referred instead to an athlete's opinion of the competence of the team of which they were a member).

An EFA of technical competence variables revealed that items of interest loaded on two factors. Measures of objective competence (Years Team, Training Age, Team Status, Athlete Status) loaded on the first factor, which was labelled ``Objective Competence'' because the measures were all objective markers of an athlete's competence.
Objective Competence explained 41\% of the total variance ($SS Loading = 2.87$).  The remaining measures of subjective competence (Ability Teammates, Ability Chinese Pros, Ability International Pros) loaded on the remaining factor.  The second factor was labelled ``Subjective Competence'', due to the fact that all measures were the product of athlete self-report.  Subjective competence explained 23\% of the variance ($SS Loading = 1.62$).  $Guttman's \lambda = .86$ and $Cronbachs \alpha = .79$ indicated that the data reduction was appropriate and reliable.

\myparagraph{Arousal\label{app9:arousalEFA}}
Items associated with physiological arousal measured pre- and Post-Drill were subjected to EFA.
Correlations between items were all reasonably high (all $r's  .41$), which suggested that one factor would be appropriate (see Table ~\ref{tab:arousalCorrTable}). The KMO index and Bartlett's test both suggested high sampling adequacy, ($KMO = .64, \chi^2(3, N = 116) = 116.12, p < .01$).  One factor, labelled ``Arousal'' was imposed on the data, which explained 58.7\% of the overall variance ($SS Loading = 1.76$).  $Guttman's \lambda = .74 and  Cronbach's \alpha = .78$  indicated that the data reduction was appropriate and reliable.


%\subsubsection{Additional data reduction}
%Measures relating to athlete perceptions of components of group and individual performance, social bonding to the team as a whole, athlete arousal, technical competence (subjective and objective measures), and athlete perceptions of team discipline were subjected to data reduction (see Appendix ~\ref{app9:trainingExperiment} Section ~\ref{app9:dataReduction} for full details).




\subsubsection{Correlation of factors}
\myparagraph{Post-Drill}
Table~\ref{tab:expcorr_postdrill} displays correlation values of key factors of interest for the Pre- to Post-Drill data.  All values were low- to moderate; except for Team Performance Components and Team Performance Vs Expectations ($.64$), and Team Click Change and Social Bonding Change ($.79$).
\begin{table}[htbp]
\resizebox{\textwidth}{!}{%
\begin{tabular}{lccccccc}
\toprule
                       & Group Confidence & Self Confidence & Team Performance & Indiv. Performance & Team Click & Social Bonding & Social Bonding \\
                       &                  &                 & vs Expectations  & vs Expectations        &            & (Group)        & (Province)     \\
                    \midrule
                       &                  &                 &                  &                        &            &                &                \\
Group Confidence       & 1                & 0.65            & -0.04            & 0                      & 0.25       & 0.32           & 0.27           \\
                       &                  &                 &                  &                        &            &                &                \\
                       &                  &                 &                  &                        &            &                &                \\
Self Confidence        &                  & 1               & -0.08            & -0.04                  & 0.19       & 0.29           & 0.27           \\
                       &                  &                 &                  &                        &            &                &                \\
                       &                  &                 &                  &                        &            &                &                \\
Team Performance       &                  &                 & 1                & 0.26                   & 0.58       & 0.33           & 0.28           \\
vs Expectations        &                  &                 &                  &                        &            &                &                \\
                       &                  &                 &                  &                        &            &                &                \\
Indiv. Performance &                  &                 &                  & 1                      & 0.26       & 0.09           & -0.05          \\
vs Expectations        &                  &                 &                  &                        &            &                &                \\
                       &                  &                 &                  &                        &            &                &                \\
Team Click             &                  &                 &                  &                        & 1          & 0.7            & 0.7            \\
                       &                  &                 &                  &                        &            &                &                \\
                       &                  &                 &                  &                        &            &                &                \\
Social Bonding         &                  &                 &                  &                        &            & 1              & 0.79           \\
(Group)                &                  &                 &                  &                        &            &                &                \\
                       &                  &                 &                  &                        &            &                &                \\
Social Bonding         &                  &                 &                  &                        &            &                & 1              \\
(Province)             &                  &                 &                  &                        &            &                &               
          \\
\bottomrule     
\end{tabular}}
\caption{Training Experiment Correlation Matrix (Post Drill)}
\label{tab:expcorr_postdrill}
\end{table}


\myparagraph{Pre- to Post-Drill}
Table ~\ref{tab:expcorr_prepost} displays correlation values of key factors of interest for the Pre- to Post-Drill data.  All values were low- to moderate, except for Team Performance Components and Team Performance Vs Expectations ($.64$).  The correlation between key outcome variables, Team Click Change and Social Bonding Change, was moderate ($.37$).
\begin{table}[htbp]
\resizebox{\textwidth}{!}{%
\begin{tabular}{lllccccc}
\toprule
                   & \multicolumn{1}{c}{Team Performance} & \multicolumn{1}{c}{Team Performance} & Indiv. Performance   & Indiv. Performance   & Team Click           & Social Bonding       & Social Bonding       \\
                   & \multicolumn{1}{c}{Components}       & \multicolumn{1}{c}{vs Expectations}  & Components           & vs Expectations      & Change               & Change (Group)       & Change (Province)    \\
          \midrule
                   & \multicolumn{1}{c}{}                 & \multicolumn{1}{c}{}                 &                      &                      &                      &                      &                      \\
Team Performance   & \multicolumn{1}{c}{1}                & \multicolumn{1}{c}{0.64}             & -0.09                & 0.07                 & -0.26                & 0.32                 & -0.11                \\
Components         & \multicolumn{1}{c}{}                 & \multicolumn{1}{c}{}                 &                      &                      &                      &                      &                      \\
                   & \multicolumn{1}{c}{}                 & \multicolumn{1}{c}{}                 &                      &                      &                      &                      &                      \\
Team Performance   & \multicolumn{1}{c}{}                 & \multicolumn{1}{c}{1}                & -0.07                & 0                    & -0.33                & 0.17                 & -0.09                \\
vs Expectations    & \multicolumn{1}{c}{}                 & \multicolumn{1}{c}{}                 &                      &                      &                      &                      &                      \\
                   & \multicolumn{1}{c}{}                 & \multicolumn{1}{c}{}                 &                      &                      &                      &                      &                      \\
Indiv. Performance & \multicolumn{1}{c}{}                 & \multicolumn{1}{c}{}                 & 1                    & 0.25                 & 0.53                 & 0.04                 & 0.1                  \\
Components         & \multicolumn{1}{c}{}                 & \multicolumn{1}{c}{}                 &                      &                      &                      &                      &                      \\
                   &                                      &                                      & \multicolumn{1}{l}{} & \multicolumn{1}{l}{} & \multicolumn{1}{l}{} & \multicolumn{1}{l}{} & \multicolumn{1}{l}{} \\
Indiv. Performance &                                      &                                      &                      & 1                    & 0.2                  & 0.01                 & 0                    \\
vs Expectations    &                                      &                                      &                      &                      &                      &                      &                      \\
                   &                                      &                                      &                      &                      &                      &                      &                      \\
Team Click         &                                      &                                      &                      &                      & 1                    & 0.37                 & 0.24                 \\
Change             &                                      &                                      &                      &                      &                      &                      &                      \\
                   &                                      &                                      &                      &                      &                      &                      &                      \\
Social Bonding     &                                      &                                      &                      &                      &                      & 1                    & 0.44                 \\
Change (Group)     &                                      &                                      &                      &                      &                      &                      &                      \\
                   &                                      &                                      &                      &                      &                      &                      &                      \\
Social Bonding     &                                      &                                      &                      &                      &                      &                      & 1                    \\
Change (Province)  &                                      &                                      &                      &                      &                      &                      &               \\
\bottomrule      
\end{tabular}}
\caption{Training Experiment Correlation Matrix (Pre-Post)}
\label{tab:expcorr_prepost}
\end{table}



\subsection{ICC values}

\myparagraph{Post-Drill data}
ICC values for variables of interest (shown in Table ~\ref{tab:trainingICC}) indicated meaningful clustering (i.e. $ICC <.10$) in the data according to experimental session (8 in total), provincial team (Beijing men's, Beijing women's, Shandong men's, Shandong women's).  Given the constraints on model complexity imposed by the relatively small sample size of the study, experiment session was chosen as a level 2 grouping variable (owing to marginally higher ICC values).  To account for clustering within each experiment session, Experiment session was modelled as a random effect using the intercept only (models that included both the intercept and the slope failed to converge due the large number of groups relative to sample size).  High variation in objective performance outcomes further supported the decision to account for variation attributable to experiment session (see Appendix~\ref{app9:objPerformance}).

% Please add the following required packages to your document preamble:
% \usepackage{booktabs}
\begin{table}[]
  \centering

  
\begin{tabular}{@{}rcccc@{}}
\toprule
\multicolumn{1}{c}{\textbf{Variable}} & \textbf{Session} & \textbf{Sex} & \textbf{Team} & \textbf{Location} \\ \midrule
Group Performance Expectations & 0.06 & 0.06 & 0.08 & 0.01 \\
 &  &  &  &  \\
Team Click & 0.14 & -0.03 & 0.09 & 0.08 \\
 &  &  &  &  \\
Social Bonding & 0.21 & -0.01 & 0.18 & 0.15 \\
 &  &  &  &  \\
Social Bonding (Provincial team) & 0.05 & 0.06 & 0.08 & -0.04 \\ \bottomrule
\end{tabular}


\caption{ICC values for experiment session, sex, provincial team, and location.}
\label{tab:trainingICC}
\end{table}



 %revealed team-level clustering in the data for Team Performance Components ($ICC = .37$), Individual Performance Components ($ICC = .22$), Team Click ($ICC = .30$), Social Bonding ($ICC = .10$) and objective measures of technical competence ($ICC = .36$).  ICC values indicated only trivial clustering according to sex all $ICC < .10$, except for the ICC for Individual Performance Components, which was .11).  As such, team (but not sex)) was included in subsequent models as a random effect (for a detailed assessment of the group-level dependencies in the collected data, see Appendix~\ref{app8:ICC})

%\myparagraph{(Pre- to Post-Drill data)}
 %ICC values indicated that variance between experimental sessions
 %variance was low to moderate, with highest values for perceptions of team click and social bonding between teams and experiment sessions ().

%\myparagraph{Pre- to Post-Drill survey}
%For the Pre- to Post-Drill survey data, experiment session was identified as a group-level variable in the multi-level structure, based on an assessment of ICC values between main variables of interest (see Table ~\ref{app9:modelSelectionPrePost} in Appendix ~\ref{app9:trainingExperiment} Section ~\ref{app9:modelSelectionPrePost}).



\subsection{Change in factors over time}
  %One exception was physiological arousal, which significantly increased following the experiment, as would be expected due to the moderate to high levels of physiological exertion associated with the training drill.
Table ~\ref{tab:factorsTime} shows that the Post-Drill measures of group component performance, individual component performance, training group Team Click, and training group social bonding, were all considerably lower than Pre-Drill measurements. The differential between athlete perceptions of confidence in group and individual ability to meet the challenges of the drill and perceptions of group and individual performance relative to expectations was also striking: the mean of questions relating to group ($63.37, SD = 21.23$) compared to individual ($49.23, SD = 26.42$) performance relative to prior expectations were much lower than initial confidence in individual ($77.33, SD = 19.19$) and group ($77.33, SD = 15.04$) ability.  Thus, key variables of interest appeared to reduce in value between Pre- and Post-Drill measures.

\begin{table}[]
\centering

\begin{tabular}{rccc}
\hline
\multicolumn{1}{l}{\textbf{}} & \textbf{Baseline} & \textbf{Pre} & \textbf{Post} \\ \hline
n & 58 & 58 & 58 \\ \hline
\multicolumn{1}{c}{\textbf{Performance}} &  &  &  \\ \hline
Group confidence &  & 77.33 (15.04) &  \\
\multicolumn{1}{l}{} &  &  &  \\
Group vs expected &  &  & 63.37 (21.23) \\
\multicolumn{1}{l}{} &  &  &  \\
Group components &  & .27 (.86) & -.28 (1.01) \\
\multicolumn{1}{l}{} &  &  &  \\
Ind. confidence &  & 77.33 (19.19) &  \\
\multicolumn{1}{l}{} &  &  &  \\
Ind. vs expected &  &  & 49.23 (26.42) \\
\multicolumn{1}{l}{} &  &  &  \\
Ind. components & -.05 (.98) & .10 (.93) & -.06 (.97) \\
\multicolumn{1}{l}{} &  &  &  \\ \hline
\multicolumn{1}{c}{\textbf{Team Click}} &  &  &  \\ \hline
Training Group &  & .19 (.85) & -.19 (.97) \\
\multicolumn{1}{l}{} &  &  &  \\
\begin{tabular}[c]{@{}r@{}}Training Group Post \\ (5 item)\end{tabular} &  &  & -.00 (.93) \\
\multicolumn{1}{l}{} &  &  &  \\ \hline
\multicolumn{1}{c}{\textbf{Social Bonding}} &  &  &  \\ \hline
Training Group &  & .12 (.81) & -.12 (.85) \\
\multicolumn{1}{l}{} &  &  &  \\
Team (Province) & .02 (.93) &  & -.02 (.84) \\
\multicolumn{1}{l}{} & \multicolumn{1}{l}{} & \multicolumn{1}{l}{} & \multicolumn{1}{l}{} \\ \hline
\multicolumn{1}{c}{\textbf{Moderators}} &  &  &  \\ \hline
Arousal &  & -.22(.94) & .23 (.92) \\
\multicolumn{1}{l}{} &  &  &  \\
\multicolumn{1}{l}{} &  &  &  \\ \hline
\end{tabular}

\caption{Mean (SD) for factors at Baseline, Pre-, and Post-Drill.}
\label{tab:factorsTime}
\end{table}

New variables were created which expressed each individual athlete's change in variables of team click and social bonding between two time points.  Values at time Pre-Drill (or Baseline in the case of the Social Bonding (Province) variable) were subtracted from values at time Post-Drill to create a new variable.  Relationships between these new variables (``Team Click Change,'' ``Social Bonding Change,'' ``Social Bonding (Province) Change'') were used to analyse study predictions. Measures of performance relative to prior expectation were not suitable for this transformation, as the pre- training drill measures of performance were not phrased in terms of prior expectation.  Instead, the Post-Drill measure of performance was used to predict change in outcome variables.

%(See Appendix ~\ref{app9:trainingExperiment} Section ~\ref{sect:prePostDrillFactors})
%\myparagraph{Construction of new variables expressing change in variables over time (Pre- to Post-Drill)}

%To find evidence for predicted \textit{positive} associations between variables within a larger trend in which mean values of most measures decreased between Pre- to Post-Drill,

%\subsubsection{Post-Drill survey}
%Results showed no obvious group-wise effects according to condition.  LMER models revealed no significant variation outcome variables according to experiment condition.

%( team performance components, or individual performance components) according to experiment condition (see Appendix ~\ref{app9:trainingExperiment} Section ~\ref{app9:conditionDifferencesIV} for a full description results).

%Similarly, LMER models revealed no significant variation in outcome variables (team click with the training group, group bonding, and bonding to team) by condition (see Appendix ~\ref{app9:trainingExperiment} Section ~\ref{app9:manipulationChecksDV} for a full description of results).


 %The relationship between perceptions of team performance relative to prior expectations and team click was modelled using an LMER with team performance expectations, condition, and their interaction included in the model as fixed effects.


%\begin{figure}
%  \centering
%  \includegraphics[width=0.5\linewidth,keepaspectratio] {images/fullOutcomeAvgSessionBoxplot-1}%
%  \label{fig:fullOutcomeAvgSessionBoxplot}
%  \caption{Average performance outcome by condition}
%\end{figure}

 %\subsubsection{Pre-Post Drill survey}
%The Pre-Post Drill survey data showed similar results.
%There was no significant variation in predictor variables (change in team performance relative to prior expectations, change in individual performance relative to prior expectations, change in perception of components of group or individual performance) according to experiment condition (see Appendix ~\ref{app9:trainingExperiment} Section ~\ref{app9:conditionDifferencesIV} for a full description results).  Similarly, outcome variables (change in team click, change in group bonding, and bonding to team (Province) according to condition) did not vary by condition (see Appendix ~\ref{app9:trainingExperiment} Section ~\ref{app9:manipulationChecksDV} for a full description of results).


%\myparagraph{Components of performance}
%Similarly,






 %revealed no overall condition-wise differences in athlete confidence in group ($\betavec .008 \CIstart -.30, .31 \CIfinish \SE .16, t(54)= .05, \pvalue .96$) or individual ($\betavec .06 \CIstart -.30, .42 \CIfinish \SE .18, t(54)= .33, \pvalue .74$) ability to meet the technical challenges of the drill.  Pre-Drill measures of arousal also did not differ according to condition ($\betavec .07 \CIstart -.29, .44 \CIfinish \SE .19, t(53)= .38, \pvalue .71$).
%For a more detailed review of manipulation checks, see Appendix ~\ref{app9:trainingExperiment} Section ~\ref{app9:manipulationChecks}.

  %$F(28,28) = 2.17, \CIstart 1.02, 4.63 \CIfinish, p = .04$

%Unless otherwise stated, all LMER models controlled for objective performance in experiment session (Trial Outcome Average), subjective and objective measures of technical competence, arousal, and personality (extraversion).
%, and 2) pay closer attention to the details of joint action between participants.



%The high difficulty prime was designed to encourage athletes to 1) generate lower expectations for individual and team performance, and 2) pay closer attention to the details of joint action between participants.  It was predicted that athletes in the high difficulty condition would report more positive perceptions of performance relative to prior expectations.








\section{Analysis of study predictions\label{sect:resultsStudyPredictions}}


\subsection{Prediction 1.a: Higher levels of uncertainty in joint action will generate Lower confidence for success of team performance}

\subsubsection{Manipulation checks}

%\myparagraph{Confidence in group and self to meet technical challenges}
LMER models controlling for Objective and Subjective Competence (fixed) and experiment session (random) were used to assess the effect of condition on athlete confidence in the abilities of the group (Team Confidence) and self (Self Confidence) to meet the technical challenges of the drill.

The effect of condition on Team Confidence was not significant ($\betavec .01 \CIstart -.42, .44 \CIfinish \SE .22, t(54)= .05, \pvalue .96$). The model did reveal significant positive effects of Objective Competence ($\betavec .49 \CIstart .27, .72 \CIfinish \SE .11, t(54)= 4.30, p <.001$) and Subjective Competence ($\betavec .49 \CIstart .25, .73 \CIfinish \SE .12, t(54)= 4.00, p < .001$) on Team Confidence, which suggested that athletes' objective and perceived technical competence influenced confidence in group ability to meet the impending challenges of the training drill.

Similarly, the effect of condition on Self Confidence was also not significant ($\betavec .09 \CIstart -.42, .59 \CIfinish \SE .26, t(54)= .05, \pvalue .74$).
The model did reveal significant positive effects of Objective Competence ($\betavec .38 \CIstart .13, .65 \CIfinish \SE .13, t(54)= 2.90, p <.001$) and Subjective Competence ($\betavec .29 \CIstart .01, .58 \CIfinish \SE .14, t(54)= 2.04, \pvalue .05$) on Individual Confidence, which suggested that an athlete's objective and perceived technical competence influenced confidence in individual ability to meet the impending challenges of the training drill.

These results did not support the prediction that higher levels of uncertainty in joint action will generate lower confidence for success of team performance in joint action.  It could be inferred that the experimental prime did not successfully manipulate athletes' explicit expectations around performance prior to the Invasion Drill.


\subsection{Prediction 1.b: More positive perceptions of team performance relative to prior expectations}

%\subsubsection{Results by condition}

It was still possible that the experimental prime influenced athletes implicitly, in ways that were not measured by athlete self-report Pre-Drill.  As such, condition-wise effects in key variables of interest were tested.

%The high difficulty prime was designed to encourage athletes to generate lower expectations for individual and team performance. It was therefore predicted that athletes in the high difficulty condition would on average report more positive perceptions of performance relative to prior expectations.  This prediction was tested for group and individual performance.

\myparagraph{Team Click}
\textit{Post-Drill:} the effect of Condition on Team Click was not significant ($\betavec .43 \CIstart -.44, 1.30 \CIfinish \SE .44, t(54)= .98, \pvalue .33$). The model did reveal significant positive effects of Subjective Competence on Team Click ($\betavec .27  \CIstart .003, .54 \CIfinish \SE .45, t(54)= 1.99, p = .05$), which suggested that higher levels of self-professed technical competence predicted feelings higher levels of team click.

\textit{Pre- to Post-Drill:} the effect of Condition on Team Click Change was not significant ($\betavec -.47 \CIstart -1.47, .54 \CIfinish \SE 2.20, t(53)= -.914, \pvalue .37$). No other main effects of the model were significant.  Results of the model suggested that Condition did not significantly predict change in athlete perceptions of team click.

\myparagraph{Social Bonding}
\textit{Post-Drill:} the effect of Condition on Social Bonding was not significant
($\betavec .33 \CIstart -.45, 1.10 \CIfinish \SE .40, t(54)= .83, \pvalue .41$). There were no other significant main effects.

\textit{Pre- to Post-Drill:} the effect of Condition on Social Bonding Change was not significant ($\betavec -.39 \CIstart -.42, .44 \CIfinish \SE .49, t(53)= -.79, \pvalue .96$). There were no other significant main effects.

These results suggested that Condition did not significantly predict athlete perceptions of social bonding to the training group, or change in perceptions of social bonding with the training group.

\myparagraph{Social Bonding (Province)}
\textit{Post-Drill:}
The effect of Condition on Social Bonding (Province) was not significant ($\betavec -.23 \CIstart 1.30, 1.37 \CIfinish \SE .39, t(53)= .18, \pvalue .75$). There were no other significant main effects of the model.

\textit{Pre- to Post-Drill:} the effect of Condition on Social Bonding (Province) was not significant ($\betavec .17 \CIstart -.23, 1.30 \CIfinish \SE .39, t(8.2)= .33, \pvalue .75$). There were no other significant main effects of the model.

Together, these results suggest that experiment condition did not predict variation in social bonding to the category of the provincial team as a whole.



\subsubsection{Exclusion of cases}
Results indicated that there were no clear observable effects of condition on variables of interest.  One possible explanation for these results is that the study design and the unconstrained nature of the training drill made it difficult to ultimately control for the quality of performance in joint action (discussed in more detail in Section~\ref{sect:discussionTrain}).  It was possible that uncontrolled variation in objective performance in each experiment session was driving perceptions of performance (versus prior expectations).  To investigate this possibility, both data subsets were analysed for a relationship between objective performance outcomes (Tries Total and Trial Outcome Average) and athlete perceptions of performance (Team Performance Vs Expectations and Individual Performance Expectations).

\textit{Tries Total:} results of an LMER model revealed that the relationship between Tries Total and Team Performance Vs Expectations was not significant ($\betavec -.02 \CIstart -.55, .51 \CIfinish \SE .27, t(12.54)= -.07, \pvalue .95$).  No other main effects in the model (arousal, objective and subjective measure of technical competence, and personality(extraversion)) were significant.

\textit{Trial Outcome Average:} Results revealed that the relationship between Trial Outcome Average and Team Performance Expectations was not significant, ($\betavec -.02 \CIstart -.16, .12 \CIfinish \SE .07, t(10.58)= -.07, \pvalue .95$). No other main effects were significant.

These results provided no justification for a relationship between objective performance outcome and perceptions of team performance relative to prior expectations.  As such, no cases were excluded from the data.

\myparagraph{Summary of results for Prediction 1.a and 1.b}
Results reported here failed to support the predictions that higher levels of uncertainty in joint action scenarios predict lower levels of expectations concerning team performance (Prediction 1.a), and higher levels of positive expectation violation (Prediction 1.b).

To investigate the extent to which the collected data could explain the remaining predictions of this study, both data subsets (Post-Drill and Pre- to Post-Drill) were collapsed into one overall sample for subsequent analysis.



\subsection{Prediction 2: More positive perceptions of team performance relative to prior expectations will predict higher levels of team click}

\myparagraph{Post-Drill}
Results of the Post-Drill data supported this prediction. There was a significant main effect of Team Performance Vs Expectations on Team Click ($\betavec .57 \CIstart .23, .90 \CIfinish \SE .17, t(5.40) = 3.21, \pvalue .02$), indicating that more positive perceptions of team performance relative to prior expectations predicts perceptions of team click (see Figure ~\ref{fig:teamPerfExpClickScatter}).  All other main effects were not significant (see Table~\ref{tab:TEM1groupPerfExpClick} in Appendix~\ref{app9:trainingExperiment}).

%Condition was included in the model as an interaction.  Results of the new model revealed a significant positive interaction effect of team performance expectation violation and condition on team click ($\betavec .41 \CIstart .07, .75 \CIfinish \SE .17, t(53) = 2.39, \pvalue .02 $).  A chi-squared test of model fit (based on assessment of AIC value for each nested model) was significant ($\chi^2 (2, N = 53) = 9.42, p = .009$), suggesting that the model in which the interaction effect featured offered a more appropriate fit of the data (for a visualisation of the results, see Figure ~\ref{fig:teamPerfExpClickScatter}).

\begin{figure}
    \centering
    \includegraphics[width=0.5\linewidth,keepaspectratio] {images/teamPerfExpClickScatter}
    \caption{Team Performance Vs Expectations predicts Team Click in the Post-Drill data ($n = 53$).  Both variables are normalised with mean of zero and SD of one.}
    \label{fig:teamPerfExpClickScatter}
\end{figure}

Model residuals were normally distributed around zero ($\resdist .99, \pvalue .97$) and an analysis of Cook's distances suggests that the model did not contain any observations that unjustifiably influenced parameter estimates (all $\cooksD < .49$), for full presentation of model assumptions, see Appendix Figure ~\ref{fig:M1Assumptions}).  These results suggest that more positive perceptions of team performance relative to prior expectations predicted stronger perceptions of team click.

%Results were in line with predictions that more positive perceptions of team performance relative to prior expectations predicted stronger perceptions of team click, an effect observed to be strongest in the high difficulty condition.
%The main effect of condition ($\betavec .76 \CIstart .12, 1.4 \CIfinish \SE .33, t(53) = 2.32, \pvalue .02 $) and team performance expectation violation on perceptions of team click with the training group ($\betavec .32 \CIstart .06, .57 \CIfinish, \SE .13, t(53)= 2.41, \pvalue .02$) were also significant.
%The average performance outcome of each Invasion Drill trial was also a significant positive predictor of team click ($\betavec .73 \CIstart .23, 1.24 \CIfinish \SE .26, t(53)= 2.86, \pvalue .006$ (see Table ~\ref{tab:M1lmer}). This result suggested that on-field performance influenced athlete perceptions of performance and team click.

\myparagraph{Pre- to Post-Drill}
Results of the Pre- to Post-Drill survey also supported this prediction.  There was a significant main effect of Team Performance Vs Expectations on Team Click Change ($\betavec .54 \CIstart .30, .78 \CIfinish \SE .12, t(53) = 4.37, p < .001$), indicating that greater positive change in perceptions of team performance relative to prior expectations predicts positive change in perceptions of team click.  All other main effects were not significant (see Table~\ref{tab:TEM1groupPerfExpClick} in Appendix~\ref{app9:trainingExperiment}).



%Results of a nested model---with condition included as an interaction---revealed a significant positive interaction effect of changes in team performance and condition on team click ($\betavec .47 \CIstart .04, .90 \CIfinish \SE .22, t(53) = 2.16, \pvalue .04 $).  While the interaction effect is clearly identifiable visually (see Figure ~\ref{fig:teamPerfExpClickScatter}),a chi-squared test of model fit was not significant ($\chi^2 (5, N = 53) = 4.49, p = .48$), suggesting that the model in which the interaction effect featured did not offer a more appropriate fit of the data.

 \begin{figure}
     \centering
     \includegraphics[width=0.5\linewidth,keepaspectratio] {images/groupPerfClickChangeCondition}
     \caption{Team Performance Vs Expectations predicts Team Click Change ($n = 53$).  Both variables are normalised as z-scores.}
     \label{fig:groupPerfClickChangeCondition}
 \end{figure}

Model residuals were normally distributed around zero ($\resdist .98, \pvalue .44$) and an analysis of Cook's distances suggests that the model did not contain any observations that unjustifiably influenced parameter estimates (all $\cooksD < .34$), for full presentation of model assumptions, see Figure ~\ref{fig:M21Assumptions} in Appendix~\ref{app9:postExperimentModelAssumptions}). Results suggested that change in perceptions of team performance were a positive predictor of increase in perceptions of team click following the training drill.  Results from both Post-Drill and Pre- to Post-Drill subsets were in line with the prediction that more positive perceptions of team performance relative to prior expectations predicts perceptions of team click.



%An LMER, with change in perceptions of team performance, condition, and their interaction included in the model as fixed effects, revealed a significant positive interaction effect of team performance vs expectations and condition on team click ($\betavec .49 \CIstart .06, .92 \CIfinish \SE .22, t(53) = 2.22, \pvalue .03 \MR \CR $).
%The main effect of change in team performance expectation on change in team click was also (highly) significant, ($\betavec .76 \CIstart .46, 1.06 \CIfinish \SE .15, t(53) = 5.018, \pvalue .00006 $). The main effect of condition, however, was not significant ($\betavec .08 \CIstart -.69, .85 \CIfinish \SE .39, t(53) = .20, \pvalue .84 $).
%These results---visualised in figure ~\ref{fig:groupPerfClickChangeCondition}---

%Model residuals were normally distributed around zero ($\resdist .98, \pvalue .45$) and an analysis of Cook's distances suggests that the model did not contain any observations that unjustifiably influenced parameter estimates (all $\cooksD < .34$), for full presentation of model assumptions, see Appendix Figure ~\ref{fig:M1PrePostAssumptions}).


\subsection{Prediction 3: Higher levels of team click will predict higher levels of social bonding (to the training group or provincial team)}

\myparagraph{Prediction 3.a: Perceptions of team click predict perceptions of social bonding to the training group}

\myparagraph{Post-Drill}
Results of the Post-Drill survey supported this prediction. The model revealed a significant main effect of Team Click on Social Bonding ($\betavec .63 \CIstart .46, .79 \CIfinish \SE .09, t(53) = 7.28, p < .01$), indicating that stronger perceptions of team click predict stronger perceptions of social bonding to the training group.  All other main effects of the model were not significant.  Model residuals were normally distributed around zero ($\resdist .99, \pvalue .80$) and there were no unjustifiably influential cases (all $\cooksD < .6$, see model assumptions in Appendix~\ref{app9:postExperimentModelAssumptions} Figure ~\ref{fig:M2Assumptions}).





 %These results suggest that athletes who felt stronger levels of team click also felt stronger levels of social bonding to their training group.
%Results of a model in which condition was included as an interaction revealed that the interaction effect of team click and condition on social bonding was not significant, ($\betavec .23 \CIstart -.08, .60 \CIfinish \SE .17, t(23.33) = 1.35, \pvalue .19$).  A chi-squared test of model fit also indicated that the model did not explain the variance with any more precision as the initial model, ($\chi^2 (5, N = 53) = 1.78, p = .88$). These results were confirmed by their visual representation, in which an interaction effect is not obvious (see Figure ~\ref{fig:groupClickBondScatter}).  Thus, the initial model was adopted as the best fit for the data.


\begin{figure}
  \centering
    \includegraphics[width=0.5\linewidth,keepaspectratio] {images/groupClickBondScatter}
    \label{fig:groupClickBondScatter}
    \caption{Team Click predicts Social Bonding in the Post-Drill data ($n = 53$).  Both variables are standardised factors ($mean = 0, SD = 1$).}
\end{figure}


%a significant positive interaction effect of team performance expectation violation and condition on team click ($\betavec .41 \CIstart .07, .75 \CIfinish \SE .17, t(53) = 2.39, \pvalue .02 $).  A chi-squared test of model fit (based on assessment of AIC value for each nested model) was significant ($\chi^2 (2, N = 53) = 9.42, p = .009$), suggesting that the model in which the interaction effect featured offered a more appropriate fit of the data (for a visualisation of the results, see Figure ~\ref{fig:teamPerfExpClickScatter}).

%A LMER model in which team click, condition, and their interaction were included as fixed effects, revealed a strong positive relationship between perceptions of team click and perceptions of social bonding to the training group, $\betavec .43 \CIstart .15, .72 \CIfinish \SE .14, t(29.22) = 2.96, \pvalue .006, \MR , \CR $.
%The main effect of condition ($\betavec .13 \CIstart .15, .72 \CIfinish \SE .29, t(18.7) = 3.15, \pvalue .006$) and the interaction effect of team click and condition on social bonding ($\betavec .26 \CIstart -0.08, .60 \CIfinish \SE .17, t(30.43) = 1.51, \pvalue .14$) were not significant.  These results did, however, appear to be trending in the predicted direction, whereby the relationship between click and bonding appeared stronger in the high difficulty condition.

\myparagraph{Pre- to Post-Drill}
Results of the Pre- to Post-Drill survey also offered support for this prediction.  A LMER revealed a significant main effect of Team Click Change on Social Bonding Change ($\betavec .27 \CIstart .01, .54 \CIfinish \SE .13, t(53) = 2.05, p < .05$), such that an increased change in team click predicted an increased change in social bonding.  All other main effects of the model were not significant. Model residuals were normally distributed around zero ($\resdist .96, \pvalue .08$), and Cook's Distances were less than $.5$, indicating that the model did not to contain any observations that unjustifiably influenced parameter estimates (see Figure ~\ref{fig:M22Assumptions} in Appendix~\ref{app9:prePostExperimentModelAssumptions} for model assumptions).



%Results of a model in which condition was included as an interaction revealed a significant interaction effect of team click and condition on social bonding, ($\betavec .82 \CIstart .38, 1.25 \CIfinish \SE .22, t(47.58) = 3.69, \pvalue .0006$).  A chi-squared test of model fit indicated that the model explained the variance with any more precision than the initial model, ($\chi^2 (5, N = 53) = 11.35, p = .04$).  Graphing the interaction effect clearly demonstrates the extent of the effect (see Figure ~\ref{fig:groupClickBondScatter}).



\begin{figure}
  \centering
    \includegraphics[width=0.5\linewidth,keepaspectratio] {images/groupClickBondingChangeCondition}
    \label{fig:groupClickBondingChangeCondition}
    \caption{Team Click Change predicts Social Bonding Change in the Pre- to Post-Drill data ($n = 53$)}
\end{figure}

%In contrast to the Post-Drill data, the Pre-Post Data revealed not  only a main effect of (change in) team click on (change in) social bonding, but also an interaction effect with condition; the effect of team click on social bonding was more pronounced in the high difficulty condition than the low difficulty condition.


%A LMER model revealed a significant interaction effect of change in team click and experiment condition on change social bonding, ($\betavec .76 \CIstart .32, 1.19 \CIfinish \SE.22, t(47.05) = 3.41, \pvalue .001, \MR , \CR $).  The main effect of change in team click was also a significant predictor of change in social bonding, ($\betavec .58 \CIstart .32, .84 \CIfinish \SE.13, t(40.03) = 4.36, p < .01 $).  The main effect of condition was not significant, ($\betavec .03 \CIstart -.79, .84 \CIfinish \SE.41, t(17.66) = .06, p < .95 $).  A scatter plot shows that the relationship between change in team click and change in bonding is most pronounced in the high difficulty condition (see figure ~\ref{fig:groupClickBondingChangeCondition}).




\myparagraph{Prediction 3.b: Perceptions of team click predict perceptions of social bonding to the provincial team}

\myparagraph{Post-Drill}
Results of the Post-Drill survey supported this prediction. The model revealed a significant main effect of Team Click on Social Bonding (Province) ($\betavec .50 \CIstart .21, .79 \CIfinish \SE .15, t(5.13) = 3.442, p = .02$), indicating that stronger perceptions of team click predicted stronger perceptions of social bonding to the provincial team.  All other main effects of the model were not significant.  Model residuals were normally distributed around zero ($\resdist .96, \pvalue .10$) and there were no unjustifiably influential cases (all $\cooksD < .7$, see model assumptions in Figure ~\ref{fig:M2TeamAssumptions} in Appendix~\ref{app9:postExperimentModelAssumptions}).



 %These results suggest that athletes who felt stronger levels of team click also felt stronger levels of social bonding to their training group.
%Results of a model in which condition was included as an interaction revealed that the interaction effect of team click and condition on social bonding was not significant, ($\betavec .23 \CIstart -.08, .60 \CIfinish \SE .17, t(23.33) = 1.35, \pvalue .19$).  A chi-squared test of model fit also indicated that the model did not explain the variance with any more precision as the initial model, ($\chi^2 (5, N = 53) = 1.78, p = .88$). These results were confirmed by their visual representation, in which an interaction effect is not obvious (see Figure ~\ref{fig:groupClickBondScatter}).  Thus, the initial model was adopted as the best fit for the data.

\begin{figure}
  \centering
    \includegraphics[width=0.5\linewidth,keepaspectratio] {images/groupClickTeamBondScatter}
    \label{fig:groupClickTeamBondScatter}
    \caption{Team Click predicts Social Bonding (Province) in the Post-Drill data ($n = 53$).}
\end{figure}


\myparagraph{Pre- to Post-Drill}
The main effect of Team Click Change Social Bonding (Province) was not significant, ($\betavec .19 \CIstart -.11, .48 \CIfinish \SE .14, t(47.51) = 1.25, p < .05$).  All other main effects of the model were not significant. This result suggested that change in perceptions of team click did not predict change in perceptions of social bonding to the provincial team.










\subsection{Prediction 4: More positive perceptions of team performance relative to prior expectations will predict higher levels of social bonding (to the group and provincial team)}


\myparagraph{Prediction 4.a: More positive violations of team performance expectations will predict higher levels of social bonding to the training group}

\myparagraph{Post-Drill survey}
A LMER model revealed that the main effect of Team Performance Vs Expectations on Social Bonding was not significant, ($\betavec .25 \CIstart -.10, .59 \CIfinish \SE .18, t(5.21) = 1.39, \pvalue .22$).  The only significant main effect of the model was arousal, ($\betavec .23 \CIstart .01, .44 \CIfinish \SE .11, t(39.37) = 2.10, \pvalue .04$), suggesting that reports of higher arousal at the end of the drill were significantly associated with higher levels of social bonding.  These results did not directly support the prediction that more positive violations of expectations around team performance would also predict higher levels of social bonding.

%However, a second model with condition included as an interaction revealed a significant interaction between team performance expectations and condition on social bonding, ($\betavec .44 \CIstart .08, .81 \CIfinish \SE .18, t(53) = 2.41, \pvalue .02$).  A chi-squared test of model fit indicated that the model explained the variance with any more precision than the initial model, ($\chi^2 (2, N = 53) = 6.75, p = .03$).  Graphing the interaction effect clearly demonstrates the extent of the effect (see Figure ~\ref{fig:groupPerfExpBondConditionScatter}).

%Model residuals of this second model were normally distributed around zero ($\resdist .97, \pvalue .31$), and Cook's Distances were less than $.6$, indicating that the model did not to contain any observations that unjustifiably influenced parameter estimates (for full presentation of model assumptions, see Figure ~\ref{fig:M3Assumptions} in Appendix ~\ref{app9:trainingExperiment}).

%\begin{figure}
%  \centering
%  \includegraphics[width=0.5\linewidth,keepaspectratio] {images/groupPerfExpBondConditionScatter}
%% predict perceptions of social bonding to training group}
% \label{fig:groupPerfExpBondConditionScatter}
%\end{figure}

%Average performance in each trial was also a significant predictor of social bonding, $\betavec .65 \CIstart .11, 1.19 \CIfinish \SE .27, t(53) = 2.37, \pvalue .02$. The main effects of team performance vs expectations ($\betavec .008  \CIstart -.27, .28 \CIfinish \SE .14, t(53) = .06, \pvalue .96$) and condition ($\betavec .56  \CIstart 1.12, 1.25 \CIfinish \SE .35, t(53) = 1.6, \pvalue .11$) were not significant. The result for the fixed effect of condition did, however, appear to be trending in the predicted direction ($p = .11$). These results indicate that the the relationship between team performance expectations vs expectations and social bonding was significant only in the high difficulty condition, as demonstrated visually in figure ~\ref{fig:groupPerfExpBondConditionScatter}. Model residuals were normally distributed around zero ($\resdist .97, \pvalue .31, and all \cooksD < .55$, see model assumptions in Appendix  ~\ref{fig:M3Assumptions}).  These results suggested that the relationship between positive violation of team performance expectations and social bonding was significant, only in the high difficulty condition, and not overall across conditions.




\myparagraph{Pre- to Post-Drill survey}
Models of the Pre- to Post-Drill data also produced similar results.  A LMER model revealed that the main effect of Team Performance Vs Expectations on Social Bonding Change was not significant, ($\betavec -0.03 \CIstart -.44, .39 \CIfinish \SE .21, t(7.57) = -.12, \pvalue .91$). The model did reveal a signifiant main effect of Team Performance Vs Expectation on Social Bonding Change, ($\betavec .28 \CIstart .02, .55 \CIfinish \SE .13, t(49.39) = 2.13, \pvalue .04$), suggesting that more positive increase in perceptions of individual performance predicted more positive increase in social bonding between Pre- and Post-Drill surveys.  However, these results did not directly support the prediction that more positive violations of expectations concerning team performance would predict higher levels of social bonding.

Taken together, results did not support the prediction that perceptions of team performance relative to prior expectations directly predict social bonding.


%A second model with condition included as an interaction revealed a significant interaction between team performance expectations and condition on social bonding, ($\betavec .92 \CIstart -1.38, -.48 \CIfinish \SE .23, t(53) = 4.04, \pvalue .0002$).  A chi-squared test of model fit indicated that the model explained the variance with any more precision than the initial model, ($\chi^2 (5, N = 53) = 11.00, p = .05$).  Graphing the interaction effect clearly demonstrates the interaction effect (see Figure ~\ref{fig:groupPerfExpBondConditionScatter}).

%Model residuals of this second model were normally distributed around zero ($\resdist .98, \pvalue .55$), and Cook's Distances were less than $.3$, indicating that the model did not to contain any observations that unjustifiably influenced parameter estimates (for full presentation of model assumptions, see Figure ~\ref{fig:M23Assumptions} in Appendix ~\ref{app9:trainingExperiment}).

%\begin{figure}
%  \centering
%  \includegraphics[width=0.5\linewidth,keepaspectratio] {images/groupPerfBondingChangeCondition}
%  \caption{Change in perceptions of team performance predict change in social bonding to the training group, only in the high difficulty condition ($n = 53$).  }
 %%\label{fig:groupPerfBondingChangeCondition}
%\end{figure}

%These results suggested that the relationship between team performance vs expectations and social bonding was only significant only in the high difficulty condition.

%The interaction effect of change in perceptions of team performance and condition significantly predicted variation in change in social bonding, ($\betavec .49 \CIstart .46, 1.38  \CIfinish \SE .53, t(53) = 4.03, \pvalue .0001, \MR , \CR $). The main effect of change in team performance was also significant, ($\betavec .21  \CIstart .16, .78 \CIfinish \SE.23, t(53) = 2.33, \pvalue .02$), as was the main effect of change in perceptions of individual performance, ($\betavec .26  \CIstart .02, .50 \CIfinish \SE.12, t(53) = 2.15, \pvalue .04$).
%For a full report of model statistics, see APPENDIX ~\ref{tab:M2PrePostOutput}.  This model suggested that the relationship between team performance expectations vs expectations and social bonding was significant only in the high difficulty condition, as demonstrated visually in figure ~\ref{fig:groupPerfExpBondConditionScatter}.

%Model residuals were normally distributed around zero (\resdist .97, \pvalue .31) and all $\cooksD < .22$, see figure ~\ref{fig:M23Assumptions} Appendix ~\ref{app9:trainingExperiment}).   For a full report on tests of model assumptions, see APPENDIX ~\ref{fig:M23Assumptions}.



\myparagraph{Prediction 4.b: More positive violations of team performance expectations will predict higher levels of social bonding to the provincial team}

\myparagraph{Post-Drill survey}
A LMER model revealed that the main effect of Team Performance Vs Expectations on Social Bonding (Province) was not significant, ($\betavec .24 \CIstart -.02, .49 \CIfinish \SE .13, t(8.62) = 1.83, \pvalue .10$). No other main effects in the model were significant.

\myparagraph{Pre- to Post-Drill survey}
 A LMER model revealed that the main effect of Team Performance Vs Expectations on Social Bonding (Province) Change was not significant, ($\betavec .16 \CIstart -.44, .39 \CIfinish \SE .21, t(8.97) = .714, \pvalue .49$).

Results of both models did not support the prediction that perceptions of team performance relative to prior expectations directly predict social bonding to the provincial team.



\subsubsection{Prediction 5: Perceptions of team click will mediate a relationship between more positive violations of expectations around team performance and social bonding to the group}

Due to the fact that the direct path between Team Performance Vs Expectations did not significantly predict social bonding (to the training group or to provincial team), a mediation analysis was not appropriate.




































\clearpage
\section{Discussion\label{sect:discussionTrain}}
%study:
This study was designed in order to investigate the phenomenon of team click in a controlled field experiment. Here, uncertainty was manipulated in order to observe its predicted effect on expectations concerning team performance.  Results did not show evidence for the primary prediction that higher levels of uncertainty in joint action will generate more positive perceptions of team performance relative to prior expectations (Prediction 1).  These results appeared to be explained by the failure of the experiment design to successfully manipulate athlete perceptions of confidence.  Granting the possibility that the manipulation was effective but was not reflected in self-report measures, data were analysed for condition-wise effects, but results suggested no evidence for study predictions.

These results thus suggest two possible implications: the experiment did not successfully manipulate uncertainty in joint action, or the hypothesis concerning the role of uncertainty in joint action is incorrect.  In this discussion I will first attempt to account for factors of experiment design and execution that support the first implication, before attending to the question of hypothesis validity.


Considering limitations with experiment design, it is highly likely that the manipulation failed to sort variation in athlete expectation by condition.  The design of this experiment enabled control over athlete expectations prior to drill, but it did not enable control for the success of performance in the drill.  Piloting of the Invasion Drill with junior athletes of the Beijing rugby program established a difficulty rating of 4.7 out of 10 (range = 3—7), relative to a less complex ``2+1+1'' drill (difficulty = 2, range = 2–4), and a more complex ``6 on 4 continuous'' drill (difficulty = 6.2, range = 4–10; n = 20).  Based on these results, it was deemed that the Invasion Drill would establish a middling difficulty.  In other words, given non-trivial levels of coordination required of athletes in the Invasion Drill, it was not expected that all trials within each experiment session would be a roaring success.  Considering also the possibility of high variation in training age and experience with rugby within each experimental session, it could be expected that performance would vary.  Indeed, results suggest within- and between-group variation in objective performance, irrespective of experimental condition (see Table~\ref{tab:trainingObjPerformanceOutcomeTable}).

Two foreseeable solutions for overcoming this issue exist.  First, the same experiment could be run with a larger sample size, and cases could be selected based on a pre-determined base level of objective (successful) performance.  As such, the effect of the manipulation of uncertainty could be examined using a subset of successful performances only.  In this study, there was no relationship between objective performance and expectation violation, and the small sample size made exclusion of data unsuitable.  Indeed, the field experiment format of this study also meant that collection of more data was practically challenging.

Alternatively, alterations to the experiment design could enable more satisfactory control for athlete performance, and, in turn, athlete expectation violation.  An ideal experiment would involve a reliable comparison between levels of positive violation of expectation, with performance held constant.  This design would enable testing of the hypothesis that team click (and social bonding) will be higher when \textit{successful} team performance comes as a surprise vs. when it does not come as a surprise.  Future pilots of this paradigm (or one like it) would establish a measurably successful manipulation, on which the outcome variables of interest to this thesis, namely team click and social bonding, could be re-examined.

In addition to these considerations, it was also possible that the experiment manipulation was too subtle, or that the training paradigm was too mundane to arouse variation in expectations surrounding team performance.  The manipulation---delivered electronically via WeChat and then verbally immediately pre-drill—was relatively subtle.  Moreover, the manipulation was designed and delivered by a researcher who spoke Chinese as a second language, and was thus perhaps unable to manufacture sufficient levels of attention and believability.  In contrast to the high-stakes, high-intensity National Tournament (see Chapter ~\ref{chap:chap:tournamentSurvey}), this study was conducted as part of athletes' usual training regime, and with very little obvious consequentiality or reward for participation or performance (athletes were not, for example,  reimbursed for participating in the experiment).  In addition, the limitations of research context, namely time constraints and access to athletes, were such that pilot phase was not exhaustive as it could have been (only 2 trials per condition were ran).  These limitations are representative of the types of methodological challenges associated with attempting to bring the laboratory to the field \citep{Xygalatas2012}.\footnote{See also Chapter~\ref{sect:generalDiscussion} for a discussion of the inverse but analogous challenges associated with bringing the field to the lab.}

Future manipulations of uncertainty and expectation in joint action should also consider the use of measures beyond self-report to ascertain manipulation effectiveness and to measure expectation violation.  As discussed in Chapter~\ref{chap:theory}, experimental manipulations of uncertainty have shown effects on various physiological and autonomic responses such as heart rate \citep{Averill1972} and skin conductance \citep{Epstein1970}. While not explicitly designed to feature as a check for effectiveness of the experimental manipulation, arousal was measured in this experiment.  However,  a follow-up analysis revealed no significant effect of of condition on athlete pre-drill arousal  ($\betavec .07 \CIstart -.29, .44 \CIfinish \SE .19, t(53)= .38, \pvalue .71$).

Despite these methodological limitations, the study did offer some confirmation for predictions.  A supplementary analysis of a collapsed sample of athletes confirmed results of the previous study.  Namely, athletes who did experience more positive perceptions of team performance relative to prior expectations also perceived higher perceptions of team click (Prediction 2).  Higher levels of team click also significantly predicted higher levels of social bonding (Prediction 3).  The direct relationship between perceptions of performance relative to prior expectations and social bonding was not significant, and thus the prediction that team click mediates a relationship between positive expectation violation and social bonding was not supported (Prediction 5). These findings extend the empirical substantiation of a general theory of team click in group exercise settings, while also leaving many important theoretical and empirical questions to be considered.  I will attend to some of these questions in the following chapter by way of concluding this thesis.


                                                          \end{CJK}
