\chapter{\label{partBintroMethod}Introduction to Part B - Experimental Studies}


  \minitoc


\section{Abstract}

In this section of the dissertation, I use findings from part A (ethnography) to inform the design of two \textit{in situ} empirical studies that draw on a broader sample of professional Chinese rugby players.  In this chapter, I outline the method through which I operationalised 1) theory from the social cognition of joint action and 2) ethnographic results in the form of both existing (validated) and novel psychological constructs.  The In addition, I introduce the two empirical studies: 1) an \textit{in situ} survey study of athletes competing in a national rugby tournament ($n = 174$) and 2) a controlled field experiment designed to isolate and test the relationship between joint action (and its component mechanisms) and team click and social bonding ($n= 58$).

\section{Vignette}



\section{Introduction}

Conviction around the existence of the phenomenon of team click is powered by strong intuition and anecdote rather than rigorous theory or empirical evidence. To what extent team click exists as a generalisable phenomenon in human populations engaged in joint activities such as sport, is largely unknown.  In this section of the dissertation, I report the design, method, and results of two studies, which were conducted to test for the existence of a relationship between joint action and social bonding (mediated by team click) in a broader population of athletes beyond the specific ethnographic setting of the Beijing men's rugby program.


\section{Method}

In this section I explain how I operationalise theory and ethnographic observations into psychometric constructs appropriate for field experiments.

\subsection{Conversion of ethnographic results}

      \subsubsection{Perceptions of Performance}

          \myparagraph{What I predicted from theory}
          \myparagraph{What I found in ethnography}

                    \begin{description}
                      \item [Attention to performance]
                        \begin{description}
                          \item [Team Performance]
                            \begin{description}
                              \item [Overall]
                              \item [Components]
                            \end{description}
                          \item [Individual Performance]
                          \begin{description}
                            \item [Overall]
                            \item [Components]
                          \end{description}
                        \end{description}
                        \item [Expectation violation around performance]
                    \end{description}



          \myparagraph{Conversion}


      \subsubsection{Team Click}

          \myparagraph{What I predicted from theory}
          \myparagraph{What I found in ethnography}

          \begin{description}
                \item [Tacit Understanding]
                \item [Team Atmosphere]
                \item [General order and organisation]
          \end{description}

          \myparagraph{Conversion}

      \subsubsection{Social Bonding}

            \myparagraph{What I predicted from theory}
            \myparagraph{What I found in ethnography}

              \begin{description}
                \item [Emotional Support]
                \item [Shared Goal]
                \item [Social Identity]
              \end{description}

            \myparagraph{Conversion}

            \begin{description}
              \item [Dunbar's bonding]
              \item [Tomasello's bonding]
              \item [Whitehouse's bonding]
            \end{description}


      \subsubsection{Moderators}

      \myparagraph{What I predicted from theory}
      \myparagraph{What I found in ethnography}

            \begin{description}
              \item [Personality]
              \item [Injury]
              \item [Fatigue]
              \item [Technical Competence]
            \end{description}


\subsection{Empirical Studies}

      \subsubsection{Tournament Survey}

      \subsubsection{Training Experiment}




\section{Discussion}
