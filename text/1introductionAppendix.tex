\chapter{\label{app1:intro}Appendix}












\section{Group hunting in Chimpanzees\label{sect:groupHunt}}
Group hunting has been well documented in various geographically distinct populations of Chimpanzees (i.e. by \textcite{Goodall1986}  in Gombe national park, by \textcite{Uehara1997} in Mahale Mountains National Park, and by \textcite{Boesch1989, Boesch1994} in the Tai National Park in Ivory Coast West Africa.  This triangulation of observations suggests a species-wide capacity for elaborate coordination combined with bouts of intense physiological exertion.   Observational studies suggest that the size and membership of hunting parties vary greatly, from a single chimpanzee to as many as 35 \citep{Stanford1994}.   In addition, hunting frequency appears to covary with fluctuations in group size (and particularly number of sexually receptive females in the group \citep{Stanford1996}.
Observational studies suggest that the size and membership of hunting parties vary greatly, from a single chimpanzee to as many as 35 \citep{Stanford1994}: Typically, a group first gathers and then simultaneously fans out in search for a victim while reducing the chances of detection by avoiding vocalisations and using slow and careful steps to minimise noise \citep{Boesch1989,Mitani2001}.  There is also a division of labor based on roles that was characterized by Boesch and Boesch (1989) as collaboration: a ``blocker;'' a ``chaser;'' an ``ambusher;'' and others remaining on the ground to track the hunt and intercept a fleeing target if the opportunity arises.

The evolutionary function of group hunting in Chimpanzees may not be limited to a nutritional shortfall hypothesis (which suggests that chimpanzees hunt to compensate for seasonal shortages in food availability).  A second hypothesis argues that male chimpanzees hunt to obtain meat that they swap for matings. A third hypothesis proposes that males use meat as a social tool to develop and maintain alliances with other males \citep{Mitani2001}



\section{The Modern Synthesis\label{sect:modernSynthesis}}
Generally speaking, rigorous and scientifically testable accounts of human behaviour have emerged in the last ~70 years, facilitated by 1) the gradual refinement of evolutionary theory over the last 200 years now known as the ``modern synthesis,'' as well as 2) the ``cognitive revolution'' of the 1950s and 60s, in which mechanisms associated with information theory, cybernetics, and computation provided useful conceptual metaphors for understanding population-level transmission and fixation of biological and cultural variants. Below, I provide a brief overview of the main assumptions, protocols, and critiques of the modern synthesis, identifying knowledge gaps and research opportunities along the way.

The modern synthesis (also known as ``neo-Darwinism'', hereafter simply MS) refers generally to the gradual maturation of evolutionary theory in the last 200 years, and specifically to the unification of the theory of evolution by natural selection (attributed to Darwin and Wallace in the second half of the 19th century) with a theory of genetic inheritance (replacing a previously popular theory of blended inheritance).  The MS was first proposed by Huxley in 1942, following successful mathematical formalisations performed by population geneticists between 1930 and 1947 (e.g., Fisher (1930) and Haldane (1932)).  Subsequent advances in molecular biology and genetics, including the verification of the structure of the DNA molecule by Wallace and Crick in 1954, paved the way for a definition of biological evolution as changes in the frequency of heritable DNA sequences in a population due to selection pressures exerted at the level of the phenotype \citep{Dawkins1976,Grafen1984}.  Shown to be mathematically plausible, the mechanism of genetic inheritance served to explain observable intra-species phenotypic variation (for which the preceding theory of blended inheritance failed to account), and confirmed Darwin's original insight that organismic change occurs via gradual population-level accumulation of adaptive traits over evolutionary time. The MS and its associated methodological innovations and empirical findings have collectively transformed scientific knowledge of the historical origins and function of biological life, including human life. The modern synthesis has been widely touted as the second most successful theory in the history of science after modern quantum physics in its ability to explain and predict the world in which we live \citep{Dunbar1996}.

The power of the MS to account for observable biological phenomena hinges on two interrelated assumptions: (1) evolutionary explanations of a biological trait are solely determined by the process of natural selection, and (2) the consequences of the process of natural selection accumulate in the germ-line in the form of statistical frequencies of programs (alleles) for specific phenotypic traits.  As Mayr first suggested in 1961, these two interrelated assumptions create conceptual space for two distinct but complimentary explanations of observable biological phenomena---one evolutionary or ``ultimate'' level, and one developmental or ``proximate'' level \citep{Mayr1961}.  Explained in detail below, the ``proximate-ultimate'' distinction is a core epistemological pillar of the MS.

Assuming that natural selection is the primary agent of evolutionary change, biologists interested in an evolutionary explanation can largely set to one side immediate processes that contribute to the development or causation of a biological trait, and instead focus on the adaptive value of a trait and its phylogenetic history. \citep{Mayr1961,Tinbergen1963}.  Proximate causal mechanisms and developmental processes are not passed on in the germ-line and are thus relatively inconsequential to macro-scale processes of evolutionary change (\citep{Dawkins1982,Grafen1991,Svensson2017} but see \citep{Laland2013,Laland2015}.  Meanwhile, biologists interested in explaining the developmental processes and immediate causal mechanisms associated with a trait can proceed under the assumption that an observable phenotypic trait is the product of interaction between the organism's internal evolved biological programs (genes) and external environmental triggers or cultural capacities emanating from gene-environment interactions within a specific phenotypic niche (SOURCE).  Importantly, proximate and ultimate levels of causation must be considered together, as distinct but complimentary components of an overall explanation of the biological phenomena\citep{Mayr1961,Tinbergen1963,Scott-Phillips2011}.
%Something on Aristotle's four questions here...philosophical/epistemological groundings
%if natural selection is the primary agent of shaping biological information (assumption 1), then it can be predicted that the programs for subsequent life contained in the germ-line (assumption 2) have been subject to gradual historical processes of selection.
%Application of the MS to biological phenomena has spawned enormously productive and wide ranging research programs, which have contributed to communication between previously disparate strands of biology (structural, functional, evolutionary) \citep{Svensson2017}.

In the case of the commonly cited example of a human infant crying \citep[taken from][]{Scott-Phillips2011,Nettle2009}), a proximate explanation of this behaviour would require an account of both the external (e.g., physical separation from the caregiver, lack of food, cold) and internal (e.g., activity of the limbic system to initiate crying or the role of endogenous opioids in the cessation of crying) factors responsible for the crying behaviour \citep[38]{Scott-Phillips2011}. An ultimate causal explanation for human infants crying includes both a description of the adaptive value of crying (e.g., crying elicits support and defence from mothers and other care-givers; infants that do not cry when in need of assistance are less likely to survive), as well as an account of its phylogenetic history \citep{Mayr1961,Tinbergen1963}.
Any explanation of biological phenomena requires an account of both proximate and ultimate levels of causation---simply knowing \textit{how} it is that an infant cries (proximate), or alternatively only knowing \textit{why} it is present in the phenotypic repertoire (ultimate) is not enough to satisfactorily account for the observed phenomena \citep[38]{Scott-Phillips2011}.  In effect, the MS protocol of distinguishing between proximate-ultimate levels of explanation constructs two simple and testable causal models on complimentary levels of biological time---one being the phenotypic lifespan of the organism (proximate), the other being the phylogenetic (evolutionary) trajectory of the species. In so doing, the proximate-ultimate distinction facilitates an efficient division of labour in the dauntingly complicated task of wrangling into a theoretically consistent and empirically substantiated scientific domain, the complex, continuous, and historical processes of biological life \citep{Mayr1961}. In this regard, the MS has facilitated communication and integration between what were previously disparate strands of biology around a single explanatory project \citep{Svensson2017}.

Ever since the first declarations of the viability of a theoretical synthesis between Darwinian natural selection and Mendelian genetic inheritance, momentum in favour of the simplifying assumptions and protocols of the MS outlined above has been coupled with theoretical undercurrents, in the form of periodical critiques, revisions, and proposed extensions to the MS that call these assumptions and protocols into question \citep[see, for example][]{Waddington1950,Gould1980,Levins1985,Ingold1990,Ingold1995,Odling-Schmee2003,Piggliuci2007}. Generally speaking, critiques boil down to the claim that the simplifying assumptions and protocols of the MS (such as the proximate-ultimate distinction) do not satisfactorily account for the causal complexity of evolutionary processes \citep{Laland2011}. In the most recent calls for an ``extended evolutionary synthesis,'' \citep[EES, see][]{Pigliucci2007}, for example, researchers claim that the dichotomy between proximate and ultimate causal models impedes adequate theorisation and measurement of the ways in which proximate-level processes are linked with ultimate-level evolutionary processes via positive and negative feedback loops of reciprocal causation \citep{Pigliucci2007,Laland2011,Laland2013,Mesoudi2013,Laland2015}. Evidence within developmental biology suggests that processes of ontogeny typically understood as proximate (and therefore inconsequential to evolutionary change) can shape and co-direct evolutionary trajectories through reciprocal processes that ``reverberate'' through the organism \citep{Laland2013}. Proponents of the EES outline key domains such as developmental bias, extra-genetic forms of inheritance (i.e., epigenetic, parental, or cultural systems), niche construction, and phenotypic plasticity, which can be responsible for initiating evolutionary episodes and shaping evolutionary outcomes within specific evolutionary niches \citep{Laland2015}.  The EES thus challenges the exclusive position of natural selection as the sole director of evolutionary change.  Instead, it has been suggested that the emphasis on natural selection in the evolutionary landscape should be relaxed to make room for modelling other demographic, social, and spatial factors that are hypothesised be responsible for the influence of population-level distributions of genetic and phenotypic variants \citep{Mesoudi2013}.  The key methodological and empirical challenge to this suggestion is that these factors often spontaneously (self-organising) emerge as system dynamics, making them harder to manipulate and measure in traditional ``snapshot'' experimental paradigms \citep{Svensson2017}.

The key claim of the EES is captured by developmental biologist and leading proponent of the EES, Keven Laland, and his inversion of E.O. Wilson's original proclamation made in 1978 against the prevailing behaviourist paradigm common within the human sciences at the time: that human culture is held on a genetic ``leash'' \citep{Wilson1978}.  In 2017, Laland claimed that genes should be more accurately depicted as dog-walkers struggling to retain control of a number of unruly dogs (niche construction, developmental plasticity, developmental biases, non-genetic inheritance, etc.) pulling in different directions at different intensities \citep{West-Eberhard2003,Laland2017}.  Evolution, in this image, is depicted by the outcome of the struggle between dog-walker and dogs.  As Laland \citep[723][]{Laland2013} and colleagues suggest, ``much of adaptive evolutionary change may have its origin in plastic responses to novel environments, later followed by genetic changes that stabilize and fine-tune those phenotypes, rather than the other way around.'' Only by modelling and quantifying the dynamic coupling (reciprocal relationship) of the dogs (developmental processes) to their owner (genes) over time can the various contributors to evolutionary change be sufficiently represented \citep{Laland2013,Laland2015}.

It is worth noting that despite the most recent calls for an EES, and beneath the highest profile debates between opposite ends of the MS theoretical spectrum (for example, the popularised debate between E.O. Wilson and Richard Dawkins over the generalisability of kin selection and/or multilevel (group) selection), consistent and productive attempts have been made by evolutionary biologists to integrate the system dynamics of evolutionary processes into empirical research programs \citep{Wray2014,Svensson2017}.  Beginning with Fisher's initial attempts to account for the reciprocal dynamics of sexual selection on gene frequencies in initial mathematical formulations of the MS \citep{Fisher1930}, there have been numerous attempts to either empirically measure or mathematically model bidirectional causation in evolutionary processes.  Such attempts can identified in areas such as coevolution (SOURCE), frequency-dependent selection \citep{Prum2010}, sexual selection \citep{Svensson2009}, speciation \citep{Mayr1965}, and canalisation \citep{Waddington1950}.  As I will discuss in more detail below, the bidirectional causation of evolutionary processes is particularly relevant to human evolution, and has been modelled in relation to human social evolution in coalition formation and cooperative networks \citep{Gavrilets2008}, as well as in approaches that attempt to model the interaction between biological and cultural systems of transmission \citep{Cavalli-Sforza1981,Cavalli-Sforza1989,Boyd1988,Henrich2007,Claidiere2007}.  Reciprocal causation is also central to the field of ``eco-evolutionary dynamics,'' in which the bidirectional feedbacks between ecological (demographic, social, spatial) and evolutionary processes (genetic change within populations) are modelled, particularly in instances in which ecological and evolutionary timescales converge \citep{Hendry2017}.
While the EES represents the most recent and vocal critique of the narrowest assumptions and protocols of the MS, it is important to keep in mind that consistent empirical and theoretical progress has been made within the existing parameters of the MS.  Indeed, many evolutionary biologists insist that the predictions of the MS are robust and resilient enough to endure continued innovation and modification, including the incorporation of nonlinear dynamics, without the need for an ``extension'' \citep{Wray2014,Lewens2017,Svensson2017}.
%epistemological TENSION here between line and circle.

Despite these areas of theoretical contentions, almost all evolutionary theorists now agree on the point that the only way to progress evolutionary approaches to knowledge beyond theoretical advocacy is to initiate methodologically innovative empirical research programs designed to test and expand the predictions of both the MS and EES, whether or not they constitute the same or separate scientific paradigms.  In particular, this includes utilising methods capable of quantifying (and qualifying) reciprocal causation in both micro and macro evolutionary processes  \citep{Wray2014,Laland2014,Laland2015,Svensson2017}.  As I explain below, the social cognition of joint action is a research domain rich in methodological and empirical opportunities, and in which key knowledge gaps in evolutionary approaches to behaviour (particularly approaches to human behaviour) can be addressed.














\section{Neurobiological reward in exercise\label{sect:neuroRewardGE}}
A series of laboratory studies involving human and non-human subjects exercising on stationary exercise equipment (treadmills, watt bikes) show that sustained aerobic exercise at a moderate intensity ($\sim70-85\%$ of maximum heart rate)---but not low ($\sim45\%$) or high ($\sim90\%$) intensities---induces activity in the endocannabinoid (eCB) system \citep{Raichlen2013}, and similar results have been obtained in studies on the opioidergic system \citep{Boecker2008}.  Endocannabinoids appear to play an influential role in exercise-specific neurobiological reward, with studies showing activation eCB activation in moderately intense exercise and in cursorial mammals, such as humans and dogs \citep[but not non-cursorial mammals, e.g., ferrets;][]{Raichlen2012}.  In addition to direct peripheral (analgesic) and central (psychological well-being and alteration) effects, eCBs are also responsible for activating ``traditional'' neurotransmitters (opioids, dopamine, and serotonin) also responsible for rewarding and reinforcing behaviour \citep{Sparling2003}.  These findings lead Raichlen to propose eCBs as a key neurobiological substrate responsible for motivating habitual engagement in aerobic exercise, by generating ``appetitive'' and hedonic associations with exercise behaviour \citep{Raichlen2012}.







Grooming:


This ultimate evolutionary explanation for group exercise has its roots in studies of social grooming in non-human primates. Dunbar and colleagues propose a neuropharmacologically mediated affective mechanism linking dyadic grooming practices with group-size maintenance \citep{Machin2011}.  The capacity for social cohesion is thought to have arisen in primates as an adaptive response to the pressures of group living.  Aggregating in groups serves to reduce threat from predation.  At the same time, it can be individually costly due to stress arising from interactions at close proximity and conflict over resources among genetically unrelated individuals.  These pressures are hypothesised to have led to selection for social bonding (e.g., via dyadic grooming).  Resulting coalitional alliances among close partners allow for the maintenance of the group by buffering the stresses of group living.  Primate social grooming, for example, is associated with the release of endorphins, presumably leading to sustained rewarding and relaxing effects.  While other neurotransmitters such as dopamine, oxytocin, or vasopressin may also be important in facilitating social interaction, endorphins are argued to reinforce individuals (who are not related or mating) to interact with each other long enough to build ``cognitive relationships of trust and obligation'' \citep[1839]{Dunbar2012}.  It is thought that, as the homo genus evolved more complex collaborative capacities for survival in interdependent group contexts, grooming-like behaviours sustained social bonding in larger groups where dyadic grooming would cumulatively take too much time \citep{Dunbar2012}.

Experimental studies suggest that neurophysiological mechanisms activated by activities that involve physical exertion and coordinated movement, such as group laughter, dance and music-making, exercise, and group ritual can bring groups closer together, mediated by the psychological effects of endogenous opioid and endocannabinoid release \citep{Cohen2009,Fischer2014a,Fischer2014,Sullivan2014,Tarr2016,Tarr2015}.  Group exercise can in this sense be understood as a form of ``grooming at a distance.''

\subsubsection{The ``social high'' theory\label{sect:socialHigh}}

Existing evidence suggests that the link between group exercise and social cohesion could be explained in part by the way in which group exercise contexts foster psychophysiological environments conducive to forging social bonds---an important component to cohesive social groups \citep{Dunbar2010}.






\subsection{Knowledge gap in the science of sport and exercise}
It is in part due to the state of the science of sport and exercise that the project of cognitive and evolutionary anthropology of group exercise is yet to be formally defined and empirically substantiated. The anthropological science of sport and exercise---a field that should offer a rich source of theory and data---is currently marred by a bifurcation between sports psychology and social and cultural anthropology, with very little substantive theoretical predictions and empirical evidence existing in the space between.

Although social scientists of the past few centuries have occasionally included sport and group exercise as pseudo ritual-like activities responsible for producing social cohesion \citep{Durkheim1965,Mauss1935,Turner1977}, nonetheless, core biological, cognitive, and evolutionary questions surrounding sport and exercise remain largely unexplored by the behavioural sciences, including anthropology \citep{Blanchard1995,Downey2005a}.    While efforts to identify proximate cognitive and psychological mechanisms underlying the social function of religion and ritual began soon after the modern evolutionary synthesis \citep{Huxley1942} and cognitive revolution \citep[e.g.,][]{Turner1986,1987}—--the continuation of which has matured as a program of research in evolutionary anthropology and the cognitive science of religion \citep{Barrett2002,Lawson1993,Sperber1996,Whitehouse2004}---an equivalent program of research is yet to emerge around sport and exercise.

\myparagraph{Sport psychology}
Sport and exercise has been meanwhile been investigated within modes of scientific analysis concerned instrumentally with either the health benefits of exercise \citep{Fiuza-Luces2013,Morris1994}, or athletic performance \citep{Beedie2015a}, all the while neglecting to consider the cognitive or evolutionary role of exercise in human psychology or sociality \citep{Balish2013,Coulter2015}. A strict instrumental focus on athletic adherence and performance in sports psychology has restricted the psychology of exercise to an analysis of an ``athletic personality'' and the diverse motivations of the ``whole person,'' including the moral, ethical components of exercise, have been neglected \citep{Coulter2015,Laborde2014}.  As such, the cognitive and evolutionary dimensions of group exercise have escaped rigorous empirical analysis within the psychology of sport and exercise.

\myparagraph{Social and cultural anthropology of sport}
A broader spectrum of motivations and experiences in group exercise are reflected in social and cultural anthropological accounts. In these various monograph-length studies, authors focus less on the affective or sensorial experience of the athlete, and more on the personal and cultural commitments and meanings negotiated at the site of the athlete’s body in various cultural contexts.

Social anthropologists and sociologists have for some time emphasised the social function of exercise and sport in diverse cultural contexts, and various attempts have been made to analyse the phenomenological experience of exercise in terms of its sociological and psychological meaning \citep{Bourdieu1978}.  Social anthropologist Joseph Alter \textcite{Alter1993}, for example, argues that, for wrestlers in north India, the body functions as a nexus through which the symbolic and material structures of the state, family, and the individual coalesce.  In a similar vein, cultural anthropologist Susan Brownell \textcite{Bronwell1995}, in a seminal ethnography of sport in China, argues that sport functions as a crucial national symbolic practice for the Chinese nation-state in a project of ``rejoining the world,'' and that the ``micro-techniques'' (c.f. Foucault, 1977) of this project entail significant cost (and rich meaning) to the individual athlete.   Similarly, French sociologist Loic Wacquant \textcite{Wacquant2004}, in an ethnography of boxers in Chicago’s south side, describes a ``social logic'' of physical activity, claiming that ``the daily dedication and high technique that training demands; the regimented diet; the control, mutual respect, and tacit understandings necessary for actual fist-to-fist competition serve to create for the boxer an island of order and virtue'' \textcite[17]{Wacquant2004}.  In a recent extension of this line of ethnographic work, researchers have attempted to theorise the cognitive implications for different social-cultural frames of belief and understanding surrounding skill-acquisition and performance in sport and exercise \citep{Downey2005bDowney2007,Marchand2010}.

These samples of the social and cultural anthropology of sport represent an attempt to interpret an ethnographically observed connection between adherence to group exercise and the social and psychological meaning that appears to be tethered to this adherence.  While compelling and rich in ethnographic detail, these accounts do not explicitly engage in the project that is the central focus of this dissertation—--i.e., an \textit{explanatory} account of group exercise in human sociality.

In sum, existing scientific explanations of sport and exercise are split along two branches---one in which focus on causal physiological, cognitive, and social mechanisms of exercise is driven by priorities of athletic performance and public health outcomes, and another in which the study of group exercise forms part of a broader discipline in which social, ethical, and moral hermeneutics are more dominant than cognitive or evolutionary theory.  This bifurcation has created a scenario in which the evidence available for the cognitive and evolutionary anthropology of group exercise has been been derived either very close to the treadmill---from laboratory paradigms in which components and effects of group exercise (e.g., exact behavioural synchrony and direct or assay measures of neuropharmacological activity) are hyper-essentialised through experimental manipulation and operationalisation.  Alternatively, evidence has been derived very far from the treadmill---by ethnographic researchers whose priorities do not include identifying or testing causal processes. Cognitive and evolutionary approaches to the phenomenon of group exercise suffer from this bifurcation in anthropology, and would benefit from attempts to address the space between the treadmill and the field.












\subsection{Existing anthropology of group exercise}
A broader spectrum of motivations and experiences in group exercise are reflected in social and cultural anthropological accounts.  In these various monograph-length studies, authors focus less on the affective or sensorial experience of the athlete, and more on the personal and cultural commitments and meanings negotiated at the site of the athlete’s body in various cultural contexts.

Social anthropologists and sociologists have for some time emphasised the social function of exercise and sport in diverse cultural contexts, and various attempts have been made to analyse the phenomenological experience of exercise in terms of its sociological and psychological meaning \citep{Bourdieu1978}.  Social anthropologist Joseph Alter \textcite{Alter1993}, for example, argues that, for wrestlers in north India, the body functions as a nexus through which the symbolic and material structures of the state, family, and the individual coalesce.  In a similar vein, cultural anthropologist Susan Brownell \textcite{Bronwell1995}, in a seminal ethnography of sport in China, argues that sport functions as a crucial national symbolic practice for the Chinese nation-state in a project of ``rejoining the world,'' and that the ``micro-techniques'' (c.f. Foucault, 1975) of this project entail significant cost (and rich meaning) to the individual athlete.   Similarly, French sociologist Loic Wacquant \textcite{Wacquant2004}, in an ethnography of boxers in Chicago’s south side, describes a ``social logic'' of physical activity, claiming that ``the daily dedication and high technique that training demands; the regimented diet; the control, mutual respect, and tacit understandings necessary for actual fist-to-fist competition serve to create for the boxer an island of order and virtue'' \textcite[17]{Wacquant2004}.  In a recent extension of this line of ethnographic work, researchers have attempted to theorise the cognitive implications for different social-cultural frames of belief and understanding surrounding skill-acquisition and performance in sport and exercise \citep{Downey2005bDowney2007,Marchand2010}.

These samples of the social and cultural anthropology of sport represent an attempt to interpret an ethnographically observed connection between adherence to group exercise and the social and psychological meaning that appears to be tethered to this adherence.  While compelling and rich in ethnographic detail, these accounts do not explicitly engage in the project that is the central focus of this dissertation—--i.e., an \textit{explanatory} account of group exercise in human sociality.


\subsection{Bifurcation in the science of sport and exercise}

Noticeable gaps in cognitive and evolutionary accounts of group exercise can be in part explained by the specific history of the study of sport.  Although social scientists of the past few centuries have occasionally included sport and group exercise as pseudo ritual-like activities responsible for producing social cohesion \citep{Durkheim1965,Mauss1935,Turner1977}, nonetheless, core biological, cognitive, and evolutionary questions surrounding sport and exercise remain largely unexplored by the behavioural sciences, including anthropology \citep{Blanchard1995,Downey2005a}.

While efforts to identify proximate cognitive and psychological mechanisms underlying the social function of religion and ritual began soon after the modern evolutionary synthesis \citep{Huxley1942} and cognitive revolution \citep[e.g.,][]{Turner1986,1987}—---the continuation of which has matured as a program of research in evolutionary anthropology and the cognitive science of religion \citep{Barrett2002,Lawson1993,Sperber1996,Whitehouse2004}---an equivalent program of research is yet to emerge around sport and exercise \citep{Blanchard1995,Downey2005a}.  Sport and exercise has meanwhile been investigated within modes of scientific analysis concerned instrumentally with either the health benefits of exercise \citep{Fiuza-Luces2013,Morris1994}, or athletic performance \citep{Beedie2015a}, all the while neglecting to consider the cognitive or evolutionary role of exercise in human psychology or sociality \citep{Balish2013,Coulter2015}. A strict instrumental focus on athletic adherence and performance in sports psychology has restricted the psychology of exercise to an analysis of an ``athletic personality'' and the diverse motivations of the ``whole person,'' including the moral, ethical components of exercise, have been neglected \citep{Coulter2015,Laborde2014}.  As such, the cognitive and evolutionary dimensions of group exercise have escaped rigorous empirical analysis.

In sum, a review of existing literature in the anthropology of group exercise exposes a bifurcation of knowledge along two branches---one in which focus on causal physiological, cognitive, and social mechanisms of exercise is driven by priorities of athletic performance and public health outcomes, and another in which the study of group exercise forms part of a broader discipline in which social, ethical, and moral hermeneutics is more dominant than cognitive or evolutionary theory.  This bifurcation has created a scenario in which the evidence available for the cognitive and evolutionary anthropology of group exercise has been been derived either very close to the treadmill---from laboratory paradigms in which components and effects of group exercise (e.g., exact behavioural synchrony and direct or assay measures of neuropharmacological activity) are hyper-essentialised through experimental manipulation and operationalisation---or, alternatively, very far from the treadmill---by ethnographic researchers whose priorities do not include identifying or testing causal processes.  The existing social high theory of group exercise and social bonding suffers from this bifurcation in anthropology, and would benefit from attempts to address the space between the treadmill and the field.  In this dissertation, I actively address the knowledge gaps in the social high theory of group exercise and social bonding, using an inclusive human science integrating psychology, cognitive and neuroscience, and anthropology \citep{Whitehouse2012,Downey2014}.




\subsubsection{
  Anthropology of extreme ritual theorises cost and meaning
  \label{sect:extremeRitual}
              }
Evidence from the adjacent field of the cognitive and evolutionary anthropology of extreme ritual suggests a relationship between extreme physiological exertion and social bonding, pro-sociality, and cooperation \citep{Whitehouse2004,Xygalatas2013,Fischer2014a}. Researchers hypothesise that extreme ritual activates evolved neurocognitive processes of pain and reward \citep{Bastian2014,vanBunderen2014}, but it is understood that these mechanisms may not necessarily translate directly to feelings of positive affect, euphoria, or collective  effervescence (as the social high theory would suggest).  It has been suggested instead that the link between psychophysiological mechanisms of extreme ritual and social bonding may be mediated by cultural processes of group membership \citep{Whitehouse2004,Whitehouse2014a}.  For example, accumulating evidence suggests that extreme ritual practices increase levels of physiological arousal and personal agency, allowing for a process whereby individuals internalise group-level expectations as part of their own self-understandings to the extreme end of ``identity fusion''---a \textit{visceral} feeling of oneness with the group  \citep[hereafter simply ``fusion''][]{Swann2010,Swann2015}.

Whitehouse and colleagues propose that ritual practices responsible for generating higher levels of fusion have endured processes of cultural group selection \citep[or technically ``cultural attraction,'' cf.][]{Claidiere2007}, owing to the assemblage of adaptive benefits which they bestow to individuals and groups, among other contextual and cognitive factors of attraction, including memorability and transmission frequency in populations \citep{Atkinson2011}.  On balance, empirical data suggest that experiences of \textit{dysphoric} (more so than euphoric) shared experiences (of pain, torture, or warfare) strongly predict fusion (and in turn willingness to perform pro-group self-sacrifice,][]{Whitehouse2014,Whitehouse2014a}. These findings have been recently formalised in a mathematical model showing how conditioning cooperation on previous shared dysphoric experience can allow individually costly pro-group behaviour to evolve  \citep{Whitehouse2017}.

The anthropology of extreme ritual offers a pertinent example of how the social high theory could be developed to incorporate missing theoretical links, such as the link identified between extreme physiological cost and profound (social) meaning.  The relationship between extreme ritual practices and fusion appears contingent on behavioural affordances that are additional to those immediate physiological and cognitive mechanisms activated by the activity itself.  In this case, psycho-social processes of self construal \citep{Markus2003}, self-defining memory \citep[cf.][]{Moffitt1994}, and group membership \citep{Tajfel1971,Turner1987,Swann2009} all appear to play a supervening role in enabling social bonding, and---at an evolutionary scale---social cohesion.




\subsubsection{Group exercise resembles ``grooming at a distance''}
The ultimate evolutionary story of the social high theory has its roots in studies of social grooming in non-human primates. Dunbar and colleagues propose a neuropharmacologically mediated affective mechanism linking dyadic grooming practices with group-size maintenance \citep{Machin2011}.  The capacity for social cohesion is thought to have arisen in primates as an adaptive response to the pressures of group living.  Aggregating in groups serves to reduce threat from predation.  At the same time, it can be individually costly due to stress arising from interactions at close proximity and conflict over resources among genetically unrelated individuals.  These pressures are hypothesised to have led to selection for social bonding (e.g., via dyadic grooming).  Resulting coalitional alliances among close partners allow for the maintenance of the group by buffering the stresses of group living.  Primate social grooming, for example, is associated with the release of endorphins, presumably leading to sustained rewarding and relaxing effects.  While other neurotransmitters such as dopamine, oxytocin, or vasopressin may also be important in facilitating social interaction, endorphins are argued to reinforce individuals (who are not related or mating) to interact with each other long enough to build ``cognitive relationships of trust and obligation'' \citep[1839]{Dunbar2012}.  It is thought that, as the homo genus evolved more complex collaborative capacities for survival in interdependent group contexts, grooming-like behaviours sustained social bonding in larger groups where dyadic grooming would cumulatively take too much time \citep{Dunbar2012}.

Experimental studies suggest that neurophysiological mechanisms activated by activities that involve physical exertion and coordinated movement, such as group laughter, dance and music-making, exercise, and group ritual can bring groups closer together, mediated by the psychological effects of endogenous opioid and endocannabinoid release \citep{Cohen2009,Fischer2014a,Fischer2014,Sullivan2014,Tarr2016,Tarr2015}.  Group exercise can in this sense be understood as a form of ``grooming at a distance.''




\subsubsection{The infancy of science of sport and exercise}
The anthropological science of sport and exercise---a field that should offer a rich source of theory and data---is currently marred by a bifurcation between sports psychology and social and cultural anthropology, with very little substantive theoretical predictions and empirical evidence existing in the space between.

Although social scientists of the past few centuries have occasionally included sport and group exercise as pseudo ritual-like activities responsible for producing social cohesion \citep{Durkheim1965,Mauss1935,Turner1977}, nonetheless, core biological, cognitive, and evolutionary questions surrounding sport and exercise remain largely unexplored by the behavioural sciences, including anthropology \citep{Blanchard1995,Downey2005a}.    While efforts to identify proximate cognitive and psychological mechanisms underlying the social function of religion and ritual began soon after the modern evolutionary synthesis \citep{Huxley1942} and cognitive revolution \citep[e.g.,][]{Turner1986}—--the continuation of which has matured as a program of research in evolutionary anthropology and the cognitive science of religion \citep{Barrett2002,Lawson1993,Sperber1996,Whitehouse2004}---an equivalent program of research is yet to emerge around sport and exercise.  The samples of social and cultural anthropology of sport mentioned above (Section ~\ref{sect:linkCostMeaning}), while compelling and rich in ethnographic detail, do not explicitly engage in the project that is the central focus of this dissertation—--i.e., an \textit{explanatory} account of group exercise in human sociality.

Sport and exercise has been meanwhile been investigated within modes of scientific analysis concerned instrumentally with either the health benefits of exercise \citep{Fiuza-Luces2013,Morris1994}, or athletic performance \citep{Beedie2015}, all the while neglecting to consider the cognitive or evolutionary role of exercise in human psychology or sociality \citep{Balish2013,Coulter2015}. A strict instrumental focus on athletic adherence and performance in sports psychology has restricted the psychology of exercise to an analysis of an ``athletic personality'' and the diverse motivations of the ``whole person,'' including the moral, ethical components of exercise, have been neglected \citep{Coulter2015,Laborde2014}.  As such, the cognitive and evolutionary dimensions of group exercise have escaped rigorous empirical analysis within the psychology of sport and exercise.

In sum, existing scientific explanations of sport and exercise are split along two branches---one in which focus on causal physiological, cognitive, and social mechanisms of exercise is driven by priorities of athletic performance and public health outcomes, and another in which the study of group exercise forms part of a broader discipline in which social, ethical, and moral hermeneutics are more dominant than cognitive or evolutionary theory.  This bifurcation has created a scenario in which the evidence available for the cognitive and evolutionary anthropology of group exercise has been been derived either very close to the (figurative) treadmill---from laboratory paradigms in which components and effects of group exercise (e.g., exact behavioural synchrony and direct or assay measures of neuropharmacological activity) are hyper-essentialised experimental operationalisation.  Alternatively, evidence has been derived very far from the treadmill---by ethnographic researchers whose priorities do not include identifying or testing causal cognitive and evolutionary mechanisms.  Cognitive and evolutionary approaches to the phenomenon of group exercise suffer from this bifurcation in anthropology, and would benefit from attempts to address the space between the treadmill and the field.
