\chapter*{Chapter 9: Abstract}
%\addcontentsline{toc}{chapter}{\nameref{ch:study1intro}}


%\markboth{{Introduction to Part I}}{Introduction to Part I\label{ch:part1intro}

This study employed a controlled experimental environment to further test the hypothesised relationship between joint action, team click, and social bonding. An \textit{in situ} survey study (Chapter~\ref{chap:tournamentSurvey}) generally confirmed the core prediction that athletes who perceived higher quality team performance in joint action would also experience higher levels of team click and social bonding. The causal mechanisms and system dynamics underlying this relationship remain unknown.  The previous study showed results that suggest that positive violation of expectations around performance could be significantly associated with team click.

In the present was designed to test the role of expectation violation as a driving mechanism for the experience of team click and social bonding in group exercise. A final sample of 58 professional Chinese rugby players ($Men = 31$) participated in a between-subjects design in which athletes' expectations for joint action training drill \citep[``Invasion drill''][]{Passos2011}) were manipulated in one of two conditions.  Athletes in the ``low difficulty'' condition were primed with information to suggest that the training drill would require minimal joint action competence (i.e., 2/10 difficulty rating).  In the ``high difficulty'' condition, athletes were primed to expect the training drill to be relatively difficult with respect to joint action competence (8/10 difficulty rating).  Pre- and Post-Drill surveys measured athletes' perceptions of main variables of interest (individual and group performance, team click, and social bonding to the training group (and to the team more generally), as well as athlete personality type and states of athlete states of arousal, fatigue, and injury.  An independent observer coded video footage of each experiment session to derive an objective measure of performance in joint action according to the parameters of the training drill.  It was predicted that athletes in the ``high-difficulty'' condition would experience higher levels of team click and social bonding due to more positive perceptions of group performance relative to prior expectations.  Athletes in the ``low difficulty'' condition would on average experience less strongly the click of joint action and social bonding because they would not experience the same level of positive violations of expectations around group performance.

Results generally confirmed these predictions.  Although there was no explicit evidence to support the effectiveness of the experimental manipulation, results did reveal significant interaction effects between perceptions of group performance relative to prior expectations and experiment condition on team click, such that the relationship between perceptions of performance and team click was significant in the high difficulty condition.  In addition, results revealed a significant interaction effect between team click and social bonding, also moderated by condition. Finally, in two separate analyses of the collected data, a moderated mediation effect was shown, whereby team click mediated the relationship between expectation violation and social bonding, only in the ``high difficulty'' condition---designed to produce highest expectation violation around performance in joint action.
