\chapter{\label{5ethnographicResults}Ethnographic Results 2}

\minitoc



\section{Abstract}
In this chapter I analyse the predictions of this dissertation in light of ethnographic evidence collected with the Beijing Men's rugby team.  I observe evidence in which the contours of generalisable cognitive mechanisms relevant to a relationship between joint action and social cohesion are visible.  I identify three core areas of analysis.  First, athletes pay close attention to their performance of joint action requirements specific to rugby union.  In particular, I notice that violation of athletes' expectations around joint action, which can be either positively or negatively valenced, is a source of strong emotional response with implications for processes of group formation.  Second, ethnographic observations reveal that the experience of ``team click'' is a prominent experience in joint action associated with rugby, and appears to be connected in athletes' reasoning to concepts of team cohesion.  Third, I identify evidence of social bonding in processes of group formation.  In addition to these three core areas of ethnographic evidence, I also flag the possibility that these three processes are moderated by within-group variation in technical competence, personality type, injury, and athlete experience of fatigue and exertion in exercise.  I conclude this chapter by considering how these ethnographic data could be combined with existing theory and evidence and operationalised in further empirical studies beyond this specific ethnographic setting.


                                      \begin{CJK}{UTF8}{gbsn}


\section{Vignette}

Story about Joint Action, Social Bonding, Team Click?



\section{Introduction}








\section{Results}
As explained in the previous section, most of the athletes in the program at the Institute transitioned from other sports, or otherwise had no specific background in sport before coming to the Institute.  Rugby's minnow status in China means that its technical and social requirements are not widely known or understood.  Thus, from a very low base of prior knowledge, athletes are forced to develop technical competence in rugby and social fluency in the social environment of the team.  The ability to effectively coordinate behaviour with others, both on and off the field, is therefore crucial to an athlete's ability to navigate the rugby program at the Institute and access the incentives available.


\subsubsection{Perceptions of performance in joint action}

During interviews with athletes, I asked them a number of questions about the costs and benefits of adherence to rugby.  I also asked them what they found to be the most difficult part of rugby.  Rugby's parts are diverse, and range from on-field technical requirements related to individual and team performance.  Individuals must learn skills of passing, tackling, running, kicking, and these individual components must be harmonised in phases of joint action with other athletes on the same team, as well as against opposition athletes.  Off the field, as discussed in the previous chapter, the rugby program at the Institute requires athletes to navigate a complex social environment, structured by hierarchical relations of power.

    \myparagraph{Performance related anxiety (predominantly Junior)}

20 of the 26 athletes interviewed responded directly to the question of rugby's most difficult part.  11 athletes cited an aspect of rugby that related to on-field coordination of behaviour in joint action, such as decision making in attack, ``awareness'' or vision in game play, coordinating running lines, executing planned plays, and so on.  Seven athletes cited aspects of individual performance, such as tackling, speed, footwork in attack; and two athletes cited both an aspect that was related to coordination of behaviours with others, and an aspect of individual performance.  The way in which the question was framed led the athlete to a consideration of the on-field requirements of rugby itself, rather than the social and political dynamics of group membership, which may explain why no athletes chose to talk respond with reference to rugby's off-field requirements.



            \subparagraph{Team Awareness}

            Many newcomers are transitioning from individual sports over to team sports...as such there is an unfamiliarity with not only the specific joint action demands of rugby, but also the norms of group membership.. an ``awareness'' of the norms and expectations of group membership at the rugby program at XNT

          P1: %Anxiety around complexity of rugby, and the ``awareness'' required to execute team based joint-action:
          Number of athletes quoted team awareness as the hardest thing?

          P2: %Anxiety included reference to specific components of performance
          %TEAM: attack, defence, support play, communication
          Some examples from interviews?


          \myparagraph{``I have become a social animal''}

          Of the group of 26 athletes who I included in the final analysis, 17 had transferred from other sports: 15 from athletics, one from football and one from basketball.  The remaining nine athletes had no specific background in sport.  When I asked athletes what rugby in particular had given them, an overwhelming response was ``team awareness'' (团队意识).  For the majority of athletes who came from the individual sport of athletics (and the athletes with no prior team sport background), the rugby team was a distinctly novel experience, involving a suite of new social norms and requirements associated with group membership.

          Yang Can:
          \begin{quotation}
            Because rugby is a team sport, you know. We’re all brothers, it's the same as fighting, we’re born brothers and we’ll die as brothers you know, like a family, all together, thinking about how to share the responsibility. Its really good...I had never experienced this feeling before.  At that time at primary school there weren’t any rules, you’d just muck around or at the end of school just leave straight away, there wasn’t this type of feeling there.
          \end{quotation}


          \begin{quotation}
            因为橄榄球就是团体项目嘛,都是兄弟,像打架一样,出生入死得兄弟嘛,就跟一家人一样,互相,多替别人考虑分担,挺好的...没有过这种感觉,那时候小学没有什么规矩的,直接玩或者放学就大直接走了走了,没这种感觉
          \end{quotation}


          Lian Jianxiang:
          \begin{quotation}
             I first I wasn’t used to it, and then later as began to interact with my teammates (brothers-shixiong) I discovered that it was more or less the same as my previous (athletics) team, but just more cohesive (tuanjie).  But there was still a difference, for example in an individual sport you need to manage yourself and that's all, whereas team sports you need to consider more, you need to consider a lot of things that relate to everyone…Now I am used to it. I like this feeling (of team sport membership) more, its so much better than individual sport, in an individual sport its just yourself, its to independent, this (rugby) is a collective  (a family) , isn’t it
          \end{quotation}


          \begin{quotation}
            转到团体项目有什么感受,学到什么新的东西?
           :之前不适应,后来和师兄一接触发现和我之前的队也差不多,要团结。但还是有差别的比如说之前个人项目管好自己就行了,团体项目考虑的比较多,要考虑好多大家一起的东西
           ?练了有一年的时间,现在习惯了吗?习惯这个感觉吗?
           :习惯了。更喜欢这个感觉,比个人好太多了,个人只不过自己,个人项目太独立了,这个是个大家么
          \end{quotation}



          Unruly undergrad Fang Chao explains how he was transformed by rugby and the team:

          \begin{quotation}
            I think before when I was doing athletics, compared to now, I think I am a completely different person.  When I first came into contact with rugby, before I was doing an individual sport, now this is a team sport. I've changed a lot in terms of my personality, before when doing athletics I thought I was very independent and self-reliant, my own person.  Now I am one person who needs to communicate a lot with other people, cooperate; I have become a social animal.
          \end{quotation}


            \begin{quotation}
              我觉得之前练田径,跟现在橄榄球比,觉得我完全不是同一个人了。橄榄球刚开始接触的时候,之前田径是个人项目,现在是团队项目。性格方便改变很多,之前练田径是觉得我是个我行我素,自己一个人,现在我一个人还需要和别人多沟通,合作,变成群恤动物
            \end{quotation}



          \begin{quotation}
            ...I think it's mainly this thing of having teammates. Before, when I was training for an individual sport, it was just me training by myself. [In that environment] it was a case of whoever trained well was successful.  But now with this team of brothers, elder teammates will take care of younger teammates. We all train together, and if you can’t do something, you can always ask your elder teammates...
          \end{quotation}


            \begin{quotation}
              :我觉得主要是师哥师弟的这一块儿,原来练个体项目都是自己练自己的,谁练好了谁厉害,但是现在师哥师弟,有师哥照顾师弟带着,互相练,我不会我可以问师哥
            \end{quotation}









          \subparagraph{Individual Social Shame (negative violations)}
                %Anxiety around individual letting the team down due to individual mistakes

                %Anxiety included reference to specific components of performance
                  %IND: contact, tackle, passing, decision making, and support play in attack






    \myparagraph{Strategy in individual performance and deflection of responsibility (more predominantly Senior athletes)}

            \subparagraph{Team Awareness: Agency over, deflection of responsibility}
         %Deflection of responsibility towards junior athletes: criticism of junior athletes for not committing, playing computer games, being complacent.  (Irony that LP was one of the most vocal when discussing computer games at the dinner table).
         %Others more generous and circumspect: WW and CSC realise that it's a progression, and that individuals
         %EXPLAIN: reduction of dissonance via deflection

          \subparagraph{Individual performance: strategy}
         %MHT, CSC, HXL, WWX, MC use of experience to strategically avoid over-exertion and injury risk.







  \myparagraph{Survey Results}

        SURVEY RESULTS: relatively equal levels of flow overall, but higher performance related anxiety in junior athletes, particularly in the scratch matches, which required high levels of technical competence and joint action coordination.

        %EXPLAIN: perception of performance in relation to social expectations of the team: joint action participation, as well as individual responsibilities (team member, self-determination)

        %Senior athletes talk with more composure regarding performance, any deflect responsibility for being the agent of team performance, and show strategy regarding how to regulate energy expenditure.





  \myparagraph{Potential Mechanism: Positive Violation of expectations Performance related exhilaration and generalised emotionality (predominantly Junior)}

                %SWH story: the buzz and glow from acquisition of skill

            \subparagraph{Ind components}
                %WZF: explanation of the first side-step
                %YC: Fending - the feeling of picking up
          \subparagraph{Team performance}
                %WZF: likens team performance to side-step
                %HXL: rugby wasn't interesting (wasn't motivated until he started to pick up the team dimensions)

          \subparagraph{Individual Social Shame (negative violations)}







        \subsubsection{Team Click}


  Junior Athletes:

  Either:
          \myparagraph{Generalised emotionality, equating click with bonding (Junior)}

  unspoken understanding linked with cohesion (tuanjie)

  Or:
          \myparagraph{inability to conceive of click, not qualified to talk about it}


  Either:
          \myparagraph{Familiarity and granularity/precision (Senior)}
          %HXL: Aura/atmosphere of the teams
          %WW: flow description
          %WZF: consideration of all the individuals
          %CSC:
          \myparagraph{Problematised}
            %Juniors for not being committed (Undergrad loafers)
            %Coach/Leadership for not supporting (LP)
            %China:
              %the system - No initiative, Chinese society too complex
              %the culture - Confucian education

              %Nostalgia for old regime pre 2013: HXL, LP,
            %E: a cognitive strategy in reorganising information in a way that maintains coherence (Nowak 2017)



    \subsubsection{Social Bonding}


      \myparagraph{Emotional Support}
      %MC, LZS, SHW, GJP



      \myparagraph{Shared Goal}
      %WW:



      \myparagraph{Identity Fusion}
      %Fun/interesting/compelling (juniors)
      %Attachment: MXK, LZS,
      %Fusion: HXL, LP,




    \subsubsection{Moderator Variables}
        \myparagraph{Technical Competence}


        \myparagraph{Personality}


        \myparagraph{Injury}


        \myparagraph{Fatigue}


                                                          \end{CJK}
