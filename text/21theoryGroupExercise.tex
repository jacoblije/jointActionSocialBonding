
\begin{savequote}[8cm]

  \qauthor{}
\end{savequote}



\chapter{A novel theory of social bonding through joint action
\label{chap:theoryGE}}

\minitoc



  \begin{CJK}{UTF8}{gbsn}


\section{A novel theory of social bonding through joint action\label{sect:novelTheory}}

As explained in the previous chapter, The AIF begins with the assumption that, on a basic level, joint action consists of two or more Bayesian inferential systems attempting to minimise free energy by modelling the hidden causes of sensory stimuli.  From sheer computational perspective, joint action is an inherently challenging task, but it appears that ``general synchronisation'' derived from the coupling of two or more agents facilitates a shared narrative that sets the foundation for functionally equivalent generative models for joint action.  This perceptual common ground resembles a situation in which ``adjustments for ``Am I?'' and ``Are You?'' and ``Where Are We?'' and stuff like that...'' (see Section~\ref{sect:teamClickIntro}) dissolve, and a transcendent ``we-mode'' of collective agency is glimpsed \citep[][]{Friston2015}.  Anecdote and observation of real-world joint action scenarios suggests that achieving the ``click'' of joint action---in spite of its uncertainty, and however momentarily it lasts---can be a deeply rewarding experience.  In the case of joint action in group exercise contexts, it appears that various features of group exercise serve to amplify both the challenges of joint action, as well as its psycho-social rewards.

The explanatory power of the AIF for team click and social connection in joint action lies in its capacity to incorporate various cognitive strategies within one overarching mandate of free energy minimisation \citep{Clark2015}.  The distinction between cortical and extra-cortical, mental and emotional, model-based and model-free cognitive processes dissolves, and in place of these dualisms, a continuum of cognitive strategies are flexibly deployed to exploit brain, body, and bio-external resources \citep{Pezzulo2013}.  Bayesian optimisation facilitates flexibility: the volume on sensory prediction error signals is either ``dialled-up'' or ``dialled-down,'' depending on their reliability judged by prior experience (inverse of variance of the prior distribution).  Thus, just as the binary distinction between mental and emotional collapses, so too does the distinction between external and internal, exteroceptive and interoceptive \citep[and proprioceptive][]{Seth2013}.

This conception of joint action allows for a full consideration of the interlocking physical, cognitive, and social processes outlined in in Chapter~\ref{chap:intro}.  Anecdotal and observational evidence suggests that team click is at once mental and embodied, cognitive as well as emotional, metaphysical (i.e., mysterious) as well as deeply grounded in visceral sensations (see Chapter~\ref{chap:intro} Section ~\ref{sect:adrian}).  The active and embodied approach of the AIF allows for an account of the surprising, visceral, and agentic dimensions of team click; while the hypothesised coupling between generative models and cultural affordances explains how visceral experience of team click can set the foundation for higher-order processes of emotional affiliation, perceptions of common goal, and social identity formation with co-participants.

In the next section, I outline the core relationships of a novel theory of social bonding through joint action.  The theory predicts three causal paths: 1) joint action to team click, 2) team click to social bonding, and 3) a direct path from joint action to social bonding, 4) mediated by team click.

I then outline the application of AIF to group exercise in particular, explaining how group exercise environments appear almost deliberately rigged to produce extreme levels of cognitive uncertainty, while at the same containing social and cultural affordances capable of attenuating such uncertainty (Section ~\ref{}).  Finally, I present a novel theory of social bonding through joint action, and outline general predictions that emerge from this theory.


%The first relationship is between joint action and team click. Team click in group exercise resembles a scenario in which individuals perceive success in joint action.  The tacit but powerful experience of surprise, agency, and certainty (about the reliability of co-actors) deriving from team click sets the foundation for higher-order cognitive processes by virtue of their connection with these processes via an intricate  helix of dynamically coupled generative models and cultural affordances.  Thus, the predicted relationship between team click and social bonding, if proven true, will have implications for ultimate processes, such as social cohesion and cultural transmission.





\subsection{More positive perceptions of success in joint action will predict higher levels of team click\label{sect:JASuccessTeamClick}}

    The components of team click outlined above indicate that this commonly observed phenomenon contains elements of positive expectation violation or surprise, deriving from an experience of tacit or implicit coordination in joint action.  The phenomenology of team click can also involve the blurring of boundaries between self and others in the team, the perception of atmosphere or aura surrounding the group, as well as a perception of reliability of teammates and the self to successfully perform joint action.




\subsection{Higher levels of team click predict higher levels of social bonding \label{sect:teamClickSocialBonding}}

    The combination of positively valenced surprise deriving from dialled-up interoceptive (``emotional'') and dialled-down exteroceptive inputs, a blurring of agency between self and other due to the strain on proprioceptive inputs during ``in the moment'' movement execution appear to provide powerful ingredients for flow on psycho-social effects.  Team click appears to be associated with both an atmosphere around the team and reliability of teammates, both of which can be understood as social cognitions.  Is team click responsible for generating feelings of emotional closeness, a sense of shared goal, and shared identity with teammates?  In other words, is it possible to identify a link between team click and social bonding?

    The link between interpersonal coordination and social bonding has been addressed in the behavioural mimicry and synchrony literatures \citep[e.g.,][]{Wheatley2012,Launay2016,Mogan2017}, but there is less substantive evidence in relation to dynamic interpersonal coordination in natural joint action settings such as those found in group exercise contexts \citep{Marsh2009,Miles2009,Lumsden2012}.  More recently, however, analysis of dynamic coupling of co-actors in joint action scenarios reveals that synchronised movement implicates an array of implicit and pre-perceptual cognitive processes of alignment and prediction error minimisation \citep{Schmidt2011}, which, in addition to more explicit forms of communication, could be central to the generation of feelings of self-other merging, self-other distinction, and perceived reliability and trust associated with social bonding \citep{Marsh2009}.


%\subsection{Team click will mediate a direct relationship between joint action and social bonding in group exercise\label{sect:JASuccessSocialBonding}}

\subsection{Cultural affordances}

Evidence suggests that informational affordances provided by the specific cultural milieu can also serve to shape patterns of behaviour relevant to joint action.  The term ``culture'' can be understood as shared elements that provide standards for perceiving, believing, evaluating, communicating, and acting among those who share a language, a historical period, and a geographical location \citep{Triandis1996}.  As cultural psychologists Kitayama and Markus (2020, p. 422) explain, culture is a:

    \begin{quote}
      ...stand-in for a similarly untidy and expansive set of material and symbolic concepts...that give form and direction to behaviour [and that] culture is located in the world, in patterns of ideas, practices, institutions, products, and artefacts.
    \end{quote}

    A key insight overlooked by the existing social high account of group exercise and social cohesion, but revealed by the paradigm shift surrounding active inference, is the sensitivity of joint action (or any cognitive process for that matter) to informational affordances provided by various layers of ecological and cultural context.  Affordances for joint action appear to be dictated by processes operating at multiple conceptual levels---from the micro-level predictive processes associated with movement action and perception, to the macro-level predictive frames offered by specific cultural and contextual niches---interact in complex processes of reciprocal causation to shape joint action.  Conceptualisation of the causal complexity of cognitive processes relevant to joint action in this way echoes a broader reconceptualisation of the causal complexity associated with change on an evolutionary timescale, which recognises that human behavioural phenomena is the result of a number of biological, cognitive, and ecological mechanisms that interact via reciprocal feedback loops spanning varying scales of time and space \citep{Fuentes2015}.

    In the context of joint action, conventional affordances can be understood as shared frames of reference---``hyper-priors'' that set the macro-contextual coordinates for joint action \citep{Clark2013}. Joint actions that involve complex sequences and divisions of labour between participants appear to rely heavily on capacities to explicitly signal intention for the assigning of roles, forward planning, and repair of failed coordination \citep{Frith2010}.  These ``coordination smoothers'' \citep{Vesper2017} often function to reduce spatial and temporal variation in action by providing a shared spatiotemporal referent for co-alignment of predictions.  Cultural conventions are thus examples of effective framing devices for joint action.  Depending on the context of the joint action, it could be subject to a pre-existing, mutually recognised power relations typical in the established culture (e.g., favouring hierarchical or egalitarian communication, \citep[see]{Cheon2011}) and the particular situational context (e.g., formal or informal).  Establishing roles, such as leader or follower, also has a similar smoothing effect, and often the affordances in the task environment shape the smoothing strategies available to co-actors \citep{Marsh2009}.
    or the ``common narrative'' that stabilises alignment of predictions \citep{Friston2015}.


    \subsection{Individual differences\label{sect:individualDifferences}}

    There is evidence to suggest that individual differences may play a role in joint action processes.  Research suggests that preexisting dispositional tendencies in sociality dimensions of personality (e.g. extroversion, agreeableness) and social orientation (locus of control, communication styles), as well as the nature of pre-existing interpersonal relationships, and technical competence in joint action \citep{Novembre2014}, will impact on the structure and quality of interpersonal movement coordination.  Individual differences may also influence the social effects of joint action \citep{Marsh2009}.

    The personality dimension of empathy—--understanding others’ thoughts and feelings—--has been linked to anticipatory mechanisms related to action simulation \citep{Sevdalis2014,Keller2014}.  In the piano studies mentioned above \citep{Novembre2012}, scores on the ``perspective-taking'' sub-scale of an empathy questionnaire correlated positively with neurophysiological measures of representing the other’s part in their own motor system, as well as how much this ``other-representation'' was relied upon for coordination \citep{Novembre2014a}.  In a synchronised finger-tapping task, Pecenka and Keller (2011) found that scores on a perspective-taking questionnaire correlated with the degree that individuals predicted event micro-timing in a tempo-changing pacing sequence.  \textcite{Richardson2007} found in a dyadic plank moving experiment, individuals’ levels of agreeableness and extroversion were positively correlated with persistence of cooperation in the task.

    In addition to personality type, social orientation and motivation have also been shown to effect interpersonal coordination.  A study of unintentional coordination revealed that prosocial-oriented individuals spontaneously synchronised arm movements with others more than pro-self-oriented individuals, whether their social/self-orientation reflected their pre-existing disposition or resulted from an experimental manipulation \citep{Lumsden2012}.  Studies have found that interacting with a late-arriving partner reduced stepping synchronisation, compared with interacting with a partner who arrived on time \citep{Miles2010}, and bodily synchrony decreased during arguments compared with affiliative conversations \citep{Paxton2013}.  A recent study addressed the relationship between locus of control (i.e. the degree to which life events are perceived to result from one’s own actions) and temporal adaptation (error correction) \citep{Fairhurst2014}.   Results indicated that individuals with an internal locus of control (who attribute the cause of events to their own actions) engaged in less phase correction than individuals with an external locus of control (who attribute events to external factors), which may reflect individual variation in predispositions towards different movement coordination strategies.

    This assertion is corroborated by a study conducted by \textcite{Schmidt1994}, which used interpersonal wrist-pendulum coordination to investigate the effects of self-reported social competence \citep[c.f.][]{Riggio1996} upon social coordination stability.  Subjects were selected to create homogeneous social competence dyads (High–High or Low–Low pairs) and heterogeneous dyads (High–Low pairs). The heterogeneous (High–Low) pairs demonstrated significantly greater stability and fewer breakdowns in coordination than the homogeneous (High–High and Low–Low) social competence pairings, suggesting that reciprocity (leader-follower) rather than symmetry (leader-leader or follower-follower) of social competence facilitates social coordination.  In this sense,  ``internal'' individuals may stabilise the tempo of their own performance (at the expense of synchrony) and take a leader role, whereas ``external'' individuals may synchronise with their partner (at the expense of maintaining a steady tempo) and take a follower role.  In sum,  dispositional tendencies in movement coordination and in sociality dimensions might set the initial conditions that make pull to the cooperation attractor stronger (or weaker) than a pull to the autonomy attractor (independent action).

    In addition to the research sketched above, there is considerable research detailing links between neuropsychiatric disorders such as autism and schizophrenia and deficits in interpersonal movement coordination \citep{Frith2013,Wheatley2016}.  Autism spectrum disorder appears to be associated with a deficit in the capacity to process the communicative function of emotional expression. In particular, while children on the autism spectrum may learn to recognise a frown in a photograph, they are relatively impaired in decoding dynamic social cues in real time \citep{Hobson1986}.  The ability to recognise emotional voice and music (in healthy participants) has been localised to the mid-posterior extent of the right superior temporal cortex---this region of the cortex is most often associated with the visual perception of biological motion \citep{Pelphrey2005}.  Biological motion, a dynamic stimulus, is known to engage a different neural pathway than static, structural features \citep{Haxby2000}.  As such, the ability to process emotion from static images may not translate to equivalent ability to process emotion from visual motion cues.

    It is interesting to note that a failure of sensory attenuation – in particular the relative strength of sensory and prior (extrasensory) precision – has been proposed as the basis of autism – whose cardinal features include an impoverished theory of mind (Happe  Frith, 2006; Lawson, Rees, Friston, 2014; Pellicano Burr, 2012; Van de Cruys et al., 2014).

    %Autism: Low cog biases Hyper reality Sensory acuity

    %schizophrenia:High cognitive biases
    %Apophenia: spontaneous perception of connections and meaningfulness of unrelated phenomen


    %Grau-Moya, J., Ortega, P. A., & Braun, D. A. (2012). Risk-sensitivity in Bayesian sensorimotor integration. PLoSComputationalBiology, 8, e1002698. doi:10.1371/journal.pcbi.1002698






    \subsection{Team click and the dark room dilemma}
    The evidence reviewed above demonstrates the human ability to flexibly deploy a range of cognitive strategies, seemingly towards the ultimate goal of free energy minimisation.  Inherent in the application of the free energy principle to group exercise is the paradox that group exercise is an activity that appears to deliberately amplify uncertainty.
    The details of these proximate mechanisms could shed light on the phenomenon of team click and its associated downstream psychosocial effects, including social bonding and social cohesion.  Joint action in group exercise, and the phenomenon of team click in particular thus offers a fascinating space in which to investigate the social cognition of active inference in joint action.  Active inference provides a theoretical frame through which the social cognition of joint action can be more clearly rendered and comprehended.

    %How, then, do we explain the widespread prevalence of group exercise contexts, which appear to be riddled with uncertainty owing to the overwhelming complexity of joint action requirements and extreme physiological demands?  The phenomenon of team click could lie at the heart of an explanation to this paradox.



    \subsection{Group exercise offers ready-made semantic inputs}



    Team Membership

    Virtue plays of nation etc

    movemen










































































  \end{CJK}
