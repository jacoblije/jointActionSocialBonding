
\chapter*{Abstract}
%\addcontentsline{toc}{chapter}{\nameref{ch:study1intro}}


%\markboth{{Introduction to Part I}}{Introduction to Part I\label{ch:part1intro}



%\section{Abstract}
In this chapter I assess the predictions of this dissertation using ethnographic evidence collected with the Beijing Men's rugby team.  Having described the specific cultural terrain of the research context in the previous Chapter (Chapter ~\ref{chap:ethnoField}), the contours of generalisable cognitive mechanisms relevant to the predictions of this dissertation become more visible. I identify three key components of athlete experience---perceptions of performance in joint action, feelings relating to team click, and processes of group membership---which are relevant to the hypothesised relationship between joint action and social bonding.  I also outline evidence for possible moderating variables of these relationships, such as technical competence, personality type, injury, and fatigue.  I conclude the chapter by summarising and discussing the results, with a particular focus on how these observations could be operationalised in further experimental studies with a larger sample of athletes beyond the Beijing men's team.
