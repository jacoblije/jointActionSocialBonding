\chapter{\label{4ethnographicField}Ethnographic Results 1}

  \minitoc



  \section{Abstract}
  In this chapter I present the results of ethnographic data collected with the Beijing Men's Rugby team between September 2015 and August 2016.  I find a range of evidence in support of the predictions set out initially in Chapters 2 and refined in Chapter 3 in light of a review of the contextual specificities of the research setting.  First, I describe the culturally specific terrain of social cognition, which is defined by a dominance of ``hierarchical relationalism.''  The dominance of this culturally specific mode of social cognition is identifiable at multiple levels of social life, from the level of the institutions in which athletes, coaches, and officials interact, to the level of group norms in which athletes and coaches participate, to the level of on-field processes of joint action and perception. Having described the details of this cultural terrain, the contours generalisable cognitive mechanisms relevant to the predictions of this dissertation become more visible. I identify three key categories of athlete experience---1) perceptions of performance in joint action, 2) feelings relating to ``team click'', and 3) processes of group membership---which I argue are relevant to the hypothesised relationship between joint action and social bonding, as well as the hypothesised mediating role of team click in this relationship. I also outline evidence for possible moderating variables of these relationships, such as technical competence, personality type, injury, and fatigue. I conclude the chapter by summarising and discussing the results, with a particular focus on how these observations could be operationalised in further experimental studies with a larger sample of athletes beyond the Beijing men's team.

                                            \begin{CJK}{UTF8}{gbsn}

  \section{Vignette: My first assignment at the National Tournament in Qingdao}

  On that Monday day that I first arrived at the Institute and rounded the Stadium to its main administrative building, almost two years had passed since the Beijing Women's rugby team's controversy at the 2013 National Games.  Fortunately, my first visit to the Institute ended up going as smoothly as I could have hoped.

  I first met with Jenny, the Vice-Principal of the Institute who was responsible for rugby.  I sat down with Jenny in her office as she finished a conversation to someone on the phone. Jenny flowed with the personality and affectations that only a true Beijing local could embody---her word endings textured with coarse yet elegant ``erhuayin''(儿化音) suffixes, and she addressed the person to which she was speaking using the respectful version of the 2nd person pronoun ``nin3.''  Once she had finished with the call, Jenny welcomed me and tactfully explained to me that rugby at the Institute had indeed experienced a dramatic fall from grace.  Jenny indicated that the head coach ZPH and his assistant SY really had their work cut out for them, and that my presence as observer and occasional coach would benefit the team.  She agreed to organise a room in the rugby program dormitory, as well as access to the Institute's canteen, in exchange for my expertise.

  Buoyed by this meeting, I made my way to where the rugby program's dormitory to meet head coach Zhu Peihou (see map X).  My connection to ZPH went back to CAU in 2008, where he had been a coach at the time. From Shandong originally, ZPH was a graduate of the Shanghai Sports University, another prominent rugby program at the time.  ZPH had been recruited from Shanghai to CAU by ZHJ to coach so that he could continue to play for the Chinese national team after he had completed his undergraduate studies.  After we had discussed my research and he had provisionally approved my plan to spend the next period with the team, I asked him about the current situation with the Beijing team.  ZPH explained that he was quite frustrated that the group of athletes he was coaching lacked experience and maturity. I asked him exactly what areas of the team's performance, and he indicated that all areas were not great, suggesting that not enough players had found that ``feel'' for gameplay and very few were motivated to train hard.  We agreed that my first assignment should be to accompany the team to the final National Tournament in the coastal city of Qingdao in Shandong province, in a week's time.

  As it turned out, the National Tournament in Qingdao was organised to take place immediately following a two-day Asia Sevens Tournament.  The Asia Sevens is an annual series of three regional rugby sevens tournaments featuring men's and women's national sevens teams.  The men's series has been held regularly since 2009, and the women's series was established in 2013. The series usually consists of two three annual Tournaments, alternating between various locations including Hong Kong, South Korea, Sri Lanka, China, Japan, Malaysia, India, Singapore, and Thailand.  In 2015, The Asia Sevens Series Tournaments were held in Colombo, Bangkok, and Qingdao. The Tournament would be played on the Saturday and Sunday, and then the National Tournament would follow immediately after on the Monday and Tuesday.

  \subsection{Asia Sevens Tournament}
  When I arrived at the stadium in time for the beginning of the Asia Sevens Series, I met a number of coaches and athletes in the stands who I knew from my time in Beijing in 2008 and then coaching in 2013. I sensed from my various interactions that there was an air of nervousness around the Tournament, particularly on behalf of the Chinese women's team.   The Qingdao tournament was the final tournament before the Olympic Qualification Tournaments, to be held in Hong Kong and Tokyo in November 2015.  The top ranked team from those two legs would qualify for the Rio Olympics in 2016.  The Chinese men's team were not in serious contention for Olympic qualification, given the clear superiority of the more established men's rugby programs like Japan and Hong Kong. The Chinese Sports Commission did, however, expect the Chinese women's team to qualify for the Olympics.

  Since the Chinese women's team was first established in 2002, China had made great strides in women's rugby, in both Asia and globally.  Not including occasional losses to closest rivals Kazakhstan, between 2002 and 2012 the Chinese women's sevens team was the dominant women's team in Asia, easily outcompeting Japan and Hong Kong, and at times was competitive against the world's best including New Zealand and Australia.  The main reason for China's dominance in the women's game during this period was that other traditional rugby nations, despite having developed professional infrastructure for the men's game, lacked almost entirely an equivalent infrastructure for the women's game. China`s state sponsored sport system, on the other hand, was relatively agnostic towards gender in sport. When it comes to the bare incentive structure of the Chinese sports system a gold medal is a gold medal, regardless of the gender of the recipient. (This is of course not to say that there are not distinct gender inequalities in relation to sport in China).  Indeed, beginning with the Chinese Women's Volleyball Team's gold medal victory at the LA Olympics in 1982, China has enjoyed a comparative advantage in women's sport due to the fact that the Chinese sport system was comparatively more supportive of women's sport.

  Following his success at the National Games in 2013, Shandong head coach Lu Xiaohui was given the responsibility as coach of the Chinese women's program in 2014.  Essentially, this involved giving Lu and  Shandong province the responsibility for the team.  The national team trained at Shandong's provincial training centre, and most of the athletes who represented China in 2014 were from Shandong.

  Alarmingly for the Chinese women's rugby team's hopes of Olympic qualification, by 2014 it had become obvious that other more traditional rugby playing nations in Asia, namely Japan and Hong Kong, had begun to make up serious ground on China and Kazakhstan in the women's game.  This naturally prompted nervousness among the Administration therefore and CRFA.  In mid 2015, it was decided by the Administration that CRFA would enlist the services of a foreign coach to bolster their campaign for Olympic qualification.  According to sources close to CRFA, apparently the original plan was for the appointed foreign coach, Ben, to act as a consultant for Lu and his existing group of coaches.  By the time I arrived in Qingdao in early September, however, the initial arrangement had since transformed into a situation in which Ben was given 100\% control over the program as head coach, and LXH was more or less sidelined as coach. Not only had Ben took over, but he had also set about reorganising the starting team and also scouting for new talent outside the squad, which was predominantly made up of Shandong athletes.

  %The complication was that before being dethroned, LXH preferred to use his own athletes.  Most of the starting team at the LDN7s in June 2015 were indeed Shandong athletes, many of the women who had won gold at the National Games in 2013.  When BG took over the reigns as head coach, as directed by the GAS and CRFA,

  When I arrived in Qingdao it appeared that tensions between the old and new guard were at their peak.  Lu and many of his favoured athletes had been relegated to the sidelines, and some had been completely removed from the squad altogether.  There were still six weeks to go before the all important first Olympic qualification tournament in Hong Kong.  I sat and watched the first few games of the women's Tournament, and I was indeed surprised to see that the Chinese women's side was missing some of its usual stalwarts, and indeed appeared in my eyes to lack the flow and familiarity that I had come to expect in Lu's clan of Shandong athletes.  In the stands I came across a group of Shandong women, one of which, Qi Gaige, I knew quite well from when she toured to Australia with the Shandong team when I was still with the Australian rugby sevens team in early 2013.  I asked Gaige why she wasn't playing for China, and it turned out that she was injured, and so wasn't eligible for selection.  But a few of the other women around her, who were all wearing Chinese national team uniforms, were all part of LXH's clan who had been effectively stood down by Ben. ``How do you think they're playing?'' (你觉得她们打得怎么样?) She asked me, after we had exchanged pleasantries. ``Not great'' I commented, hesitating, not knowing how much I should prime her, but also feeling obliged to be honest: ``it feels like they’re not coordinating together very well at the moment.  What do you think?'' (一般吧。感觉她们的配合不太好,目前。你觉得呢?) I asked.  ``They’re out there playing as individuals, not playing as a team! They can't get it together; there's no shared goal.'' `(她们都在打个人的,不打团体提的。打不到一块儿去啊,没有共同的目标.)  ``Hmm. Yes it does look like that.'' (嗯嗯,看起来像是) ``Hey, Lijie...'' she asked me quietly, ``...don’t you think they’re not even playing as well as our Shandong team could play?'' (嘿,李杰,是不是她们现在打的没有我们山东队打的好,是吧?)

  Later that day I bumped into an assistant coach of the Chinese women's side, who had been working under Lu and now Ben since 2014.  Li Sheng was a big and booming Qingdao local, a CAU graduate, and a member of the Beijing Men's team from 2010-2013.  I asked Li about the current situation with the Chinese women's side and their prospects 6 weeks out from the first Olympic qualifier.  ``Chinese athletes must see the (individual) benefits if they are going to go all out'' (中国的运动员必要看到个人的利益才会全力以赴的) he insisted, and he went on to explain why for these athletes, there were no obvious benefits available sufficient to motivate them.  There were indeed very few material benefits associated with representing China in rugby at the national level.  Athletes were payed a nominal USD100 per month on top of their provincial contracts when training and touring with the national program.  If athletes were injured while playing for China, CRFA at the time did not have access to sufficient health insurance to cover the costs of treatment, and athletes had no choice but to return to their provincial programs and seek treatment at the expense of the province.  The less tangible benefits of playing for China, for example, access to high quality coaching, or the pride of representing the country in a sport, or even the promise of a trip to the Olympics, were heavily outweighed by other less tangible costs: long stints of time away from family, the constant risk of falling out of favour with provincial programs.  In effect, the lack of incentives at the national level meant that athletes were by definition more committed to their provincial systems---the programs that provided athletes with the benefits that they were most interested in obtaining, such as tertiary education, future employment, and a modest---but compared to CRFA--a reasonable salary (most national level players were paid 3-8k RMB/month). ``Its no wonder these athletes aren't performing well,'' (难怪这个队伍的表现不好),Li exclaimed.

  \subsection{National Tournament}
  On the first day of the National Tournament I met Beijing head coach Zhu in the stands before their first game, and he instructed me to keep an eye on the games that Beijing would play, and offer any feedback about things they could work on.  Of course, eager to being my observation of the athletes I would go on to spend many months together with, I obliged, and sat down with a notebook and watched the days play.  In their first game, Beijing obviously lacked coherence in attack and defence.  Their basic skills, for example, passing accuracy and work in contact needed more work, and it did not seem as though they had anyone who was performing the role of leader and providing the team with direction on the field.  I felt that they lacked an element of maturity and patience that would be required to win tight games of rugby sevens. It was clear that the team was made up of many inexperienced athletes, many of whom had barely played in serious official Tournaments such as this one.

  Regardless, Beijing ended up winning both its games on Day 1. I couldn't help but be puzzled by the atmosphere when the team huddled together around head coach Zhu after their victory over the PLA in their final game of Day 1.  While the younger athletes were concentrated intently on Zhu's every word, it was apparent that some of the older players were less focussed, in fact many of them appeared to not even be listening to Zhu.  In particular, I noticed that Wang Chongyi, a former Beijing men's team representative and assistant men's coach at the time, was as physically withdrawn from the huddle as possible while still maintaining connection to it by holding loosely on to the shirts of the athletes either side of him. Wang was seemingly uninterested in what Zhu was saying.  Some of the senior athletes also appeared less engaged with what Zhu was saying.  I was puzzled by this piece of team theatre and wondered if I was simply reading too much into it, and that perhaps the posturing and performance of team unity was less emphasised in this context than the contexts in which I had played rugby elsewhere. As was common on my journeys through the world of rugby in China, I often did not quite grasp all the details and pieces of the puzzle that contextualised the interactions I was having until after the fact.

  Despite the obvious deficiencies in Beijing's performance that weekend, they did well enough over the two days to make the final of the Tournament and indeed win the National Tournament.  This achievement probably owed more to the relative weakness of the other provincial outfits, more than Beijing's out and out strength.  In an unfortunate twist of fate, however, the Beijing side was later disqualified from the Tournament, due to the fact that they fielded an underage player, who was only 15 at the time.  It was clear that Zhu and his team could not catch a break.  After reconvening back in Beijing, Zhu gave athletes the rest of September off training, and instructed all to reconvene at the Institute at the start of October to begin off-season training.


















  \section{Results}

  In this Chapter I describe the research context, and analyse evidence for its culturally specific terrain.  The results presented in this chapter set the foundation for the following chapter, in which I analyse evidence that supports the core preidctions relating to joint action, team click, and social cohesion.


  \subsection{Rugby at the Institute after 2013}
  Having let the dust settle on the embarrassment of the women's program's widely publicised disqualification from the 2013 National Games, in 2014 the Institute decided to quietly continue with both the men's and women's programs, in preparation for the 2017 National Games.  In April 2014, more than six months after the National Games, the men's program was resurrected with the appointment of a new head coach. The junior athletes from the previous National Games cycle were recalled back to the Institute to resume training, and the new head coach Zhu Peihou was charged with finding new talent. A former Chinese representative and CAU coach, Zhu had previously coached the Anhui women's team during their 2013 National Games campaign.  Zhu appointed his close colleague and graduate from the Shanghai Institute of Sport's rugby program, Shi Yan, as assistant coach.  Both Zhu and Shi were employed by the Institute on a contract basis, rather than becoming full-time official employees of the Institute.

  The women's program was inactive for a full two years after 2013, and was only just starting to re-activate when I arrived in October 2015.  Former Beijing women's rugby representative (2010-2013) and Beijing local Ma Jiale was appointed as head coach, and former CAU graduate, Chinese National Team representative, and Beijing men's rugby representative athlete and Beijing local Wang Chongyi was appointed as assistant coach.

  Thus, rugby was resurrected at the Institute, but was no longer in centre stage. The Beijing men's team endured a series of mediocre performances during the 2014 and 2015 seasons, and clearly lacked experience, talent, and institutional support from the Institute.  A handful of senior athletes who had played in the era of the 2013 National Games remained, and two in particular, Han Xiaolong and Lu Peng were promoted to a transitional athlete-coach status. Unlike Women's assistant coach and former athlete Wang Chongyi, however, both Han and Lu were originally from Shandong province and so did not automatically have Beijing residency required to make them eligible for full time employment at the Institute. As such, their future place at the Institute was uncertain, and as I found out from both during the course of my ethnographic research, their ability to stay at the Institute would depend on the result the team could achieve at the 2017 National Games.

  It is important to note that both coaches appointed to the women's program in 2015, Ma and Wang were---unlike Zhu, Shi, Han, and Lu---full-time permanent employees of the Institute. As explained in the previous chapter, China's infamously rigid ``Hukou'' (户口)residency system means that it is still the case that only individuals with Beijing residency can hold permanent employment roles at government institutions such as the Institute (\textit{shiye danwei} 事业单位).  Chinese citizens born outside of Beijing can become Beijing residents if offered employment, but due to Beijing's swelling population, the eligibility criteria for this process of naturalisation has become more and more stringent, and fewer and fewer applications are successfully processed, particularly in industries like sport.\footnote{This links to the question of the ``Quality'' of athletes}

  Despite being only a shell of its former glory, the rugby program at the Institute nonetheless offered attractive incentives to prospective athletes.  The difficulties of Han and Lu in gaining Beijing residency made it clear to more junior athletes that there was little promise a passage to official Beijing residency or full-time employment at the Institute, but the program did offer the much more realistic opportunity of attending the Beijing Sports University (BSU)---considered to be the country's most prestigious sports universities and one of China's top ``brand universities'' (\textit{mingpai daxue} 民牌大学).  The mass exodus of experienced senior athletes from the rugby program meant that junior athletes from the pre-2013 era were now in a position to represent Beijing at a national level, and therefore attain the official athletic standard of a ``Master Sportsperson'' (\textit{yundong jianjiang} 运动健将), and thus eligibility to attend BSU.


  \subsection{Participants}
  For the ethnographic section of this dissertation, I analysed data on a total of 26 Athletes ($avg. age = 21.3, range = 18-27, SD = 2.96$) and four coaches, who were not included in the main analysis but provided important contextual information.  Athletes were included in data analysis if they participated in 1) a semi-structured interview, 2) at least one informal survey relating to experiences of rugby training and group membership, and 3) at least 2 months of training at the Institute.  See Table ~\ref{tab:ethnoDescriptivesTable} for a summary of athlete attributes, including team status, contract status, etc.

  % latex table generated in R 3.5.0 by xtable 1.8-2 package
% Wed Jun 20 16:44:16 2018
\begin{table}[ht]
\centering
\begin{tabular}{ll}
  \hline
row & Overall \\ 
  \hline
n &    26 \\ 
  Age (mean (sd)) & 20.96 (3.17) \\ 
  ResearchCategory = Senior (\%) &    10 (38.5)  \\ 
  TrainingAge (mean (sd)) &  3.34 (2.02) \\ 
  YearsInTeam (mean (sd)) &  2.59 (1.80) \\ 
  AthleteStatus (\%) &     \\ 
     Master Sportsperson &    10 (47.6)  \\ 
     Level 1 &     6 (28.6)  \\ 
     Level 2 &     5 (23.8)  \\ 
  ContractStatus (\%) &     \\ 
     Permanent Employee &     1 ( 3.8)  \\ 
     Full Time Contract &     7 (26.9)  \\ 
     Training Contract &     5 (19.2)  \\ 
     Student Contract &     6 (23.1)  \\ 
     Trial &     7 (26.9)  \\ 
  EducationLevel (\%) &     \\ 
     Graduate &     1 ( 3.8)  \\ 
     Undergraduate &    12 (46.2)  \\ 
     High School &    10 (38.5)  \\ 
     Middle School &     3 (11.5)  \\ 
  HomeProvince (\%) &     \\ 
     Shandong &    11 (42.3)  \\ 
     Beijing &     6 (23.1)  \\ 
     Jiangsu &     3 (11.5)  \\ 
     Liaoning &     2 ( 7.7)  \\ 
     Hebei &     2 ( 7.7)  \\ 
     Heilongjiang &     1 ( 3.8)  \\ 
     Fujian &     1 ( 3.8)  \\ 
  PreviousSport (\%) &     \\ 
     Athletics &    16 (61.5)  \\ 
     None &     8 (30.8)  \\ 
     Basketball &     1 ( 3.8)  \\ 
     Football &     1 ( 3.8)  \\ 
   \hline
\end{tabular}
\caption{Athlete Desciptives (n = 26)} 
\label{tab:ethnoDescriptivesTable}
\end{table}


  The team consisted of one full-time employee (Su Hailiang, by virtue of the fact that he was a Beijing resident), seven fully contracted athletes (\textit{xieyi}), five provisionally-contracted athletes \textit{shixun}, and six training team athletes \textit{erjiban} who do not receive a salary but received training, food, and board (see Table ~\ref{}).  The remaining athletes (seven) were classed as athletes in training \textit{jixun} and were effectively on a trial arrangement until they showed promise or else withdrew from the squad, either voluntarily or upon suggestion by the head coach.  Most athletes were from urban and rural areas of northern China (Shandong (11), Beijing (5), Jiangsu (3), Liaoning (2), Hebei (2), Anhui (1), Henan (1), Heilongjiang (1)). Athletes were, generally speaking, and from what I could gather, from relatively modest socio-economic backgrounds.

  The average rugby training age (years spent playing and training in a rugby program) of Beijing athletes was 3.12 years ($range = 0.16 –-- 10 years$).  17 Athletes had a background in other sports (15 athletes from track and field, one from football, one from basketball), usually beginning part-time or full-time physical training at the age of 11-13.  Those who transferred to rugby from other sports did so either at the beginning of senior high school (16 years) or at university age (18 years).  The rest of the group (9) had no particular sporting background before starting training with the Program, and were scouted by the head coach of the Program or by school athletics coaches based on their basic athletic attributes (running speed, strength, coordination, and potential for physical growth).  Of the 26 athletes in the squad, three junior athletes who were part of the squad when I arrived in September 2015 since left, and three new athletes arrived during the time I performed research. This flux of non-contracted players in and out of the Program was quite common, the reasons for which I explain below (HYPERLINK).

  %Contracted senior athletes ($average age = 24.3 years$) had trained for an average of 5.4 years, whereas the average training age of junior non-contracted athletes (average age = 19.3 years) was 1.7 years.


  \subsubsection{Training schedule}

   The Beijing men's rugby team competed against other provinces in five national tournaments held in different locations across the country   every year between March and September. The period in which I conducted my first stint of ethnographic research (September 2015 –-- February 2016), therefore, constituted the off-season and pre-season components of the training year.  Due to cold weather in the north of China during winter and spring, teams from northern China (e.g. Beijing, Liaoniang, and Shandong provinces) often elected to train   at other domestic or international training locations depending on amount of program funding available and the training strategy of each program.  In 2015, before an unexpected change in coaching team at the end of December (explained below in Section HYPERLINK), the head coach of the Beijing Men's team had planned to travel to Yunnan province in early 2016 for one month of altitude training before moving closer to sea-level somewhere in the south of China for one month (February/March).  Following the coaching leadership change, the team did not leave Beijing until after Chinese New Year (25th February). Training during this period was therefore consistently stationed at the Institute in Beijing, and as such subject to occasional disruption due to Beijing's cold winter weather and air pollution.

   All athletes lived and trained 6 days a week at the Institute, and would occasionally attend university or high school classes as part of their ongoing education commitments.  Below is a table of a typical weekly training schedule (see Figure ~\ref{tab:tournamentData}). A typical week consisted of 10 two and a half hour (150 minute) training sessions, seven of which were on-field rugby sessions, three of which were strength and conditioning sessions (not involving a rugby-specific skills).  In addition, two one hour evening skills sessions were also allocated for junior athletes to hone their basic skills of passing, catching, and game-play.  Athletes lived full-time on campus in the Institute's dormitory accommodation (usually 3 athletes per room), and were permitted to take overnight leave on the weekend after the conclusion of Saturday morning training.  Athletes from Beijing or with family in Beijing would usually take this leave, while the remaining athletes would spend weekends at the Institute.  Generally speaking, the rugby program would break at the end of the national season in September for two weeks, and occasionally around Chinese New Year for 7-10 days, unless New Year interrupts pre-season training plans, in which case training would continue in spite of this national holiday.

    \newgeometry{margin=0.5cm} % modify this if you need even more space
    \begin{landscape}
      \begin{table}[htpb]\caption{Weekly Training Schedule}
        \begin{center}
          \begin{small}
              \begin{tabular}{| c | c | c | c | c | c | c | c |}
                \hline
                & \bf M & \bf T & \bf W & \bf T & \bf F & \bf S & \bf S \\
                \hline
                0600 & Training &  &  & & & & \\
                \hline
                0900 &  & Training & Training & Training & Training & Training &  \\
                  \hline
                1500 & Training & Training & & Training & Training & Training &  \\
                  \hline
                1900 &  & Training (junior athletes) & & Training (junior athletes) & & & \\
                   \hline
              \end{tabular}
                  \label{tab:tournamentData}
            \end{small}
          \end{center}
        \end{table}
    \end{landscape}
    \restoregeometry




  \subsubsection{Participant Observation}

    \subsubsection{Interviews}

    In addition to 26 semi-structured interviews, I also conducted 6 unstructured interviews with three members of the Program (the head coach and the two most senior athletes) and two former coaches of the Institute.  These unstructured interviews were not included in the main analysis but provided important ethnographic context for the main analysis.

    \subsubsection{Informal Surveys}

    All 26 Athletes participated in the post-interview survey tasks.  For post-training surveys, 12 Athletes (age, etc.) participated in the survey following the Beep Test training , 16 Athletes (age, etc.) participated in the survey following the first match-like training session, and 15 Athletes (age, etc.) participated in the second match-like training session a week later.

    A total of 23 Athletes (age, etc.) participated in the general survey administered twice with a three month interval. This general survey occurred either side of the change in coaching staff over the Christmas period of 2015.




  \subsection{The Research Context}


  \subsubsection{Team Factions}

  After a short time at the Institute, my conversations with (predominantly senior) athletes and coaches helped me identify a number of factions in the team, which appeared to align not only with each athlete's position in the rugby program, but also by the incentives that each athlete was pursuing through participation in the program.

  \myparagraph{What was left of the Old Guard (2)}
  At the top of the hierarchy, senior players Han and Lu were interested predominantly in the prospect of gaining full time employment and Beijing residency through their continued involvement in the program. The Institute employed both athletes on full-time contracts (Ixia), but neither were formal permanent employees (Zhengshi) of the Institute.  Lu had finished his undergraduate studies at CAU, while Han had failed to complete his studies at CAU, and had instead transferred to BSU to complete his undergraduate degree. Han was still completing the second year of his undergraduate studies when I arrived in 2015.

  According to both Han and Lu, formal permanent employment and Beijing residency were part of the incentives that were originally tabled in contract negotiations in 2010 when both athletes were enticed to the Institute from CAU as young and promising athletes.  The Institute reportedly set performance targets for the programs of a Women's gold medal and a top-three finish for the men.  Given the chaos that transpired at the 2013 National Games, however, the Institute's promises did not materialise for athletes like Han and Lu, even though the men technically achieved their performance goal.
      \footnote{As it turned out, many of the senior athletes in the men's team pre-2013 were already Beijing residents, due to the fact that many were from CAU.  Thus, the Institute's promise of residency applied to only a small portion of athletes that they recruited.  After many athletes were enticed away from Beijing to the Tianjin rugby program in early 2014, the only first team athletes to remain at the Institute were Han and Lu.  Both expressed frustration to me about not receiving the incentives that both felt they had rightfully earned, and this sentiment motivated a lot of complaining and problematisation of the program.}
  When I consulted outside observers on this situation, some suggested that the fact that both Han and Lu had not been granted the incentives that were promised to them was a deliberate mechanism, commonly employed by leaders of sports institutes to motivate commitment to the team at least until the end of the 2017 National Games.  Suffice to say, both Han and Lu sat atop the athlete hierarchy of the men's team in an uncomfortable position in which they were not confident that they would achieve what they wanted from their involvement in the program, particularly given that hopes for success in 2017 were uncertain at best, due to the way in which the men's team had been decimated after the events of 2013. During my time at the Institute, out of all of the athletes, I interacted most directly and personally with Han and Lu, given that they were the most senior athletes and we had known each other previously from my time spent at CAU.

  \myparagraph{The new senior athletes (8)}

  Below Han and Lu was a collection of eight athletes who were now considered members of the ``first team'' (Zhuli).   These athletes were either on full-time contracts (Xieyi), or otherwise had been promised full-time contracts (in the case of one athlete who had transferred from Shandong, and another graduate from CAU). But unlike Han and Lu, none of these athletes were particularly central to the senior team pre-2013.  Many of these athletes were recruited between 2010-2013, from other programs at the Institute or from other sport programs such as football and athletics.  As such, very few of these new ``senior athletes'' expressed to me any strong short-term aspirations for attaining permanent employment and residency (although some may have harboured such aspirations in the longer term). Five of these athletes were in the process of completing their studies at BSU, and the remaining three had already completed their undergraduate degrees at CAU.

  %WW, MHT, WWX, WZF, SHL

  \myparagraph{The unruly undergrads (5)}

  Below the senior athletes were a group of five athletes who had been recruited into the program between 2010-2013, but had not subsequently progressed through the ranks to the first team. This group had, however, managed to qualify to study at BSU by virtue of their participation in national level tournaments in 2014 and 2015. As such, four of these athletes were beginning their first year of study at BSU, and the remaining athlete was preparing to do so. There was a clear distinction in technical competence between these athletes and the first team, and as such there was not a whole lot of competition between these athletes and the first team athletes for a spot in the first team.  Three of these athletes are on ``training contracts,'' while the remaining two are still on ``student'' contracts.  I call these athletes the unruly undergrads, as they are often criticised by coaches and more senior athletes for being constantly distracted by university life, and not having any motivation to commit to the rugby program, given that the key incentive available to them (university attendance) has already been awarded to them.


  \myparagraph{The aspiring athletes from Chaoyang Sports School (5)}

  Below the unruly undergrads are a group of five athletes who are also high school students and athletes at the rugby program at the Beijing Chaoyang Sports Institute, a city-level high school institute located in the east of Beijing.  These athletes had been playing rugby for one to three years, and were all attached to the Institute as ``second training class'' (\textit{erjiban} 二级班).  Second training class athletes received coaching, board, and food, but no form of remuneration.  All athletes are aspiring to transition to full time members at the Institute and representatives of the Beijing men's team.  In order to qualify for admission to the Institute as a contracted athlete, these athletes must first complete high school and attain the standard of a level 1 athlete (一级运动员), which can be achieved by representing the Beijing Youth Rugby team at a national level youth tournament.  Once these athletes become full members of the program at the Institute, they will then be able to pursue subsequent opportunities such as university attendance.

  \myparagraph{The hopefuls on trial (6)}
  Finally, the remaining group of six athletes were all at the Institute on a ``trial'' basis.  These athletes usually appeared via some connection to either the head coach of the Institute or via the relational network of one of the Principals at the Institute, as was the case with SHW (see Chapter 2 Vignette HYPERLINK). The position of these athletes at the Institute was deeply uncertain.  Most had transitioned from athletics or another mainstream sport in which they had achieved a minimum standard of performance in their event (Level 1) thus making them eligible to attend the Institute.  Most, however, had not played rugby before, and so had a large gap in technical competence that each was attempting to address.


  %\subsection{Daily Life at the Institute}
  %Head coach Zhu and assistant coach Shi looked after the day-to-day organisation of team schedules and training. Zhu and Shi were also assisted by a further two player-coaches, Han and Lu, who were in the gradual process of transitioning from athlete to coach status.  These player-coaches were included in the analysis as Athletes.  One of the Institute's four Principals, Jenny, was responsible for the management and administration of the rugby program, and as such was occasionally present at team meetings and national competitions.

  %Athletes were either already contracted with the Program as students, full-time contracts, or formal employees of the institute (in the case of only one athlete and the head coach) (n=13), or aspiring to become students, or contracted athletes (n=13).


  \subsubsection{An abrupt leadership transition}
  Zhu and Shi were ostensibly in charge of the team when I entered in 2015, despite their being signs that their authority may not have been rock solid.  If I had understood as much about the machinations of politics and power in Chinese sport then as I do now, then perhaps I would have understood Wang's disengagement from the post-Match team huddle was a clear signal that the writing was on the wall for Zhu, and that Wang---a Beijing resident and Asian Games Bronze medallist---would eventually take his place atop the Men's program.  Key senior players like Lu and Wei Wenxin complained to me constantly about Zhu and his approach to coaching and team management.  I could also tell the women's coach Zhu and his communication style frustrated Ma.  And then there was the interaction with being forced to take on SHW, and the lack of support from Institute leadership for off-season training plans overseas... It was also the case the Zhu had barely taken an on-field training session since I had arrived to do my research.  After a few weeks of me settling in and observing, Zhu would suggested that I take over half of the on field sessions.  There was a period in which he barely took training that he announced to the team that he was focussing on the injured players instead...

  In hindsight, it is quite clear that Zhu and Shi were perhaps only ever going to be a temporary stop-gap for the Institute's program; outsiders employed to come in and do the thankless job of resurrecting a sporting program rugby team from a shameful fall from grace.  But at the time, only four months into my first stint of ethnographic research and still largely unacquainted with (and without access to) the deep details of politics at the Institute, I was not prepared for what happened when I returned to Beijing from two weeks at home in Australia over Christmas.

  When I returned to Beijing on the 31st December, I stayed the night at my friend Kai's place, and the next morning I messaged head coach Zhu to ask when would be best to return to the Institute to resume research.  His reply was brief and somewhat odd: ``You should be able to [return to the Institute], you should probably make contact with Han or maybe Coach Wang.'' (应该可以,你联系一下小龙呗,或王导呗).  This reply was a little bit opaque, but I didn't challenge it as was the polite norm with someone like Zhu who was my senior.  Instead I contacted Han, as directed by Zhu, and asked if I could go to the Institute that evening and if the team had the afternoon off: ``Mmmm yeah.. we're off training today. I'm currently out (not at the Institute). Teacher Zhu and Coach Shi have left...now Wang is head coach.  You're still in Room 113 for the time being...'' (嗯嗯 休息 我在外面 朱老师跟石导走了. 现在由王重一主教练 你还是先住113).  I didn't really register what he was saying and replied:
  ``So Zhu and Shi have gone home to rest today too?'' (朱,石,回家休息是么)
  ``No, they are no longer coaching Beijing...When you get back we can meet and I will explain it all to you, ok? '' (不是的 是不在北京做教练了等你回来了 我们见面再好好跟你说说吧).

  And like just like that, Zhu and Shi were out of the job, and Wang had assumed the role of head coach of the men's team.  Han and Lu became his assistant coaches, and some weeks later, Zhu Jing, a classmate of Wang from CAU who represented Xinjiang province in the 2013 National Games, joined Wang as assistant coach in Beijing.

  In the aftermath of the change of leadership, I heard gossip that Zhu had pushed the Institute to provide more funding towards the team's off-season training program, which Zhu wanted to take either to Fiji in a best case scenario, otherwise to high altitude in Kunming (in China's southwest) and then down south to Guangzhou. After the Institute (presumably Jenny) refused to support these plans, Zhu allegedly made the call to resign from the role, saying that there was no way he could do his job properly.

  Soon after the change, there appeared to be a renewed energy in the team.  Senior players Han and Lu had been promoted and appeared to respond to this promotion with positivity and enthusiasm.  Zheng Hongjun, the old boss of Chinese rugby, had coached Wang, Han, and Lu and they all saw eye-to-eye about training techniques.  The training schedule immediately reverted to an older more familiar format, and the content of training was dedicated largely to basic skills, fitness, in order to support a style of play that Zheng pioneered in Chinese rugby.  Zhu had been attempting to lead the Beijing team towards a different style of play in the years that he had been coach.  I got on well with Wang (we had known each other from CAU days in 2008 and 2013), and so I also experienced an increase in positivity and feeling of increased inclusion after he took control of the team.







  \section{Analysis of Study Predictions: Culturally specific terrain}

  In this section I provide evidence for the specific predictions set out in the previous Chapters.  In particular, I make two broad predictions concerning my ethnographic setting.  First, I predict that my research context of professional rugby in contemporary China will entail culturally specific affordances that will be relevant to enabling and constraining certain modes of social cognition.  Second, I predict that within this culturally specific terrain, contours of generalisable cognitive mechanisms relating to joint action and social cohesion will be identifiable.  In the sections below, I review evidence for each of these predictions.

  I predict that the research context of professional rugby in China will necessitate culturally specific cognitive affordances that will enable and constrain the social cognition and joint action in patterned and predictable ways.  China is home to a dynamic indigenous psychology (see Chapter 3 Section X HYPERLINK), which is the product of a number of distinct but interwoven historical trajectories.  Among these trajectories are two core strands of interest, namely: 1) millennia of Dynastic rule involving institutionalised norms of social interaction (commonly generalised as ``Confucianism'' or ``hierarchical relationism'') combined with draconian legal and bureaucratic mechanisms of governmentality, and 2) the rejuvenation of China as a modern Marxist/Leninist socialist nation-state in the shadow of 150 years of colonial domination and humiliation at the hands of foreign (non-Han) actors (including Manchurian rulers of the Qing dynasty).  The modern project of competitive sport serves as a fascinating area of analysis in which the dynamic interaction between and interweaving of these two cultural-historical trajectories can be witnessed.

  Conventional theories within Western social psychology predict that identity can be generated (or at least expressed) through a subjective psychological calculation of distance between categories of self and group, such that a 1:1 overlap between self and group identity represents identity fusion \citep{Swann2009}.  From this theory also stems a theory of motivation, such that motivation for pro-social behaviour is inversely proportional to the distance between categories of self and group. 1:1 fusion between self and group also entails maximal motivation to perform (potentially costly) pro-social activity on behalf of the group or individuals to which one identifies as fused.

  In the case of the men's rugby program at the Institute, however, concerns attention to relations appears to dominate over attention to categories of social identity as the primary mode of social interaction.  While distinct categorical modes of cognition related to the ``team'' and the institute are particularly salient in the context of the imported team sport of rugby, these categories do not appear to capture attention and perception in the same way as relational concerns.  Indeed, group processes relating to the team and identification with the team are cast in relational terms, utilising relational metaphors such as family and brotherhood.

  In this section, I argue, in line with existing research, that the distinctiveness of cultural terrain of professional rugby in China can be observed at multiple conceptual levels: 1) the social institutional level, 2) negotiation of group norms, and 3) processes of (joint) action and perception.  I will offer observational evidence at each of these levels, before proceeding to an analysis of generalisable cognitive mechanisms of joint action and social cohesion that are identifiable across this culturally specific terrain.

  \subsubsection{Institution as platform for social activity of hierarchical relational networks}

  The rugby program at the Institute entails two key institutional categories, namely the Institute (and the province or city that it represents, Beijing), and the Beijing team of which each athlete is a member, and over which the head coach presides.  My ethnographic observations suggest that while these institutional categories are important platforms for social interaction, they do not play a dominant role in enabling and shaping social interaction at an institutional level.  Instead, concerns for regulating and managing hierarchically structured relational networks, particularly to one's own ``guanxi'' network, dominate attention.

  As explained in detail in Chapter 3 (HYPERLINK), the Institute is one
  of four sports institutes in Beijing, and is home to seven sport programs.  One principal and four vice-principals overlook these seven sports, each principal being primarily responsible for one of the seven sports.  The Institute is of course held together by a bureaucratic backbone of regulations administered by city and central government.  These regulations set the parameters within which social interaction can take place.

  \myparagraph{Athlete motivations for rugby}
  After I completed each semi-structured interview, I asked each of the 26 athletes to complete an activity in which they were asked to rank (from most important to least important) their motivations for adherence to rugby at the Institute.  I selected these motivations based on preliminary unstructured interviews and conversations with athletes, coaches, and other knowledgeable observers.  The motivations included: ``Education'', ``For Beijing'', ``Family'', ``Gain Respect of Others'', ``For Teammates'', ``Employment'', ``Beijing Residency'', ``Money'', ``Enjoyment'', ``Popularity (with prospective romantic partner).'' I wrote each motivation down on a piece of paper and scattered them randomly on the surface of a table in my room where I conducted interviews.

  Results of this activity revealed the following mean rank of motivations:

  \begin{enumerate}
    \item Family (2.34)
    \item Education (2.38)
    \item To gain other's respect (3.50)
    \item For Beijing (3.58)
    \item Employment (3.96)
    \item For Teammates (4.00)
    \item Enjoyment (5.69)
    \item Money (5.85)
    \item Beijing Residency (6.31)
    \item Popularity (with prospective romantic partner) (8)
  \end{enumerate}

  It was telling to see that motivations of family and education ranked as the highest of athletes' motivations for rugby.  When asked in semi-structured interviews, almost all athletes agreed that their most prominent explicit motivation for playing rugby was to pursue life-course opportunities of education and employment. The opportunity for tertiary education was more proximal to athletes than employment opportunities, perhaps explaining education's higher rank.

  %F: motivations for personal strategic life-course opportunities, more explicit than teammates
      %Rugby/Employment: Old heads (1st team) and undergrads (BYH, MXK etc.)
      %junior: Young bucks (Chaoyang crew) and freshers (SHW etc..)
        %family strong, team weaker but still present; coach??


  There were a number of instances in which I observed the mood of athletes fluctuate according to the state of their attempts to gain entrance to BSU.  Two senior athletes, Ma Haitao and Cui Suocheng, despite long qualifying as a Champion Athlete, were involved in complex bureaucratic journeys in an attempt to gain admission to BSU.  One day early in my first stint of training I was taking a training session in which we were focussing on defence, and Shuocheng suddenly appeared to have given up all energy.  He all of a sudden stopped doing the prescribed tackling drill.  I took him aside and began to take him through some of the details he missed last week when he was away.  He refused to listen and said, “I get it, I just don’t want to do it, I’m sick of it, I’ve had enough of this drill (练够了!)”  “You’ve had enough?” I asked, “Alright, if you’ve had enough then get off the field!”  I snapped at him, motioning to the stands for him to go and sit down. I was stunned that he first said that he didn’t want to practice.  And as soon as I said get off the field he immediately did so and went and sat on a tackle bag.  Later he came back over to the subsequent training drill and started offering advice to the more junior athletes about their technique

  Some days later I found an opportunity on our way back from the dormitory to ask head coach Zhu about the situation with Shuocheng, and Zhu explained that he was experiencing difficulty finalising his contract transfer from Shandong to Beijing province, which was interrupting his ability to process his university application (Shuocheng had moved from Shandong provincial team in 2014).  Ma Haitao, another senior athlete, was also in the process of trying to apply for university (after years of delays).  One day he spoke to me at length about the way in which these difficulties were impacting on his mood and his ability to focus on training. ``It impacts me so strongly'' he said as we walked to the canteen, ``I’ll probably need to have a sleep or take a break from training before I can recover from these sort of frustrations - the endless process of getting this form and completing that form, its so frustrating''

  The fact that (immediate) family was the highest ranked social category above team-related categories of ``For Beijing'' (4th) and ``For Teammates'' (6th) was indicative of the primacy of motivation related to immediate relational networks of each individual.  While representing Beijing was clearly important, and indeed rugby was the source of gaining respect from others (3rd), these motivations appeared to be coordinated by concerns for family and life-course opportunities.  Very few athletes cited ``For Enjoyment'' (clear 7th) as the prime motivation for adherence to rugby.  It would not be inconceivable to see this as a strongly cited motivation for adherence to rugby in settings in which rugby exists as a popular mainstream sport.

  \myparagraph{The coach's clan}
  An example of the strength of relational networks over the institutions in which they are active can be seen in the power of the coach in coordinating the commanding the respect of the athletes below him or her.  The activities of the coach appear able to heavily influence the athletes who are able to enter and exit the team.  The change in head coach half way through my first stint of ethnographic research gave me the fortunate opportunity to witness the implications of this transition for the social organisation.

  Interestingly, when I returned to Beijing after a 10 day travel break in February, almost a month after Wang had replaced Zhu, two of the athletes who were on trial under Zhu---XG and LJX---had since left.  When I asked senior athlete Wang Wei about the disappearance of these two students, he suggested that ``they had interests with Old Zhu...Old Zhu said that he could solve their ``school problems'' [i.e. get them into university]. They were here to try to get in to University.'' (和老朱有利益关系,老朱说可以解决上学的问题 都是过来上学的). Talk of arrangements of this nature in sport was hushed but common, and it came as no real surprise to me by that stage.  As it turned out, on the day that I sat on the side of training talking to Wang Wei about Zhu and the trial students connected to Zhu, a group of four new trial athletes had come from a high school in the neighbouring province of Hebei.  These athletes were apparently introduced to Wang somehow through his relational network, and they arrived at the Institute and joined training for the day. This coincidence was a telling indication of the power of the coach to coordinate the members and activities of the team. The important thing to pay attention to is the fact that the Institute and the team serve as platforms for activates of relational networks to unfold. An athlete is only categorically related to the team by virtue of his or her capacity to adequately participate in and foster social relationships that play out on that institutional platform.

  \myparagraph{Leaders have the final say}
  One striking example of the dominance of hierarchical relationism over categorical social identification in motivating social attention and action came towards the end of my second stint of ethnographic research, in the summer of 2016 before the National Championship Tournament in which I performed the survey study (Chapter 6 HYPERLINK).  Less than a week before the team was set to depart for Qianan in Hebei Province, a very large and quite overweight young man appeared at training, dressed in Beijing rugby team apparel.  It was clear that the young man was completely unfamiliar with rugby, and nor did he appear to display any direct athletic potential relevant to rugby (he was quite heavy set and unagile and almost definitely unable to run).  I was somewhat puzzled by this new arrival, although I had previously witnessed the coming and going of the son of one of Vice-Principal Wang, who was affectionately called ``Chubby'' (小胖).
  Chubby was a normal high school student without a specific background in sport, but would join the team's training during his school holiday periods.  Chubby had become a popular and welcomed presence in the team; he assumed the role of most junior member of the team and diligently performed all the tasks expected of him in this role (carrying training gear to training, filling water bottles, etc.).  And over time, Chubby became considerably less Chubby, and more and more competent in the skills of rugby.

  I had not thought too much about the motivations for Chubby's adherence to rugby during school holidays, apart from assuming that the Vice-Principal wanted his son to participate in a team sport with many positive educational and health benefits.  Perhaps this assumption was somewhat naive.  That is at least what I thought once I discovered the reason for the arrival of this next young man before the year's most important Tournament.  As Wang later explained to me, this was also a high school student and was related in some more or less direct way to another Vice-Principal at the Institute (Wang didn't specify who, apart from referring to them simply as a ``Leader'' (领导)).  The student had arrived to familiarise himself with rugby, because he was going to be named in the Beijing team for the weekend's Tournament.  At some point, he was going to take the field in this Tournament, and in so doing attain the athlete standard of a ``Champion Athlete.''  (The obvious motivation was become eligible to attend BSU through the athlete pathway).

  I considered this to be one of the most absurd events that I witnessed during my fieldwork---it chafed against all of my intuitions as an athlete and a citizen of a Western advanced democracy.  A young, roughly 17 year old overweight student with no athletic ability and no familiarity with the game of rugby was planning (under the will and organisation of the seniors in his relational network) to take one of the 12 positions available to the Beijing team (leaving one legitimate Beijing athlete out of the side and unable to compete), so that he could take the field and therefore receive an undeserved passage to a prestigious university.  ``The Leader has the final say, I can't do anything about it'' (领导说的算,我没办法)said Wang, helplessly, when I pressed him on it.  Wang was only half way through his first year as head coach, and perhaps had been targeted as a coach who could be manipulated.  Indeed, it was quite possible that rugby more generally had been identified as a in which these types of activities were permissible.  Rugby after all was a gold gate sport of little standing in China---and this manoeuvre was unlikely to raise many eyebrows beyond those of the Australian anthropologist on the sidelines of an empty stadium. Clearly, this was an instance when the power enacted in the relational network was more dominant than the power of the social institution in which these hierarchically structured relationship were allowed to take place.

  Other Examples:
  - Jing at team meeting calling for ``Big Family'' ; LP's anger at Zhu after the meeting
  - 不要站错队 -- Zhu Jing trying to organise relational factions within the team
  F: HXL with contract and residency permits (HXL's period of withdrawal)
  F: ZPH when SHW came in above his control.

  SIGNIFICANCE: These instances demonstrate that the relational mode of interacting captures attention and perception. Any sanctity around the institutional boundaries of the team or the Institute appears to be secondary to activity that appears on the surface to be strategic, but in reality stems from a deeply relational (social) logic.

  %\footnote{This assertion was probably true for XG; LJX's story was a bit more complex. LJS did appear He had a tough run with injury, and had persisted for some time through a number of painful shoulder injuries, noticeably stressed by the situation).


  %E: Any sanctity around the team appears secondary to more immediate strategic goals

  %P: Power of regulation:
  %F: BYH, SHL, and WCY privileged position: contrast with HXL and LP (and ZPH).
  %E:Setting up regulations in institutions to incentivise and constrain, but not something that generates trust in the category of the institution itself.

  %F: ZPH --> WCY transition:
     %WWX criticism of ZPH; subsequent alignment with WCY & ZJ


      \subsubsection{Social norms of group membership}

  I also observer that the dominance of a relational mode of social interaction also played out strongly when athletes were in situations in which group norms of the team had to be navigated.  Although the categories of team and the overtures of egalitarianism and self-sacrifice that came with that notion in the sport of rugby in particular were present and salient in the discourse of coaches and athletes, it gradually became obvious to me that the hierarchical structure of relationships between team members were the focal point of athletes' attention.  Cultural psychology predicts that individuals from East Asian cultures will tend to attend to the harmonisation of the group as the core social priority \citep{Yuki2003}.  This prediction easily leads to an assumption that East Asian subjects will be less traditionally self-focussed than Western European or North American subjects.  Interestingly, while I did notice a dominant attention to the harmonisation of relationships, the strategies utilised to harmonise group relations were not limited to self-less acts.  Indeed, I noticed that promotion of the self in-group situations was, somewhat paradoxically, a key strategy of group membership.  I will attempt to explain this point through ethnographic observations below.


  \myparagraph{Team as family, teammates as brothers}

  The concept of ``family'' was central to the rugby program's public discourses surrounding social norms of group membership. From vice-principal Jenny, through to the head coaches, and the athletes themselves, family was the metaphor most commonly used to convey the requirements of each athlete in-group.  The metaphor of family contained both the solidarity and emotional support of shared membership, as well as a justification for the hierarchical structure of the group.

  As a general rule, when directly addressing elder athletes, an athlete would add the suffix ``\textit{ge}'' (哥) to the elder athlete's first or last name, to indicate that the elder athlete was relationally equivalent to an elder brother.  When addressing Han Xiaolong, the most senior athlete in the group, for example, athletes would commonly use \textit{Long Ge} (龙哥).  When referring to the coach, on the other hand, athletes would use either the formal and respectful ``Teacher'' (\textit{Laoshi} 老师), ``Coach'' (\textit{jiaolian} 教练), or the more colloquial suffix abbreviation of \textit{dao}}. The conventions for the coach did not therefore have an explicitly familial structure, and were based more on Confucian traditions of master-apprentice relationships. When referring to a teammate in the third person, athletes would often refer to them as either their senior apprentice \textit{Shige} (师哥) or junior apprentice \textit{Shidi} (师弟), depending on whether the teammate was older or younger.

  This system of naming conventions, built around the hierarchical familial and Confucian master-apprentice relationships, is one example of the ways in which social norms of the group were modelled and reproduced.  The concept of ``team'' (\textit{tuandui} 团队) was also highly salient in group discourse. And indeed, as per the traditional discourses that frame rugby as a character-building team sport, ``team spirit'' (\textit{tuandui jingshen} 团队精神) and ``team work'' (\textit{tuandui peihe} 团队配合) were commonly evoked as ideal ethics of group membership.   Over time, however, it became clear to me that athletes' understanding of the concept of team was closely aligned with the concept of family.

  In an official team meeting approximately two months into my first stint of fieldwork, vice principal Jenny provided the most elegant and articulate example of discursively framing the group as a family.

  \begin{quotation}
  ...I very much welcome everyone to join the Institute's rugby team. I also hope that in study, living, and training, everyone will treat each other like their own siblings, with mutual tolerance and understanding.  I hope you are able to create a very good team atmosphere, and it is by no means easy, because everyone comes from all corners of the world. Everyone is an independent individual, more or less autonomous, prone to be relatively selfish, and may also have poor self-management skills.  But this all changes when everyone comes to this big family, because rugby is a team sport. In a team sport, everyone in the team needs to be twisted together into one single rope in order for that rope to send out a force. I don't wish to see cliques of twos and threes; or to take the field with seven hearts, or five hearts---rugby doesn't work like that. I wish that everyone shares only one heart. Only in this way will our team be able to achieve good results
  \end{quotation}


  \begin{quotation}
    ...非常欢迎大家加入橄榄球的队伍,也希望在以后学习生活训练过程中希望大家像亲兄弟姐妹一样,互相宽容互相帮助互相体谅,能够形容一个非常好的气氛,因为大家来自五湖四海,非常不容易。每个人都是独立的个体,多多少少都比较自主、比较自私、可能自我管理能力较差。来到这个大家庭之后,因为橄榄球是个团体的项目、团队的项目,需要大家队伍要拧成一股绳,发出一股力。我不希望大家三个一群两个一伙,上场之后七条心、五条心、不是这样。我只希望大家一条心。只有这样我们队伍才能打出好成绩。
  \end{quotation}



  The concept of family offered dual affordances for processes of group membership.  On the one hand, the family represents a group in which every member contributes to and receives collective strength and support, by virtue of the solidarity achieved through group membership.  On the other hand, the family, and indeed, the team,  entails hierarchy.  In the formal setting of the team meeting, Jenny emphasised the former dimension of the family.  But it was clear from behaviours in less formal settings that the family not only provided a concept for group solidarity, but it   also afforded the devotion of attention to fostering a hierarchical network of relationships.  The concept of team and its associated ethics were insufficient as  affordances for explicitly representing hierarchical relationism.

  \myparagraph{The coach is King...or Emperor?}
  Not considering the Institute leadership, the coach sits clearly at the top of the apex of the group, followed by the senior athletes (Han and Lu - the old guard were also transitioning coaches), and then from there in order of age and seniority.  When conducting the post-interview activities with Feng Yang, a young athlete who had been enticed away from the Shanghai team to Beijing by the promise of attending BSU, he asked me why I didn't have ``For your Coach'' as one of the motivation categories for rugby:  ``Is there a Coach option in here?  I think there should be a Coach option!'' (这里没有教练吗?我觉得应该有个教练!)

  Admittedly, I had not originally considered this as a core category of athlete motivation, mainly because at the time I was not as attentive to the hierarchical structure of the group, and the possibility that the coach would be singularly identified as a motivator of adherence to rugby (in distinction to the team more generally).  But the series of semi-structured interviews with athletes indeed demonstrated the centrality of the agency of coaches in their trajectories.

  All athletes who had transitioned from a different sport previously (17) cited their previous coach as the agent of the transition: their coach introduced them to the Institute or facilitated a trial with the Institute, or something similar.  It was clear that these coaches were all important figures in their lives. As explained in the section above, the head coach of sporting programs is a key coordinator of opportunities and resources for athletes, and therefore it is understandable that the coach would be a central focus of social attention.

  As I became more aware of the prominence of the coach in the team hierarchy, I began to notice instances in which social norms concerning the coach were enacted.  I first began to notice the stature of head coach Zhu during meal times.  I noticed that Zhu and Shi had developed a habit of arriving at meals at least 15-20 minutes after all of the senior athletes who ate at the 1st Level canteen.  This timing meant that athletes were more or less finished eating by the time that Zhu and Shi had served their own meal from the buffet trays and prepared to sit down.  At this point, many of the senior athletes in the canteen would hastily finish their meals and leave the table before Zhu and Shi arrived (each sport program at the Institute was assigned a set number of circular tables at which to sit and eat, thus the coaches invariably ate with the athletes). At the sight of Zhu and Shi entering the canteen, some of the athletes, commonly Wei Wenxin and Wang Wei, would quietly cry out under their breaths something like ``Quick, Old Zhu is coming!  Let's go let's go!'' (快点,老朱来了,走吧走吧!) and attempt to leave without having to interact with the coaches.  I noticed that members of the Old Guard, particularly Han and Lu (who were technically speaking coaches themselves) often ignored the urgency of departing before the coaches arrived, and would stay and keep the coaches company. Other times, if they had well and truly finished their meals or had somewhere else to be, they would join the other senior athletes in escaping and clearing their trays in the canteen's kitchen.

  I later asked Han why athletes were so keen to escape, and he suggested that if they were still present at the table when Zhu and Shi arrived to eat, then they would feel strong social obligation to accompany the coaches through the full course of their meal, and only leave once the coaches had finished eating.  On some occasions, the Zhu and Shi would arrive slightly earlier, before some athletes were able to finish their own meals.  In these instances, I noticed a combination of strategies, all of which involved seeking permission or receiving permission from Zhu to leave the table.  Zhu would often notice when athletes were finished their meals and would actively grant permission to the athletes (吃完了你可以走吧).  Often athletes would wait for this permission rather than asking for it themselves.  Han explained to me that Zhu and Shi had deliberately developed a habit of arriving late to meals so as to reduce the frequency of scenarios in which this social tension arose.  It was clear that both parties perhaps wished not to have to engage in this social ritual of respect to authority of the coach, but were unable to ignore it.

  Indeed, I soon discovered that I was also afforded this treatment as a coach, although I could tell that because I didn't feature as centrally in the orbit of relations in the same way as the head coach, the social charge of my presence in these situations was much less intense.  Athletes were happier to freely announce to me that they had finished, and I awkwardly insisted that they leave.  When Wang became head coach, the tension in these situations was drastically reduced. Wang after all had himself been an athlete and played alongside many of the athletes who he was now eating with.  He usually chose to arrive at around the same time as the other athletes.  Even though athletes were more comfortable with Wang than with Zhu, they still had to enact the ritual of asking for leave if they had finished before Wang.  This convention around meals is indicative of the importance of hierarchical relations in structuring social attention of group members.

  \myparagraph{The ``I'' in team}
  The clear social authority of the coach over athletes was understandable and in line with existing research on hierarchical relationalism.  What I was less prepared for were instances of self-promotion in social context.  These behaviours appeared odd to me, particularly given my egalitarian impulses.  I started to notice this in interviews when I would ask about each athletes' journey to rugby.  Although athletes would recognise in passing the agency of their coach or close relation who facilitated their journey to the Institute, when I asked whose final decision it was to play rugby, almost all athletes would emphasise that coming to the Institute was ultimately their own personal decision.  Often this personal agency was made in distinction to their parent's will.  Senior athlete, Ma Haitao, after explaining how he had been introduced to rugby via one of his family friends (who had attended CAU), explained how he came to the decision:

  \begin{quotation}
    My Father didn't know this sport.  Because in China very rarely will people know what this (rugby) is, so my Parents didn't know. From a young age I've been away from home doing sport, and so generally I decide these things...I made the decision myself.  At the time I arrived wheeling a suitcase and had a bag on my back.  I came at the time with a classmate...
  \end{quotation}


    \begin{quotation}
      我爸不知道这个项目。因为中国很少知道有这个,他们不知道。我从小就在外面练体育,所以我一般是自己决定这些东西。 (2) :自己决定的。当时自己背个包投的箱子来。当时和一个同学,他是农大的。
    \end{quotation}


  A junior aspiring athlete from Beijing's outer suburbs was even more proud of his decision to play rugby:
  \begin{quotation}
    My parents at the time thought [rugby] was quite dangerous, and they didn't let me play.  But they weren't as stubborn as I was.  It's the case with everything, I decide...If I to do something I have to make the decision myself, and then no one can change my decision
  \end{quotation}


    \begin{quotation}
      父母当时觉得很危险,不让我练。但是拗不过我,他什么事都。我自己决定了...
      就是说我拿什么我一定拿的主意,任何人改变不了...
    \end{quotation}



  Senior athlete and CAU graduate from Dandong in Liaoning province also had a similar story about self-determination:

  \begin{quotation}
    I decided myself. How I came to the decision was actually quite interesting.  At the time when we first came in it was really difficult.  At the time fourty or more of us were selected, but many people wouldn't let us train, including our head teacher at high school.  A lot of class teachers said to me that there was no hope for rugby, rugby in China has no development [potential], and there are barely any universities that you can attend [through their rugby programs].  And so then they wouldn't let us (the students who selected rugby) train, only us! After that I said to my Mum and Dad that I really liked this sport, and then my parents also extremely supported me with this.  When I got in to CAU (through their rugby program) my parents were extremely supportive.
  \end{quotation}


    \begin{quotation}
      我自己做的决定...决定特别有意思。当时我们进来的时候特别艰难,当时我们选了四十多个人,好多人不让我们练,包括我们的班主任,好多授课老师跟我说没什么希望,橄榄球在中国没有什么发展, 也没有几所大学能上。然后不让我们练,就——我们! 后来我就跟我爸妈说我特别喜欢这个项目,然后我爸妈也非常支持我,我靠农大的时候他们非常支持我。
    \end{quotation}


  %Johnny Zhang's kids
  The emphasis on the virtue of self-determination and self-reliance was powerfully illustrated to me outside the context of the Institute, when I agreed to take a rugby session with a group of young school children (6-8yrs) for a CAU coach Johnny Zhang.  The training session was designed to be explicitly educational, and so I felt that I should attempt to tie some of the elements of team work in rugby to virtues of friendship.  Towards the end of the session, when trying to emphasise the importance of teamwork in a team relay drill, I asked the children: ``Who do you rely on most in this life?'' (你在生活中最依赖谁?)(obviously I was hoping to lead them to the answer that they had to rely most on their family and friends for support, what I thought was a relatively obvious response in either Confucian or Anglo-Protestant ethics!).  To my surprise, a chorus of two or three more outspoken children responded: ``Myself!'' (自己!) This was not the reply I was looking for. But it did say something about the emphasis on self-reliance and self-cultivation as an important virtue of group membership.

  %PHOTO: Johnny Zhang's kids

  %Here in this statement is no real feeling that the kids should be taking responsibility for their behaviour, and on the contrary, there is almost like an acceptance of the fact that the kids are rowdy, and that the only way to address that problem is to perform the role of herding them by yelling and screaming and shouting.

  The active cultivation of the self was a recurrent theme in interviews with athletes when asking about their role in the team and how they ought best contribute to the team.

  HSQ: self-cultivation for the benefit of the group
  :有一点,但是我觉得我毕竟是主要还是想把自己做好,做好自己的基础上为再考虑大家。
  MXK:
  :现在我感觉主要的动力就是想证明自己,给现在的教练看,因为我觉得现在的教练对我不太认可,所以我现在最大的目标就是想让他们认可我
  FC:
  ?你觉得你现在的位置是什么?现在需要注意什么?:我没想那么多,就想自己把球打好,自己好好练


  % HXL at team dinner
  The importance of self-assertion and promotion within the context of the team was performed at a team dinner held on a Sunday night (when everyone had returned from leave) a few weeks after Wang took the reigns as head coach.  This team dinner, like the team dinners that I had previously attended, was held at a Shandong restaurant a 10 minute walk south-west of the Institute's Main Gate.  It was the first team ritual orchestrated by Wang, and as such Wang initiated a ceremony of organised drinking and speeches.  I entered late and noticed that many of the junior athletes appeared nervous and up tight. Almost 26 people were crowded around a large (but not that large) banquet table.  Wang sat at the equivalent of the ``head'' of the circular table, the position opposite the entrance to the private dining room.  Either side of Wang sat Han and Lu (the assistant coaches) and then the senior athletes in order of seniority, all the way until the opposite side of the table, where the most junior athletes were positioned.

  Wang began the proceedings with three toasts (customary as the host), and then from there the senior athletes spoke in turn.  After each toast, the entire team had to drink a portion of their rice wine (and then beer after the rice wine had been drunk).  After the senior athletes had all had a chance to talk, the coaches directed each vague cohort of athletes (i.e., the very youngest athlete on trial, the aspiring Chaoyang students, the unruly undergraduates, and then the remaining senior team) to stand up and say a few sentences and then drink to the coaching staff (and senior athletes).

  The proceedings were relatively rigid to begin with, and from my position to the side of the table, the scene resembled an awkward face-off between the senior athletes and coaches at one end of the table, and junior athletes on the other end.  But once the alcohol kicked in, social interaction began to flourish. Speeches became bold and fluid with both self-assertive pronouncements and self-critical admissions, and the spaces between formal toasts were filled with one-on-one sideline conversations between athletes and coaches, junior and senior intermingled.

  Han's first speech, which came after Wang's three toasts, was one of the most bold and noteworthy:

  \begin{quotation}
    I don’t know if all of you have your own personal goal?  I at least know that I have a goal, and that is to win next year’s National Games gold medal.  That is what motivates me; that is what I am sacrificing for.  I don’t know exactly what your goals are, but if you don't have a goal then your (personal) goal should be to help me achieve my goal.  I toast to everyone helping me achieve my dream of winning a National Games gold medal!
  \end{quotation}


    \begin{quotation}
        我不知道你们都有自己的目标吗? 我至少知道我有一个目标,那就是拿到明年全运会的金牌。 这就是我的动力,就是我所牺牲的。 我不确定你的目标是什么,但如果你没有目标,那你的目标应该是帮助我实现我的目标。 我向所有人致敬,帮助我实现自己的梦想(赢得全国运动会金牌!
    \end{quotation}


  I was once again automatically struck by this bold speech.  Han was the most senior athlete in the team, he was the former team captain and was now in transition to coach.  I suspect that the intuitions around team membership that I had generated in rugby contexts elsewhere prepared me for a speech in which Han would take responsibility for curating an egalitarian idea of group membership.  ``We are all equal under the category of the team,'' he ought to say; we are all in this together.   Instead, in his prime-time team captain talk, Han was intuitively compelled to demonstrate the strength of his own personal conviction in order to generate solidarity. Han's speech appeared to me on the surface of things to be deeply self-promotional act (at the expense of an egalitarian ethic of equal access to the resources of team membership). But on another level, Han's speech could be interpreted as being deeply prosocial, as it expressed the strength of his motivation as a leader of the team. On this level, Han's speech expressed that he was upholding his part in the system of hierarchically structured relationship that made up the Beijing men's rugby team.  In this way, Han's dream was the group's dream, and Han's prosociality needed not be mediated through reference to the category of team and an egalitarian membership to that category.

  Later in the evening, as the speeches progressed around the table, I did hear some of the more junior senior players make reference to team ideals.   CAU graduate Pan Qiyu suggested in his speech that ``we all need to put ourselves after the team; this is a team sport'' (我们都有把自己放在团队之下); and old head Wang Wei: ``this is a team sport; team first, self second'' (这是个团队项目。团队第一,自己第二)。 Importantly, however, these assertions were not egalitarian in their essence, and instead suggested an individual subservience to the team, rather than an egalitarian, shared ownership of the team by each of its members (a more common notion expressed in Western team sport contexts that I have been a part of previously).  Indeed, when Pan and Wang were referring to the team, it could be assumed that what they had in mind was a relational network of individuals structured in a fashion that mirrored the seating arrangement. On one level, the team was united together as equals: all of its members sat at a round table in which everyone had a seat. But on another level this circle was imbued with a hierarchical structure through which relations of power that flowed from the coach at one end to the most tenuous and junior athlete sat nervously awaiting orders at the other end.

  While these two levels of social interaction could conceivably be identified as mutually co-occurring in equivalent settings across human cultures and throughout time, it is important to notice that this specific cultural milieu preferenced attention to the second, relational level as the dominant mode of social activity.  On this relational level of social interaction, categories of ``team'' and notions of team membership were present and salient in social discourse, but these categories appeared to function not as direct mechanisms for psychologically mooring individual identity to group identity (as social identity theory would have it), but instead they functioned as resources for reproducing a hierarchical network of relationships in which each individual was defined and visible only in his or her relations to others in that network \cite{Yuki2003}.

  %For discussion?
  The theory of social identification suggests that for social identity to become prominent and meaningful for an individual, self identity (momentarily) supressed.  Social identification thus works in a hydraulic fashion, such that when one category of identity (i.e. the group) is promoted, the other category (the self) retreats \citep{Swann2009}.  Identity fusion, on the other hand, describes a situation in which both social and self identity are fused, such that when the group category is salient so too is the category of the self; I am strong when the group is strong.  Existing research relating to social identification theory and identity fusion suggests that identity fusion represents an extreme pro-social manifestation of group membership, responsible for extreme self-sacrifice on behalf of the group.  My observations with the Beijing men's team suggested that identity fusion was almost instead a default position, owing to the dominant (relational) mode in which social interaction was processed.  Each individual, in order to survive in the group environment was always already fused by virtue of the group norms to which he or she was required to adhere.

  The point here is that what might be crucial for processes of social cohesion might not necessarily be the quantity that is measured through identity fusion constructs. Athletes in this context would likely declare that they are always already fused, and thus fusion would probably not provide an accurate read of the in-group variation in social cohesion. Instead, what do vary in this group setting are mechanisms related to technical competence, and the quality of cognitive processes relating to joint action.  What does vary is an individual's capacity to ``click'' with another, based on individual and collective competence generated through on-field action and off-field social interaction.

  The fact that bonding and emotional commitment can be achieved and expressed in a predominant and chronic relational mode challenges out understanding of what bonding is.  Imagine it to be an association or affiliation with an abstract category of the group - fusion, while it improves on self-categorisation theory, still contains the inherently categorical mode of association...



  \myparagraph{The entire system must be aligned}
  I encountered a number of instances in which it was clear that the categories of self and team were not distinct, and were instead appeared largely porous and secondary to relational concerns. I observe athletes who appear to rely on relational explanations for explaining social interaction and their agency in these interactions.
  SIGNIFICANCE:  These instances show that the hierarchical relationism dominates attention and interaction in processes of group formation.

  One example occurred the first morning I arrived back for the second major stint of ethnography in the summer of 2016. I arrived back to a terrible incident in which some of the rugby team had been effectively poisoned accidently by pesticide sprayed on the hedges outside their dormitory windows.  A few days before I arrived back, the groundskeepers had carelessly sprayed pesticide on the hedges outside the windows of the bottom-floor dormitory rooms of which the rugby team were inhabitants.  Some of the pesticide had made it in to the rooms of the athletes, and had caused a few of the athletes to develop throat irritations coughs.  Wei Wenxin, the new team captain replacing the promoted Han Xiaolong, was one of the most seriously affected by the pesticide, and when I sat down with him at breakfast the first day I returned, he explained the story with deep anger and outrage.  Importantly, when I asked him what was going to happen with all of this, he said to me ``the Leadership will have to say something on this, we are all waiting for how Leadership to respond.'' (领导,我们在看领导怎么说). While the role of political leadership in an organisation such as the Institute was no doubt crucially important, I found it interesting that Wei was so emphatic about the role Leadership ought to play in the matter.

  During my second extended period of ethnographic research, which was well into the transition from head coach Zhu to head coach Wang, I shared a dormitory room with head Coach Wang's offsider, assistant coach Zhu Jing.  Zhu Jing had recently arrived from the rugby program at Xingjiang province, after he was invited by Wang to join him as assistant coach.  Zhu was needed to help provide support to Wang as well as lighten the load on Han and Lu, who despite transitioning to the official position of coach, were still required to train and play, at least until the National Games in 2017.

  Zhu Jing was a new arrival to the team, and was thus on unstable ground - at the time he had no official contractual relationship with the Institute.  Zhu was neither a particularly celebrated rugby player during his time playing, nor was he a particularly experienced or successful coach.  What he did have, however, was a close relational connection to Wang (they were classmates at CAU), and thus, presumably, he possessed the valuable virtue of loyalty---useful to Wang in his attempt to stabilise his leadership of the team.  In my eyes, Zhu had everything to prove in terms of his value to the team and the Institute.  If I were in his position, I thought, I would have done everything possible to help head coach Wang and the Team to improve performance.  As part of this, no doubt, I would seek to signal diligence and commitment in the way I went about my business as coach.

  After about two or three weeks into our roommate relationship, it became clear to me that Zhu Jing was not so concerned with signalling diligence, at least not to me, and not to the athletes either, at least as far as I could tell.  Most mornings I would wake up at 0700 hrs. to attend breakfast with the athletes and then collate my field notes and prepare for training starting at 0900. Zhu on the either hand would reliably sleep through until roughly 0850 each morning, and then turn up to training usually at around 0920 after athletes had completed their warm up.  After lunch, Zhu would return to bed, usually from 1300 until 1500-1530, depending on when afternoon training was scheduled. Admittedly, afternoon siestas are an institutional part of China's work life, but routine indulgence in a plus two-hour siesta was surely taking the siesta to its extreme.  I rarely witnessed Zhu spend any time working on preparing training schedules or other forms of professional development.  As it turned out, became clear that his former classmate also noticed Zhu's approach to his job, head coach Wang.

  After the final Tournament in Qian An in July 2016, the team went out together for dinner.  This dinner was much less ritualised, and in fact the junior athletes were in one room, and the senior athletes and coaches were in another room, free to socialise freely within their more natural social factions.  Towards the end of the evening, after the effects of alcohol had well and truly set in, a heated discussion developed between head coach Wang and his assistant Zhu. The discussion began with one of Wang's numerous toasts to the group at the table. Wang talked about the importance of the rugby program differentiating itself as an excellent team at the university, and the need for all senior members of the team (seated at the table) to contribute to this project.  It became clear that Wang wished to emphasise the need for Zhu to ``lift his game'' in this regard. Wang explained that as head coach he was burdened with a lot of administrative work that detracted from his ability to manage athletes and training.  Given that the team was without a dedicated manager, he suggested that Zhu ought to improve his contribution. ``As an assistant coach, you need to conduct yourself at a high standard'' (当助理教练你要作为一个高度) said Wang directly to Zhu, indicating that his current standard was not acceptable.  Zhu retorted, directing attention away from his personal standards to the global situation of the team at the Institute.  Zhu suggested that the most important thing was that the leadership of the program supported the program, and only then could the program achieve a high standard: ``you have to have the support of leadership'' (必须有领导的支持).  Lu and Han, the next most senior team members present, also became involved in the discussion, with Lu siding more with Zhu and Han siding more with Wang (as far as I could remember).

  The next morning, as both Zhu and I lay in our beds hung-over after the team dinner the night before, he insisted Zhu revived the discussion with me, seeking my support.   Zhu insisted that his ability as an individual was fundamentally limited if the Leadership did not appear to offer him support.  If the team does not have the faith and support from the Institute (which in this case, it clearly did not have, at least since 2013), then how was Zhu supposed to perform his role? My deep-seated ``Protestant ethic'' raged within my unstable stomach and I couldn't help but challenge him on this assertion.  After all, I thought that the clear way to counteract lack of support from above was to demonstrate self-competence, to signal diligence and willingness to move independently, despite or in spite of the level of support form leadership.

  As I looked at the state of the team, there were clear problems in the team that needed addressing by someone like Zhu.  There was an apparent lack of clean and clear team discipline, a lack of focus, intensity, and technical precision during training, a lack of physical fitness necessary to survive the challenges of a high level rugby tournament problems, and so on.  These were all things within Zhu's remit, I thought.  In this chicken and egg scenario, I thought that the team was a chicken with the capacity to independently lay an egg that would change the Institute's view of the rugby program.

  But Zhu pushed back:

  \begin{quotation}
    I have experienced this with Xinjiang during the last National Games. The Leadership started off being very positive about rugby, promising this and that, but then in the end, nothing came through.  The support of leadership is hugely important in China: money, incentives, all of it.  If you don't have support from Leadership then you can't get anywhere.
  \end{quotation}

  Zhu was obviously defensive about the situation, attempting to deflect any blame or responsibility for his action (or lack thereof).

  But perhaps his claims weren't completely unreasonable in the specific environment---an environment in which hierarchical relationism was a dominant mode of social attention.  In a way, Zhu was referring to the importance of the harmony of the whole system of social relations in which he was embedded as but one node. Individuals can only act with agency if they are in an environment that supports that agency, suggested Zhu. In the hierarchical system of the Institute, the crucial node in the system is the paternal benevolence of Leadership.  I couldn't help but connect Zhu's stance to the stance I had received from Wei Wenxin a few weeks earlier when I first arrived back to the Institute for my second stint of ethnography: ``We are waiting for the Leadership to act on this.''

  \myparagraph{``Its all very complicated (in China)''}
  Many of the athletes and coaches with whom I would informally converse would insist that China's social interaction was ``too complex'' (太复杂了), to a degree that I---a Western/outside observer---would struggle to fathom.  While I didn't grasp the meaning of this at the time, I feel as though what my interlocutors were referring to could be related to the attention devoted to fostering and harmonising a relational network in which the individual is understood to be a single node rather than an autonomous, bounded entity. One day at breakfast, for example, Han started dreaming:

  \begin{quotation}
    It would be great to live in the West, wouldn’t it Li Jie?  An hour’s work is an hour’s pay.  Not like China.  What’s it like it in China?  `You’re mine! (says your boss) You have to do whatever I say.'
  \end{quotation}


  \begin{quotation}
    在西方生活多好,多吧,李杰?一份货是一分钱。中国就不是了。中国是什么呢?你就是我的人! 你要听我的!
  \end{quotation}



  The act of comparing China and ``overseas'' (\textit{guowai} 国外) or ``the West'' (\textit{xifang} 西方) became a common theme when I would discuss issues with predominantly senior athletes and coaches.
  ``Long live Chairman Mao'' I said jokingly one day during my first stint of ethnography as I caught up to the group of senior athletes headed to the canteen for lunch.  ``Long live CCP'' Han replied with a smile and developing the theme.  Ma Haitao then turned to me and asked: ``Do you all in the West have that type of person'' referring to Mao.  ``Well, not recently, but in the West people have always believed in religion --- this question you asked is really a question about what people put their faith in, aren't you?'' (这个问题实际上是人家的信仰,对吧?) I said.  ``Yes, but I think in China people only believe in this'' (对对对,但是在中国人只信钱) he said, as he rubbed his thumb, index, and middle fingers together---the universal sign language for money. ``Chinese have too much faith in money.'' (中国人太信钱了).

  When I asked senior athletes about athletes in China, particularly in relation to team sports, I received similar tropes of general  criticism about the deficiencies in the Chinese psyche.  One day I was talking with Han and Lu on the sidelines of training, just after Lu had returned from representing the Chinese national team at two international tournaments. It was the first time I had explained the cover story for my research to Lu.  I told him that I was studying team membership and cooperation, and said that I was trying to use some theories from Western social science, but that I wasn't sure how well they were holding in my observations in China. ``Of course'' said Lu, ``That's because Chinese people are too smart (for your theories)!''  (当然,因为中国人太聪明了).  From this quip we then started talking earnestly and at some depth about some of the issues with rugby in China (Lu had just been away with the national team, and had plenty of complaints).  As a former Australian professional rugby player and therefore default representative of ``best practice'' in rugby, I was an obvious person to field these complaints. Lu explained that the national men's team faced issues that were structurally similar to the women's team, relating to payment, medical care, and training methods (mentioned in the opening vignette in this chapter, HYPERLINK).  For example, Lu suggested that Chinese coaches, as a general rule, don't believe athletes when they say they have an injury, and instead force them to train through an injury without a concern for regulating training loads. I couldn't help but retort with the obvious return of Lu's original serve ``But Chinese people are too smart, aren't they?'' as an obvious joke.  Han then chimed in with a sigh: ``the word `smart' is a word with derogatory connotations'' (贬义词), suggesting that Lu's original use of the word smart was actually a suggestion that there is a crafty, sly or cunning (smart) dimension to the Chinese psyche that leads to problems such as those identifiable in Chinese sport and rugby in particular.

  Han continued this theme one Saturday night in the dormitory institute, when I joined a few athletes who had not taken leave over the weekend. Senior athletes Han, Cui, and Wang Zhenfeng and their junior athlete Men Cheng, a Beijing local who was still affiliated with Chaoyang sports high school, had got together to cook a hot pot meal in their dormitory room.  We were talking about relationships, and we got back on to the topic of comparing China with the West.  Han explained to me his view that:

  \begin{quotation}
    Chinese people are experts at tricking, they even trick themselves: in the West, if a man cheats on his girlfriend he will regret it and feel that he has mistreated her.  But in China, men trick themselves into thinking that nothing is wrong.
  \end{quotation}


  Junior athlete Meng Cheng was emboldened by the rice wine that we had been drinking and was noticeably eager to contribute his opinion to the discussion:

  \begin{quotation}
    It's a type of belief. From a young age, you are taught to believe it, in the same way that we grow up thinking that having a son is better than having a daughter.
  \end{quotation}

  EXPLAIN/SUMMARISE:

  %HXL: return to his period of absence from the team after final tournament.
  %\myparagraph{The ``team'' in hierarchical relationalism}
  %Jing's speech, official
  %Example: Last fitness session ``一团一团!''
  %Performed for me, for the cameras...




  \subsubsection{Action-Perception}
  In this section I present evidence for culturally specific terrain as it is observable on the level of action and perception.  Most of these observations are derived from my time observing and leading training sessions with the athletes. In particular, I notice evidence of porous boundaries between individual athletes, and instead responsibility for the creation of action and perception in joint action appears to be distributed among many individuals.  In the team context, it appears that the structure and direction in joint action and attention relies heavily on the agency of coaches and senior athletes.

  One of the most stressful experiences for me during fieldwork at the Institute was navigating the traffic of athletes and coaches every mealtime at the canteen.  Meals were served in buffet style at a long table in the centre of the canteen. Upon entry to the canteen, one would pick up a metal meal tray, bowl, and a set of chopsticks and choose a place to start self-serving foot from the buffet.  I found it extremely difficult to predict where people were trying to move to, particularly when the crowds were at their height at the beginning of lunch or dinner. I also had trouble at times navigating the unwritten rules of the traffic on the running track at the Institute.  Sometimes I would also have the same jarring experience when walking in a crowded street in Beijing. On some basic level, my intuitions for action and perception did not quite line up with my environment.  As I would soon discover, the environment at the Institute would also challenge my intuitions for action and perception in the game of rugby.

  \myparagraph{Consistent factions, on and off the field}
  After a few weeks of observing training and a few sessions in which I myself took the team for training, the different factions in the team became quite clearly expressed on the field.  The new arrivals and athletes on trial were either on the sidelines learning and practicing the basics, or very tentatively beginning to regular join team drills. Above the athletes on trial, the young hopefuls from Chaoyang had begun to get a hang of the basics of the game and could more or less hold their own in most drills, although tended to lack ``awareness'' in more complex team drills. The young hopefuls could not be faulted for their effort, more for their lack of experience. Then, the ability of the unruly undergrads varied by individual, however there was a general consensus among coaches and senior athletes that the undergrads suffered from a lack of motivation and responsibility, and that this lack played out on the field.  Finally, the senior athletes were clearly a level above the rest, despite each individual athlete possessing his own strengths and weaknesses.  Among them, Lu and Han were clearly the most experienced.

  Given the relatively clear stratification of abilities, the team had become accustomed to grouping in drills such that athletes of similar abilities almost always trained together in same groups, as opposed athletes randomly assorting into new subunits for each training drill (as was normal practice in rugby training sessions I had become accustomed to outside of China). In a basic passing drill involving 4 athletes per sub unit (in which athletes would pass the ball between each other while running up and down the field), for example, usually the most senior four athletes present would naturally form the first sub-unit, followed by the next most senior (or competent), and so on, until the most junior and least competent/experienced athletes would form a sub unit and attempt to emulate the preceding sub units.  When talking to senior players about this custom in training, I was told that the habit was necessary for two factors. First, given that the differential in ability between the most senior and most junior athlete was so large, sorting according to seniority and competence (as opposed to randomly) ensured that the senior athletes could ensure the quality of their practice.  Second, by sorting together in sub units and doing the drill first, senior athletes could demonstrate the ideal standard of drill execution for the rest of the team to emulate.  Although from a rugby point of view, I worried at times about the way in which these customs would limit the diversity of joint action combinations between team members, I did feel like this reasoning was justified based on the constraints that the team faced in terms of marked variation in technical competence.

  When talking to the senior athletes and coaches about the quality of training (or lack thereof), the most problematised of all the team factions was the group of unruly undergrads.  The basic conclusion was that the undergrads lack initiative and focus; they were too distracted by university, computer games, and pursuing romantic love. Indeed, having achieved the core goal of gaining access to university, the unruly undergrads were criticised for not possessing sufficient intrinsic motivation to dedicate them to rugby.

  \myparagraph{}

  After a few weeks of observing training I began to take my own sessions with the team, focussing on defence and ``contact'' work, and team play.\footnote{Contact refers to skills of rugby that involve body-on-body physical contact in addition to defence.}  After taking training for approximately two weeks, one day I took a session in which, for the first time, almost all the senior athletes were unable to train, either due to injury or unavailability.  The impact on the quality of training was palpable.  The junior athletes who were present were passive and unresponsive to instructions, and unable to direct the athletes below them.

  Over time, a clear pattern emerged, in which the quality of training was almost directly correlated with participation of senior athletes. Without senior athletes helping direct training, the group---particularly the younger athletes---was generally slower to react to new information.  Then following this, more senior athletes would usually become actively irritated by the poor quality of training, and begin to openly criticise the junior athletes.  This point might be obvious and predictable, given the stark within-group variation in technical competence.  I could not help but sense, however, that, even when controlling for variation in competence within the team, there was something related to dominant social mode of hierarchical relationism that was also driving the relationship.

  One day following a training session that I took, in which many senior athletes were absent,  I walked back from the training pitch to the dormitory with assistant coach Shi and Han, and we began to discuss the issue. Coach Shi postured his opinion on the situation:

  \begin{quotation}
    Some athletes think its ok to just be an athlete here and get to BSU, and not pursue anything higher or anything more.  Actually, it's ok to think like that. But if that's the case then sorry, you'll have to watch from the sidelines.
  \end{quotation}


  \begin{quotation}
    有的球员觉得好在这当运动员上体大就可以了,不去追求更高的更多的。其实,可以这样想,但是,对不起,你要靠边看着。
  \end{quotation}



  Han explained that he could see the reasons for this tendency for more junior athletes to lose focus, but insisted that the undergrads were becoming habitually un-focussed during training:

  \begin{quotation}
    It is becoming a habit.  When we were at CAU at that time during university, we would also lack focus at times and so on, but back then, as soon as coach Zheng would take training, we would all be immediately focussed!
  \end{quotation}


  \begin{quotation}
      成为一种惯性,我们在农大那个时候也会不关注等等,但是那个时候郑老师一带我们就全在关注
  \end{quotation}



  Shi's take on the issue provided an explanation for a lack of focus at the more general level (i.e. on the level of general motivation for adherence to rugby).  Han's suggestion, on the other hand, offers a more proximate explanation, i.e., that the make-up of the group at training, including coaching staff, has an impact on athlete's level of focus and attention.  Although it was conceivable that the structural incentives available for athletes would impact on general motivation for adherence to rugby and its collective technical and social challenges, in this section I am particularly interested in exploring the ways in which the culturally specific terrain of rugby in China could shape action and attention.  Considering existing evidence from cultural psychology concerning the distributed self and relational self-construal, for example, it is plausible to predict that in a setting in which hierarchical relationalism is most dominant, the presence of figures of authority could be crucial for maintaining the quality of attention and focus in joint action.

  \myparagraph{Joint action is distributed throughout at hierarchical network of actors}

  %lack of verbal communication and active signalling:
  One of my first observations of joint action in training was how rarely athletes pro-actively used vocal communication in order to facilitate joint action.  Admittedly, this was a habit that required explicit training, and not necessarily an innate intuition.  Given the structure of game play in rugby, in particular the fact that each team can only pass the rugby ball backwards from the position of the ball carrier, it means that both teams face off against each other in two lines (attack and defence), direction with only 180 degrees of vision. Vocal communication provides auditory information about the location of other athletes, helping broaden an athlete's awareness of the location and intentions of other athletes in a situation in which visual information is relatively scarce (SOURCE).  The lack of communication could be explained as simply a lack of adequate instruction.  But I couldn't help but notice that there was a relative ease with which athletes were able to coordinate action without excessive explicit vocal or other communication signals.  Considering the possibility that the boundaries of the self and group were perhaps more porous than in other rugby contexts in which I had encountered in the past, it was possible that the lack of communication could also be explained by the fact that individuals didn't feel that it was a priority to distinguish themselves as an entity separate from the joint action in which they were engaged.

  Panoramic view of joint action? Traffic.



  %Open scrutiny
  The inverse of the observable lack of communication in generating joint active was the observable wealth of active, unreserved scrutiny of each other's joint action once it had transpired.  Coaches and senior players predominantly drove this activity, and it was common to see both publically scrutinise the quality of athletes' joint action in a way that contained no politeness or concern for the way in which the athlete under scrutiny would receive the scrutiny. But at the same time, rarely did such open scrutiny contain any trace of malice or intent to harm or hurt the other athlete.  At times this process of open scrutiny during training would result in a conflict between two senior athletes of similar team stature, in situations in which the relationship was hierarchical, the scrutiny would be received without too much protest, and the training would continue.

  As an expert observer of rugby, I gradually developed the impression that both the observable comfort and ease in the lack of communication in generating joint action, and the relative comfort and ease associated with an excess of communication that analysed others' action, were indications that the culturally specific terrain of joint action in the Beijing men's rugby team entailed porous psychological boundaries between self and group.  Without strong boundaries between self, other, and group, there would be less attention directed towards the need to distinguish between self and other in joint action.  Secondly, without the same strong boundaries would enable open scrutiny between team members without the need to package this scrutiny in polite modes which serve to respect the boundaries between self and other.


  \myparagraph{The benevolent authority of joint action}

  My observation of and participation in training helped me understand that the way in which the coach---the linchpin of hierarchical network of social relations in the team---was also the coordinator of attention and focus in joint action.

  During my time coaching I occasionally felt as thought I stepped, somewhat uncomfortably, into the role of authority figure for athletes.  One day we were training in the indoor running track in order to avoid Beijing's winter smog, and so I directed the athletes to practice some passing drills in the confined space.  I was becoming frustrated with the level of commitment from some of the unruly passive undergrads, and on this day I ended up reacting to one athlete in particular, Fang Chao.  Fang Chao was participating in a passing drill in which he was standing in an offside position to catch the ball in front of the athlete passing the ball---a clear violation of one of the most basic rules of rugby (that you must pass the ball backwards, to an athlete in a position behind your own).  Fang Chao, as usual, was standing with a vacant look on his face going through the motions of the drill without any grasp of its purpose or attention to the details (let alone the basic detail of the need to pass the ball backwards according to the basic rules of the game).  I couldn't help but call him out on it: ``What the hell are you doing!?'' I asked, pointing at the fact that he was in front of the athlete passing the ball.  ``You're standing in front of him!'' One of the more senior athletes mumbled under his breath behind me: ``He just doesn't get it.''

  Coincidently I was scheduled to interview Fang the following day.  When I asked him what makes him feel social guilt or dissonance, he responded, citing our interaction at training the day before.

  \begin{quotation}
    FC: I feel as though I feel social guilt everyday! \\
    JT: What about yesterday, for example? \\
    FC: Yesterday I was dazed, because I felt that few days of training had been very tiring, so I couldn't concentrate...but I was actually very happy (that you called me out), when you yell at me I think that next time I will at least react better.  The main thing is that when I can't do something to the level that the coach requests, I feel very uncomfortable.  I will remember for next time, though.  I might not change immediately after the first time I get pulled up, but I will change, little by little... \\
  \end{quotation}


  \begin{quotation}
    FC: 我感觉我每天练得很内疚!
    JT: 比如昨天?昨天我骂你的时候么?
    FC: 当时我愣神儿了, 因为我觉得这两天训练比较累,注意力集中不起来。?你知道我在训练当中这样骂没有什么意思,:我知道,其实我特别高兴,您骂我我就觉得下次走神立马会反应过来。主要是我自己做不到教练说的要求,会感觉不舒服...我会记住下次,比如你第一次说我做得不好,可能一次改不好,一点一点会改好.
  \end{quotation}



  These interchanges with Fang could be taken as evidence to suggest that it is explicitly understood that the coach's role is to direct attention of athletes in joint action, and that without a coach or senior athlete performing this role, the attention and focus is not necessarily self-sustained.

  During an interview with senior athlete Lu Peng, we arrived at the conversation of the role of the coach in directing and coordinating action and perception of athletes, potentially to the detriment of performance in rugby. Lu reminisced on his time at CAU under the coaching of old Boss of the Beijing team and Chinese rugby, Zheng Hongjun.

  \begin{quotation}
    ...one mistake or a bad decision, say if you didn't pass the ball when you should have, then you'd be taken off.  It doesn't matter how well you played before that, one basic mistake, for example, you were under a lot of pressure and you didn't judge the play properly, and you didn't pass the ball, but instead you took the ball forward, maintained possession...I don't think its such a big issue.
    %Its not like we're talking about that type of situation where there is an obvious two-on-one scoring opportunity and you didn't pass the ball (to the open player)...
    But he (Zheng) won't accept it.  If you don't pass the ball (in that situation) he will yell at you, the pressure is extreme.  So we became very careful when we played, which developed a terrible habit, which was when you dropped the ball ``baang!'' (when we made a mistake like that we knew ourselves that we had made a mistake, but we'd still look to the coach for confirmation!!)
  \end{quotation}


  \begin{quotation}
    ...一点失误或者判断不好不传球就下来,你以前打的再好,一个简单的失误,比如说这个球压力很大。你在场上你美判断好,然后你没传,但是说我可以向前,保留球权,我觉得问题不是太大。
    %他并不是说想那种很明显的二打一机会我不传...
    但是他不行,不传球就骂你,压力非常大, 我们打球的时候初伏小心。最后黑的什么习惯:“呗儿”(掉球)。。。 我们一失误了自己也知道失误。先看教练!
  \end{quotation}





  \myparagraph{Football match in January}
  For the first training session after Wang took the reigns as head coach of the team in early January 2016, he decided to play an informal game of half-pitch association football.  Wang told the team he wanted this to be an opportunity for everyone to relax a little bit and reconvene after the New Year holiday break, and of course the abrupt change in coaching staff. After a gentle warm up, the athletes split into two teams (of mixed age and ability) and played roughly 60 minutes of football, with Wang as the referee.  I joined in on one of the teams also.  The football game contained the two key elements above---both a relative silence in coordinating joint action, as well as a constant stream of criticism and critique of teammate and opposition alike, mainly in the direction of senior to junior athletes.  There was also a dramatic incident following what I thought was an obvious goal scored by our side, in which the ball scooted through the bottom left hand corner of the net-less goal posts. My interpretation was that Wang Wei, always a cheeky and crafty operator, saw an opportunity to challenge the decision  pleaded boldly with the referee (Wang), and he was joined immediately by others in his team to create a wild chorus of complaint and hysteria, motivated to challenge Wang on his decision to award a goal.  Astonishingly, Wang agreed to reverse the decision, and the goal was withdrawn.

  Later, upon reflection, while I of course couldn't verify the accuracy of my own perception of the goal (or indeed `no-goal'), I did feel that many of the stereotypical components of joint action, which I had noticed previously during rugby training sessions, were on display.  There was a clear lack of politeness between individuals in both joint action execution (lack of communication) as well open scrutiny---the direction of which was largely from senior to junior.  Importantly, there was also a public appeal to authority (Wang, the referee), and a belief that his decision as referee was far from final (as the culture of some sports, such as rugby, would suggest) but rather could be contested and appealed.  The team in which I was embedded was not the sanctimonious egalitarian space in which bounded individuals politely navigated and negotiated each other to reach a sense of coherence.  Rather, the team was a rowdy family in which coordination and disagreement were already distributed throughout the group, and therefore publically scrutinised, only with the blessing of the Leadership, of course, whoever that may be in the particular context.


  \myparagraph{The art of saving face in joint action}
  The hierarchy of relationships in the Beijing men's team added yet another structural dimension to the distributed psychological system of joint action, described above. On the face of it, a system in which 1) lack of active communication in joint action, and 2) an excess of scrutiny of individuals following joint action were the defining parameters could be susceptible to chaos and conflict.  My observations of the actions of senior athletes and coaches suggest that these actors were capable of constraining and directing these two parameters of the system in order to ensure its reproducibility.   First, it appeared that coaches and senior athletes sought to mitigate the risk that a habitual lack of active communication in joint action could lead to poor quality performance by offering direction and models of joint action that could be emulated by junior athletes, therefore bypassing the need for overt communication (leader/follower dynamic). Second, senior athletes and coaches sought to regulate the direction in which scrutiny of joint action flowed, in order to  ensure that criticism was not chaotic and therefore threatening to the integrity of the system itself (scrutiny generally moved top-down or horizontally, but not so much bottom-up).

  A few examples shed light on these dynamics.

  The first example involves open scrutiny of action. I was sitting in the canteen with a small group of senior athletes and Coach Shi after we had all finished eating.  Senior player Lu, one of the most talkative at dinner conversations, usually about the intricacies of computer games (such as League of Legends) was driving a conversation about the ability of specific Chinese rugby players. Again, just as I observed on the training field, there was in my eyes a distinct openness of discussion about the idiosyncrasies, strengths, and deficiencies of athletes, some of whom were former Beijing athletes, others were from other provinces.  In addition to scrutinising joint technical competence, I commonly found myself in conversations with Chinese rugby coaches, athletes and enthusiasts about an athlete's quantifiable attributes such as height, weight, and non-rugby attributes (bench press, squat, 100m sprint)---attributes I personally spent very little energy on when making generating an opinion or impression about a player.

  Eventually the discussion shifted to a direct and open discussion of Lu Zhongsheng, or ``Big Mouth Monkey'' (大嘴猴) as he was affectionately nicknamed. Indeed, Big Mouth was an example of an athlete---who had transitioned from athletics to rugby during 2010-2013---who had extremely attractive physical and athletic attributes (height, weight, strength, speed), but who had yet to mature in his grasp of rugby-specific technical skills and intelligence during game-play.  Lu continued to drive the conversation, drawing attention to Big Mouth's inability to catch the rugby ball following a kick off by the opposition. ``If you ought to train anything as a Forward, it should be catching kick offs.'' The tone of the conversation driven by Lu appeared to me neither malicious towards Big Mouth, forgiving. It was, however, direct and critical, and it was allowed to develop, presumably as it was driven by Lu.  Never did I witness a public conversation at mealtime in which Lu himself, or Han for that matter, were the subject of scrutiny. Towards the end of the conversation, others besides Lu began to join in on the criticism, to the point where the group began to laugh at Big Mouth as he tried desperately to defend their criticisms of his rugby ability.
  Finally, coach Shi intervened, saying ``Ok ok enough, or else he will soon lose face!'' (好了好了,他快丢脸了!).  In this instance, Shi in his role as coach was able to diffuse the pressure on Big Mouth that began to mount once he began to challenge the criticism and others joined Lu in scrutinising.

  In the case above, the odds were stacked against Big Mouth, and he had little room to move from criticism.  Indeed, his kick off receipts were objectively substandard, and this criticism was being delivered by one of the most senior members of the group.  I noticed other times, however, when these situations contained more room for contestation.  Six months after that incident with Lu and Big Mouth, I was sitting with the Beijing team in a meeting room at the Institute of their as the team analysed the video footage of the Tournament, two days after its completion.  Head coach Wang led the session, and the athletes were asked to point out issues as the video rolled. Wang had prepared his own list of analyses and criticisms of issues with the team's performance, on which the video footage shed light.

  Senior athletes featured most prominently in the video footage, by virtue of the fact that they were usually selected in the starting team.  As such, many of the issues that were identified by Wang required a response or engagement by athletes who were involved in the specific incident displayed in the footage. Immediately this process generated a tense atmosphere, as Wang was attempting to expose technical issues with individual and team performance, and senior athletes were attempting to negotiate these criticisms.  Joint action scenarios in rugby are inherently complex, multi-faceted, and experienced differently by each co-actor, meaning that there is occasionally considerable room for divergence of opinion regarding the same phase of play. Nonetheless, in a public scenario such as this, which was directed by Wang, I assumed that idea was to make collective progress and draw general conclusions that would help the team's progress.

  What ended up playing out, however, was a series of agile defences from senior athletes to Wang.  Often these defences would involve senior athletes criticising more junior athletes also involved in the joint action, for not doing their bit in the joint action.  This strategy resembled a pattern of training sessions in which senior athletes would constantly deflect responsibility for mistakes, and instead ridicule more junior players for their lack of support, or the quality of their pass, or some other component of the joint action. Unless the mistake was extremely obviously the fault of the senior athlete, I very rarely witnessed a senior player publically admit to a senior athlete that they had got it wrong. It is of course unlikely that the fault in joint action would be the result of the more competent (senior) co-actor, but it did happen occasionally happen.  The point here is that there is not a culture of active self-criticism, and instead, instances like the video session exposed an art of saving face.  Indeed, the video session more clearly exposed this art of self-preservation, because unlike in training, there were fewer opportunities for senior athletes to blame junior athletes, and the video provided a more objective 3rd person referent that made responsibility for the outcome of joint action less deflectable.

  The video session possibly also became a contest because Wang, the director of the session, provided a certain amount of room in which the senior athletes could contest his or others interpretation, just as Wang Wei had contested his referee call in January during the football match (HYPERLINK). During my interview with Lu, he provided a rich description of his experience doing the same type of video analysis session with Zheng Hongjun:

  \begin{quotation}
    ...especially when we were analysing video: we'd spend three hours analysing a seven-minute video...So much detail: he'd analyse every little chicken feather and garlic skin (every triviality) for each of us. For example, if the ball was slippery and you didn't pass the ball properly in that play because the ball was slippery, it would become your personal issue....I
  \end{quotation}


  \begin{quotation}
      尤其是我们看录像,7分钟的录像看三个小时。很细。鸡毛蒜皮都会给你分析到。你比如说我就是球滑没传好,这个小失误都会上升到你人的问题...
  \end{quotation}



I asked Lu why he thought Zheng was so strict and critical of athletes in these sessions.  He replied:

\begin{quotation}
    It's due to belief and habit. Before it was extremely strict, we were all scared to bits, on whichever day, every detail of every video you would remember so clearly. `The video is coming my way!'  Shit, it was a whole serve of cursing with the blood of dogs, it would scare you to death, cursing like the blood of dogs, in front of all these people, the pressure was extremely high, normally people can’t accept this atmosphere! Normal people can’t come back from that!
\end{quotation}


  \begin{quotation}
      信仰和习惯。以前就是非常严格,看录像我们都吓坏了,哪天你,每一个录像每一个细节我们都记的很清楚。'马上到我马上到我!''一顿狗血骂人,吓死你,骂得你狗血里面XX,当了那么多人的面 压力非常大,一般人在这个气氛下受不了。一般人出不来!
  \end{quotation}


What was at stake in these sessions, it seemed, was each individual's ``face'' ---his or her social identity.  In the case of the video session that Wang convened, senior athletes contested and defended themselves, but in the story recounted by Lu, it appeared as if Zheng allowed no such room for contestation, at least by Lu.



  \section{Discussion}



  The cultural hyper-priors of JA - Team Click - SB <- factors of attraction for cultural evolution.
  Relational and Categorical modes in interaction and flux in Chinese sport.



  \section{Conclusion/Contribution}
  Cross-cultural psychology / anthropology, CAT


                                                          \end{CJK}
