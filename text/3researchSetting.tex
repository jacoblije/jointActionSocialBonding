
\begin{savequote}[8cm]

  \qauthor{}
\end{savequote}


\chapter{\label{chap:researchSetting}Research Setting}


\minitoc




                                          \begin{CJK}{UTF8}{gbsn}


%\section{The Temple of the God of Agriculture Sports Institute}
I first visited the Beijing Temple of the God of Agriculture Sports Technology Institute (\textit{Beijingshi xiannongtan tiyujishu yundong xuexiao} 北京市先农坛体育技术运动学校,
hereafter the Institute) first thing in the morning on my first Monday in Beijing.  I entered via the main entrance in the south and made my way west to the main administration building by hugging the southwest perimeter of the 30,000 seat capacity, multi-purpose sport stadium that spatially dominates the Institute's campus (see map ~\ref{fig:beijingXNT}). My aim that morning was to confirm the details of my proposed research with the vice principal of the Institute responsible for the rugby program, as well as the head coach of the rugby program.

I knew both the vice principal and the head coach from previous times spent in China studying (2006 and 2008) and coaching and coaching rugby (2013) prior to my doctoral research.  Although I had already received positive responses from both during the planning stages of my research, I still needed to confirm arrangements for research face-to-face.

I received mixed reactions when I announced to friends within the Chinese rugby community in Beijing that I was preparing to conduct research with the Beijing rugby team at the Institute. Those reasonably well acquainted with the politics of rugby in China, rugby in Beijing, or sport in China more generally, warned me about becoming too involved.  Perhaps they were worried that I might suffer a similar fate to the group of coaches and athletes involved in the National Games in 2013.  My friend and Chinese rugby elder Adrian, for example (see Chapter~\ref{chap:intro} Section ~\ref{sect:adrian}), warned me to be careful, citing that the Institute was riddled with ghosts: ``There really are ghosts there, I'm telling you Lijie, you should keep your distance'' (真的有鬼啊,我告诉你李杰,你最好离远点吧).  I naturally scoffed at these warnings, which were admittedly tongue-in-cheek.  I was intrigued, however, to learn more about the cultural, political, and psychological---as well as ancient cosmological---principles that lay beneath the recent history of the Institute.

The Institute is located in the heart of Beijing, just to the west of Yongding gate in on the South 2nd Ring Road. As one can imagine given Beijing's 3000-year history, the centrally located land on which the Institute sits was not always home to sport facilities and professional athletes.  The Institute takes its name from the temple that was built on the land in during the Ming Dynasty in the 15th century.  Yongding gate marks the southern end of the city's ancient north-south axis, which also includes, from south to north: Tiananmen Square, the Forbidden City, Jingshan Park, and the Drum and Bell Towers (see Figure ~\ref{fig:beijingTemplesXNT}).  For thousands of years, Chinese cities have been laid out on a north-south axis according to the principles of feng shui. The auspicious power or energy (\textit{qi} 气) of each monument along this axis is believed to flow upward from the south (south being the most auspicious and therefore important of the Four Directions in Chinese cosmology).

In addition to this spatial dimension of cosmological order, dynastic powers also sought to reinforce order temporally, through regular performances of ceremonial sacrifices.  In the Ming and Qing Dynasties, Emperors (or their commissioned representatives) used the Temple of the God of Agriculture (hereafter the Temple of Agriculture) during the middle month of Autumn and Spring (according to the Chinese lunar calendar), to perform Middle Sacrifices (in a system of Grand, Middle and Common sacrifices) in honour of the Harvest (\textit{nong} 农)---one of the four main cosmological principles, in addition Heaven (\textit{tian} 天), Earth (\textit{di} 地), and Ancestors (\textit{zu} 祖)\citep[98]{Brownell2008}.  However, when the Qing Dynasty (1644---1912) finally buckled under the pressure of Western Imperial occupation and popular revolutionary political movements in 1912, Confucian Sacrifices in Beijing's various temples ceased.

But as one form of ritual expired in China, another form began.
The embrace of the practice and spectacle of modern sport by the Republic of China (ROC, 1912-1948) and the People's Republic of China (PRC, 1949-present) has been so resounding and fundamental that sport stadiums (and the sporting events for which these stadiums were specifically constructed) have supplanted ancient temples and rites along Beijing's sacred north-south axis as a medium of communicating state order \citep{Brownell1995}.  The flow of auspicious cosmological energy now begins with the Temple of Agriculture Sports Institute (founded in 1930) in the south, and ends 21km to the north where the National Stadium, National Aquatic Centre, and the Olympic Green---the iconic monuments of the Beijing 2008 Summer Olympics---are situated.  Since re-joining the International Olympic Committee in 1979, China's athletes have participated at every Summer and Winter Olympics (except for the 1980 Moscow Summer Olympics, which China joined 65 other nations in boycotting following the Soviet Union's invasion of Afghanistan in 1979), winning over 600 medals in 28 different sports.  China's performance on the international stage is facilitated by a now enormous state-sponsored competitive sport system (\textit{jingji tiyu tizhi} 竞技体育体制), which consists of thousands of sport programs housed in secondary and tertiary level education institutions and specialist sport institutes throughout China's 34 provincial level regions.  According to China's National Bureau of Statistics, in 2016 China's sport industry reached a total scale of RMB 1.9 trillion yuan (USD300 billion).  Since 2014, when the central government first declared plans for sport to become a ``pillar industry'' of China's modern economy, the sport industry has been growing by an average of 18.2\% per year.  The central government has set a target total scale of 5 trillion yuan (USD800 billion) or 2\% of GDP by 2025.\footnote{For comparison, in 2016 the scale of the US sport industry was USD400 billion. At present, the Chinese sport industry is dominated by manufacturing (sport equipment, apparel, etc).  There is therefore considerable room in the Chinese sport sector for much higher value inputs, such as the development of the sports services (tournaments, leagues, activities).}

Sport first became associated with the Temple of the God of Agriculture, in the form of a horse-racing track at the southern gate of the Temple at the end of the Qing dynasty. But, beginning in this period of history, not all of the energy that flowed from the sacred temple has been auspicious.  As I will explain below, the arrival of rugby to the Institute has been a predominant source of much of the most recent cosmological turbulence.  Although I was personally captivated by the fact that the Institute was located in such a culturally meaningful location in Beijing, I was also slightly apprehensive about the state in which I was entering the Program and the Institute.  It was with a slight pang of nervousness, therefore, that I made my way from the southern gate, around the stadium, and towards the main office building to meet the Institute's vice-principal.

\begin{figure}[htbp]
  \centering
  \includegraphics[scale =.5]{images/beijingXNT.png}
  \caption{Location of the Temple of the God of Agriculture Sports Technology Institute  (Source: Google Maps)}
  \label{fig:beijingXNT}
\end{figure}


\begin{figure}[htbp]
    \centering
  \includegraphics[scale =.1]{images/beijingTemplesXNT.png}
  \caption{Locations of former Qing dynasty temples and modern sport stadiums along Beijing's sacred north-south axis (Brownell 2008)}
  \label{fig:beijingTemplesXNT}
\end{figure}

%width = \linewidth,


\section{Introduction}

In this chapter, I introduce the empirical research setting in which I test a general theoretical account of team click in group exercise.  To situate a theory of social bonding through joint action in the context of rugby in China, I introduce two core layers of context.  The first and most immediate layer pertains to the specific parameters of joint action in rugby union.  The second layer pertains to the cultural system in which rugby is enmeshed: contemporary China.  To re-emphasise my aim: in this thesis I do not attempt to explicitly address the question of how variation according to activity or culture alters a general account of team click in group exercise.  Rather, I seek to situate general theoretical claims within a specific group exercise setting, so that these claims can be meaningfully tested.


CULTURAL AFFORDANCES/coordination smoothing:
Both layers of context---rugby and China---are best understood using the concept of affordances \citep{Ramstead2016}.
Coordination smoothers for joint action: shared referents, ideas, concepts that make joint action easier. Rugby will generate uncertainty; China will provide cultural affordances which will be used to reduce and mitigate uncertainty.


In the sections below, I calibrate research hypotheses to the context of rugby in China, by generating expectations concerning the ways in which professional Chinese rugby players will experience joint action.  Rugby union is an interactive, field-based team sport that requires of its participants high levels of physiological exertion and socially coordinated movement---both factors are known to be linked to social bonding \citep{Cohen2017,Davis2015}. In addition, rugby is anecdotally and colloquially associated with social bonding in many of the contexts in which it is commonly played \citep{Dunning2005}.  The combination of highly complex and physiologically costly joint action demands, and anecdotally observable social effects makes rugby union an ideal arena to investigate the predicted explanatory role of team click in a theory of social bonding through joint action (Hypothesis 1).  Specifically, high levels of uncertainty (owing to its complexity, intensity, and competitiveness) and interdependence (due to the fact that almost all action in rugby is joint action) in rugby makes rugby an activity particularly well suited to testing theoretical claims regarding the mechanisms of uncertainty and positive expectation violation that are hypothesised to underly the joint action --- social bonding link (Hypothesis 2).

The cultural milieu of China will also reliably shape patterns of experience and expression of joint action.  Numerous strands of psychological and anthropological research have converged on the proposal that cultural systems can shape patterns of attention and behaviour in non-negligible ways (SOURCE). With the longest continuous cultural history in the world, China boasts a rich, diverse, and complex cultural terrain for the study of human behavioural phenomena such as group exercise. Of most relevance to situating a general account of team click in group exercise is the way in which the cultural milieu of contemporary China (broadly construed) reliably facilitates a \textit{relational} mode of social cognition, which reliably shapes patterns of action and perception, self-construal, group formation, and institutional norms \citep{Nisbett2003}.  Evidence suggests that relational social cognition will encourage preferential emphasis on holistic (over logical) reasoning, attention to and maintenance of particularistic and hierarchically organised social relationships (over deontological commitments to categorical social identities such as self, group, or nation), and an ethic of personal cultivation (over self-less egalitarianism) \citep{Liu2009}.
I expect these processes to bear upon athlete experiences of joint action, team click, and social bonding, by patterning modes of action and perception in joint action, expressions of group membership to the team, as well as broader socio-institutional processes in which athletes are embedded.

I focus my research on a population of male and female professional Chinese rugby players.  I begin with an in-depth ethnographic study of the Beijing men's rugby team ($n = 26$), before extending my reach to a broader population, during a National rugby tournament, and by conducting a controlled field experiment ($n = 58$).


\section{Rugby in China}
In this section I provide a historical background to rugby in China and my ethnographic setting in particular.  The history of professional rugby in China is situated within a specific history of modern sport in China.  In turn, the emergence of modern sport and exercise in China is entangled with a history of interaction and conflict with, and embrace of foreign influence during the 19th, 20th, and 21st centuries.

Prior to the introduction of Anglo-American interactional team sports and Northern European calisthenics in the late 19th century, physical activity that could be grouped in the broad category of sport and exercise was limited to a small collection of physical cultures indigenous to China, practices imported much earlier in history from south Asian religious traditions (e.g. Buddhism and Hinduism), and some activities imported from non-Han dynasties such as Manchurian horse racing in the Qing dynasty.  Historical records suggest that participation in almost all of these activities was limited to the imperial ruling classes and urban elites \citep{Ge2005}.

Increased foreign imperial influence in China during the 19th century brought with it the introduction of an array of ideas and practices to China's urban elite ruling classes, including novel physical cultures of dress, adornment, leisure, and not least, physical exercise.  Sport first manifested in China in the promotion of Anglo-American competitive sports by North American Christian missionary organisations such as the Young Men’s Christian Association (YMCA), and the incorporation of Northern European calisthenics routines into military exercises \citep[240]{Morris2004}.  Such techniques were soon popularised within elite intellectual communities as pedagogical tools designed to foster an explicit link between the strength of the physical body and the strength of the Chinese nation \cites[32]{Morris2004}[49]{Brownell1995}.\footnote{The belief that sport and exercise were important pedagogical tools in the development of physically and mentally strong subjects of modernisation---an idea Increasingly popular at the time in Europe, North America, and Japan, \citep{Elias1986}---gelled with the values of a nationally (and internationally) motivated Chinese urban elite}

\textit{Tiyu} (体育), the term in modern Chinese that most closely translates to sport, was one of many neologisms inherited from the Western social sciences via its Japanese translation.  Importantly, tiyu encompasses more than just the modern Anglo-American competitive sports (roughly translatable to ``yundong'' (运动) that the English word connotes.  The modern Chinese concept of sport refers to an entire culture and discipline of the body that is deeply intertwined with the political project of Chinese modernisation and advancement \citep{Morris2004}.\footnote{As Lydia Liu (1995: 58) points out, even the notion of ``China'' itself as a term linked to a national imaginary, only began to emerge as such through interaction with Western missionary discourses concerning ``China'' during the 2nd half of the 19th century.}

During the Republican Era (1912-1949), sport in China began to develop among urban elite along two main strands—--``competitive sports'' (\textit{jingji tiyu} 竞技体育) and ``games and calisthenics'' (\textit{ticao} 体操).  A significant aspect of competitive sports was the public spectacle of the ``games meets'' (\textit{yundonghui} 运动会), in which the performance of emerging national and international political identities could take place.  As early as 1908, the Chinese sport community enshrined the Modern Olympic Games (\textit{Aolinpike yundonghui} 奥林匹克运动会) as the pinnacle of participation in an international community of nations, and as such, the quadrennial global ritual has since preoccupied a collective Chinese sporting consciousness, and a Chinese national consciousness more broadly (\citep{Jarvie2008;Barme2009;Brownell2008;Morris2004;Xu2008}.

Along with many other facets of society after 1949, when the CCP took power in 1949, sport was institutionalised in line with Soviet bureaucratic models of governance.  In 1952 the ``State Sports (and Physical Culture) Commission'' (\textit{guojia tiyu yundon wieyuanhui} 国家体育运动委员会) (hereafter the Sports Commission) was established, which acted as the central State organ responsible for the administration of ``sport for the masses'' (\textit{qunzhong tiyu} 群众体育), ``physical culture education'' (\textit{tiyujiaoyu} 体育教育), as well as an elite competitive sport (\textit{jingji tiyu tixi} 竞技体育).
The competitive sport system was designed with the intention of creating a fast track for the development of world class athletic talent, in lieu of a sports system as advanced as other (predominantly Western) nations, whose development pathways for athletes were more organically embedded within existing social and educational institutions \citep{Brownell2008}.  By creating model athletes capable of performing and advocating the healthy, egalitarian and militaristic body promoted by the Party, competitive sport was designed to kick start more widespread engagement in ``sport for the masses'' and ``sport education''\citep[56]{Brownell1995}.

The competitive sport system was designed with the intention of creating a fast track for the development of world class athletic talent, in lieu of a sports system that was as advanced as other (predominantly Western) nations \citep[whose development pathways for athletes were more organically embedded within existing social and educational institutions; see][]{Brownell2008}.  By creating model athletes capable of performing and advocating the healthy, egalitarian and militaristic body promoted by the Party, competitive sport was designed to kick start more widespread engagement in ``sport for the masses'' and ``sport education''\citep[56]{Brownell1995}.  Publicly, the proletarian body and the propagation of an ideology of active cultivation of the physical body was centralised in CCP propaganda \citep[58]{Brownell1995}.  The body of the worker (\textit{gongren} 工人), peasant (\textit{nongren} 农), and soldier (\textit{bingyuan} 兵员), as well as the body of the athlete (\textit{yundongyuan} 运动员), were glorified for their ``capacity for manual labour'' (\textit{laodongli} 劳动力 )---the ideological foundation for the ``proletarian revolution'' (\textit{wuchanjieji dageming} 无产阶级革命) (Ge and G. 2005: 91).  The ``emancipation ''(\textit{fanshen} 翻身) and glorification of the physical, labouring body is particularly explicit in the propaganda posters of the early Mao era \citep[87]{Ge2005} (see Figure ~\ref{fig:motherlandStrength}).

\begin{figure}[htbp]
  \includegraphics[width = \linewidth,scale=.7]{images/motherlandStrength.png}
  \caption{Strengthen Physique to Defend Motherland (1950)}
  \label{fig:motherlandStrength}
\end{figure}

The athlete was thus positioned as a representative of Chinese nation building, and a dedicated sports system was erected to facilitate the fast-tracking of athletes to elite performance.  One of the symptoms of the inorganic erection of a sporting infrastructure was that elite sport institutes became  separated from education institutions.

The widening gap between education and sport in China became the subject of public scrutiny in the early 1980s, but China's sporting success on an international stage during this period delayed policy response.  The PRC won a total of 32 medals at the 1984 Los Angeles Olympics---its first official appearance at the Olympics since it boycotted the games in 1952 due to a dispute with the Republic of China (now Chinese Taipei) over the use of ``China.''  Importantly, 15 of these 32 medals were gold, and this powerful display of strength on the international stage was an enormous moment for modern Chinese nationalism in the reform era \citep{Brownell2008}.  When China produced a much less impressive performance in the summer Seoul Olympics in 1988, winning only 5 gold medals (and a total of 28), latent public criticism of way in which reform era sport had become isolated from society readily surfaced and a ``crisis in Chinese sports'' was declared \citep[199]{Brownell1995}.  Amidst broader social anxieties concerning not only the alarming quantity of the Chinese population (\textit{renkou guoduo} 人口过多), but also the problem of population \textit{quality} (\textit{renkou suzhi} 人口素质), the athlete in China was problematised as lacking sufficient ``cultural quality'' (\textit{wenhua suzhi} 文化素质) in accordance with his or her elevated social status as a ``representative'' (\textit{daibiao} 代表) of the Chinese nation on an ever-expanding international stage (General Administration of Sport 2009a; Brownell 1995: 95).

In 1989 the Sports Commission adopted a policy modelled on the US college sports system, of ``combining sport and education'' (\textit{tijiao jiehe} 体教结合).  In an attempt to move away from a reliance on sport boarding schools and full-time sports training centres for the development of athletic talent, ``high level tiyu programs'' (\textit{gaoji tiyu xiangmu} 高级体育项目) were embedded within existing stand alone high schools and universities so as to ensure the ``all-round development''(\textit{quanmian fazhan} 全面发展) of the athlete \citep[203]{Brownell1995}.  As part of an emphasis on a broader range of sports and their perceived potential to facilitate community engagement, international relations, as well as commercial opportunities, various sports programs, including many non-Olympic sports such as rugby, were inducted into the Chinese sports system for the first time\citep[70]{Knuttgen1990}.  As I explain below, China's first rugby program was a high level sports program, embedded at the Chinese Agricultural University in Beijing in 1990.

While China has experienced widespread social and economic transformation since the death of Mao in 1976, and while the sport system in particular has experienced numerous waves of policy reform,  the core organisation and the logic of the Chinese sport systems remains largely in tact.  Sports like football and basketball have matured as standalone enterprises with supporting market-based consumer industries, but most other sports in China (i.e., all other Olympic events, including rugby) exist primarily due to the support of the enormous state-sponsored sport system.  Whereas the commercial basketball and football industries might offer a small percentage of prospective athletes incentives of fame and fortune, the benefits of a state-sponsored sports programs like rugby are more modest.  Chinese youth either gravitate or are ushered by their parents towards sporting careers primarily due to potential life-course opportunities such as access to tertiary education and post-athletic career employment.

Thus, the athlete in China thus occupies a unique position, as both a symbolic representative of Chinese nationalism, as well as the target of various incentives pertaining to various life-course opportunities. American cultural anthropologist Susan Brownell, in her monograph ``Training the Body for China'' (1995), was the first to provide a comprehensive development of the subjectivity of Chinese athletes in the (post-Mao) reform era in China.  In reference to the unprecedented success of the Chinese women’s volleyball team in the 1980s, including winning China's first ever gold medal in a team event at the LA Olympics in 1984, Brownell explains how elite level sport functioned as a crucial symbolic practice for China in the process of ``re-joining the world'' (1995: 86).  As a participant in the sports system as a student-athlete herself, Brownell draws on first-hand ethnographic experience of training and existing as subject to the state-administered ``microtechniques of power'' (citing \cite{Foucault1977}) designed to cultivate athletes in post-Mao China. This unique empirical contribution serves to situate the athlete in China between tensions and shifts of an ever-transforming social terrain structured by contradictory forces of top-down state control and the emerging logic of the free market.

Since Training the Body for China, very little ethnographic research into the experience of athletes in China has been published. Since her ethnographic research was conducted in the mid 1980s, processes of the market-based economy have transformed the nature of incentives and choices available to athletes.  Yet, the competitive sport system, of which the Institute in Beijing is one of many examples remains strongly in tact.  Generally speaking, athletes rarely gravitate to the Institute as part of a duty to be a symbolic representative of Chinese nationalism.  Rather, they manoeuvre to the Institute and to the rugby program for the attractve life-course opportunities it affords.

\setting{Rugby in China}

\myparagraph{Rugby as an university level amateur sport (1990-2009)}
Although reportedly existing in China within colonial and expatriate circles for more than a century \citep[210]{Reason1979}, and as a modified military exercise as early as the 1930s \citep[135]{Morris2004}, rugby was a late entrant into the Chinese sport system, established as a ``high level sport program'' at  Chinese Agricultural University in 1990.  The advent of rugby in China was thus part of the process of combining sport and education of Chinese sport initiated in the late 1980s.

The program was originally made up of existing CAU students who expressed interest in the novel activity, but by its second year, the program earned status as a High Level Sport Program and was subsequently advertised to student-athletes across the country \citep[2]{Xu2010}.  Between 1990 and 2009, rugby programs based on this original CAU model were established within over 30 regular universities and specialist sports colleges in cities throughout China.  In addition to these programs, a number of social rugby clubs (\textit{shehui julebu} 社会俱乐部; organisations completely independent of the state sports system) began to form in major cities with high expat populations (e.g., Shanghai, Beijing, Chengdu, Qingdao).

For the first 20 years of its existence in China, Rugby was part of a large collection of ``cold-gate'' sports (\textit{lengmen xiangmu}, a term that refers to a profession, trade or branch of learning that receives little attention) in China, which had a relatively small participation base compared to other interactive team sports like basketball or football.  The Chinese Rugby Football Association (CRFA) was established in 1997, and both the men and women’s national teams, made up of players predominantly from CAU, but also from other well-established programs based at the Shanghai Sports University, Shenyang Sports College, and the People’s Liberation Army Sports College. China consistently competes against other nations in the Asia Pacific region (most notably in the Asian Games and the East Asian Games), and is also occasionally involved in top-tier international tournaments such as the International Hong Kong Sevens.

\myparagraph{Rugby in China 2010 - 2013 \label{sect:rugbyinChina}}
Olympic status transformed rugby almost overnight from its former position as an amateur sport played at university level by a handful of universities.  In 2010, rugby (in its seven-a-side version of rugby sevens) was included as one of 33 events to be held at the 2013 National Games in the city of Shenyang. This decision spurred provinces to set up professional rugby programs at provincial sports institutes, to compete at the National Games in 2013.  Ten of China's collection of 32 provinces and municipalities that participate in the National Games have full time men's and women's rugby programs.  For Olympic sports in China, results in the National Games (\textit{quanguo yundonghui} 全国运动会) decide the amount of funding a province and its constituent sporting institutes and programs receive.  As such, the National Games are the most important measure of success \citep{Hong2002} for athletes, coaches, and administrators in the competitive sports system.

When rugby union was officially inducted into the state sponsored sports system in 2010, a total of five full time Men's (Beijing, Shandong, People's Liberation Army (PLA), Liaoning, and Shanghai) and six Women's (Beijing, Shandong, Anhui, Liaoning, Shanghai, and Jiangsu) provincial programs were immediately established, which signalling an intention by these provinces to invest in the sport for the long term.  With the establishment of professional provincial rugby programs catering for tertiary aged athletes (17 years and above) already established, these provinces could subsequently initiate the establishment of city level rugby programs catering for high school aged athletes (10 - 16 years).  In this way, a previously non-existent development pathway for athletes, coaches, and officials began to emerge in provinces interested in investing in the sport.

In addition to these full-time provincial programs, three part-time men's (Inner Mongolia, Heilongjiang, and Xinjiang) and two part-time women's (Sichuan and Xingjiang) programs were established, in which these provinces temporarily employed rugby athletes from university programs. In addition, Hong Kong fielded both a Men's and a Women's side, bringing the total of Men's and Women's teams eligible to compete in the National Games to nine and ten, respectively.

\subsubsection{The National Games 2013 \label{sect:fallFromGrace}}
Two provinces in particular identified an opportunity to achieve a favourable result at the National Games by heavily investing in this debutant sport in 2010.  The Beijing men's and women's programs (based at the Institute) managed to attract a large amount of China's existing rugby talent from where it was previously based at the CAU.  Importantly, among Beijing's recruits was the unofficially touted ``Boss''  (\textit{Laoda} 老大) of Chinese rugby, Chinese national coach Zheng Hongjun.  Meanwhile, Shandong province, a traditional powerhouse in other provincial sports, succeeded in attracting the majority of the remaining rugby talent.  The pull to Shandong was strong for a large majority of rugby players in China at the time, many of which were originally from Shandong.  Importantly, the talent transferred to Shandong province also included coaching staff, namely Zheng Hongjun's student and soon to be rival, former Chinese Women's Team coach, Lu Xiaohui.  Besides Beijing and Shandong, Jiangsu and Anhui province were strong contenders for the Women's gold medal, while the People's Liberation Army (PLA) and Hong Kong in particular were strong contenders for top spot in the men's competition.

Beijing's results in the two years leading into the 2013 national games were strongest overall across the men's and women's teams.   However, the traditionally strong Hong Kong men's and women's teams had only occasionally participated in these tournaments due to conflicting international tournaments.  In the semi-finals of the National Games, held in Shenyang at the beginning of September 2013, the Beijing men came up against Hong Kong, while the Shandong men played off against the PLA.  Beijing lost to their stronger and more favoured opponents, and Shandong beat the PLA.  Meanwhile in the women's tournament, both Beijing and Shandong advanced to the final without faltering.  The stage was set: the traditional favourites, Beijing, led by the reining Boss of Chinese rugby, would face Shandong---the underdogs---lead by the Boss's cunning apprentice come challenger.

The men's final was played first, and in somewhat of an upset, Shandong edged out Hong Kong to win the gold medal by one try (one five-point touch down).  In the women's final, scores were level until early in the 2nd half when Shandong went ahead by two tries to nil.  At that point, the Beijing women's team, allegedly under instruction from their coach Zheng Hongjun, suddenly stopped playing.  After being asked by the referee and match officials to continue, the Beijing athletes stood firm and refused to play on, forming a huddle on their side of half-way in the middle of the field. Shandong had no choice but to continue to play out the rest of the 2nd half, running in try after try, until the final score at full time was a farcical 71-0 \citep{Sina2013}.  Shandong was declared victorious, while Beijing called foul play, claiming that the Spanish referee had been unfairly adjudicating the match in Shandong's favour.  The details and dramas of this now well-known story in China's sporting history (known as ``The 2013 National Games Match Strike Scandal'' (13年全运会巴塞门) ) require more detailed development in a format that exists beyond the scope of this particular dissertation.  Suffice to say, the repercussions of this incident for the Beijing provincial rugby program were extremely costly.

\subsection{The Temple of the God of Agriculture Sports Institute}
The rugby match-striking-gate of 2013 led to a sudden fall from grace for the Beijing rugby programs.  Between 2010 and 2013, the Institute Leadership, excited about the prospect of unprecedented success at the national games,  immediately elevated the rugby programs to top-priority status.  Rugby received unrivalled institutional and financial support in the hope that both teams would be crowned National champions---what would have been the Institute's first National Games gold medals since 2004.  During this period, the rugby program attracted a high profile commercial sponsorship deal from Beijing Capital Steel(北京首钢), which enabled the Institute to invest in a team of foreign coaches from New Zealand to come to Beijing on a periodic basis to consult on training and preparation. Both teams also travelled twice to New Zealand for two three-month stints of off-season training and competitions.  Between 2010-2013, the rugby team lived in the Institute's best available accommodation, and ate their meals at the Institute's highest level canteen, reserved for National-level champions.  Right up until the National Games in 2013, the men's and women's teams had met the high expectations set for them, winning all but one of seven national tournaments each.  All indications were positive for Beijing to take home two gold medals.  However, as explained above, the National Games in Shenyang in September 2013 did not transpire as Beijing would have hoped.

In the end, Beijing came away with one bronze medal (men's team) and one face-destroying disqualification for the women (official review of match referee performance found no evidence of clear foul adjudication).  During my time at the institute, assistant coach of the Beijing men's team, Shi Yan, told me quietly one evening that the Beijing women's rugby team was the first Beijing team in the 48-year history of the National Games not to receive the ``medal for civilised spirit''  (awarded by the Beijing Mayor to all Beijing representatives in the National Games).  All rugby coaches and many senior athletes of the 2013 National Games campaign have since left the Temple of Agriculture, either retiring or moving to other provinces.  The rugby program was all but abandoned at the end of 2013, with athletes from both teams being told to take a break for an undetermined length of time.  It wasn't until April 2014 that the men's program was resurrected with the appointment of a new head coach.

\myparagraph{Rugby at the Institute after 2013}
Having let the dust settle on the embarrassment of the women's program's widely publicised disqualification from the 2013 National Games, in 2014 the Institute decided to quietly continue with both the men's programs in preparation for the 2017 National Games.  In April 2014, more than six months after the National Games, former Chinese representative and CAU coach Zhu Peihou was appointed as new head coach.  The junior athletes from the previous National Games cycle were recalled back to the Institute to resume training, and coach Zhu was charged with finding new talent to fill the ranks of the team.  The women's program was inactive for a full two years after 2013, and was only just starting to re-activate after I arrived, in November 2015.  Thus, rugby was resurrected at the Institute, but was no longer in centre stage.

The Beijing men's team endured a series of mediocre performances during the 2014 and 2015 seasons, and clearly lacked experience, talent, and institutional support from the Institute.  A handful of senior athletes who had played in the era of the 2013 National Games remained, and two in particular, Han Xiaolong and Lu Peng were promoted to a transitional athlete-coach status. Unlike Women's assistant coach and former athlete Wang Chongyi, however, both Han and Lu were originally from Shandong province and so did not automatically have Beijing residency required to make them eligible for full time employment at the Institute.  As such, their future place at the Institute was uncertain, and as I found out from both during the course of my ethnographic research, their ability to stay at the Institute would depend on the result the team could achieve at the 2017 National Games: a medal at the National Games would qualify them for a fast-tracked and Institute-sponsored application for Beijing residency application.\footnate{China's infamously rigid residency (\textit{``Hukou''} 户口) system means that only individuals with Beijing residency can hold permanent employment roles at government institutions such as the Institute (\textit{shiye danwei} 事业单位).  Chinese citizens born outside of Beijing can become Beijing residents if offered employment, but due to Beijing's swelling population, the eligibility criteria for this process of naturalisation has become more and more stringent, and fewer and fewer applications are successfully processed, particularly in industries like sport.}

Despite being only a shell of its former glory, the rugby program at the Institute nonetheless offered attractive incentives to prospective athletes.  The difficulties of Han and Lu in gaining Beijing residency made it clear to more junior athletes that there was little promise of a passage to official Beijing residency or full-time employment at the Institute.  However, the program did offer a much more realistic opportunity of attending the Beijing Sports University (BSU)---considered to be the country's most prestigious sports universities and one of China's select ``big brand universities'' (\textit{mingpai daxue} 民牌大学).  The mass exodus of experienced senior athletes from the rugby program meant that junior athletes from the pre-2013 era were now in a position to represent Beijing at a national level, and in so doing attain the official athletic standard of a ``Master Sportsperson'' (\textit{yundong jianjiang} 运动健将).  A Master Sportsperson was automatically eligibility to attend BSU through its arrangement with the Institute.

It was in this context that I entered the Institute and began ethnographic research.
Professional sport program: incentives.


\section{Method}
The empirical research analysed in this thesis consists of an in-depth ethnographic study of the BJM and two field experimental studies.
Ethnography: describe method and setting briefly:
Setting, participants:

I conducted a three stretches of participant observation at the Institute, totalling ten months between September 2015 and September 2017 (1st stretch = 7 months, 2nd stretch = 2 months, 3rd stretch = 1 month).  During these stretches, I lived full-time at the Institute and attended (and often participated in, predominantly as coach) training sessions, team meetings, meals, and any other activities relevant to the rugby program.  In addition to participant observation, I also conducted semi-structured interviews and administered a series of informal surveys following training and at different intervals throughout my time at the Institute.  I recorded field notes using Evernote (Version 7.4.1), an electronic note taking software that was synchronised across my mobile and personal computer devices.

The institute was home to both a men’s and women’s rugby team, each with usually approximately 20-30 athletes and 2-4 coaches per team.  However, when I arrived to conduct research, only the men’s team was in active training, and so I dedicated my ethnographic research to the Beijing men’s team only.  I analysed data on a total of 26 athletes ($age = 20.96, range = 17-27, SD = 3.17$)   Athletes were included in data analysis if they participated in 1) a semi-structured interview, 2) at least one informal survey relating to experiences of rugby training and group membership, and 3) at least 2 months of training at the Institute.  See Table ~\ref{tab:ethnoDescriptivesTable} for a summary of athlete attributes, which will be discussed at more length below.

One of four vice-principals of the Institute---Jenny---was responsible for the administration of the rugby program.  I sought permission to conduct research at the Institute was from relevant authorities and directly from athletes at the beginning of the first research period in September 2015.

In addition to researching athletes, I also engaged coaches, officials, and informants with relevant knowledgeable of rugby and sport in China. Data collected on these individuals were not included in the main analysis, but provided valuable contextual information.
All my interactions with athletes took place in Modern Standard Chinese (Mandarin or \textit{putonghua} 普通话).  To record these interactions, I would either interrupt conversation to ask permission to start the audio recording device within the Evernote application on my mobile phone.  Alternatively, in the case of shorter or unplanned interactions that were relevant to my research,  I transcribed shorter conversations using my notebook or phone immediately following these interactions.  Every week or fortnight I collated, summarised, and organised these notes by date and by theme.  interviews were transcribed into written Chinese by a native Chinese speaking research assistant using a ``verbatim'' method \citep[i.e., including an account of all verbal and important nonverbal (coughs, pauses, etc.) utterances, see][269-70]{Poland2003}.  I analysed interviews in Chinese and only translated into English data extracts that were included in the main analysis of this thesis.

\myparagraph{The life of an athlete in the Beijing rugby team}
The Beijing men's rugby team competed against other provinces in five national tournaments held in different locations across the country every year between March and September.  The period in which I conducted my first stretch of ethnographic research (August 2015 --- February 2016), therefore, constituted the off-season and pre-season components of the training year.  Due to cold weather in the north of China during winter and spring, teams from northern China (e.g. Beijing, Liaoniang, and Shandong provinces) often elected to train at other domestic or international training locations depending on amount of program funding available and the training strategy of each program.  In 2015, before an unexpected change in coaching team at the end of December (explained below in Section HYPERLINK), the head coach of the Beijing Men's team had planned to travel to Yunnan province in early 2016 for one month of altitude training before moving closer to sea-level somewhere in the south of China for one month (February/March).  Following the coaching leadership change, the team did not leave Beijing until after Chinese New Year (25th February). Training during this period was therefore consistently stationed at the Institute in Beijing, and as such subject to occasional disruption due to Beijing's cold winter weather and air pollution.

All athletes lived and trained 6 days a week at the Institute, and would occasionally attend university or high school classes as part of their ongoing education commitments.  Below is a table of a typical weekly training schedule (see Figure ~\ref{tab:trainingSchedule}). A typical week consisted of 10 two and a half hour ($\approx 150 minute$) training sessions, seven of which were on-field rugby sessions, three of which were strength and conditioning sessions (not involving a rugby-specific skills).  In addition, two one hour evening skills sessions were also allocated for junior athletes to hone their basic skills of passing, catching, and game-play.  Athletes lived full-time on campus in the Institute's dormitory accommodation (usually 3 athletes per room), and were permitted to take overnight leave on the weekend after the conclusion of Saturday morning training.  Athletes from Beijing or with family in Beijing would usually take this leave, while the remaining athletes would spend weekends at the Institute.  Generally speaking, the rugby program would break at the end of the national season in September for two weeks, and occasionally around Chinese New Year for 7-10 days, unless New Year interrupts pre-season training plans, in which case training would continue in spite of this national holiday.

\subsubsection{Positionality of the researcher}
        1. P: My unique positionality as a researcher requires further consideration due to potential demand characteristics and
        2. Recap my unique position:
            1. Professional rugby player, spend many years in China, coached in 2013: I was known as a “friend of rugby in China” and as an expert in rugby
        3. P: my unique position as researcher involved many possible “demand characteristics” that are worth of consideration
            1. When briefing athletes on my motivations for research, I explained that I was generally interested in researching ``athlete's on-field and off-field experience of rugby.''  The unique position
            2. A foreigner:
            3. Essentialised as  a spokesperson for the West.. (common trope in China)
            4. Out of place ultimately
        1. Language: pro fluent, but still non-native, this means I may have missed
        2. My own values and influence: egalitarian and categorical intuitions of thought in a relational
            1. Therefore at times I take a reflexive stance (eg modes of social cognition).
        3. E:



    1. Road map for this chapter:
        1. P: In the sections that follow, I outline key considerations (PREDICTIONS?) for the ways in which joint action parameters of rugby, and the cultural context of China will bear upon a general account of team click in group exercise.
            1. Each consideration is formulated with the support of ethnographic evidence collected during my time at the Institute.
        1. Rugby: high uncertainty leads
        2. For China




\section{Rugby}
Rugby Union (hereafter simply rugby) is an interactional team sport played on a rectangular field (100m x 70m), by two teams of 15 players each, who contest possession of an egg-shaped ball that can be used to score points \citep{IRB2014}.  Descending from a variety of locally-specific folk games played in pre-industrial England, all loosely grouped as ``football,'' rugby developed within the elite public school system as a deliberate physical activity arbitrated by rules and regulations, before circulating through the arteries of Great Britain's colonial empire as a leisurely pastime—--a ``sport'' \citep{Dunning2005}.  In 1996, rugby became a professional sport and is played as such in Western Europe, the Southern hemisphere (Australia, New Zealand, South Africa, and Argentina), and Japan. Rugby sevens---the specific focus of this dissertation---is a modified version of the conventional 15-a-side game involving teams of 7-a-side, and 14-minute games played in a Tournament structure over two or more days (rather than a one-off 80-minute match between two teams).  Rugby sevens has grown in popularity more recently, particularly since its introduction to the Olympics for the 2016 Games in Rio de Janeiro.  More so than the traditional version of the game, rugby sevens is played by countries all over the world, and attracts more balanced participation by men and women.

High levels of uncertainty, and social interdependence associated with rugby will shape athletes’ experience (of perceptions of performance, team click, and social bonding in joint action).

UNCERTAINTY: rugby involves high levels of uncertainty due to complex and varied joint action requirements
                1. P: Diverse joint action requirements:
Rugby players run with the ball, pass or kick the ball to other attackers or towards open space on the field, enter collisions as either ball-carrier, tackler, or support player, and contest possession of the ball by grappling in ``rucks'' and ``mauls.''\footnote{In distinction to American Football (NFL), in rugby only the defensive team is only permitted to tackle the the ball-carrier from the attacking team.}

   P: Dynamical movement Complexity:
      1. Nested joint actions and goals across multiple timescales
      2. Multiple sensory modalities
      3. multiple agents in in the moment JA
   P: Physiological costs —> impact on cognitive capacity
   P: Competitiveness
   P: In-the-moment, on-line
   Expectation 1: Rugby will be experienced as challenging, owing to high levels of uncertainty in joint action.


INTERDEPENDENCE: rugby’s joint action involves high levels of interdependence
     P: almost all action is joint action,
        1. F:The performance of almost all technical actions require either direct or indirect consideration of the position and intentions of other athletes in relation to that action.
        2. Even the most seemingly independent technical operation among these skills---for example, an athlete kicking the rugby ball to open space on the field---requires the consideration of positioning of other athletes.
            1. Individual versus team performance:  Because all action is joint action, I do not delineate joint action from individual action in rugby, rather, dimensions of joint action, one more dominantly team-based, and one more dominantly individually-based (categorise performance according to its individual-level and team-level perceptions).
            1. Individual:
                1. For example, when athletes are participating in attack, athletes may be concerned with elements of individual-level contribution to joint action, such as the quality of their passing technique, or their positioning as ball-receiver, or their ability to make decisions.
                2.  In defence, athletes may attend primarily to their one-on-one tackling performance, or their effectiveness in contesting possession in rucks or mauls.
            2. Team:
                1. Athletes for example may develop an impression of how the team (or a specific subunit of the team) is managing to coordinate the defensive or attacking line, or how well the team is communicating throughout a game, or supporting each other in attack and defence.
                2. Various contextual and individual-level factors may underpin variation variation in the extent to which individuals attend to either level of performance. role of uncertainty on confidence / expectations for team performance


  Rugby is a sport traditionally associated with the colloquial interpretation of ``social bonding,''in all-male social organisation common in the elite educational spaces of England and Commonwealth countries in which rugby originally developed \citep{Dunning2005,Richards2007,Collins2008}.\footnote{Recently, rugby union has been the site of much criticism due to the fact all-male social groups that cohere around the sport of rugby appear to support and sustain hyper-masculine and hyper-normative behaviours, including gender-related violence \citep{Cosslett2014}.}

  Rugby is also a sport associated with social virtues:

``Rugby is a game for barbarians played by gentlemen,'' or so the saying goes. \footnote{The origins of this oft-cited adage are unclear.  The phrase is supposedly the adopted motto of the British Barbarians Football Club, established in 1890 \citep[34]{Dunning2005}.  The complete phrase reads ``Rugby is a game for barbarians played by gentlemen, football is a game for gentlemen played by barbarians.''  However, official club history cites its original motto as, ‘Rugby Football is a game for gentlemen in all classes, but for no bad sportsman in any class' \citep[vii]{Starmer-Smith1977}.  Some sources attribute the saying to British writer and poet Oscar Wilde (1854-1900) \citep{Fleenor2015}}.

Different inflections on this adage have been reproduced by people in all parts of the world that rugby has reached, including China \citep[see][]{Taylor2010}).  Presumably, this saying has survived due to its ability to tether the nature of rugby's physical requirements with socially valorised moral virtues of fair play, cooperation, and integrity.  The current slogan of World Rugby (rugby's international governing body) is ``Building character since 1886,'' \citep{WorldRugby2017}.  This phrase was presumably chosen to draw an association between the joint action requirements of rugby and the moral and social character that can be generated through participation in rugby.

        3. Expectation 2:  interdependence in rugby's joint action will encourage processes characteristic of social bonding
        4. SUMMARISE:  Uncertainty and interdependence in joint action in rugby will mean that
        1. Performance will be experienced as an inherent challenge
        2. Rugby's joint action will encourage processes characteristic of social bonding


1. INSTITUTE: Evidence from ethnographic observations supports and extends this these expectations.

  Expectation 1: Rugby will be experienced as challenging, owing to high levels of uncertainty in joint action.
                1. P: Athletes start from scratch, low familiarity with rugby’s joint action (rugby is a minnow sport in China).
                2. F: Descriptives
                3. F:Transfer from individual sports, or no sports, at age of ~16-18.
                4. F: high variation in technical competence:
                    1. Rough groupings: (Old guard (2), starting team, reserves, students, newcomers) (describe factions)
                5. E: Not only uncertainty inherent in the prescribed parameters of JA, but also amplified by the fact that many adherents are new to rugby.
  Expectation 2: interdependence in joint action will encourage processes characteristic of social bonding
                1. P: Athletes motivated to rugby for individual life-course opportunities of education and employment, not necessarily due to social virtues
                    1. F: Motivations for rugby (table, rank).
                1. At the same time, while the Institute was not the traditional social rugby club or school team found in England or New Zealand, athletes  but Live together, eat together, sleep together, and develop a level of interdependence and social identity (processes characteristic of social bonding).
                    1. F: My ethnographic observations (beyond these interview data) confirm that members of the Beijing team developed strong bonds of friendship with their teammates.  Athletes trained, ate, slept, and bathed together (in communal bathrooms).  Indeed, some athletes spent the majority of their formative high school, university, and young-adult years together.
                    2. F: Interdependence “Team Requirements of rugby (ethno examples):
                        1. "I have become a social animal” - adjustment to team requirements
                        2. F: Explicit declarations of group membership
                        3. Old ethno results stuff on group membership; pictures from wechat, etc
                1. E: Something about rugby that will encourage social processes: interdependence, group membership, despite the existence of individual incentives as explicit motives, and unfamiliarity of individuals with the idea of team.
                2.

Discuss in relation to hypotheses:

                1. Hypothesis 1:
                    1. perceptions of performance (team performance over individual)
                    2. interdependence on-field encourages social bonding
                    3. Therefore suitable setting in which to test the role of team click in this process (Hypothesis 1).
                1. Hypothesis 2:
                    1. Uncertainty in JA will lead to lower expectations for team performance (a possibility associated with the experience of challenge)







\section{China}
The cultural setting shapes attention, perception, and action in social interaction, which has implications for how relationships between JA, TC, and SB in GE will be expressed in different settings
        1. P: I expect that social cognition of joint action in China will be associated with a more relational mode of social cognition
            1. Relational mode of cognition (Edit! what’s the point??)

A gradual accumulation of research across the human sciences suggests the existence of non-negligible variation in social cognition according to culture.  Anthropologists have for some time emphasised meaningful cultural variation in processes of social group formation \citep{Strodtbeck1961,Kluckhohn1961,Mead1967,Fei1992}, and more recently cultural psychologists have sought to demonstrate this variation in experimental paradigms \citep{Markus1991,Nisbett2001}. %Indigenous CHINESE psychology

A core finding from cross-cultural psychology concerns the correlation between cultural variation and modes of group formation.  This research has produced a theoretical spectrum of processes of group membership, the two poles of which are usually described as ``categorical'' and ``relational'' \citep{Hofstede1980,Brewer2007}.  In the case of some cultural niches, traditionally samples from modern and industrialised ``Western'' societies (such as the USA), social identity formation revolves around the autonomy and immutability of social categories of ``self'' and ``group.''  By contrast, in societies in which a ``relational'' mode is dominant (for example, in East Asian countries such as Japan, China, Korea, etc.), social identity is defined primarily according to degrees of social embeddedness and interdependence with others comprising their in-groups\citep{Leung2012}.  A considerable amount of evidence has amassed within cultural and social psychology to suggest that the existence of meaningful contrasts between samples of East Asian populations (predominantly undergraduates of Japanese, Chinese, and South Korean universities) and Western populations (predominantly undergraduates of North American and Western European universities) in domains of attention and perception \citep{Peng1997,Nisbett2003}, psychological construal of social categories of self and group \citep{Markus1991}, behavioural tendencies in social interaction \citep{Yuki2003}, and institutional norms \citep{Liu2017}.

Modes of group membership have been shown to vary not only across cultures (i.e., East Asian versus Western European or North American), but also within cultures \citep{Henrich2014}, within social groups (according to sex and personality differences \citep{Yuki2014}), and even within individuals \citep[depending on contextual and situational primes, see][]{Lee2014,Wong2005}.  The prominence of one mode of membership over another in broad ethnic or cultural groups (e.g., The West versus East Asia) appears to be associated with the durable persistence of cultural and linguistic institutions that afford particular patterns of social cognition.

Categorical modes of social cognition have formed the foundation of Anglo-American social psychology of the 20th Century \citep{Liu2005}.  The canonical ``social identification'' paradigm, of social psychology, for example, requires that an individual make an identification between abstract categories of the self and the in-group or out-group \citep{Turner1987}.  Group membership is achieved when the perceived differences between the self and other in-group members are smaller than the perceived differences between in-group and out-group members \citep{Yuki2014}. Categories of self, other, and group are construed as independent and autonomous constructs that moor an individual's perception of social identity.

In distinction to a categorical mode of group membership, a relational mode of group membership requires that individuals attend more predominantly to maintaining and harmonising intra-group relationships, rather than engaging in intergroup categorical comparisons \citep{Yuki2003}.  In a relational mode of group membership, social identity derives less from a calculation of psychological distance between abstract categories of self and in-group, and more a degree of commitment to cultivating a network of hierarchically structured---but more or less self-centred and self-enhancing---relationships \citep{Liu2009,Nisbett2003}.

In addition to the concept of relational social cognition, an understanding of cultural psychological processes relevant to contemporary China can be further bolstered by contributions from Chinese indigenous psychology---a field dedicated to specifying Chinese psychological processes from say East Asian populations more generally \citep{Liu2009}—a ``Chinese mode of relational cognition'':

    1. \begin{enumerate}
    2.   \item \textbf{Holistic reasoning} is commonly employed to harmonise social relationships and avoid interpersonal conflict, via the principle of the  ``middle way'' \textit{zhongyong zhidao} 中庸之道).
    3.   \item \textbf{Hierarchical relationism} acts as a guide for managing hierarchically organised familial and social networks, known as ``guanxi'' (\textit{guanxi} 关系).
    4.   \item \textbf{Personal cultivation} is employed as a means of accumulating ethical virtue, or ``human heartedness'' (\textit{ren} 仁).
    5. \end{enumerate}



















In this chapter I introduce the empirical research setting in which I test a general theoretical account of team click in group exercise.  As mentioned in the introduction to this thesis, anthropology is now well positioned to contribute in different ways to scientific understandings of human behaviour and sociality, due to its diverse methodological toolkit.  I address research questions and hypotheses of this thesis through a combination of ethnographic and field-experimental research, which I restricted to one specific group exercise context—rugby union in China.  To emphasise, in this thesis I do not attempt to explicitly address a question of how variation activity or culture will alter relationships between joint action, team click, and social bonding hypothesised herein.  Rather, I seek to situate a general account of team click in group exercise in a specific group exercise setting so that theoretical claims pertaining to this account can be meaningfully tested.

 in part theoretical and in part opportunistic.  First, rugby union is a group exercise context defined by dynamic joint action involving multiple agents and engaging a full suite of sensorimotor modalities across various timescales.  Rugby also involves extreme levels of physiological exertion, further contributing to the overall environment of cognitive uncertainty.  These specific parameters of joint action in rugby make it highly suitable as a setting in which to test the theoretical claim that higher levels of uncertainty in joint action will generate higher levels of team click and social bonding in group exercise (Hypothesis 2).

Second,



In this chapter, I situate and contextualise a general account of team click within the specific research context of rugby union in China, which allows me to set the scene for the following chapter (Chapter X), in which I use ethnographic data to more directly assess of the validity of specific theoretical claims.  Following these two ethnographic chapters, I assess results against research hypotheses in order to generate specific predictions to be tested in the survey and field experimental studies (Chapters ~\ref{chap:tournamentSurvey} and ~\ref{chap:trainingExperiment}, respectively).

%These two ethnographic chapters lay the foundation for a progression to an \textit{in situ} survey study of a high stakes National rugby tournament  ($n = 174, male = 90$), and a controlled field experiment ($n = 58, male = 30$) with athletes sampled from two Provincial teams in China (Beijing and Shandong).

%Based on an assessment of ethnographic evidence in Chapter X, I formulate specific preditions that I then test in the survey and field experiment studies.
%\myparagraph{Situating the research through ethnography}

Before being able to address the important empirical question of how Chinese professional rugby players experience joint action, team click, and social bonding in group exercise, it is necessary to first to introduce the relevant details of the specific group exercise context, so that answers to this question become more easily interpretable.  In this instance, the two layers of context most relevant to athletes’ experience of joint action are: 1) the parameters of joint action prescribed by the sport of rugby union, and 2) the broader cultural milieu of contemporary China.  Below I review some of the key considerations for both layers of context that are of relevance to athletes experience of joint action, team click, and social bonding.

%In this chapter I introduce the research setting in which the empirical studies of this dissertation are conducted: rugby union in contemporary China.  In the previous chapters, I outlined a novel theory of social bonding through joint action and suggested its applicability to group exercise contexts.

\section{Rugby Union\label{sect:rugbyUnion}}
Rugby Union (hereafter simply rugby) is an interactional team sport played on a rectangular field (100m x 70m), by two teams of 15 players each, who contest possession of an egg-shaped ball that can be used to score points \citep{IRB2014}.  Descending from a variety of locally-specific folk games played in pre-industrial England, all loosely grouped as ``football,'' rugby developed within the elite public school system as a deliberate physical activity arbitrated by rules and regulations, before circulating through the arteries of Great Britain's colonial empire as a leisurely pastime—--a ``sport'' \citep{Dunning2005}.  In 1996, rugby became a professional sport and is played as such in Western Europe, the Southern hemisphere (Australia, New Zealand, South Africa, and Argentina), and Japan. Rugby sevens---the specific focus of this dissertation---is a modified version of the conventional 15-a-side game involving teams of 7-a-side, and 14-minute games played in a Tournament structure over two or more days (rather than a one-off 80-minute match between two teams).  Rugby sevens has grown in popularity more recently, particularly since its introduction to the Olympics for the 2016 Games in Rio de Janeiro.  More so than the traditional version of the game, rugby sevens is played by countries all over the world, and attracts more balanced participation by men and women.

\subsection{Joint action in rugby \label{sect:jointActionRugby}}
Rugby is a highly interactive and physiologically demanding sport in all forms and at all levels at which the game is currently played.  It requires players to participate in frequent bouts of intense activity at and above the aerobic threshold such as sprinting, physical collisions, tackles, and grappling, separated by short bouts of low intensity aerobic activity such as walking and jogging \cite{Duthie2003}.  Tackling, being tackled, rucking and mauling all involve full-body movements and demand maximal efforts of strength and power \citep{Elloumi2012}.  Athletes require high levels of general and specific strength as well as stability and mobility.  Expressing strength quickly (as power) is required to break through tackles, accelerate at a high speed to make tackles, or jump to catch a ball.  At the elite level in particular, the physiological costs (including the risk and severity of injury) and complexity of joint action requirements of rugby are amplified \citep{Coughlan2011}.

Rugby also requires high levels of behavioural interdependence between team members due to the complexity and uncertainty of interactive coordination tasks. In rugby, almost all action is joint action, in the sense that the performance of almost all technical actions requires either direct or indirect consideration of the position and intensions of other athletes in relation to that action.  Rugby players run with the ball, pass or kick the ball to other attackers or towards open space on the field, enter collisions as either ball-carrier, tackler, or support player, and contest possession of the ball by grappling in ``rucks'' and ``mauls.''\footnote{In distinction to American Football (NFL), in rugby only the defensive team is only permitted to tackle the the ball-carrier from the attacking team.}  Even the most seemingly independent technical operation among these skills---for example, an athlete kicking the rugby ball to open space---requires the consideration of positioning of other athletes.

This brief description of rugby's extreme physiological and cognitive demands demonstrates the suitability of rugby as a group exercise context in which a general account of team click can be tested.
%From this point forward, I use rugby to refer to the version of the game that is the focus of this dissertation, i.e., rugby sevens.

%The parameters of joint action typical of rugby union make the sport highly suited to test the the theory of social bonding through joint action (Chapter~\ref{chap:theory} Section ~\ref{sect:activeInfGE}. Rugby union is an interactive, field-based team sport that requires of its participants high levels of physiological exertion and socially coordinated movement---both factors are known to be linked to social bonding \citep{Cohen2017,Davis2015}.  In addition, the game is anecdotally and colloquially associated with social bonding in many of the contexts in which it is commonly played \citep{Dunning2005}. Thus, the combination of highly complex and costly and physiological and cognitive joint action demands, and anecdotally observable social effects makes rugby union an ideal arena to investigate the predicted explanatory role of team click in a theory of social bonding through joint action.  In this section, I introduce the history of the group exercise context of rugby union, and the specific parameters of joint action typical to the sport.



Almost all action in rugby can be classed as joint action.  From the point of view of the action of a single team, however, joint action in rugby can be grouped into two modes: aspects of \textit{team performance} and aspects of \textit{individual performance}.  Team performance usually entails joint action that requires explicit coordination with teammates (such as coordination of attack and defensive line, passing between two or more players, etc).  Components of individual performance, by contrast, entails a focus on individual actions that contribute to larger joint action goals (such as tackling, individual passing, effectiveness in contact encounters at the breakdown, etc.)




\subsubsection{The ``structure'' of joint action in rugby}
Like many equivalent team sports in which a single ball (or similar object) is contested, such as basketball, association football, and ice hockey, game play in rugby typically involves a series of sub-phases in which attacking and defending subunits of athletes contest possession of the ball \citep{Passos2011}.  This structure requires athletes to continually perform similar joint action schemas with teammates and against opponents.   The highest order of joint action in rugby consists of 14 athletes (seven athletes per team).  These 14 athletes coordinate around the shared goal of completing a 14-minute game in which one team competes against the other team for victory.  Lower order goal-directed joint actions are nested within this overarching frame.  Athletes coordinate their movements around shared goals of attack or defence, depending on which team is in possession of the ball at any given time.

The goal of attacking subunits is to penetrate the defensive line. If the attack is halted by the defence, then the attacking team attempts to re-secure possession of the ball at the point at which the defensive side halts the advance of the attack---known as the ``breakdown.''  A breakdown occurs after a ball-carrier is tackled and brought to ground, and subsequently a contest for possession of the ball is permitted.  The goal of the defensive subunit is to halt the ball carrier and subsequently successfully contest possession of the ball at the breakdown.  Subunits of attack and defence usually require immediate coordination between 2-4 athletes per team.  For example, three attackers attempting to out-manoeuvre two defenders ($n = 5$), or an attacking player running to retrieve possession of the ball at a ruck containing four other players---a ball carrier, a tackler, and one player from each side contesting possession in the ruck ($n = 5$).

Athletes commit to multiple hierarchically nested goals---some of which they share with the entire field of athletes (for example, the shared goal of playing a game, or adhering to the rules of rugby), while other goals they share only with their own teammates (e.g., winning the game, controlling possession of the ball), or with a select subset of their teammates (e.g., coordinating with a teammate in attack to outsmart the defence).  The joint action that pertains to each shared goal plays out over various spatial and temporal scales.  For example, it usually takes two days and multiple facilities to complete a tournament; 15 minutes and one 70x100m rugby field to complete one game; and one second and a confined physical space (as small as 1x1m) for two attackers to complete a pass.  This brief makes clear that the ``structure'' \citep[cf.][]{Keller2014} of joint action in rugby is complex and multi-layered (see Chapter~\ref{chap:theory}, Section ~\ref{sect:structureJA}).


% P: Cognitive threshold for maintaining social bonds
\subsubsection{The cognitive uncertainty of joint action in rugby}
Various properties of joint action typical to rugby serve to threaten the likelihood of achieving successful interpersonal coordination.  In particular, the complexity of rugby's various hierarchically nested joint action goals, the time pressure and cognitive load associated with monitoring, adapting to and predicting on-line and in-the-moment joint action between multiple autonomous actors under conditions of extreme physiological exertion, and the competitive structure of rugby (in which half of participants are pitted against the other half with the explicit goal of foiling interpersonal coordination of their opposition), dictates that joint action in rugby is defined by extreme levels of  uncertainty. These factors, considered from a cognitive point of view, suggests that actually achieving success in rugby's joint action is a highly improbable proposition. As such, the AIF for joint action predicts that the reward structure for team click in rugby will be extremely high (see Chapter~\ref{chap:theory}, Section ~\ref{sect:sect:surprise}).

%As outlined in Chapter~\ref{chap:theory}, current research suggests that humans have devised a number of effective cognitive strategies for establishing and sustaining joint action.  These strategies span a continuum, with cognitively taxing interoceptive predictive modelling on one end, and direct (extra-neural) coupling on the other (see Chapter~\ref{chap:theory}, Section ~\ref{sect:solutionsJA}).

Rugby players---from relative novices to elite professional athletes---demonstrate a capacity to overcome the improbability of success in joint action.  Given the challenges inherent to joint action mentioned above, it is likely that successful joint action in rugby will require of its participants flexibly deployment of cognitive strategies designed to engage an entire spectrum of natural and conventional affordances .  At the same time, however, the on-line and ``in-the-moment'' nature of joint action in rugby, combined with the extreme physiological exertion suggests that cognitive solutions to on-field joint action in rugby will tend to rely on those strategies on the continuum that recruit more automatic action-perception links and direct coupling of movement system components \citep{RKiouak2016,Novembre2014}.


\subsubsection{Individual and team performance in rugby}

The dynamic, interactive, and multi-level structure of joint action in rugby  joint action in rugby can be categorised according to its individual-level and team-level perceptions.  For example, when athletes are participating in attack, athletes may be concerned with elements of individual-level contribution to joint action, such as the quality of their passing technique, or their positioning as ball-receiver, or their ability to make decisions.  In defence, athletes may attend primarily to their one-on-one tackling performance, or their effectiveness in contesting possession in rucks or mauls.  At the same time, athletes may also develop perceptions concerning team-level performance.  Athletes for example may develop an impression of how the team (or a specific subunit of the team) is managing to coordinate the defensive or attacking line, or how well the team is communicating throughout a game, or supporting each other in attack and defence.  Various contextual and individual-level factors may underpin variation variation in the extent to which individuals attend to either level of performance.

In line with theory discussed in Chapter~\ref{chap:theory}, variation in perceptions of team and individual performance could be underwritten by cognitive processes of active inference, such that relatively more attention to individual level of performance could be due to higher tuning of proprioceptive feedback to predictive models; whereas greater attention to team-level performance could be associated with greater tuning of predictive models to exteroceptive feedback.




\subsection{Evidence for team click and social bonding in rugby}
There is very little available evidence to substantiate a direct, empirical link between joint action, team click, social bonding in the case of rugby in particular \citep[but for a discussion, see][]{Davis2015}.  Despite this dearth of empirical research, rugby is a sport heavily associated with the popular interpretation of ``social bonding,'' particularly in all-male social organisation common in the elite educational spaces of England and Commonwealth countries in which rugby originally developed \citep{Dunning2005,Richards2007,Collins2008}.\footnote{Recently, rugby union has been the site of much criticism due to the fact all-male social spaces cultivated by rugby appear to support and sustain hyper-masculine and hyper-normative behaviours, including gender-related violence \citep{Cosslett2014}.}

``Rugby is a game for barbarians played by gentlemen,'' or so the saying goes.
  \footnote{The origins of this oft-cited adage are unclear.  The phrase is supposedly the adopted motto of the British Barbarians Football Club, established in 1890 \citep[34]{Dunning2005}.  The complete phrase reads ``Rugby is a game for barbarians played by gentlemen, football is a game for gentlemen played by barbarians.''  However, official club history cites its original motto as, ‘Rugby Football is a game for gentlemen in all classes, but for no bad sportsman in any class' \citep[vii]{Starmer-Smith1977}.  Some sources attribute the saying to British writer and poet Oscar Wilde (1854-1900) \citep{Fleenor2015}}.
Different inflections on this adage have been reproduced by people in all parts of the world that rugby has reached, including China \citep[see][]{Taylor2010}).  Presumably, this saying has survived due to its ability to tether the nature of rugby's physical requirements with socially valorised moral virtues of fair play, cooperation, and integrity.  The current slogan of World Rugby (rugby's international governing body) is ``Building character since 1886,'' \citep{WorldRugby2017}.  This phrase was presumably chosen to draw an association between the joint action requirements of rugby and the moral and social character that can be generated through participation in rugby.

In sum, details of the physiological demands, joint action complexity, and social-historical trajectory of rugby detailed in this section suggest that rugby is extremely suited to an investigation of the social bonding effects of joint action in group exercise.  Rugby involves various types of complex behavioural coordination between team mates and opposition.  Joint action tasks involve sub-phases ranging from dyadic interaction, to interaction of small groups, to entire teams, and squads.  The coordinated activity at each of these scales could have important implications for social bonding.  In this dissertation, my main theoretical focus is joint action that takes place on the playing field, but I am also attentive to the off-field processes of interpersonal coordination and alignment, which also likely relates to---and affords on---field joint action.



\section{China}

In addition to the parameters of joint action specific to rugby, the various cultural contexts in which rugby is played may also have meaningful implications for behaviour observable in these contexts.  Sporting anecdote indicates that different teams from different places and times appear to play the same game in very different ways, often appearing to embody different ``styles'' of play \citep{Bourdieu1990,Taylor2010}.  It has been demonstrated that shared cultural knowledge can serve to structure joint action scenarios in ways that help ``smooth'' coordination by providing pre-loaded expectations between co-participants \citep{Vesper2017}.  The AIF to suggests joint action is enabled and constrained not simply by the couplings between basal models and their immediate physical affordances.  Rather, successful joint action is contingent upon a snug fit between an entire assemblage of hierarchically ordered expectations and affordances pertaining to personal, cultural, and ecological trajectories \citep{Clark2013}.  Thus, a careful consideration of the culturally specific affordances relevant to coordination of joint action and group membership in China is crucial to subsequent empirical analyses concerning professional rugby in China.
% Vollan2017 - cooperation under authoritarian conditions

Very little direct empirical evidence that links joint action and social bonding within the Chinese cultural context.  However, extensive indirect evidence, from historical and contemporary psychological literatures, suggests ways in which a relationship between joint action and social processes of group formation is uniquely articulated in the Chinese context \citep{Weed2011}.  This evidence suggests that China's specific cultural trajectory contains a unique combination of affordances for joint action.  Thus, although the cognitive mechanisms theorised within the AIF to facilitate joint team click and social bonding in joint action relate to universal (thermodynamic, system-theoretical, and evolutionary) principles, and  are understood to be generalisable across the continuum of human cultural trajectories, it is important to consider the culturally specific contours of these mechanisms and dynamics. In this way, cultural variation can be understood to demarcate fields of affordances which enable and constrain particular regimes of attention and trajectories of behaviour and sociality.  Thus, the AIF offers a powerful compliment to theories of cultural evolution such as CAT, which attempt to account for the various factors of attraction (in addition to Darwinian selection) that dictate population level transmission and fixation of cultural variants \citep[cf.][]{Claidiere2014}.

In the sections below, I outline evidence for the way in which an the cultural context of China (broadly and flexibly construed) uniquely shapes social cognitive processes of action and perception, self-construal and group membership, and institutional norms \citep{Liu2009}.\footnote{The problematics of referring to China as a cultural category (see Liu1995, etc).}

%Indigenous Chinese psychology is the product of a distinct historical trajectory, and has been facilitated by specific socio-cultural institutions.

\subsection{Relational and categorical modes of group membership}

A gradual accumulation of research across the human sciences suggests the existence of non-negligible variation in social cognition according to culture.  Anthropologists have for some time emphasised meaningful cultural variation in processes of social group formation \citep{Strodtbeck1961,Kluckhohn1961,Mead1967,Fei1992}, and more recently cultural psychologists have sought to demonstrate this variation in experimental paradigms \citep{Markus1991,Nisbett2001}.

A core finding from cross-cultural psychology concerns the correlation between cultural variation and modes of group formation.  This research has produced a theoretical spectrum of processes of group membership, the two poles of which are usually described as ``categorical'' and ``relational'' \citep{Hofstede1980,Brewer2007}.  In the case of some cultural niches, traditionally samples from modern and industrialised ``Western'' societies (such as the USA), social identity formation revolves around the autonomy and immutability of social categories of ``self'' and ``group.''  By contrast, in societies in which a ``relational'' mode is dominant (for example, in East Asian countries such as Japan, China, Korea, etc.), social identity is defined primarily according to degrees of social embeddedness and interdependence with others comprising their in-groups\citep{Leung2012}.

Much of Anglo-American social psychology of the 20th Century is rooted in the assumption that the cognition of group membership is universally a categorical process \citep{Liu2005}.
  \footnote{This is of course unsurprising given the fact that experimental social psychology matured into an empirical science in North America following WWII}.
The canonical ``social identification'' paradigm, of social psychology, for example, requires that an individual make an identification between abstract categories of the self and the in-group or out-group \citep{Turner1987}.  Group membership is achieved when the perceived differences between the self and other in-group members are smaller than the perceived differences between in-group and out-group members \citep{Yuki2014}. Categories of self, other, and group are construed as independent and autonomous constructs that moor an individual's perception of social identity.

In distinction to a categorical mode of group membership, a relational mode of group membership requires that individuals attend more predominantly to maintaining and harmonising intragroup relationships, rather than engaging in intergroup categorical comparisons \citep{Yuki2003}.  In a relational mode of group membership, social identity derives less from a calculation of psychological distance between abstract categories of self and in-group, and more a degree of commitment to cultivating a network of hierarchically structured---but more or less self-centred and self-enhancing---relationships \citep{Liu2009,Nisbett2003}.

According to experimental evidence, individuals culturally accustomed to a dominant categorical mode of group membership tend to endorse categories of self and group as psychological realities (in the sense that social categories generate material psychological effects).  Categorical group processes facilitate fast and effective identification with arbitrary minimal groups \citep{Diehl1990,VanBavel2014}, arousal of intrapersonal cognitive dissonance between the self and experimentally constructed in-group \citep{Festinger1957,Stone2001}, higher levels of cooperation with categorically similar strangers in economic games \citep{Yuki2005,Yuki2003}, and greater attention to and memory recall \citep{Buchan2006,Ng2016}.  By contrast, the inverse is usually observed in experiments where relational processes of group membership are made more prominent or salient.  It has been noted, for instance, that minimal group experimental paradigms have had very little (if any) success in East Asian (particularly Japanese) contexts \citep[586]{Liu2009}.  Instead, relational group processes appear to allow for the arousal of cognitive dissonance only when it is constructed interpersonally (as opposed to intrapersonally) between an individual and specific individuals to which that individual is connected by a meaningful social relationship \citep{Hoshino-Browne2005}.  Likewise, individuals with predominantly relational group awareness are more willing to cooperate with and attend to strangers with whom they share relational rather than categorical ties \citep{Ng2016,Yuki2005}.

Both modes of group membership have been shown to shape attention, cognition, and social behaviour \citep{Nisbett2003}. As such, these divergent modes of social cognition could have important implications for the hypothesised relationship between joint action and social bonding in rugby in China.  Nisbett and colleagues suggest that, as a general rule, East Asians subjects:

  \begin{quote}
    ...tend to be holistic, attending to the entire field and assigning causality to it, making relatively little use of categories and formal logic, and relying on `dialectical' reasoning...whereas Westerners tend to be more analytic, paying attention primarily to the object and the categories to which it belongs and using rules, including formal logic, to understand its behavior \citep[291]{Nisbett2001}.
  \end{quote}

Modes of group membership have been shown to vary not only across cultures (i.e., East Asian versus Western European or North American), but also within cultures \citep{Henrich2014}, within social groups (according to sex and personality differences \citep{Yuki2014}), and even within individuals \citep[depending on contextual and situational primes, see][]{Lee2014,Wong2005}.  The prominence of one mode of membership over another in broad ethnic or cultural groups (e.g., The West versus East Asia) appears to be associated with the durable persistence of cultural and linguistic institutions that afford particular patterns of social cognition.  More recent research suggests that context-specific socio-ecological factors such as the level of relational mobility in any given environment may mediate divergent modes of group membership \citep{Oishi2010,Takagishi2014,Yuki2005}.  Thus, while durable, categorical and relational modes of group membership should be understood in a dynamical sense, as attractors (statistical points in n-dimensional space towards which social cognitive processes tend to converge), rather than immutable characteristics of a given behavioural ecology.

Divergent modes of group membership, documented predominantly by cultural and social psychology, are most appropriately understood as social cognitive \textit{tendencies} \citep{Nisbett2003}.  These tendencies are contingent on various socio-ecological factors and afforded by durable cultural affordances such as language and institutional norms.\footnote{There is preliminary evidence to suggest that East Asian subjects are under certain circumstances capable of behaving according to the tenets of a categorical mode of group membership when (experimental) conditions make such an identity adaptive \citep{Hong2000}. The inverse could perhaps be predicted of typically Western subjects.}  The enduring continuity of cultural affordances in East Asia is such that a relational mode of group membership is reliably dominant and observable.  In the following section, I consider the details of Chinese cultural affordances and their relevance to observable patterns of social interaction and behavioural coordination.

\subsection{Tenets of an indigenous Chinese psychology\label{sect:indigPsych}}
%indigenous Chinese psychology
Understanding the cultural contours of the social cognition of joint action in China requires an engagement not only with contemporary findings from cross-cultural social psychology of group membership, but also with what social psychologist James Liu terms an ``indigenous Chinese psychology'' \citep{Triandis1996,Liu2009}.  For Liu, the construct of an indigenous Chinese psychology is important for deepening theorisations of observable behaviour in China beyond the globally dominant Western modes of scientific knowledge production (including productions within human sciences such anthropology, psychology, and cognitive science).  As outlined above, a considerable amount of evidence has amassed within cultural and social psychology to suggest that the existence of meaningful contrasts between samples of ``East Asian'' populations (predominantly undergraduates of Japanese, Chinese, and South Korean universities) and ``Western'' populations (predominantly undergraduates of North American and Western European universities) in domains of attention and perception \citep{Peng1997,Nisbett2003}, psychological construal of social categories of self and group \citep{Markus1991}, behavioural tendencies in social interaction \citep{Yuki2003}, and institutional norms \citep{Liu2017}.
However, Liu argues that theoretical generalisations based on this evidence alone run the risk of being frail to the behavioural diversity observable both between the East Asian nations (Japan, China, Korea among others), and within each individual nation itself (for example, the vast internal cultural variation in China between North and South; East and West \citep[see, for example,][]{Henrich2014}).

%multi-ethnic dynamics??
Liu and colleagues thus argue that conventional social and cultural psychological approaches require bolstering through the utilisation of a ``representational'' account of social psychology \citep{Liu2005}. in which socially shared representations of history are central to creating, maintaining, and changing psychological identity and patterns of social interaction.  The representational account is very much in line with the theoretical framework utilised in this dissertation, in which cultural affordances are understood as causally relevant to social interaction, owing to the informational affordances they contain for enabling and constraining certain patterns of behaviour.  The task of testing the causal relevance of cultural affordances for the social cognition of joint action poses a practical challenge, however, owing to the rich diversity of cultural variants emanating from Chinese history.

Despite such diversity, both Chinese and Western institutions of knowledge production (academia, media, and so on) generally cohere around common tenets of an indigenous Chinese psychology.  In particular, it is generally agreed that contemporary China is the product of an ongoing interaction between two distinguishable historical processes: 1) two millennia of cultural continuity associated with the ancient development of a singularly successful, multilingual Chinese civilisation, and 2) a more recent engagement with---including, importantly, perceived sufferings and failings at the hands of, and hopes of rejuvenation within---global activities of commerce, governance, knowledge production, nation-building, and international relations \citep{Liu2009}.  In essence, the argument is that these two historical processes have produced a number of indigenous socio-cultural institutions that support the representational affordances and social behavioural tendencies of an indigenous Chinese psychology.

Of these affordances and tendencies, the most relevant to the social cognition of joint action are: 1) ethically-prescribed Confucianism, 2) formidable legal and bureaucratic systems of state governance, and 3) the modern creation of Chinese nationalism, fuelled by the revolutionary spirit of both Marxist/Leninist dialectical materialism, as well as faith in scientific and technological advancement \citep{Barme2009}.  I will expand on these components of an indigenous Chinese psychology below.

\subsubsection{Confucianism}
Confucian values pervade all levels of social interaction in Contemporary China. What is understood as Confucianism in contemporary discourse emerges from deep historical roots in folk-cultural axioms \citep{Wang2009}, agricultural modes of production \citep{Talhelm2014,Fei1992}, dynastic rule, and modern reinventions of these cultural forms by contemporary processes of governance and knowledge production by the nation-state \citep{Hwang1999,Liu2014}.  Confucian philosophy provides a number of resources for directing and harmonising social interaction, all of which stem from an epistemology of holism.

Confucianism, broadly construed, entails three main dimensions relevant to the social cognition of joint action:

\begin{enumerate}
  \item \textbf{Holistic reasoning} is commonly employed to harmonise social relationships and avoid interpersonal conflict, via the principle of the  ``middle way'' \textit{zhongyong zhidao} 中庸之道).
  \item \textbf{Hierarchical relationism} acts as a guide for managing hierarchically organised familial and social networks, known as ``guanxi'' (\textit{guanxi} 关系).
  \item \textbf{Personal cultivation} is employed as a means of accumulating ethical virtue, or ``human heartedness'' (\textit{ren} 仁).
\end{enumerate}

These three dimensions of Confucianism can be identified in cultural affordances that pervade all spaces of contemporary Chinese social life.  In the context of the social cognition of joint action, Confucianism can be understood as one dominant hyper-prior or coordination smoother for joint action in contemporary China.  I review the details of these three dimensions below.

\myparagraph{Holism}
Holism is an epistemology that originated in Daoism before being reappropriated originally by Confucius (551–479 BCE) and then later by a lineage of Confucian scholars during the Warring States Period.  Holism involves less emphasis on reason in the Western epistemological sense (i.e., the search for ultimate knowledge via reduction), and instead suggests that manifest and latent aspects of reality come in and out of being through an interaction between the ``Receptive'' (\textit{Yin} 阴) and ``Creative'' (\textit{Yang} 阳) principles inherent in the universe.  The interaction of these two fundamental principles (rather than the deterministic, causal principle of a single, omnipotent God, for example), give birth to ever-changing phenomena.  The postulate of holism means that in Confucian thinking, ``it is the dynamics among the elements, rather than the elements themselves, that serve as the primary units of analysis'' \citep[156]{Ji2010}. Above all, holism is commonly employed to explain human social behaviour.  In the case of knowledge of social interaction, Holism supports the reasoning that the focal point of interest should not be unchanging human biology, but rather the dynamic and evolving patterns of family, groups, society, and culture.  The institutionalisation of holistic reasoning in Chinese society via thousands of years of continuous civilisation has implications for contemporary social interaction \cite{Nisbett2003}.


\myparagraph{Hierarchical relationism}
Hierarchical relationism builds on the dialectical reasoning of holism to provide cues and directives for managing reciprocity and responsibility in particular kinship and extra-kin social relationships \citep[\textit{renqing} 人情][]{Maehr1980}.  In this system, the individual stands simultaneously in several different relationships with different people: as a junior in relation to parents and elders, and as a senior in relation to younger siblings, and students. While juniors are considered in Confucianism to owe their seniors reverence, seniors also have duties of benevolence and concern toward juniors. The Five Relationships are: 1) ruler to ruled, 2) father to son, 3) husband to wife, 4) elder brother to younger brother, and 5) friend to friend. Four of the five key relationships concern unequal hierarchical relationships, the only exception being friendship, in which reciprocity and respect is emphasised.  Notably, none of the prescriptions concern strangers.

This prescription for social interaction amounts to a strong contrast to the practical ethics dominant in Western Europe and North America, which blends Enlightenment philosophy and Christian values to emphasise egalitarian concerns such ``respect for thy neighbour'' and Good Samaritanism \citep{Liu2005}.  These ideals rely crucially on internalisation of abstract categories of self, society, and citizenship.  In a morally centred Confucian worldview, by contrast, deeds done in the context of a father-son relationship, as compared to that between siblings, or between ruler-ministers are interpreted through different moral lenses \citep{Liu2011}.  A universal morality applying to all situations and individuals (e.g., Kant’s categorical imperative) is not only impossible, but also undesirable \citep{Bedford2003}.

Writ large to the public domain of social interaction and exchange, the principals of hierarchical relationism appear to have stabilised as three layers of ``treatment'' in social Relationships \citep{Liu 2009}.  Hwang’s (1987) work on face and favour hypothesises that Chinese have a precise set of rules to govern resource allocation and exchange: 1) close, affective ties are governed by a need rule, 2) mixed ties by a \textit{renqing} or human relations rule, and 3) instrumental, or distant relationships are governed by a fairness rule.  Such a set of rules for resource exchange in social interaction allows for maintaining committed social relationships (what you need I will give you, and what I need you will give me), but also for using a tit-for-tat type of fairness rule to govern the expansion of the social network to accommodate new members, initially by an instrumental fairness rule, and then over time to a \textit{renqinq} or human relations rule.


\myparagraph{Self-cultivation}
The dominance of concerns for the social domain in Confucian ethics can give the impression that attention to the group takes precedence over attention to the individual self.  This line of thinking, while potentially commensurate with some psychological and anthropological evidence deriving from modern Japan \citep{Kitayama2010b}, is a common misconception of indigenous Chinese psychology \citep{Tu1998}.  In fact, whether applied to Japan or to China, constructing a mutually exclusive individual-versus-group binary is problematic because it derives from the (Western) epistemological standpoint, in which the key psychological processes of group behaviour are understood to be categorical, rather than relational.  Categorical group membership operates through a process of ``de-personalisation,'' whereby individuality is temporarily submerged within conformity to a group proto-type containing idealised characteristics of the group \citep{Turner1987}.  By contrast, cultural niches in which a relational mode of group membership is culturally dominant, often promote active cultivation and promotion of self as ultimately a virtue of group membership.

Experimental evidence has demonstrated that self-consistency is not as important in East Asian cultures compared to Western cultures. \textcite{Suh2002}, for example, showed that East Asians described themselves much less consistently across situations than did European Americans.  \textcite{Kitayama1999} have also noted that in East Asian cultures (Japan), the self-consistent person is not evaluated as favourably as is the person who adjusts his or her behaviour to fit the demands of the situation.  This evidence suggests that whereas stability and consistency is central to patterns of self-construal common in the West (i.e., in categorical modes of group membership), flexible relationship management is more central to patterns of self-construal in the East \citep{Nisbett2001}.

In Confucianism, cultivation of the self is central to accumulating social virtue (\textit{ren} 仁), and vice-versa \citep{Hwang2012}. As Tu explains, ``one's ethical intelligence will only be enhanced when the mind is properly cultivated.''  In this conception, acts of self-cultivation, rather than being interpreted as anti-social, are instead often celebrated as expressions of pro-sociality.  Within the context of Confucian ethics---underpinned by dialectic holism and hierarchical relationism---self-cultivation arouses little or no dissonance between categories of self and group because these categories are generally understood to be fluid and flexible rather than autonomous and immutable \citep{Nisbett2003}.  Far from being a derogated act (as in the tall poppy syndrome common cultural milieus such as Australia!), in Confucianism, self-cultivation is understood as an appropriate and even ultimate path or ``way'' towards sagely transcendence \citep[106]{Hwang2012}.

The notion of self-cultivation has been widely interpreted and revised through many waves of Confucian scholarship. As such, prescribed practices of self-cultivation are wide-ranging and variably interpreted in contemporary Chinese society.  Self-cultivation and can include anything from self-regulation of emotions, including socially-derived emotions of shame (a.k.a., ``face'') \citep{Matsumoto2012}, to cultivation of vital bodily and spiritual ``energy'' (\textit{qi} 气) through daily activities revolving around diet and exercise \citep{Farquhar2012}, to dedication to, and gradual mastery of, scholarly and artistic endeavours \citep{Slingerland2000}.

In the context of an indigenous Chinese psychology, in which the dominant organiser of social attention is maintenance of harmony in particularistic relational networks, the categorical boundaries between self and group are more fluid and flexible.  In this regime, the self can serve as a useful vehicle for overt pro-social behaviour, without arousing conflict between other's interests.\footnote{Early Confucian thought contained a deep skepticism and suspicion towards the idea of an abstract social space in which individuals ostensively diminish emphasis on their own particularistic interests and instead attend to the creation of an egalitarian collective identity.  This proposal for social organisation was understood by Confucius to be insincere and vulnerable to exploitation \citep{Bollas2013}.  This emphasis is captured by one of the most well known quotes from the Analects, in which Confucius urges a primary focus on the self before attending to the activities of the social sphere beyond: ``Don’t complain about the snow on your neighbour’s roof when your own doorstep is unclean'' (各人自扫门前雪,莫管他人瓦上霜) \citep{Leys1997}.}

When intersecting with an individual's role-based moral obligations deriving from hierarchical relationism, self-cultivation can take the form of overt social acts of leadership \citep{Woods2011}.  \textcite{Farh200} note that a long history of ``paternalistic leadership'' in China is underwritten by an intertwining of Confucian and legalistic traditions \citep{Farh2000}.  Defined as a ``a style that combines strong discipline and authority with fatherly benevolence and moral integrity couched in a personalistic atmosphere'' \citep[91]{Cheng2004}, paternalistic leadership promotes an ethic in which the individual acts in a dignified manner, exhibiting high self-confidence, tight control of information, and an unwillingness to delegate authority \citep{Liu2003}.

\subsubsection{Civil institutions}
A second, and less emphasised dimension of the Chinese cultural trajectory also responsible for shaping social interaction is the formidable legal and bureaucratic institutions.  China's legal and government institutions have served to administer state control and power over the particularistic social interactions in both dynastic and modern history \citep[]{Liu2017}.  In practice, Confucian ideals of benevolence and morality were accompanied in Imperial China by a comprehensive, well-articulated, but also draconian legal system that functioned to limit and regulate the particularistic social activities of hierarchical relationism \citep{Fitzgerald1985}.  Magistrates and the law were revered, and the legal system was regarded legal more as a system of last resort than a method for ordering daily social life \citep{Liu2009}.  In addition, Chinese society was rare and distinctive among ancient civilisations in practicing bureaucratic meritocracy, with examinations that allowed ordinary people to become officials of state (thereby increasing the fortunes of their entire clan).  Nevertheless, meritocracy made Chinese high culture esteemed, and created a psychology where education was cherished as a primary marker of Chinese identity, and a route of upward mobility \citep{Spence1990}.\footnote{Unlike practices in ancient Rome, where citizenship, as the official marker of Roman identity, was a legal matter of heredity and wealth, there was no formal categorical demarcation between citizens and non-citizens in ancient China.  Rather, there were degrees of power and practice between the high culture of the Confucian elite and the low culture of the peasant class \citep[see][]{Liu2014}.}.

China's cultural historical trajectory is one which the state functioned not as a union or federation in which independent interests engaged in exchange and debate.  Rather, the Chinese state operated as a benevolent authority that sought to regulate particularistic activities via extensive (and at times draconian) bureaucratic mechanisms.  As I explain in the ethnographic sections of this dissertation, this specific affordances helps to explain, for example, the fact that institutions in China appear to function more as platforms for the performance of relational social processes (with limit incentives and boundaries to shape behaviour), and less as spaces that define those processes themselves (or indeed define the behaviour of members of the institutions).

\subsubsection{Chinese Nationalism}
Finally, representations of Chinese nationalism are central to an indigenous Chinese psychology, and have specific implications for the practice of modern sport in contemporary China.  Chinese nationalism is a comparatively recent development in Chinese history, and one that has been conceived by Chinese (as well as non Chinese) elites and intellectuals in a fraught context of suffering and failures at the hands of Western imperialism over the last two centuries \cite{Liu2009,Liu1995}.  Initial responses by Chinese elites to the collapse of traditional Chinese civilisation at the hands of Western Imperial influence was to problematise Chinese social and ethical virtues as flawed and backward \citep[see][]{Levenson1959}.  The Confucian vision of state as ``family and other particularistic social relations writ large'' became viewed as one of the reasons why Chinese people were unable to mobilise successful resistance to European and then Japanese invasion.  Thus, beginning in the 19th century, and culminating with the ``New Culture Movement'' (\textit{xinwenhua yundon} 新文化运动), generations of intellectual and political leaders in China sought to instil the virtues of nationalism and patriotism to Chinese people as a means of mobilising resistance to external aggression \citep{Pye1996}.  As I explain in more detail in Section ~\ref{sect:introEnMasse} below, this project led to the importation \textit{en masse} of novel institutions and social practices, including modern sport.

As Liu explains, processes of interaction with Western imperialism and the construction of Chinese nationalism had an important impact on indigenous Chinese psychology:

\begin{quote}
  In the Treaty-port concessions, all members of Chinese society, Manchu or Han, nobility or commoners, were considered as inferior by Westerners.  Such a categorical system where one group identity transcends all other social positions like class, gender, and profession, was a shock for Chinese, whose traditional model of social relations is more consistent with role theory (Stets and Burke, 2000) than this type of group-based self-categorization...foreign imperialism threatened not just the ruling class of China, but the very existence of Chinese culture itself, a nationalistic form of categorical consciousness was required for survival.  \citep[584-5]{Liu2009}
\end{quote}

Since initial problematisation and contestation of Chinese national consciousness by various Chinese and non-Chinese actors at the end of the 19th and beginning of the 20th century, from the mid 20th century onwards, the Chinese Communist Party (CCP) have taken on the self-described mandate to bring about the ``great rejuvenation of China'' (\textit{zhonghua fuxing} 中华复兴).  This project has undergone various iterations since its conception in the early 20th Century.

The advent of Chinese nationalism contained a problematic double bind.
The humiliations and failures of traditional Chinese virtues at the hands of Western Imperial rule meant that Chineseness was problematised by intellectuals and elites.  At the same time, these same intellectuals and elites were intent on celebrating Chinese national identity in order to participate in a in an emerging world of nations.  Sinologist Geremie Barme suggests that the embrace and popularisation of communism and Marxist/Leninist dialectic materialism by the CCP served to reconcile the potentially crippling symbolic contradiction inherent in the project of Chinese nationalism. In this sense, Marxist-Leninist thought functioned ontologically as a replacement to Confucian holism, allowing Chinese elites to neutralise the tension between traditional Chinese values and Chinese nationalism \citep{Barme2009}.  In addition, the active historical framing of the Qing dynasty as an ethnically Manchu (in distinction to a Han) project also serves to release a majority Han Chinese population from implicit culpability for humiliations and failings \citep{Barme2015a}.

Upon this foundation, the CCP has built a number of iterations of Chinese national identity, under a series of different headlines propogated by its central leadership. These headlines began most notably with the Maoist politics of cultural revolution (\textit{wenhua dageming} 文化大革命), before Chinese nationalism was drastically reimagined, following the death of Mao, in an embrace by Deng Xiaoping of ``(economic) reform and opening up'' (\textit{gaige kaifang} 改革开放).  After Deng, Jiang Zemin and Hu Jintao each offered additions to this general project of ``Socialism with Chinese Characteristics'' (中国特色社会主义), such as the promotion of ``Scientific Development'' (\textit{kexue fazhan} 科学发展),and now Xi Jinping's thought on ``Socialism with Chinese Characteristics for a New Era'' (习近平新时代中国特色社会主义思想). In the era of Chinese nationalism following the death of Mao, Confucian values have made a resurgence into the official rhetoric of CCP leadership and Chinese national identity \citep{Billioud2007}.


\myparagraph{Team click and social bonding in China}
While the egalitarian assembly of ``team'' is a relatively novel cultural affordance in China \citep[see][]{Morris2004}, historical evidence suggests that behavioural coordination in joint action in particular has been the focus of thorough ethical consideration.  Sinologist and cognitive scientist Edward Slingerland (2014) locates a key role for joint action in resolving social tensions inherent in collective action problems.  Common to many collective action problems is the threat of free-riding and defection \citep{Cosmides2013}.  As Slingerland explains:

\begin{quote}
   ...the logic of civilised life makes us very keen to distinguish reliable cooperators from unreliable defectors...what we want then is a particular type of desirable...behaviour where there is absolutely no gap between action and motivation.  \citep[192]{Slingerland2014}.
\end{quote}

The paradoxical Chinese idiom of \textit{wei wuwei} (为无为), translated awkwardly into English as ``trying not to try'' or ``effortless action,'' is promoted in many ancient Chinese philosophical corpuses as a behavioural solution to conflicting incentives involved in coordination problems: effortlessness in performing socially desirable actions (virtues) communicates a level of mutual trust and commitment that is ``hard-to-fake.''  The phenomenology of effortless action is described at length in many of these corpuses, and is a key dimension of many traditional Chinese martial arts such as \textit{Taiqi} and Wushu \citep{Morris1998}.  While mass participation in Anglo-American competitive sports is only a recent phenomenon in China \citep{Brownell2008}, the social valorisation of certain types and styles of movement, particularly the phenomenology of flow in (joint) action, appears to enjoy a long history.

Just as click in joint action has traditionally been afforded in China by resources other than the concept of team, social bonding in China appears to be mediated more predominantly by relational and familial (as opposed to categorical) affordances.  Beyond the specific mandates of the five relationships in Confucianism (four of which are familial and none of which concern strangers), the metaphor of family dominates many extra-kin social relationships between individuals, and between the individual and the state.  The state is represented in Confucianism as family and other particularistic social relations writ large \citep[579]{Liu2009}.  Traditionally, the father-son kinship relationship is utilised to naturalise the socially constructed relationship between lord and minister.  As Slingerland explains: ``Parents naturally love their children, and children naturally love and respect their parents, and they both know that they're stuck with one another no matter what...'' \citep[178]{Slingerland2014}. Likewise, the metaphor of family is often utilised to naturalise the notion of an extra-kin social group, for example a company or a sporting team \citep{Brownell2008}.
It is important to emphasise that, whereas (Anglo-American) notions of group membership generally hinge upon a psychological representation of egalitarian equivalence between immutable categories of self and social group, the utilisation of the family metaphor for group processes relies on the individual identifying group membership through a participation in hierarchically structured (familial-like) \textit{relationships} \citep{Fei1992}.


\subsubsection{Summary and predictions based on cultural affordances of contemporary China}
In the sections above, I outlined three core affordances of indigenous Chinese psychology that could play important roles in shaping and enabling particular patterns of social cognition in contemporary China.  Long standing and culturally institutionalised Confucian values---supported by a legal and bureaucratic institutional back-bone---has facilitated the development of a social identity practiced within a context of particularistic social relations.  Subsequently, two centuries of confluence and antagonism between Chinese and Western modernity has facilitated the emergence of more categorical forms of binary inclusions and exclusions consistent with other national identities in the global system of nation-states \citep{Liu2009}.  Evidence from cultural psychology suggests that these tenets of an indigenous Chinese psychology have direct influence on attention and action \citep{Nisbett2001}.  As such, these affordances will shape the trajectories of behaviour and experience in joint action in China in patterned and predictable ways.

It is possible to predict that under conditions such as those in contemporary China, in which a relational mode of group membership is made salient, dominant, or adaptive, subjects will preferentially attend to the management of social relationships as a behavioural priority.  In the case of a group setting such as that encountered in a team sport like rugby, individuals may attempt to to harmonise relationships with in-group members according to Confucian principles of holism, hierarchical relationism (guanxi), and self-cultivation to the extent that the bureaucratic parameters of the institutions allow.  These tendencies will be more salient and observable than tendencies to endorse social and institutional categories as psychological realities. From the point of view of the AIF, these predictions are consistent with the role of cultural affordances in shaping regimes of attention and loops of action and perception \citep{Ramstead2016,Roepstorff2010}.







\section{Rugby in China}

\subsection{Physical activity in Chinese modern history}
The history of sport and exercise in China is in many fascinating ways emblematic of broader processes of China's modernisation.  The introduction of modern sport to China was itself a project explicitly designed to be a vehicle for transforming China into a modern nation state \citep{Morris2004}.  As such, sport in China offers a highly visible public arena of human activity, in which the often fraught and turbulent processes of China's modern history were choreographed and performed.

China's interaction with sport represents a significant engagement with novel physical cultures of movement and interaction.  Prior to the introduction of Anglo-American interactional team sports and Northern European calisthenics in the late 19th century, physical activity that could be grouped in the broad category of sport and exercise was limited to a small collection of physical cultures indigenous to China, practices imported much earlier in history from south Asian religious traditions (e.g. Buddhism and Hinduism), and some activities imported from non-Han dynasties such as Manchurian horse racing in the Qing dynasty.  Historical records suggest that participation in almost all of these activities was limited to the imperial ruling classes and urban elites \citep{Ge2005}.  Thus, the emergence of modern sport and exercise in China is entangled with a history of interaction and conflict with, and embrace of foreign influence during the 19th, 20th, and 21st centuries.  In this section, I sketch the main transitions of this history in order to situate rugby in them.

  \subsubsection{Introduction, rejection, and embrace of sport and exercise in China 1842-1912 \label{sect:introEnMasse}}

The history of modern competitive sport in China is born out of the history of trade and mercantilism between China and Western Europe.
By the beginning of the 19th century, China had established flourishing international trade relationships with Western empires.  Generally speaking, China traded in tea, silk, and porcelain, meeting high demand in emerging middle class households of newly industrialised nations of Great Britain, France, and Germany.  In return for these goods, merchants from Western empires initially paid in silver, until it became clear that a trade imbalance was emerging between China and the West, owing to the fact that tea, silk, and porcelain where much more renewable than silver \citep{Fay2000}.  By the late 19th century, Western powers and merchants began to seek mechanisms through which this trade imbalance could be rectified.  The spike in demand for opium among China's urban middle classes offered trade merchants a convenient opportunity to trade with China using a currency much more replenishable (and addictive) than silver.

By the mid-19th century, however, a series of conflicts arose between Qing dynasty rulers and the Western empires over the regulation and trade of opium, leading to the first Opium War (1939-1942).  The resulting Treaty of Nanking (1842), imposed upon the Qing dynasty by a league of eight imperial powers, initiated a period of colonial occupation of China that drastically weakened the political power and legitimacy of the Qing Dynasty and increased the influence of foreign powers within China's major inland and port cities.

Increased foreign influence in China during this period brought with it the introduction of an array of ideas and practices to China's urban elite ruling classes, including novel physical cultures of dress, adornment, leisure, and not least, physical exercise.  Increasingly popular at the time in Europe and North America was the belief that sport and exercise were important pedagogical tools in the development of physically and mentally strong subjects of post-Enlightenment modernisation \citep{Elias1986}.  This belief gelled with the values of a nationally (and internationally) motivated Chinese urban elite, and subsequently manifested in China in the promotion of Anglo-American competitive sports by North American Christian missionary organisations such as the Young Men’s Christian Association (YMCA), and the incorporation of Northern European calisthenics routines into military exercises \citep[240]{Morris2004}.  Such techniques were soon popularised within elite intellectual communities as pedagogical tools designed to foster an explicit link between the strength of the physical body and the strength of the Chinese nation \cites[32]{Morris2004}[49]{Brownell1995}.

The introduction of novel political ideas to China in the late 19th century involved importation \textit{en masse} of novel linguistic, cultural, and social categories and practices from the West and Meiji Japan (1868-1912) \citep{Liu1995}. \textit{Tiyu} (体育), the term in modern Chinese that most closely translates to sport, was one of many neologisms inherited from the Western social sciences via its Japanese translation.  Importantly, tiyu encompasses more than just the modern Anglo-American competitive sports (roughly translatable to ``yundong'' (运动) that the English word connotes.  Instead, the modern Chinese notion of sport refers to an entire culture and discipline of the body that is deeply intertwined with the political project of Chinese modernisation and advancement \citep{Morris2004}.\footnote{As Lydia Liu (1995: 58) points out, even the notion of ``China'' itself as a term linked to a national imaginary, only began to emerge as such through interaction with Western missionary discourses concerning ``China'' during the 2nd half of the 19th century.}

The two-character phrase \textit{tiyu} is a contraction of a longer four-character phrase \textit{shenti jiaoyu} (身体教育), a direct translation of Herbert Spencer’s notion of  ``a physical education'' that first appeared in Chinese reformist intellectual Yan Fu’s ``On Strength'' 1895 \citep[9-10]{Morris2004}.  Now naturalised within the modern Chinese vernacular, compound words such as the ``body'' (\textit{shenti} 身体) and ``education'' (\textit{jiaoyu} 教育), as well as other fundamental conceptual social categories such as ``society'' (\textit{shehui} 社会) and ``culture'' (\textit{wenhua} 文化) all first appeared in their modern form as an arsenal of translated neologisms made popular by Chinese intellectuals in the late 19th and early 20th century who were grappling with ways to transform a dynastic realm crippled by colonial occupation and feudal backwardness into a strong nation in a system of modern nations \citep{Pusey1983;Liu1995;Huters2005}.   Convinced that the body, \textit{shenti}, was crucial to realising this transformation, intellectuals rigorously subscribed to the Spenserian ideal of a physical education, which combined with a moral and intellectual education, as a ``cultivation of the whole moral, intellectual, physical, and aesthetic self'' (\textit{dezhitimei quanmian fazhan} 德智体美全面发展) \citep[10]{Morris2004}.

%\subsubsection{The Boxer Rebellion}
Novel regimes of sport and exercise were not uniformly embraced in China during this period, however.  The most organised movement of anti-foreign, anti-colonial, and anti-Manchurian resistance during the period 1842-1912 came in the form of an army of Han Chinese subjects from rural areas of Shandong was known as the ``Militia United in Righteousness'' (\textit{yihetuan} 义和团).  The Militia trained in martial arts and meditation and explicitly rejected Western forms of physical culture such as sport and exercise \citep{Brownell2008}.  The Militia's first target in the ``Boxer Rebellion''
  \footnote{1900-1901, members of the Militia were known in English as the ``Boxers,'' due to the fact that they had been practitioners of martial arts that included boxing.}
was the Horse Racing track in Beijing situated at the south gate of the Temple of God of Agriculture---a site that had become heavily associated with the activity of Manchurian ruling class and the influence of foreign legions.

In a harsh and somewhat ironic twist of fate for the Militia, the influence of modern sport infiltrated Beijing's sacred sites even further in 1901, when the troops of the Eight-Nation Alliance (United Kingdom, The United States of America, Germany, Japan, Russia, Italy, and Austro-Hungary) arrived in Beijing to relieve the besieged international legions.  During this period, troops from the UK occupied the Temple of Heaven to the east of Yongding gate, while US troops occupied the Temple of Agriculture to the west \citep{Brownell2008}. Compared to many other public spaces in Beijing, the flat, open expanses of Beijing's temple grounds were conducive to the playing of Western sports common in garrison life such as field hockey and association football.  Throughout the final years of the Qing Dynasty, the sporting activities organised within the grounds of the Temple of Heaven and Temple of Agriculture attracted the participation of troops from other legions, as well as Western missionary organisations such as the YMCA, and local Chinese elites and revolutionaries \citep{Steel1985}.

It was in this period of the late 19th century and early 20th century that modern sport as a cultural practice first fixated in China. However, this first wave of introduction was extremely patchy and riddled with conflict.  The spaces in which sport was played were extremely limited to urban, elite, and foreign enclaves, and the conditions under which these processes occurred were often dictated heavily by the will of foreign influence.  As I explain below, it was not until the revolutions of the early and mid 20th century that sport and exercise became inducted into Chinese state processes and reached populations in China beyond urban elite centres.

\subsubsection{Republican Era (1912-1949)}
The combination of a weakened dynastic regime and the influx and development of new political and social ideas among China's urban intellectual elite led eventually to the Chinese revolution in 1911, and the establishment of the Republic of China (ROC) in 1912 \citep{Mitter2008}. During this initial period, sport in China began to develop among urban elite along two main strands—--``competitive sports'' (\textit{jingji tiyu} 竞技体育) and ``games and calisthenics'' (\textit{ticao} 体操).  Traditional Chinese martial arts also made a resurgence during this period.  Although initially delegitimised within the New Culture Movement (\textit{xinwenhua yundong} 新文化运动)as elements of feudal superstition, members of the National Essence Movement (\textit{guocui yundong} 国粹运动) reappropriated martial arts by deliberately aligning these practices with rational modernist ideas about sport and the body, at a time when sport became more significant for the articulation of an emerging national identity in the face of imperial powers \citep[38]{Brownell1995}\citep[45]{Morris2004}.  A significant aspect of competitive sports was the public spectacle of the ``games meets'' (\textit{yundonghui} 运动会), in which the performance of emerging national and international political identities could take place.  As early as 1908, the Chinese sport community enshrined the Modern Olympic Games (\textit{Aolinpike yundonghui} 奥林匹克运动会) as the pinnacle of participation in an international community of nations, and as such, the quadrennial global ritual has since preoccupied a collective Chinese sporting consciousness, and a Chinese national consciousness more broadly (\citep{Jarvie2008;Barme2009;Brownell2008;Morris2004;Xu2008}.

The most southern point of Beijing's sacred north-south axis continued to feature prominently in the development of sport in the ROC.  In 1914, the YMCA organised the first ever multi-sport event within the walls of the Temple of Heaven in Beijing, which the ROC's nationalist government later labelled the Second National Games (the label of the First National Games of the Republican era was retrospectively assigned to the ``First National Athletic Alliance of Regional Student Teams'' multi-sport event hosted by the YMCA in Nanjing in October 1910) \citep[441]{Li2015}. The active reappropriation of the sacred spaces of the Qing Dynasty through activities such as team sport, allowed the foreign legions and local Chinese elites to perceive their occupation of these sites as a demonstration of superiority over the Qing court, and facilitated the ROC's priority of displacing traditional reverence of the throne \citep{Hevia1990}.

Beijing lost its importance as the centre of state power when the ROC moved the capital to Nanjing between 1928-1948.  In 1937, however, after the space surrounding the Temple of Agriculture had been repurposed for a range of leisure and amusement activities (including sport), a large sports stadium was constructed on the land directly to the south east of the main altar.  With a capacity of 10,000 people and a dirt association football field in the middle, the Beiping Public Stadium (as it was originally named) was the second ever modern sports stadium to be built in the ROC---the first being built two years previously in Shanghai for the hosting of the National Games in 1935.  The grounds immediately surrounding the stadium subsequently became the site for the Beijing Municipal Elite Sport Training facility, the predecessor to the current Institute.

By the mid 20th century, sport in China and the explicit pedagogy of physicality and modernism that adherents to sport in China seeked to promote, were far from universal or hegemonic during the Republican era \citep{Morris2004}. Sport and exercise existed predominantly as an urban, elite and male-dominated activity.  Nonetheless, these developments laid the foundation for a more deliberate and widespread adoption and institutionalisation of sport by the CCP after 1949.

\subsubsection{Sport in the People's Republic of China (1949-1976)\label{sect:sportPRC}}
Sport was transformed and politicised in radical ways following the establishment of the People’s Republic of China in 1949 (hereafter PRC).  As a key member in the New Culture Movement and the May Fourth Student Revolution, CCP Chairman and first President of the PRC Mao Zedong was a proponent of the Spenserian logic of physical education.
  \footnote{A testament to Mao’s personal commitment to the tiyu project is the subject of his first ever publication in the New Culture movement's magazine New Youth (\textit{Xin qingnian} 新青年) entitled ``A Study of Physical Education'' (\textit{Tiyuzhiyanjiu} 体育之研究) (1917).}
When the CCP took power in 1949, the proletarian body and the propagation of an ideology of active cultivation of the physical body was centralised in CCP propaganda \citep[58]{Brownell1995}.  The body of the worker (\textit{gongren} 工人), peasant (\textit{nongren} 农), and soldier (\textit{bingyuan} 兵员), as well as the body of the athlete (\textit{yundongyuan} 运动员), were glorified for their ``capacity for manual labour'' (\textit{laodongli} 劳动力 )---the ideological foundation for the ``proletarian revolution'' (\textit{wuchanjieji dageming} 无产阶级革命) (Ge and G. 2005: 91).  The ``emancipation ''(\textit{fanshen} 翻身) and glorification of the physical, labouring body is particularly explicit in the propaganda posters of the early Mao era \citep[87]{Ge2005} (see Figure ~\ref{fig:motherlandStrength}).

    \begin{figure}[htbp]
      \includegraphics[width = \linewidth,scale=.7]{images/motherlandStrength.png}
      \caption{Strengthen Physique to Defend Motherland (1950)}
      \label{fig:motherlandStrength}
    \end{figure}

Along with many other facets of society after 1949, sport was institutionalised in line with Soviet bureaucratic models of governance.  In 1952 the ``State Sports (and Physical Culture) Commission'' (\textit{guojia tiyu yundon wieyuanhui} 国家体育运动委员会) (hereafter the Sports Commission) was established, which acted as the central State organ responsible for the administration of ``sport for the masses'' (\textit{qunzhong tiyu} 群众体育), ``physical culture education'' (\textit{tiyujiaoyu} 体育教育), as well as an elite competitive sport (\textit{jingji tiyu tixi} 竞技体育).  The competitive sport system was designed with the intention of creating a fast track for the development of world class athletic talent, in lieu of a sports system as advanced as other (predominantly Western) nations, whose development pathways for athletes were more organically embedded within existing social and educational institutions \citep{Brownell2008}.  By creating model athletes capable of performing and advocating the healthy, egalitarian and militaristic body promoted by the Party, competitive sport was designed to kick start more widespread engagement in ``sport for the masses'' and ``sport education''\citep[56]{Brownell1995}.

\begin{figure}[htbp]
  \centering
  \includegraphics[scale=.5]{images/maoXNT.jpg}
  \caption{Mao Zedong congratulating members of the St Petersburg Zenit FC following a fixture against China in 1952. Source: baidu.com}
  \label{fig:maoXNT}
\end{figure}

The reinstatement of Beijing as China's capital immediately following the establishment of the People's Republic of China in 1949, had immediate implications for the Institute situated at the Temple of God of Agriculture.  The existing stadium was enlarged to a capacity of nearly 30,000, and lights were added to enable hosting training and events at night.  The stadium was host to many important sporting and political events between the years of 1949 to 1976, including a number of International football matches attended by high profile CCP members, including Mao Zedong and Zhou Enlai (see Figure ~\ref{fig:maoXNT}).

Despite becoming heavily entwined with political processes of the PRC during 1949-1976, sport and its development also became severely hampered by many factors during this period.  On the one hand, the internal political, social, and economic chaos of The Great Leap Forward (1955-58) and the Cultural Revolution (1966-77) detracted from a focus on development of sporting infrastructure and grass-roots community participation.  On the other hand, PRC's exclusion from membership in the International Olympic Committee (IOC) and thus participation in the Olympic games, limited China's ability to participate in sporting events on the International stage.






\subsubsection{Reform era tiyu (1976-2000)\label{sect:reformEra}}
The death of Mao and the end of the Cultural Revolution in 1976 signalled the beginning of widespread social and economic transformations in China, in which the development of sport was heavily implicated.  American cultural Anthropologist Susan Brownell’s work, ``Training the Body for China'' (1995) was the first and most comprehensive attempt at an anthropology of sport in China. The research that forms the basis of Brownell's monograph was conducted during the mid 1980s at a time when China was only just beginning to interact politically and economically with an international community.

In reference to the unprecedented success of the Chinese women’s volleyball team in the 1980s, including winning China's first ever gold medal in a team event at the LA Olympics in 1984, Brownell (1995: 86) explains how elite level sport functioned as a crucial symbolic practice for China in the process of ``re-joining the world.''  As a participant in the sports system as a student-athlete herself, Brownell draws on first-hand ethnographic experience of training and existing as subject to the state-administered ``microtechniques of power'' (citing \cite{Foucault1977}) designed to cultivate athletes in post-Mao China.  Brownell interrogates the role of the athlete in the perpetuation of a hyper-visible and generalisable moral order, cast in official terms as a ``socialist spiritual civilisation'' (\textit{jingshen wenming} 社会精神文明) (1995: 156).  Brownell explains that the position of the athlete in reform era China was one characterised by the tensions and shifts of an ever-transforming social terrain in China,  structured by contradictory forces of top-down state control and the emerging logic of the ``free market.''
Brownell's contribution offers the first conceptualisation of the subjectivity of Chinese athletes in the reform-era in China.

One of the most immediate transformations to effect the Chinese sports system after the death of Mao was the restoration of the High School University Entrance Examination (\textit{Gaokao} 高考, hereafter Gaokao), following the end of the Cultural Revolution in 1976 \citep[198]{Brownell1995}.  School curricula were immediately redesigned around the Gaokao, and as a result, schools quickly reduced emphases on sport programs as they were seen to draw student’s attention and energy away from academic study.  Subsequently, a situation emerged where the only option for prospective athletes was to attend a specialist sports boarding school in which a scholastic education was not emphasised or was abandoned all together in favour of intense physical training.

%Position of athletes was downgraded: derogated as a decision to sacrifice your spring (青春).

China's sporting success on an international stage in the early 1980s delayed public scrutiny of this widening gap between education and sport. The PRC won a total of 32 medals at the 1984 Los Angeles Olympics---its first official appearance at the Olympics since it boycotted the games in 1952 due to a dispute with the Republic of China (now Chinese Taipei) over the use of ``China.''  Importantly, 15 of these 32 medals were gold, and this powerful display of strength on the international stage was an enormous moment for modern Chinese nationalism in the reform era \citep{Brownell2008}.  When China produced a much less impressive performance in the summer Seoul Olympics in 1988, winning only 5 gold medals (and a total of 28), latent public criticism of way in which reform era sport had become isolated from society readily surfaced and a ``crisis in Chinese sports'' was declared \citep[199]{Brownell1995}.  Amidst broader social anxieties concerning not only the alarming quantity of the Chinese population (\textit{renkou guoduo} 人口过多), but also the problem of population \textit{quality} (\textit{renkou suzhi} 人口素质), the athlete in China was problematised as lacking sufficient ``cultural quality'' (\textit{wenhua suzhi} 文化素质) in accordance with his or her elevated social status as a ``representative'' (\textit{daibiao} 代表) of the Chinese nation on an ever-expanding international stage (General Administration of Sport 2009a; Brownell 1995: 95).

In response to this public sentiment, in 1988 former army general Wu Shaozu (伍绍祖) was appointed head of national sports commission and tasked with implementing reform measures that would help the ``societisation'' of the Chinese sport system.  In 1989 the Sports Commission adopted a policy modelled on the US college sports system, of ``combining sport and education'' (\textit{tijiao jiehe} 体教结合).  In an attempt to move away from a reliance on sport boarding schools and full-time sports training centres for the development of athletic talent, ``high level tiyu programs'' (\textit{gaoji tiyu xiangmu} 高级体育项目) were embedded within existing stand alone high schools and universities so as to ensure the ``all-round development''(\textit{quanmian fazhan} 全面发展) of the athlete \citep[203]{Brownell1995}.  As part of an emphasis on a broader range of sports and their perceived potential to facilitate community engagement, international relations, as well as commercial opportunities, various sports programs, including many non-Olympic sports such as rugby, were inducted into the Chinese sports system for the first time\citep[70]{Knuttgen1990}.

%Above all, the  democratisation of sports programs to include non-Olympic sports was driven by a persistent faith---built-in to the logic of `\textit{tiyu} form its inception in China---in the ability of sport to produce citizens of a certain \textit{quality} \citep[7]{Woronov2003}.

Reform measures in sport continued into the mid 1990s before being interrupted by China's first chance at hosting the Olympics.  In an attempt to reduce the monopolisation of power and resources in the sports system, in 1993 Wu Shaozu broke up the six major sporting bodies of the Sports Commission into 23 sports management centres, with the ultimate goal of placing every sport under the management of an independent sporting association.  In 1994, the first professional Chinese Football League was established, followed soon after by the professional Chinese Basketball League in 1995.  In 1998, the Sports Commission rebranded as the General Administration of Sport (hereafter GAS) to accord with this direction of institutional reform.  But, one year earlier in 1997, it was decided (by the powers above in the CCP) that Beijing would bid for the 2008 Olympics. As such, yet another shift in focus occurred in Chinese sport that distracted the course of reform.

By this stage in its history in China, success on the world stage at the Olympics and the increasing consumption of Chinese and international sport by China's emerging middle class meant that sport had become an important part of the Chinese national fabric \citep{Brownell}. By 2001, China had hosted the Asia Games (1990), and were preparing to bid for the Olympics.  Sport thus offered a symbolic stage on which China was able to choreograph its engagement with the world and maturation as a modern nation state.

\subsubsection{Beijing Olympics and the lost decade of sport reform (2002-2012)}

Many political and social commentators within China refer to the period under the leadership of President Hu Jintao and Premier Wen Jiabao as the ``lost decade'' (\textit{shiqu de shinian} 失去的十年) in the PRC's modern development \citep{Johnson2012}.  The main criticism by commentators is that despite China's economic rise, social and political reform during this period stagnated in comparison, leading to a deepening of problems ranging from corruption to degradation of the natural environment, to structural imbalances in Chinese economic and financial systems \citep{Barme2014,Minzner2018}.  Many commentators adopt the same stance in relation to the Chinese sport industry.

Indeed, once the bid for the Olympics was announced as successful in 2001, sport reform ground to a halt as priority shifted to winning as many gold medals as possible \citep{News2017}.  Wu Shaozu left GAS in 2000, and his two successors Yuan Weimin (袁伟民, 2000-2004) Liu Peng (刘鹏, 2004-2016) did not actively return to the project of reform, and instead continued to invest in Olympic performance.  Even though many sports had since established independent associations, these associations had to be directly affiliated with one of the 23 GAS sport management centres. Rugby, for example, became affiliated with the ``Management Centre for Small Ball Sports'' (\textit{xiaoqiu guanli zhongxin} 小球管理中心), which was also home to sports such as Golf and Ten-pin Bowling.  The structure of the sports system exists today largely unchanged, although its name changed in 1998 to ``The General Administration of Sport in China'' (\textit{Guojia tiyu zongju} 国家体育总局, hereafter GAS).

The combination of high profile scandals and corruption (match fixing, doping reports, etc.) and the persistence of China's narrow performance-focussed sports system produced palpable public discontent during the lost decade of Chinese sport.  Low investment in public team sport and deterioration of indicators of public health contributed to lack of public participation.  During this period, sports management centres swelled into sovereign entities managing large accounts. These centres thus became vulnerable to corruption and graft, and served to hinder reform despite their original design as reform facilitators.  The Chinese Basketball Association and Chinese Football Association suffered year after year of losses, and sports such as table tennis and volleyball struggled to secure a sturdy commercial foundation.  Critical murmurings remained sufficiently muffled in public discourse, however, by the strong performances of Chinese Olympic athlete delegations in Sydney 2000 (28 gold, 58 total) and Athens 2004 (32 gold, 63 total). The choreography of Chinese sporting strength on the world stage reached its pinnacle when Beijing hosted the 2008 Olympics, with the Chinese athlete delegation winning 48 gold medals in a total haul of 100 medals.  It appears that the highly visible success on the world stage distracted from much needed reforms and further institutionalised a problematic incentive structure revolving around success in Olympic sports.

%\subsubsection{the impact of competitive Olympic logic on team sports}



\subsubsection{Xi Jinping Era Sport reform and Beijing 2022 Winter Olympics (2013 - present)}

Unlike his predecessors in the post-Mao era of the PRC, when President Xi Jinping came to power at the end of 2013, he wasted no time in signalling strong intentions for immediate and major political reform as part of his tenure.  The institutional reforms initiated by Xi have had had implications for sport.  In the second half of 2014, sport was earmarked to become a ``major pillar'' of domestic economic consumption.  In 2015, the GAS released a number of policies charting a course for the transformation of the sports industry from an industry dominated by manufacturing (55\%) to a services dominated industry with a total scale of USD 800 billion by 2025 (for context, the total scale of the comparatively mature US sport industry was USD 400 billion in 2016).  In 2016, CCP Standing Committee Member, Deputy Party Secretary and former Vice Major of Beijing Gou Zhongwen was appointed Director of GAS.  Gou's appointment appeared to be a surprise move, given that Gou had no previous experience in sport. Gou is however reported to be one of Xi's trusted CCP colleagues and therefore his appointment signalled a seriousness in relation to finally executing long-awaited structural reform in Chinese sport.  Signals of this reform process include the appointment of Chinese Basketball deity Yao Ming as director of the Chinese Basketball Association (the first time a non-Bureaucrat athlete had been appointed to a top sports administration position), the removal of coaching cliques (such as the removal of Liu Guoliang as head of table tennis), and the establishment of new high performance model called ``Team China'' (\textit{beizhanban} 备战办) to kick-start elite performance ahead of the Tokyo 2020 Olympics and the Beijing Winter Olympics in 2022.  The process of reform has continued to impact on sport in China and rugby in particular.  However, these changes were largely immaterial during the period I was conducting research. The fact that a concerted effort to initiative reform was taking place around the time of my research served to highlight the problems that were the focus of reform policies.

%Late 2016:  GZW blindsided sports centre managements, calling them monopolies and saying too much power rests in the hands of their chiefs.


\section{The structure, regulations, and incentives of the Chinese Sports System}
As explained above, the Chinese Sports system has evolved gradually since it was erected at the beginning of the PRC in the 1950s. The organisational structure, and the ways in which its athletes, coaches, officials, and administrators are incentivised, however, have not changed significantly since the their inception, despite several waves of policy reform.





In this section I detail the organisational structure of the Chinese sports system, and explain the incentives that propel athletes to participate in programs such as those that exist for rugby in China.

\subsubsection{Organisational structure of the Sports System}
At the bottom of the hierarchical structure of the Sports Commission are local sports commissions (county, township and city), above which are the provincial and municipal sports commissions; and at the top is the National Sports Commission, located in Beijing \citep[59]{Brownell1995}.  The Sports Commission was responsible for all sports training centres and sports programs, of which there were many types.  On one extreme, the elite professional arm of the Sports Commission---a ``national system'' (\textit{juguo tizhi} 举国体制)--- presides over all full-time national, provincial, and municipal professional sports teams (\textit{tigongdui} 体工队).  The main objective of this national system is to cultivate elite athletes to compete on a national and international level, in events such as the National Games (\textit{Quanguo yundonghui} 全国运动会), the Asian Games (\textit{Yazhou yundonghui} 亚洲运动会), and most importantly, the Olympic games (\textit{Aolinpike yundonghui} 奥林匹克运动会).  Due to an overwhelming Olympic-focus, all sports under the umbrella of professional arm of the Sports Commission are either Olympic sports, or Chinese martial arts (Guojia tiyu zongju 2009a).  Outside of this professional arm, elite sport programs are embedded within secondary and tertiary education institutions in a number of different ways, under the banner of the ``high school and university sport system'' (\textit{gaoxiaotizhi} 高校体制) (Guojia tiyu zongju 2009a).  At a high school level, elite sports programs are offered at ``extracurricular sports schools'' (\textit{yeyu tixiao} 业余体校), as well as regular high schools that focus on one or two sport programs in particular \citep[59]{Brownell1995}. At a tertiary level, a number of specialist sport colleges operate at national, provincial/municipal and local levels.

\subsubsection{Athlete Incentives}
Despite valorisation of the athlete as part of the proletarian revolution during the Mao era (see Section ~\ref{sect:sportPRC}), and some celebration of the nation's top athletes during the post-Mao reform era, a career as a professional athlete remains far from highly desirable or coveted in China, at least not by urban elites.  It is popularly understood that becoming an athlete in China entails sacrificing one's youth (\textit{qing chun} 青春); athletes train incredibly hard and are forced to endure bitterness (\textit{chiku nailao} 吃苦奶酪. It is commonly understood that for women in particular, the physiological demands of being an athlete (weights training, training outside, training during one's menstrual cycle) can threaten one's femininity and fertility, thus harming marriage prospects \citep{Bronwell1995}.  Although recent commercial growth of the sport industry in China has supported the rise and valorisation of ``super star'' athletes in Chinese sport, such as NBA athlete Yao Ming, or 110m hurdles Olympic Champion Liu Xiang, for most wealthy and educated urban Chinese families, the prospect of becoming an athlete does not compare to other life-course trajectories such as pursuing education.  The history of separation between the competitive sport system and the education system is also a factor in discouraging Chinese youth and their families from pursuing sport if it threatens other more attractive life course possibilities within mainstream education institutions.

Nonetheless, for many sectors of society, becoming an athlete---in the past, and still today---offers two main attractive life-course opportunities: education and employment.  Entry into a competitive sport institute such as the Institute in Beijing promises access to a tertiary education and employment opportunities that tertiary level education qualifies. As I explain in more detail below, with employment also comes other benefits such as residency permits and more privileged access to public goods as healthcare and education.  As such, sport offers athletes the opportunity for social mobility.  Traditionally, many athletes in China's competitive sport system are from more rural and less affluent areas.  Athletes are drawn to professional programs for the opportunity to study, find employment, and gain access to residency in urban centres and the access to education, healthcare, and other public goods that residency affords.


\myparagraph{National Athlete Technical Standards}
The Chinese competitive sport system is populated by hundreds of thousands of athletes, coaches, officials, and administrators based at hundreds of institutions throughout the country.  Professional training programs (such as the rugby program at the Institute in Beijing) exist at national, provincial, and city levels.  These programs are hosted either in standalone sport institutes, or by universities and high schools.

A series of national standards for athletes, coaches, and officials, allows for the systematic regulation of admission and performance in this system (see Appendix ~\ref{chap:researchSetting} Section ~\ref{sect:athleteStandards}).


 Standards are set out by the National governing body of each sport.  Athletes can qualify for entry to a professional sport program if they meet the nationally-defined technical standards for performance (see Figure ~\ref{}).  The lowest level of qualification for an athlete is ``Level 3'', and, as of 2010, can be achieved by recording an official level of performance in one of the following sports: Athletics, Football, Basketball, Volleyball, Table Tennis, Chinese Martial Arts, or Swimming.  Access to most ``high level sport programs'' based at universities or high schools require at least a Level 2 athlete qualification.  Admission to the Institute---a professional provincial program---necessitated a minimum Level 1 standard.  For a male athlete, a Level 1 standard could be achieved by running 11.2 seconds or faster in the 100m sprint (athletics), jumping 6m in the long jump (athletics), or achieving a top-three result in a national level tournament in football or basketball.  Once gaining entrance to the Institute, athletes were eligible to attend university if they achieved the next level of athlete qualification, that of a ``Master Sportsperson'' (\textit{yundong jianjiang} 运动健将).  This level of qualification could be achieved by representing Beijing at a national level tournament (竞体司2014).






\subsubsection{Summary: the logic of the Chinese Sports System}

Sport in China began as a deliberate social project driven by revolutionary goals of modernisation against the backdrop of what was perceived to be a corrupted dynastic rule.  With the establishment of the PRC, sport became a central ideological piece in China's participation with the world, which was heavily geared around participation in international multisport events such as the Olympics and the Asian Games. This explicit focus on sport as a nationalistic project has ultimately had implications for the ability of sport in China to move beyond a fixation on Olympic performance, as is demonstrated by the problematisation of the ``lost decade'' of Chinese sport and the continual challenge of developing grass-roots participation.  Despite aggressive reforms in recent years, the challenge of reform continues, and it has become clear that the founding logic of the sports system runs deep.  Thus, most athletes in China appear to be attracted to their sport by the life-course opportunities of education and future employment that participation in the professional competitive sport system offers.










  \section{The history of rugby union in China}



    \subsection{Rugby as an university level amateur sport (1992-2009)}
Although reportedly existing in China within colonial and expatriate circles for more than a century \citep[210]{Reason1979}, and as a modified military exercise as early as the 1930s \citep[135]{Morris2004}, rugby was a late entrant into the Chinese sport system, established as a ``high level sport program'' in 1990.  The advent of rugby in China was thus part of the ``democratisation'' and ``combining sport and education'' of Chinese sport initiated by Wu Shaozu during the late 1980s (see Section ~\ref{sect:reformEra}).   The introduction of rugby into China was initiated originally at CAU by professor Shi Zhengsheng (施振声) who was introduced to rugby by his
supervising professor while completing vocational studies at Azabu University in Japan 1987-1989 \citep{Xu2010}.  Following Shi’s return to CAU, an exchange relationship was set up between the two universities, and throughout 1990, coaches and referees from Azabu University came to CAU to help set up the necessary infrastructure required for a rugby program.  On the 12th December 1990, China’s first rugby union team was created.  The program was originally made up of existing CAU students who expressed interest in the novel activity, but by its second year, the program earned status as a High Level Sport Program and was subsequently advertised to student-athletes across the country \citep[2]{Xu2010}.  Between 1990 and 2009, rugby programs based on this original CAU model were established within over 30 regular universities and specialist sports colleges in cities throughout China.  There are also a number of social rugby clubs (\textit{shehui julebu} 社会俱乐部), organisations completely independent of the state sports system, in major cities with high expat populations (e.g., Shanghai, Beijing, Chengdu, Qingdao).

The Chinese Rugby Football Association (CRFA) was established in 1997. Both the men and women’s national teams, made up of players predominantly from CAU, but also from other well-established programs based at the Shanghai Sports University, Shenyang Sports College, and the People’s Liberation Army Sports College. China consistently competes against other nations in the Asia Pacific region (most notably in the Asian Games and the East Asian Games), and is also occasionally involved in top-tier international tournaments such as the International Hong Kong Sevens.

For the first 20 years of its existence in China, Rugby was part of a large collection of ``cold-gate'' sports (\textit{lengmen xiangmu}, a term that refers to a profession, trade or branch of learning that receives little attention) in China, which had a relatively small participation base compared to other interactive team sports like basketball or football.  While football and basketball have matured as standalone enterprises with supporting market-based consumer industries, most other sports in China (i.e., all other Olympic events, including rugby) exist primarily due to the support of the enormous state-sponsored sport system.  Whereas the commercial basketball and football industries might offer a small percentage of prospective athletes incentives of fame and fortune, the benefits of a state-sponsored sports programs like rugby are more modest.  Chinese youth either gravitate or are ushered by their parents towards sporting careers primarily due to potential life-course opportunities such as access to tertiary education and post-athletic career employment.


\subsection{Rugby in China 2010 - 2013 \label{sect:rugbyinChina}}
Olympic status transformed rugby almost overnight from its former position as an amateur sport played at university level by a handful of universities.  In 2010, rugby (in its seven-a-side version of rugby sevens) was included as one of 33 events to be held at the 2013 National Games in the city of Shenyang. This decision spurred provinces to set up professional rugby programs at provincial sports institutes, to compete at the National Games in 2013.  Ten of China's collection of 32 provinces and municipalities that participate in the National Games have full time men's and women's rugby programs.  The most important measure success in sport is derived from results in the National Games (\textit{quanguo yundonghui} 全国运动会) \citep{Hong2002}.  The amount of funding a province and its provincial sporting institutes and programs receive is decided to a large extent by results at the national games.

When rugby union was officially inducted into the state sponsored sports system in 2010, a total of five full time Men's (Beijing, Shandong, People's Liberation Army (PLA), Liaoning, and Shanghai) and six Women's (Beijing, Shandong, Anhui, Liaoning, Shanghai, and Jiangsu) provincial programs were established, signalling a intention to invest in the sport for the long term.  With professional provincial rugby programs catering for tertiary aged athletes (17 years and above) already established, these provinces could also initiate the establishment of city level rugby programs catering for high school aged athletes (10 - 16 years).  In this way, a previously non-existent development pathway for athletes, coaches, and officials began to emerge in provinces interested in investing in the sport.

% a quadrennial multi-sport event hosted on rotation by provincial capital cities
 Full time programs could draw fully on the institutional resources of their respective provinces to offer athletes a range of attractive life-course opportunities relating to education, employment, and permanent residency.  In the case of the Men's and Women's programs based at the Institute, for example, athletes were attracted by the offer of much sought after Beijing permanent residency, the opportunity to attend a well-renowned Beijing university, and the chance to remain at the Institute as an employee after their career as an athlete.  All of these opportunities were of course conditional on various hurdles of measurable performance.  Resources of provincial sport institutes also included athletes from adjacent programs such as athletics or association football.  Full time rugby programs soon began to attract transition athletes from these sports, which were often overcrowded due to their traditional popularity.

In addition to these full time provincial programs, three part-time men's (Inner Mongolia, Heilongjiang, and Xinjiang) and two part-time women's (Sichuan and Xingjiang) programs were established, in which these provinces temporarily employed rugby athletes from university programs. In addition, Hong Kong fielded both a Men's and a Women's side, bringing the total of Men's and Women's teams eligible to compete in the National Games to nine and ten, respectively.


\subsubsection{The National Games 2013 \label{sect:fallFromGrace}}
A few provinces in particular identified an opportunity to achieve a beneficial result at the National Games by heavily investing in this debutant sport.  The Beijing men's and women's programs (based at the Institute) managed to attract a large amount of China's existing rugby talent from where it was previously based at the CAU, Beijing.  Importantly, among Beijing's recruits was the unofficially touted ``Boss''  (\textit{Laoda} 老大) of Chinese rugby, Chinese national coach Zheng Hongjun.  Meanwhile, Shandong province, a powerhouse in other provincial sports, succeeded in attracting the majority of the remaining rugby talent.  The pull to Shandong was strong for a large majority of rugby players in China at the time, many of which were originally from Shandong.\footnote{(Indeed, a large proportion of athletes more generally are from Shandong, \citep[see][]{Taylor2010}.}  Importantly, the talent transferred to Shandong province also included coaching staff, namely Zheng Hongjun's student and soon to be rival, former Chinese Women's Team coach, Lu Xiaohui.  Besides Beijing and Shandong, Jiangsu and Anhui province were strong contenders for the Women's gold medal, while the People's Liberation Army (PLA) and Hong Kong in particular were strong contenders for top spot in the men's competition.

Beijing's results in the two years leading into the 2013 national games were strongest overall across the men's and women's teams.   However, the traditionally strong Hong Kong men's and women's teams had only occasionally participated in these tournaments due to conflicting international tournaments.  In the semi-finals of the National Games, held in Shenyang at the beginning of September 2013, the Beijing men came up against Hong Kong, while the Shandong men played off against the PLA.  Beijing lost to their stronger and more favoured opponents, and Shandong beat the PLA.  Meanwhile in the women's tournament, both Beijing and Shandong advanced to the final without faltering.  The stage was set: the traditional favourites, Beijing, led by the reining Boss of Chinese rugby, would face Shandong---the underdogs---lead by the Boss's cunning apprentice come challenger.

The men's final was played first, and in somewhat of an upset, Shandong edged out Hong Kong to win the gold medal by one try (one five-point touch down).  In the women's final, scores were level until early in the 2nd half when Shandong went ahead by two tries to nil.  At that point, the Beijing women's team, under instruction from their coach Zheng Hongjun, suddenly stopped playing.  After being asked by the referee and match officials to continue, the Beijing athletes stood firm and refused to play on, forming a huddle on their side of half-way in the middle of the field. Shandong had no choice but to continue to play out the rest of the 2nd half, running in try after try, until the final score at full time was a farcical 71-0 \citep{Sina2013}.  Shandong was declared victorious, while Beijing called foul play, claiming that the Spanish referee had been unfairly adjudicating the match in Shandong's favour.  The details and dramas of this now well-known story in China's sporting history (known as ``The 2013 National Games Match Strike Scandal'' (13年全运会巴塞门) ) require more detailed development in a format that exists beyond the scope of this particular dissertation.  Suffice to say, the repercussions of this incident for the Beijing provincial rugby program were extremely costly.

\subsection{Temple of the God of Agriculture Sports Institute}
The rugby match-striking-gate of 2013 led to a sudden fall from grace for the Beijing rugby programs.  Between 2010 and 2013, the Institute Leadership, excited about the prospect of unprecedented success at the national games,  immediately elevated the rugby programs to top-priority status.  Rugby received unrivalled institutional and financial support in the hope that both teams would be crowned National champions---what would have been the Institute's first National Games gold medals since 2004.  During this period, the rugby program attracted a high profile sponsorship deal from Beijing Capital Steel (北京首钢), which enabled the Institute to invest in a team of foreign coaches from New Zealand to come to Beijing on a periodic basis to consult on training and preparation. Both teams also travelled twice to New Zealand for two three-month stints of off-season training and competitions.  Between 2010-2013, the rugby team lived in the Institute's best available accommodation, and ate their meals at the Institute's highest level canteen, reserved for National-level champions.  Right up until the National Games in 2013, the men's and women's teams had met the high expectations placed on them, winning all but one of seven national tournaments each.  All indications were positive for Beijing to take home two gold medals.  However, as explained above, the National Games in Shenyang in September 2013 did not transpire as Beijing would have hoped.

In the end, Beijing came away with one bronze medal (men's team) and one face-destroying disqualification for the women.  The assistant coach of the Beijing men's team, Shi Yan, told me quietly one evening that the Beijing women's rugby team was the first Beijing team in the 48-year history of the National Games not to receive the ``medal for civilised spirit''  (awarded by the Beijing Mayor to all Beijing representatives in the National Games) (SOURCE).  All rugby coaches and many senior athletes of the 2013 National Games campaign have since left the Temple of Agriculture, either retiring or moving to other provinces.  The rugby program was all but abandoned at the end of 2013, with athletes from both teams being told to take a break for an undetermined length of time.  It wasn't until April 2014 that the men's program was resurrected with the appointment of a new head coach.  It was in this context that I entered the Institute and began ethnographic research.




\section{Conclusion}

                                                          \end{CJK}
