
\chapter{\label{app4:ethnoSetting}Appendix: Ethnographic setting and methods}

                                                \begin{CJK}{UTF8}{gbsn}


Accounting for human behavioural phenomena requires the consideration of a number of biological, cognitive, and ecological mechanisms that interact via reciprocal feedback loops spanning multiple scales of time and space \citep{Fuentes2015}.  The cognitive inputs to joint action in real world settings are rarely limited to essentialised elements administered in laboratory paradigms.  It is now known that cognitive processes relevant to joint action are distributed throughout brains, bodies, and the physical environment of the ecological niche in which it is situated.

\subsection{Social cognition of joint action among professional rugby players in China: recalibration of predictions}

In Chapter ~\ref{chap:theory}, I make a series of novel theoretical predictions concerning a relationship between joint action and social bonding. In particular, I isolate the experience of team click as a psychological construct that captures the subjective experience of optimally coordinated interpersonal movement in a team sport context.  Team click  contains the psychological elements responsible for mediating a relationship between joint action and social bonding.  This novel theory of social bonding through joint action was formulated with a broad variety of joint action settings in mind.  As such, the theory is broadly relevant all joint action scenarios in which co-actors coordinate behaviours to bring about change in the environment.

Locating evidence for this general theory in real world settings of human behaviour necessitates a consideration of the cultural and ecological affordances responsible for patterning social cognition in any given context.  In the case of this dissertation, for example, t can be expected that generalisable cognitive mechanisms and systems dynamics of joint action (hypothesised in Chapter ~\ref{chap:theory} will operate within culturally specific terrain of rugby in contemporary China.  To comprehend the affordances that shape the social cognition of joint action in rugby in China, attention must be paid to both the 1) joint action parameters of rugby, 2) the historical-cultural context of the PRC, and 3) the recent history of rugby's development in China (see Chapter ~\ref{chap:researchSetting}).  Only by considering these contextual factors is it possible to bridge the interpretive gap between real world human behaviour and formal theory.






















\section{Method\label{sect:methodChap4}}

  \subsection{Interviews}
    \subsubsection{Script for semi-structured interviews \label{sect:semiStructured}}


  Introduction:
  - Brief explanation of my research
  - When was the first time you came into contact with rugby?
  - Where are you from?
  - History with sport before rugby?

  Family:
  What does your family think about rugby?
  Do they support you playing rugby?
  Whose decision was it to start playing rugby?
  Do they worry about you getting injured?

  Perceived Costs Of Rugby:
  Opportunity Cost: what would you be doing if you weren’t here playing rugby?
  How do you feel about the fitness requirements of rugby?
  What is the feeling like when you’re out on your feet and can’t go on?
  Injury: Have you had any major injuries?
  What do you think is the hardest thing about rugby?

  Perceived Benefits of Rugby:
  Do the following motivate you to play rugby? (counter-balanced order)
  - Education
  - Your parents / Family (i.e., to make them proud/content)
  - Beijing Residency
  - Future Employment
  - Teammates
  - Earn Respect from (from society, family, friends)
  - Find a girlfriend
  - Fun
  - Represent Beijing

  What is something new that you have learnt through rugby?

  Team Membership:
  What role do you play in the team?
  What is the most important thing for you to do in your current role in the team?

  Dissonance/Personal Shortcomings?
  Have you ever felt like you have let the team down?  (neijiu 内疚: failure to uphold obligation to another)
  In what situations do you feel like that?
  How do you react to those feelings?

  Flow/team click:
  Have you ever experienced the team playing extremely well together; everything clicking, like everyone on the field has a ``tacit understanding'' of each other (\textit{moqi} 默契)?
  When was this experience?
  What did it feel like?
  How do you think that “team click” can be achieved?


  \subsubsection{Post-interview activities\label{sect:postInterview}}


  Sorting Task: rank your motivations for playing rugby
  - Education
  - Your parents / Family (i.e., to make them proud/content)
  - Beijing Residency
  - Future Employment
  - Teammates
  - Earn Respect from (from society, family, friends)
  - Find a girlfriend
  - Fun
  - Represent Beijing
  (Order randomised)

  Social Network Task:
  1.	The three most competent rugby players in the team
  2.	The three people in the team most willing to help others
  3.	Your three closest friends in the team



  \subsection{Informal Surveys}



        \subsubsection{Flow State Scale\label{sect:flowStateScale}}

  Can all athletes who participated in today's training session please respond to each of these questions in a private message. Please do not communicate with other athletes, I want to know about your own experience.  If there are any questions you can ask me directly.

  Please answer the following questions based on your experience in the just-concluded competition or activity. These questions relate to the various ideas and feelings you may have experience during the competition or activity you just completed. There is no right or wrong answer. Think about how you feel during the competition/activity and then grade each question using a scale of 1 to 7: Strongly disagree 1; Strongly agree 7).


  \begin{enumerate}
    \item Just now my attention was completely devoted to executing the activity
    \item Just now it was as if everything was happening automatically
    \item Just now I was not concerned about how I was performing
    \item Just now it was as if time changed (either slowed or accelerated)
    \item Just now I was clearly aware of what I wanted to achieve
    \item Just now my abilities were matched to the high demands of the activity
    \item Just now I was really enjoying the experience
    \item Just now I experienced a challenge, but I believed that my skills could meet this challenge.
    \item Just now I was not concerned with how others may be evaluating me
  \end{enumerate}

  Thanks for your cooperation!




  所有参加上午训练的球员请私信给我回答下面的九个问题。
  请不要和其他球员沟通,我想知道你们自己的感受。要是有问题可以直接问我。


  状态流畅量表-2 (CFSS-2)
  请根据你在刚刚结束的竞赛或活动中的体验回答下列问题。 这些问题与你在刚刚完成的竞赛或活动过程中可能体验到的各种想法和感受有关。答案无对错之分。思考一下你在竞赛/活动过程中的感受,然后采用下面的等级划分回答问题。  (等级划分1到7分:完全不同意1分; 完全同意7分)

  \begin{enumerate}
    \item 刚刚我的注意力完全正在进行的活动上
    \item 刚刚行动似乎是自然而然发生的
    \item 刚刚我不关心自己的表现如何
    \item 刚刚时间似乎改变了(要么是减慢了,要么是加快了)
    \item 刚刚我清楚的意识到自己想要做什么
    \item 刚刚我的能力与情境的高要求相匹配
    \item 刚刚我真的很享受那种体验
    \item 刚刚我遇到了挑战,但我相信自己的技能能够应付这一挑战
    \item 刚刚我不关心别人可能会如何评价自己
  \end{enumerate}

    谢谢配合!





        \subsubsection{General Information Survey \label{sect:generalSurvey}}


  Think about the last five weeks of training.  What has been your experience of the following (grade each question using a scale of 1 to 7):

\begin{enumerate}
  \item The intensity of training during this period (1 very low; 3.5 is normal; 7 extremely intense)
  \item The difficulty of training during this period (1 very low difficulty; 3.5 normal; 7 extremely high difficulty)
\end{enumerate}

Now think about your personal and team performance during this period.  How do you feel about:

\begin{enumerate}
  \item Your overall individual performance in training (1 very poor, 3.5 normal, 7 very good)
  \item The overall performance of the team (1 very poor; 3.5 normal, 7 very good)
  \item The role that you personally play in the team (1 very small role; 3.5 average role; 7 very important role)
  \item  Your personal agency in the team (1 extremely weak;  3.5 average; 7 extremely strong)
\end{enumerate}


想一想刚过去的五周训练。你对以下的问题有什么经验感受?(等级划分1到7分:完全不同意1分; 完全同意7分)
\begin{enumerate}
  \item 这个阶段训练的强度 (1分 非常小,3.5分 正常,7分 非常大)
  \item 这个阶段训练的难度 (1分 特别低,3.5分 正常,7分 特别高)
\end{enumerate}

想一想你对个人和整个团队的表现:

\begin{enumerate}
  \item 这个阶段你个人在训练当中的表现 (1分 特别差; 3.5分 正常;7分 特别好)
  \item 这个阶段整体在训练当中的表现 (1分 特别差; 3.5分 正常;7分 特别好)
  \item 这个阶段你个人在团队发挥的作用 (1分 作用很小; 3.5分 正常作用; 7分 作用很大)
  \item 这个阶段你在团队的个人力量 (1分 比较小,3.5分 中等; 7分 比较大)
\end{enumerate}





\subsection{Informal Surveys}

      \subsubsection{Flow State Scale\label{sect:flowStateScale}}

Can all athletes who participated in today's training session please respond to each of these questions in a private message. Please do not communicate with other athletes, I want to know about your own experience.  If there are any questions you can ask me directly.

Please answer the following questions based on your experience in the just-concluded competition or activity. These questions relate to the various ideas and feelings you may have experience during the competition or activity you just completed. There is no right or wrong answer. Think about how you feel during the competition/activity and then grade each question using a scale of 1 to 7: Strongly disagree 1; Strongly agree 7).


\begin{enumerate}
  \item Just now my attention was completely devoted to executing the activity
  \item Just now it was as if everything was happening automatically
  \item Just now I was not concerned about how I was performing
  \item Just now it was as if time changed (either slowed or accelerated)
  \item Just now I was clearly aware of what I wanted to achieve
  \item Just now my abilities were matched to the high demands of the activity
  \item Just now I was really enjoying the experience
  \item Just now I experienced a challenge, but I believed that my skills could meet this challenge.
  \item Just now I was not concerned with how others may be evaluating me
\end{enumerate}

Thanks for your cooperation!




所有参加上午训练的球员请私信给我回答下面的九个问题。
请不要和其他球员沟通,我想知道你们自己的感受。要是有问题可以直接问我。


状态流畅量表-2 (CFSS-2)
请根据你在刚刚结束的竞赛或活动中的体验回答下列问题。 这些问题与你在刚刚完成的竞赛或活动过程中可能体验到的各种想法和感受有关。答案无对错之分。思考一下你在竞赛/活动过程中的感受,然后采用下面的等级划分回答问题。  (等级划分1到7分:完全不同意1分; 完全同意7分)

\begin{enumerate}
  \item 刚刚我的注意力完全正在进行的活动上
  \item 刚刚行动似乎是自然而然发生的
  \item 刚刚我不关心自己的表现如何
  \item 刚刚时间似乎改变了(要么是减慢了,要么是加快了)
  \item 刚刚我清楚的意识到自己想要做什么
  \item 刚刚我的能力与情境的高要求相匹配
  \item 刚刚我真的很享受那种体验
  \item 刚刚我遇到了挑战,但我相信自己的技能能够应付这一挑战
  \item 刚刚我不关心别人可能会如何评价自己
\end{enumerate}

  谢谢配合!





      \subsubsection{General Information Survey \label{sect:generalSurvey}}


Think about the last five weeks of training.  What has been your experience of the following (grade each question using a scale of 1 to 7):

\begin{enumerate}
\item The intensity of training during this period (1 very low; 3.5 is normal; 7 extremely intense)
\item The difficulty of training during this period (1 very low difficulty; 3.5 normal; 7 extremely high difficulty)
\end{enumerate}

Now think about your personal and team performance during this period.  How do you feel about:

\begin{enumerate}
\item Your overall individual performance in training (1 very poor, 3.5 normal, 7 very good)
\item The overall performance of the team (1 very poor; 3.5 normal, 7 very good)
\item The role that you personally play in the team (1 very small role; 3.5 average role; 7 very important role)
\item  Your personal agency in the team (1 extremely weak;  3.5 average; 7 extremely strong)
\end{enumerate}


想一想刚过去的五周训练。你对以下的问题有什么经验感受?(等级划分1到7分:完全不同意1分; 完全同意7分)
\begin{enumerate}
\item 这个阶段训练的强度 (1分 非常小,3.5分 正常,7分 非常大)
\item 这个阶段训练的难度 (1分 特别低,3.5分 正常,7分 特别高)
\end{enumerate}

想一想你对个人和整个团队的表现:

\begin{enumerate}
\item 这个阶段你个人在训练当中的表现 (1分 特别差; 3.5分 正常;7分 特别好)
\item 这个阶段整体在训练当中的表现 (1分 特别差; 3.5分 正常;7分 特别好)
\item 这个阶段你个人在团队发挥的作用 (1分 作用很小; 3.5分 正常作用; 7分 作用很大)
\item 这个阶段你在团队的个人力量 (1分 比较小,3.5分 中等; 7分 比较大)
\end{enumerate}








                                                        \end{CJK}
