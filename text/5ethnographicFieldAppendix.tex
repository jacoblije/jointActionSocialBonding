



\chapter{\label{app5:ethnoField}Appendix: The culturally specific terrain of rugby in China}



















\section{Introduction}

In the ethnographic section of this dissertation, I make two sets of interrelated predictions.  The primary set of predictions derive from a novel theory of social bonding through joint action outlined in Chapter ~\ref{chap:theory}.  In addition to these primary predictions, I also predict that the specific group exercise setting of rugby in China will exhibit contextual specificity that will shape the contours of cognitive mechanisms and system dynamics of joint action and social bonding.  In this chapter, I report evidence for the second set of predictions of predictions---the culturally specific terrain of rugby in China---in order to lay the foundation for ethnographic evidence for the first set of predictions.

Prevailing theory from the social cognition of joint action suggests that culturally specific terrain can function as informational affordances---``hyper-priors'' or ``coordination smoothers''---that functions to enable and constrain observable patterns of action, and perception, processes of group membership, and adherence to institutional norms \citep{Clark2015}.  Describing the terrain of the Beijing men's rugby team is thus an important first step to identifying ethnographic evidence for the hypothesised relationship between joint action, team click, and social bonding.

The culturally specific terrain relevant to rugby in China has formed through an ongoing interaction between the specific history of rugby and modern sport in China and facets of an indigenous Chinese psychology.  China is home to a dynamic indigenous psychology (see Chapter ~\ref{chap:researchSetting} Section ~\ref{sect:indigPsych}), which is the product of a number of distinct but interwoven historical trajectories.   Millennia of dynastic rule involving institutionalised norms of social interaction (commonly generalised as ``Confucianism'' or ``hierarchical relationism''), combined with draconian legal and bureaucratic mechanisms of governmentality.  At the same time, China represents to itself a narrative of rejuvenation as a modern Marxist/Leninist socialist nation-state, in the shadow of 150 years of colonial domination and humiliation at the hands of foreign (non-Han) actors (including Manchurian rulers of the Qing dynasty).  The project of modern competitive sport serves as a fascinating arena of social behaviour in which in which the dynamic interaction between, and interweaving of, these two cultural-historical trajectories is clearly choreographed.

Conventional theories from within Western social psychology predict that social identity can be generated through a subjective process that is equivalent to calculating psychological distance between social categories of self and group \citep{Tajfel1971}.  As an extension of group identification theory \citep{Turner1987}, ``identity fusion '' represents extreme case that an individual perceives a 1:1 overlap between categories of self and group \citep{Swann2009}.  Related to these conceptions of social identification are theories of social motivation.  Motivation for pro-social pro-social behaviour, for example, is often conceived of as being inversely proportional to an individual's perceived distance between categories of self and group. 1:1 fusion between self and group entails maximal motivation to perform (potentially costly) pro-social activity on behalf of the group or individuals to which one identifies as fused \citep{Swann2015}.

In the case of China, however, social identity appears to be not solely driven by attention to perceived discrepancy between social categories of self and group, but more so by concerns for regulating and harmonising one's position within a network of hierarchically organised social relationships \citep{Liu2009}.  While categorical modes of social identification, relating for example to the category of the ``team,'' and the institute are particularly salient in the context of the imported team sport of rugby, these categories do not appear to capture attention and perception in the same way as relational concerns.  In this Chapter, I argue that the distinctiveness of cultural terrain of professional rugby in China can be observed at multiple conceptual levels, spanning the social institutional level,  the level of group norms, and processes of (joint) action and perception.  I offer observational evidence at each of these levels, before proceeding in the following chapter to an analysis of generalisable cognitive mechanisms of joint action and social bonding that are identifiable within this culturally specific terrain.
