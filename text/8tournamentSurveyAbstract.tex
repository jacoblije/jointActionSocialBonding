
\chapter*{Chapter 8: Abstract}
%\addcontentsline{toc}{chapter}{\nameref{ch:study1intro}}


%\markboth{{Introduction to Part I}}{Introduction to Part I\label{ch:part1intro}

This study analysed the hypothesised relationship between joint action, team click, and social bonding in a population of professional Chinese rugby players in a naturalistic real-world setting.  Specifically, the study tested the prediction that click mediates a relationship between perceptions of team performance and social bonding.  The study took place in the context of a two-day National Rugby Sevens Tournament in Qianan, Hebei Province, China (hereafter the Tournament).  Self-report survey measures and data gleaned from official performance records were collected at time points before, during, and after the Tournament.  Athletes responded to survey questions on perceptions of individual and team performance, perceptions of team click, perceptions of social bonding, as well as items measuring technical competence, personality type, perceptions of exertion and fatigue, and injury status. These items were developed based on existing theory and analysis of ethnographic observations of the Beijing men's rugby team.

%(among other items)
%When statistically controlling for perceptions of individual performance success, technical competence of athletes, and objective measures of individual and team performance in the Tournament,

Results revealed significant associations between 1) more positive perceptions of team performance and perceptions of team click, 2) perceptions of team click and perceptions of social bonding, and a direct relationship between 3) more positive perceptions of team performance feeling of social bonding.  A mediation analysis revealed that the relationship between perceptions of team performance and social bonding was fully mediated by perceptions of team click,  suggesting that athletes felt more bonded to their teammates when they felt the click of successful joint action. Results also revealed a positive relationship between perceptions of team performance (relative to prior expectations) and team click (but not social bonding), which suggested that more positive violations of expectations concerning joint action could be an important mechanism in the hypothesised relationship between joint action, team click, and social bonding.  Further controlled experimental studies are required to better investigate proximate mechanisms implicated in social bonding in joint action.


%that  However, results only partially supported the relationship between expectation violation and team click, but not the direct relationship between expectation violation and social bonding.

%(specifically a factor containing athlete responses to a range of technical components of team performance)
%($male = 93, M = 21.67, SD = 3.67, range = 17-32$)
%Data were analysed according to predictions derived from existing theory of the social cognition of joint action and social bonding, and ethnographic analysis presented in Part A.
