\chapter{\label{chap:ethnoSetting} Ethnographic Setting}

  \minitoc



\section{Abstract}
In this chapter I update predictions in light of the contextual details provided in Chapter 3.  I then outline the ethnographic method through which I collected data to test these predictions.

                                        \begin{CJK}{UTF8}{gbsn}

\section{My first assignment at the National Tournament in Qingdao \label{sect:qingdaoVignette}}

On that Monday day that I first arrived at the Institute and rounded the Stadium to its main administrative building, almost two years had passed since the Beijing Women's rugby team's controverShiat the 2013 National Games.  Fortunately, my first visit to the Institute ended up going as smoothly as I could have hoped.

I first met with Jenny, the Vice-Principal of the Institute who was responsible for rugby.  I sat down with Jenny in her office as she finished a conversation to someone on the phone. Jenny flowed with the personality and affectations that only a true Beijing local could embody---her word endings textured with coarse yet elegant ``erhuayin''(儿化音) suffixes, and she addressed the person to which she was speaking using the respectful version of the 2nd person pronoun ``nin3.''  Once she had finished with the call, Jenny welcomed me and tactfully explained to me that rugby at the Institute had indeed experienced a dramatic fall from grace.  Jenny indicated that the head coach Zhu and his assistant Shi really had their work cut out for them, and that my presence as observer and occasional coach would benefit the team.  She agreed to organise a room in the rugby program dormitory, as well as access to the Institute's canteen, in exchange for my expertise.

Buoyed by this meeting, I made my way to where the rugby program's dormitory to meet head coach Zhu Peihou (see map X).  My connection to Zhu went back to CAU in 2008, where he had been a coach at the time. From Shandong originally, Zhu was a graduate of the Shanghai Sports University, another prominent rugby program at the time.  Zhu had been recruited from Shanghai to CAU by ZHJ to coach so that he could continue to play for the Chinese national team after he had completed his undergraduate studies.  After we had discussed my research and he had provisionally approved my plan to spend the next period with the team, I asked him about the current situation with the Beijing team.  Zhu explained that he was quite frustrated that the group of athletes he was coaching lacked experience and maturity. I asked him exactly what areas of the team's performance, and he indicated that all areas were not great, suggesting that not enough players had found that ``feel'' for gameplay and very few were motivated to train hard.  We agreed that my first assignment should be to accompany the team to the final National Tournament in the coastal city of Qingdao in Shandong province, in a week's time.

As it turned out, the National Tournament in Qingdao was organised to take place immediately following a two-day Asia Sevens Tournament.  The Asia Sevens is an annual series of three regional rugby sevens tournaments featuring men's and women's national sevens teams.  The men's series has been held regularly since 2009, and the women's series was established in 2013. The series usually consists of two three annual Tournaments, alternating between various locations including Hong Kong, South Korea, Sri Lanka, China, Japan, Malaysia, India, Singapore, and Thailand.  In 2015, The Asia Sevens Series Tournaments were held in Colombo, Bangkok, and Qingdao. The Tournament would be played on the Saturday and Sunday, and then the National Tournament would follow immediately after on the Monday and Tuesday.

\subsection{Asia Sevens Tournament}
When I arrived at the stadium in time for the beginning of the Asia Sevens Series, I met a number of coaches and athletes in the stands who I knew from my time in Beijing in 2008 and then coaching in 2013. I sensed from my various interactions that there was an air of nervousness around the Tournament, particularly on behalf of the Chinese women's team.   The Qingdao tournament was the final tournament before the Olympic Qualification Tournaments, to be held in Hong Kong and Tokyo in November 2015.  The top ranked team from those two legs would qualify for the Rio Olympics in 2016.  The Chinese men's team were not in serious contention for Olympic qualification, given the clear superiority of the more established men's rugby programs like Japan and Hong Kong. The Chinese Sports Commission did, however, expect the Chinese women's team to qualify for the Olympics.

Since the Chinese women's team was first established in 2002, China had made great strides in women's rugby, in both Asia and globally.  Not including occasional losses to closest rivals Kazakhstan, between 2002 and 2012 the Chinese women's sevens team was the dominant women's team in Asia, easily outcompeting Japan and Hong Kong, and at times was competitive against the world's best including New Zealand and Australia.  The main reason for China's dominance in the women's game during this period was that other traditional rugby nations, despite having developed professional infrastructure for the men's game, lacked almost entirely an equivalent infrastructure for the women's game. China's state sponsored sport system, on the other hand, was relatively agnostic towards gender in sport. When it comes to the bare incentive structure of the Chinese sports system a gold medal is a gold medal, regardless of the gender of the recipient. (This is of course not to say that there are not distinct gender inequalities in relation to sport in China).  Indeed, beginning with the Chinese Women's Volleyball Team's gold medal victory at the LA Olympics in 1982, China has enjoyed a comparative advantage in women's sport due to the fact that the Chinese sport system was comparatively more supportive of women's sport.

Following his success at the National Games in 2013, Shandong head coach Lu Xiaohui was given the responsibility as coach of the Chinese women's program in 2014.  Essentially, this involved giving Lu and  Shandong province the responsibility for the team.  The national team trained at Shandong's provincial training centre, and most of the athletes who represented China in 2014 were from Shandong.

Alarmingly for the Chinese women's rugby team's hopes of Olympic qualification, by 2014 it had become obvious that other more traditional rugby playing nations in Asia, namely Japan and Hong Kong, had begun to make up serious ground on China and Kazakhstan in the women's game.  This naturally prompted nervousness among the Administration therefore and CRFA.  In mid 2015, it was decided by the Administration that CRFA would enlist the services of a foreign coach to bolster their campaign for Olympic qualification.  According to sources close to CRFA, apparently the original plan was for the appointed foreign coach, Ben, to act as a consultant for Lu and his existing group of coaches.  By the time I arrived in Qingdao in early September, however, the initial arrangement had since transformed into a situation in which Ben was given 100\% control over the program as head coach, and LXH was more or less sidelined as coach. Not only had Ben took over, but he had also set about reorganising the starting team and also scouting for new talent outside the squad, which was predominantly made up of Shandong athletes.

%The complication was that before being dethroned, LXH preferred to use his own athletes.  Most of the starting team at the LDN7s in June 2015 were indeed Shandong athletes, many of the women who had won gold at the National Games in 2013.  When BG took over the reigns as head coach, as directed by the GAS and CRFA,

When I arrived in Qingdao it appeared that tensions between the old and new guard were at their peak.  Lu and many of his favoured athletes had been relegated to the sidelines, and some had been completely removed from the squad altogether.  There were still six weeks to go before the all important first Olympic qualification tournament in Hong Kong.  I sat and watched the first few games of the women's Tournament, and I was indeed surprised to see that the Chinese women's side was missing some of its usual stalwarts, and indeed appeared in my eyes to lack the flow and familiarity that I had come to expect in Lu's clan of Shandong athletes.  In the stands I came across a group of Shandong women, one of which, Qi Gaige, I knew quite well from when she toured to Australia with the Shandong team when I was still with the Australian rugby sevens team in early 2013.  I asked Gaige why she wasn't playing for China, and it turned out that she was injured, and so wasn't eligible for selection.  But a few of the other women around her, who were all wearing Chinese national team uniforms, were all part of LXH's clan who had been effectively stood down by Ben. ``How do you think they're playing?'' (你觉得她们打得怎么样?) She asked me, after we had exchanged pleasantries. ``Not great'' I commented, hesitating, not knowing how much I should prime her, but also feeling obliged to be honest: ``it feels like they’re not coordinating together very well at the moment.  What do you think?'' (一般吧。感觉她们的配合不太好,目前。你觉得呢?) I asked.  ``They’re out there playing as individuals, not playing as a team! They can't get it together; there's no shared goal.'' `(她们都在打个人的,不打团体提的。打不到一块儿去啊,没有共同的目标.)  ``Hmm. Yes it does look like that.'' (嗯嗯,看起来像是) ``Hey, Lijie...'' she asked me quietly, ``...don’t you think they’re not even playing as well as our Shandong team could play?'' (嘿,李杰,是不是她们现在打的没有我们山东队打的好,是吧?)

Later that day I bumped into an assistant coach of the Chinese women's side, who had been working under Lu and now Ben since 2014.  Li Sheng was a big and booming Qingdao local, a CAU graduate, and a member of the Beijing Men's team from 2010-2013.  I asked Li about the current situation with the Chinese women's side and their prospects 6 weeks out from the first Olympic qualifier.  ``Chinese athletes must see the (individual) benefits if they are going to go all out'' (中国的运动员必要看到个人的利益才会全力以赴的) he insisted, and he went on to explain why for these athletes, there were no obvious benefits available sufficient to motivate them.  There were indeed very few material benefits associated with representing China in rugby at the national level.  Athletes were payed a nominal USD100 per month on top of their provincial contracts when training and touring with the national program.  If athletes were injured while playing for China, CRFA at the time did not have access to sufficient health insurance to cover the costs of treatment, and athletes had no choice but to return to their provincial programs and seek treatment at the expense of the province.  The less tangible benefits of playing for China, for example, access to high quality coaching, or the pride of representing the country in a sport, or even the promise of a trip to the Olympics, were heavily outweighed by other less tangible costs: long stints of time away from family, the constant risk of falling out of favour with provincial programs.  In effect, the lack of incentives at the national level meant that athletes were by definition more committed to their provincial systems---the programs that provided athletes with the benefits that they were most interested in obtaining, such as tertiary education, future employment, and a modest---but compared to CRFA--a reasonable salary (most national level players were paid 3-8k RMB/month). ``Its no wonder these athletes aren't performing well,'' (难怪这个队伍的表现不好),Li exclaimed.

\subsection{National Tournament}
On the first day of the National Tournament I met Beijing head coach Zhu in the stands before their first game, and he instructed me to keep an eye on the games that Beijing would play, and offer any feedback about things they could work on.  Of course, eager to being my observation of the athletes I would go on to spend many months together with, I obliged, and sat down with a notebook and watched the days play.  In their first game, Beijing obviously lacked coherence in attack and defence.  Their basic skills, for example, passing accuracy and work in contact needed more work, and it did not seem as though they had anyone who was performing the role of leader and providing the team with direction on the field.  I felt that they lacked an element of maturity and patience that would be required to win tight games of rugby sevens. It was clear that the team was made up of many inexperienced athletes, many of whom had barely played in serious official Tournaments such as this one.

Regardless, Beijing ended up winning both its games on Day 1. I couldn't help but be puzzled by the atmosphere when the team huddled together around head coach Zhu after their victory over the PLA in their final game of Day 1.  While the younger athletes were concentrated intently on Zhu's every word, it was apparent that some of the older players were less focussed, in fact many of them appeared to not even be listening to Zhu.  In particular, I noticed that Wang Chongyi, a former Beijing men's team representative and assistant men's coach at the time, was as physically withdrawn from the huddle as possible while still maintaining connection to it by holding loosely on to the shirts of the athletes either side of him. Wang was seemingly uninterested in what Zhu was saying.  Some of the senior athletes also appeared less engaged with what Zhu was saying.  I was puzzled by this piece of team theatre and wondered if I was simply reading too much into it, and that perhaps the posturing and performance of team unity was less emphasised in this context than the contexts in which I had played rugby elsewhere. As was common on my journeys through the world of rugby in China, I often did not quite grasp all the details and pieces of the puzzle that contextualised the interactions I was having until after the fact.

Despite the obvious deficiencies in Beijing's performance that weekend, they did well enough over the two days to make the final of the Tournament and indeed win the National Tournament.  This achievement probably owed more to the relative weakness of the other provincial outfits, more than Beijing's out and out strength.  In an unfortunate twist of fate, however, the Beijing side was later disqualified from the Tournament, due to the fact that they fielded an underage player, who was only 15 at the time.  It was clear that Zhu and his team could not catch a break.  After reconvening back in Beijing, Zhu gave athletes the rest of September off training, and instructed all to reconvene at the Institute at the start of October to begin off-season training.



\section{Introduction to the ethnographic setting}
In the present study, I address these knowledge gaps in evolutionary approaches to group exercise through an ethnographic study of the social cognition of joint action among professional Chinese rugby players.  Based on emerging research from the social cognition of joint action, in which it is increasingly understood that dynamical coupling of lower-cognitive mechanisms associated with movement regulation set the foundations for higher-cognitive processes such as social bonding and declarations of group membership, I predict a relationship between athletes' experience and perceptions of joint action and social bonding.  Within this relationship I isolate the psychological construct of ``team click,'' which refers to athletes' tacit sense of quality of coordination in joint action.

Accounting for human behavioural phenomena requires the consideration of a number of biological, cognitive, and ecological mechanisms that interact via reciprocal feedback loops spanning multiple scales of time and space \citep{Fuentes2015}.  Theory and empirical research from the emerging field of the social cognition of joint action suggest that the affordances of particular cultural environments can act to enable and constrain observable behaviour in patterned, and therefore predictable ways.  The cognitive inputs to joint action in real world settings are rarely limited to essentialised elements administered in laboratory paradigms.  It is now known that cognitive processes relevant to joint action are distributed throughout brains, bodies, and the physical environment of the ecological niche in which it is situated.  Ethnographic research offers one method through which the complexity of informational affordances for joint action, and the sensitivity of cognitive mechanisms of joint action to these affordances, can be examined empirically.

The proliferation of anthropological approaches to human behaviour in the last 50 years, while at times threatening the overall coherence of the discipline as a whole \citep{Beller2012}, has also produced diverse theoretical and methodological options for documenting human variation \citep{Fuentes2016a}.  Anthropology is thus well placed to expand upon accounts of group exercise, via methods ranging from ethnographic exploration capable of uncovering novel dimensions of behaviour and generating testable hypotheses, to quantitative techniques---e.g., experimental and mathematical simulation paradigms---capable of testing hypotheses \citep{Epstein2006,Fuentes2016}.


\subsection{Social cognition of joint action among professional rugby players in China: recalibration of predictions}

In Chapter ~\ref{chap:theory}, I make a series of novel theoretical predictions concerning a relationship between joint action and social bonding. In particular, I isolate the experience of team click as a psychological construct that captures the subjective experience of optimally coordinated interpersonal movement in a team sport context.  Team click appears to contain the psychological elements that could be responsible for mediating a relationship between joint action and social bonding.  This novel theory of social bonding through joint action was formulated with a variety of joint action settings in mind. As such, the theory is broadly relevant all joint action scenarios in which co-actors coordinate behaviours to bring about change in the environment.

Locating evidence for this general theory in real world settings of human behaviour necessitates a consideration of the cultural and ecological affordances responsible for patterning social cognition in any given context.  In the case of this dissertation, for example, t can be expected that generalisable cognitive mechanisms and systems dynamics of joint action (hypothesised in Chapter ~\ref{chap:theory} will operate within culturally specific terrain of rugby in contemporary China.  To comprehend the affordances that shape the social cognition of joint action in rugby in China, attention must be paid to both the 1) joint action parameters of rugby, 2) the historical-cultural context of the PRC, and 3) the recent history of rugby's development in China (see Chapter ~\ref{chap:researchSetting}).  Only by considering these contextual factors is it possible to bridge the interpretive gap between real world human behaviour and formal theory.

As such, in this ethnographic study of the Beijing men's rugby team, I make two sets of predictions concerning observable behaviour.  First, I make predictions about the specific cultural terrain of the research Context.  Considering existing evidence from anthropology, cultural psychology, and indigenous Chinese psychology, I expect the culturally ancient and socially dominant mode of social cognition known as ``hierarchical relationalism'' to interact with inputs from the modern history of sport in China and the specific history of the team sport of rugby in China, to produce distinct patterns of behaviour. In particular, I predict to find evidence that the social attention of athletes, coaches, and officials of the Beijing rugby program is captured predominantly by the activity of fostering and harmonising relationships.  In this environment, I expect that categories such as the self and the team function less as resources of social identity in and of themselves (as may be predicted by classical social identification theory applied to typically ``Western'' social contexts).  Instead, I expect that these categories will be flexibly utilised in service of fostering a hierarchical network of relationships.  In keeping with existing research, I expect to observe evidence of this interaction at the level of 1) institutional norms, namely the Institute and the rugby program, 2) processes of group membership, namely in expression and endorsement of group identity, and 3) action and perception when participating in the joint action requirements of rugby.  I outline evidence for this first set of predictions in Chapter ~\ref{chap:ethnoField}.

With this specific cultural terrain in mind, I subsequently expect to observe evidence of a positive relationship between joint action, team click, and social bonding.  I predict that perceptions of team and individual performance in joint action will be a core organiser of social attention.  In particular, athletes will pay close attention to
the quality of on-field performance vis-a-vis expectations for performance internalised within individuals and disseminated by figures of authority such as the coach and senior athletes.  I expect athletes to be more or less acquainted with the experience of optimal coordination in joint action, a.k.a. team click, and I predict that team click will be tightly tethered to inferences about the social group.  Finally, I predict to find evidence that athletes derive powerful social resources from processes of group membership in rugby, ranging from emotional support, perception of solidarity with others owing to a shared goal or direction, and social identity from perceived group membership.

%3) possibly mediated by expectation violation,
%4) a direct relationship between perceptions of joint action and social bonding.

This ethnographic study is reported over three chapters.  In the present chapter, I outline the method through which I collected ethnographic data, as well an overall description of the data I collected during my time with the Beijing men's rugby team.  In the following two chapters I analyse ethnographic evidence for the two sets of study predictions.  In Chapter ~\ref{chap:ethnoField}, I identify evidence for informational affordances for joint action specific to the sport of rugby and the cultural milieu of competitive sport in contemporary China.  In Chapter ~\ref{chap:ethnoResults}, I report results that pertain specifically to the core predictions of this dissertation pertaining to the relationship between joint action and social bonding.  The present chapter is designed to set the scene for an in depth analysis of evidence for the study's core predictions in Chapters ~\ref{chap:ethnoField} and ~\ref{chap:ethnoResults}.


\section{Method}


\subsection{Research Setting and Participants}

I conducted ethnographic research with athletes and coaches of the Beijing Provincial Rugby Sevens Program (the Program) based at the Beijing Temple of God of Agriculture Sport Institute (the Institute).  The Program consisted of a men's and women's rugby sevens teams, each with approximately 20-30 athletes and 2-4 coaches per team.  Athletes trained full time and lived in dormitory accommodation at the Institute.  One of four vice-principals of the Institute was responsible for the administration of the program.  Permission to conduct research at the Institute was sought from relevant authorities and directly from athletes at the beginning of the first research period in September 2015. The University of Oxford’s Central University Research Ethics Committee approved this study (SAME/CUREC1A/15-059).

% from the vice-principal responsible for rugby at the Institute and the head coach of the rugby Program prior to arriving in Beijing in 2015.  Permission to conduct research was sought directly from athletes at the

\subsection{Materials}

  \subsubsection{Participant Observation}
   I conducted a number of stretches of participant observation with the Program between September 2015 and September 2017.  During these stretches, I lived full-time at the Institute and attended training sessions, team meetings, meals, and participated in any other activities relevant to the rugby program.  I recorded field notes using Evernote (Version 7.4.1), an electronic note taking software that is synchronised across my mobile and personal computer devices.

  \subsubsection{Interviews}
  I conducted and recorded a combination of ad-hoc exploratory interviews (unstructured) with a range of research participants with knowledge of rugby in China, as well as scheduled and directed interviews (semi-structured) with athletes.  The script for semi-structured interviews was produced based on a combination of existing theory and initial ethnographic observations.

  \subsubsection{Informal Surveys}
  In addition to conducting participant observation and interviews, I also issued a series of informal surveys to measure athletes' motivations for, and perceptions of joint action and group membership in the rugby Program.  The surveys were chosen and designed based on a combination of existing theory and initial ethnographic observations.


\subsection{Procedure}

In May 2015, 4 months prior to beginning ethnographic research, I contacted the vice-principal responsible for the rugby Program and the head coach of the rugby Program to ask for permission to conduct research at the Institute.  Following affirmative responses from both, I made plans to conduct two periods of in-depth ethnographic research: 1) 6 months between September 2015 and March 2016, 2) 6 weeks during July-August 2016.

Soon after arriving in Beijing at the end of August in 2015, I discovered that the Beijing Rugby Program was at the time limited to its men's Program.  At the time, the women's rugby Program was yet to be resurrected after the humiliating ``match strike incident'' of the 2013 National Games (see Chapter ~\ref{chap:researchSetting} for a detailed explanation).  The Women's program would later be resurrected at the start of 2016, in time to participate in qualification tournaments for the 2017 National Games.  Due to this limitation, I decided to focus my attention on the men's program, which at any one time consisted of 25-30 athletes and three or four coaches.

I was provided with research access to the Beijing men's team, a room in the Institute's dormitory, and access the Institute's 1st level canteen.\footnote{The Institute had two canteens in which athletes and coaches ate all of their meals.  Athletes and coaches who had represented Beijing at national level competitions were entitled to eat at the 1st Level Canteen (\textit{yixian shitang} 一线食堂), whereas all other athletes at the Institute, or athletes who were on temporary trial at the Institute, ate at the 2nd Level Canteen (\textit{erxian shitang} 二线食堂).}  In exchange for these provisions, I agreed to assist the Program by consulting the coaching staff and doing some coaching myself.

All my interactions with research participants took place in Modern Standard Chinese (Mandarin or \textit{putonghua} 普通话).  To record these interactions, I would either interrupt conversation to ask permission to start the audio recording device within the Evernote application on my mobile phone.  Alternatively, I would transcribe conversations using my notebook or phone immediately following interactions.  Every week or fortnight I collated, summarised, and organised these notes by date and by theme.


  \subsubsection{Interviews}
I conducted unstructured interviews with athletes and coaches on an ad-hoc basis, often when an informal discussion developed into a conversation relevant to my research questions.  In such instances, I would interrupt discussion with the research participant and ask permission to record the remainder of the discussion.

Semi-structured interviews were conducted by appointment in my dormitory room at the Institute at a period 2 months into my first stint of participant observation.  During semi-structured interviews, I asked athletes about questions about personal background (including where they were from and their family situation), motivations for adherence to rugby, perceived costs and benefits of adherence to rugby, perceptions of joint action, and group membership (for a detailed script of semi-structured interviews, see Appendix ~\ref{app4:ethnoSetting} Section ~\ref{sect:semiStructured}).  Questions served only as a loose structure for conversation, and at times either the athlete or I departed from these questions to talk about other dimensions of experience associated with rugby at the Institute.  The order in which athletes participated in semi-structured interviews was randomised.

I conducted all interviews in Modern Standard Chinese (Mandarin) and interviews were recorded with participant consent.  Once all interviews were recorded, interviews were transcribed into written Chinese by a native Chinese speaking research assistant using a ``verbatim'' method \citep[i.e., including an account of all verbal and important nonverbal (coughs, pauses, etc.) utterances, see][269-70]{Poland2003}.  I checked each transcript for accuracy by comparing the script against the original audio recording during the first phase of open coding analysis (see Section ~\ref{sect:dataAnalysis} below for full explanation of procedure for data anlaysis).  I analysed interviews in Chinese and only translated into English data extracts that were included in the main analysis of this dissertation.

%    \subsubsection{Structured}

\subsubsection{Surveys}

 I conducted a number of informal surveys designed to measure athletes' experience of joint action and group membership in training sessions.

   \myparagraph{Post-interview surveys}
   Following semi-structured interviews, I asked each athlete to rank 10 different possible motivations for adherence to rugby from most important to least important. Possible motivations for rugby consisted of: \textit{to gain access to education}, \textit{to represent Beijing}, \textit{to do Family proud}, \textit{to gain respect from others}, \textit{for (the benefit of) teammates}, \textit{for employment opportunities}, \textit{for money}, \textit{for enjoyment}, \textit{to find a partner}. In addition, athletes were asked to report their 1) three closest friends in the team, 2) the three team members most willing to sacrifice on behalf of the team, and 3) three most competent athletes in the team (see Appendix ~\ref{sect:postInterview} for a full script). Athletes answered these questions using a pen and paper. I later collated and uploaded these responses to Evernote.

   \myparagraph{Chinese Flow State Scale-2}
    I conducted informal surveys following three training sessions: 1) a session in which (predominantly junior) athletes ran an aerobic fitness test involving continuous running in a straight line ``shuttle runs''  at and above the aerobic threshold for approximately 25 minutes (known as the ``Beep Test''), and 2) two 60-minute training sessions spread one week apart involving training scenarios that emulated high-intensity interaction and exertion of match conditions.  After each of these sessions, I administered to each participating athlete via WeChat nine items selected from a validated Chinese version \citep{Liu2012} of the Flow State Scale-2 \citep{Jackson2002}.  The items were selected to measure each of the nine conceptual dimensions of the flow experience: challenge-skills balance, action-awareness merging, clear goals, unambiguous feedback, total concentration on the task at hand, sense of control, loss of self-consciousness, transformation of time, and autotelic experience \citep{Csikszentmihalyi1990}).  All survey items used a 7-point Likert scale.  Responses were collected within one hour of activity completion, with the aim of gathering the data as close to the finish of an activity as possible, while minimising intrusion on the participants \citep{Jackson2004}. For full details concerning survey, see Appendix ~\ref{app4:ethnoSetting} Section ~\ref{sect:flowStateScale}.


    \myparagraph{General survey administered at two time points (longitudinal)}
    I asked athletes to comment on experiences of joint action and group membership at two points in time spread three months apart.  These survey items included experience of agency in the team (weak-strong), perceived role in the team (very small-very important), perceptions of individual performance (poor-good), perceptions of team performance (poor-good), training intensity (\textit{qiangdu})(low-high) and difficulty (low-high).  All survey items were measured using a 7-point Likert scale. For a full description of survey questions, see Appendix ~\ref{app4:ethnoSetting} Section ~\ref{sect:generalSurvey}.



\subsection{Data analysis\label{sect:dataAnalysis}}
Field notes from participant observation, interview scripts, and informal survey responses formed a corpus of ethnographic data that was subjected to a process of ``thematic analysis'' \citep{Braun2006}.  As Braun and Clark \textcite[10]{Braun2006} explain, ``A theme captures something important about the data in relation to the research question, and represents some level of patterned response or meaning within the data set.''  Identification of recurring themes was guided by (but not limited to) the research question and theoretical predictions of this dissertation.  Themes were identified on both explicit (semantic) and implicit (latent) levels of the data \citep{Boyatzis1998}.  Theoretical predictions and relevant existing research concerning the social cognition of joint action helped direct analysis of the latent level of the data.

The thematic analysis involved three stages that unfolded in a recursive (rather than linear) fashion \citep{Braun2006}. In phase one, I familiarised myself with the each data source in the corpus and tagged relevant extracts with theoretically-guided ``codes.'' For example, upon encountering Hongwei's description of his position in the team in his interview transcript (cited in the Introduction ~\ref{}), I tagged this with codes such as ``group membership,'' ``mutual support,'' ``emotional support,'' ``knowledge of team roles,'' ``signalling commitment to team'' etc.  My coding system was thus directed by (but not limited to) pre-identified theoretical variables: 1) athlete perceptions and expectation violations surrounding joint action, 2) perceptions and feelings associated with the phenomenon of ``team click,'' and 3) understandings of and feelings relating to ``group membership'' and social bonding.  In addition, I looked for evidence of 4) possible moderator variables, such as technical competence and personality type.  For each data set, I created a data frame using Microsoft Excel (Version 14.7.1) in which research participants formed the rows, and distinct codes formed individual columns. Data extracts from interviews and field notes were imputed into the matrix, with an emphasis on including data surrounding each extract, in order to preserve context \citep[see][]{Bryman2001}.

In phase two, I sorted the different codes into potential themes and collated all the relevant coded data extracts within the identified themes and judged on the dual criteria internal homogeneity of codes within themes (coherence) and heterogeneity of codes between themes (distinction) \citep{Patton1990}.  I then produced a master data-frame (participants \times themes), including data extracts from all data sets. In phase three, I generated a definition of each theme, and a refined list of data extracts capable of representing that theme in subsequent analysis \citep{Braun2006}.




\section{Summary of ethnographic data}
In this section I summarise the the ethnographic data that I collected between September 2015 and September 2017.

\subsection{Rugby at the Institute after 2013}
As mentioned above, having let the dust settle on the embarrassment of the women's program's widely publicised disqualification from the 2013 National Games, in 2014 the Institute decided to quietly continue with both the men's programs in preparation for the 2017 National Games.  In April 2014, more than six months after the National Games, former Chinese representative and CAU coach Zhu Peihou was appointed as new head coach.  The junior athletes from the previous National Games cycle were recalled back to the Institute to resume training, and coach Zhu was charged with finding new talent to fill the ranks of the team.  Zhu had previously coached the Anhui women's provincial team during their 2013 National Games campaign.  Zhu appointed his close colleague and graduate from the Shanghai Institute of Sport's rugby program, Shi Yan, as assistant coach.  Both Zhu and Shi were employed by the Institute on a contract basis, rather than becoming full-time official employees of the Institute.

The women's program was inactive for a full two years after 2013, and was only just starting to re-activate after I arrived, in November 2015.  Former Beijing women's rugby representative (2010-2013) and Beijing local Ma Jiale was appointed as head coach, and former CAU graduate, Chinese National Team representative, Beijing men's rugby representative athlete, and Beijing local resident Wang Chongyi was appointed as assistant coach.

Thus, rugby was resurrected at the Institute, but was no longer in centre stage. The Beijing men's team endured a series of mediocre performances during the 2014 and 2015 seasons, and clearly lacked experience, talent, and institutional support from the Institute.  A handful of senior athletes who had played in the era of the 2013 National Games remained, and two in particular, Han Xiaolong and Lu Peng were promoted to a transitional athlete-coach status. Unlike Women's assistant coach and former athlete Wang Chongyi, however, both Han and Lu were originally from Shandong province and so did not automatically have Beijing residency required to make them eligible for full time employment at the Institute.  As such, their future place at the Institute was uncertain, and as I found out from both during the course of my ethnographic research, their ability to stay at the Institute would depend on the result the team could achieve at the 2017 National Games.

It is important to note that both coaches appointed to the women's program in 2015, Ma and Wang were---unlike Zhu, Shi, Han, and Lu---full-time permanent employees of the Institute. As explained in the previous chapter, China's infamously rigid residency system (``Hukou'' (户口)) means that only individuals with Beijing residency can hold permanent employment roles at government institutions such as the Institute (\textit{shiye danwei} 事业单位).  Chinese citizens born outside of Beijing can become Beijing residents if offered employment, but due to Beijing's swelling population, the eligibility criteria for this process of naturalisation has become more and more stringent, and fewer and fewer applications are successfully processed, particularly in industries like sport.

%\footnote{This links to the question of the ``Quality'' of athletes}
Despite being only a shell of its former glory, the rugby program at the Institute nonetheless offered attractive incentives to prospective athletes.  The difficulties of Han and Lu in gaining Beijing residency made it clear to more junior athletes that there was little promise of a passage to official Beijing residency or full-time employment at the Institute. But the program did, however, offer the much more realistic opportunity of attending the Beijing Sports University (BSU)---considered to be the country's most prestigious sports universities and one of China's top ``brand universities'' (\textit{mingpai daxue} 民牌大学).  The mass exodus of experienced senior athletes from the rugby program meant that junior athletes from the pre-2013 era were now in a position to represent Beijing at a national level, and therefore attain the official athletic standard of a ``Master Sportsperson'' (\textit{yundong jianjiang} 运动健将).  A Master Sportsperson was automatically eligibility to attend BSU through its arrangement with the Institute.


\subsection{Participants}
For the ethnographic section of this dissertation, I analysed data on a total of 26 athletes ($avg. age = 20.96, range = 17-27, SD = 3.17$) and four coaches, who were not included in the main analysis but provided important contextual information.  Athletes were included in data analysis if they participated in 1) a semi-structured interview, 2) at least one informal survey relating to experiences of rugby training and group membership, and 3) at least 2 months of training at the Institute.  See Table ~\ref{tab:ethnoDescriptivesTable} for a summary of athlete attributes, including team status, contract status,


% Please add the following required packages to your document preamble:
% \usepackage{booktabs}
\begin{table}[]
\centering
\begin{tabular}{@{}rl@{}}
\toprule
\multicolumn{1}{l}{}            Variable                 & Value        \\ \midrule
n                                                & 26           \\
Age (mean (sd))                                  & 20.96 (3.17) \\
Research Category = Senior (\%)                   & 10 (38.5)    \\
Training Age (mean (sd))                          & 3.34 (2.02)  \\
Years In Team (mean (sd))                          & 2.59 (1.80)  \\
\multicolumn{1}{l}{}                             &              \\
\multicolumn{1}{l}{\textbf{Athlete Status (\%)}}  &              \\
Master Sportsperson                              & 10 (47.6)    \\
Level 1                                          & 6 (28.6)     \\
Level 2                                          & 5 (23.8)     \\
\multicolumn{1}{l}{\textbf{Contract Status (\%)}} &              \\
Permanent Employee                               & 1 ( 3.8)     \\
Full Time Contract                               & 7 (26.9)     \\
Training Contract                                & 5 (19.2)     \\
Student Contract                                 & 6 (23.1)     \\
Trial                                            & 7 (26.9)     \\
\multicolumn{1}{l}{\textbf{Education Level (\%)}} &              \\
Graduate                                         & 1 ( 3.8)     \\
Undergraduate                                    & 12 (46.2)    \\
High School                                      & 10 (38.5)    \\
Middle School                                    & 3 (11.5)     \\
\multicolumn{1}{l}{\textbf{Home Province (\%)}}   &              \\
Shandong                                         & 11 (42.3)    \\
Beijing                                          & 6 (23.1)     \\
Jiangsu                                          & 3 (11.5)     \\
Liaoning                                         & 2 ( 7.7)     \\
Hebei                                            & 2 ( 7.7)     \\
Heilongjiang                                     & 1 ( 3.8)     \\
Fujian                                           & 1 ( 3.8)     \\
\multicolumn{1}{l}{\textbf{Previous Sport (\%)}}  &              \\
Athletics                                        & 16 (61.5)    \\
None                                             & 8 (30.8)     \\
Basketball                                       & 1 ( 3.8)     \\
Football                                         & 1 ( 3.8)     \\ \bottomrule
\end{tabular}

\caption{Descriptives statistics of athletes included in ethnographic analysis (n = 26)}
\label{tab:athleteDescriptives}
\end{table}


The team consisted of one permanent employee (\textit{zhengshi} 正式)\footnote{Senior athlete Su Hailiang was the only permanent employee, by virtue of the fact that he was a Beijing resident.}, seven full time contracted athletes (\textit{xieyi} 协议), five athletes who were provisionally contracted on a training contract, (\textit{shixun} 试训), and six athletes who were on "Student" level contracts (\textit{erjiban} 二级班).  Athletes contracted as students did not receive a salary but received in-kind support in the form of training, food, board, and health insurance at the Institute (see Table ~\ref{tab:athleteDescriptives}).  The remaining athletes (seven) were technically classed as athletes ``in camp'' (\textit{jixun} 集训), and were effectively on a trial arrangement until they showed promise or else withdrew from the squad, either voluntarily or upon suggestion by the head coach.  Most athletes were from urban and rural areas of northern China (Shandong (11), Beijing (6), Jiangsu (3), Liaoning (2), Hebei (2), Henan (1), Heilongjiang (1)). Athletes were, generally speaking, and from what I could gather, from relatively modest socio-economic backgrounds.

The average rugby training age (years spent playing and training in a rugby program) of Beijing athletes was 3.34 years ($range = 0.16 –-- 10 years, SD = 2.02$).  18 of the 26 athletes had a background in other sports (16 athletes from track and field, one from football, one from basketball).  These athletes usually began part-time or full-time physical training at the age of 11-13.  Those who transferred to rugby from other sports did so either at the beginning of senior high school (16 years) or at university age (18 years).  The remaining eight athletes had no particular sporting background before rugby.  These athletes were either scouted by the head coach of the Program or by school athletics coaches based on their basic athletic attributes (running speed, strength, coordination, and potential for physical growth).  Of the 26 athletes in the squad, three junior athletes who were part of the squad when I arrived in September 2015 left before August 2016; and three new athletes arrived during the time I performed research.  This flux of athletes in and out of the Program was quite common, due to the fact that recruitment often occurred informally via personal networks.




\subsection{Training schedule}

 The Beijing men's rugby team competed against other provinces in five national tournaments held in different locations across the country   every year between March and September. The period in which I conducted my first stint of ethnographic research (September 2015 –-- February 2016), therefore, constituted the off-season and pre-season components of the training year.  Due to cold weather in the north of China during winter and spring, teams from northern China (e.g. Beijing, Liaoniang, and Shandong provinces) often elected to train   at other domestic or international training locations depending on amount of program funding available and the training strategy of each program.  In 2015, before an unexpected change in coaching team at the end of December (explained below in Section HYPERLINK), the head coach of the Beijing Men's team had planned to travel to Yunnan province in early 2016 for one month of altitude training before moving closer to sea-level somewhere in the south of China for one month (February/March).  Following the coaching leadership change, the team did not leave Beijing until after Chinese New Year (25th February). Training during this period was therefore consistently stationed at the Institute in Beijing, and as such subject to occasional disruption due to Beijing's cold winter weather and air pollution.

 All athletes lived and trained 6 days a week at the Institute, and would occasionally attend university or high school classes as part of their ongoing education commitments.  Below is a table of a typical weekly training schedule (see Figure ~\ref{tab:tournamentData}). A typical week consisted of 10 two and a half hour (150 minute) training sessions, seven of which were on-field rugby sessions, three of which were strength and conditioning sessions (not involving a rugby-specific skills).  In addition, two one hour evening skills sessions were also allocated for junior athletes to hone their basic skills of passing, catching, and game-play.  Athletes lived full-time on campus in the Institute's dormitory accommodation (usually 3 athletes per room), and were permitted to take overnight leave on the weekend after the conclusion of Saturday morning training.  Athletes from Beijing or with family in Beijing would usually take this leave, while the remaining athletes would spend weekends at the Institute.  Generally speaking, the rugby program would break at the end of the national season in September for two weeks, and occasionally around Chinese New Year for 7-10 days, unless New Year interrupts pre-season training plans, in which case training would continue in spite of this national holiday.

  \newgeometry{margin=0.5cm} % modify this if you need even more space
  \begin{landscape}
    \begin{table}[htpb]\caption{Weekly Training Schedule}
      \begin{center}
        \begin{small}
            \begin{tabular}{| c | c | c | c | c | c | c | c |}
              \hline
              & \bf M & \bf T & \bf W & \bf T & \bf F & \bf S & \bf S \\
              \hline
              0600 & Training &  &  & & & & \\
              \hline
              0900 &  & Training & Training & Training & Training & Training &  \\
                \hline
              1500 & Training & Training & & Training & Training & Training &  \\
                \hline
              1900 &  & Training (junior athletes) & & Training (junior athletes) & & & \\
                 \hline
            \end{tabular}
                \label{tab:tournamentData}
          \end{small}
        \end{center}
      \end{table}
  \end{landscape}
  \restoregeometry




\subsubsection{Participant Observation}

  \subsubsection{Interviews}

  In addition to 26 semi-structured interviews, I also conducted 6 unstructured interviews with three members of the Program (the head coach and the two most senior athletes) and two former coaches of the Institute.  These unstructured interviews were not included in the main analysis but provided important ethnographic context for the main analysis.

  \subsubsection{Informal Surveys}

  All 26 athletes participated in at least one of the post-interview survey tasks. 12 athletes ($age = 20.92(3.64), range = 16 - 27$) participated in the survey following the Beep Test training; 16 athletes ($2.81 (3.25) range = 17 - 27$) participated in the survey following the first match-like training session, and 15 athletes ($21.81 (3.25) range = 17 - 27$) participated in the second match-like training session a week later.  A total of 23 athletes ($21.21 (3.27) range = 17-27$) participated in each of the the general surveys, administered one in late November 2015, and a second time in early February 2016. This general survey occurred either side of the change in coaching staff over the Christmas period of 2015.


\subsection{Team Factions\label{sect:teamFactions}}

After a short time at the Institute, my conversations with (predominantly senior) athletes and coaches helped me identify a number of factions in the team, which appeared to align not only with each athlete's position in the rugby program, but also by the incentives that each athlete was pursuing through participation in the program. The team could be roughly split into senior and junior athletes.

Senior athletes had represented the Beijing men's team or were contracted to the Institute. Junior athletes, by contrast, were still yet to do so.  Senior athletes  ($n = 10, average age = 24.3, SD = 1.57, range = 22 - 27$) had trained for an average of 5.4 (1.48) years.  Junior athletes ($n = 16, average age = 19.3, SD = 1.78, range = 17 - 22)$) had trained for an average of 2.18 (1.34) years.


\subsubsection{Senior Athletes}

Within the category of Senior and Junior athletes were a number of obvious sub categories in which athletes tended to cluster or self-identify.  Of course these categories involved some level of flux and fluidity as athletes progressed up the ranks, or due to personal friendships across category boundaries.

\myparagraph{What was left of the Old Guard (2)}
At the top of the hierarchy were senior players Han Xiaolong (27 years) and Lu Peng (26 years). Han and Lu had been playing rugby for eight and seven years respectively---the longest of any athletes in the team.  Han's education situation was somewhat of an anomaly, because at the time of this ethnographic study he was still yet to finish his undergraduate degree (due to delays incurred during the 2010-2013 National Games campaign). Lu had well and truly finished his undergraduate degree, but had not continued to keep studying beyond undergraduate level. Han and Lu were ultimately both potentially interested in the prospect of gaining full time employment at the Institute, and thus Beijing residency.  At the time of conducting this research, the Institute employed both athletes on full time contracts, but neither were formal permanent employees of the Institute.

According to both Han and Lu, formal permanent employment and Beijing residency were part of the incentives that were originally tabled in contract negotiations in 2010 when both athletes were enticed to the Institute from CAU as young and promising athletes.  The Institute reportedly set performance targets for the programs of a Women's gold medal and a top-three finish for the men.  Given the chaos that transpired at the 2013 National Games (see Chapter ~\ref{chap:researchSetting} Section ~\ref{sect:fallFromGrace}), however, the Institute's promises did not materialise for athletes like Han and Lu, even though the men technically achieved their performance goal.
    \footnote{As it turned out, many of the senior athletes in the men's team pre-2013 were already Beijing residents, due to the fact that many were from CAU.  Thus, the Institute's promise of residency applied to only a small portion of athletes that they recruited.  After many athletes were enticed away from Beijing to the Tianjin rugby program in early 2014, the only first team athletes to remain at the Institute were Han and Lu.  Both expressed frustration to me about not receiving the incentives that both felt they had rightfully earned, and this sentiment motivated a lot of complaining and problematisation of the program.}
When I consulted outside observers on this situation, some suggested that the fact that both Han and Lu had not been granted the incentives that were promised to them was a deliberate ploy by the the leadership of the Institute to motivate commitment to the team at least until the end of the 2017 National Games.  Suffice to say, both Han and Lu sat atop the athlete hierarchy of the men's team in an uncomfortable position in which they were not confident that they would achieve what they wanted from their involvement in the Program.  Particularly given that hopes for success in 2017 were uncertain at best, due to the way in which the men's team had been decimated after the events of 2013, it did not appear that either would come away with what they wanted out of the situation.  During my time at the Institute, out of all of the athletes, I interacted most directly and personally with Han and Lu, given that they were the most senior athletes and we had known each other previously from my time spent at CAU.

\myparagraph{The new senior athletes (8)}
Below Han and Lu was a collection of eight athletes who were now considered members of the ``first team'' (\textit{zhuli} 主力).   These athletes were either on full time contracts, or otherwise had been promised full time contracts since they had represented Beijing at a national level.  Unlike Han and Lu, none of these new senior athletes were central to the senior team pre-2013, and thus occupied a distinctly different space in the team hierarchy.  Many of these athletes were recruited between 2010-2013, from other programs at the Institute or from other sport programs such as football and athletics.  As such, very few of these new ``senior athletes'' expressed to me any strong short-term aspirations for attaining permanent employment and residency (although some may have harboured such aspirations in the longer term). Five of these athletes were in the process of completing their studies at BSU, and the remaining three had already completed their undergraduate degrees at CAU.

%WW, MHT, WWX, WZF, SHL
\subsubsection{Junior Athletes}

\myparagraph{The unruly undergrads (5)}

Below the senior athletes were a group of five athletes who had been recruited into the program between 2010-2013, but had not subsequently progressed through the ranks to the first team. This group had, however, managed to qualify to study at BSU by virtue of their participation in national level tournaments in 2014 and 2015, after the program at the Institute was resurrected.  As such, four of these athletes were beginning their first year of study at BSU, and the remaining athlete was preparing to do so. There was a clear distinction in technical competence between these athletes and the first team, and as such there was not a whole lot of competition between these athletes and the first team athletes for a spot in the first team.  Three of these athletes were on ``training contracts,'' while the remaining two were still on ``student'' contracts.  I call these athletes the ``unruly undergrads,'' because they were often criticised by coaches and more senior athletes for being constantly distracted by university life, and not having any motivation to commit to the rugby program.  Many senior team members inferred that because the key incentive available to them (university attendance) has already been awarded to them, they lacked material motivation to endure the costs associated with rugby.


\myparagraph{The aspiring athletes from Chaoyang Sports School (5)}

Below the unruly undergrads are a group of five athletes who are also high school students and athletes at the rugby program at the Beijing Chaoyang Sports Institute, a city-level high school institute located in the east of Beijing.  These athletes had been playing rugby for one to three years, and were all attached to the Institute through student contracts.  Second training class athletes received coaching, board, and food, but no form of remuneration.  All athletes are aspiring to transition to full time members at the Institute and representatives of the Beijing men's team.  In order to qualify for admission to the Institute as a contracted athlete, these athletes must first complete high school and attain the standard of a Level 1 athlete (\textit{yiji yundongyuan} 一级运动员), which can be achieved by representing the Beijing Youth Rugby team at a national level youth tournament.  Once these athletes become full members of the program at the Institute, they will then be able to pursue subsequent opportunities such as university attendance.

\myparagraph{The hopefuls on trial (6)}
Finally, the remaining group of six athletes were all at the Institute on a ``trial'' basis.  These athletes usually appeared via some connection to either the head coach of the Institute or via the relational network of one of the Principals at the Institute, as was the case with SHW (see Chapter ~\ref{chap:theory} Section ~\ref{sect:SHW}). The position of these athletes at the Institute was deeply uncertain.  Most had transitioned from athletics or another mainstream sport in which they had achieved a minimum standard of performance in their event (Level 1) thus making them eligible to attend the Institute.  Most, however, had not played rugby before, and so had a large gap in technical competence that each was attempting to address.




\subsection{An abrupt leadership transition\label{sect:leadershipTrans}}
Zhu and Shi were ostensibly in charge of the team when I entered in 2015. There were perhaps some signs (if I were either looking closely or knew where to look) that their authority was not rock solid.  If I had understood then as much about the machinations of politics and power in Chinese sport as I do now, then perhaps I would have understood Wang's disengagement from the post-Match team huddle in Qingdao was a clear signal that the writing was on the wall for Zhu, and that Wang---a Beijing resident, full-time employee of the Institute, and Asian Games Bronze medallist---would eventually take his place atop the Men's program. Upon reflection, during the first four months of my ethnographic research (i.e., the ``Zhu era'' of leadership), key senior players like Lu and Wei Wenxin complained to me constantly about Zhu and his approach to coaching and team management.  I could also tell the women's coach Ma Jiale was frustrated by Zhu and his communication style.  And then there was the interaction with being forced to take on SHW, and the lack of support from Institute leadership for off-season training plans overseas... It was also the case the Zhu had barely taken an on-field training session since I had arrived to do my research.  After a few weeks of me settling in and observing, Zhu would suggested that I take over half of the on field sessions.  There was a period in which he barely took training that he announced to the team that he was focussing on the injured players instead. Indeed, in hindsight all of these observations could have been interpreted as writing on the wall.

In reality, coach Zhu and coach Shi were perhaps only ever going to be temporary stop-gaps for the Institute's rugby program.  Zhu and Shi were both outsiders, employed to come in and do the thankless job of resurrecting a rugby program from a shameful fall from grace.  But at the time, only four months into my first stint of ethnographic research and still largely unacquainted with (and without access to) the deep details of politics at the Institute, I was not prepared for what happened when I returned to Beijing from two weeks at home in Australia over Christmas.

When I returned to Beijing on the 31st December, I stayed the night at my friend Kai's place, and the next morning I messaged head coach Zhu to ask when would be best to return to the Institute to resume research.  His reply was brief and somewhat odd: ``You should be able to [return to the Institute], you should probably make contact with Han or maybe Coach Wang.'' (应该可以,你联系一下小龙呗,或王导呗). I found this reply to be a little bit confusing and opaque, but I didn't challenge it, as was the polite norm with someone like Zhu who was my senior.  Instead I contacted Han, as directed by Zhu, and asked if I could go to the Institute that evening and if the team had the afternoon off: ``Mmmm yeah.. we're off training today. I'm currently out (not at the Institute). Teacher Zhu and Coach Shi have left...now Wang is head coach.  You're still in Room 113 for the time being...'' (嗯嗯 休息 我在外面 朱老师跟石导走了. 现在由王重一主教练 你还是先住113).  I didn't really register what he was saying and replied:

\begin{quotation}
  JT:So Zhu and Shi have gone home to rest today too? \\
  HXL: No, they are no longer coaching Beijing...When you get back we can meet and I will explain it all to you, ok? \\
\end{quotation}

\begin{quotation}
  JT:朱,石,回家休息是么 \\
  HXL: 不是的 是不在北京做教练了等你回来了 我们见面再好好跟你说说吧
\end{quotation}


And like just like that, Zhu and Shi were out, and Wang had assumed the role of head coach of the men's team.  Han and Lu became his assistant coaches, and some weeks later, Zhu Jing, a classmate of Wang from CAU who represented Xinjiang province in the 2013 National Games, joined Wang as assistant coach in Beijing some three weeks after Wang took over.

In the aftermath of the change of leadership, I heard gossip that Zhu had pushed the Institute to provide more funding towards the team's off-season training program.  Zhu  made it quite clear that he wanted to take the team either to Fiji in a best case scenario, otherwise to high altitude in Kunming (in China's southwest), after which they would train in Guangzhou before the first National Tournament in March.  After the Institute (presumably Principal Jenny) refused to support these plans, Zhu allegedly made the call to resign from the role, saying that there was no way he could do his job properly without support of the Leadership.

Soon after the change, there appeared to be a renewed energy in the team.  Senior players Han and Lu had been promoted and appeared to respond to this promotion with positivity and enthusiasm.  Wang, Han, and Lu all shared the same coach (Zheng Hongjun, the old boss of Chinese rugby) and so they all saw eye-to-eye about training techniques, based on the common habits instilled by Zheng's approach).  The training schedule immediately reverted to an older more familiar format, and the content of training was dedicated largely to basic skills and fitness, in order to support a style of play that Zheng pioneered in Chinese rugby.  Zhu had been attempting to lead the Beijing team towards a different style of play in the years that he had been coach, and this often frustrated the more senior athletes in the team who had learnt rugby under Zheng's regime.  I got on well with Wang (we had known each other from CAU days in 2008 and 2013), and so I also experienced an increase in positivity and feeling of increased inclusion after he took control of the team.


\section{}




                                                          \end{CJK}
