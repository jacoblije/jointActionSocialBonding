Humans display an unmistakable tendency come together and move together.  Amidst all of the scientific questions that we ask of ourselves (and of biological life more broadly), how and why we move together are two fundamental questions that have evaded sufficiently detailed and rigorous consideration, until recently.  The key assertion of this dissertation is that a scientific explanation for the prevalence of physiologically exertive and socially coordinated movement (group exercise) in human sociality can benefit from closer attention to the mechanisms of joint action (defined herein as the dynamic coordination of behaviour between two or more co-actors).  I propose novel theoretical synthesis and conduct focussed empirical research to explore and substantiate this assertion.

The ubiquity of group exercise in the human record suggests a relationship between group exercise and social cohesion.  Existing evidence supports a ``social high'' theory of group exercise and social bonding, whereby sustained moderate intensity physiological exertion and exact behavioural synchrony combine to generate a psychophysiological environment conducive to affiliation and trust.   However, examination of anecdotal and ethnographic observations reveal a broader spectrum of profiles and experiences in group exercise, many of which involve extreme physiological cost, complex coordination, as well as rich social and cultural meaning-making.  These knowledge gaps in the social high theory appear to involve an interlocking of physical, cognitive, social, and cultural processes that cannot be satisfactorily explained through attention to only one discrete process in isolation \citep{Kenrick2001}.

To address this theoretical challenge, I consider the utility of the emerging paradigm of ``active inference'' \citep{Friston2010} to more fully account for the dynamical properties inherent in social cognition of joint action that have been traditionally overlooked by cognitive and evolutionary explanations of human sociality.  I focus specifically on the experience of ``team click'' in complex multi-agent joint action typical of group exercise contexts.  Team click is an anecdotally widespread and theoretically grounded construct that pertains to the subjective feeling associated with optimal interpersonal coordination.  The dynamical and unifying active inference framework allows me to formulate a novel theory of social bonding through joint action, in which the traditionally distinct (albeit complimentary) physiological, cognitive, and social processes of information transfer are more satisfactorily integrated.

To empirically evaluate this theory, I conducted ethnographic and field-experimental research with professional Chinese athletes who participate in the competitive interactional team sport of rugby union.  Results provide evidence for a relationship between athletes' perceptions of success in joint action and social bonding, mediated by the phenomenon of team click.  On average, athletes who report more positive perceptions of team performance report higher levels of perceived team click.  In turn, higher levels of team click predict higher levels of social bonding.  Finally, mediation analyses show that a direct relationship between joint action and social bonding is mediated by team click.  Results reported here suggest that the experience of positive violation of expectations of team performance may be at the heart of the affective, agentic, and visceral experience of team click, and may in turn set the foundation for higher-order processes of social bonding.

Considered in light of existing debates on the cognitive and evolutionary mechanisms of social cohesion, these results suggest that processes of bio-psycho-social alignment achieved through interpersonal movement regulation and coordination could be central to the formation of durable social bonds and the transmission of cultural variants between individuals and within groups.  Thus, by considering the cognitive and evolutionary explanation for group exercise in human sociality, this thesis supports the need for a more dynamical understanding of humans' social cognition and evolutionary trajectory.






Alternate:

This thesis tests the hypothesis that the phenomenon of team click mediates a relationship between joint action and social bonding in group exercise contexts.  Team click refers to a visceral and agentic experience of optimal performance in joint action.  Although anecdotal evidence is pervasive, theory-driven investigation into the phenomenon is lacking.  In this thesis I develop a theoretical grounding for team click based on the proposed relationship between dynamical coordination of physical movement and social communication in joint action.  I formulate a novel theoretical account of social bonding through joint action, which I test with ethnographic, survey, and field experimental data collected with a population of professional rugby players in China.  Results support the prediction that more positive perceptions of successful team performance in joint action leads to higher levels of social bonding in group exercise, mediated by higher levels of team click.  Results also suggest that positive violation of expectations surrounding team performance may play an important role as a precedent for team click in complex and uncertain joint action scenarios.  From these results I infer that team click establishes an embodied and social template for more abstract cognitive processes characteristic of social affiliation and group membership.  Empirical evidence presented in this dissertation warrants further research into  the role of team click in group exercise contexts will have implications for scientific understandings of humans' culturally rich and socially cohesive evolutionary niche.
