
\section{Abstract}
In this chapter I present the results of ethnographic data collected with the Beijing Men's Rugby team between September 2015 and August 2016.  I find a range of evidence in support of the predictions set out in Chapter 3.  First, ethnographic data show evidence  for a culturally specific terrain of rugby at the Temple institute, particularly the interaction of two different modes of cognition--relational and categorical--within a turbulent modern Chinese history in which cultural practices including sport have been imported en masse in service of a modern Chinese nation state. In particular, the chronic dominance of the relational mode of cognition appears to pervades the structure of the institutions, social norms, and on-field action-perception processes associated with joint action of rugby at the Temple institute.  Instituttion:  Norms: Active Inference:

Cultural modes of group membership and self construal act as ``hyper-priors'' or ``coordination smoothers'' for the structuring of active inference.  The specific context of rugby, and the position of the researcher as an expert observer of rugby from a Western context, while not without problematics, nonetheless serves as an interesting way of exposing the divergent modes of cultural cognition, and denaturalises both received intuitions about rugby as well as China.  Allows us to avoid the potential pitfalls of ethnographic method and resist reifying either the practice or the cultural milieu as an either/or relational/categorical system.

Second, I show how within this culturally specific terrain, contours of testable proximate mechanisms relating to joint action, team click, and social bonding can be identified.   In terms of athlete perceptions of performance, athletes appear to develop strong subjective perceptions of individual and team performance based on specific technical components of rugby.   Athletes---predominantly junior athletes---on the one hand display evidence of anxiety when expectations around team and individual performance are negatively violated.  Conversely, when the same expectations concerning team and individual performance are positively violated, the same athletes are noticeably exhilarated, energised, and motivated.

More senior athletes, by contrast, display the ability to relate to perceptions of individual and team performance expectations with more personal agency and emotional control.  More senior athletes talk in terms of an ability to strategically mitigate the costs of physical exertion and injury risk of joint action in rugby,

These observations are substantiated by survey results that show that mean performance related anxiety of junior athletes is higher than that of senior athletes in the starting team.

These results suggest that perceptions of team and individual performance in joint action are formulated in relation to specific components of team and individual performance in joint action, as well as more general team and individual expectations structured by the specific team context, social norms, and institutional constraints.

In terms of Athletes' feelings relating to notions relating to the construct of ``team click,'' Athletes relate to optimal team performance either with strong but generalised emotionality (predominantly junior athletes), or with a more restrained but more specific understanding of the dimensions and factors required in order to achieve peak team performance (predominantly senior athletes) (perhaps rooted in self-defining memory -- whitehouse?).

In regards to dimensions of social bonding, athletes show evidence prioritise feelings of emotional support and shared goal between teammates, as well as a level of personal connection to team roles and team identity.  Testimonies of team identity vary from the initial excitement and positivity of new arrivals towards, to the realisation of irreversible attachment to rugby of middle-rung athletes, to the out and out identity fusion between senior athletes and rugby, whereby athletes recognise that their self-concept is inextricably with rugby and the Temple institute team in particular.


Finally, contained within the collected data is evidence that individual differences may moderate relationships between perceptions of joint action, team click, and social bonding.  Namely, an athlete's experience and competence with the technical and social dimensions of joint action of rugby at the Temple Institute may impact upon the ways in which they formulate and respond to expectations surrounding and team performance.  Individual differences in technical competence may also have flow-on effects for the formulation of feelings concerning team click and social bonding. It appears that individual personality types may moderate perceptions of performance in joint action, as well as feelings of team click and social bonding. In addition, other facts such as the injury status of the athlete and the level of fatigue and exertion experienced in joint action scenarios of rugby may impact on an athlete's ability to feel the ``click'' of joint action.  These individual differences should be considered in ongoing studies designed to test the relationship between joint action, team click, and social bonding.


Contribution: (See Cohen2007 and CohenWhitehouse2012)
The first time ethnographic evidence has been provided for mechanisms of joint actions
(Wacquant: body and soul), Body for China (Brownell1995)

This ethnographic theorises divergent cultural modes


Cultural specificities may act as ``factors of attraction'' \citep{Sperber2014} that constrain and direct the fixation of cultural variants within and between populations.

\section{Vignette: Qingdao}
Qingdao:
Kai's hometown was Qingdao, and so we travelled down together on the high speed train after Kai finished work on the Friday evening in time to watch the Asia Sevens Tournament on the Saturday.  The Asia Sevens is an annual series of three regional rugby sevens tournaments featuring men's and women's national sevens teams.  The men's series has been held regularly since 2009, and the women's series was established in 2013. The series usually consists of two three annual Tournaments, alternating between various locations including Hong Kong, South Korea, Sri Lanka, China, Japan, Malaysia, India, Singapore, and Thailand.  In 2015, The Asia Sevens Series Tournaments were held in Colombo, Bangkok, and Qingdao. Would be played on the Saturday and Sunday, and then the National Tournament would follow immediately after on the Monday and Tuesday.


Olympic Qualification:

When I arrived at the stadium in time for the beginning of the Asia Sevens Series, I met a number of coaches and athletes in the stands who I knew from my time in Beijing in 2008 and then coaching in 2013. I sensed from my various interactions that there was an air of nervousness around the Tournament, particularly on behalf of the Chinese women's team.   The Qingdao tournament was the final tournament before the Olympic Qualification Tournaments, to be held in Hong Kong and Tokyo in November 2015.  The top ranked team from those two legs would qualify for the Rio Olympics in 2016.  The Chinese men's team were not in serious contention for Olympic qualification, given the clear superiority of the more established men's rugby programs like Japan and Hong Kong. The Chinese Sports Commission did, however, expect the Chinese women's team to qualify for the Olympics.

Since the Chinese women's team was first established in 2002, China had made great strides in women's rugby, in both Asia and globally.  Not including occasional losses to closest rivals Kazakstan, between 2002 and 2012 the Chinese women's sevens team was the dominant women's team in Asia, easily outcompeting Japan and Hong Kong, and at times was competitive against the world's best including New Zealand and Australia.  The main reason for China's dominance in the women's game during this period was that other traditional rugby nations, despite having developed professional infrastructure for the men's game, lacked almost entirely an equivalent infrastructure for the women's game. China`s state sponsored sport system, on the other hand, was relatively agnostic towards gender in sport. According to the incentive structures of the Chinese sports system a gold medal is a gold medal, regardless of the gender of the recipient. (This is not to say that there are not distinct gender inequalities in relation to sport in China).
Indeed, beginning with the Chinese Women's Volleyball Team's gold medal victory at the LA Olympics in 1982, China has enjoyed a comparative advantage in women's sport due to the fact that the Chinese sport system was comparatively more supportive of women's sport.

Alarmingly for the Chinese women's rugby team's hopes of Olympic qualification, by 2014 it had become obvious that other more traditional rugby playing nations in Asia, namely Japan and Hong Kong, had begun to make up serious ground on China and Kazakstan in the women's game.  This naturally prompted nervousness among the GAS and CRFA.  In mid 2015 it was decided by the GAS and CRFA that they would enlist the services of a foreign coach.  According to sources close to CRFA, apparently the original plan was for the appointed foreign coach, BG, to act as a consultant for LXH and his existing group of coaches.  By the time I arrived in Qingdao in early September, however, the initial arrangement had since transformed into a situation in which BG was given 100\% control over the program as head coach, and LXH was more or less sidelined as coach.

The complication was that before being dethroned, LXH preferred to use his own athletes.  Most of the starting team at the LDN7s in June 2015 were indeed Shandong athletes, many of the women who had won gold at the National Games in 2013.  When BG took over the reigns as head coach, as directed by the GAS and CRFA, he set about reorganising the starting team and also scouting for new talent outside the squad, which was predominantly made up of Shandong athletes.

When I arrived in Qingdao it appeared that tensions between the old and new guard were at their peak.  LXH and many of his favoured athletes had been relegated to the bench, and some had been completely removed from the squad altogether.  There were 6 weeks to go before the all important first Olympic qualification tournament in Hong Kong.  I sat and watched the first few games of the women's Tournament, and I was indeed surprised to see that the Chinese women's side was missing some of its usual stalwarts, and indeed appeared in my eyes to lack the flow and familiarity that I had come to expect in LXH's clan.  In the stands I came across a group of Shandong women, one of which, QGG I knew quite well from when she toured to Australia with the Shandong team when I was still with the Australian rugby sevens team in early 2013.  I asked QGG why she wasn't playing for China, and it turned out that she was injured, and so wasn't eligible for selection.  But a few of the other women around her, who were all wearing Chinese national team uniforms, were all part of LXH's clan who had been effectively stood down by BG. ``How do you think they're playing?'' She asked me, after we had exchanged pleasantries. ``Not great'' I commented, hesitating, not knowing how much I should prime her, but also feeling obliged to be honest: ``it feels like they’re not coordinating together very well at the moment.  What do you think?'' (一般吧。感觉她们的配合不太好,目前。你觉得呢?) I asked.  ``They’re out there playing as individuals, not playing as a team! They can't get it together; there's no shared goal.'' `(她们都在打个人的,不打团体提的。打不到一块儿去啊,没有共同的目标.)  ``Hmm. Yes it does look like that.'' ``Hey, Lijie...'' she asked me quietly, ``...don’t you think they’re not even playing as well as our Shandong team could play?'' (嘿,李杰,是不是她们现在打的没有我们山东队打的好,是吧?)

As was common on my journeys through the world of rugby in China, I often did not quite grasp all the details and pieces of the puzzle that contextualised the interactions I was having until after the fact.

Later that day I bumped into LS, who had been assistant coach of China with LXH and now BG since 2014.  LS was a Qingdao local, a CAU graduate, and a member of the Beijing Men's team from 2010-2013.  I asked him about the current situation with the Chinese women's side and their prospects 6 weeks out from the first Olympic qualifier.  ``Chinese athletes need to see the (individual) benefits if they are going to put their bodies on the line and put in for each other'' he insisted, and he went on to explain why for these athletes, there were no obvious benefits available sufficient to motivate them.  There were indeed very few material benefits associated with representing China in rugby at the national level.  Athletes were payed a nominal USD100 per month on top of their provincial contracts when training and touring with the national program.  If athletes were injured while playing for China, CRFA at the time did not have access to sufficient health insurance to cover the costs of treatment, and athletes had no choice but to return to their provincial programs and seek treatment at the expense of the province.  The less tangible benefits of playing for China, for example, access to high quality coaching, or the pride of representing the country in a sport, or even the promise of a trip to the Olympics, were heavily outweighed by other less tangible costs: long stints of time away from family, the constant risk of falling out of favour with provincial programs...  In effect, the lack of incentives at the national level meant that athletes were by definition more committed to their provincial systems---the programs that provided athletes with the benefits that they were most interested in obtaining, such as tertiary education, future employment, modest but compared to CRFA, a reasonable salary (most national level players were paid 3-8k RMB/month). ``Its no wonder these athletes aren't performing well,'' LS exclaimed.

EXPLANATION: weakness of the institution?

Parallel to Beijing?


Stand talk about attributes and critical of technique etc.
Basketball:



National Women's teams: Li Sheng
Beijing Team: Huddle at the end










\section{Descriptives}

\subsection{The Beijing Provincial Men's Rugby Team}
During my time performing ethnogrpahic research at the Temple Institute, I collected data on a total of 26 Athletes ($avg. age = 21.3, range = 17-27, SD = 2.96$). All 26 participated in one scheduled, semi-structured interview.  All athletes participated in at least one informal survey relating to experiences of rugby training.  See table ~\ref{} for a summary of athlete attributes, including team status, contract status, etc.


Athletes were either already contracted with the Temple institute as either students, full-time contracts, or formal employees of the institute (in the case of ) (n=13) or aspiring professional athletes (n=13) who live and train 6 days a week at Xiannongtan and occasionally attend university or high school classes as part of their ongoing education---what can only be termed part-time education.  A head coach and assistant coach look after the day to day organisation of team schedules and training, and these two coaches are assisted by a further two player-coaches, who are in the gradual process of moving from athlete to coach status.  One of Xiannongtan's four principals is responsible for the management and administration of the rugby program, and is occasionally present at team meetings and national competitions.

Athletes are all from relatively modest socio-economic backgrounds, many hailing from suburban and rural areas of northern China (Shandong (11), Beijing (6), Jiangsu (3), Liaoning (2), Hebei (2), Anhui (1), Henan (1), Heilongjiang (1)).  The squad consists of 10 fully contracted senior athletes (\textit{xieyi}), three provisionally-contracted athletes \textit{shixun}, and six student-athletes \textit{erjiban} who do not receive a salary but receive training, food, board, and educational support.  The remaining athletes (7) are classed as athletes in training \textit{jixun} and are effectively on trial until they either show promise or withdraw from the squad either voluntarily or upon suggestion by the head coach.  Provided that they meet the relevant academic and athletic requirements, contracted and student athletes are able to attend the Beijing Sports University—considered to be the country's most prestigious sports university and one of China's ``top brand universities'' (\textit{mingpai daxue}).

The average rugby training age (years spent playing rugby) of Beijing athletes is 3.12 years ($range = 0.16 – 10 years$).  Contracted senior athletes ($average age = 24.3 years$) have trained for an average of 5.4 years, whereas the average training age of junior non-contracted athletes (average age = 19.3 years) was 1.7 years.  Over half of the athletes have a background in other sports (15 athletes from track and field, one from football, one from basketball), usually beginning part-time or full-time physical training at the age of 11-13.  Those who transferred to rugby did so either at the beginning of senior high school (16 years) or at university age (18yrs).  The rest of the group (n = ?) had no particular sporting background before starting at Xiannongtan, and were scouted by school athletics coaches based on their basic athletic attributes (running speed, strength, coordination, and potential for physical growth).  Of the 28 athletes in the squad, three junior athletes who were part of the squad when I arrived in September 2015 have now left, and three new athletes have arrived. This movement of non-contracted players in and out for trials is quite common.



      \subsection{Participant Observation}
      For 6 months in 2015-16 and 6 weeks during July-August 2016, I lived and trained full time with the team at XNT, during which time I took daily field notes using a note taking application (Evernote, version 6.11), which was synced between my mobile phone and personal computer. I collated, summarised, and tagged these notes weekly or fortnightly.

Notes observing training
Taking training 2-3 sessions per week, mainly on defence

      \subsection{Interviews}



      \subsubsection{Surveys}
        % Beep test
        % Scratch matches (2)
        % motivations for rugby?
        % closest friends
        % Competence
        % prosocial


\section{Analysis of Study Predictions}


  \subsection{Culturally specific terrain}

    \subsubsection{Institution as platform}
    %P: Institution as platforms with incentives and constraints, but not sacred in and   of themselves, less sanctimony around the institution than my intuitions prepared me for.

    %F: motivations are not for teammates, but for personal strategic life-course opportunities.
        %Rugby/Employment: Old heads (1st team) and undergrads (BYH, MXK etc)
        %junior: Young bucks (Chaoyang crew) and freshers (SHW etc)
        %family strong, team weaker but still present; coach??
        %F: CSC: not motivated, kicking the bucket and conversation with ZPH
        %F: MHT explaining connection between his mood and study
    %E:

    %P:coaches organise clans :
    %F: ZPH --> WCY transition:
       %WWX criticism of ZPH; subsequent alignment with WCY & ZJ
       %F: When ZPH left, he took with him XG and LJX (LJX stayed committed to the institution of BJM, present at National Games 2017).
       %F: WCY's clan from Shunyi, Hebei
    %E:

    %P: Power of regulation:
    %F: BYH, SHL, and WCY privileged position: contrast with HXL and LP (and ZPH).
    %E:Setting up regulations in institutions to incentivise and constrain, but not something that generates trust in the category of the institution itself.

    \subsubsection{Social norms}
  % HXL team Dinner speech and Johnny Zhang's Kids... what is going on here?
  % Team as team, but also as family (Team meeting, WCY, etc)
          % Self-determination in relational system:
                  %Younger players: Coach agency but my choice (figures?)

                  %Younger player aware of team roles
                  %Senior players: WW, CSC aware of cycle from young to old, whereas others are very critical of younger players not motivated.
          % ``Its all very complicated (in China)''

          %ZJ: can't move without leadership support (ZJ) conversation

  %HXL: return to his period of absence from the team after final tournament.

    \subsubsection{Action-Perception}

%can these processes be located in action-perception (active inference)?

          % Action scrutiny:
            %reference to BBall:

          % distributed dissonance:
            %training would fall apart when senior players absent
            %


  \subsection{Contours of Generalisable Mechanisms}

    \subsubsection{Performance}

      \myparagraph{Performance related anxiety (predominantly Junior)}

          \subparagraph{Team Awareness}
        %Anxiety around complexity of rugby, and the ``awareness'' required to execute team based joint-action:
        %Anxiety included reference to specific components of performance
        %TEAM: attack, defence, support play, communication

          \subparagraph{Individual Social Shame}
        %Anxiety around individual letting the team down due to individual mistakes
        %Anxiety included reference to specific components of performance
        %IND: contact, tackle, passing, decision making, support play in attack

      \myparagraph{Positive Violation of expectations: Performance related exhilaration and generalised emotionality (predominantly Junior)}
        %SWH story: the buzz and glow from acquisition of skills
        \subparagraph{Ind components}
        %WZF: explanation of the first side-step
        %YC: Fending - the feeling of picking up
        \subparagraph{Team performance}
        %WZF: likens team performance to side-step
        %HXL: rugby wasn't interesting (wasn't motivated until he started to pick up the team dimensions)

  \myparagraph{Strategy in individual performance and deflection of responsibility (more predominantly Senior athletes)}

          \subparagraph{Team Awareness: Agency over, deflection of responsibility}
       %Deflection of responsibility towards junior athletes: criticism of junior athletes for not committing, playing computer games, being complacent.  (Irony that LP was one of the most vocal when discussing computer games at the dinner table).
       %Others more generous and circumspect: WW and CSC realise that its a progression, and that individuals
       %EXPLAIN: reduction of dissonance via deflection
        \subparagraph{Individual performance: strategy}
       %MHT, CSC, HXL, WWX, MC use of experience to strategically avoid over-exertion and injury risk.


      \myparagraph{Survey Results}

      SURVEY RESULTS: relatively equal levels of flow overall, but higher performance related anxiety in junior athletes, particularly in the scratch matches, which required high levels of technical competence and joint action coordination.

      %EXPLAIN: perception of performance in relation to social expectations of the team: joint action participation, as well as individual responsibilities (team member, self-determination)

      %Senior athletes talk with more composure regarding performance, any deflect responsibility for being the agent of team performance, and show strategy regarding how to regulate energy expenditure.





      \subsubsection{Team Click}

Either:
        \myparagraph{Generalised emotionality, equating click with bonding (Junior)}
Or:
        \myparagraph{inability to conceive of click, not qualified to talk about it}


Either:
        \myparagraph{Familiarity and granularity/precision (Senior)}
        %HXL: Aura/atmosphere of the teams
        %WW: flow description
        %WZF: consideration of all the individuals
        %CSC:
        \myparagraph{Problematised}
          %Juniors for not being committed (Undergrad loafers)
          %Coach/Leadership for not supporting (LP)
          %China:
            %the system - No initiative, Chinese society too complex
            %the culture - Confucian education
          %Nostalgia for old regime pre 2013: HXL, LP,
          %E: a cognitive strategy in reorganising information in a way that maintains coherence (Nowak 2017)

  \subsubsection{Social Bonding}

    \myparagraph{Emotional Support}
    %MC, LZS,
    \myparagraph{Shared Goal}
    %WW:

    \myparagraph{Identity Fusion}
    %Fun/interesting/compelling
    %Attachment: MXK, LZS,
    %Fusion: HXL, LP,

  \subsubsection{Moderator Variables}
      \myparagraph{Technical Competence}
      \myparagraph{Personality}
      \myparagraph{Injury}
      \myparagraph{Fatigue}

\section{Discussion}

The cultural hyper-priors of JA - Team Click - SB <- factors of attraction for cultural evolution.
Relational and Categorical modes in interaction and flux in Chinese sport.

\section{Conclusion/Contribution}
Cross-cultural psychology / anthropology, CAT

recap


























%%%%%%%%%%%%%%%%%%%%%%%%%%%%%%%%%%%%%%%%%%%%%%%%%%%%%%%%%%%%%%%%%%%%%%%%%%%%%%%%%%%%%%%



I arrived in Beijing late on a Friday evening at the end of August in 2015.  My close friend Kai---a former Chinese National rugby team representative, and now a lawyer working in Beijing---picked me up at the airport and drove me back to his home.  The plan was to stay with Kai until I was able to make solid arrangements with the Beijing Temple of Agriculture Sports Institute, the home of the Beijing Provincial Rugby Team.

After a day of acclimatising and running errands, on Saturday evening I was invited to a dinner hosted by Adrian, a respected elder within the Chinese rugby community of Beijing.  Adrian was the captain of the second ever class of rugby players to graduate from the Chinese Agricultural University in Beijing---the birthplace of rugby in China.
I first met Adrian through Kai in 2013,

 while coaching in China.  I was originally introduced to Adrian by Kai, a close friend of mine who I met earlier during another stint in Beijing in 2008.

 Kai was also at the dinner, as was Mr Shi, a sports television producer at Chinese Central Television (CCTV).  CCTV had just accepted the rights to the Rugby World Cup, which World Rugby---the international governing body of rugby---had made available to CCTV in an attempt to promote the development of the game in non-traditional rugby playing nations.


I arrived in Beijing late on a Friday evening at the end of August in 2015.  My close friend Kai---a former Chinese rugby player, graduate the Chinese Agricultural University (Chinese rugby's birthplace), and now a lawyer working in Beijing---met me at the airport and drove me back to his home.  When we got to his home, Kai turned on the television and we caught up while some footage from a rugby documentary played in the background.  As it so happened, the international rugby world was on the verge of another Rugby World Cup, which was being hosted by England in the coming months. World Rugby, the world governing body of rugby union, had made the television broadcast rights for the World Cup available to Chinese Central Television (CCTV), in an attempt to promote the game globally.  Having accepted the rights to the tournament, which is the 3rd largest sporting event in the world behind the Olympics and the Football World Cup, CCTV were in search of Chinese rugby experts to help produce the 48-match tournament.  As Kai quickly explained, CCTV's search had led them to the Chinese rugby community based in Beijing, who were almost all, like Kai, graduates of the Chinese Agricultural University---the birthplace of rugby in China and the base for the Chinese National Rugby team between 1996 and 2010.  In fact, CCTV's search led them first to Adrian, the captain of one of the first CAU rugby teams (1992), and currently working for a large international sport organisation in Beijing.  Adrian had then contacted his younger university brother (师弟) Kai, who like him was fluent in English and able to assist in sourcing and translating rugby materials relevant to the broadcast. The CCTV producer responsible for the broadcast, Mr Shi, had scheduled a dinner with Adrian and Kai on Saturday (tomorrow) night to thank them for their willingness to assist in the production.  I was also invited to the dinner. I wasn't scheduled to meet with the Principle and head coach of the Beijing Temple of Agriculture Sports Institute until the following Monday, so I agreed to accompany Kai.







First week in China (introduction vinnete):
- ``rou'': mysterious carnal sport --> social community
- XNT fall from grace:
- Qingdao:
    - basketball (actionScrutiny)
    - Asia 7s: (Qi gaige & Li sheng) (system, incentives)
    - National 7s: Beijing:WCY out of the huddle


XNT:
- Sport as a upward mobility strategy:
    - Many from athletics (IND) to rugby (TEAM)
- Fall from grace --> Lack of click?
- System: coach ()
- Zero to one: new comers, uni students, senior players, old motherlandStrength

In this system, how do they interaction

%%%%%%%%%%%%%%%%%%%%%%%%%%%%%%%%%%%%%%%%%%%%%%%%%%%%%%%%%%%%%%%%%%%%%%%%%%%%%%%%%%%%%%
ABSTRACT:
Fieldwork in Beijing



The Beijing provincial men's and women's rugby programs are based at the Xiannongtan Sports Institute in Beijing (one of Beijing's four major sports institutes, and home to seven different full-time sports programs, hereafter Xiannongtan).  These athletes represent Beijing at a provincial level, playing against other provinces in annual tournaments, and every four years at the all-important National Games.  Five athletes have previously represented China in international rugby sevens tournaments.  There is also a Beijing women's rugby team at Xiannongtan, but I was not able to accompany the activities of both teams closely enough to perform adequate ethnographic research.
