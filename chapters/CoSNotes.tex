Frith  & Frith (2010: 171) outline a clear model for how ostensive signalling can lead to cooperation through a process of ``closing the loop''.  ``Common knowledge ''is important:
``You need to know that the alien knows that you are signalling. Further, you want the alien to believe that you know that it trusts you.''

This is all well and good, and no doubt central to processes of cooperation.  The thrust of my thesis, however, is that this same process of ``closing the loop'' can also occur via implicit signaling, or perhaps more accurately: the process of closing the loop implicates many implicit lower cognitive processes involved in movement regulation.


Learning by prediction (Frith 2007: 98): dopamine and endorphins yolked together
dopamine is the error signal trying to lead the brain to the endorphin sweet spot


Something about the engineer that is more like the shaman than the scientist.  insights are pre-empirical, there is an intuition and a knowing already there as they explore the world and notice the imperfections and intricacies... a knowing that isn't learnt in PhDs or
