\chapter*{Chapter 6: Abstract}
%\addcontentsline{toc}{chapter}{\nameref{ch:study1intro}}


%\markboth{{Introduction to Part I}}{Introduction to Part I\label{ch:part1intro}



%Important point that the extension of empirical affordances hasn't been shown before... need to show empirical evidence to substantiate the claim that cultural milieu can function as affordances.

In this chapter I present results of ethnographic data collected with the Beijing Men's Rugby team between September 2015 and August 2016.  I find a range of evidence in support of the predictions set out initially in Chapter ~\ref{chap:theory} and refined in Chapter ~\ref{chap:researchSetting} in light of a review of the contextual specificities of the research setting.  First, I describe the culturally specific terrain of social cognition, which is defined by a dominance of a relational mode of social cognition in processes of contemporary Chinese sociality.  The dominance of this culturally specific mode of social cognition is identifiable at multiple levels of social life, from the level of the social institutions in which athletes, coaches, and officials interact, to the level of group norms in which athletes and coaches participate, to the level of on-field processes of joint action and perception.  I present and analyse ethnographic evidence for each of these levels.
