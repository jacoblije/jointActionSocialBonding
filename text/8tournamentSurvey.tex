mixed effectsacross
\begin{savequote}[8cm]

  \qauthor{}
\end{savequote}


\chapter{\label{chap:tournamentSurvey}Team click in a National Tournament}
                                            \begin{CJK}{UTF8}{gbsn}

\minitoc

\section{Introduction\label{sect:introSurvey}}
Team click is a visceral and socially agentic experience of optimal team performance in joint action.
The phenomenon of team click in group exercise contexts is anecdotally pervasive, but theoretical justifications and empirical substantiation for the mechanisms that generate team click are lacking.  Thus, in this study, I tested a general account of team click by surveying a population of professional athletes before, during, and after a high-stakes, high-intensity rugby Tournament.   It was hypothesised that, when levels of uncertainty were high---levels of uncertainty in joint action in the Tournament were presumed to be high---team click would mediate a relationship between more positive perceptions of team performance and higher levels of social bonding (H1).  In addition, it was hypothesised that positive violation of expectation surrounding team performance will lead to higher levels of team click, and, in turn, social bonding (H2b).

Empirical and theoretical evidence outlined in Chapter~\ref{chap:theory} supported the core claim that team click will explain a relationship between positive perceptions of team performance and social bonding.  Theory suggests that perceptions of team click may arise in situations in which the high levels of uncertainty in joint action are met with high levels of dynamical coordination---of both physical movement and social communication between co-actors \citep{Semin2008}.  In previous \mccorrect{Chapters}, ethnographic evidence suggested that the experience of uncertainty of joint action, and on-and off-field forms of coordination were present and salient \mccorrect{dimensions of athletes daily experience at the Institute} (see Chapter~\ref{chap:ethnoField}).  In addition, I confirmed that athletes widely recognised and experienced team click in rugby’s joint action, albeit infrequently in the case of the Beijing rugby team (see Chapter~\ref{chap:ethnoResults}).  In particular, positive expectation violation appeared to precede team click,  and team click appeared to prompt automatic appeal to social rationales for team performance.  Taken together, this evidence supports the possibility that positive expectation violation in joint action serves to draw attention to visceral and socially agentic perceptions of team click, for which social rationalisations characteristic of social bonding function to explain (H2).   These findings validated research hypotheses and motivated further empirical research.

\myparagraph{The Tournament}
``The 2016 National Rugby Sevens Championship'' (2016全国7人制橄榄球冠军赛; hereafter ``The Tournament'')was a rugby sevens tournament held 16-17th July 2016 in Qianan, Hebei Province, China.  Athletes competed for their provincial teams at high intensity and for high stakes (to be crowned overall National Champion) in a series of 14-minute rugby sevens games spread over two days.  The athletes surveyed in the Tournament represented the top level of professional rugby players in China at the time.  In this study, it was assumed that uncertainty in group exercise would be constantly high, and that varying levels of physical and social coordination would be present in response to this high uncertainty (see Chapter~\ref{chap:theory}).

\myparagraph{Survey measures}
The key predictor variable of this study was athlete perceptions of \textit{team} performance.  Ethnographic evidence suggested that athletes would reflect on and scrutinise various technical components of joint action, which could be delineated as ``more-joint'' action and therefore more uncertain (team performance), and ``less-joint'' action and therefore less uncertain (individual performance); see Chapter~\ref{sect:uncertaintyJA}.  Thus, before and after the Tournament, athletes were asked to report their perceptions concerning performance of technical components of team (coordination in defence, attack, support play, on-field communication) and individual (passing technique, tackling, decision making, effectiveness in contact) performance.  It was expected that athletes would reflect on team and individual performance as separate constructs.  Additionally, measures of individual performance would be used as statistical controls in analyses of study predictions, to further distinguish perceptions of team-level contributions to joint action from individual-level contributions. To investigate the predicted role of positive expectation violations of team performance as a preceding mechanism to team click and social bonding, athletes were also asked to report their perceptions of overall team and individual performance relative to prior expectations.

Team click was measured using a number of novel items designed capture the phenomenon's visceral and socially agentic dimensions (as outlined in Chapter~\ref{sect:visceralAgency}).  Viscerality was measured by questions about 1) \textit{moqi} (a Chinese term specifically denoting tacit understanding, but also used to describe more general ``click'' in joint action), 2) team aura (\textit{qichang} 气场;distinguished from atmosphere (\textit{qifen} 气氛) to denote atmosphere specific to the team), and 3) a novel visual item designed to reflect perceptions of coherence in joint action.  Social agency was measured by asking athletes to report perceptions of 1) their own abilities being extended by the abilities of their teammates, and 2) the perceived reliability of others (and self) to perform prescribed roles in joint action (two separate items).

Social bonding is known to entail a continuum of affective and more explicit or declarative processes, ranging from emotional connection to specific others, to more general \mccorrect{attachment, to more culturally mediated affordances such as group identity} \citep{Dunbar2010,Whitehouse2014}.  Accordingly, athletes were asked questions designed to  access both affective and explicit ends of this continuum.  Two novel measures were used to assess: 1) athlete perceptions of emotional support from teammates and 2) common goal with teammates (both more tacit, affective measures of social bonding).  Athlete identification with the team was measured using a scale validated to measure identification of self with the prototypical features of the group \citep[Group Identification, see][]{Turner1987}, as well as a scale validated to measure an individual's ``visceral sense of oneness'' or fusion with the group \citep[Identity Fusion; see][]{Swann2009}).

Together, these survey items were designed to operationalise hypothesised relationships between perceptions of team performance, team click, and social bonding.  In general, athletes were asked to reflect on their ``sense perceptions'' or ``feelings'' (\textit{ganjue} 感觉) associated with aspects of performance, team click, and social bonding.  Feeling was used instead of ``perception'' (\textit{zhijie} 直觉), to encourage athletes to reflect on their own personal sensorimotor impressions of participation in joint action, and to reduce the extent to which athletes would appeal to more explicit or de-personalised explanations for team or individual performance (for example, sourced from feedback provided by their coach, or by objective measures of performance such as games won or lost).

Acknowledging the possibility that individual differences in technical competence and personality type may modulate the processes outlined above, this study also measured technical competence and personality type of athletes.  Based on ethnographic evidence, it was expected that athletes with higher levels of technical competence would be more able to mitigate the uncertainty associated with rugby’s joint action, and would therefore perceive higher levels of team click.  Meanwhile, it was expected that higher levels of extraversion may serve as a proxy for a propensity shown by more extraverted individuals to ascribe agency in joint action to referents external to the self \citep[]{Keller2014}.  Accordingly, measures of technical competence---both objective measures (e.g., training age and years spent in the team) and subjective measures (e.g., competence relative to other athletes)—--and personality type were also recorded, based on the possibility that these variables may also bear upon perceptions of joint action in group exercise.   However, owing the many unresolved questions in these literatures, these measures were examined as open questions, and no strict predictions were formulated.

In addition to these main variables of interest, perceptions of fatigue and injury were also recorded.  Finally, data concerning athletes' objective performance in the tournament was also collected to enrich analysis and provide statistical control for variation in factors such as overall team performance in the Tournament, as well as an individual's overall time spent on the field and markers of performance during the Tournament (i.e., total minutes played; total points scored).

Three subsets of the collected data (Post-Tournament, Pre- to Post-Tournament, and Overall Tournament) were analysed for relationships between variables of interest within individuals and over time.  Clustering of responses according to groupings such as sex, team, an repeated measures of individuals was assessed (and controlled for statistically where necessary).

In sum, this study aimed to test the core hypothesis of this thesis \mccorrect{(H1)}, as well as aspects of the secondary hypothesis \mccorrect{(H2)}.  Specific predictions are outlined in Table~\ref{tab:tournamentPredictions}.

\input{images/tournamentSurveyPredictions}


%Based on these considerations, the following predictions were formulated and tested:
    %  \begin{description}

    %  \end{description}
        %\end{description}\item[Prediction 1.1:] More positive perceptions of team performance will predict higher levels of team click
        %\item[Prediction 1.2:] Higher levels of team click will predict higher levels of social bonding
        %\item[Prediction 1.3:] More positive perceptions of team performance will predict higher levels of social bonding
        %\item[Prediction 1.4:] Team click will mediate a %relationship between more positive perceptions of team %performance and social bonding
        %\item[Prediction 2.1:] More positive perceptions of %team performance relative to prior expectations will %predict higher levels of team click
        %\item[Prediction 2.2] More positive perceptions of team %performance relative to prior expectations will predict %higher levels of social bonding
%      \end{description}




\clearpage

\section{Method}
\subsection{Participants}
174 Chinese professional adult rugby players from 8 men's provincial teams and 7 women's provincial teams ($M(age) = 21.67$ ($SD = 3.67$, $range = 16 - 32$, $male = 93$) were surveyed once before (2-4 days before), twice during (once each day of the two day tournament, following the 2nd or 3rd game of each day), and once after the Tournament.  The University of Oxford's Central University Research Ethics Committee approved this study (SAME/CUREC1A/15-059).

\subsection{Materials}

\subsubsection{The Tournament}
In this study, 15 teams in total participated in the Tournament: 7 women's teams and 8 men's teams. The men's competition was split into two pools of 4 teams each, and the women's competition was split into one pool of 4 and one pool of 3 teams (see Table~\ref{tab:poolStructureTable}).  A rugby sevens tournament is generally made up of 16 teams and requires two full days to complete.  Within each tournament, participating teams are first divided into smaller pools (four teams per pool), and spend Day 1 of the tournament playing 3 14-minute games against each team in their pool. On Day 2 of the tournament, teams are re-grouped according to results from Day 1.  The top 8 teams of the tournament compete for overall championship in a knock-out (winning teams advance, losing teams are knocked-out of further contention for overall winner, but may still contest lower placings) stage of the tournament, while any remaining teams outside the top 8 compete for the remaining placings below these 8 teams \citep[][]{WorldRugby2018}. At the time that this study was conducted, the playing time for the grand-final was extended to 20 minutes (10 minutes per half, as opposed to the usual 7 minutes).



\myparagraph{Women's tournament}
Teams in Women's Pool A played only two games each on the first day, while teams in Women's Pool B played 3 games each. On the second day, the top two teams from each women's pool competed in the women's knock-out (which consisted of only a semi-final and grand-final).  The remaining 3 teams who did not qualify for the knock-out phase played each other once to determine the remaining places (5th-7th place).  Therefore, athletes in the women's teams played 4 or 5 games spread over two days.

\myparagraph{Men's tournament}
Athletes in the men's team each played 3 games on day one, before playing in a knock-out phase on day two, which involved 8 teams in a quarter-semi-grand final structure. The winning teams from the quarter-finals played in a semi-final, and then either a grand final (for the two undefeated teams) or a 3rd-4th place playoff final (for the two losing teams from the semi-finals). The 4 teams who lost in the quarter-finals played in a subsidiary semi-final and final to decide the 5th-6th place and 7th-8th place results. Thus, athletes in the men's teams played a total of 6 games over two days.


% Please add the following required packages to your document preamble:
% \usepackage{booktabs}
% Please add the following required packages to your document preamble:
% \usepackage{booktabs}
\begin{table}[]
  \centering
\begin{tabular}{@{}lcclcc@{}}
\toprule
  & \textbf{Women's Pool A} & \textbf{Women's Pool B} &  & \textbf{Men's Pool A} & \textbf{Men's Pool B} \\ \midrule
1 & Jiangsu                 & Anhui                   &  & Shandong              & Tianjin               \\
2 & Shandong                & Shanghai                &  & Beijing               & PLA*                  \\
3 & Tianjin                 & Beijing                 &  & Hebei                 & Anhui                 \\
4 &                         & Fujian                  &  & Shanghai              & Fujian                \\ \bottomrule
\end{tabular}

\caption{Tournament pool structure for women's (left) and men's (right) tournaments. *PLA: People's Liberation Army}
\label{tab:poolStructureTable}
\end{table}


\subsection{Surveys}
Surveys were designed primarily to measure athletes' perceptions of performance, team click, and social bonding in joint action.  Surveys were generated using Qualtrics software (Qualtrics version 9, Provo, UT). Surveys were translated into Chinese and then back-translated by two independent native Chinese speaking translators from Beijing Sports University.

Surveys were administered at 4 different time points: once before, twice during, and once after the Tournament.  Pre- and Post-Tournament surveys were administered online using a social networking software called WeChat. Surveys administered before and after the Tournament were completed by athletes within the WeChat application, using their personal mobile phone devices and Internet access.  Due to the constraints of the Tournament setting, in particular athletes' lack of access to mobile phones and Internet immediately following games, hard copy (paper) surveys rather than electronic surveys were administered Mid-Tournament.


%% Summary table of key variables, ned to adapt from Ch6
%<<surveyMeasureSummaryTable, eval=T, echo=F>>=
%  #create all columns:
%  Items <- c("Performance(Ind)", "Performance(Group)", "Performance(Team)",
%             "TeamClick(Group)", "TeamClick(Team)", "SocialBonding(Group)", %"SocialBonding(Team)", "Arousal", "Exertion")
%  Baseline <- c(" ",""," ",""," ",""," ","","")
%  Pre <- c(" "," ",""," ",""," ",""," ","")
%  Post <- c(" "," ",""," "," "," "," "," "," ")

%  surveyMeasureSummary <- data.frame(Items, Baseline, Pre, Post, stringsAsFactors = FALSE)

%  summaryTable <- xtable(surveyMeasureSummary,
%                                caption = "Survey items measured at each time point"
%                                label = "tab:surveyItemsByTime")
%  align(summaryTable) <- "llccc"
%  print(summaryTable, file="surveyItemsByTime.tex")
%@

\subsubsection{Tournament Survey Items\label{survey:Tournament Survey Items}}

Tournament survey items were designed and selected to collect data on 4 main areas of athlete experience.  Questions included items on subjective experience of team and individual performance, feelings of team click, and feelings of social bonding to their team.  In addition, athletes answered questions about exertion, fatigue, injury status, as well as objective and subjective measures of technical competence, perceptions of team discipline, and individual personality type (see Table~\ref{tab:tournamentSurveyItemsTime}). The surveys (initially designed in English) were translated into modern Chinese by the researcher and back translated by two independent native Chinese speaking translators from Beijing Sports University to verify accuracy.


% Please add the following required packages to your document preamble:
% \usepackage{booktabs}
\begin{table}[]
  \centering
  \begin{tabular}{@{}rcccc@{}}
  \toprule
  \multicolumn{1}{c}{\textbf{Item}}                   & \textbf{Pre}          & \textbf{Day 1}        & \textbf{Day 2}        & \textbf{Post}         \\ \midrule
  \multicolumn{1}{c}{\textbf{Team Performance}}       &                       &                       &                       &                       \\ \midrule
  Components of                                       & \cmark &                       &                       & \cmark \\
  Vs expectations                                     &                       & \cmark & \cmark & \cmark \\ \midrule
  \multicolumn{1}{c}{\textbf{Individual Performance}} &                       &                       &                       &                       \\ \midrule
  Components of                                       & \cmark &                       &                       & \cmark \\
  Vs expectations                                     &                       & \cmark & \cmark & \cmark \\ \midrule
  \multicolumn{1}{c}{\textbf{Team Click}}             &                       &                       &                       &                       \\ \midrule
  6 item survey                                       &                       &                       &                       & \cmark \\
  3 item survey                                       & \cmark & \cmark & \cmark & \cmark \\ \midrule
  \multicolumn{1}{c}{\textbf{Social Bonding}}         &                       &                       &                       &                       \\ \midrule
  6 item survey                                       & \cmark &                       &                       & \cmark \\
  3 item survey                                       & \cmark & \cmark & \cmark & \cmark \\ \midrule
  \multicolumn{1}{c}{\textbf{Moderators}}             &                       &                       &                       &                       \\ \midrule
  Fatigue                                             &                       & \cmark & \cmark & \cmark \\
  Injury                                              & \cmark & \cmark & \cmark & \cmark \\
  Objective competence                                & \cmark &                       &                       &                       \\
  Subjective competence                               & \cmark &                       &                       &                       \\
  Personality                                         & \cmark &                       &                       &                       \\
  ID Variables                                        & \cmark &                       &                       &
  \end{tabular}
\caption{Main survey items administered over the course of the Tournament.  ID Variables included team, age, training age, etc.}
\label{tab:tournamentSurveyItemsTime}
\end{table}


%This study was specifically designed to test the relationship between perceptions of team performance and team click and social bonding.
 %distinguish between perceptions of team and individual performance,


\myparagraph{Team and individual performance}
To isolate athlete perceptions of team performance from perceptions of individual performance, athletes were asked to report perceptions of team and individual performance separately.  Athletes reported their perceptions of two dimensions of performance: 1) components of performance (team and individual), and 2) overall quality of performance relative to prior expectations (team and individual). As in Chapter~\ref{chap:tournamentSurvey}, components of team performance and components of individual performance were selected according to aspects of team performance commonly scrutinised in rugby (see Table~\ref{tab:compPerform}).

%and Chapter~\ref{chap:ethnoResults} Section~\ref{sect:ethnoPerformComponents}

% Please add the following required packages to your document preamble:
% \usepackage{booktabs}
\begin{table}[]
  \centering
  
  
\begin{tabular}{@{}lll@{}}
\toprule
\textbf{Performance} & \textbf{\begin{tabular}[c]{@{}l@{}}Technical \\ Component\end{tabular}} & \multicolumn{1}{c}{\textbf{Item}} \\ \midrule
\multicolumn{1}{c}{\textbf{Team}} & Defence & \begin{tabular}[c]{@{}l@{}}How do you feel about your team's \\ coordination of the defensive line?\end{tabular} \\
 &  &  \\
 & Attack & \begin{tabular}[c]{@{}l@{}}How do you feel about your team's \\ coordination of the attacking line?\end{tabular} \\
 &  &  \\
 & Support play & \begin{tabular}[c]{@{}l@{}}How do you feel about your team's \\ support play?\end{tabular} \\
 &  &  \\
 & \begin{tabular}[c]{@{}l@{}}On-field \\ communication\end{tabular} & \begin{tabular}[c]{@{}l@{}}How do you feel about your team's \\ on-field communication?\end{tabular} \\
 &  &  \\
\multicolumn{1}{c}{\textbf{Individual}} & Passing technique & \begin{tabular}[c]{@{}l@{}}How do you feel about your passing \\ technique?\end{tabular} \\
 &  &  \\
 & \begin{tabular}[c]{@{}l@{}}Support play \\ in attack\end{tabular} & \begin{tabular}[c]{@{}l@{}}How do you feel about your support \\ play in attack?\end{tabular} \\
 &  &  \\
 & \begin{tabular}[c]{@{}l@{}}One-on-one \\ defence\end{tabular} & \begin{tabular}[c]{@{}l@{}}How do you feel about your 1on1 \\ defence?\end{tabular} \\
 &  &  \\
 & \begin{tabular}[c]{@{}l@{}}Effectiveness \\ in contact\end{tabular} & \begin{tabular}[c]{@{}l@{}}How do you feel about your \\ effectiveness in contact over the past \\ month?\end{tabular} \\
 &  &  \\
 & \begin{tabular}[c]{@{}l@{}}Decision making \\ in game play\end{tabular} & \begin{tabular}[c]{@{}l@{}}How do you feel about your decision \\ making in game-play?\end{tabular} \\
 &  &  \\ \bottomrule
\end{tabular}

\caption{Components of team and individual performance administered Pre- and Post-Tournament}
\label{tab:compPerform}
\end{table}

%/Users/jacob1/Documents/2017/Research/DPhil/Dissertation/finalDocuments/jointActionSocialBonding/images/componentsOfPerformance.tex

In the Pre-Tournament survey, athletes were asked about their impression of team or individual performance in the past month, for example: ``how do you feel about your team's coordination of the defensive line over the past month?''  In the Post-Tournament survey athletes were asked about their impressions of the same components of performance as they were perceived during the Tournament, e.g. ``How do you feel about your passing technique during the Tournament?'' Athletes responded to each item by moving a toggle left or right from its default centre position on a continuous 100-point scale: 0 - ``Extremely poor'', 100 - ``Extremely good.''  Given the time constraints associated with delivering surveys during the Tournament itself (immediately following individual games), athlete perceptions of components of team and individual performance were only included in the Pre- and Post-Tournament surveys.

Items measuring athlete perceptions of overall quality of \textit{team and individual performance relative to prior expectations} were included each Mid-Tournament survey, and in the Post-Tournament survey.  In the case of the Mid-Tournament surveys, athletes were asked: ``Overall, how do you feel about your individual performance/the performance of the team in this game?'' (100 point continuous scale centred at zero: -50 --- ``Much worse than expected'', 0 --- ``As expected'', 50 ---  ``Much better than expected''). In the Post-Tournament survey, athletes answered the same questions about individual and team performance, but in relation to the Tournament as a whole.

In the Pre-Tournament survey, athletes answered questions relating to overall team and individual performance in relation to the month of training and competition prior to date of the survey (``Overall, how well do you feel you/your team has been performing in training and competition over the past month?'').
Without a specific or immediate focus for perceptions of overall performance, it was unnatural to frame these items in terms of prior expectations.  Instead, overall performance was rated on a continuous scale (0 = ``Extremely poor,'' 100 = ``Extremely well'').  While these items did not provide a measure of performance in relation to prior expectations \textit{per se}, they provided a baseline control measure of attitudes towards performance.\footnote{Unless otherwise stated, all Pre-Tournament survey items used ``in the past month'' as a time referent; Mid-Tournament survey items used ``in this game,'' and the Post-Tournament survey items used ``in the Tournament'' as time referents.}



%In addition, in the Pre-Tournament survey Athletes were also asked to report the extent to which 1) the quality of recent individual performances influences their mood and 2) the extent to which recent performance influences their confidence for future performance (see Appendix~\ref{app8:tournamentSurvey} Section~\ref{app8:performancePre} for a full description of these additional performance measures).

%(see Chapter~\ref{chap:ethnoResults} Section~\ref{sect:teamClick} for a full explanation)
\myparagraph{Team Click\label{sect:teamClickSurvey}}
Athletes responded to a number of questions regarding their experience of team click. Survey items pertaining to team click were generated by utilising concepts commonly encountered by the researcher during ethnographic observations of the Beijing provincial team and other Chinese provincial teams. Items measuring team click are displayed in Table~\ref{tab:teamClickSurveyItems}.  All items were novel measures. Athletes responded to each item by moving a toggle left or right from its default centre position on a continuous 100-point scale (0 --- ``Extremely weak,'' 100 --- ``Extremely strong'').

Tacit Understanding is an English translation of a Chinese term \textit{moqi}, which is often used in team sport contexts to express the idea of click in joint action (see Chapter ~\ref{sect:researchSettingMethod} and~\ref{sect:teamClickExperience}).  Team Aura \textit{qichang} (气场), is a term taken from Chinese \textit{qigong} (气功), literally meaning ``field of energy.'' \textit{Qichang} is commonly used to describe the atmosphere generated when the team is performing well.  The pictorial measure of team click was a visual item with 5 responses, ranging from less to more coordinated arrangements of dots, designed to represent the coordination of the team on the field (Click Pictorial, see Figure~\ref{fig:clickPictorial}).
For a detailed explanation of all Pre-Tournament survey items relevant to team click see Appendix~\ref{app8:clickPre}.

Only 3 items pertaining to team click were included in the Mid-Tournament survey due to time constraints (Tacit Understanding, Team Aura, and the Click Pictorial measure).

% Please add the following required packages to your document preamble:
% \usepackage{booktabs}
\begin{table}[]
  \centering
  
  
\begin{tabular}{@{}ll@{}}
\toprule
\textbf{Item} & \textbf{Description} \\ \midrule
 &  \\
\textbf{Viscerality} &  \\
 &  \\
Tacit Understanding & \begin{tabular}[c]{@{}l@{}}How strong has the tacit understanding \\ been between team members?\end{tabular} \\
 &  \\
Team Aura & How is the aura in/around the team? \\
 &  \\
Click Pictorial & \textit{Visual Item} \\
 &  \\
  &  \\
\textbf{Social Agency} &  \\
 &  \\
Reliability of Others & \begin{tabular}[c]{@{}l@{}}To what extent have you felt that you can \\ rely on others to perform their roles on the \\ field  (for example, in key moments of \\ competition or training)?\end{tabular} \\
 &  \\
Reliability for Others & \begin{tabular}[c]{@{}l@{}}During the past month, to what extent \\ have you felt that others can rely on you \\ to perform your role on the field (for example, \\ in key moments of competition or training)?\end{tabular} \\
 &  \\
Ability Extended & \begin{tabular}[c]{@{}l@{}}When coordinating with others on the field in \\ the past month, do you feel that your individual \\ ability is extended by the ability of your team \\ mates?\end{tabular} \\
 &  \\ \bottomrule
\end{tabular}
  
\caption{Survey items designed to measure the experience of team click in joint action.}
\label{tab:teamClickSurveyItems}
\end{table}



  \begin{figure}[htbp]
    \includegraphics[width = \linewidth]{images/teamClickPictorial.png}
    \caption{Click Pictorial Scale}
    \label{fig:clickPictorial}
  \end{figure}


  \myparagraph{Social Bonding\label{sect:socialBondingSurvey}}
Athletes answered items relating to feelings of social bonding to their team and teammates.  Survey items related to social bonding are described in Table~\ref{tab:socialBondingSurveyItems}. Two novel measures of social bonding were constructed to reflect ethnographic results: Emotional Support and Common Goal.  Both measures used 100-point continuous scales, (0 - ``Extremely weak'', 100 - ``Extremely strong'').

Group Identification and Identity Fusion items were validated multi-item constructs (5 and 7 items respectively) measured using 5-point Likert scales. The Fusion Pictorial scale is shown in Figure~\ref{fig:fusionPictorialGroup}.  See Appendix~\ref{app8:bondingPre} for a detailed explanation of all items relevant to social bonding.  Given time constraints, only 3 items pertaining to social bonding were included in the Mid-Tournament survey (Emotional Support, Common Goal, and Identity Fusion Pictorial (Team)).

% Please add the following required packages to your document preamble:
% \usepackage{booktabs}
\begin{table}[]
  \centering
  
  
  
  
\begin{tabular}{@{}ll@{}}
\toprule
\multicolumn{1}{c}{\textbf{Item}} & \multicolumn{1}{c}{\textbf{Description}} \\ \midrule
Emotional Support & How emotionally supportive does the team feel? \\
 &  \\
Common Goal & \begin{tabular}[c]{@{}l@{}}How strong is the feeling that everyone is \\ working towards a shared goal?\end{tabular} \\
 &  \\
\begin{tabular}[c]{@{}l@{}}Group Identification\\ Verbal Scale\end{tabular} & \textit{\begin{tabular}[c]{@{}l@{}}An individual's personal identification with the \\ stereotypical features of the in-group (see Mael 1992)\end{tabular}} \\
 &  \\
\begin{tabular}[c]{@{}l@{}}Identity Fusion\\ Verbal Scale\end{tabular} & \textit{\begin{tabular}[c]{@{}l@{}}A five item verbal scale designed to measure \\ construct of Identity Fusion (see Swann et al. 2009)\end{tabular}} \\
 &  \\
\begin{tabular}[c]{@{}l@{}}Identity Fusion\\ Pictorial Scale\end{tabular} & \textit{\begin{tabular}[c]{@{}l@{}}A single item visual scale designed to measure \\ construct of Identity Fusion to three target \\ in-groups: an athletes' 1) team, 2) family, \\ and 3) nation (see Swann et al. 2009)\end{tabular}} \\
 &  \\
Pictorial Rank & \textit{\begin{tabular}[c]{@{}l@{}}The rank order in which athletes experience \\ fusion to team, family, and nation  \\ (see Whitehouse et al. 2014)\end{tabular}} \\
 &  \\ \bottomrule
\end{tabular}
  
  
  
  
\caption{Survey items designed to measure athlete experience of social bonding}
\label{tab:socialBondingSurveyItems}
\end{table}



\begin{figure}[htbp]
  \includegraphics[width=\linewidth]{images/Identity_Fusion_Pictorial_Scale.png}
  \caption{Identity Fusion Pictorial Scale}
  \label{fig:fusionPictorialGroup}
\end{figure}




\myparagraph{Moderator Variables}
\textit{Fatigue, Exertion, and Injury Status:} Following each Mid-Tournament survey and in the Post-Tournament survey, athletes were asked about their mood (including dimensions of arousal, excitement, and nervousness), feelings of fatigue (``How fatigued do you feel as a result of the game/tournament?''), perceived physical exertion \citep[Borg RPE scale;][]{Borg1990} and perceived mental exertion \citep[see][]{Noakes2012a}. Athletes were also asked about their injury status in each Mid-Tournament survey and in the Post-Tournament survey (see Appendix~\ref{app8:injuryStatus}).

\textit{Technical Competence:} Athletes were asked to report their perceptions of individual technical competence relative to 1) other teammates, 2) other current professional Chinese rugby players, and  3) professional rugby players from other countries.  Athletes also provided information on rugby-related attributes, which also provided a more objective indicator of technical competence.  These measures included: 1) rugby training age (number of years of experience training for rugby, to the nearest number of years), 2) the number of years spent training with the provincial teams (to the nearest year) and 3) whether the athlete is a usual member of the provincial program's starting team or the reserves.  See Appendix~\ref{app8:technicalCompetence} for a detailed explanation of all items relating to technical competence.

\textit{Personality:} A 10-item personality measure was included in the Pre-Tournament survey to measure athlete personality type according to the ``Big 5'' personality types \citep[Ten Item Personality Index - TIPI.  The 5 types are: Extraversion, Agreeableness, Conscientiousness, Emotional Stability (Neuroticism reverse coded), and Openness to Experiences; see][]{Gosling2003}. See Appendix~\ref{app8:TIPI} for a detailed explanation.

\myparagraph{Identification Variables}
In the Pre-Tournament survey athletes were asked about basic identification variables (athlete name, date of birth, team membership, etc).  For a detailed description of additional items included in the Tournament surveys, see Appendix~\ref{app8:surveyItems}.



\subsubsection{Measures of objective performance\label{app8:objectivePerformance}}
Following the completion of the Tournament, tournament officials from the Chinese Rugby Football Association (CRFA) provided performance data in electronic format. These data included results for each game, minutes played and points scored by individual athletes in each game, substitutions made during each game, and video footage of every game played during the Tournament.  Based on the data provided, a number of objective performance variables were created for use as statistical controls (described in Table~\ref{tab:tournamentPerformance}).

%including the final rank of each team in their respective competition (men's and women's, reverse coded so that the top ranked team was awarded the highest value), a team's total number of wins minus total number of losses, each individual athlete's total number of minutes played throughout the course of the Tournament, each individual athlete's total number of points scored in the Tournament, the average number of times an individual athlete was part of the starting team throughout the Tournament.


% Please add the following required packages to your document preamble:
% \usepackage{booktabs}
\begin{table}[]
\centering


\begin{tabular}{@{}ll@{}}
\toprule
\multicolumn{1}{c}{\textbf{Item}} & \multicolumn{1}{c}{\textbf{Description}} \\ \midrule
Final rank & \begin{tabular}[c]{@{}l@{}}A rank was given to each team based on \\ performance in their respective competition \\ (men's and women's)\end{tabular} \\
 &  \\
Wins - losses & \begin{tabular}[c]{@{}l@{}}A team's total number of losses was subtracted \\ from its total number of wins\end{tabular} \\
 &  \\
Total minutes & \begin{tabular}[c]{@{}l@{}}Total number of minutes played throughout \\ the Tournament by each individual athlete\end{tabular} \\
 &  \\
Total points & \begin{tabular}[c]{@{}l@{}}Total number of points scored throughout \\ the Tournament by each individual athlete\end{tabular} \\
 &  \\
Total points & \begin{tabular}[c]{@{}l@{}}An average measure indicating likelihood \\ of an athlete being selected in the starting team.\end{tabular} \\ \bottomrule
\end{tabular}


\caption{Measures of objective performance in the Tournament}
\label{tab:tournamentPerformance}
\end{table}









\subsection{Procedure}

\subsubsection{Pre-Tournament survey}
Two months prior to the Tournament, I contacted the head coach of each provincial team and officials from CRFA to seek permission for the study.  After receiving permission from all participating teams and CRFA, 5 days prior to the Tournament I asked the coach or manager of each team to create a virtual message group on WeChat, the members of which included the coach/manager of the team, the athletes competing in the Tournament, and the researcher.  The WeChat group could be accessed by the athletes on their personal mobile phone devices with Internet connection.  Once the WeChat group was set up, I posted a standard message in each group in which I introduced the study and provided the link to the Qualtrics survey for the athletes to complete in their own time (for English and Chinese versions of the script, see Appendix~\ref{app8:tournamentSurvey} Section~\ref{app8:studyIntro}).

Upon opening the link to the survey, athletes read a detailed brief about the survey, provided consent, and demonstrated their ability to answer the survey questions by changing the position of a virtual sliding bar toggle that would feature in many of the survey questions.  Athletes were then asked a number of questions grouped by the following categories: perceptions of team and individual performance, team click, social bonding, technical competence, and personality type. The order in which each item appeared within these categories was randomised for each survey participant. At the end of the survey, athletes were asked to provide basic identification variables such as age, sex, team, position, and injury status.  The Pre-Tournament survey took approximately 15 minutes to complete.  Once the data collection window for the Pre-Tournament survey had ceased, survey responses were collated in Qualtrics.

\subsubsection{Mid-Tournament surveys}
Copies of the Mid-Tournament survey were printed on A4 paper for athletes to complete with a pen or pencil within 30 minutes of the end of the second or third game of each day.  All surveys were administered and collected by the researcher, occasionally with minimal assistance from team staff who handed out surveys or pens. After receiving permission from the team coach or manager, I approached each team approximately 10-20 minutes following the completion of the game, and administered a hard copy of the Mid-Tournament survey to each athlete.  Data collection occurred on the side of the Tournament field after athletes had completed their cool-down routines.  The Mid-Tournament took approximately 3 minutes to complete. Completed surveys were collected by the researcher and sealed in envelopes labelled by team. Athletes were surveyed following the second game of each day (or in the case of two of the teams, following the third game of Day 1, and the second game of Day 2).  Survey responses were later manually collated and data were imputed into a .csv file using Microsoft Excel (Version 14.7.1).  An hypothesis blind research assistant performed an accuracy check on manually imputed survey data by randomly sampling 20 Mid-Tournament surveys and ensuring agreement with .csv file for one survey item per each main variable of interest (team performance, team click, social bonding, technical competence, and personality).

\subsubsection{Post-Tournament survey}
The Post-Tournament survey was administered via the same WeChat group that was set up for the Pre-Tournament survey. Data collection for the Post-Tournament survey began the day after the completion of the Tournament, and finished 4 days later. Athletes were asked to respond to survey items framed in terms of their experience of the Tournament.

\subsubsection{Performance data}
Following the completion of the Tournament, game-by-game performance data were collected from the CRFA Tournament statistician.  Data were stored on an encrypted external hard disk. These data were later manually imputed into a data frame in Microsoft Excel, before being imported into RStudio to be merged with other survey data (Pre-tournament survey, Mid-Tournament surveys, Post-Tournament surveys, and Post-Tournament survey) for statistical analysis.

Survey responses and tournament performance data were collated and imported into RStudio (Version 1.0.136), where they were cleaned and reorganised for statistical analysis. Collated data were then combined with other survey and performance data to be analysed in RStudio.






\clearpage
\section{Data analysis}

\subsection{Roadmap for analysis of study predictions}
The predictions of this study were analysed using 3 subsets of the collected data: 1) Post-Tournament, 2) Pre- to Post-Tournament, and 3) Overall Tournament.  Post-Tournament data were analysed for a relationship between perceptions of team performance, team click, and social bonding following the tournament.  Pre- to Post-Tournament data were analysed for a \textit{change} in variables of interest as a result of the Tournament.  The Overall Tournament dataset (i.e., Pre-, Mid-, and Post-Tournament surveys combined) was analysed to ascertain whether the predicted relationships between variables of interest were consistent throughout the Tournament for individuals.  By analysing multiple observations for the same athlete over a number of time points, it was possible to better account for intra- and inter-individual variation in the collected data, enabling more robust inferences regarding study predictions.

\subsubsection{Data structure and model selection\label{sect:dataStructureModelSelection}}
The \textit{in situ} nature of this study meant that the collected data contained multiple levels of dependency: observations from multiple time points were nested within the same individual athlete; athletes were nested within their respective teams; teams were nested within the men's and women's competitions (i.e., according to sex).  In addition, the data were unbalanced, meaning that an uneven number of observations were recorded for each of the 15 teams over 8 separate time points at which data were collected.  Linear mixed-effects regression (LMER) models were thus used to avoid the violation of the assumptions of independence and equality of variance due to dependencies, and at the same time deal with the problem of missing data \citep{Quene2004,Field2012}.  The LMER can be expressed in notation form as follows:

    \begin{align}
      Y_{ij} & \sim  (\beta_{0} + u_{0j}) + (\beta_{1} + u_{1j})X_{ij} + \varepsilon_{ij}\\
           & \varepsilon_{ij} \sim \mathcal{N}(0,\sigma^{2})
    \end{align}

Where $Y_{ij}$ denotes the $i^{th}$ observation for group $j$, $(\beta_{0} + u_{0j})$ denotes the fixed and random intercept, $(\beta_{1} + u_{1j})$ the random and fixed slope, and $\varepsilon_{ij}$ denotes the error term.  Errors are assumed to be normally distributed with mean of zero.
The random structure of the model (j) was determined by assessing Intra Class Correlation (ICC) values for main group-level variables of interest: team, competition (men's or women's), or individual athlete (in the case of repeated measures analyses).

For the following analyses, multilevel linear models were fit with Maximum Likelihood parameter estimation method using the \textit{lme4} package (Bates and Sarkar, 2006) in the R environment (R Development Core Team, 2006). Model fit was judged by comparing the Akaike Information Criteria (AIC) and Bayesian Information Criteria (BIC).  In addition, marginal and conditional $R^2$ values of equivalent models were compared and used as an indication of model effect size.\footnote{Nakagawa and Schielzeth (2013) have proposed a formula for calculating the proportion of variance explained by the fixed factor of the model (marginal $R^2$, compared to an intercept-only or null model), and the proportion of variance explained by the combination of the fixed factor plus the random factor, (conditional $R^2$, compared to the null).}

%\footnote{The chi-squared test based on -2Log Likelihood scores was not possible when comparing models with different sample sizes owing to inclusion of covariates with missingness.}using a chi-squared test based on -2Log Likelihood score when possible.

Unless otherwise stated, all LMER models controlled for individual performance, technical competence, objective performance measures (final performance rank of team in the Tournament, total individual points scored, average appearances in the starting team), fatigue and personality (Extraversion).

%\myparagraph{Post-Tournament}
%The Post-Tournament data were analysed as described above.

\myparagraph{Data manipulation for Pre- to Post-Tournament analysis}
 The number of observations available in the Pre-Post Tournament data were insufficient to construct a mixed-effects repeated measures design in which both the intercept and slope of observations could vary according to each individual athlete nested within their given team over time.\footnote{Allowing both the intercept and slope to vary requires twice the number of observations to estimate the random effects ($2\times174 = 248$), whereas the available number of observations was only 198.}
 Alternative models for suitable for repeated measures designs, such as RM-ANOVA or ANCOVA, are incapable of allowing each fixed factor (the regression coefficient or slope) to vary randomly according to higher level factors (individual and team), and are also unable to handle unbalanced designs.  As a compromise, change scores in variables of interest (individual and team performance, team click, social bonding, and fatigue) were calculated by subtracting Pre- from Post-Tournament scores for each athlete. The calculation of change scores reduced the complexity of the data structure down to two levels of analysis (individual athlete and team), and meant that relationships between these change variables could be modelled using a linear mixed effects regression.  Change scores of relevant factors were introduced to the model as fixed effects, and their slopes and intercepts were allowed to vary according to team (random effect).

 In regards to analysis of Prediction 2.a and 2.b, it was not possible to directly compare the Pre- and Post-Tournament measures of Team Performance Vs Expectations, given that the item measuring athlete perceptions of overall team performance in the Pre-Tournament survey was not framed in terms of expectation violation (but instead in terms of more or less ``good'' or ``poor'').  Instead, the Post-Tournament measure of Team Performance Vs Expectation only was used to predict change in feelings of team click between Pre- and Post-Tournament measurements.


\myparagraph{Constraints of the Overall Tournament data}
The items administered to athletes during the Mid-Tournament were reduced to accord with athlete schedules and convenience following games.  As such, variables available for analysis in the Overall Tournament data were limited to those that appeared in the Mid-Tournament surveys.  Time constraints meant that components of performance were not assessed in the Mid-Tournament survey, and only appraisals of overall individual and team performance relative to prior expectations were included.  The Mid-Tournament survey contained 3 Team Click items (Unspoken Understanding, General Atmosphere, and Click Pictorial), and 3 Social Bonding items (Emotional Support, Shared Goal, and Fusion Pictorial) among other items.

\myparagraph{Mediation analysis}
%The hypothesised path of relationships outlined in predictions of this study (specifically: Team Performance Components $\rightarrow$ Team Click $\rightarrow$ Social Bonding) suggests the possibility that Team Performance Components exerts its influence on Social Bonding indirectly, via feelings of Team Click.
To test the prediction that team click mediates a relationship between team performance and social bonding, mediation analyses were conducted using linear mixed effects regressions in the Causal Mediation Analysis package in R (Version 4.4.5).  To make inferences concerning the average indirect and total effects, quasi-Bayesian Markov Chain Monte Carlo (MCMC) method based on normal approximation and 1000 simulations was used to estimate the 95\% Confidence Intervals \citep{Tofighi2016a,Imai2010}. MCMC estimation is a form of non-parametric bootstrapping whereby the sampling distribution for the effect of interest is not assumed to be normal but is instead simulated from the model estimates and their asymptotic variances and covariances \citep{Preacher2008}.  For a full justification for, and methods associated with mediation analysis, see Appendix~\ref{app8:mediationAnalysis}.

\subsection{Data Reduction\label{Ch5:dataReduction}}
 To make analysis of predicted relationships more tractable and parsimonious, data reduction was required.  Data reduction allows for a reduction in multicollinearity between predictor variables of interest while retaining as much variance as possible in the observed data \citep{Yong2013}.  The survey items of this study were designed to collectively access latent psychological constructs  (e.g., Team Click and Social Bonding).  As such, a data reduction technique capable of modelling the theoretical structure of these data was preferred over a procedure that merely reduced the collected data to its common variance.  Exploratory Factor Analysis (EFA) was thus chosen as the most suitable data reduction technique over Principal Component Analysis (for a full explanation of EFA, and a more detailed justification for EFA over other available techniques, see Appendix~\ref{app8:EFA}).

An EFA was performed for each subset of the data (Post-Tournament, Pre- to Post-Tournament change, and Overall Tournament), to reduce data to key variables of interest: components of team and individual performance, team click, social bonding, and technical competence (objective and subjective measures).  Measures relating to exertion and fatigue were also consolidated as a factor.  Prior to factors being extracted, correlation matrices of each group of variables were subjected to two common sampling adequacy measures: the Kaiser-Meyer-Olkin (KMO) index and Bartlett's test of sphericity.  Factor loadings of $> .3$ were considered adequate, and only items that loaded on one factor were accepted \citep{Field2012}.  Sum of Squares Loadings (SS Loadings) for each factor were also reported \citep{Dziuban1974}.  Finally, two reliability measures (Guttman's $\lambda$ and Cronbach's $\alpha$) were reported to provide estimates of the average correlation of all items that could pertain to the underlying construct \citep[values  $> .5$ were considered acceptable; see][]{Benton2015,Tabachnick2007}.  Factor scores were calculated as standardised z-scores with a mean of approximately zero, and standard deviation of approximately one.

\subsubsection{Team Performance Components and Individual Performance Components}
The theoretical predictions of this dissertation concentrate in particular on athletes' team-level perceptions of joint action (see Chapter~\ref{sect:onfieldCoordination}).  As such, perceptions of individual and team components of performance were analysed separately.  This separation allowed for concentration on perceptions of individual and team performance on team click and social bonding. In addition, perceptions of success in individual performance components were used as a statistical control for perceptions of success in team performance.


\subsubsection{Perceptions of team performance relative to prior expectations}
The predictor variable of interest for \mccorrect{Prediction 2.a and 2.b,} perceptions of team performance relative to prior expectations, was a single item measure, and did not require data reduction.  Given that many of the other outcome variables of interest were transformed (via EFA) into standardised z-scores ($mean \approx 0, SD \approx 1$), perceptions of team performance relative to prior expectation was also standardised ($mean \approx 0, SD \approx 1$) for consistency and accurate generation of estimates within subsequent linear mixed effects models \citep{Bates2015}.

%Perceptions of individual and team performance relative to prior expectations was included in only the Mid- and Post-Tournament surveys.































\clearpage

\section{Results}


\subsection{Descriptive Statistics}

  \subsubsection{Participants}

Data were collected for 174 adult rugby playing athletes ($male = 93, M(age) = 21.67, SD = 3.67, range = 17-32$) accross 8 time points. Objective performance data were collected after each of the 6 games for all 174 athletes who participated in the Tournament; survey responses were recorded for a total of 165 unique athletes at 4 different time points: once Pre-Tournament, twice during the Tournament, and once Post-Tournament (see Table~\ref{tab:surveyDataByTime}).

% $n = 120$)(once each day of the two day tournament, following the 2nd or 3rd game of each day, $n = 164$) ($n = 118$)(mean athletes per team = 11.6 ($SD =1.06$), ---


% Please add the following required packages to your document preamble:
% \usepackage{booktabs}
\begin{table}[]
  \centering
  
  
\begin{tabular}{@{}lcc@{}}
\toprule
\textbf{Time} & \textbf{\begin{tabular}[c]{@{}c@{}}Survey \\ (Men, Women)\end{tabular}} & \textbf{\begin{tabular}[c]{@{}c@{}}Objective Performance\\ (Men, Women)\end{tabular}} \\ \midrule
Pre-Tournament & 120 (68, 52) & - \\
 &  &  \\
Day 1 &  &  \\
\multicolumn{1}{r}{Game 1} & - & 174 (93, 81) \\
\multicolumn{1}{r}{Game 2} & 129 (60, 69) & 174 (93, 81) \\
\multicolumn{1}{r}{Game 3} & 22 (8, 14) & 174 (93, 81) \\
 & \multicolumn{1}{l}{} & \multicolumn{1}{l}{} \\
Day 2 &  &  \\
\multicolumn{1}{r}{Game 4} & - & 174 (93, 81) \\
\multicolumn{1}{r}{Game 5} & 163 (91, 72) & 174 (93, 81) \\
\multicolumn{1}{r}{Game 6} & - & 174 (93, 81) \\
 &  &  \\
Post-Tournament & 118 (65, 53) & - \\
 & \multicolumn{1}{l}{} & \multicolumn{1}{l}{} \\ \bottomrule
\end{tabular}
  
  
\caption{Survey and objective performance data collected at each time point during the Tournament}
\label{tab:surveyDataByTime}
\end{table}


Of the 174 athletes competing in the Tournament, 120 were surveyed during a 4 day window before the Tournament (69\% of the total sample, $male = 68$).  On Day 1 of the Tournament, a total of 151 athletes (87\% of sample, $male = 68$) were surveyed: 129 athletes ($male = 60$) in 11 teams were surveyed after their 2nd game of the day, and 22 athletes ($male = 8$) in two teams were surveyed after their 3rd game. Two of 11 teams (Hebei men's and Fujian men's) were not surveyed due to timing and logistical constraints.  On Day 2 of the Tournament, a total of 163 athletes (94\% of sample, $male = 91$) in 14 teams were surveyed after their second game of the day. One team (Shanghai Women's) was not surveyed due to timing and logistical constraints.  A total of 100 athletes (57\% of the sample, $male = 59$) completed both the Pre- and Post-Tournament surveys, and a total of 99 athletes completed all 4 surveys (57\% of the sample, $male = 59$). Logistical challenges relating to data collection meant that observations were missing for athletes across the 4 survey time points. Missingness in the survey data ranged from 15-19\% at any one of the 4 survey time points.\\


\subsubsection{Variables of interest over time}
Table~\ref{tab:rawVariablesByTime} displays the basic summary statistics (mean and standard deviation) for variables of interest collected at 4 time points during the Tournament.  The central tendency of variables within the categories of performance, team click, social bonding, and fatigue and exertion were above the mid-point of each scale.  In addition, Mid-Tournament measurements tended to be lower than Pre- or Post-Tournament measures for each category.

% Please add the following required packages to your document preamble:
% \usepackage{booktabs}
\begin{table}[]
  \centering

\begin{tabular}{@{}lrlll@{}}
\toprule
\multicolumn{1}{c}{\textbf{Item}} & \multicolumn{1}{c}{\textbf{Pre}} & \multicolumn{1}{c}{\textbf{Day 1}} & \multicolumn{1}{c}{\textbf{Day 2}} & \multicolumn{1}{c}{\textbf{Post}} \\ \midrule
\multicolumn{1}{c}{\textbf{\begin{tabular}[c]{@{}c@{}}Team \\ Performance\end{tabular}}} &  &  &  &  \\
Vs expectations & 71.11 (21.87) & 54.43 (30.63) & 53.19 (32.98) & 64.36 (23.61) \\
Attack & 66.00 (24.95) & \multicolumn{1}{c}{-} & \multicolumn{1}{c}{-} & 65.33 (20.26) \\
Defence & 65.06 (22.85) & \multicolumn{1}{c}{-} & \multicolumn{1}{c}{-} & 62.42 (22.50) \\
Communication & 62.37 (24.21) & \multicolumn{1}{c}{-} & \multicolumn{1}{c}{-} & 65.25 (21.26) \\
Support play & 64.36 (23.53) & \multicolumn{1}{c}{-} & \multicolumn{1}{c}{-} & 65.75 (19.72) \\
 &  &  &  &  \\
\multicolumn{1}{c}{\textbf{\begin{tabular}[c]{@{}c@{}}Individual \\ Performance\end{tabular}}} &  &  &  &  \\
Vs expectations & 68.92 (21.32) & 39.43 (27.25) & 40.92 (27.93) & 56.36 (23.47) \\
Passing technique & 65.74 (24.94) & \multicolumn{1}{c}{-} & \multicolumn{1}{c}{-} & 58.41 (24.25) \\
1 on 1 defence & 59.21 (26.64) & \multicolumn{1}{c}{-} & \multicolumn{1}{c}{-} & 57.64 (23.57) \\
Effectiveness in contact & 65.66 (24.85) & \multicolumn{1}{c}{-} & \multicolumn{1}{c}{-} & 62.15 (24.81) \\
Support play & 69.54 (23.23) & \multicolumn{1}{c}{-} & \multicolumn{1}{c}{-} & 62.62 (22.70) \\
Decision making & 65.48 (23.80) & \multicolumn{1}{c}{-} & \multicolumn{1}{c}{-} & 61.22 (21.43) \\
 &  &  &  &  \\
\multicolumn{1}{c}{\textbf{Team Click}} &  &  &  &  \\
Tacit understanding & 71.58 (20.77) & 55.92 (26.88) & 55.30 (29.43) & 72.72 (19.95) \\
Team aura & 75.51 (23.27) & 65.74 (31.95) & 64.32 (33.39) & 78.45 (21.34) \\
Click Pictorial & 3.87 (1.24) & 3.46 (1.49) & 3.33 (1.70) & 3.93 (1.04) \\
Reliability of others & 67.43 (28.01) & \multicolumn{1}{c}{-} & \multicolumn{1}{c}{-} & 68.00 (23.09) \\
Reliability for others & 62.38 (25.40) & \multicolumn{1}{c}{-} & \multicolumn{1}{c}{-} & 63.45 (25.80) \\
Ability extended & 70.45 (25.83) & \multicolumn{1}{c}{-} & \multicolumn{1}{c}{-} & 72.25 (19.27) \\
 &  &  &  &  \\
\multicolumn{1}{c}{\textbf{Social Bonding}} &  &  &  &  \\
Emotional support & 70.12 (26.21) & 67.29 (30.56) & 67.53 (32.55) & 79.67 (18.84) \\
Common goal & 77.66 (24.28) & 76.34 (30.50) & 71.42 (35.47) & 86 (15.56) \\
Fusion (Pictorial) & 4.26 (1.25) & 4.06 (1.47) & 3.85 (1.69) & 4.33 (1.19) \\
Fusion (Verbal) & 4.03 (.74) & \multicolumn{1}{c}{-} & \multicolumn{1}{c}{-} & 4.00 (.71) \\
Group Identification & 4.34 (.78) & \multicolumn{1}{c}{-} & \multicolumn{1}{c}{-} & 4.29 (.67) \\
 &  &  &  &  \\
\multicolumn{1}{c}{\textbf{Other}} &  &  &  &  \\
Fatigue & \multicolumn{1}{c}{-} & 55.04 (24.94) & 62.26 (24.37) & 69.27 (21.24) \\
Injury status & 81.28 (23.36) & 79.70 (22.45) & 73.29 (29.86) & 76.14 (26.91) \\ \bottomrule
\end{tabular}
    
    
\caption{Mean (SD) of key variables of interest measured at Pre-Tournament (n = 120), Day 1 (n = 149), Day 2 (n = 163), and Post-Tournament (n = 118). }
\label{tab:rawVariablesByTime}
\end{table}



In relation to performance measures, athletes appeared on average to be more critical of their own and their team's performance (relative to prior expectations) when surveyed immediately after games on Day 1 and Day 2 than they were following the completion of the Tournament (note, however, that survey items relating to individual and team performance administered Pre-Tournament were not posed in relation to athlete expectations, and thus could not be directly compared to subsequent Mid- and Post-Tournament measures).  The same pattern was identifiable in team click variables, with the central tendency of Mid-Tournament measures of Tacit Understanding and Team Aura 10-15\% lower than Pre- or Post-Tournament measures.  This is also the case for variables related to social bonding: Emotional Support and Common Goal in particular showed a steep increase from Mid-Tournament measurements to the Post-Tournament measurement.  The same pattern was identifiable for variables relevant to fatigue.
%Pre-, Mid-, and Post-Tournament survey items are reviewed in more detail in Appendix~\ref{app8:tournamentSurvey}, Sections ~\ref{app8:descriptivesPre}\nobreakdash~\ref{app8:descriptivesPost}.














\subsection{Data Reduction\label{Ch5:dataReduction}}


\subsubsection{Perceptions of performance}

\myparagraph{Post-Tournament}
Items concerning team components of performance (team defence, team attack, team support play, and on field communication) were subjected to EFAs.  Correlations between team component performance items were very high (all $r's > .5$), which suggested that one factor would be appropriate . The KMO index and Bartlett's test both suggested high sampling adequacy, ($KMO = .79$, $\chi^2(6, N = 118) = 342.14$, $p < .001$).  One factor, labelled ``Team Performance Components'' was imposed on the data, which explained 72.8\% of the overall variance ($SS Loading = 2.91$). $Guttman's \lambda =.90$ and $Cronbach's \alpha = .91$ indicated that the data reduction was appropriate and reliable.

%(see Appendix~\ref{app8:tournamentSurvey} Table~\ref{tab:22teamPerformancePostCorr})

Items relating to individual component performance (passing technique, support play in attack, effectiveness in contact, 1on1 defence, and decision making in attack)  were subjected to an EFA.  Correlations between individual component performance items were also very high (all $r's > .5$), which suggested that one factor would be sufficient (confirmed by sampling adequacy tests, $KMO =  .84$, $\chi^2(10, N = 118) =  326.38$, $p < .001$).  One factor, labelled ``Individual Performance Components'' explained 62.1\% of the overall variance (SS Loading = 3.10).
$Guttman's \lambda =.88$ and $Cronbach's  \alpha = .89$ both indicated that the data reduction was appropriate.


To confirm that the theoretically motivated separation of Team Performance Components from Individual Performance Components was appropriate for the collected data, a follow up EFA was conducted, in which team and individual performance component variables were combined in one matrix.  Sampling adequacy measures indicated high suitability ($KMO = 0.83$, $\chi^2(36, N = 118) = 726.60$, $p < .01$).  As expected, an EFA extracted two factors: Individual performance measures loaded on one factor (proportion of variance = .34, $SS Loading = 3.09$), and team performance measures loading on a second factor (proportion of variance = .32, $SS Loading = 2.90$). $Guttman's \lambda =.93$ and $Cronbach's \alpha = .90$ indicated that the data reduction was appropriate.

\myparagraph{Pre- to Post-Tournament}
Correlations between components of team performance were very high (all $r's > .65$), and the suitability of the correlation matrix for factor extraction was confirmed by sampling adequacy tests ($KMO = .82$, $\chi^2(6, N = 238) = 717.55$, $p < .001$).  As such, one factor (``Team Performance Components Change'') was imposed on items concerning team component performance, which explained 74.5\% of the overall variance in items relating to team performance (SS Loading = 2.97). $Guttman's \lambda =.90$ and $Cronbach's \alpha = .92)$ indicated that the data reduction was appropriate.  Correlations between 5 items relating to individual performance were also sufficiently high (all $r's > .45$, $KMO = .83$, $\chi^2(10, N = 238) = 633.82$, $p < .001$), suggesting that one factor would be appropriate for individual performance.  One factor (``Ind Performance Components Change'') was imposed, which explained 59.9\% of the overall variance (SS Loading = 2.99).  $Guttman's \lambda =.87$ and $Cronbach's \alpha = .88$ indicated that the data reduction was appropriate.

\myparagraph{Overall Tournament}
Athlete perceptions of components of performance were not administered in the Mid-Tournament surveys, and thus were not eligible for Overall Tournament analysis.




\subsubsection{Team Click}

\myparagraph{Post-Tournament}
All 6 items relevant to team click were included in an EFA: Tacit Understanding, Team Aura, Team Click Pictorial, Reliability Of Others, Reliability For Others, and Abilities Extended.  Medium to strong correlations between variables of interest ($r's > .3$) and sampling adequacy measures suggested that imposing one factor was appropriate, $KMO =  .69$, $\chi^2(15, N = 118) = 182.73$, $p < .001$.  One factor labelled ``Team Click'' was extracted from the data, which explained 34.5\% of the overall variance ($SS Loading = 2.07$).  Guttman's $\lambda =.76$ and Cronbach's $\alpha = .75$ indicated that the data reduction was appropriate.  These reliability statistics provided confidence that the novel pictorial click measure related strongly to the ethnographically derived click items (for example, Tacit Understanding and Team Aura).


\myparagraph{Pre- to Post-Tournament}
Pre- and Post-Tournament measures of the same 6 team click items were collated and subjected to EFA. Correlations between variables were moderate to high, indicating that one factor was suitable (all $r's > .3$, $KMO = .78$,$\chi^2(15, N = 238) = 336.41$, $p < .001$).  One factor, ``Team Click Change'', was imposed, which explained 37.6\% of the overall variance (all items loadings $> .4$, SS Loading = 2.26).  $Guttman's \lambda =.76$ and $Cronbach's \alpha = .76$ indicated that the data reduction was appropriate.
%$\chi^2(9, N = 238) = 31.52 $, $p < .001$

\myparagraph{Overall Tournament}
Tacit Understanding, Team Aura, and Click Pictorial---the 3 team click items that were included in all 4 surveys---were subjected to EFA.  Correlations between variables were high, suggesting that imposing one factor was appropriate (all $r's > .62$, $KMO = .7$, $\chi^2(3, N = 440) = 723.67$).  The factor ``Team Click Tournament'' explained 70.3\% of the variance (SS Loadings = 2.11).  $Guttman's \lambda =.83$ and Cronbach's $\alpha = .87$ indicated that the data reduction was appropriate.

%, and a Chi-squared test indicated that one factor was sufficient to account for the variance of these items ($\chi^2 (9, ) = 46.36$, $p < .001$).

%Interestingly, when click and bonding measures were analysed together, the following loading were observed:
%Loadings:
%                       Factor1 Factor2
%unspokenUnderstanding7  .710
%generalAtmosphere7      .713
%clickPictorial7         .672  -0.106
%reliabilityOfOthers7    .128   .539
%reliabilityForOthers7           .424
%abilityExtended7       -0.110   .956
%emotionalSupport7       .656   .131
%sharedGoal7             .838
%fusionPictorialTeam7    .424
%fusionVerbal7           .138   .281


\subsubsection{Social Bonding}
Five survey items related to social bonding (Emotional Support, Common Goal, Identity Fusion Verbal Scale, Group Identification Verbal Scale, and the Identity Fusion Pictorial (Team, Family, Country) were subject to data reduction.


%A correlation matrix of all 5 social bonding variables (~\ref{tab:4bondingPostCorr})
\myparagraph{Post-Tournament}
Correlations between 5 social bonding variables indicated that Group Identification Verbal Scale did not share common variance with other variables (all correlations were $<.1$, except for the verbal measures of Identity Fusion ($r =.36$). As such, Group Identification was excluded from analysis.  Sampling adequacy variables suggested that the remaining subset of variables were appropriate for analysis, $KMO = .65$, $\chi^2(10, N = 118) = 108.22$, $p < .001$.  EFA was performed on 4 remaining items, imposing one factor labelled ``Social Bonding'', which explained 34.5\% of the overall variance ($SS Loading = 1.60$, $Guttman's \lambda =.66$ and $Cronbach's \alpha = .65$).

\myparagraph{Pre- to Post-Tournament}
The same 4 survey items related to feelings of social bonding were subjected to EFA.  Correlations between variables were generally high (all $r's > .3$, $KMO = .66$, corrtest Bartlett: $\chi^2(6, N = 238) = 218.95$, $p < .001$), indicating that one factor would be appropriate.  One factor, labelled ``Social Bonding Pre Post'' was extracted, which explained 41.9\% of the overall variance (SS Loading = 1.68).  $Guttman's \lambda =.68$ and $Cronbach's \alpha = .71$ indicated that the data reduction was appropriate.
% $\chi^2 (df=2) = 10.3 $, $p < .01$,

\myparagraph{Overall Tournament}
The 3 social bonding variables that featured consistently in all surveys (Emotional Support, Common Goal, and Fusion Pictorial) were subjected to EFA.  Correlations between variables were high, suggesting that imposing one factor was appropriate (all $r's > .63$, $KMO = .71$, $\chi^2(3, N = 440) =  759.30$, $p < .001$).  The factor imposed for Social Bonding (``Social Bonding Tournament'') explained 71.7\% of the variance ($SS Loadings =  2.15$), and $Guttman's \lambda =.84$ and $Cronbach's \alpha= .88$ indicated that the data reduction was reliable.


\subsubsection{Technical Competence}
Eight items relevant to technical competence were analysed in a correlation matrix to assess relatedness.  Medium correlations among measures of objective competence (three out of 4 items correlated at $> .3$) and among measures of subjective competence (all items except for team competence measure correlated at $> .3$) suggested that the data could be explained by two underlying factors. Team Ability Chinese Provinces was dropped from analysis due to low correlation with other competence variables, possibly because the item did not ask about an individual athlete's competence (it referred instead to an athlete's opinion of the competence of the team of which they were a member). An examination of the KMO measure of sampling adequacy ($KMO = .67$), and the Bartlett sphericity test indicated that two factors were adequate, $\chi^2(21, N = 120) = 239.71$, $p < .001$.

An EFA of technical competence variables revealed that items of interest loaded on two factors. The exception was the variable Starting Reserve, which failed to load on either factor, and was thus dropped from analysis. Measures of objective competence (Years Team, Training Age, and Age) loaded on the first factor, which was labelled ``Objective Competence'' because the measures were all objective markers of an athlete's competence.  Objective Competence explained 26.4\% of the total variance ($SS Loading = 1.85$). The remaining measures of subjective competence (Ability Teammates, Ability Chinese Pros, Ability International Pros) loaded on the second factor.  The second factor was labelled ``Subjective Competence'', due to the fact that all measures were the product of subjective self-report.  Subjective competence explained 23.8\% of the variance ($SS Loading = 1.67$). $Guttman's \lambda =.74$ and $Cronbach's \alpha = (.67)$ indicated that the data reduction was appropriate and reliable.

\subsubsection{Additional data reduction}
Data reduction on survey items pertaining to perceptions of fatigue and exertion (hereafter Fatigue) are reported in full detail in Appendix~\ref{app8:fatigueEFA}.


\subsubsection{Correlations between key factors of interest}
Correlations values of factors of interest are shown in Tables~\ref{tab:surveycorr_post} and~\ref{tab:tournamentFactorsMatrixOverall}.  In the Post-Tournament and Pre- to Post-Tournament subsets, correlation values for outcome variables of team click and social bonding were moderate to high, but less than a generally accepted threshold of .70 \citep{Field2012}.  In the Overall Tournament subset, the correlation value between team click and social bonding was $.75$, suggesting that models in which team click predicts social bonding in the Overall Tournament subset should be treated with caution.

\input{images/tournamentSurveyCorrMatrixPost}
\input{images/tournamentSurveyCorrMatrixPrePost}
\input{images/tournamentSurveyCorrMatrixOverall}




\subsection{ICC values}

\myparagraph{Post-Tournament}
Computing ICC values for the Post-Tournament factors of interest identified team-level clustering in the data for Team Performance Components ($ICC = .37$), Individual Performance Components ($ICC = .22$), Team Click ($ICC = .30$), Social Bonding ($ICC = .10$) and objective measures of technical competence ($ICC = .36$).  ICC values indicated only trivial clustering according to sex all $ICC < .10$, except for the ICC for Individual Performance Components, which was .11).  As such, team (but not sex) was included in subsequent models as a random effect (for a detailed assessment of the group-level dependencies in the collected data, see Appendix~\ref{app8:ICC}).


\subsubsection{Pre- to Post-Tournament ICC Values}
Team-level clustering was identified for
Team Performance Components ($ICC = .15$), Individual Performance Components ($ICC = .32$) and Team Click ($ICC = .16$).  Social Bonding ($ICC = .01$) did not exhibit within-group variance relative to between group variance. ICC values for sex were trivial (all $ICC < .10$).

\subsubsection{Overall Tournament ICC Values}
All ICC values for Team and Tournament were trivial (all $ICC < .10$).

\subsubsection{Random effects}
Based on these ICC values, Team was included as a random effect for Post-Tournament and Pre- to Post-Tournament models.
Specifically, the intercepts and slopes of the response variable were allowed to vary according to team (i.e., random intercept model, see \textcite{Pinheiro2000}; for an application, see \textcite{Oberauer2006}).
Specifying Team as a random effect helped statistically account for clustering of observations within teams.

For the Overall Tournament analysis, neither team nor tournament were included as random effects, due to ICC values for Team and Tournament that were negligent.  Athlete was included in the random structure to account for repeated measures of individuals.















\clearpage




\section{Analysis of study predictions}


%\subsection{Prediction 1: More positive perceptions of team performance predict higher levels of Team Click}


\subsection{Prediction 1.a: More positive perceptions of team performance predict higher levels of team click\label{sect:prediction1a}}

This prediction was tested by assessing the relationship between perceptions of components of team performance and perceptions of team click in the Post-Tournament and Pre- to Post-Tournament data.\footnote{The Overall Tournament data did not contain consistent measures of components of team performance and so was not analysed in relation to Prediction 1.a.}

\myparagraph{Post-Tournament}
Results of the Post-Tournament analysis support this prediction. The model revealed a significant positive effect of Team Performance Components on Team Click ($\beta = .65$ ($95\% CI = .44, .88$), $SE = .11$, $t(13.13) = 5.94$, $p < .01$, marginal $R^2 = .56$, conditional $R^2 = .61$).  All other fixed effects did not significantly predict team click, but the inclusion of these fixed effects in the model did significantly improve the overall model fit, as indicated by a comparison of AIC/BIC values between \mccorrect{iterations} of the model (see Appendix~\ref{app8:prediction1a} Table~\ref{tab:MLM1aJointActionSuccessClick}).

Model residuals were normally distributed around zero ($W = .97, p = .06$), and individual cases had low influence on the model (Cook's Distances all $< .25$, see Appendix~\ref{app8:prediction1a} Figure~\ref{fig:MLM1aAssumptions}). Results of this model support the prediction that more positive perceptions of team performance predict higher feelings of team click (see Figure~\ref{fig:jasClickModelSLope}).

\begin{figure}[htbp]
  \centering
\includegraphics[scale = .5]{images/jasClickModelSlope}
  \caption{Prediction 1.a (Post-Tournament): Team Performance Components predicts Team Click.  Both variables are factors expressed as z-scores ($mean \approx 0, SD \approx 1$).  The line of best fit has been adjusted based on parameter estimates of the LMER ($n = 98$).}
  \label{fig:jasClickModelSLope}
\end{figure}



\myparagraph{Pre- to Post-Tournament}
Results of the Pre- to Post-Tournament data also supported the prediction.  The model revealed a significant positive relationship between Team Performance Components Change and Team Click Change ($\beta = .51$ ($95\% CI =  .29, .72$), $SE = .11$, $t(10.80) = 4.63$, $p < .01$, marginal $R^2 = .40$, conditional $R^2 = .47$).  No other main effects were significant.  Model residuals were normally distributed around zero ($W = .98, p = .24$), and individual cases had low influence on the model (Cook's Distances all  $< .20 $; see Appendix~\ref{app8:prediction1a} Table~\ref{tab:MLM21aJointActionSuccessClick} for results of all iterations of the model and Figure~\ref{fig:MLM21aAssumptions} for model assumptions).

This model supported the prediction that positive change in perceptions of components of team performance are associated with increase in feelings of team click (see Figure~\ref{fig:jasClickDeltaModelSLope}).  On average, athletes who reported a positive increase in perceptions of components of team performance also reported a positive increase in feelings of team click throughout the Tournament.

\begin{figure}[htbp]
  \centering
\includegraphics[scale=.5]{images/jasClickDeltaModelSlope}
  \caption{Prediction 1.a (Pre- to Post-Tournament): Team Performance Components Change predicts Team Click Change. The slope has been adjusted according to the parameter estimates of the linear model ($n = 97$).}
  \label{fig:jasClickDeltaModelSLope}
\end{figure}

In sum, results suggest more positive perceptions of components of team performance were associated with higher levels of perceived team click. A positive increase in perceptions of Team Performance Components predicted a positive increase in feelings of Team Click between Pre- and Post-Tournament surveys, suggesting that athletes who perceived an increase in success of components of team performance also experienced an increase in feelings of team click. Together, these results support a relationship between joint action and team click.






%\subsection{Prediction 1.c: Team Performance Components predicts Team Click, moderated by Team Performance Vs Expectations\label{sect:Prediction1c}}

%The prediction that more positive violation of expectations surrounding team performance moderated a direct relationship between perceptions of components of team performance and team click was tested in the Post-Tournament and Pre - to Post-Tournament data set.

%\subsubsection{Post-Tournament}
%Team Performance Vs Expectations was added to the model as an interaction presented in Prediction 1.a. Results revealed that the interaction between Team Performance Components and Team Performance Vs Expectations was not significant, $\beta = .002$ ($95\% CI =  -.0062, .0063$), $SE = .08$, $t(39.7) = .026$, $p = .98$, marginal $R^2 = .56$, conditional $R^2 = .65$ (see Appendix~\ref{app8:tournamentSurvey}, Table~\ref{tab:MLM1cPerformanceClickInteraction}).  The inclusion of the interaction term failed to improve upon the fit of the previous model, judging by the relative goodness of fit, $AIC(1.c) = 217.91$ compared to $AIC(1.a) = 209.48$, $SD = .52 $, $\chi^2(18, N = 97) = 3.56$, $ p =.74$.

%\subsubsection{Pre- to Post-Tournament}
%The interaction effect of Team Performance Vs Expectations and Team Performance Components on Team Click Change was not significant, $\beta = .004$ ($95\% CI =  -.002, .01$), $SE = .003$, $t(30.14) = 1.42$, $p = .17$, and failed to improve the model fit, $\chi^2 = 2.72$, $ p = .44$ (see Table~\ref{tab:MLM21ccTeamPerfExpcClickInt}).  This result suggests that perceptions of team performance relative to prior expectations did not have an additive effect on the positive relationship between perceptions of Team Performance Components and Team Click.

%\myparagraph{Prediction 1.c: Summary of results}
%Results did not offer support for the prediction for an interaction effect of Team Performance Components and Team Performance Vs Expectations on team click.









\subsection{Prediction 1.b: Higher levels of team click will predict higher levels of social bonding}

\myparagraph{Post-Tournament}
 Results supported the prediction.  A LMER model revealed a significant relationship between Team Click and Social Bonding, $\beta = .64$ ($95\% CI =  .47, .81$), $SE = .09$, $t(13.73) = 7.49$, $p < .01$, marginal $R^2 = .49$, conditional $R^2 = .51$ (see Appendix~\ref{app8:prediction1b} Table~\ref{tab:MLM2aTeamClickBonding} for a full description of results).
 \mccorrect{No other main effects were significant} Model residuals were normally distributed around zero ($W = .98, p = .14$), and individual cases had low influence on the model (Cook's Distances all $< .15$, see Appendix~\ref{app8:prediction1b} Figure~\ref{fig:MLM2aAssumptions} for a full report of model assumptions).
 The model supported the prediction that higher levels of team click are associated with higher levels of social bonding (see Figure~\ref{fig:clickBondModelSlope}).



  \begin{figure}[htbp]
    \centering
  \includegraphics[scale=.5]{images/clickBondModelSlope.pdf}
    \caption{Prediction 1.b (Post-Tournament): Team Click predicts Social Bonding. The slope reflects the model parameter estimates of the LMER ($n = 97$).}
    \label{fig:clickBondModelSlope}
  \end{figure}



  \myparagraph{Pre- to Post-Tournament}
Results of the Pre- to Post survey data also supported the prediction.
The LMER model revealed a significant positive effect of Team Cick Change on Social Bonding Change, ($\beta = .37$ ($95\% CI =  .10, .64$), $SE = .14$, $t(8.75) = 2.65$, $p = .03$, marginal $R^2 = .17$, conditional $R^2 = .23$; see Appendix~\ref{app8:prediction1b} Table~\ref{tab:MLM22acClickcBonding} for a full description of model estimates).  Model residuals were normally distributed around zero ($Shapiro-Wilk = .98, p = .15$). and individual cases had low influence on the model (Cook's Distances all $< .6$, see Appendix~\ref{app8:prediction1b} Figure~\ref{fig:MLM22aAssumptions} for model assumptions).  This model suggests that athletes who experienced an increase in feelings of team click also experienced an increase in feelings of social bonding towards their team (see Figure~\ref{fig:clickBondDeltaModelSlope}).


  \begin{figure}[htbp]
    \centering
  \includegraphics[scale=.5]{images/clickBondDeltaModelSlope.pdf}
    \caption{Prediction 1.b (Pre- to Post-Tournament): Team Click Change predicts Social Bonding Change. The slope reflects parameter estimates of the LMER ($n = 97$).}
    \label{fig:clickBondDeltaModelSlope}
  \end{figure}


\myparagraph{Overall Tournament}
A model of the Overall Tournament data also supported the prediction. The LMER revealed a significant positive relationship between Team Click and Social Bonding, $\beta = .64$ ($95\% CI = .55, .74$), $SE = .05$, $t(87.4) = 13.84$, $p < .01$, marginal $R^2 = .49$, conditional $R^2 = .66$. No other main effects were significant.

Examination of model residuals revealed that they were not normally distributed around zero, ($W = .92, p < .01$), owing to high negative skew ($-1.24$) and high kurtosis (5.12) (see Appendix~\ref{app8:prediction1b} Figure~\ref{fig:MLM31bAssumptions}).  Exclusion of outliers according to Tukey's method \citep[observations above and below 1.5x the Inter Quartile Range (IQR); see][]{Tukey1977}, and log transformation of the outcome variable  failed to markedly improve the normality of residuals (Outlier removal: $W = .92, p < .01$, and Log transformation: $W = .96, p < .01$, respectively).  To resolve this assumption violation, the outcome variable was first subjected to outlier-removal, and then subsequently log-transformed, which appeared to improve the distribution of residuals only somewhat ($W = .98, p = .0002$).
The failure of this model to converge suggested that it was not a robust indication of the relationship between Team Click and Social Bonding in the Overall Tournament data.

%Log transformation of the outcome variable ($W = .92, p < .001$) and outlier removal  procedures improved the model fit only marginally (not within the accepted bounds of normality).


In sum, athlete perceptions of team click predicted athlete perceptions of social bonding in the Post-Tournament and Pre- to Post-Tournament data sets.  The Overall Tournament model of this relationship was not reliable due to the violation of model assumptions.






\subsection{Prediction 1.c: More positive perceptions of team performance will predict higher levels of social bonding}


\myparagraph{Post-Tournament}
The model revealed a significant effect of Team Performance Components on Social Bonding, $\beta = .41$ ($95\% CI =  .14, .69$), $SE = .14$, $t(21.98) = 2.92$, $p < .01$, marginal $R^2 = .27$, conditional $R^2 = .42$.  The model also revealed a significant positive effect of Subjective Competence on Social Bonding, $\beta = .09$ ($95\% CI =  .02, .16$), $SE = .03$, $t(90.76) = 2.52$, $p = .01$. No other main effects were significant (see Appendix~\ref{app8:prediction1c} Table~\ref{tab:MLM3aJointActionSuccessBonding}).

Model residuals were non-normally distributed ($W = .95, p = .0007$), owing to relatively large negative skew (-.87) (see Appendix ~\ref{app8:prediction1c} Figure~\ref{fig:MLM3aAssumptions}).  The graphic representation of the data indicated possibility of outliers biasing the model (see Figure~\ref{fig:jasBondModelSlope}).
Exclusion of outliers (based on Tukey's method, see Section~\ref{sect:prediction1b}) improved the distribution of model residuals, but still violated the assumption of normality ($Shapiro-Wilk = .96, p = .009$).  Individual cases had low influence on the model (Cook's Distances all < .10).

Due to the \textit{positive} skew of the model residuals, the outcome variable was transformed by taking the log of the reversed scores of the outcome variable (i.e., $log10(k - y)$), where $k$ is a constant value from which each score for $y$ is subtracted so that the distribution of the outcome variable is reversed \citep{Howell2012}.\footnote{Reversing the distribution of the outcome variable allows the logarithmic function to normalise the distribution of the variable, by pushing them from the left hand side of the distribution towards the centre.}
Transformed values were then returned to their original direction for analysis \citep{Field2012}.  Rebuilding the model with a log-transformed outcome variable appeared to improve the fit of the model more than the outlier-removed model, and avoided the removal of any observations.

This adjustment achieved normal distribution of residuals ($W = .98, p = .08$), and the R-squared values for the model improved (marginal $R^2 = .27$, conditional $R^2 = .43$; see Appendix~\ref{app8:prediction1c} Figure~\ref{fig:MLM3aLogAssumptions} for model results and assumptions).  Results of the log-transformed model were taken as satisfactory evidence for a significant positive relationship between perceptions of Team Performance Components and feelings of Social Bonding.  Results also suggested that on average athletes who reported higher levels of subjective competence also reported higher levels of social bonding.
%While Levene's Test for Equality of Variance indicated that the assumption of homoscedasticity was met at the group-level, $F(13,83) = .80, p = .66$,


  \begin{figure}[htbp]
    \centering
  \includegraphics[scale=.5]{images/jasBondModelSlope.pdf}
    \caption{Prediction 1.c (Post-Tournament): Team Performance Components predicts Social Bonding. The line of best fit reflects the log-adjusted LMER model ($n = 97$).}
    \label{fig:jasBondModelSlope}
  \end{figure}


\myparagraph{Pre- to Post-Tournament}
Results of a LMER model revealed a significant positive effect of Team Performance Components Change on Social Bonding Change ($\beta = .36$ ($95\% CI =  .19, .54$), $SE = .09$, $t(90) = 4.02$, $p < .01$, marginal $R^2 = .18$, conditional $R^2 = .18$), suggesting that on average, athletes who experienced a positive increase in perceptions of components of team performance also experienced an increase in perceptions of social bonding (see Figure~\ref{fig:jasBondDeltaModelSlope}). All other fixed effects were not significant (see Appendix~\ref{app8:prediction1c} Table~\ref{tab:MLM23aJointActionSuccessBonding}). Model residuals were normally distributed around zero ($W = .98, p = .19$), and individual cases had low influence on the model (Cook's Distances all $< .5$; see Appendix~\ref{app8:prediction1c} Figure~\ref{fig:MLM23aAssumptions} for a full description of model assumptions).  This model supported the prediction that more positive perceptions of components of team performance significantly predict higher levels of social bonding.


%he model also revealed a significant \textit{negative} effect of Individual Performance Components Change on Social Bonding ($\beta = -.23$ ($95\% CI =  -.44, -.02$), $SE = .11$, $t(97) = -2.12$, $p = .04$), indicating that athletes whose attitudes towards their own performance increased in positivity showed an average decrease in feelings of social bonding between Pre- and Post-Tournament measures.




  \begin{figure}[htbp]
    \centering
  \includegraphics[scale=.5]{images/jasBondDeltaModelSlope.pdf}
    \caption{Prediction 1.c (Pre- to Post-Tournament): Team Performance Components Change predicts Social Bonding Change. The line of best fit reflects the parameter estimates of the LMER ($n = 97$).}
    \label{fig:jasBondDeltaModelSlope}
  \end{figure}




%  3) more positive perceptions of joint action success will predict higher levels of social bonding, driven by more positive 3.a) perceptions of components of team performance, or 3.b) violation of team performance expectations, or 3.b) an interaction between the two predictors.




\subsection{Prediction 1.d: Team click will mediate a relationship between more positive perceptions of team performance and social bonding}


  \myparagraph{Post-Tournament}
  Results of the mediation analysis revealed significant average indirect effect of Team Performance Components on Social Bonding attributable to Team Click, $\beta = .37, 95\% CI = .20 , .59, p < .001$.  When controlling for the effect of Team Click on Social Bonding, the average direct effect between Team Performance Components and Social Bonding was no longer significant, $\beta = -.006, 95\% CI = -.27 , .23, p = .96 $ (see Figure~\ref{fig:MLM4aMediationAnalysis}). The direct effect diminished such that including Team Performance Components in the model produced a total effect that was marginally \textit{smaller} than the indirect effect alone, $\beta = .36, 95\% CI = .13 , .61, p = .01$. These results suggest that feelings of team click fully mediate the relationship between perceptions of team performance and social bonding.


  %Residuals of the mediation model were normally distributed around zero, ($W = .99, p < .28$, see Figure~\ref{fig:MLM4aAssumptions} for model assumptions).

  \begin{figure}
    \centering
    \includegraphics[width=0.9\linewidth,keepaspectratio] {images/tournamentSurveyMediationFigure}
    \caption{Prediction 1.d Mediation Model (Post-Tournament): Team Click fully mediates the relationship between Team Performance Components and Social Bonding ($n = 98$).}
    \label{fig:tournamentSurveyMediationFigure}
  \end{figure}


  \begin{figure}[htbp]
    \centering
    \includegraphics[scale = .5]{images/MLM4aMediationEffectsOffline1}
    \caption{Prediction 1.d Mediation Effects (Post-Tournament): Team Click fully mediates the relationship between Team Performance Components and Social Bonding, such that the direct effect (ADE) is no longer significant ($n = 98$)}
    \label{fig:MLM4aMediationAnalysis}
  \end{figure}


  %\subsubsection{2.4.a Mediation Analysis: $\Delta$Team Performance Components $\rightarrow$ $\Delta$Team Click $\rightarrow$ $\Delta$Social Bonding}


  \myparagraph{Pre- to Post-Tournament}
  Results from the models reported above demonstrate significant relationships between Team Performance Components Change and Team Click Change (Prediction 1.a), and Team Click Change and Social Bonding Change (Prediction 1.b).  In addition, a direct relationship between Team Performance Components Change and Social Bonding Change was also significant (Prediction 1.c).  Accordingly, a mediation analysis was performed to formally test whether a change in feelings of team click over the course of the Tournament mediated a direct relationship between change in perceptions of team performance and changes in perception of social bonding.\\

  Results of the mediation analysis revealed that the average indirect effect of Team Performance Components Change on Social Bonding Change attributable to Team Click Change was not significant, $\beta = .14, 95\% CI = -.004 , .30, p = .06$, but trended in the predicted direction (see Figure~\ref{fig:MLM24aMediationAnalysis}).  When controlling for the effect of Team Click Change on Social Bonding Change, the average direct effect between Team Performance Components Change and Social Bonding Change was, however, significant, $\beta = .27, 95\% CI = .07 , .48, p < .001$.  The total effect of the meditation was also significant ($\beta = .41, 95\% CI = .21 , .63, p < .001$; see Figure~\ref{fig:tournamentSurveyMedPrePost}).

  These results suggest a marginally-significant \textit{partial} mediation effect of Team Click Change on the relationship between Team Performance Components Change and Social Bonding Change.  This (marginally significant) result offers some support for the indirect effect observed in the Post-Tournament data.


  %While the indirect effect of Team Performance Components Change on Social Bonding Change (mediated by Team Click Change) was only marginally significant, this result does provide some support for the indirect effect observed in the Post-Tournament data.



  \begin{figure}
    \centering
    \includegraphics[width=0.9\linewidth,keepaspectratio] {images/tournamentSurveyMedPrePost}
    \caption{Prediction 1.d Mediation Model (Pre- to Post-Tournament): Team Click Change partially mediates the relationship between Team Performance Components Change and Social Bonding Change ($n = 97$).}
    \label{fig:tournamentSurveyMedPrePost}
  \end{figure}



  \begin{figure}[htbp]
    \centering
    \includegraphics[scale=.5]{images/MLM24aMediationAnalysisOffline1.pdf}
    \caption{Prediction 1.d Mediation Effects (Pre- to Post-Tournament): Team Click Change partially mediates the relationship between Team Performance Components Change and Social Bonding Change, such that the direct effect (ADE) remains significant ($n = 97$).}
    \label{fig:MLM24aMediationAnalysis}
  \end{figure}




%\subsubsection{Prediction 4.b: Team Click mediates the relationship between Team Performance Vs Expectations and Social Bonding}

%\textit{This mediation analysis was not reported due to the unreliability of the models that constitute each path.}
















\subsection{Prediction 2.a: More positive perceptions of team performance relative to prior expectations will predict higher levels of team click\label{sect:prediction1b}}

\myparagraph{Post-Tournament}
Results of the model revealed a significant positive relationship between Team Performance Vs Expectations and Team Click, $\beta = .52$ ($95\% CI =  .28, .77$), $SE = .12$, $t(14.00) = 4.24$, $p < .08$, marginal $R^2 = .40$, conditional $R^2 = .56$.  In addition, there was a significant effect of Fatigue ($\beta = .18$ ($95\% CI =  .01, .35$), $SE = .08$, $t(91.00) = 2.06$, $p = .04$), and a marginally significant effect of Final Rank ($\beta = .09$ ($95\% CI =  .0005, .18$), $SE = .05$, $t(36.34) = 1.97$, $p = .06$).  All other fixed effects did not significantly predict Team Click (see Appendix~\ref{app8:prediction2a} Table~\ref{tab:MLM1bteamExpectationsClick} for step-wise iterations of the model).

Model residuals were normally distributed around zero ($W = .98, p = .28$), and individual cases had low influence on the model (Cook's Distances all $< .2$, see Appendix~\ref{app8:prediction2a} Figure~\ref{fig:MLM1bAssumptions}).  The relationship between Team Performance Vs Expectations and Team Click is represented in Figure~\ref{fig:teamPerfClickModelSlope}.  Results support the prediction that more positive perceptions of team performance relative to prior expectations are associated with higher levels of team click.   In addition, results suggested that higher levels of Fatigue, and greater objective performance in the Tournament (Final Rank) also influenced ratings of Team Click.


  \begin{figure}[htbp]
    \centering
  \includegraphics[scale=.5]{images/teamPerfClickModelSlope.pdf}
    \caption{Prediction 2.a (Post-Tournament): Team Performance Vs Expectations predicts Team Click. The slope has been adjusted according to the predictions of the linear model ($n = 97$).}
    \label{fig:teamPerfClickModelSlope}
  \end{figure}


\myparagraph{Pre- to Post-Tournament}
The model revealed a significant positive relationship between Team Performance Vs Expectations and Team Click Change ($\beta = .30$ ($95\% CI =  .09, .51$), $SE = .10$, $t(16.80) = 2.82$, $p = .01$, marginal $R^2 = .12$, conditional $R^2 = .22$), indicating that athletes who reported more positive violations of expectations concerning their team's performance also reported higher levels of team click.  No other main effects were significant.  Model residuals were not normally distributed around zero ($W = .96, p = .003$), due to positive skew ($.85$).  Log-transformation of the outcome variable did not markedly improve the non-normality of residuals ($Shapiro-Wilk = .96, p = .008$). Exclusion of outliers according to Tukey's method improved the model fit such that residuals were normally distributed, ($Shapiro-Wilk = .99, p = .89$), and individual cases had low influence on the model (Cook's Distances all $< .10$, see Appendix~\ref{app8:prediction2a} Figure~\ref{fig:MLM21bOutAssumptions}).

%\footnote{For a justification of transformation and exclusion methods, see Appendix~\ref{app8:normality}.}

%(see Figure~\ref{fig:MLM21bAssumptions} in Appendix~\ref{app8:MLM21b})

The adjusted model supported the original significant main effect of Team Performance Vs Expectations on Team Click Change, ($\beta = .36$ ($95\% CI =  .16, .55$), $SE = .10$, $t(9.78) = 3.66$, $p < .01$, marginal $R^2 = .16$, conditional $R^2 = .55$). The model also revealed a significant \textit{negative} main effect of Individual Performance Vs Expectations on Team Click Change ($\beta = -.22$ ($95\% CI =  -.06, -.38$), $SE = .08$, $t(91.90) = -2.74$, $p = .001$), suggesting that more positive perceptions of individual performance relative to prior expectations predicted decrease in team click over the course of the Tournament.
All other main effects of the adjusted model were not significant (see Appendix~\ref{app8:prediction2a} Table~\ref{tab:MLM21bOutLogComparison} for a comparison of adjusted models).  The slope of the outlier-adjusted model is included in a graphical representation of the relationship (see Figure~\ref{fig:teamPerfClickDeltaModelSlope}).

These results generally confirmed the prediction that, on average, athletes who experienced more positive perceptions of team performance relative to prior expectations following the Tournament experienced more positive increases in Team Click.  By contrast, athlete s who on average experienced more positive expectation violation in relation to \textit{individual} performance reported less increase in Team Click.


%The adjusted model also revealed a significant negative effect of violations of expectations around \textit{individual} performance on team click, $\beta = -.008 (95\% CI =  -.01, .001), SE = .004, t(91.79) = -2.28, p = .03$,  which could suggest that athletes who experienced more positive violations about their own performance did not feel the team click as strongly.

    %
% Table created by stargazer v.5.2 by Marek Hlavac, Harvard University. E-mail: hlavac at fas.harvard.edu
% Date and time: Tue, Jun 27, 2017 - 09:21:13
\begin{table}[!htbp] \centering 
  \caption{M2.1b cTeamClick ~ cPerformanceExpectations} 
  \label{tab:MLM21bcTeamPerfExpcClick} 
\begin{tabular}{@{\extracolsep{5pt}}lccc} 
\\[-1.8ex]\hline 
\hline \\[-1.8ex] 
 & \multicolumn{3}{c}{\textit{Dependent variable:}} \\ 
\cline{2-4} 
\\[-1.8ex] & \multicolumn{3}{c}{cTeamClick} \\ 
\\[-1.8ex] & (1) & (2) & (3)\\ 
\hline \\[-1.8ex] 
 (constant) & 0.10 & $-$0.57 & $-$0.64 \\ 
  & (0.13) & (0.30) & (0.44) \\ 
  & & & \\ 
 cTeamPerformanceExpectations &  & 0.01$^{*}$ & 0.01$^{*}$ \\ 
  &  & (0.004) & (0.005) \\ 
  & & & \\ 
 cIndPerformanceExpectations &  &  & $-$0.003 \\ 
  &  &  & (0.004) \\ 
  & & & \\ 
 objectiveCompetence &  &  & $-$0.13 \\ 
  &  &  & (0.11) \\ 
  & & & \\ 
 subjectiveCompetence &  &  & $-$0.16 \\ 
  &  &  & (0.09) \\ 
  & & & \\ 
 finalRank &  &  & 0.02 \\ 
  &  &  & (0.06) \\ 
  & & & \\ 
 minutesTotal &  &  & 0.003 \\ 
  &  &  & (0.005) \\ 
  & & & \\ 
 pointsTotal &  &  & 0.003 \\ 
  &  &  & (0.01) \\ 
  & & & \\ 
\hline \\[-1.8ex] 
Marginal R-squared &  &  & .40 \\ 
Conditional R-squared &  &  & .47 \\ 
Observations & 99 & 99 & 97 \\ 
Log Likelihood & $-$130.59 & $-$127.00 & $-$123.15 \\ 
Akaike Inf. Crit. & 267.19 & 266.01 & 270.30 \\ 
Bayesian Inf. Crit. & 274.97 & 281.58 & 301.20 \\ 
\hline 
\hline \\[-1.8ex] 
\textit{Note:}  & \multicolumn{3}{r}{$^{*}$p$<$0.05; $^{**}$p$<$0.01; $^{***}$p$<$0.001} \\ 
\end{tabular} 
\end{table} 

    %
% Table created by stargazer v.5.2 by Marek Hlavac, Harvard University. E-mail: hlavac at fas.harvard.edu
% Date and time: Tue, Jun 27, 2017 - 09:21:15
\begin{table}[!htbp] \centering 
  \caption{M2.1b cTeamClick ~ cPerformanceExpectations (adjusted models)} 
  \label{tab:MLM21bOutLogComparison} 
\begin{tabular}{@{\extracolsep{5pt}}lcc} 
\\[-1.8ex]\hline 
\hline \\[-1.8ex] 
 & \multicolumn{2}{c}{\textit{Dependent variable:}} \\ 
\cline{2-3} 
\\[-1.8ex] & cTeamClick & clickFactorChangePrePostOut \\ 
 & log-transformed & outliers removed \\ 
\\[-1.8ex] & (1) & (2)\\ 
\hline \\[-1.8ex] 
 (constant) & 0.91$^{***}$ & $-$0.14 \\ 
  & (0.15) & (0.34) \\ 
  & & \\ 
 cTeamPerformanceExpectations & 0.004$^{*}$ & 0.02$^{***}$ \\ 
  & (0.002) & (0.004) \\ 
  & & \\ 
 cIndPerformanceExpectations & $-$0.001 & $-$0.01$^{*}$ \\ 
  & (0.001) & (0.004) \\ 
  & & \\ 
 objectiveCompetence & $-$0.04 & 0.07 \\ 
  & (0.04) & (0.09) \\ 
  & & \\ 
 subjectiveCompetence & $-$0.05 & $-$0.02 \\ 
  & (0.03) & (0.08) \\ 
  & & \\ 
 finalRank & 0.003 & $-$0.08 \\ 
  & (0.02) & (0.04) \\ 
  & & \\ 
 minutesTotal & 0.001 & 0.001 \\ 
  & (0.002) & (0.004) \\ 
  & & \\ 
 pointsTotal & 0.001 & 0.003 \\ 
  & (0.002) & (0.01) \\ 
  & & \\ 
\hline \\[-1.8ex] 
Marginal R-squared & .14 & .17 \\ 
Conditional R-squared & .25 & .20 \\ 
Observations & 97 & 93 \\ 
Log Likelihood & $-$12.38 & $-$98.81 \\ 
Akaike Inf. Crit. & 48.75 & 221.62 \\ 
Bayesian Inf. Crit. & 79.65 & 252.01 \\ 
\hline 
\hline \\[-1.8ex] 
\textit{Note:}  & \multicolumn{2}{r}{$^{*}$p$<$0.05; $^{**}$p$<$0.01; $^{***}$p$<$0.001} \\ 
\end{tabular} 
\end{table} 



    \begin{figure}[htbp]
      \centering
    \includegraphics[scale=.5]{images/teamPerfClickDeltaModelSlope.pdf}
      \caption{Prediction 2.a (Pre- to Post-Tournament): Team Performance Vs Expectations predicts Team Click Change. The slope reflects the parameter estimates of the LMER model ($n = 97$).}
      \label{fig:teamPerfClickDeltaModelSlope}
    \end{figure}


  \myparagraph{Overall Tournament}
Analysis of the Overall Tournament subset confirmed the prediction.
The model revealed a significant relationship between Team Performance Vs Expectations and Team Click ($\beta = .72$ ($95\% CI = .62, .80$), $SE = .05$, $t(185.29) = 16.06$, $p < .01$, marginal $R^2 = .63$, conditional $R^2 = .69$).
The model also indicated that Ind Performance Vs Expectations significantly predicted Team Click, $\beta = .12$ ($95\% CI =  .04, .20$), $SE = .04$, $t(294.70) = 2.83$, $p < .01$ (all other main effects were not significant (see Appendix~\ref{app8:prediction2a} Table~\ref{tab:MLM31ateamPerfClickTournament} for a full description of results).  The relationship between Team Performance Vs Expectations  and Team Click is represented graphically in Figure~\ref{fig:teamPerfClickOverallModelSlope}.

The residuals of the model were normally distributed around zero, ($W = .99, p = .11$), and individual cases had low influence on the model (Cook's Distances all $< .05$; see Appendix~\ref{app8:tournamentSurvey} Figure~\ref{fig:MLM31aAssumptions} for a full report of model assumptions).  Results of the model suggest that athletes whose expectations around team performance were more positively violated also experienced stronger feelings of team click.  The model also revealed main effects of Individual Performance Vs Expectation, which suggests that perceptions of individual performance may also play contribute to producing perceptions of team click.

   \begin{figure}[htbp]
     \centering
   \includegraphics[scale=.5]{images/teamPerfClickOverallModelSlope.pdf}
     \caption{Prediction 2.a (Overall Tournament): Team Performance Vs Expectations predicts Team Click.  The slope reflects the parameter estimates of the LMER model ($n = 306$).}
     \label{fig:teamPerfClickOverallModelSlope}
   \end{figure}


%\myparagraph{Prediction 2.a: Summary of results}
Results from each subset supported the prediction that more positive expectation violation around team performance is associated with team click.  There was some evidence to suggest that more positive violations of expectations concerning individual performance and (self-reported) technical competence may also predict team click (Overall Tournament).  Results also showed an opposite effect, whereby higher levels of expectation violation concerning individual performance predicted lower levels of team click (Pre- to Post-Tournament).




\subsection{Prediction 2.b: More positive violations of team performance expectations predicts higher social bonding}

\myparagraph{Post-Tournament}
The model revealed a significant positive relationship between Team Performance Vs Expectations and Social Bonding, $\beta = .35$ ($95\% CI = .06, .64$), $SE = .14$, $t(12.80) = 2.39$, $p = .03$, marginal $R^2 = .20$, conditional $R^2 = .40$.  The model also revealed a significant positive effect of Subjective Competence on Social Bonding, $\beta = .19$ ($95\% CI =  .02, .36$), $SE = .08$, $t(90.37) = 2.24$, $p = .03$, such that athletes who provided higher ratings of their own technical competence in rugby (measured before the Tournament) reported higher levels of social bonding.

Model residuals were not normally distributed, ($W = .91, p < .01$), owing to large negative skew ($-1.33$) and high kurtosis ($.49$). Re-running the model with a log-transformed outcome variable appeared made the best improvement to the distribution of model residuals (compared to an outlier-removed model), but residuals of the log-transformed model were non-normally distributed ($W = .97, p = .04$).  Owing to non-normally distributed residuals, this model does not provide robust support for the prediction that team performance expectation violations relates directly to social bonding (see Appendix~\ref{app8:prediction2b} Figures~\ref{fig:MLM3bAssumptions} and~\ref{fig:MLM3bLogAssumptions} for model assumptions).

%(see Appendix~\ref{app8:tournamentSurvey} Section~\ref{app8:MLM3b} Table~\ref{tab:MLM3bExpectationsBonding}).


%  the distribution of model residuals appeared most normal and individual cases had low influence on the model (Cook's Distances all $< .15$, see Figures ~\ref{fig:MLM3bAssumptions} and ~\ref{fig:MLM3bLogAssumptions} in Appendix~\ref{app8:MLM3b} for a comparison of model assumptions between the original and log-transformed model).

  \myparagraph{Pre- to Post-Tournament}
  %\subsubsection{2.3.b $\Delta$Social Bonding $\sim$ Team Performance Vs Expectations}
  A model revealed a marginally significant main effect of Team Performance Vs Expectations on Social Bonding Change $\beta = .20$ ($95\% CI =  -.0004, .02$), $SE = .11$, $t(97) = 1.87$, $p = .07$, marginal $R^2 = .05$, conditional $R^2 = .05$.  Examination of model residuals revealed that they were not normally distributed around zero ($Shapiro-Wilk = .96, p = .008$).  Re-running the model with a log-transformed outcome variable and outliers excluded improved the normality of residuals ($Shapiro-Wilk = .98, p = .24$).
  However the effect of Team Performance Vs Expectations on Social Bonding was no longer significant, $\beta = .04$ ($95\% CI =  .02, .10$), $SE = .03$, $t(95) = 1.29$, $p = .10$, marginal $R^2 = .03$, conditional $R^2 = .09$.  These results suggest that perceptions of team relative to prior expectations between Pre- and Post-Tournament measurements did not suitably explain observed variation in Social Bonding Change (see Appendix~\ref{app8:prediction2b} Figure~\ref{fig:MLM3bLogAssumptions} for model assumptions).



  \myparagraph{Overall Tournament}
  The initial model failed to converge with the random slope and intercept model structure.  As such, the model was simplified to estimate only the random slope. This simplified model revealed a significant positive relationship between Team Performance Vs Expectations and Social Bonding, $\beta = .47$ ($95\% CI =  .53, .74$), $SE = .06$, $t(306.00) = 8.26$, $p < .01$, marginal $R^2 = .40$, conditional $R^2 = .40$.  The model also revealed a significant relationship between Individual Performance Vs Expectations and Social Bonding, $\beta = .25$ ($95\% CI =  .002, .009$), $SE = .06$, $t(306) = 4.36$, $p < .01$.  Model residuals were non-normally distributed, ($W = .97, p < .01$), with negative skew ($-.60$), and higher than normal kurtosis ($> 2$).  A model in which the outcome was log-transformed following removal of outliers provided the best possible fit for the available data (see Appendix~\ref{app8:prediction2b} Figure~\ref{fig:MLM32aLogAssumptions}).  While the distribution of errors was still non-normal following transformation ($W = .99, p = .03$),  error terms appeared much more evenly distributed around zero than the original model, albeit with a slight negative skew ($-.10$).
  The adjusted model was suggestive of a relationship, but was not taken as confirmation for Prediction 2.b, due to irreconcilable violation of model assumptions.

  %See Table~\ref{tab:MLM32ateamPerfBondingTournament} in Appendix~\ref{app8:tournamentSurvey} for full description of model estimates).
  %provided confirmation that the significant positive effect of Team Performance Vs Expectation on Social Bonding ($\beta = .07$ ($95\% CI =  .12, .20$), $SE = .02$, $t(336.80) = 7.98$, $p < .01$, marginal $R^2 = .36$, conditional $R^2 = .36$).


   \begin{figure}[htbp]
     \centering
   \includegraphics[scale=.5]{images/teamPerfBondOverallModelSlope.pdf}
     \caption{Prediction 2.b (Overall Tournament): Team Performance Vs Expectations predict Social Bonding in the  data. The line of best fit reflects the parameter estimates of the adjusted LMER model ($n = 271$). \textit{Note: the assumption of normality of residuals was violated and figure is presented for reference only}.}
     \label{fig:teamPerfBondOverallModelSlope}
   \end{figure}


In sum, athlete perceptions of team performance relative to prior expectations did not predict athlete perceptions of social bonding during the Tournament.
















\clearpage

\section{Discussion\label{sect:tournamentDiscussion}}
The results of this study suggest that team click mediates a relationship between joint action and social bonding in group exercise.  This relationship was substantiated by findings that, in a high-stakes national rugby tournament in which levels of uncertainty in joint action were assumed to be high, athletes who reported more positive perceptions of team performance were more likely to perceive higher levels of team click (Prediction 1.a), and athletes who perceived higher levels of click in joint action perceived higher levels of social bonding (Prediction 1b).  A multi-level mediation analysis using the Post-Tournament data set showed that the significant direct relationship between perceptions of Team Performance Components and Social Bonding (Prediction 1.c) was fully mediated by Team Click. In addition, a marginally significant partial mediation was shown in the Pre- to Post-Tournament data set.  These results represent the first empirical substantiation of a general theory of team click in a real-world group exercise setting.

%Considering the relatively small variation in change in team click Pre- to Post-Tournament, to observe a marginally significant indirect effect of Team Performance Components on Social Bonding (mediated by team-click) in the Pre-Post-Tournament data is noteworthy.

Results also suggested that positive violations of expectations surrounding team performance may be an important precedent for the visceral and socially agentic experiences associated with team click (Prediction 2.a).  Results did not support the predicted direct relationship between expectation violation and social bonding (Prediction 2.b), which indicated that while expectation violation in joint action might be an important factor in drawing attention to visceral and socially agentic perceptions of team click \citep{Chetverikov2014}, it might not be a strong enough mechanism to directly drive associations between these perceptions and more elaborate processes characteristic of social bonding, such as perceived emotional support and common goal with teammates. This result indicated the need for further analysis of positive expectation violation in a controlled experimental setting.

%Precise details concerning this mechanism  mechanisms of the continuum of physical and social coordinative mechanism hypothesised by the AIF remain largely unknown.

In addition to predicted main effects, models on occasion revealed significant effects of moderator variables of interest.  Fatigue positively predicted variation in Team Click (in a model of the relationship between Team Performance Vs Expectations and Team Click using the Post Tournament data; Prediction 2.a).  Ethnographic evidence suggested that fatigue may have had a \textit{negative} impact on athletes' ability to mitigate the uncertainties of on-field joint action.  Under fatigue, athletes reported a lack of on-field awareness (\textit{yishi}), which, contrary to results reported here, could suggest that fatigue has a negative impact on perceptions of team click.

However, research has proposed that physiological exertion in exercise could also be associated with experiences of euphoria (e.g., ``flow'' and the ``runner's high'') owing to down-regulation of cortical processes \citep{Dietrich2011}, and exercise-specific neuropharmacological reward \citep{Raichlen2013}.  \mccorrect{Dietrich and colleagues propose that cortical structures implicated in decision-making and self-monitoring could at times be detrimental to achieving dynamical coordination in action due to their belonging to a subsystem of cognitive function} \citep[i.e., a deliberate ``system 2'' as opposed to more automatic and implicit ``system 1''; see ][]{Kahneman2011} characterised by slower and more deliberate executive function, rather than in-the-moment movement control.  By contrast, viewed from the perspective of the AIF, fatigue-related strain on cognitive processes would likely reduce the reliability (precision-weighting) of (more uncertain/less reliable) exteroceptive inputs, favouring instead the more proximate and reliably interoceptive information \citep{Pezzulo2014}.  Thus, these results could be interpreted as being commensurate with the claim advanced in this thesis, that the visceral experience of team click is generated in environments in which interoceptive information is the most reliable available source for inference.

In the same model, the objective performance measure of Final Rank also predicted variation in team click.  This result was theoretically expected: better performing teams would necessarily achieve higher levels of coordination in response to higher levels of uncertainty in joint action.  However, this result also signalled a methodological limitation.  In a high-stakes Tournament, athletes are provided with explicit feedback about joint action, in the form of winning and losing each game, scoring points, or feedback from coaches.   Although some of these factors were addressed through statistical controls, it could be expected that more explicit and external information available in the environment would have informed athlete perceptions of team performance, team click, and social bonding. From a theoretical perspective (i.e., based on claims of the AIF), it was possible that athlete perceptions of team click were influenced by affordances less grounded in individual sensorimotor experiences, and instead more contingent socially- (rather than sensorily-) mediated sources of information \citep{Ramstead2016}.  Thus, this result represented a potential problematic confound to the proposed causal relationship between joint action, team click, and social bonding.  In essence, athletes may have reported higher levels of team click because they thought they ought to report such perceptions as a member of a better performing team. As I discuss below, this issue should be addressed through controlled experimental design.

The results of this study showed that, in general, perceptions of team performance (relative to perceptions of individual performance) were more powerful as predictors of team click and social bonding.  However, on two occasions athlete perceptions of individual performance relative to prior expectations predicted variance in team click. These results were conflicting and thus difficult to interpret.  In the Pre- to Post- Tournament analysis, a log-adjusted model revealed a \textit{negative} main effect of Individual Performance Vs Expectations on Team Click (see Section~\ref{sect:prediction1b}), which suggested that individuals who felt higher levels of positive expectation violation concerning individual performance perceived lower levels of team click (measured as a function of change between Pre- and Post-Tournament surveys).  Meanwhile, the Overall Tournament analysis revealed an opposite effect: a significant positive relationship between Individual Performance Vs Expectations and Team Click (see Overall Tournament results in Section~\ref{sect:prediction1b}). This result suggested that athletes who experienced more positive violations conerning individual performance also reported higher levels of team click.

Considered together, these results could suggest the co-presence of two opposing modes of joint action evaluation \citep{Friston2015a}, one in which individuals tend to perceive more positive violations of individual performance expectations as distinct from team click, and a mode in which more positive violations of expectations surrounding individual performance may enhance perceptions of team performance.

It is possible to imagine real-world scenarios in which athletes feel satisfied or surprised about their individual contributions to joint action, but nonetheless do not perceive that the team clicked.  Consider Beijing senior player Lu Peng, for example, who was often critical of team performance and avoided admission of individual deficiencies.  Likewise, it is also imaginable that more positive perceptions of individual performance relative to expectations could have an additive effect on experience of team click and social bonding.  Dynamic interactive joint action can involve a blurring of boundaries between self and others and, thus cognitions and perceptions attributed to individual performance and team performance could share common variance \citep{Friston2015}.  These results could be interpreted as suggesting that the ``we-mode'' of cognition characteristic of team click also contains a strong ``I-mode'' as its sensorimotor foundation \citep{Gallotti2013,VanderWel2015}.  Variation in the differential recruitment of these modes could be explained by individual variation in tendencies for movement in joint action \citep{Keller2014}.


%In the scenario in which athletes perceived that rated individual performance higher and team click comparatively lower,
%performance were associated with more positive perceptions of team click, it could be expected that athletes with higher objective and subjective levels of technical competence
 %to exceed others in the team (or Tournament) would also report lower levels of team click.  However, in the model in which perceptions of individual performance predicted an increase in team click (Prediction 1.a, Overall Tournament data), subjective performance also contributed to this positive increase.
%This result may have been due to the on-average lower levels of team click reported in the Mid-Tournament results (see Table ~\ref{tab:tab:tournamentSurveyItemsTime}).  Athletes were more critical


Subjective and objective measures of technical competence were included in analyses based on theoretical and ethnographic evidence to suggest that familiarity with technical components of joint action may condition the relationship between perceptions of joint action, team click, and social bonding.  In this study, there was no clear evidence that Objective Competence (training age, years in team, age) significantly altered predicted relationships of interest.  Subjective Competence significantly predicted Social Bonding in one model (prediction 1.c), but these results were not consistent in relationships between all variables of interest.

It is possible that these measures failed to access the (largely pre-perceptual) mechanisms of prediction error management hypothesised to underpin cognition and affective dispositions in joint action \citep{Clark2013}.  It is also possible that the nature of the intensity and consequentiality of the Tournament in which athletes were participating was high enough that no one---experts athletes included---was immune to the uncertainty and of joint action.  Indeed, the ability of competitive group exercise contests to consistently arouse high levels of psychological stress and uncertainty could be an important reason for their cultural evolutionary success over recent centuries \citep{Dunbar2010,Whitehouse2004}.

Self-reported measures of technical competence (Subjective Competence---ability relative to teammates, Chinese opponents, or international professionals)—did significantly
predict Team Click (when included as a control variable in a model of the relationship between Team Performance Vs Expectations and Team Click in the Overall Tournament data, Prediction 2.a), and Social Bonding (in a model of the direct relationship between Team Performance Components and Social Bonding in the Post-Tournament data, Prediction 1.c).  These results suggested that more technically competent athletes (according to subjective self report) may have experienced higher levels of team click and social bonding, owing to more precise ability to achieve dynamical coordination in joint action, and therefore experience the visceral and socially agentic experiences associated with this achievement in instances in which joint action was successful.  It was possible that athletes who were willing to self-report higher levels of technical competence may also have been more willing to report higher levels of other desirable variables such as team click and social bonding. Thus, these results should be treated with caution.

In addition to technical competence, a measure of extraversion was included in models to control for the possibility that preexisting dispositional tendencies could impact on processes of team click and social bonding \citep{Marsh2009,VonRueden2015}.  In particular, it is believed that extraversion may be associated with a propensity to seek out opportunities for coordination \citep{Richardson2007a}, and to attribute agency in these interactions to sources beyond the self \citep[also known as external locus of control; see][]{Morris1981}. For example, \textcite{Richardson2007a} found that extraversion positively correlated with willingness to sustain cooperation in joint action in a dyadic plank moving task.  In the present study, results revealed no significant effects of extraversion on variables of interest.

The uncontrolled and \textit{in-situ} design of this study posed many challenges to causal inference.  To address this issue, theoretically motivated predictions were tested using 3 subsets of data, which included multiple observations per individual athlete, thus increasing the amount and reliability of evidence with which to assess study predictions. In addition, linear mixed-effects regression models allowed for clustering of observations (according to individual or team) could be factored into account. Nonetheless, results should be treated as preliminary and suggestive only.

Hypotheses pertaining to a general account of team click in group exercise thus require further attention via a controlled experimental designs.  An experiment in which perceptions of team performance and expectation violation were manipulated \mccorrect{as a function of uncertainty}, and explicit feedback around performance was eliminated, could allow for the assessment of the role of the cognitive processes of movement coordination in joint action in generating team click and social bonding.  These considerations were factored into the design of experimental study, which is presented in the next chapter.
                                                \end{CJK}
