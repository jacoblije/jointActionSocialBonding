Humans come together and move together.  Amidst all of the scientific questions that we ask of ourselves (and of biological life more broadly), how and why we move together are two fundamental questions that have evaded direct and rigorous consideration, until recently.  The key assertion of this dissertation is that a scientific explanation of the prevalence of physiologically exertive and socially coordinated movement (group exercise) in human sociality would benefit from greater attention to the component mechanisms of joint action (defined herein as the dynamic coordination of behaviour between two or more co-actors).  I propose novel theoretical synthesis and conduct focussed empirical research to explore and substantiate this assertion.

The ubiquity of group exercise suggests a relationship between group exercise and social cohesion. Existing evidence supports a ``social high'' theory of group exercise and social bonding, whereby sustained moderate intensity physiological exertion and exact behavioural synchrony combine to generate a psychophysiological environment conducive to affiliation and trust.   However, anecdotal and ethnographic evidence exposes a broader spectrum of profiles and experiences in group exercise, many of which involve extreme physiological cost, complex coordination, as well as rich social and cultural meaning-making.   I identify the emerging ``active inference'' framework \citep{Friston2010} as the most suitable theory for explaining the interlocking physical, cognitive, social, and cultural complexity of these instances of group exercise.

I focus specifically on the experience of ``team click'' in complex multi-agent joint action typical of group exercise contexts.  Team click is a theoretically grounded construct that pertains to the subjective feeling associated with optimal interpersonal coordination.  I propose a novel theory of social bonding through joint action, mediated by team click.

To evaluate this theory empirically, I conducted research with professional Chinese athletes who participate in the competitive interactional team sport of rugby union.  I utilised ethnographic and field-experimental methods in order to examine the specific relationship between joint action and social bonding.

Results provide evidence for a relationship between athletes' perceptions of success in joint action and social bonding, mediated by the phenomenon of ``team click.''  On average, athletes who report more positive perceptions of team performance report higher levels of perceived team click.  In turn, higher levels of team click predict higher levels of social bonding. Finally, mediation analyses show that a direct relationship between joint action and social bonding is mediated by team click. Results reported here suggest that the experience of positive violation of expectations of team performance may be at the heart of the affective, agentic, and visceral experience of team click, and may set the foundation for processes of social bonding.

Considered in light of existing debates on the process mechanisms of social cohesion, these results suggest that processes of psychophysiological alignment achieved through interpersonal movement regulation and coordination could be central to the formation of durable social bonds and the transmission of cultural practices between individuals and within groups.  This dissertation therefore contributes to an evolutionary explanation for group exercise in human sociality.
