



\begin{savequote}[8cm]

  I do not see any way to avoid the problem of coordination and still understand the physical basis of life.
  \qauthor{--- H. H. Pattee  \textit{The role of instabilities in the evolution of control hierarchies} 1976}

\end{savequote}



\chapter{\label{chap:intro}Introduction}



\minitoc





                                          \begin{CJK}{UTF8}{gbsn}

\section{Group exercise as carnal mystery\label{sect:adrian}}


On the first night that I arrived in Beijing for research in August 2015, Adrian, Kai, and I sat in the upstairs area of the Korean BBQ restaurant in a quiet willow lined street just inside Beijing's East 4th Ring Road.  Adrian was the host of the dinner, and so he naturally held the floor in conversation while the three of us waited for his colleague Mr Shi to arrive.\footnote{I use real names for participants unless otherwise stated.}

Adrian was a veritable elder of Chinese rugby union (hereafter simply rugby).  He had been captain of the second class of rugby players to graduate from the Chinese Agricultural University (CAU), the home of China's first official rugby union program, established in 1990.  Kai---also a CAU graduate (2007)---was a close friend of mine of many years, and had invited me to join at the dinner soon after I touched down earlier that day.  I was eager to catch up with old friends as well as begin my fieldwork, and so, despite my jet lag, I accepted the invitation.

Adrian reminisced fondly about his time playing rugby at CAU, as well as his time after graduation playing with the Beijing Devils, a rugby club in Beijing whose members were predominantly expats.  He assured us that rugby in China was, in those days, fun and free-spirited.  Not like today, now that rugby has become a professional program in the state-sponsored sport system.  Adrian talked about going on a rugby tour to the UK with the Beijing Devils:  ``Everyone only just scraped together the money to go on tour.  We all paid our own way. Sometimes you'd get a bit of help from someone or whatever. We did it because we loved the game, not for any other reason.''   Kai and I listened intently.  All of a sudden I realised that this conversation could be relevant, so I started taking notes.

%(owing to the Olympic status of the modified form of the game, rugby sevens, see Chapter~\ref{chap:researchSetting})

When Mr Shi finally arrived, Adrian continued in his nostalgic mode, but naturally shifted his target audience from Kai and me to his guest.  Adrian began to describe in rich detail the experience of camaraderie between his Beijing Devils team mates when they participated in a overseas rugby tours.  At one point Adrian interrupted his own story to make an explanatory aside directed at Mr Shi, accommodating for the fact that Mr Shi was relatively unacquainted with the sport: ``This sport, rugby: it's actually very mysterious. If you haven't played it yourself you might not know this type of feeling,'' (英式橄榄球这个项目其实特别神秘,没玩过的话您可能不知道这种感觉) Adrian respectfully suggested to Mr Shi.  ``Because rugby, you know, you're all on the field together, there's body contact...'' (因为英式橄榄球么,大家在场上有身体接触) he paused to find the right phrasing,  ``It's a very \textit{carnal} type of feeling'' (是一种``肉''的感觉).  His attempts to enrich his communication by gesticulating had led him to have both of his hands clenched as fists in front of him like they were cradling a rugby ball or gripping the steering wheel of a car---a lit cigarette smouldering between the index and middle finger of his right hand.  Adrian concluded by looking into the distance and repeating: ``Its very mysterious.'' (特别神秘) He shook his head as if baffled and finally released his clenched fists to dab the ash from his cigarette into the ashtray in front of him.  After taking another drag from his cigarette he finally added: ``So it means this rugby circle here in China is very tight...'' (所以在橄榄球这个圈子特别亲)---a short pause for another dab of his cigarette--- ``...but it doesn't mean that this circle is not also complete chaos!'' (但这不是说这个圈儿也不乱!)
  \footnote{Circle (\textit{quanzi} 圈子) is a common colloquial way to refer to a social group or community of people in modern standard Chinese.}
The wisdom of Adrians's final punchline was confirmed with a knowing chuckle from all of us, including Mr Shi. Adrian concluded his performance silently, by taking a long, satisfying drag of his cigarette and looking off into the most distant corner of the restaurant.

I was captivated---but also somewhat surprised---by Adrian's monologue.  I was not expecting, so early into my fieldwork, to happen upon a declaration in which a link between the carnal (\textit{rou} 肉) or visceral sensations associated with on-field joint action, and social processes of interpersonal affiliation (\textit{qin} 亲) and group cohesion of the rugby community (\textit{quanzi} 圈子) was so explicitly and spontaneously emphasised.  It was clear that rugby's visceral dimension continued to capture Adrian emotionally; some 10 years after he had finished playing rugby his fists still clenched with energy, and his head still shook with amazement.

I was also intrigued that Adrian cited the source of his emotional capture as at once both very specific (derived from playing together with others on the field) and, at the same time, ultimately ``mysterious.''  The aim of my research was to contribute to an explanation the human behavioural phenomenon of group exercise in terms of its social, evolutionary, cognitive, and physiological causal processes and dynamics.  In essence, the aim is to somehow move from mystery to scientific mechanism.  At this first dinner in Beijing, Adrian's comments both captured the phenomenological mystery of group exercise, and pointed me in the direction of the underlying explanatory mechanisms.

Needless to say, I left that first dinner eager to investigate the sources of Adrian's experience of mysterious carnality and social connection in rugby's joint action.  My next stop, on Monday morning, was the Temple of God of Agriculture Sports Institute, where I began ethnographic research with the Beijing Provincial rugby program.


                            \begin{center}

                            \end{center}





\section{Group exercise as evolutionary puzzle}
Competitive team sports, whirling Sufi dervishes, late-night electronic music raves, Masi ceremonial dances, or the fitness cults of Cross-fit and Soul Cycle---endless examples can be plucked from across cultures and throughout time to exemplify the human compulsion to come together and move together.  How can we explain the prevalence of these activities in the human record?  In this dissertation I contribute to a scientific understanding of ``group exercise''---defined herein as physiologically exertive and socially coordinated movement---by way of a focussed study of the social cognition of joint action among professional Chinese rugby players.

Because physical movement is a metabolically expensive task, it is justifiable from an evolutionary standpoint only if the benefits somehow outweigh the costs.  Using this basic calculus, it is easy to imagine how group exercise would have served important survival functions in our ancestral past.  Activities involving group exercise such as hunting, travel, communication, and defence all appear to confer immediate and obvious benefits to individuals and groups \citep{Sands2010}.

In more recent domains of human history, however, the task of explaining the persistent recurrence of group exercise is more complicated.  At least since the late Pleistocene era (approx. 500ka), and particularly since the Holocene transition from hunter-gatherer to agricultural (approx. 11ka), and later industrial and post-industrial societies, group exercise can be identified in shared cultural practices as varied as religion, organised warfare, music, dance, play, and sport.  But unlike group hunting or defence, the fitness-relevant benefits of group exercise in cultural practices such as sport, music, or dance are not always as obvious.  On the contrary, many of these activities appear on the face of it to entail extreme time and energy costs for very little immediate reward.

Thus, the prevalence of group exercise in a diverse array of shared cultural practices in the more recent human record presents an evolutionary puzzle.  A solution to this puzzle requires a more nuanced calculus that incorporates an appreciation of humans' species-unique evolutionary trajectory, defined by increasingly complex cognitive and cultural capacities, including technical manipulation of extra-somatic materials and ecologies; advanced theory of mind; and information-rich, malleable, and scalable communication systems \citep{Roepstorff2010,Clark2015,Fuentes2016}.  A theory capable of satisfactorily explaining group exercise within humans' distinctive evolutionary parameters is yet to be fully formulated \citep{Cohen2017}.

As Adrian's monologue demonstrates, the experience of group exercise appears to be at once a physical and social phenomenon.  The fact that these dimensions appear to coalesce in experiences of group exercise suggests that cultural activities in which group exercise features may have played an important role in human-specific processes of social cohesion \citep{Dunbar2010,Whitehouse2004,Cohen2017}.  In  this dissertation, I advance existing cognitive and evolutionary understandings of group exercise by formulating and testing a novel theoretical relationship between group exercise and the glue of group cohesion: social bonding.

Existing theories of group exercise and social bonding do not yet satisfactorily account for the variation in, and complexity of, interpersonal movement coordination common to many real-world settings of group exercise.  I concentrate in particular on the proximate cognitive mechanisms  ``joint action,'' defined herein as any form of social interaction whereby two or more individuals coordinate their actions in space and time to bring about a change in the environment \citep{Sebanz2006}.  I draw attention to the bio-psycho-social effects known to occur when active, in-the-moment, and on-line joint action functions successfully---i.e., when joint action ``clicks'' between co-actors.  I propose the phenomenon of ``team click'' as a construct that captures the phenomenology of optimal performance in dynamic joint action.  More importantly, I propose team click as a psychological construct that can help explain a hitherto under-appreciated causal pathway between uncertainty of joint action and social bonding.  Team click offers a vehicle through which the various dynamical interlocking dimensions of experience in group exercise can be analysed in terms of their cognitive and evolutionary significance.

%As I demonstrate in this introduction, novel theoretical synthesis---in addition to empirical research---is required in order to more fully align scientific explanations for group exercise with their mysterious, carnal, and embodied (i.e., subjective) dimensions.

In the chapters that follow, I formulate and test a general account of team click in group exercise, which I situate and test in the group exercise context of rugby in China.  I conduct a series of three empirical studies with professional Chinese rugby players.  These studies include 1) an ethnographic study of the Beijing men's rugby team (Chapters~\ref{chap:ethnoSetting}\nobreakdash~\ref{chap:ethnoResults}), 2) an \textit{in situ} survey study of professional Chinese rugby players during a National rugby tournament (Chapter~\ref{chap:tournamentSurvey}), and 3) a controlled field experiment with a sample of professional Chinese rugby players recruited from two of China's provincial rugby programs (Chapter~\ref{chap:trainingExperiment}).  In each of these studies I find evidence to support the research hypotheses and specific predictions of this thesis.  Specifically, more positive perceptions of success in team performance predict higher levels of team click; higher levels of team click predict higher levels of social bonding, and in some instances, team click mediates---either fully or partially---a direct positive relationship between perceptions of success in team performance and social bonding.

This thesis progresses in a step-wise manner: I start with broad and rich ethnographic observation and analysis, and build on these observations towards a more quantitative verification of hypothesised mechanisms.  Findings from each study offer initial substantiation of a general account of team click in group exercise.

%In so doing, this dissertation sheds new light cognitive and evolutionary processes in human behaviour, psychology, and sociality.

In what remains of this chapter, I review existing theories of group exercise and social cohesion (Section~\ref{sect:GESoCo}), and point to empirical and theoretical knowledge gaps that require attention (Section~\ref{sect:empKnowGaps}).  In particular, I identify the under-theorised relationship between successful performance in complex joint action and perceptions of team click.  I conclude by previewing a general account of team click and the main empirical components of this dissertation (Section~\ref{sect:components}).

%and conclude with an outline of the chapters of the dissertation (Section~\ref{sect:chapters}).


\section{Existing scientific explanations of group exercise\label{sect:GESoCo}}

In this section I outline existing cognitive and evolutionary accounts of group exercise.  In this dissertation, I define group exercise broadly as any activity that minimally entails 1) sustained physical exercise \citep[i.e., structured physical activity that reaches at least low intensity physiological exertion of 45\% of max heart rate or above; see][]{Caspersen1985}, and 2) coordination of behaviour between two or more individuals in time and space to bring about change in the environment \citep[a.k.a., joint action, see][]{Sebanz2006,Vesper2010}.  A broad definition of this nature allows for the identification of instances of group exercise in a wide array of contexts throughout the human record, from music making and dance, to ritual practices, to competitive sport, warfare, play, and even pair-bonding activities such as sexual intercourse.  Anecdotal and ethnographic evidence pertaining to humans' subjective experience of all of these activities suggests that group exercise commonly entails a prominent and unmistakable visceral dimension. But, as I explain below, the natural scientific origins of this dimension of experience have remained unclear.


\subsection{Evolutionary origins of group exercise}
Archeological and primatological evidence suggests that human's capacity for group exercise runs deep into our evolutionary trajectory.  Humans have evolved distinct morphological and physiological adaptations that enable sustained aerobic exercise.  Adaptations include derived skeletal features that support stable bipedal running and respiratory function \citep[see ][]{Bramble2004}, as well as exercise-specific neurobiological reward \citep{Raichlen2012}.  The archeological record indicates that a capacity for sustained physical exercise may have emerged at some point in the transition from Pan (i.e., Australopithecus) to Homo, and could be associated with activities such as persistence hunting, defence of territory, and communication over vast distances such as those conceivable in woodland or grassland savanna, as opposed to dense canopy rainforests of pre-human ancestors \citep{Sands2012}.\footnote{While all species of the Homo genus (Sapiens, Erectus, and Habilis) appear to show evidence of morphological adaptation for running, none of these species appear to  be natural sprinters.  Even elite human sprinters (capable of sustaining maximum speeds of only 10.2 meters per second for less than 15 seconds) are slow compared to mammalian cursorial specialists such as horses, greyhounds and antelopes, who can maintain maximum galloping speeds of 15–20 meters per second for several minutes \citep{Garland1983}.  Moreover, running is more costly for humans than for most mammals, demanding roughly twice as much metabolic energy per distance travelled than is typical for a mammal of equal body mass \citep{Taylor1982}.  This suggests that humans have evolved capacities suited specifically to endurance running at moderate intensities, rather than short bouts of high intensity sprinting \citep{Bramble2004}.}

The co-occurence of physical exertion with joint action has been identified in activities of non-human primates, suggesting that the human propensity for group exercise scaffolds on top of strong evolutionary foundations laid well before our last common ancestor with Chimpanzees ($\sim 5-7$ mya).  Group hunting in Chimpanzees is by now well-documented and is known to be both spontaneous, highly organised \citep[for example involving divisions of set roles][]{Boesch1989}, and causally linked to social bonding \citep{Mitani2001}.  Chimpanzees behave similarly in group territorial ``warfare'' against members of neighbouring groups \citep{Boehm1992,Wilson2014a}.
In addition to group hunting and territorial conflicts, Chimpanzees also engage in exertive group activities such as group laughter \citep{Waller2005}, or coordinated display patterns of tree buttress drumming, pant-hooting, and branch-dragging \citep[for example, observed as part of a ``rain dance,'' see][]{Goodall2000,Whiten2001}.  Meanwhile, bonobos display a propensity for highly arousing and energetic socio-sexual encounters, which are hypothesised to quell intra- and inter-group tensions (relating to competition over food or reproductive resources) through the activation of neuropharmacological reward mechanisms \citep{Dunbar1992,Parr2005,Clay2015}.  Taken together, this evidence suggests that group exercise underpins various adaptive social behaviours in non-human primates, ranging from hunting, to group defence, and even social affiliation and diplomacy.

\subsection{Group exercise as the object of social scientific analysis\label{sect:GEsoSci}}
Social scientists and anthropologists have long speculated that cultural activities in which group exercise features are somehow central to the function of human sociality.  Sociologist Emile Durkheim emphasised the emotional importance of shared ritual practice, which he thought could ``change the conditions of psychic activity'' (p. 469).  This group-level property of ``collective effervescence,'' Durkheim argues, ``strengthen[s] the bonds attaching the individual to the society'' and has important behavioural consequences relating to prosociality and well-being (Durkheim 1915/1965; pp. 257-258). ``Once the individuals are gathered together,'' Durkheim notes of an indigenous Australian tribe engaged in ritual dance, ``a sort of electricity is generated from their closeness and quickly launches them to an extraordinary height of exaltation...''  As a compliment to his more famous essay on the role of shared cultural practices in generating ties of social reciprocity---`The Gift''---Marcel Mauss (Durkheim's nephew) also drew attention to the importance of the visceral, embodied dimension to human social life in his 1935 essay ``Techniques of the Body'' \citep{Mauss1935}.
    \footnote{The intuition that the body is at the centre of social processes has been developed throughout an entire French social scientific lineage.  Mauss was Durkheim's nephew, and phenomenologist Maurice Merleau-Ponty \citep{Merleau-Ponty1956}, in turn was influenced by Mauss.  Anthropologist Pierre Bourdieu \citep{Bourdieu1990}, and contemporary sociologist Loic Wacquant \citep{Wacquant2004} are also direct descendants of this line of inquiry, which invariably fixates on the causal relevance of ``embodied practice'' to social phenomena.}
Meanwhile, anthropologist Victor Turner highlighted the drama of ritual performance (invariably involving group exercise) in creating a playful and liminal space separate from the usual structure of everyday social life in which ``spontaneous communitas'' and ``humankindness'' between ritual participants is fostered \citep{Turner1974}.  These accounts share a common focus on the relationship between embodied psychophysiological processes and social processes.

While initial anthropological accounts of group exercise in human sociality lacked an a rigorous scientific framework through which explanatory mechanisms could be tested and verified, the importance of the visceral and emotional dimensions of group exercise to human sociality was nonetheless intuitively grasped and duly emphasised.

\subsection{The sidelining of group exercise through the modern evolutionary synthesis and the cognitive revolution \label{sect:visceralSideline}}

Generally speaking, more testable scientific accounts of human behaviour have emerged only in the last $\sim$70 years,  enabled by 1) the gradual refinement of evolutionary theory, now known as the ``modern evolutionary synthesis'' (hereafter MES) and 2) the ``cognitive revolution'' (hereafter CR) of the 1950s and 60s \citep{Laland2010}.  The MES amounts to a unification of the theory of evolution by natural selection (attributed to Darwin and Wallace in the second half of the 19th century) with a theory of genetic inheritance \citep[replacing a previously popular theory of blended inheritance, see][]{Calcott2013}.  The MES thus paved the way for a ``gene-centred'' understanding of biological evolution, as changes in the frequency of heritable DNA sequences in a population due to selection pressures exerted at the level of the phenotype \citep{Grafen1984}.  Meanwhile, concepts from information theory, cybernetics, and computation provided the necessary language to describe cognition and evolution as variational, digital, and sequential processes of change \citep{Yockey2005}.  Forefront to initial applications of cognitive and evolutionary approaches to human behaviour were questions of the biological origins of human sociality, including our species-unique capacities (e.g., language and culture) for complex patterns of cooperation and coordination \citep{Wilson1975,Chomsky1965}.

Together, MES and CR enabled anthropologists to draw upon empirical findings and theoretical frameworks from neighbouring fields of cognitive science, psychology, evolutionary biology, and behavioural ecology, to develop hypotheses concerning proximate psychological and cultural causes of, and ultimate (evolutionary) explanations for human sociality \cite[e.g.][]{Dawkins1976,Wilson1978,Sperber1996,Whitehouse2004,Dunbar1996}.  However, the necessarily rudimentary parameters of initial MES models of human cognition and evolution also limited the extent to which these models could account for a full spectrum of behaviour observable in humans' evolutionary niche.

%see Appendix~\ref{sect:modernSynthesis} for a more detailed explanation of MS, CR, and their applications to human behaviour)
In essence, the theoretical assumptions of MES models have served to direct attention of empirical research towards cognitive mechanisms that are fixed by processes of natural selection \citep{Lickliter2003,Kenrick2001}.  The gene-centred orthodoxy of the MES posits two interrelated assumptions.  The first is that population-level distribution of gene frequencies is determined solely by natural selection \citep[including associated stochastic mechanisms of mutation, and genetic drift][]{Grafen1991}.  The second assumption follows from the first, by suggesting that an organism's developmental processes are not causally relevant to evolutionary change \citep[also known as the ``phenotypic gambit''; see][]{Grafen1984}.  Assuming that developmental processes are not passed on in the germ-line, immediate processes that contribute to the development of an observable behaviour can be largely set to one side, and instead focus can be placed on the adaptive value of a trait and its phylogenetic history \citep{Mayr1961,Dunbar1996}.

While these assumptions were arguable necessary in order to reduce the overwhelming complexity of biological processes and make scientific analysis of human behaviour tractable \citep{Mayr1961}, they have also served to direct scientific attention towards a limited subset of cognitive mechanisms.  The primary vehicle of MES approaches to human behaviour has been the principle of inclusive fitness. Working within the MES assumptions outlined above, \textcite{Hamilton1964} demonstrated that the adaptive value of an observable behavioural trait can be calculated as the sum of direct and indirect fitness benefits to an organism over its lifespan \citep{Grafen2006}.  Inclusive fitness therefore helps explain how social behaviours that appear seemingly costly to the individual phenotype could be \textit{indirectly} beneficial in the case that such behaviours increased the reproductive success of other individuals carrying the same gene (expressed famously in Hamilton's rule of $rb > c$).  Considered from the point of view of inclusive fitness, cognition, social learning, and (cultural) communication systems (among other aspects of the human evolutionary niche) are understood as (adaptive) extensions of the phenotype, which have evolved according to their contribution to the inclusive fitness of the organism over evolutionary time.\footnote{Evolutionary biologist Richard Dawkins, for example, famously proposed that human culture should be understood as a digital, variational, and heritable evolutionary system in its own right that furnishes humans' ``extended phenotype'' \citep{Dawkins1982}.  Just as evolutionary biologists model genes, anthropologists can model ``memes'' as units of cultural selection that transmit and fixate within populations.}

In response to this line of MES reasoning, researchers began to hypothesise the existence of evolved cognitive mechanisms that either 1) function according to their contribution to inclusive fitness \citep[an approach that has since matured into the field of evolutionary psychology][]{Cosmides1992}, or else 2) are conducive to the effective \citep[i.e. ``cumulative'' see][]{Tomasello1993} transmission of (adaptive) cultural variants \citep[an approach that has since matured into the field of cultural evolution][]{Cavalli-Sforza1981,Boyd1988}.
As such, cognitive mechanisms pertaining to aspects of social decision making \citep[in the case of game-theoretical models of cooperation, see][]{Cosmides1989,West2011} or memory and social learning \citep[in the case of gene-culture coevolutionary models][]{Henrich2003} have been most actively examined for evidence of their ultimate evolutionary significance \citep{Badcock2012}.  Focus on these cognitive mechanisms and evolutionary processes was further assisted by the furnishing of MES with a theory of change advanced by the CR, which conceived of species-unique aspects of human cognition as primarily logical, grammatical, or symbolic processes of digital (and a-modal) information transfer.\footnote{Refer, for example, to the language of digital computation utilised by Ernst Mayr in his explication of the need to distinguish between proximate and ultimate levels of biological phenomena;][]{Mayr1961}.}  In brief, the assumptions deployed by gene-centred approaches to human evolution indirectly constrain empirical attention to a subset of cognitive mechanisms that contribute to inclusive fitness---either directly by producing adaptive traits, or via transmission of cultural variants.  The visceral or embodied dimensions of human experience have been traditionally sidelined as largely inconsequential to a gene-centred MES and an amodal theory of informational change.

To be sure, researchers have worked steadily over the past 70 years to nuance cognitive and evolutionary approaches to human behaviour, which has enabled more inclusive explanations of observable features of human behaviour and sociality.  Evolutionary psychologists, for example, propose that evolved cognitive mechanisms (adaptations) may not predict survival or reproductive success in proximate contexts in which they are currently observable \citep[as per the assumption utilised in human behavioural ecology known as the ``phenotypic gambit;'' see][]{Grafen1984}.  Rather, evolved cognitive mechanisms influence behaviour in ways that were likely to have performed an adaptive function over evolutionary time \citep[a theoretical formulation known as the ``environment of evolutionary adaptiveness'' (EEA)][]{Cosmides1992a,Buss1998}.  Models of cultural evolution adjust population genetic models to take into account the observable differences between cultural and genetic information, such as culture's capacity to support one-to-many transmission, the blending of cultural variants, and non-randomly guided variation \citep{Cavalli-Sforza1981,Boyd1988}.  These adjustments are part of the concession that human cultural variants are not as dependent on high fidelity replication as their genetic cousins, but instead are shaped by evolved cognitive biases that favour the acquisition and transmission of some cultural variants over others due to their memorability or effectiveness \citep[i.e., context sensitivity][]{Henrich2007}.
Meanwhile, the functional role of emotion has been foregrounded---in neuroscience \citep{Damasio1994}, cognitive science \citep{Lazarus1982}, and various strands of developmental \citep{Campos1989}, social \citep{Parrott2001}, positive \citep{Fredrickson2001}, and cultural \citep{Nisbett2003} psychology.
Within evolutionary psychology, it has been recognised that emotion may have evolved as a superordinate mechanism for regulatory function \citep{Cosmides2000}, and may play an important (albeit subordinate) role in regulatory function of decision making \citep{Dalgleish2004} and communication \citep{Rime2009}.  In all of these approaches, however, the foundational assumptions of the MES (selection at the level of the phenotype according to inclusive fitness) and CR (primacy of explicit, propositional, or semantic processes of information transfer) remain largely in-tact.  As such, the role of physiological, social, and ecological dimensions of human cultural activity---including those that feature group exercise---are not afforded causal primacy, and therefore remain deemphasised in accounts of the natural origins of human life \citep{Badcock2012}.

To summarise, the first 70 years of cognitive and evolutionary approaches to human behaviour and sociality have produced explanatory models that tend to direct empirical attention to species-unique traits that can be understood to enhance inclusive fitness (in the case of evolutionary psychology) or otherwise facilitate cultural transmission (in the case of cultural evolution).  While the MES has been groundbreaking for understanding human behaviour and sociality, these attempts do not yet satisfactorily account for the mysteriously visceral dimension to group exercise reported by Adrian and identified by Durkheim.  In this thesis, I propose that a closer attention to the proximate mechanisms and dynamical properties of joint action can help bolster cognitive and evolutionary approaches to understanding the role of group exercise in human sociality.


%, which hinge on the cognitive capacities---e.g. imitation, teaching, and memory---for ``cumulative culture'' \citep{Tomasello1993}).

\subsection{Bondedness and sociality\label{sect:bondednessSociality}}
More recent waves of scholarship have addressed the sidelining of physicality in human cognition and sociality.  As Anthropologists Robin Dunbar and Susanne Shultz have aptly pointed out, humans do not merely adhere to a``dung fly'' model of sociality, by incidentally \textit{aggregating} in time and space according to genetically hard-wired protocols \citep[see][]{Wilson1975}.  Rather, humans actively \textit{congregate} in structured, cohesive groups, which are held together by emotionally-mediated relational bonds \citep[777]{Dunbar2010}.  What makes human sociality unique, according to Dunbar, is the quality of ``bondedness''---the human capacity to forge social bonds with genetically unrelated members, which results in a more adaptive ecological niche \citep[see][]{Odling-Smee2003}.
Because being able to maintain the effective functionality of a group through time may have very significant individual fitness benefits for its members, the emergent property of sociality itself can be understood as part of the individual's fitness strategy \citep{Dunbar2010b,Nowak2010}.

Dunbar argues that human bondedness and sociality cannot simply entail cognitive processes \citep[at least not in the way cognitive processes are narrowly rendered by game-theoretic and gene-culture coevolutionary models;][]{Dunbar2010}.  Rather, bonded relationships involve two parallel and distinct mechanisms—--a cognitive mechanism (derived from what Dunbar calls the ``social brain'' \citep{Dunbar1998}, or what has been otherwise understood as an evolved ``norm psychology'' \citep{Chudek2011}), and a physiological mechanism, in the form of an emotionally based form of attachment \citep{Dunbar2010b}.  By appealing to the physiological basis of relational ties between individuals, Dunbar scientifically reinstates the visceral dimension to human sociality that is observable in cultural activities involving group exercise.

  %This and which was temporarily absent in initial waves of cognitive and evolutionary approaches to human behaviour.

Humans' evolved capacity for social bonding is thought to have arisen in primates as an adaptive response to the pressures of group living.  Aggregating in groups serves to reduce threat from predation, but at the same time can be individually costly due to stress arising from interaction at close proximity, and conflict over resources among genetically unrelated individuals.  These costs can lead the group to disband, and are hypothesised to have led to selection for social bonding via dyadic grooming, as the coalitional alliances that arise among grooming partners allow for the maintenance of the group by buffering the stresses of group living \citep{Dunbar2012}.  Primate social grooming leads to the release of endorphins \citep[a type of endogenous opioid, see][]{Keverne1989}, presumably leading to sustained rewarding and relaxing effects \citep{Dunbar2010}.  While other neurotransmitters such as dopamine, oxytocin and/or vasopressin may also be important in facilitating social interaction, it has been suggested that endorphins allow individuals who are not related or mating to interact with each other long enough to build ``cognitive relationships of trust and obligation'' \citep[1839]{Dunbar2012}.

As the homo genus evolved more complex collaborative capacities for survival in interdependent group contexts \citep[see][]{Dunbar1998,Tomasello2012a}, grooming-like behavioural technologies (such as group laughter, music making and dance, and collective ritual) also evolved \citep[via processes of multi-level cultural group selection, see][]{Wilson2008} to sustain social bonding in larger group sizes where dyadic grooming would take too much time \citep{Dunbar2012,Tarr2014,Launay2016}.
When mood-elevating effects are experienced in a group, they seem to lead to participants embodying each other's affective experiences, resulting in more positive, trusting, and cooperative relationships among participants \citep{Dunbar2012}.

% \citep[see][]{Mayr1961,Tinbergen1963}

Bondedness provides a testable theory for rigorously comprehending what Durkheim described as collective effervescence.  Accounting for bondedness in human sociality requires, on an ultimate, evolutionary level of analysis, the loosening of a strict singular determinacy of selection on cultural variants.  In its place, bondedness suggests a multi-level (cultural group selection) or niche-construction approach, whereby individual-level benefits can be generated through the production and evolution of cultural activities involving genetically unrelated social actors operating within a shared ecological niche \citep{Dunbar2012,Laland2010,Laland2015}.  On a behavioural or ``proximate'' level of analysis, bondedness requires a conceptual broadening of mechanisms of cognition to incorporate the causal role of physiological (as opposed to purely cognitive) mechanisms. In essence, social bonding provides the immediate ``social'' glue for social cohesion \citep[see][]{Lakin2003,Bastian2014a}, and this achievement depends crucially on physiological (and not just cognitive) processes.  Bondedness thus offers a suitable vehicle for an analysis of the visceral dimension of group exercise.  As I argue below, however, existing theories of group exercise and social cohesion contain various gaps that can be addressed by more focussed attention to mechanism of physical and social coordination in joint action.

\subsection{The social high theory of group exercise and social bonding \label{sect:socialHigh}}

Research focussed on the proximate physiological, cognitive, and social mechanisms associated with group exercise confirms that group exercise is responsible for generating a psychophysiological environment conducive to social bonding.  Anthropologist Emma Cohen and colleagues have recently identified bi-directional causal links between the two essential ingredients of group exercise---1) physiological exertion and 2) interpersonal movement coordination---and their common psychophysiological effects, including increased pain tolerance, athletic performance, positive affect, wellbeing, pro-sociality, and cooperation \citep{Davis2015}.  This evidence amounts to what can be called the ``social high'' theory of group exercise and social bonding \citep[hereafter ``the social high theory,'' see][]{Cohen2017}.  Here, social bonding is understood as the psychological experience of increased social closeness, which facilitates affiliation between non-kin group members \citep{Tarr2014}.  I outline the key principles of the social high theory below, before identifying some empirical and theoretical aspects of the relationship between group exercise and social bonding which are not addressed by this theory, and which thus require further attention.

\myparagraph{Physiological exertion\label{sect:physExertion}}
The social high theory proposes a causal link between physiological exertion and social bonding.  Group exercise necessarily entails rigorous physiological exertion.
The health and wellbeing benefits associated with regular physical exercise---including reduced risk of cardiovascular disease, autonomic dysfunction, and early mortality; as well as enhanced neurogenesis, cognitive ability, and mood---are becoming increasingly well-known \citep{Blair1994,Nagamatsu2014}. Evidence suggests that strenuous and prolonged physical exercise is modulated by the same neuropharmacological systems responsible for regulating pain, fatigue, and reward \citep{Boecker2008,Raichlen2013}.  Neurobiological rewards in exercise are associated with both central effects (improved affect, sense of well-being, anxiety reduction, post-exercise calm) and peripheral effects (analgesia), and appear to be dependent for their activation on exercise type, intensity, and duration \citep{Dietrich2004}.  Exercise-specific activity of neurobiological reward systems offers a plausible explanation for commonly reported sensations of positive affect, anxiety reduction, and improved subjective well-being during and following exercise---extremes of which are popularly referred to as the ``runner's high'' \citep{Dietrich2004,Boecker2008,Raichlen2012}.  This neurobiological evidence maps on to more extensive literature concerning the psychological effects of exercise, which indicates a duration and intensity ``sweet spot'' for exercise and positive affect, whereby moderate-intensity exercise for durations of $\sim45$ minutes appears most optimal \citep{Reed2006}.

It is possible that the function of exercise-induced positive affect extends to the realm of social bonding, particularly when achieved in group exercise contexts \citep{Cohen2009,Machin2011}.  Endocannabinoids and opioids have been implicated in mammalian social bonding \citep{Fattore2010,Keverne1989}, and in humans specifically, there is evidence that endorphins (a particular class of endogenous opioids) mediate social bonding \citep{Dunbar2012,Shultz2010}.

\myparagraph{Interpersonal movement coordination\label{sect:synchrony}}
The social high theory also proposes a causal link between joint action and social bonding.  Experimental evidence (predominantly from social psychology) suggests that time-locked coordination of behaviour between two or more individuals is conducive to psychological processes of self-other merging, liking, trust, and psychological affiliation.  In these contexts, interpersonal coordination is primarily operationalised as behavioural synchrony---i.e., stable time- and phase-locked movement of two or more independent components (limbs, bodies, fingers, etc.) \citep{Pikovsky2007}. Researchers suggest that synchrony enables a tight attentional union between individuals who match the timing and content of their actions, leading to the enhancement of interpersonal similarity and the blurring of self-other boundaries in cognitive processing and recall \citep{Cohen2017}.  Relative to non-synchronous group activities, synchrony increases social bonding and pro-social behaviour \citep{Reddish2013,Reddish2013a,Wiltermuth2009,Tarr2014}---an evolutionarily important outcome of bonded relationships.  Recent studies have also found that, compared to solo and non-synchronous group exercise, synchronous group exercise leads to significantly greater post-workout pain threshold \citep{Cohen2009,Sullivan2014,Sullivan2013a, Sullivan2013b}.

A recent meta analysis of the behavioural synchrony literature in social psychology suggests three candidate mediators of the relationship between behavioural synchrony and social bonding: 1) lower cognitive affective mechanisms implicating neuropharmacological reward systems (e.g., opioidergic and dopaminergic systems), 2) neurocognitive action-perception networks responsible for the experience of self-other merging, and 3) processes of group-centred cognition responsible for perception and reinforcement of cooperation \citep{Mogan2017}.  The current balance of existing evidence suggests that affective physiological mechanisms may be more relevant to joint action involving larger group sizes in which generalised feelings of euphoria and pro-sociality are common \citep[e.g., mass religious rituals or music festivals, see][]{Weinstein2016}, whereas neurocognitive mechanisms linking joint action and social bonding may be more applicable to smaller group sizes in which individuals can share intentions through ostensive communicative signals and implicit movement regulation cues \citep{Lang2017}.  Studies linking synchrony with social bonding and cooperation are supported by a literature than connects nonconscious mimicry with liking and affiliation \citep{VanBaaren2009}.

\myparagraph{The social high $=$ exertion $\times$ synchrony}
In addition to recorded independent effects of exertion synchrony,  preliminary evidence suggests that exertion and coordination in group exercise interact to produce social effects \citep{Jackson2018}.  Social features of the exercise environment (for example, perceived social support, level and quality of behavioural synchrony, etc.) modulate exercise-induced mechanisms of pain and reward \citep{Cohen2009,Sullivan2014,Tarr2015,Davis2015,Weinstein2016}. This work is bolstered by existing literature on the social modulation of pain \citep{Eisenberger2012a}, and links between pain and prosociality \citep{Bastian2014a}.  The social high theory thus combines these two bodies of literature to tell a story in which positive affect---associated with neuropharmacologically-mediated pain analgesia and reward—--is extended to the social group via synchrony-activated cognitive mechanisms of self-other merging, and the perception and reinforcement of in-group cooperation.


\section{Empirical knowledge gaps in the relationship between group exercise and social cohesion\label{sect:empKnowGaps}}
While the social high theory has served to empirically flesh out---particularly at the proximate level of analysis---the relationship between group exercise and bondedness, the theory does not account for a full spectrum of profiles and subjective experiences in group exercise.  Even a cursory survey of human sociality reveals that group exercise scenarios often deviate markedly from the narrowly defined profile (the exertion $\times$ coordination sweet spot) and subjective experience of group exercise set out by the prevailing social high theory.  In this section, I review some of the empirical gaps that remain unexplained by the social high theory.  In particular, group exercise contexts often involve extreme (and not just moderate) levels of physiological exertion, as well as complex coordination demands (beyond exact synchrony).  In addition, the effects of participation in group exercise appear to extend well beyond those of a feel-good social high, and into the realm of rich meaning making and fine-grained sensitivity to the click of joint action.  I suggest that these empirical gaps in the social high theory point to a more fundamental \textit{theoretical} gap in cognitive and evolutionary approaches to human behaviour.  In short, the vast majority of cognitive and evolutionary approaches to human behaviour are unable to sufficiently account for the dynamical interlocking of physiological and social processes. In this thesis, I suggest that the dynamical dimensions of group exercise contexts serve to bring this theoretical gap into sharp focus.

%In this dissertation I propose to develop cognitive and evolutionary understandings of group exercise through closer attention to the immediate causal processes and psycho-social effects of interpersonal movement.  I suggest that the carnal mystery that Adrian attempted to articulate on my first night in Beijing can be explained in part by the way in which extreme physiological exertion and interpersonal movement coordination combine in rugby to enable and constrain physiological, psychological, emotional, and social processes.

\myparagraph{Group exercise involves both extreme physiological cost and profound meaning\label{sect:linkCostMeaning}}
Anecdotal and observational evidence suggests that group exercise contexts often entail extreme levels of psychophysiological exertion.  High-stakes professional competitive sporting contexts (international-level sports such as rowing or ultra marathon running), extreme adventure sports (big wave surfing, free-diving), high-intensity contact (rugby union, American football, ice hockey) and combat sports (MMA, boxing, wrestling) are known to involve extreme physiological demands, often including high levels of pain.  Although it is expected that extremely physiologically costly exercise contexts will involve activation of neurobiological reward mechanisms outlined above, some contexts may on average exceed (or alternatively never reach) the intensity and duration sweet-spot for optimal activation of neurobiological reward \citep{Raichlen2013} or positive affect \citep[see Appendix~\ref{sect:neuroRewardGE} for more details on exercise-specific neuropharmacological effects)]{Ekkekakis2011,Reed2006}.

At the same time, group exercise involving extreme physiological, psychological, and social costs also appear to offer participants and observers an opportunity for profound meaning.  Many people do not engage in exercise \textit{just} enjoyment or health; rather, in some contexts sport forms part of a life of purpose and self-discovery \citep[see, for example][]{Jackson1995,Jones2004,White2011}.  Modern sport, for example, has always been much more than ``just a game,'' and instead offers an arena in which virtues and vices are learned, and the ``morality plays''—--of community, nation, or globe—--thus performed \citep{Elias1986,McNamee2008}.  Psychological and physiological resilience in exercise contexts is lauded as virtuous, as is evidenced by the numerous idioms in the English language that receive currency in exercise lore: ``when the going gets tough, the tough get going,'' ``no pain, no gain,'' ``you get out what you put in'' and so on \citep{Sarkar2014}.  Activities in which group exercise feature, such as organised warfare, offer even more obvious examples of socially coordinated and physiologically exertive activity that is not for the mere purposes of a social high.

Whereas the social high theory predicts motivation for exercise based on ``hedonic'' enjoyment, anecdotal and ethnographic perspectives emphasise instead the ethical and moral dimensions of experiences, which social anthropologists contextualise within political processes relating to the construction of the self, community, and nation-state \citep{Alter1993,Brownell1995,Downey2005,Wacquant2004}.
For example, anthropologists and sociologists have for some time emphasised the social function of exercise and sport in diverse cultural contexts, and various attempts have been made to analyse the phenomenological experience of exercise in terms of its sociological and psychological meaning \citep{Bourdieu1978}.

Social anthropologist Joseph \textcite{Alter1993}, for example, argues that for wrestlers in north India, the body functions as a nexus through which the symbolic and material structures of the state, family, and the individual coalesce.  In a similar vein, in a seminal ethnography of sport in China, cultural anthropologist Susan \textcite{Brownell1995}, argues that sport functions as a crucial national symbolic practice for the Chinese nation-state in a project of ``rejoining the world,'' and that the ``micro-techniques'' (c.f. Foucault, 1977) of this project entail significant cost to (and rich meaning for) the individual athlete.   Similarly, French sociologist Loic Wacquant \textcite{Wacquant2004}, in an ethnography of boxers in Chicago's south side, describes a ``social logic'' of physical activity, claiming that the costs associated with ``the daily dedication and high technique that training demands; the regimented diet; the control, mutual respect, and tacit understandings necessary for actual fist-to-fist competition serve to create for the boxer an island of order and virtue'' \textcite[17]{Wacquant2004}. In many instances, it may be that the primary psychological motivation for exercise is not immediate, hedonic wellbeing, but instead \textit{eudemonic} wellbeing, or the psychological awareness of a process through which life becomes ``well-lived'' \citep{Fave2009,Huta2013}.

\myparagraph{Group exercise demands complex coordination which can lead to team click\label{sect:linksComplexClick}}
A similar connection between physiological, cognitive, and social processes in group exercise can be identified in processes of joint action typical of group exercise contexts.  Currently, the social high theory has identified exact behavioural synchrony as an idealisation of coordinated joint action in group exercise.  While some group exercise contexts do contain high levels of behavioural synchrony \citep[rowing, synchronised swimming, diving, mass calisthenics, and forms of dance such as ballet, see][]{McNeill1995}, exact, in-phase synchronisation or matching of movements between co-actors is not typical of most instances of group exercise.  More generally beyond group exercise, interpersonal coordination is more often achieved through flexible, function-specific assemblages of complimentary and contrasting behaviours \citep[for example, coordination in an interactional team sport, a dyadic conversation, or an ensemble music performance, see][]{Fusaroli2014}.  Real-world instances of joint action in group exercise usually entail various distinct action elements, organised hierarchically within a sequence \citep{Schmidt1975,Rosenbaum2009}.  Successful execution of the structure of real-world joint action requires temporal and spatial precision and flexibility of movement across multiple timescales and sensorial modalities \citep{Sebanz2006,Pacherie2012}.
All of these factors amount to potentially extreme levels of computational uncertainty in joint action contexts typical of group exercise \citep{Bernstein1967}, whose mitigation and regulation necessitate mechanisms of coordination of physical movement and social communication \citep{Turvey1978}.  In essence, successful coordination in joint action typical of group exercise contexts requires considerable cognitive resources \citep{Turvey1978}, which are not budgeted for when joint action is modelled as exact in-phase synchrony \citep{Keller2014}.


% As discussed below in Section~\ref{sect:pathBeyondSynch}, this reliance on synchrony could occlude important causal mechanisms in a relationship between joint action and social bonding.

At the same time, participants in group exercise contexts involving complex joint action often scrutinise the quality of coordination, and derive powerful psychological reward when complex joint action clicks \citep{Jackson1992}.  Technically demanding group exercise contexts such as competitive interactional team sports or music-making and dance, depend upon fine-grained precision of coordination of behaviours between individuals: the movements and goals of one individual must align precisely in time and space with the movements and goals of another.  For highly skilled expert practitioners, often the ecstasy of group activity is contingent not just on on reaching a certain level of physiological exertion, or resting on exact synchronisation of behaviours with others, but on the extent to which performance in joint action satisfies or exceeds implicit and explicit \textit{expectations}.  Consider the passage below, taken from a series of interviews that psychologist Susan Jackson performed with elite paired figure skaters:
%Psychologist Susan Jackson has accumulated considerable evidence of elite level athletes' subjective experience of ``flow'' in joint action:
  \begin{quotation}
    It was just one of those programs that clicked. I mean everything went right, everything felt good . . . it's just such a rush, like you feel it could go on and on and on, like you don't want it to stop because it's going so well.  It's almost as though you don't have to think, it's like everything goes automatically without thinking . . . it's like you're in automatic pilot, so you don‘t have any thoughts.  You hear the music but you're not aware that you're hearing it, because it's a part of it all. \citep[168]{Jackson1992}.
  \end{quotation}

Here, achieving success in technically demanding joint action is the source of rush, exhilaration, and amazement.  The source of this exhaltation may be due to the achievement of success in joint action, when the chances of success are highly improbable due to extreme levels of joint action complexity.

The psychological literature of ``flow'' and optimal human performance \citep[see][]{Csikszentmihalyi1992} suggests that athletes engaged in team coordination often report total absorption in and complete focus on the task at hand, a transformation of the experience of time (either speeding up or slowing down), and a blurring or transcendence of individual agency, or a ``loss of self''   \citep{Csikszentmihalyi1992,Jackson1995,Jackson1999,McNeill1995}.  Research suggests that flow often occurs in scenarios in which there are clear goals inherent in the activity, as well as unambiguous feedback concerning extent to which goals are either being achieved or not.  In addition, scenarios most conducive to the experience of flow are those in which the technical requirements are challenging but achievable if practitioners are able to extend (slightly) beyond their normal capabilities\citep{Fong2015}.
The coalescence of these factors is intrinsically rewarding and autotelic, activating both ``hedonic'' and ``eudemonic'' dimensions of subjective well-being \citep{Huta2010,Fave2009}.

In sum, the social high theory is not equipped to explain instances of group exercise that deviate from a profile of moderate intensity exertion, exact synchrony, and a feel-good social high.  Anecdote and ethnographic observation suggest that group exercise contexts also involve---in addition to mood-elevating effects---extreme levels of physiological cost, rich psychological meaning making, cognitive complexity, and feelings of team click.  Importantly, extreme physiological cost appears to be tethered to cognitive and social processes of meaning making and social identity (Section~\ref{sect:linkCostMeaning}).  Similarly, cognitive complexity in joint action appears to be linked to embodied, affective experiences of flow, eudemonic wellbeing, and team click.  Together, these empirical gaps in the science of group exercise and human sociality suggest the importance of dynamic interlocking between physical movement and social communication in order for the social bonding effects of group exercise to be realised.

% mechanisms for the generation of psychophysiological experience of group exercise.


\subsection{Attempts to address coordination of physical and social processes in social interaction}

For some time, attempts have been made to scientifically account for the physical and embodied basis of social cognition.  Researchers wihtin cognitive science have argued that processes of information transfer in joint action are fundamentally shared between brains, bodies, and physical features of a task-specific environment \citep{Hutchins1995,Kirsh2006,Susi2001}.  Theoretical formulation and empirical substantiation of this understanding of social cognition in joint action has, however, proven more difficult to achieve \citep[due in part to the tendency of the MES and CR to preference the functional role of symbolic and amodal cognitive processes; see][]{Semin2008,Yufik2013}.
But, more recent advances in neuroimaging technologies \citep{Frith2007}, neurocomputational theories of brain function \citep{Friston2010,Frith2010,Yufik2013,Clark2013}, and constructive attempts to extend the theoretical paradigm of human social cognition to account for inter-individual processes of interaction and coordination \citep[e.g.][]{Sebanz2006,Semin2008,Dale2014} have coalesced as a theoretical paradigm with testable predictions.

As I outline below, there is mounting evidence to suggest that humans' dual capacity for coordinating complex physical movement and elaborate social communication are tethered by a functional imperative to regulate high levels of uncertainty in social interaction.  In this thesis, I define uncertainty in information-theoretic terms as a quantity that reduces an individual's ability to successfully predict, anticipate, establish, and sustain joint action \citep{Shannon1963}.  As I explain in more depth in the following chapter, joint action, relative to individual action, is defined by high levels of uncertainty, owing to its cognitive complexity.  Moreover, joint action scenarios typical of group exercise contexts are defined by extreme levels of uncertainty relative to quotidian social interactions.

In emerging dynamical approaches to social cognition of joint action, the ultimate---thermodynamically inspired---mandate to reduce uncertainty drives dynamic coordination between neural models that predict the likely causes of sensory stimuli,  and extra-neural affordances that function to verify and enhance these predictions \citep{Friston2015,Ramstead2016}.  In effect, this approach suggests that higher levels of coordination between neural models and their relevant affordances will lead to more adaptive behaviour, by virtue of reducing uncertainty inherent in an organism's exchanges with the environment \citep{Firston2010,Ramstead2017}.  When applied to joint action, this prediction suggests that higher levels of (actual or perceived) coordination of physical movement generates higher levels of social alignment between co-actors \citep[]{Semin2008,Wheatley2012}.

Evidence for the imperative of coordination can be identified on the level of physical movement in the form of ``functional interpersonal synergies'' \citep{Riley2011} between co-actors and action-perception coupling between sensorimotor processes of skilled practitioners in joint action \citep{Novembre2014}.  Meanwhile, evidence for the imperative of coordination on the level of social communication can be identified in strategies such as the attribution of agency facilitate delineation roles and responsibilities in joint action \citep{Wolpert2003,Sato2008,VanderWel2012VanderWel2012}, or the use of type-based social heuristics of evaluation (e.g., personality type or social standing) \citep{Moutoussis2014}---both of which enhance predictability of future action by reducing computational uncertainty.
In addition, participation in systems of explicit communication---contingent upon shared recognition and participation by groups of others \citep{Ramstead2016}---functions to ``smooth'' coordination of joint action by providing shared referents for action planning and prediction \citep{Vesper2017}.
Importantly, all of these strategies of physical and social coordination function to reduce the uncertainty inherent in social interaction.

In sum, existing evidence deriving from dynamical approaches to social cognition supports an investigation into the role of uncertainty and coordination in joint action as a source of social bonding in group exercise.  Research suggests the possibility that higher levels of coordination in physical movement could generate higher levels of coordination of social communication, identifiable as processes characteristic of social bonding.  In this thesis, I propose that the activation of the pathway between coordination of physical movement and social bonding requires, crucially, that joint action  ``clicks.''
In the following chapters, I first develop a general account of team click in group exercise,  before situating and testing this account in the real-world group exercise setting of rugby in China.



\section{When the rugby team in China clicks}
To summarise the ground covered thus far: this thesis is driven by the overarching aim of contributing to an evolutionary explanation for puzzling ubiquity of group exercise in the more recent human record.  Existing cognitive and evolutionary accounts suggest a relationship between group exercise and social cohesion, due in part to the way in which group exercise contexts uniquely generate social bonding between participants.  The social high theory posits that social bonding in group exercise can be explained by the way in which physiological exertion and interpersonal movement coordination combine to generate a psychophysiological environment conducive to affiliation and trust.  However, the social high theory is unable to---and, indeed, does not attempt to---account for a full spectrum of experiences of group exercise.  In particular, anecdotal and ethnographic evidence suggests that group exercise is defined by instances in which physical and social processes appear to coordinate: the physiological cost of group exercise is associated with rich social meaning, while the cognitive complexity of joint action in group exercise is associated with perceptions of team click.  I suggest that a dynamical approach to the social cognition of joint action may help shed light on these gaps.

%In the case of joint action in particular, I suggest that higher levels of uncertainty in joint action are associated with higher levels of coordination of physical movement and social communication, in instances in which joint action is performed successfully.  In the case of group exercise contexts in which levels of uncertainty are invariably high, team click may explain how the complexity of joint action can generate social bonding.
\subsection{Research questions\label{sect:researchQuestions}}
The aims of this thesis lead lead me to formulate four key research questions  (see Table~\ref{tab:researchQuestions}).  The core question I seek to answer in this thesis is how joint action generates social bonding in group exercise contexts (thesis question).  In order to directly address this question, I develop a general theoretical account of team click in group exercise (Chapter~\ref{chap:theory}), which I then situate (Chapters~\ref{chap:researchSetting}\nobreakdash~\ref{chap:ethnoField})and subsequently test, by drawing upon ethnographic and field experimental data (Chapters ~\ref{chap:ethnoResults}\nobreakdash~\ref{chap:trainingExperiment}).  Throughout this process I pay specific attention to the component mechanisms of team click (mechanism question).

\input{images/researchQuestionsIntroduction}

As such, in this thesis, I focus the bulk of my attention on testing the claim that team click explains a link between joint action and social bonding (thesis question), and interrogating team click's components mechanisms (mechanism question).   I do not completely lose sight of questions relating to the role of individual or cultural variation, or the implications of the phenomenon of team click for broader evolutionary questions of social cohesion and cultural evolution.  However, considering humble scope of this dissertation, I do not formulate specific hypotheses for these questions, but rather maintain awareness of them in the background of analysis as secondary concerns.  In addition, I return to a discussion of team click in an evolutionary perspective in the general discussion (Chapter~\ref{chap:generalDiscussion}).

This thesis progresses in a step-wise fashion, both theoretically and empirically.  First, I review existing research from anthropology, psychology, and cognitive science in order to transition from research questions to research hypotheses pertaining to a general account of team click in group exercise.  Second, I situate and contextualise these hypotheses within the research setting of rugby in China, in order to assess their validity and explanatory potential for real-world settings of group exercise.  Third, I formulate and test selected predictions derived from research hypotheses within either more representative or more controlled field experimental paradigms.
This gradual, step-wise development of theoretical formulation and empirical application is suitable for this thesis, in which I attempt to chart novel theoretical territory.


\subsubsection{Research setting and method\label{sect:researchSettingMethod}}
I focus my empirical research on a group exercise setting well suited to addressing my research aims and questions.  ``Rugby'' and ``China'' are two words that are not usually mentioned in the same sentence.  Rugby union (hereafter ``rugby'') is a dynamic field-based contact sport that requires of its participants high levels of physiological exertion and complex coordination of joint action  (see Chapter~\ref{sect:rugbyUnion} for a more detailed explanation). ``A game for barbarians played by gentlemen,'' rugby first took root in the elite education institutions of Britain's colonial empire, and in many of the contexts in which it is commonly played, rugby is anecdotally and colloquially associated with experiences of team click in joint action, as well as social processes of group membership \citep{Dunning2005}.

China has enthusiastically adopted sport and exercise at different stages throughout the nation's turbulent modern history.  But for most of this history, rugby accorded neither with a dominant Olympic-centred logic of the state sport system, nor with dominant cultural dispositions and modes of understandings physicality \citep[which derived from hundreds of years of continuous history of Confucian and Daoist traditions of thought, see][]{Morris2004}.  Rugby was eventually officially established in China in 1990, as a university sport program at the Chinese Agricultural University (CAU).  But, in 2009, when rugby became an Olympic sport in the form of ``rugby sevens'' (the modified seven-a-side version of rugby), rugby was immediately ``embosomed'' (\textit{huaibao} 怀抱) by the state-sponsored sport system and afforded various institutional resources and infrastructure \citep{Xu2010}.  At the time of writing, more than ten of China's 34 provincial level prefectures now have full time men's and women's professional programs.

In addition to these theoretical motivations, my personal qualifications as a researcher afforded me with a unique opportunity for research in this particular setting.  My combination of expertise in rugby union (as an ex-professional rugby player), experience in China (professionally fluent in Modern Standard Chinese (Mandarin)), and familiarity with rugby in China specifically (e.g., as coach of the Chinese national men's youth team in 2013) meant that I had privileged access to research opportunities ranging from in-depth ethnography to \textit{in situ} field experimental studies (for a full explanation of research qualifications, see Appendix ~\ref{app3:qualPositionResearch}).

Considering the novelty of the theoretical claims proposed in this thesis concerning the phenomenon of team click, and combined with my unique qualifications as a researcher, in-depth ethnographic and field experimental research in rugby in China offered a fruitful avenue for research.  Through my years training with (and coaching) Chinese rugby teams, I already knew that there was a specific lexicon for describing optimal coordination in joint action.  The most commonly used term to describe optimal yet effortless coordination in joint action is \textit{moqi}(默契; pronounced ``more-chee'').  \textit{Moqi} refers specifically to the phenomenon of ``tacit understanding'' between two or more actors engaged in a joint activity \citep{Pleco2018}. More generally, and much like the English verb ``click,'' \textit{moqi} can be used to describe any instance of social activity that involves tacit, unspoken understanding or coordination between co-actors: from sport, to music, even to Chinese comedic ``cross talk'' (\textit{xiang sheng} 相声).  In addition to tacit understanding, I would also often hear athletes describe the importance of the ``atmosphere'' (\textit{qifen} 气氛) or ``aura'' (\textit{qichang} 气场) between co-actors or within a team.  Thus, I was confident that I would find evidence for the experience of team click in group exercise.

\subsection{Preview of findings}
In general, ethnographic evidence revealed that athlete experience of team click were associated with subjective perceptions of on-field performance and processes characteristic of social bonding.  The tacit and embodied ``awareness'' (\textit{yishi} 意识) required of athletes can be both the source of high levels of social anxiety and stress, as well as the source of extreme exhilaration and motivation when joint action clicks.  In addition, the athletes I studied derived rich resources from their adherence to rugby for processes of personal and social identity formation, and it appeared that these resources increased with familiarity and experience with both on- and off-field requirements for membership.  This novel evidence of the relationship between joint action, team click, and social bonding validated my research hypotheses, and motivated the formulation of specific predictions and field experimental methods designed to test them.

Subsequently, I conducted two quantitative studies ``in the field.''  While control and sampling constraints of field experimental methods do do allow for causal inference and hypothesis testing in a strict sense (based on the assumptions of stringent randomisation and controlled conditions), they nonetheless over an opportunity to derive meaningful (i.e., statistical) associations between variables of interest based on theoretically-motivated predictions \citep{Xygalatas2013}.  In this phase of my research, I once again advanced in a step-wise manner.  First, I found a significant positive relationships between perceptions of team performance, team click, and social bonding \textit{in situ} during a National rugby tournament ($n = 174$).  These results then motivated a between-subjects training experiment paradigm ($n = 58$), in which I attempted to manipulate uncertainty in joint action as a predictor of more positive expectation violation, team click, and social bonding.  Results in this study failed to confirm predictions; there were no condition-wise effects of uncertainty on perceptions of confidence, positive expectation violation, team click, or social bonding.   However, the a secondary analysis of all participants collapsed into one sample did provide some confirmatory results, suggesting that the null findings can be explained by limitations in the study design, as opposed to ill-formulated hypotheses.

In sum, this thesis scientifically interrogates the moment in which the team clicks, by formulating a general theoretical account of team click in group exercise, and testing this account in rugby in China.  Results provide novel (albeit provisional) evidence for a relationship between joint action, team click, and social bonding, and suggest that scientific explanations for the prevalence of group exercise in human sociality can stand to benefit from closer attention to the mechanisms of joint action.














\end{CJK}
