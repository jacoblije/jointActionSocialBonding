\chapter{\label{app2:theory}Appendix: Research Setting}

\begin{CJK}{UTF8}{gbsn}





  \section{Introducing the research setting}

Introducing the specific group exercise context of this dissertation therefore requires a consideration of 1) the micro-level components and dynamics of joint action typical of rugby union (specifically rugby 7s), and 2) a description of the way in which these micro-level processes interact with, and are shaped by, the specific historical and cultural trajectory in which they are embedded---namely, sport in the People's Republic of China (PRC, or simply China).


  Deciphering evidence for theories of human behaviour in real world settings presents a formidable scientific challenge.  \textit{In situ}, human behaviour confronts the observer ``all at once;'' the precise causal mechanics operating to produce the observable behaviour are unclear or confounded.  On the other hand, the upshot of observing behaviour \textit{in situ} and all at once is that the phenomena is laid bare in richer detail.  Within anthropology, methods for modelling the hidden causes of behaviour are now diverse, ranging from participant observation to experimentation to mathematical simulation paradigms.   But each of these methods involves tradeoffs, and there is also considerable disagreement within the discipline concerning what claims can be made once evidence is recorded \citep{Whitehouse2012}.

  Overcoming these challenges is of extreme importance for the human sciences, particularly considering accumulating evidence to suggest that the informational resources contained in the so called ‘’real world’’ itself may be central to patterning observable behaivour.  In the case of the social cognition of joint action, for example, it has been shown that cognitive resources for social interaction are not limited to those that are located within the brain or beneath the skin, and instead are distributed between and throughout other brains, bodies, and features of the task environment.


  Cultural variants contain both explicit and implicit cues and signals for action, and evidence suggests that social interaction functions best in situations where there is a snug fit between individuals' implicit cultural expectations and explicit rules for engagement \citep{Vollan2017}.  As outlined in Chapter ~\ref{chap:theory}, there is evidence to suggest taht shared cultural knowledge can act as a ``coordination smoother'' \citep{Vesper2017} for joint action, enhancing the effectiveness and efficiency of joint action between co-participants who share a similar informational framework.  In the predictive coding paradigm, cultural habits and frames of reference act as ``hyper-priors'' that set the macro-contextual coordinates for predicting sensory inputs \citep{Clark2013}.   There is some evidence to suggest that co-participants rely on ``frames of reference'' for joint action execution \citep{Ray2018}.

  Contextual affordances for joint action appear to be dictated by processes operating at multiple conceptual levels.  From the micro-level predictive processes associated with movement action and perception, to the macro-level predictive frames offered by specific cultural and contextual niches, these affordances interact in complex processes of reciprocal causation to shape joint action (SOURCE).  As such, it can be predicted that processes of joint action, and subjective experience of phenomena such as team click and social connection will be contingent, at least in part, on the particular structure of affordances of the specific joint action scenario, as well as the cultural setting in which the joint activity takes place.  Conceptualisation of the causal complexity of cognitive processes relevant to joint action in this way echoes a broader reconceptualisation of the causal complexity associated with change on an evolutionary timescale, which recognises that human behavioural phenomena is the result of a number of biological, cognitive, and ecological mechanisms that interact via reciprocal feedback loops spanning varying scales of time and space \citep{Fuentes2015}.
  The Role of Culture and attraction? CAT (Sperber2014, )
  Click may be important to understanding processes of cultural (and genetic?) information transfer.



  \subsection{The role of anthropology}
  The theoretical approach to social cognition of joint action (and human evolution more broadly) described above accords neatly with anthropology's long-standing concern for attending to the distinctiveness of cultural trajectories, and offers a space for reconciliation between anthropology and cognitive and evolutionary approaches to human behaviour \citep{Whitehouse2012}.  Prior to appropriate acknowledgement of the complexity of cognitive processes and psychological phenomena, researchers within the human cognitive, behavioural, and evolutionary sciences (for example, cultural and developmental psychology) have been prone to overlooking local cultural specificity when seeking to generalise to the human species results of studies conducted with mainly Western subjects and methods \citep{Henrich2010d}.  Research agendas and the specific experimental designs to which they give rise are shaped by the historically and culturally contingent assumptions and priorities---predominantly of ``WEIRD'' (Western, Educated, Industrial, Rich, and Democratic) societies and experimental samples.  It is now clear that to understand the complexity of observable behavioural phenomena, systematic documentation of variation within---and not simply between---cultural niches is required \citep{Fuentes2016}.  Anthropology is thus well placed to expand upon accounts of group exercise, via methods ranging from ethnographic exploration capable of uncovering novel dimensions of behaviour and generating testable hypotheses, to quantitative techniques---e.g., experimental and mathematical simulation paradigms---capable of testing hypotheses.




  In the sections that follow, I introduce three main layers of context relevant to processes of joint action and social bonding in rugby in China.  I outline 1) the history and the joint action parameters of the sport of rugby union, 2) the historical cultural context of contemporary China, and 3) the specific history of rugby union in contemporary China.













  Below, I review evidence for the relevance of mechanisms of interoceptive predictive modelling, action-perception links, and direct extra-neural coupling to achieving success in joint action.

  %\myparagraph{Interoceptive predictive modelling}
  The multi-layered hierarchical nesting of joint action goals in rugby suggests the need for interoceptive hierarchical predictive modelling in order for participants to organise behaviour in time and space such that it optimally contributes to achieving each of these goals. Rugby players rehearse a range of technical skills individually and together over extended periods of time to set the foundation---an implicit common ground \citep[see][]{Noy2017}---for successful interpersonal coordination.  In order to achieve success in the multiple hierarchically nested shared joint action goals associated with rugby, athletes must demonstrate both precision and flexibility of interpersonal movement regulation across multiple sensorial modalities and multiple time scales \citep{Keller2014}. Successful joint action in rugby must therefore rely crucially on the operation of interoceptive predictive models capable of representing the complexity of multiple hierarchically nested joint action goals and optimally allocating cognitive resources towards the achieving of these goals.

  EVIDENCE?

  However, the cognitive demands of joint action in rugby are such that they could restrict the effectiveness of interoceptive predictive models for joint action execution.  Evolutionary Anthropologist Robin Dunbar \textcite{Dunbar1992} proposes that the ratio of human neocortex size to total brain volume imposes an upper cognitive limit on real-time coordination of behaviour of approximately four to five individuals.  The group size of joint action subunits in rugby sevens (e.g., ranging form at least two attackers and one defender (n = 3), to at most seven attackers and seven defenders (n= 14)) falls at or above this this limit.  As such, it is likely that neurocognitive mechanisms dedicated to conscious monitoring and modelling of joint action are put under high cognitive load during joint action scenarios common to rugby.

  Other temporal and spatial constraints of rugby likely jeopardise the ability of interoceptive predictive modelling to support the execution of joint action.  The time pressure associated with the ``in-the-moment'' and ``on-line'' demands of on-field coordination in rugby impose considerable constraints on computation.  As explained in Chapter ~\ref{chap:theory} Section ~\ref{sect:interoceptiveModelling}, interoceptive predictive modelling is an accurate and high fidelity mechanism for facilitating joint action, but it is certainly not high-speed, nor is it highly-efficient \citep{Kahneman2011}. Successful joint action in rugby relies on high speed coordination---a task for which interoceptive predictive modelling is not particularly well suited.

  EVIDENCE: time pressure and cognition

  The parameters of joint action in rugby also impose spatial restrictions on joint action, which also limits ability of individuals to generate interoceptive predictive models to facilitate interpersonal coordination. Joint action in rugby engages multiple sensory modalities, including visual, auditory, and somatosensory (e.g. haptic).  Inputs to these modalities derive from both exteroceptive and proprioceptive inputs.  The specific rules and conventions of joint action in rugby also function to restrict the use of certain sensory modalities in certain circumstances.

  For example, the rules of rugby dictate that the goal of each team is to advance the rugby ball forward towards the goal line of the opposition \citep{IRB2014}. However, athletes at the same time are only allowed to pass the rugby ball backwards from the position of the ball carrier.  This constraint on joint action dictates that both teams face off against each other in two groups (attack and defence), with the attacking team advancing forward but only passing backwards.  In this situation, the ball carrier's ability to see the rest of his or her team mates is considerably limited (because they are necessarily located behind (or behind and to the side of) the ball carrier.  In this scenario, vocal communication can substitute for visual information about the location of other athletes.  Fundamentally, however, an athletes's ability to generate interoceptive feedback is limited by a restriction of sensory inputs.

  The physiological demands of rugby union also impose limits on cognitive function and thus impinge on the function of IPM in joint action.  Neuroscientist Arne Dietrich proposes that intense physical exercise is a behaviour that puts the organism under enormous stress, forcing it to make energetic trade-offs in the brain \citep{Dietrich2004b,Dietrich2011}.  Dietrich suggests that one experiences of flow and the ``runner's high'' in exercise could be associated with the down-regulation of energetically and inessential regions of the prefrontal cortex devoted to IPM.

  Finally, the competitive structure of joint action in rugby could also serve to compromise the function of IPM.  Like many other Anglo-American interactive team sports, rugby is a competitive activity in which one team of athletes attempts to outplay another team.  Essentially, while one team of athletes is attempting to coordinate behaviours in joint action, the opposing team is attempting---at the exact same moment---to foil and disrupt this coordination.  From a cognitive perspective, this scenario  (commonly identifiable in competitive team sport require coordination in joint action under conditions of extreme uncertainty.


  Thus, while it can be predicted that the joint action demands of rugby will recruit processes of interoceptive modelling in order to predict and organise behaviour in service of hierarchically nested joint action goals, it is also likely that in on-field joint action scenarios the efficacy of IPM will be limited.



   group size parameters of joint action in rugby will impose constraints on the function of these mechanisms, particularly in the case of on-field joint action \citep{Mogan2017}.



  \myparagraph{Action-perception links and direct (extra-neural) coupling}






  The establishment of functional interpersonal synergies between athletes could set the foundation for processes of affiliation and cohesion\citep{Marsh2009}.  It is also likely that high levels of physiological exertion associated with rugby activates neuropharmacological mechanisms of reward.  This mechanism may be important for promoting more generalised social bonding to a larger group of athletes or to in-group members associated with the team, but who do not coordinate in on-field joint action.


  Interocpetive predictive modelling may be more relevant to the off-field demands of joint action in rugby.  Each starting team of seven athletes is complemented by a further five reserves to make up a total squad of 12 who compete in a tournament setting.  In addition, the size of squads that train together outside of official tournaments can range anywhere from 16 to 28.  These group sizes also within the cognitive limits for maintaining face-to-face intimate relationships \citep[thought to be in the realm of 15-25, see][]{Dunbar1992,Dunbar2010}.



  The high-intensity and ``in the moment'' nature of joint action in rugby puts a high cognitive strain on athletes' ability to coordinate successfully.

  COGNITIVE LOAD of on-line joint action.

  The on-line and ``in-the-moment'' nature of rugby makes it hard to employ costly and slow mental models

  Restriction of sensory modalities: ``vision'', junior athletes struggle to couple with environment via action-perception links (extra-neural), not time to simulate during online action: see it do it... need the motoric underpinning.

  Restriction of sensory modalities: ``vision'', junior athletes struggle to couple with environment via action-perception links (extra-neural), not time to simulate during online action: see it do it... need the motoric underpinning.


  \myparagraph{Action-perception links}

  Rugby requires training






  \myparagraph{Direct coupling}
  Despite the uncertainty inherent in rugby's joint action scenarios, there is evidence to suggest that complex coordination dynamics emerge from these scenarios, particularly when co-actors are technically competent and familiar with each other. Passos and colleagues demonstrate the existence of dynamic coupling between dyads and subunits of attack and defence in rugby \citep{Passos2011,Correia2014}, for example functional coupling tendencies emerge between attacking dyads and adapt to specificities of the task environment \textcite{Passos2011}.  Correia and colleagues \textcite{Correia2014} show that coupling tendencies also emerge between co-actors of opposing teams in rugby union, for example, in a 1-on-1 attacker/defender sub-phase.  These results are confirmed in similar joint action contexts in other equivalent sports such as basketball and association football \citep{Duarte2013}. It is likely that the establishment and maintenance of functional interpersonal synergies in rugby joint action depend on an athlete's perception of affordances of the task-specific cognitive system made up of constraints including other athletes, the physical environment, and the rules of the game \citep{Passos2012}.




  Essentially, when taking the field in competitive joint action scenarios, athletes commit to making a series of relatively low-probability bets (predictions) in which they expect to successfully coordinate their behaviour with teammates.  It is conceivable, in this situation, that individual and collective technical competence and shared understanding of the predictions and behavioural tendencies of co-actors would help mitigate the uncertainty inherent in the these bets.  While athletes' predictions are complicated by the counter-action of opponents, and therefore rendered less probable, they are, in an ideal situation at least, based on a level of trust in the viability of individual and joint capacity for movement execution and coordination.  Considered from a predictive coding paradigm, this type of scenario creates the conditions for the activation of heightened neurocognitive reward, particularly when the high stakes bets come off \citep{Chetverikov2016}.


  In this situation, utilising multiple sensory modalities can help broaden an athlete's awareness of the location and intentions of other athletes in a situation in which visual information about teammates is relatively scarce (SOURCE)


  Analysis of this level of activity draws attention to the neurocognitive mechanisms and coordination dynamics associated with establishing and maintaining interpersonal coordinative relationships between co-actors.  Competitive team sports are unique in their ability to orchestrate a joint action environment of high informational uncertainty due to the unconstrained actors in the cognitive system (in the form of athletes of the opposing team).  It is conceivable that action perception links and direct coupling with resoruces of the task environment would buffer against high uncertainty, allowing athletes to make high risk/high reward (albeit educated) predictions about the outcome of joint action.  The ecstasy of the ``click'' of joint action in team sports like rugby, therefore, could arise from successful formation of functional interpersonal synergies in environments of high uncertainty on the playing field.













\end{CJK}
