
\chapter{\label{app4:ethnoSetting}Ethnographic Setting Appendix}


\section{Script for semi-structured interviews \label{sect:semiStructured}}

Introduction:
- Brief explanation of my research
- When was the first time you came into contact with rugby?
- Where are you from?
- History with sport before rugby?

Family:
What does your family think about rugby?
Do they support you playing rugby?
Whose decision was it to start playing rugby?
Do they worry about you getting injured?

Perceived Costs Of Rugby:
Opportunity Cost: what would you be doing if you weren’t here playing rugby?
How do you feel about the fitness requirements of rugby?
What is the feeling like when you’re out on your feet and can’t go on?
Injury: Have you had any major injuries?
What do you think is the hardest thing about rugby?

Perceived Benefits of Rugby:
Do the following motivate you to play rugby? (counter-balanced order)
- Education
- Your parents / Family (i.e., to make them proud/content)
- Beijing Residency
- Future Employment
- Teammates
- Earn Respect from (from society, family, friends)
- Find a girlfriend
- Fun
- Represent Beijing

What is something new that you have learnt through rugby?

Team Membership:
What role do you play in the team?
What is the most important thing for you to do in your current role in the team?

Dissonance/Personal Shortcomings?
Have you ever felt like you have let the team down?  (neijiu 内疚: failure to uphold obligation to another)
In what situations do you feel like that?
How do you react to those feelings?

Flow/team click:
Have you ever experienced the team playing extremely well together; everything clicking, like everyone on the field has a “tacit understanding” of each other (\textit{moqi} 默契)?
When was this experience?
What did it feel like?
How do you think that “team click” can be achieved?


\subsection{Post-interview activities\label{sect:postInterview}}


:
Sorting Task: rank your motivations for playing rugby


Social Network: write down:
1.	The three most competent rugby players in the team
2.	The three people in the team most willing to help others
3.	Your three closest friends in the team



\section{Informal Surveys}



      \subsection{Flow State Scale\label{sect:flowStateScale}}

Can all athletes who participated in today's training session please respond to each of these questions in a private message.  Please do not communicate with other athletes, I want to know about your own experience.  If there are any questions you can ask me directly.
(Please rate each question with a score between 1 and 7: Strongly Disagree 1; Strongly Agree 7).

In the training session:

1. Just now my attention was completely on executing the activity
2.
I am not concerned with how others may be evaluating me (loss of self-control)

      所有参加训练的球员请私信给我回答下面的九个问题。

      请不要和其他球员沟通,我想知道你们自己的感受。要是有问题可以直接问我。
      (等级划分1到7分:完全不同意1分; 完全同意7分)
      在上午的比赛过程中:
      1. 刚刚我的注意力完全正在进行的活动上 (1focus - Q4 from questionnaire )
      2. 刚刚行动似乎是自然而然发生的 (2agency - Q9)
      3. 刚刚我不关心自己的表现如何 (3selfperform - Q22)
      4. 刚刚时间似乎改变了(要么是减慢了,要么是加快了)(4timewarp - Q6)
      5. 刚刚我清楚的意识到自己想要做什么 (5clearintent - Q2)
      6. 刚刚我的能力与情境的高要求相匹配 (6skillchallenge - Q8)
      7. 刚刚我真的很享受那种体验 (7enjoy - Q7)
      8. 刚刚我遇到了挑战,但我相信自己的技能能够应付这一挑战 (8skillchallenge - Q1)
      9. 刚刚我不关心别人可能会如何评价自己  (9criticism - Q14)

      谢谢配合!



      \subsection{Group Membership}



 
