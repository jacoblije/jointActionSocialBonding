\chapter{\label{app5:ethnoField}Appendix: Uncertainty and coordination at the Institute}

  \begin{CJK}{UTF8}{gbsn}




\section{Open scrutiny of performance\label{app5:openScrutinyLZS}}

Open scrutiny of on-field joint action pervaded many off-field settings too.  I was sitting in the canteen with a small group of senior athletes and Coach Shi after we had all finished eating.  Senior player Lu, one of the most talkative at dinner conversations---usually about the intricacies of computer games such as League of Legends----was on this occasion driving a conversation about the technical abilities of certain Chinese rugby players known to the group at the table.  Just as I had observed on the training field, there was in my eyes a distinct openness of discussion about the idiosyncrasies, strengths, and deficiencies of athletes, some of whom were former Beijing athletes, others were athletes from other provinces.

Former Beijing athlete Wang Shengshi, for example, received general mild acclaim for his physical physique, strength, and speed; while also being heavily scrutinised for his deficiencies in rugby's game play.\footnote{I commonly found myself in conversations with Chinese rugby coaches, athletes and enthusiasts about an athlete's quantifiable attributes such as height, weight, and non-rugby attributes (bench press, squat, 100m sprint)---attributes I personally spent very little energy on when making generating an opinion or impression about a player.}  Eventually the discussion shifted to a direct and open discussion of Lu Zhongsheng, or Big Mouth Monkey (Lu Zhongsheng).  Big Mouth was an example of an athlete---who had transitioned from athletics to rugby during 2010-2013---who had extremely attractive physical and athletic attributes (height, weight, strength, speed), but who had yet to mature in his grasp of rugby-specific technical skills and intelligence during game-play.

Lu continued to drive the conversation, drawing attention to Big Mouth's inability to catch the rugby ball following a kick off by the opposition. ``If you ought to train anything [as a forward], it should be catching kick offs'' (你要练啥你该练踢接球!)  The tone of the conversation driven by Lu was direct and critical, but appeared to me neither malicious towards Big Mouth, nor was it polite or accommodating in regards to his technical deficiencies.  Importantly, the development and flow of the conversation was underwritten by Lu's authority as senior athlete.  Never did I witness a public conversation at mealtime in which Lu himself, or Han Xiaolong for that matter, were the subject of scrutiny.

Towards the end of the conversation, others besides Lu began to join in on the criticism, to the point where the group began to laugh in chorus at Big Mouth as he tried desperately to defend their criticisms of his rugby ability.  Finally, as this moment of public humiliation reached its crescendo, coach Shi intervened, saying ``Ok ok enough, or else he will lose face!'' (好了好了,他快丢脸了!).  In this instance, Shi, in his role as coach, was able to socially diffuse the pressure that had mounted on Big Mouth.\footnote{Indeed, in my eyes, Big Mouth's kick off receipts were objectively substandard, and this message was being delivered by one of the most senior members of the group. The criticism was valid from Lu, but perhaps so too was coach Shi's decision to put an end to the criticism---to preserve Big Mouth's social standing as a senior athlete.}  This evidence suggests that attention to coordination of physical movement may take on a form that is deeply relational (and hierarchical) in nature.  In essence, attempts to regulate the uncertainty of joint action associated with rugby's joint action may involve a form of coordination in which attention is directed towards relationships (between athletes and coaches, or between junior athletes and senior athletes (rather than via a relationship between social categories of self and group, for example).






\subsubsection{The entire system must be aligned \label{sect:systemAligned}}
I encountered a number of instances in which it appeared that athletes perceived social coordination to extend well beyond the interplay of immediate social categories of self and the group.  One example occurred the first morning I arrived back for the second major stint of ethnography in the summer of 2016.

I arrived back to a terrible incident in which some of the rugby team had been effectively poisoned accidentally by pesticide sprayed on the hedges outside their dormitory windows.  A few days before I arrived back, the groundskeepers had carelessly sprayed pesticide on the hedges outside the windows of the bottom-floor dormitory rooms of which the rugby team were inhabitants.  Some of the pesticide had made it in to the rooms of the athletes via open windows, and had caused a few of the athletes to develop throat irritations and coughs.
Senior athlete Wei Wenxin, who replaced Han Xiaolong as team captain after Han was promoted to assistant coach following the change in head coach in January 2016), was one of the most seriously affected by the pesticide, and when I sat down with him at breakfast the first day I returned, he explained the story with deep anger and outrage.  Importantly, when I asked him what was going to happen with all of this, he said to me ``the Leadership will have to say something about this, we are all waiting to see how Leadership will respond.'' (领导们必须要说话,我们在看领导们怎么说).  I found it interesting that Wei was so emphatic about the primary role Leadership ought to play in adjudicating this matter.  The role of political leadership in an organisation such as the Institute was no doubt crucially important to manage such issues, but the fact that Wei was so emotional and emphatic about this fact---almost as if the Leadership had a paternal role to play---was revealing of the emphasis Wei placed on an the need for alignment between levels of the social system in which he was situated.

My close interactions with the coaching staff confirmed the importance of hierarchical interdependence.  During my second extended period of ethnographic research I shared a dormitory room with head Coach Wang's offsider, assistant coach Zhu Jing.  Zhu Jing had recently arrived from the rugby program at Xingjiang province, after Wang invited Zhu to join him as assistant coach.   Wang needed Zhu for support: to be his loyal leftenant, and to lighten the load on player-coaches Han and Lu. Despite transitioning to the official position of coach, Han and Lu were still required to train and play, at least until the National Games in 2017.

Zhu Jing was a new arrival to the team, and was thus on unstable ground. At the time he had no official contractual relationship with the Institute, and his position was therefore tenuous and contingent on his relationship with head coach Wang.  Zhu was neither a particularly celebrated rugby player during his time as an athlete at CAU, nor was he a particularly experienced or successful coach.  What he did posses, however, was a close relationship to Wang (they were classmates at CAU). Zhu was valuable to Wang for his loyalty---useful to Wang in his attempt to stabilise his leadership of the team---an inherently slippery platform for social activity (see Section ~\ref{}). In terms of his value to the team as a rugby coach and the Institute, however, Zhu still had much to prove.

If I were in his position, I thought, I would have done everything possible to help head coach Wang and the team—understood as  a category demarcating the athletes and coaches only---to improve performance.  As part of this, no doubt, I would seek to signal diligence and commitment in the way I went about my business as a coach.  After about two or three weeks into our roommate relationship, it became clear to me that Zhu Jing was not so concerned with signalling diligence, at least not to me, and not to the athletes either, at least not in a way that I personally recognised.

Most mornings I would wake up at 7:15am to attend breakfast with the athletes and then collate my field notes and prepare for morning training, starting at 9:00am.  Zhu, on the other hand, would reliably sleep through until roughly 8:50am each morning, and then turn up to training usually at around 9:20am after athletes had completed their warm up.  After lunch, Zhu would return to bed, usually from 1:30pm until 3:00pm, depending on when afternoon training was scheduled.  \footnote{Afternoon siestas are admittedly an institutional in China's work life, but routine indulgence in a two-hour nap was surely taking the institution to its extreme.}  I rarely witnessed Zhu spend any time working on preparing training schedules or other forms of professional development.

As it turned out, head coach Wang also began to notice the way in which Zhu was approaching to his job, and decided to broach the issue with him.  After the final Tournament in Qian An in July 2016, the team went out together for dinner.  This dinner was much less ritualised than the first team dinner organised by Wang earlier that year.  Junior athletes were in one room, and the senior athletes and coaches were in another room, free to socialise within their more naturally occurring social factions.  Towards the end of the evening, after the effects of alcohol had well and truly set in, a heated discussion developed between head coach Wang and assistant coach Zhu. The discussion began with one of Wang's numerous toasts to the group at the table.  Wang said that it was important that the rugby program distinguished itself as an excellent team at the Institute, and there was a need for all senior members of the team (seated at the table) to contribute to this project.

It became clear that Wang wished to emphasise the need for Zhu in particular to ``lift his game'' in this regard. Wang explained that as head coach he was burdened with a lot of administrative work that detracted from his ability to manage athletes and training.  Given that the team was without a dedicated team manager (who would usually manage daily concerns of athletes and organise team logistics), Wang suggested that Zhu ought to improve his contribution. ``As an assistant coach, you need to conduct yourself at a high standard'' (当助理教练你要作为一个高度) said Wang directly to Zhu, indicating that his current standard was not acceptable. Wang spoke politely, but the public nature of these remarks suggested that he was using the opportunity to formally criticise Zhu.

Zhu retorted defiantly, directing attention away from his personal standards to the global situation of the team at the Institute.  Zhu suggested that the most important thing was that the leadership of the program supported the program, and only then could the program achieve a high standard: ``you have to have the support of leadership'' (必须有领导的支持).  Lu and Han, the next most senior team members present, also became involved in the discussion, with Lu siding more with Zhu and Han siding more with Wang.

The next morning, Zhu and I lay in our dormitory room beds, hung-over from ceremonial drinking after the team dinner the night before.  Zhu revived the dinner incident with me, seeking my support for his position.  Zhu insisted that his ability as an individual was fundamentally limited if the Leadership did not appear to offer him support.  I was in no state to record or write down Zhu's remarks, but his message was clear. If the team does not have the faith and support from the Institute (which in this case, it clearly did not have, at least since 2013), then how was Zhu supposed to perform his role?   I could see the point he was making, but the combination of my own deep-seated intuitions about ethics of group membership and self-conduct an unstable stomach, meant that I couldn't help but challenge Zhu on his line of reasoning, and I asked him if he thought there was anything that he should be doing better himself...

After all, I thought that the clear way to counteract lack of support from the powers above was to demonstrate self-competence, to signal diligence and willingness to move independently, despite or in spite of the level of support form leadership.  As I looked at the state of the team, there were clear problems that needed addressing by someone like Zhu.  In the eyes of someone with professional expertise in rugby, there was an apparent lack of clear team discipline, a lack of focus, intensity, and technical precision during training, a lack of physical fitness necessary to survive the challenges of a high level rugby tournament.  These were all things within Zhu's power to address and improve, or so I thought.

But again, Zhu pushed back.  He told me that he had experienced this situation before with Xinjiang Province during the last National Games.  The Leadership started off being very positive about rugby, promising this and that, but then in the end, nothing came through.  The support of leadership is hugely important in China: money, incentives, all of it.  If you don't have support from Leadership then you can't get anywhere.  Zhu was obviously defensive about the situation, attempting to deflect any blame or responsibility for his action (or lack thereof).

Perhaps Zhu's claims were not completely unreasonable in the specific environment---an environment in which hierarchical relationism was a dominant mode of social attention.  Zhu was referring in a sense to the importance of the harmony of the whole system of social relations in which he was embedded as but one node situated at but one level. Individuals can only act with agency if the relationships to which individuals attend also sanction and support that agency, Zhu was effectively suggesting.  In the hierarchical system of relationships at the Institute, the crucial node in the system is the paternal benevolence of Leadership.  In this dominant paradigm, social institutions of self, team, and institute do not provide resources for agency and action in and of themselves---as stable moorings of psychological and social resources \citep{Yuki2003}.  Rather, these social categories appear to act as platforms for the fostering of particularistic relationships through which social coordination is located.  In this way, Zhu's stance mirrored the explanation I had received from Wei Wenxin a few weeks earlier, the morning I first arrived back to the Institute for my second stint of ethnography: ``We are all waiting to see how Leadership will respond.''


  \end{CJK}
