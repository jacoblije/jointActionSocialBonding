

\begin{savequote}[8cm]

  I do not see any way to avoid the problem of coordination and still understand the physical basis of life.

  \qauthor{--- H. H. Pattee  \textit{The role of instabilities in the evolution of control hierarchies} 1976}
\end{savequote}






\chapter{\label{introduction}Introduction}

\minitoc
\section{Abstract}
In this chapter I present the results of ethnographic data collected with the Beijing Men's Rugby team between September 2015 and August 2016.  I find a range of evidence in support of the predictions set out in Chapter 3.  First, ethnographic data show evidence  for a culturally specific terrain of rugby at the Temple institute, particularly the interaction of two different modes of cognition--relational and categorical--within a turbulent modern Chinese history in which cultural practices including sport have been imported en masse in service of a modern Chinese nation state. In particular, the chronic dominance of the relational mode of cognition appears to pervades the structure of the institutions, social norms, and on-field action-perception processes associated with joint action of rugby at the Temple institute.  Instituttion:  Norms: Active Inference:

Cultural modes of group membership and self construal act as ``hyper-priors'' or ``coordination smoothers'' for the structuring of active inference.  The specific context of rugby, and the position of the researcher as an expert observer of rugby from a Western context, while not without problematics, nonetheless serves as an interesting way of exposing the divergent modes of cultural cognition, and denaturalises both received intuitions about rugby as well as China.  Allows us to avoid the potential pitfalls of ethnographic method and resist reifying either the practice or the cultural milieu as an either/or relational/categorical system.

Second, I show how within this culturally specific terrain, contours of testable proximate mechanisms relating to joint action, team click, and social bonding can be identified.   In terms of athlete perceptions of performance, athletes appear to develop strong subjective perceptions of individual and team performance based on specific technical components of rugby.   Athletes---predominantly junior athletes---on the one hand display evidence of anxiety when expectations around team and individual performance are negatively violated.  Conversely, when the same expectations concerning team and individual performance are positively violated, the same athletes are noticeably exhilarated, energised, and motivated.

More senior athletes, by contrast, display the ability to relate to perceptions of individual and team performance expectations with more personal agency and emotional control.  More senior athletes talk in terms of an ability to strategically mitigate the costs of physical exertion and injury risk of joint action in rugby,

These observations are substantiated by survey results that show that mean performance related anxiety of junior athletes is higher than that of senior athletes in the starting team.

These results suggest that perceptions of team and individual performance in joint action are formulated in relation to specific components of team and individual performance in joint action, as well as more general team and individual expectations structured by the specific team context, social norms, and institutional constraints.

In terms of Athletes' feelings relating to notions relating to the construct of ``team click,'' Athletes relate to optimal team performance either with strong but generalised emotionality (predominantly junior athletes), or with a more restrained but more specific understanding of the dimensions and factors required in order to achieve peak team performance (predominantly senior athletes) (perhaps rooted in self-defining memory -- whitehouse?).

In regards to dimensions of social bonding, athletes show evidence prioritise feelings of emotional support and shared goal between teammates, as well as a level of personal connection to team roles and team identity.  Testimonies of team identity vary from the initial excitement and positivity of new arrivals towards, to the realisation of irreversible attachment to rugby of middle-rung athletes, to the out and out identity fusion between senior athletes and rugby, whereby athletes recognise that their self-concept is inextricably with rugby and the Temple institute team in particular.


Finally, contained within the collected data is evidence that individual differences may moderate relationships between perceptions of joint action, team click, and social bonding.  Namely, an athlete's experience and competence with the technical and social dimensions of joint action of rugby at the Temple Institute may impact upon the ways in which they formulate and respond to expectations surrounding and team performance.  Individual differences in technical competence may also have flow-on effects for the formulation of feelings concerning team click and social bonding. It appears that individual personality types may moderate perceptions of performance in joint action, as well as feelings of team click and social bonding. In addition, other facts such as the injury status of the athlete and the level of fatigue and exertion experienced in joint action scenarios of rugby may impact on an athlete's ability to feel the ``click'' of joint action.  These individual differences should be considered in ongoing studies designed to test the relationship between joint action, team click, and social bonding.


Contribution: (See Cohen2007 and CohenWhitehouse2012)
The first time ethnographic evidence has been provided for mechanisms of joint actions
(Wacquant: body and soul), Body for China (Brownell1995)

This ethnographic theorises divergent cultural modes


Cultural specificities may act as ``factors of attraction'' \citep{Sperber2014} that constrain and direct the fixation of cultural variants within and between populations.





\section{Introduction}

re-cap:
1.research question
2. soc cog joint actions
3. Rugby-  and China-specific predictions
4. Predictions









%justifications for ethnography
Recognising that evolutionary change may involve a multilevel model of reciprocal causation requires, the recognition that accounting for human behavioural phenomena will require the consideration of a number of biological, cognitive, and ecological mechanisms that interact via reciprocal feedback loops spanning multiple scales of time and space \citep{Fuentes2015}.  Prior to widespread acknowledgement of this complexity, researchers within the human cognitive, behavioural, and evolutionary sciences have been prone to overlooking local cultural specificity and causal complexity when seeking to generalise to the human species results of studies conducted with mainly Western subjects and methods \citep{Henrich2010d}.  The issue has been shown to be more than a basic sampling problem, given that research agendas and the specific experimental designs to which they give rise are also shaped by the historically contingent assumptions and priorities of ‘‘Weird’’ societies, calling into question the methods of fields such as cultural and developmental psychology \citep{Whitehouse2012}.  It is now clear that the complexity of observable behavioural phenomena can only be addressed by embracing a plurality of methodologies to systematically document variation within and between cultural niches.  The proliferation of anthropological approaches to human behaviour in the last 50 years, while at times threatening the overall coherence of the discipline as a whole \citep{Beller2012}, has also produced diverse theoretical and methodological options for documenting human variation \citep{Fuentes2016a}.
Anthropology is thus well placed to expand upon accounts of group exercise, via methods ranging from ethnographic exploration capable of uncovering novel dimensions of behaviour and generating testable hypotheses, to quantitative techniques---e.g., experimental and mathematical simulation paradigms---designed to test hypotheses \citep{Epstein2006,Fuentes2016}.

In the present study, I address these knowledge gaps in evolutionary approaches to group exercise through a ethnographic study of the social cognition of joint action among professional Chinese rugby players.  Based on emerging research from the social cognition of joint action, in which it is increasingly understood that dynamical coupling of lower-cognitive mechanisms associated with movement regulation set the foundations for higher-cognitive processes such as group membership, I interrogate the relationship between athletes' experience and perceptions of joint action and attitudes towards group membership.  Within this space I isolate the psychological construct of ``team click,'' which refers to athletes' tacit sense of quality joint action coordination.   I then outline the specific predictions of the ethnographic component of this dissertation, before describing the method via which I collected primary data.


















\section{Method}




\subsection{Research Setting and Participants}

I planned to conduct ethnographic research with athletes and coaches of the Beijing Provincial Rugby Sevens Program based at the Beijing Temple of God of Agriculture Sport Institute.  The Program consists of Men's and Women's rugby sevens teams, each with approximately 20-30 athletes and 2-4 coaches per team.  The Program is supervised by one of four vice-principals of the Institute.   Permission to conduct research at the Institute was sought from the vice-principal responsible for rugby at the Institute and the head coach of the rugby Program prior to arriving in Beijing in 2015.  Permission to conduct research was sought from athletes directly at the beginning of the first research period in September 2015. The University of Oxford’s Central University Research Ethics Committee approved this study (SAME/CUREC1A/15-059).

\subsection{Materials}

  \subsubsection{Participant Observation}
  I planned to conduct a number of stretches of participant observation with the Program between September 2015 and September 2017.  I planned to live full-time at the Institute and attend training sessions, team meetings, meals, and participate in any other relevant activities.

  \subsubsection{Interviews}
  I planned to conduct and record a combination of ad-hoc exploratory (unstructured) interviews and scheduled and directed (semi-structured) interviews with research participants.  The script for semi-structured interviews would be designed based on a combination of existing theory and initial ethnographic observations, in order to  understand athlete motivations for participating in rugby, as well as their perceptions of joint action and group membership in rugby.

  \subsubsection{Informal Surveys}
  In addition to conducting participant observation and interviews, I also planned to issue surveys to measure athletes' motivations for, and perceptions of joint action and group membership in the rugby Program.  The surveys would be designed based on a combination of existing theory and initial ethnographic observations.


\subsection{Procedure}

In May 2015, 4 months prior to beginning ethnographic research, I contacted the vice-principal responsible for the rugby Program and the head coach of the rugby Program to ask for permission to conduct research at the Institute.  Following affirmative responses from both, I made plans to conduct two periods of in-depth ethnographic research: 1) 6 months between September 2015 and March 2016, 2) 6 weeks during July-August 2016.

%  \subsubsection{Participant Observation}
Soon after arriving in Beijing at the end of August in 2015 to begin research, I met in person with the vice-principal and head coach of the rugby program to confirm permission for research and to discuss logistics.  I discovered that the Beijing Rugby Program was limited to a Men's Program.  At the time, the Women's rugby Program had yet to be resurrected after the humiliating ``match strike incident'' of the 2013 National Games (see Chapter 3 ~\ref{} for a detailed explanation). The women's program would later be resurrected at the start of 2016, in time to participate in qualification tournaments for the 2017 National Games.  As such, I decided to focus my attention on the Men's program, which at any one time consisted of 25-30 athletes and four coaches.

Both the vice-principal and head coach agreed to provide me with access to the rugby program, a room in the Institute's dormitory, and access the the Institute's 1st level canteen, in exchange for assisting the Program with rugby knowledge and coaching.   Institute had two canteens in which athletes and coaches ate all of their meals.  Athletes and coaches who had represented Beijing at national level competitions were entitled to eat at the 1st level canteen, whereas all other athletes at the Institute, or athletes who were on temporary trial at the Institute, ate at the second level canteen.

During these periods I lived full time with the team at the Institute, attending training sessions, team meetings, meals, and other activities with the team.   All my interactions with research participants took place in Modern Standard Chinese (Mandarin).  I took notes using an electronic note taking application (Evernote, version 6.11) which automatically synced and stored notes created on either my mobile phone of personal computer. I collated, summarised, and tagged these notes weekly or fortnightly.





\subsubsection{Training schedule}

  Every year between March and September, the Beijing men's rugby team competed against other provinces in five national tournaments held in different locations across the country. The period in which I conducted my first stint of ethnographic research (September –-- February) therefore constituted the off- and pre-seasons for these yearly competitions, during which time teams were stationed either in their home province, or at other domestic or international training locations depending on amount of program funding and training strategy.  In 2015, before an unexpected change in coaching team at the end of December, the head coach of the Beijing Men's team had planned to travel to Yunnan province in early 2016 for one month of altitude training before moving to sea-level somewhere in the south (February/March).  Following the coaching leadership change, the team did not leave Beijing until after Chinese New Year (25th February). Training during this period was therefore consistently stationed at the Institute in Beijing, and as such subject to occasional disruption due to Beijing's cold winter weather and air pollution.

  Below is a table of a typical weekly training schedule (see Figure ~\ref{}). A typical week consisted of 10 two and a half hour training sessions, seven of which are on-field rugby sessions, three of which were strength and conditioning sessions (not involving a rugby-specific skills).  In addition, two one hour evening skills sessions were also allocated for junior athletes to hone their basic skills of passing, catching, and game-play.  Athletes lived full-time on campus in the Institute's dormitory accommodation (usually 3 athletes per room), and were permitted to take overnight leave on the weekend after the conclusion of Saturday morning training.  Athletes from Beijing or with family in Beijing would usually take this leave, while the remaining athletes would spend weekends at the Institute.  Generally speaking, the rugby program would break at the end of the national season in September for two weeks, and occasionally around Chinese New Year for 7-10 days, unless New Year interrupts pre-season training plans, in which case training would continue in spite of this national holiday.

  \newgeometry{margin=0.5cm} % modify this if you need even more space
  \begin{landscape}
    \begin{table}[htpb]\caption{Weekly Training Schedule}
      \begin{center}
        \begin{small}
            \begin{tabular}{| c | c | c | c | c | c | c | c |}
              \hline
              & \bf M & \bf T & \bf W & \bf T & \bf F & \bf S & \bf S \\
              \hline
              0600 & Training &  &  & & & & \\
              \hline
              0900 &  & Training & Training & Training & Training & Training &  \\
                \hline
              1500 & Training & Training & & Training & Training & Training &  \\
                \hline
              1900 &  & Training (junior athletes) & & Training (junior athletes) & & & \\
                 \hline
            \end{tabular}
                \label{tab:tournamentData}
          \end{small}
        \end{center}
      \end{table}
  \end{landscape}
  \restoregeometry



  \subsubsection{Interviews}

Unstructured interviews were conducted with athletes and coaches on an ad-hoc basis, often when an informal discussion developed into a conversation relevant to my research questions. In such instances, I would interrupt discussion with the research participant and ask permission to record the remainder of the discussion using the digital audio recording feature on the Evernote application on my mobile phone.

Semi-structured interviews were conducted by appointment in my dormitory room at the Institute 2-3 months in to the first period of participant observation.  During semi-structured interviews, I asked athletes about their personal background (including their family situation), their motivations for adherence to rugby, perceived costs and benefits of adherence to rugby, perceptions of joint action, and group membership. For a detailed script of semi-structured interviews, see Appendix ~\ref{} Figure ~\ref{}.  Questions served only as a loose structure for conversation, and at times either I or the athlete departed from these questions to talk about other dimensions of experience associated with rugby at the Institute.  The order in which athletes participated in semi-structured interviews was randomly assigned.

I conducted all interviews in Modern Standard Chinese (Mandarin) and were recorded with participant consent using digital audio recording feature on the Evernote application on my mobile phone or laptop computer.  Once all interviews were recorded, interviews were transcribed into written Chinese by a native Chinese speaking research assistant using a ``verbatim'' method \citep[i.e., including an account of all verbal and important nonverbal (coughs, pauses, etc) utterances, see][269-70]{Poland2003}.  I checked each transcript for accuracy by comparing the script against the original audio recording during the first phase of open coding analysis (see Section Data Analysis below). I analysed interviews in Chinese and only translated into English data extracts that were included in the main analysis of this dissertation.

%    \subsubsection{Structured}

\subsubsection{Surveys}

   I conducted a number of informal surveys designed to measure athletes' experience of joint action and group membership in training sessions.

\myparagraph{Post-interview surveys}
   Following semi-structured interviews, I asked each athlete to rank 10 different possible motivations for adherence to rugby from most important to least important. Possible motivations for rugby consisted of: \textit{to gain access to eduction}, \textit{to represent Beijing}, \textit{to do Family proud}, \textit{to gain respect from others}, \textit{for (the benefit of) teammates}, \textit{for employment opportunities}, \textit{for money}, \textit{for enjoyment}, \textit{to find a partner}. In addition, athletes were asked to report their 1) three closest friends in the team, 2) the three team members most willing to sacrifice on behalf of the team, and 3) three most competent athletes in the team (see Appendix ~\ref{} for full script). Athletes answered these questions using a pen and paper. I later collated and uploaded these responses to Evernote.


   \myparagraph{Post-training surveys}
    I conducted informal surveys following three training sessions: 1) a session in which (predominantly junior) athletes ran an aerobic fitness test involving straight-line running shuttle-running at and above the aerobic threshold for approximately 30 minutes (known as the ``Beep Test''), and two 90-minute training sessions spread one week apart involving training scenarios that emulated high-intensity match conditions.  After each training session, I administered to each participating athlete via WeChat nine items selected from a Chinese version of the Flow State Scale 2 \citep{Liu2012} designed to measure the nine conceptual dimensions of the flow experience: challenge-skills balance, action-awareness merging, clear goals, unambiguous feedback, total concentration on the task at hand, sense of control, loss of self-consciousness, transformation of time, and autotelic experience \citep{Csikszentmihalyi1990}).  All survey items used a 7-point Likert scale. For full survey details, see Appendix ~\ref{} Section ~\ref{}.


    \myparagraph{General survey administered at two time points (longitudinal)}
    In addition, I asked athletes to comment on experiences of joint action and group membership at two points in time spread three months apart (and either side of the change in coaching staff over the Christmas period of 2015).  These survey items included experience of agency in the team (weak-strong), perceived role in the team (central-marginal), perceptions of individual performance (weak-strong), perceptions of team performance (weak-strong), training intensity (\textit{qiangdu})(light-heavy) and difficulty (easy-hard).  All survey items used a 7-point Likert scale.



\subsection{Data analysis}
Field notes from participant observation, interview scripts, and informal survey responses formed a corpus of ethnographic data that was subjected to a recursive process of ``thematic analysis'' \citep{Braun2006}.  As Braun and Clark \textcite[10]{Braun2006} explain, ``A theme captures something important about the data in relation to the research question, and represents some level of patterned response or meaning within the data set.'' Identification of recurring themes was guided by (but not limited to) the research question and theoretical predictions of this dissertation, outlined initially in Chapters 1 and Chapter 2 (and then refined in Chapter 3 in accordance with the specific research context of rugby in China).  Themes were identified on both explicit (semantic) and implicit (latent) levels of the data \citep{Boyatzis1998}. Theoretical predictions and relevant existing research concerning the social cognition of joint action helped direct analysis of the latent level of the data.

The thematic analysis involved three stages that unfolded in a recursive (rather than linear) fashion \citep{Braun2006}. In phase one, I familiarised myself with the each data set in the corpus (field notes, interview transcripts, and informal survey responses) and tagged relevant data extracts with theoretically-guided ``codes.'' For example, upon encountering Hongwei's description of his position in the team in his interview transcript (cited in the Introduction ~\ref{}), I tagged this with codes such as ``group membership,'' ``mutual support,'' ``emotional support,'' ``knowledge of team roles,'' ``signalling commitment to team'' etc.  My coding system was thus directed by (but not limited to) pre-identified theoretical variables relating to 1) athlete perceptions and expectation violations surrounding joint action, 2) perceptions and feelings associated with the phenomenon of ``team click,'' 3) understandings of and feelings relating to ``group membership'' and social bonding, as well as 4) possible moderator variables of technical competence and personality type.  For each data set, I created a data frame using Microsoft Excel (Version 14.7.1) in which research participants formed the rows, and distinct codes formed individual columns. Data extracts from interviews and field notes were imputed into the matrix, with an emphasis on including data surrounding the code's target, in order to preserve context \citep[see][]{Bryman2001}.

In phase 2, I sorted the different codes into potential themes and collated all the relevant coded data extracts within the identified themes and judged on the dual criteria internal homogeneity of codes within themes (coherence) and heterogeneity of codes between themes (distinction) \citep{Patton1990}.  I then produced a master data-frame (participants x themes), in which data extracts from all data sets were included.  In phase 3, I generated a definition of each theme, and a refined list of data extracts capable of representing that theme in subsequent analysis \citep{Braun2006}.






\section{Results}


\subsection{Descriptives}
I analysed data on a total of 26 Athletes ($avg. age = 21.3, range = 18-27, SD = 2.96$) and four coaches.  Athletes were included in data analysis if they participated in 1) a semi-structured interview, 2) at least one informal survey relating to experiences of rugby training and group membership, and 3) at least 2 months of training at the Institute.  See Table ~\ref{tab:ethnoDescriptivesTable} for a summary of athlete attributes, including team status, contract status, etc.

% latex table generated in R 3.5.0 by xtable 1.8-2 package
% Wed Jun 20 16:44:16 2018
\begin{table}[ht]
\centering
\begin{tabular}{ll}
  \hline
row & Overall \\ 
  \hline
n &    26 \\ 
  Age (mean (sd)) & 20.96 (3.17) \\ 
  ResearchCategory = Senior (\%) &    10 (38.5)  \\ 
  TrainingAge (mean (sd)) &  3.34 (2.02) \\ 
  YearsInTeam (mean (sd)) &  2.59 (1.80) \\ 
  AthleteStatus (\%) &     \\ 
     Master Sportsperson &    10 (47.6)  \\ 
     Level 1 &     6 (28.6)  \\ 
     Level 2 &     5 (23.8)  \\ 
  ContractStatus (\%) &     \\ 
     Permanent Employee &     1 ( 3.8)  \\ 
     Full Time Contract &     7 (26.9)  \\ 
     Training Contract &     5 (19.2)  \\ 
     Student Contract &     6 (23.1)  \\ 
     Trial &     7 (26.9)  \\ 
  EducationLevel (\%) &     \\ 
     Graduate &     1 ( 3.8)  \\ 
     Undergraduate &    12 (46.2)  \\ 
     High School &    10 (38.5)  \\ 
     Middle School &     3 (11.5)  \\ 
  HomeProvince (\%) &     \\ 
     Shandong &    11 (42.3)  \\ 
     Beijing &     6 (23.1)  \\ 
     Jiangsu &     3 (11.5)  \\ 
     Liaoning &     2 ( 7.7)  \\ 
     Hebei &     2 ( 7.7)  \\ 
     Heilongjiang &     1 ( 3.8)  \\ 
     Fujian &     1 ( 3.8)  \\ 
  PreviousSport (\%) &     \\ 
     Athletics &    16 (61.5)  \\ 
     None &     8 (30.8)  \\ 
     Basketball &     1 ( 3.8)  \\ 
     Football &     1 ( 3.8)  \\ 
   \hline
\end{tabular}
\caption{Athlete Desciptives (n = 26)} 
\label{tab:ethnoDescriptivesTable}
\end{table}



Athletes were either already contracted with the Program as either students, full-time contracts, or formal employees of the institute (in the case of ) (n=13), or aspiring to become students, or contracted athletes (n=13).  All athletes lived and trained 6 days a week at the Institute, and would occasionally attend university or high school classes as part of their ongoing education commitments.  A head coach and assistant coach looked after the day to day organisation of team schedules and training, and these two coaches were assisted by a further two player-coaches, who were in the gradual process of transitioning from athlete to coach status.  One of the Institute's four vice-principals was responsible for the management and administration of the rugby program, and is occasionally present at team meetings and national competitions.

Athletes were, generally speaking, and from what I could gather, from relatively modest socio-economic backgrounds.  Most athletes were from suburban and rural areas of northern China (Shandong (11), Beijing (6), Jiangsu (3), Liaoning (2), Hebei (2), Anhui (1), Henan (1), Heilongjiang (1)).  The squad consists of 10 fully contracted senior athletes (\textit{xieyi}), three provisionally-contracted athletes \textit{shixun}, and six student-athletes \textit{erjiban} who do not receive a salary but receive training, food, board, and educational support (see Table ~\ref{}).  The remaining athletes (7) were classed as athletes in training \textit{jixun} and were effectively on a trial-arrangement until they showed promise or else withdrew from the squad, either voluntarily or upon suggestion by the head coach.  Provided that they met the relevant academic and athletic requirements, contracted and student athletes could attend the Beijing Sports University—--considered to be the country's most prestigious sports university and one of China's top ``brand universities'' (\textit{mingpai daxue} 民牌大学).

The average rugby training age (years spent playing rugby) of Beijing athletes was 3.12 years ($range = 0.16 –-- 10 years$).  Contracted senior athletes ($average age = 24.3 years$) had trained for an average of 5.4 years, whereas the average training age of junior non-contracted athletes (average age = 19.3 years) was 1.7 years.  Over half of the athletes had a background in other sports (15 athletes from track and field, one from football, one from basketball), usually beginning part-time or full-time physical training at the age of 11-13.  Those who transferred to rugby did so either at the beginning of senior high school (16 years) or at university age (18 years).  The rest of the group (9) had no particular sporting background before starting training with the Program, and were scouted by the head coach of the Program or by school athletics coaches based on their basic athletic attributes (running speed, strength, coordination, and potential for physical growth).  Of the 26 athletes in the squad, three junior athletes who were part of the squad when I arrived in September 2015 have since left, and three new athletes arrived during the time I performed research. This movement of non-contracted players in and out of the Program was quite common.


  \subsubsection{Interviews}

  In addition to 26 semi-structured interviews, I also conducted 6 unstructured interviews with three members of the Program (the head coach and the two most senior athletes) and two former coaches of the Institute.

  \subsubsection{Informal Surveys}

  All 26 athletes participated in the post-interview survey tasks.  For post-training surveys, 12 athletes participated in the survey following the Beep Test training, 16 athletes participated in the survey following the first match-like training session, and 15 athletes participated in the second match-like training session a week later.  A total of 23 athletes participated in the general survey administered twice with a three month interval.







\section{Analysis of Study Predictions}


  \subsection{Culturally specific terrain}


    Prediction: Given the specific history of sport and rugby in China, and considering evidence concerning institutional, and social norms, as well as cultural cognition, I predicted that ethnographic evidence for the core predictions of this dissertation would be located in culturally specific terrain.


    \subsubsection{Institution as platform}

Whose team? Our team?


    %P: Institution as platforms with incentives and constraints, but not sacred in and   of themselves, less sanctimony around the institution than my intuitions prepared me for.

    %F: motivations are not for teammates, but for personal strategic life-course opportunities.
        %Rugby/Employment: Old heads (1st team) and undergrads (BYH, MXK etc)
        %junior: Young bucks (Chaoyang crew) and freshers (SHW etc)
        %family strong, team weaker but still present; coach??
        %F: CSC: not motivated, kicking the bucket and conversation with ZPH
        %F: MHT explaining connection between his mood and study

    %E: Any sanctity around the team appears secondary to more immediate strategic goals

    %P:coaches organise clans :
    %F: ZPH --> WCY transition:
       %WWX criticism of ZPH; subsequent alignment with WCY & ZJ
       %F: When ZPH left, he took with him XG and LJX (LJX stayed committed to the institution of BJM, present at National Games 2017).
       %F: WCY's clan from Shunyi, Hebei
    %E:

    %P: Power of regulation:
    %F: BYH, SHL, and WCY privileged position: contrast with HXL and LP (and ZPH).
    %E:Setting up regulations in institutions to incentivise and constrain, but not something that generates trust in the category of the institution itself.

    \subsubsection{Social norms}

    There's no I in team:
    Tension between encouraging self-determination, but then needing the ``team'' to move

  % HXL team Dinner speech and Johnny Zhang's Kids... what is going on here?
  % Team as team, but also as family (Team meeting, WCY, etc)
          % Self-determination in relational system:
                  %Younger players: Coach agency but my choice (figures?)


                  %Younger player aware of team roles
                  %Senior players: WW, CSC aware of cycle from young to old, whereas others are very critical of younger players not motivated.

          % ``Its all very complicated (in China)''

          %ZJ: can't move without leadership support (ZJ) conversation

  %HXL: return to his period of absence from the team after final tournament.

    \subsubsection{Action-Perception}



%can these processes be located in action-perception (active inference)?

          % Action scrutiny:
            %reference to BBall:

          % distributed dissonance:
            %training would fall apart when senior players absent
            %


  \subsection{Contours of Generalisable Mechanisms}

    \subsubsection{Performance in Joint Action}

      \myparagraph{Performance related anxiety (predominantly Junior)}

          \subparagraph{Team Awareness}
        %Anxiety around complexity of rugby, and the ``awareness'' required to execute team based joint-action:
        %Anxiety included reference to specific components of performance
        %TEAM: attack, defence, support play, communication

          \subparagraph{Individual Social Shame}
        %Anxiety around individual letting the team down due to individual mistakes
        %Anxiety included reference to specific components of performance
        %IND: contact, tackle, passing, decision making, support play in attack

      \myparagraph{Positive Violation of expectations: Performance related exhilaration and generalised emotionality (predominantly Junior)}
        %SWH story: the buzz and glow from acquisition of skills
        \subparagraph{Ind components}
        %WZF: explanation of the first side-step
        %YC: Fending - the feeling of picking up
        \subparagraph{Team performance}
        %WZF: likens team performance to side-step
        %HXL: rugby wasn't interesting (wasn't motivated until he started to pick up the team dimensions)

  \myparagraph{Strategy in individual performance and deflection of responsibility (more predominantly Senior athletes)}

          \subparagraph{Team Awareness: Agency over, deflection of responsibility}
       %Deflection of responsibility towards junior athletes: criticism of junior athletes for not committing, playing computer games, being complacent.  (Irony that LP was one of the most vocal when discussing computer games at the dinner table).
       %Others more generous and circumspect: WW and CSC realise that its a progression, and that individuals
       %EXPLAIN: reduction of dissonance via deflection
        \subparagraph{Individual performance: strategy}
       %MHT, CSC, HXL, WWX, MC use of experience to strategically avoid over-exertion and injury risk.


      \myparagraph{Survey Results}

      SURVEY RESULTS: relatively equal levels of flow overall, but higher performance related anxiety in junior athletes, particularly in the scratch matches, which required high levels of technical competence and joint action coordination.

      %EXPLAIN: perception of performance in relation to social expectations of the team: joint action participation, as well as individual responsibilities (team member, self-determination)

      %Senior athletes talk with more composure regarding performance, any deflect responsibility for being the agent of team performance, and show strategy regarding how to regulate energy expenditure.





      \subsubsection{Team Click}

Either:
        \myparagraph{Generalised emotionality, equating click with bonding (Junior)}
Or:
        \myparagraph{inability to conceive of click, not qualified to talk about it}


Either:
        \myparagraph{Familiarity and granularity/precision (Senior)}
        %HXL: Aura/atmosphere of the teams
        %WW: flow description
        %WZF: consideration of all the individuals
        %CSC:
        \myparagraph{Problematised}
          %Juniors for not being committed (Undergrad loafers)
          %Coach/Leadership for not supporting (LP)
          %China:
            %the system - No initiative, Chinese society too complex
            %the culture - Confucian education
          %Nostalgia for old regime pre 2013: HXL, LP,
          %E: a cognitive strategy in reorganising information in a way that maintains coherence (Nowak 2017)



  \subsubsection{Social Bonding}

    \myparagraph{Emotional Support}
    %MC, LZS, SHW, GJP
    \myparagraph{Shared Goal}
    %WW:

    \myparagraph{Identity Fusion}
    %Fun/interesting/compelling
    %Attachment: MXK, LZS,
    %Fusion: HXL, LP,




  \subsubsection{Moderator Variables}
      \myparagraph{Technical Competence}
      \myparagraph{Personality}
      \myparagraph{Injury}
      \myparagraph{Fatigue}



























\section{Discussion}

The cultural hyper-priors of JA - Team Click - SB <- factors of attraction for cultural evolution.
Relational and Categorical modes in interaction and flux in Chinese sport.

\section{Conclusion/Contribution}
Cross-cultural psychology / anthropology, CAT

recap


























%%%%%%%%%%%%%%%%%%%%%%%%%%%%%%%%%%%%%%%%%%%%%%%%%%%%%%%%%%%%%%%%%%%%%%%%%%%%%%%%%%%%%%%






First week in China (introduction vignette):
- ``rou'': mysterious carnal sport --> social community
- XNT fall from grace:
- Qingdao:
    - basketball (actionScrutiny)
    - Asia 7s: (Qi gaige & Li sheng) (system, incentives)
    - National 7s: Beijing:WCY out of the huddle


XNT:
- Sport as a upward mobility strategy:
    - Many from athletics (IND) to rugby (TEAM)
- Fall from grace --> Lack of click?
- System: coach ()
- Zero to one: new comers, uni students, senior players, old motherlandStrength

In this system, how do they interaction
