
%\begin{savequote}[8cm]
%By long years of military experience he knew, and with the wisdom of age understood, that it is impossible for one man to direct hundreds of thousands of others struggling with death, and he knew that the result of a battle is decided not by the orders of a commander-in-chief, nor the place where the troops are stationed, nor by the number of canons or of slaughtered men, but by that intangible force called the spirit of the army, and he watched this force and guided it in as far as that was in his power\dots
%  \qauthor{--- Leo Tolstoy \cite{Tolstoy1956}}
%\end{savequote}

\chapter{\label{generalDiscussion}General Discussion}


\section{Overview}

 %- what this thesis was all about and set out to do, and
 %- a brief summary of what each chapter did (ie. what it was dealing with)
 %  ...this is basically the 'abstract' for the final Discussion chapter, and should be <1 page...dont talk about results just the overall remit of the whole thesis, though you can refer back to the theory as outline in Chapter xyz)

Group exercise in human sociality is an intriguing puzzle, and scientific explanations for its cross-cultural and historical prevalence, as well as its various component costs and are currently limited.  This dissertation aims to develop cognitive and evolutionary understanding of group exercise through a close study of the social cognition of joint action in group exercise.  In particular, I investigated the relationship between joint action, ``team click,'' and social bonding among professional rugby players in China.

This dissertation contains four main components.  First, given the lack of existing theory accounting for a link between diverse and often complex real-world joint action and social bonding, I formulate \textit{a novel theory of social bonding through joint action} (Chapter ~\ref{chap:theory}).  In particular, I identify the widely recognisable phenomenon of ``team click'' as an instance in which the bonding effects of joint action appear to be maximised.  Subsequently, I test this novel theory through three step-wise empirical studies of professional Chinese rugby players.

I begin with an ethnographic study of the Beijing men's rugby team (Chapters ~\ref{chap:ethnoSetting}\nobreakdash~\ref{chap:ethnoResults}), in which I attempt to identify evidence for ``in-the-world'' existence of
theorised processes of social cognition in joint action.  Before I proceed to a discussion of evidence hypothesised relationships between joint action, team click, and social bonding (Chapter ~\ref{chap:ethnoResults}), I first account for the way in which the specific cultural terrain of contemporary China serves to pattern processes of joint action and group membership (Chapter ~\ref{chap:ethnoSetting}).

Bolstered by the confirmatory results of this ethnographic study, I seek to apply the theory of social bonding through joint action to a larger population of professional Chinese rugby players.  In honour of the function of cultural affordances to structure and pattern processes of social cognition in joint action, in this dissertation I commit to a focussed study of rugby players in China, rather than making a direct leap to group exercise settings of which I am less familiar.  I design a survey study which I administered to athletes before, during, and after a high stakes National Rugby Tournament (\citep{chap:}).  Subsequently, I developed a controlled experiment paradigm in which I varied perceived difficulty of joint action in a between-subjects design.

In sum, this dissertation offers a novel theoretical approach to scientific explanations of group exercise, as well as initial empirical evidence to suggest the viability of a hitherto poorly understood relationship between joint action and social bonding.




\subsection{Summary of findings}

    %Moving through chapter by chapter (of those that were not theoretical), summarise what your overall findings were for each...
    %  -1 paragraph per empirical chapter.
    %    -You could do this for the ethnographic section too...
    %    -to write this, you can basically paraphrase/copy things you have said in each chapter's abstract.
    %    - For each paragraph, as you move through chapter by chapter, include a page reference to where that chapter begins...this way the reader can easily flip back to the chapter to re-familiarise themselves with any specific elements

In this section I summarise this dissertation's main findings.  Given that a novel theory of social bonding through joint action is a core contribution of this dissertation, I also include a summary of its key components, in addition to the key findings of the three empirical studies.

\subsubsection{A novel theory of social bonding through joint action}
While anecdote and ethnographic observation suggest a relationship between interpersonal movement coordination and social bonding, this relationship has not been thoroughly theorised in cognitive and evolutionary explanations of group exercise.  The prevailing social high theory of group exercise and social bonding currently utilises behavioural synchrony as a stand-in for successful coordination in joint action (see Chapter ~\ref{chap:intro} Section ~\ref{sect:socialHigh}).  While behavioural synchrony is an obvious place to start, a focus on behavioural synchrony alone threatens to occlude the complexity of real-world joint action scenarios in group exercise, as well as the cognitive complexity associated with the process of two or more brains and bodies ``clicking'' together in time and space.  In this dissertation I consider the argument that joint action is underwritten by a generalised synchronisation on a \textit{cognitive} (rather than behavioural) level \citep[see Chapter ~\ref{chap:theory} Section ~\ref{sect:sect:activeInfJA};][]{Friston2015,Friston2015a,Kelso2013,Fusaroli2014}.  In this conception, the experience of ``in the zone'' in joint action is not necessarily contingent on exact equivalence or alignment of behaviour, but rather on an underlying interdependence (a dynamic coupling) between the cognitive processes of action and perception of two or more individuals.  For group exercise contexts involving complex joint action tasks spanning multiple time scales and recruiting multiple sensory modalities, generalised synchronisation offers a much more powerful theoretical foundation for understanding processes of social cognition of joint action.

The theory of ``active inference'' \citep{Friston2010} provides a useful and testable framework in which the ``visceral'' and mysterious dimensions of social connection in joint action are laid bare.  In distinction to traditional models of cognition and action, active inference offers a way in which to conceive of perception, action, emotion, and prediction as flexibly integrated into one overarching inferential process whose cognitive resources are distributed throughout brains, bodies, and bio-external affordances of the task environment.  In particular, I account for a relationship between 1) perceptions of joint action and team click (Chapter ~\ref{chap:theory} Section ~\ref{sect:sect:JASuccessTeamClick}), 2) team click and social bonding Chapter ~\ref{chap:theory} Section ~\ref{sect:teamClickSocialBonding}, and 3) perceptions of joint action and social bonding, mediated by team click (Chapter ~\ref{chap:theory} Section ~\ref{sect:sect:JASuccessSocialBonding}).  In addition, I also account for the influence of individual and cultural variation on the theoretical model of social bonding through joint action.

\subsubsection{Ethnography of the Beijing men's rugby team}

Equipped with a novel theory of social bonding through joint action, I take a deep dive into the world of professional Chinese rugby players. Over three separate stints of ethnographic research (10 months in total), I follow the Beijing men's rugby team as they train, compete, and socialise at the Beijing Temple of God of Agriculture Sports Institute.  This ethnographic study reveals two sets of findings.  First, I document the way in which dominant cultural affordances structure and pattern social attention and behaviour  (Chapter ~\ref{chap:ethnoField}).  I identify evidence of these patterns of social cognition as they play out at the level of institutional behaviour, group membership to the team, and on the rugby field itself.  In particular, consideration of culturally dominant schemas of holistic reasoning, hierarchical relationism, and self-cultivation (all loosely grouped under the banner of ``Confucianism'') can help account for observable emphases on fostering and maintaining relational ties over the preservation of integrity of social categories of institution, team, and even self-and-other.  A consideration of the cultural affordances that facilitate social interaction in the Beijing men's rugby team is crucial for the subsequent task of identifying evidence for the core predictions of this dissertation.

Only with an understanding of culturally dominant modes of social cognition could I understand, for example, how a complete stranger to rugby could train for one week and represent Beijing at a professional National Tournament (see Chapter ~\ref{chap:ethnoField} Section ~\ref{sect:leadersFinalSay}); that Han Xiaolong's self-focussed speech at a team dinner was ultimately a pro-social act (see Chapter ~\ref{chap:ethnoField} Section ~\ref{sect:IinTeam}); or that active and open scrutiny of joint action by authoritative members was a central mechanism for regulating quality of interpersonal coordination in a hierarchical system in which leaders were motivated to preserve face and boundaries between self and other were relatively porous (see Chapter ~\ref{chap:ethnoField} Section ~\ref{sect:JAdistributedHierarchy}).

Equipped with this evidence of affordances specific to the Beijing men's rugby team, I subsequently outline evidence pertaining to the core predictions of this dissertation.  First, I show that athletes devote considerable attention to perceptions of joint action.  Athletes focus both on their own contribution to joint action, as well as perceptions of joint action as a team.  Within individual and team performance, athletes share general impressions of performance, as well as attention to specific components of performance (e.g., tackling, passing, coordination in attack, defence, and so on).  When recounting experiences of performance in joint action, athletes appear to fixate on instances in which their expectations surrounding joint action are either positively or negatively violated.  Positive violation of expectations in relation to individual and collective contributions to joint action in particular can produce feelings of exhilaration; whereas negative violations of expectations around individual and team performance can induce rumination and stress (see Chapter ~\ref{chap:ethnoResults} Section ~\ref{sect:expectationViolation}).  Taken together, these results lend support the Predictive Joint Action Model (PJAM) proposed by Pesquita and colleagues, and in particular call attention to the role of surprise (expectation violation) in generating salient social experiences. Specifically, the experience of positive violations of expectation around team performance could explain a relationship between joint action, team click, and social bonding.

In addition to perceptions of performance in joint action, most athletes also report positive or optimal experiences of performance in joint action.  I find evidence for tacit understanding between and reliability of teammates, aura or atmosphere around the team, flow and coherence in joint action, and a blurring of boundaries between self, other, and team.  In this dissertation I group these components of peak team performance under the banner of ``team click.''    These results provide evidence for a relationship between perceptions of joint action and team, such that more positive perceptions of joint action are associated with components of team click.

I also find evidence that athletes attend closely to processes of group membership.  Despite the explicit utilitarian motivations to adhere to rugby, which run either parallel or at times in competition with collective goals of the team, it is clear that athletes experience high levels of emotional support, shared goals, and social identity by virtue of adherence to rugby.  Ethnographic results suggest that the experience of click in joint action may set the foundation for more abstract processes of social bonding, such as social identity formation, or perceptions of a common goal. The story of Sun Hongwei (see Chapter ~\ref{chap:theory} Section ~\ref{sect:SHW}) is a powerful demonstration of the qualitative difference that feeling the click of joint action makes to an individual's sense of agency and perceived support in a social environment.

In addition to evidence in support of the key predictions of this dissertation, I also find evidence in support of hypothesised moderators of the relationship between joint action and social bonding.  Generally speaking, more junior athletes in the group tend to reflect on concepts relating to performance and team click with more emotional arousal but less granular detail about how such experiences arise.  More senior athletes, by contrast, speak with less emotional arousal and more granular detail about how, for example, team click feels and what processes of joint action are responsible for its occurrence (see Chapter ~\ref{chap:ethnoSetting} Section ~\ref{sect:teamClick}).  These results suggest that the accumulation of technical competence in joint action could allow athletes to develop more nuanced schemas for joint action which mean that emotionality of these experiences is reduced.  More junior athletes will possess more rudimentary models for predicting joint action, which could translate to more volatile emotionality (surprise).  In addition to technical competence and experience, I also find evidence that personality and factors such as fatigue and injury could play a role in moderating a relationship between joint action, team click, and social bonding (see Chapter ~\ref{chap:ethnoResults} Section ~\ref{sect:mods}).

\subsubsection{\textit{In situ} survey study}

Results of a survey study conducted \textit{in situ} at a National Rugby Tournament provided more robust confirmation of these findings in a larger and more representative population of Chinese rugby players.  Before, during, and after the Tournament, Athletes responded to a series of questions concerning perceptions of 1) performance in joint action, 2) team click, and 3) social bonding.  Athletes also answered questions concerning objective and subjective ratings of technical competence, personality type, injury status, and fatigue (as well as general ID variables such as age, sex, team membership). In addition, I collected objective performance data for each individual competing in the Tournament.

I operationalised key variables of interest by reducing survey items to factors, before testing the linear relationship between these factors.  Linear Mixed Effects Regression models controlled for various moderator variables (technical competence, personality type, fatigue, and objective performance in the Tournament), as well as variance attributable to an individual athlete's team membership (or alternatively the random variance associated with repeated samplying from individuals, in the case of the repeated measures analysis of overall Tournament data).  Results revealed positive significant relationships between perceptions of performance in joint action and team click, more positive expectation violation and team click, and team click and social bonding.  In addition, a mediation analysis suggested that team click fully mediated a relationship between joint action success and team click.  LMER models in which positive expectation violation around team performance were included as the predictor variable were not suitably robust to support a mediation model.  In sum, these results provide real-world \textit{in situ} for a
theory of social bonding through joint action.

\subsubsection{Controlled field experiment}
Results of the field experiment confirmed results of the survey study (and ethnography) in a more controlled setting.  In this novel experimental paradigm, athletes performed the same training drill, but received different pre-experiment primes in which the difficulty of the impending drill was framed as either very difficult or not very difficult at all.  Athletes were given no explicit feedback concerning success or failure of their participation in joint action (as they would in a competitive game, for example, in which one side is declared winner and the other side is declared as the loser). In addition, the training drill required only low to moderate intensity physiological exertion, which allowed for a greater focus on the role of joint action complexity and uncertainty as the source of positive violation of expectations.  Effectively, this controlled for the possibility that extreme levels of physiological exertion found in a tournament setting contributed to a relationship between joint action and social bonding.

Results of the experiment revealed an interaction effect of expectation violation and condition, such that more positive violations of expectations around team performance predicted team click, only in the high difficulty condition.  Similarly, team click predicted social bonding, only in the high difficulty condition.  The direct relationship between expectation violation and social bonding was significant only when modelled as a ``mediated moderation''---i.e., the relationship between expectation violation and social bonding was significant when explained by the effect of team click on social bonding in the high difficulty, but not the low difficulty condition.

In summary, results from the empirical components of this dissertation provide promising initial evidence of a relationship between joint action and social bonding in the case of professional rugby union players in contemporary China.  The ethnographic study of the Beijing men's rugby team shows clear evidence that athletes focus attention on performance in joint action.  In addition, athletes recognise the phenomenon of team click and naturally associate team click with processes of group membership.  In particular, the experience of surprise in joint action, in the form of positive and negative violations of expectations surrounding performance, appears to be a powerful source of positive and negative affect, and may provide an interceptive basis for social bonding in group exercise.

Results from the two further empirical studies confirm these results.  A survey study of professional Chinese athletes in a high stakes professional Tournament revealed a strong significant relationship between perceptions of success in joint action and social bonding, mediated by team click. More positive violations of expectation around team performance significantly predicted team click, but the direct relationship between expectation violation and social bonding was not significant.  Promisingly, a controlled field experiment replicated the basic trend of these results.  In particular, team expectation violation predicted team click and social bonding, only in the condition designed to amplify uncertainty around joint action and therefore increase positive expectation violation.



  \subsection{Contributions to the literature}
%Contributions to the literature (arrange this section around the core questions that your thesis centred around...so this section is not about chapter by chapter results, but cutting across the whole thesis,
%    -what were the overall contributions to the literature that your thesis hereby provides?
%    - Mention which chapters' results fed into each contribution.
%    - This section is a genuine synthesis, across the chapters, of all your findings...it is very important for the examiner to see that you can zoom out and name and summarise the 'golden threads' of the thesis contributions.
%    - These will also be mentioned in your Chapter 1 of thesis, preemptively, and you can signpost your reader to this last chapter's section for them to read about them in more detail.
%    - Each contribution can be maybe 1-2 pages long, but it depends really on how you are partitioning the material.

---> First empirical (ethnographic and behavioural) application of PJAM and active inference

Joint action --> team click?
surprise, emotion, agency (interceptive embodied focus)

Team Click --> social bonding?
Generalised synchronisation (narrative, implicit common ground)
ground for higher cognitive processes of group membership

Cultural variation:
recasts cultural psychology using affordance theory, collapses binary distinction between east and west, etc.



Group exercise to social bonding (update the social high theory?)

Group exercise to social cohesion

Cognitive and evolutionary explanations of GE





\section{Limitations and directions for future research}
  %(brief paragraph saying that you will explore some important considerations regarding the interpretation of your results, limitations of your studies (if you can identify a series of main limitations, list them here, then go into them one by one in the next section) and avenues for future research.

\subsection{Limitations}
  %2.1 Limitations of the studies
Overall:
-Theoretical:
    The correlation between team click and social bonding
-Empirical:
  Ethnography and self-report as main sources.
  Surprise needs to be better operationalised

-Ethnography:
    Language/culture barrier/ positionality of researcher


-Survey Study:
    Expectation violation;
    No measure of flow?

-Controlled Experiment:
    Sampling: size, only Beijing and Shandong, etc.
    execution: modified conditions,
    Design: The lack of implicit interoceptive preparation; it was all ``priming'' -- exteroceptive; how to make it more interoceptive? (like the thief situation)
    (Discuss the placebo effect and cognitive dissonance annecdotes!)
    Beedie
    Aronson youtube



\subsection{Generalisability of results and outstanding research questions}
   %- Generalisability is to do with, for example, to what extent your findings may be specific to China, or more generalisable to other cultural contexts etc.
[China]:



[Group Exercise]:
  Beyond high stakes interactive/competitive sport?


   %- Outstanding research questions is to do with what you think future research should focus on...



   %- you could split these two sections up, if you see fit, and change the order in which you include them, up to you).

\section{General implications of findings}
  %(what does this all have to do with bigger picture (e.g. evolutionary) theories about bonding etc... basically this section, like in your Intro chapter, should be a massive zooming out to bigger picture stuff..1-2 pages)




The combination of predictive coding and active inference provide the theoretical machinery necessary for a thermodynamic understanding of cognition.  The result is a welcomed conceptual shift away from individual-centred computational models of information processing (originally inspired by the mechanics of the electronic computer), which tend to render cognition as the final product of a linear sequence of sensory perception, amodal mental representation, and action selection \citep{Lewis2005}.  By contrast, thermodynamic cognition offers a model in which perception, representation, emotion, and action are functionally and temporally integrated in the service of informational processes of free energy minimisation.  Importantly,  to conceive of social cognition in this way, as an embodied, embedded, and immediate process of inference, centralises the role of automatic movement regulation strategies---traditionally classed as ``lower-cognitive'' processes---in establishing and maintaining the transfer of information between individuals, within groups, and throughout populations---traditionally thought to be executed by  ``higher-cognitive'' processes \citep{Claidiere2014}.
In the following section I outline the degrees of freedom problem associated with human movement, and explain how humans have devised novel cognitive solutions to it.  Many of these solutions can be observed in joint action scenarios.


[Implications for understanding Group Exercise]


[Implications for cognitive and evolutionary Anthropology]
      * Supports CAT over meme theory and gene-culture
          * Makes a challenge to representationalism of CAT
      * Recast fusion from an active inference perspective, for example


      * INTEROCEPTIVE FOUNDATIONS of RELIGIOUS BELIEF: (Pezzulo2013:909)
      Note, however, that even in the bogeyman case, interoception is not the only available information.
          * Part of the grounding and the understanding of nonperceptual concepts such as the bogeyman are in terms of tales heard from parents or friends (cultural inputs);

          Indeed, people imagining a thief (or the bogeyman) in the night, or even Capgras patients, are not performing any irrational inference. Rather, their inference maximizes the probability of the correct hypothesis, given the evidence and the relative weights of the information sources..
    This brings Cohen (2007) back in - in environments of high uncertainty, belief of this sort is entirely rational given the sensory depravity or circumstances…!
—> the neurology of disadvantage (Lende 2012)


[Implications for understanding cognition]

  - Move on from Dual systems inference:
      This view is at odds with dual-process theories that posit an emotional fast route (called “System 1”) and a slower reasoning route (called “System 2”), with only the latter being rational (Kahneman, 2003; Stanovich & West, 2000).
      - In the framework that I have proposed, sensory and interoceptive evidence can be considered simultaneously or one before the other, depending on how quickly the information sources contribute to the predictive coding inference.
      * Still, sensory or interoceptive dynamics are not two separate systems:
          * They are both expressions of a single process that is fully rational in how it considers the statistics of both the sensorium and interoceptive information.
          * Understanding the interactions between sensory and interoceptive dynamics could be highly relevant for the emerging field of computational psychiatry (Montague, Dolan, Friston, & Dayan, 2012).

[Implications for understanding of evolution, e.g. EES]
    - Plural causation versus linear causation:
    - Active inference is to cognition as CAT is to gene-culture, as EES is to Modern Synthesis….


    - Metastability is the unit of selection!!

[Understanding Placebo effect]



[Coordination Dynamics and Mental Health]

  - Autism - Schizophrenia spectrum

[Artificial Intelligence, Agency, & Consciousness]


[Understanding Athlete transition!]:  the endurance of models without interoceptive acknowledgement


\section{Conclusion}
 %(like...1 page...the final word!!)
