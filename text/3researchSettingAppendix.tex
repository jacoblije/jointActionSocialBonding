\chapter{\label{app3:researchSetting}Appendix: Research Setting}

\begin{CJK}{UTF8}{gbsn}







  \section{Qualifications of the researcher\label{app3:qualPositionResearch}}

  Before arriving in Beijing in 2015 to begin my doctoral research, the last time I was in China was two years earlier in 2013, when I spent eight months coaching the Chinese men's youth rugby 7s team in the lead up to the Nanjing Asian Youth Olympics.  Before that, I had spent one year studying on Exchange at Beijing University in 2008, and another year before that on an intensive Chinese language course at Liaoning University, Shenyang, in 2006---my first trip to China.  Rugby featured heavily in both instances.  In 2006, an Australian classmate and friend Ed had caught wind of the fact that there was a rugby program down the road from Liaoning University at the Shenyang Sports College (SSC).  Despite the fact that we had both been diligently attending class and courageously deploying our elementary Chinese to order food at restaurants and befriend local taxi drivers, Ed and I were, nonetheless, three months into our intensive language exchange and feeling that our Chinese skills were floundering.  We suspected that this was in large part due to the fact that we had met very few local Chinese people our age.  So one afternoon we rode our bikes over to the Shenyang Sports College in time for the rugby team's afternoon training session.  Less than six months later, we were boarding an overnight train from Shenyang to Shanghai with the SSC rugby team to compete in the annual Shanghai Rugby 7s Tournament.  We had become closely integrated into the community of rugby athletes at SSC, due in part to the common language of rugby that we all shared, and perhaps mostly due to the overwhelming hospitality of the SSC athletes and coaches.  The decision to find the rugby team may have also helped us improve our Chinese. Ed and I were the only two in our cohort to finish the year in Shenyang with a Level 6 in the Chinese Proficiency Exam, which qualified us to study alongside Chinese local students at an undergraduate level.

  Buoyed by this experience with the SSC rugby team in 2006, I followed a similar template two years later when I arrived at Beijing University on exchange from Sydney University to study sociology at Beijing University.  I had just finished working at the 2008 Beijing Olympics. At that time in Beijing, the only Chinese rugby program was based at the Chinese Agricultural University, a forty minute cycle north of Beijing University.  The CAU rugby program was the strongest in the country: CAU consistently outperformed its rivals at the time (Shanghai Sports Institute, the People's Liberation Army, and SSC) and it was awarded with the responsibility of hosting the Chinese national team.  When the International Olympic Committee announced in late 2009 that rugby would be played in the 2016 Rio De Janeiro Olympics, it was subsequently decided in 2010 that rugby would be inducted into the state sponsored sports system and played in the next Chinese National Games in 2013.  Following this announcement, many of CAU's athletes and coaches dispersed to various professional provincial rugby programs, the main ones being Beijing and Shandong.

  %It was during my time training and generally ``hanging out'' at CAU that I met and developed a strong friendship with Kai, who was at the time playing for CAU and China, while also finishing a Master's degree in Labour Law.  I also met and developed relationships with many rugby players, coaches, and general fans of the Beijing rugby community.

  Between 2009 and 2013 I returned to Australia to finish my undergraduate degree, during which time my own rugby career also rapidly developed.  After a successful season in the Sydney Premiership competition in 2009, I was selected to play for the Australian Rugby Sevens Team. I represented Australia from 2009 through to the end of 2012.  In 2013, during the 9 month gap between my Australian rugby contract ending and the start of my graduate studies at Oxford University, I returned to China to coach the Chinese Youth Men's 7s program in their lead-up to the 2013 Nanjing Asian Youth Olympics.  Along with a small team of Chinese coaches and management, I coached a core group of roughly 25 athletes aged between 15 and 18 years old. We trained 6 days a week for approximately 6 months, with only occasional breaks for National holidays, or for athletes to return to their home provinces to complete compulsory exams.  The program was based predominantly in Anhui province, and we travelled from Anhui to other provinces further afield to find suitable practice opportunities against provincial programs.

  This combination of experience in China and expertise with rugby enabled me privileged access to this particular research setting.

  %Soon after the completion of the Asian Youth Olympics in Nanjing in 2013, the Chinese National Games were held in Shenyang. Rugby was played---for the first time in National Games history---with dramatic consequences, explained above.


\section{Analysis of ethnographic data\label{app3:ethnoAnalysis}}
As Braun and Clark \textcite[10]{Braun2006} explain, ``A theme captures something important about the data in relation to the research question, and represents some level of patterned response or meaning within the data set.''  Identification of recurring themes was guided by (but not limited to) the research questions and research hypotheses outlined in Chapter 2.  Themes were identified on both explicit and implicit levels of the data \citep{Boyatzis1998}.  \footnote{The story of Sun Hongwei (see the opening vignette of Chapter~\ref{sect:SHW}) serves as a good example of the contrast between implicit and explicit levels of data.  In our interview, Hongwei was able to articulate many conceptions relating to joint action and social bonding, all of which were recorded as explicit declarations.  At the same time, I observed in training that Hongwei, while fluent in his declarations in interview, was far form fluent in his more embodied declarations on the field.  The combination of implicit and explicit data offered points of contrast and comparison that were fruitful for richer analysis.}

Thematic analysis involved three stages that unfolded in a recursive (rather than linear) fashion \citep{Braun2006}. In phase one, I familiarised myself with the each data source in the corpus and tagged relevant extracts with theoretically-guided ``codes.'' For example, upon encountering Hongwei's description of his position in the team in his interview transcript (cited in Chapter~\ref{sect:SHW}), I tagged this with codes such as ``group membership,'' ``mutual support,'' ``emotional support,'' ``knowledge of team roles,'' ``signalling commitment to team'' etc.

My coding system was thus directed by (but not limited to) pre-identified research questions rendered as open and investigative questions, i.e.,

\begin{enumerate}
  \item How did athletes experience team performance and social bonding in joint action?
  \item Did athletes experience team click? If so, what experiences were associated with team click?
  \item How did athletes experience uncertainty in group exercise?
  \item Did athletes experience expectation violation in group exercise?
  \item Were moderator variables (such as technical competence and personality type, as well as the relationship between physiological exertion, fatigue, and injury and joint action) salient when discussing team performance, team click and social bonding?
  \item How did the cultural context modulate experiences of social bonding
\end{enumerate}

For each data set, I created a data frame using Microsoft Excel (Version 14.7.1) in which research participants formed the rows, and distinct codes formed individual columns. Data extracts from interviews and field notes were imputed into the cells of this matrix, with an emphasis on including data surrounding each text extract, in order to preserve context \citep[see][]{Bryman2001}.

In phase two, I sorted the different codes into potential themes and collated all the relevant coded data extracts within the identified themes and judged on the dual criteria internal homogeneity of codes within themes (coherence) and heterogeneity of codes between themes (distinction) \citep{Patton1990}.  I then produced a master data-frame ($ participants \times themes$), in the cells of which data extracts from all data sets were included.

In phase three, I generated a definition of each theme, and a refined list of data extracts capable of representing that theme in subsequent analysis \citep{Braun2006}.




\subsubsection{Surveys\label{sect:procSurveys}}

 I conducted a number of informal surveys designed to measure athletes' experience of joint action and group membership in training sessions.

   \myparagraph{Post-interview surveys}
   Following semi-structured interviews, I asked each athlete to rank 10 different possible motivations for adherence to rugby from most important to least important. Possible motivations for rugby consisted of: \textit{to gain access to education}, \textit{to represent Beijing}, \textit{to do Family proud}, \textit{to gain respect from others}, \textit{for (the benefit of) teammates}, \textit{for employment opportunities}, \textit{for money}, \textit{for enjoyment}, \textit{to find a partner}. In addition, athletes were asked to report their 1) three closest friends in the team, 2) the three team members most willing to sacrifice on behalf of the team, and 3) three most competent athletes in the team (see Appendix ~\ref{sect:postInterview} for a full script). Athletes answered these questions using a pen and paper. I later collated and uploaded these responses to Evernote.

   \myparagraph{Post-training surveys}
    I conducted informal surveys following three training sessions: 1) a session in which (predominantly junior) athletes ran an aerobic fitness test involving continuous running in a straight line ``shuttle runs''  at and above the aerobic threshold for approximately 25 minutes (known as the ``Beep Test''), and 2) two 60-minute training sessions spread one week apart involving training scenarios that emulated high-intensity interaction and exertion of match conditions.  After each of these sessions, I administered to each participating athlete via WeChat nine items selected from a validated Chinese version \citep{Liu2012} of the Flow State Scale-2 \citep{Jackson2002}.  The items were selected to measure each of the nine conceptual dimensions of the flow experience: challenge-skills balance, action-awareness merging, clear goals, unambiguous feedback, total concentration on the task at hand, sense of control, loss of self-consciousness, transformation of time, and autotelic experience \citep{Csikszentmihalyi1990}).  All survey items used a 7-point Likert scale.  Responses were collected within one hour of activity completion, with the aim of gathering the data as close to the finish of an activity as possible, while minimising intrusion on the participants \citep{Jackson2004}. For full details concerning survey, see Appendix ~\ref{app4:ethnoSetting} Section~\ref{sect:flowStateScale}.


    \myparagraph{General survey administered at two time points (longitudinal)}
    I asked athletes to comment on experiences of joint action and group membership at two points in time spread three months apart.  These survey items included experience of agency in the team (weak-strong), perceived role in the team (very small-very important), perceptions of individual performance (poor-good), perceptions of team performance (poor-good), training intensity (\textit{qiangdu})(low-high) and difficulty (low-high).  All survey items were measured using a 7-point Likert scale. For a full description of survey questions, see Appendix ~\ref{app4:ethnoSetting} Section~\ref{sect:generalSurvey}.




\subsection{Interviews}
  \subsubsection{Script for semi-structured interviews \label{sect:semiStructured}}


Introduction:
- Brief explanation of my research
- When was the first time you came into contact with rugby?
- Where are you from?
- History with sport before rugby?

Family:
What does your family think about rugby?
Do they support you playing rugby?
Whose decision was it to start playing rugby?
Do they worry about you getting injured?

Perceived Costs Of Rugby:
Opportunity Cost: what would you be doing if you weren’t here playing rugby?
How do you feel about the fitness requirements of rugby?
What is the feeling like when you’re out on your feet and can’t go on?
Injury: Have you had any major injuries?
What do you think is the hardest thing about rugby?

Perceived Benefits of Rugby:
Do the following motivate you to play rugby? (counter-balanced order)
- Education
- Your parents / Family (i.e., to make them proud/content)
- Beijing Residency
- Future Employment
- Teammates
- Earn Respect from (from society, family, friends)
- Find a girlfriend
- Fun
- Represent Beijing

What is something new that you have learnt through rugby?

Team Membership:
What role do you play in the team?
What is the most important thing for you to do in your current role in the team?

Dissonance/Personal Shortcomings?
Have you ever felt like you have let the team down?  (neijiu 内疚: failure to uphold obligation to another)
In what situations do you feel like that?
How do you react to those feelings?

Flow/team click:
Have you ever experienced the team playing extremely well together; everything clicking, like everyone on the field has a “tacit understanding” of each other (\textit{moqi} 默契)?
When was this experience?
What did it feel like?
How do you think that “team click” can be achieved?


\subsubsection{Post-interview activities\label{sect:postInterview}}


Sorting Task: rank your motivations for playing rugby
- Education
- Your parents / Family (i.e., to make them proud/content)
- Beijing Residency
- Future Employment
- Teammates
- Earn Respect from (from society, family, friends)
- Find a girlfriend
- Fun
- Represent Beijing
(Order randomised)

Social Network Task:
1.	The three most competent rugby players in the team
2.	The three people in the team most willing to help others
3.	Your three closest friends in the team



\subsection{Informal Surveys}



      \subsubsection{Flow State Scale\label{sect:flowStateScale}}

Can all athletes who participated in today's training session please respond to each of these questions in a private message. Please do not communicate with other athletes, I want to know about your own experience.  If there are any questions you can ask me directly.

Please answer the following questions based on your experience in the just-concluded competition or activity. These questions relate to the various ideas and feelings you may have experience during the competition or activity you just completed. There is no right or wrong answer. Think about how you feel during the competition/activity and then grade each question using a scale of 1 to 7: Strongly disagree 1; Strongly agree 7).


\begin{enumerate}
  \item Just now my attention was completely devoted to executing the activity
  \item Just now it was as if everything was happening automatically
  \item Just now I was not concerned about how I was performing
  \item Just now it was as if time changed (either slowed or accelerated)
  \item Just now I was clearly aware of what I wanted to achieve
  \item Just now my abilities were matched to the high demands of the activity
  \item Just now I was really enjoying the experience
  \item Just now I experienced a challenge, but I believed that my skills could meet this challenge.
  \item Just now I was not concerned with how others may be evaluating me
\end{enumerate}

Thanks for your cooperation!




所有参加上午训练的球员请私信给我回答下面的九个问题。
请不要和其他球员沟通,我想知道你们自己的感受。要是有问题可以直接问我。


状态流畅量表-2 (CFSS-2)
请根据你在刚刚结束的竞赛或活动中的体验回答下列问题。 这些问题与你在刚刚完成的竞赛或活动过程中可能体验到的各种想法和感受有关。答案无对错之分。思考一下你在竞赛/活动过程中的感受,然后采用下面的等级划分回答问题。  (等级划分1到7分:完全不同意1分; 完全同意7分)

\begin{enumerate}
  \item 刚刚我的注意力完全正在进行的活动上
  \item 刚刚行动似乎是自然而然发生的
  \item 刚刚我不关心自己的表现如何
  \item 刚刚时间似乎改变了(要么是减慢了,要么是加快了)
  \item 刚刚我清楚的意识到自己想要做什么
  \item 刚刚我的能力与情境的高要求相匹配
  \item 刚刚我真的很享受那种体验
  \item 刚刚我遇到了挑战,但我相信自己的技能能够应付这一挑战
  \item 刚刚我不关心别人可能会如何评价自己
\end{enumerate}

  谢谢配合!





      \subsubsection{General Information Survey \label{sect:generalSurvey}}


Think about the last five weeks of training.  What has been your experience of the following (grade each question using a scale of 1 to 7):

\begin{enumerate}
\item The intensity of training during this period (1 very low; 3.5 is normal; 7 extremely intense)
\item The difficulty of training during this period (1 very low difficulty; 3.5 normal; 7 extremely high difficulty)
\end{enumerate}

Now think about your personal and team performance during this period.  How do you feel about:

\begin{enumerate}
\item Your overall individual performance in training (1 very poor, 3.5 normal, 7 very good)
\item The overall performance of the team (1 very poor; 3.5 normal, 7 very good)
\item The role that you personally play in the team (1 very small role; 3.5 average role; 7 very important role)
\item  Your personal agency in the team (1 extremely weak;  3.5 average; 7 extremely strong)
\end{enumerate}


想一想刚过去的五周训练。你对以下的问题有什么经验感受?(等级划分1到7分:完全不同意1分; 完全同意7分)
\begin{enumerate}
\item 这个阶段训练的强度 (1分 非常小,3.5分 正常,7分 非常大)
\item 这个阶段训练的难度 (1分 特别低,3.5分 正常,7分 特别高)
\end{enumerate}

想一想你对个人和整个团队的表现:

\begin{enumerate}
\item 这个阶段你个人在训练当中的表现 (1分 特别差; 3.5分 正常;7分 特别好)
\item 这个阶段整体在训练当中的表现 (1分 特别差; 3.5分 正常;7分 特别好)
\item 这个阶段你个人在团队发挥的作用 (1分 作用很小; 3.5分 正常作用; 7分 作用很大)
\item 这个阶段你在团队的个人力量 (1分 比较小,3.5分 中等; 7分 比较大)
\end{enumerate}





\section{Ethnographic observations}

\subsection{Rugby}

\subsubsection{Motivations for adherence to rugby\label{sect:}}


During fieldwork I witnessed a number of instances in which an athlete's mood fluctuated according to the state of their attempts to secure life-course opportunities of education and employment.

Two senior athletes, Ma Haitao and Cui Suocheng, despite long qualifying as a Champion Athlete, were involved in complex bureaucratic journeys in an attempt to gain admission to BSU.  One day early in my first stint of training I was taking a training session in which we were focussing on defence, and Shuocheng suddenly appeared to have given up all energy.  He all of a sudden stopped doing the prescribed tackling drill.  I took him aside and began to take him through some of the details he missed last week when he was away.  He refused to listen and said, ``I get it, I just don’t want to do it, I’m sick of it, I’ve had enough of this drill'' (练够了!)''You’ve had enough?'' I asked, ``Alright, if you’ve had enough then get off the field!''  I snapped at him, motioning to the stands for him to go and sit down. I was stunned that he first said that he didn’t want to practice.  Shuocheng immediately followed my instruction and sat on a tackle bag on the side of the training field.  Later on in the training session, he came back over to the subsequent training drill and started offering advice to the more junior athletes about their technique.

A few days later I found an opportunity on our way back from training to the dormitory to ask head coach Zhu about the situation with Shuocheng.  Zhu explained that Shuocheng was experiencing difficulty finalising his contract transfer from Shandong to Beijing province, and that this procedure was interrupting his ability to process his university application (Shuocheng had moved from Shandong provincial team in 2014). To attain prized life-course opportunities through adherence to rugby was obviously a core motivation for most athletes.  But it is worthy mentioning how seriously these concerns appeared to grip athletes, to the extent that, in Shuocheng's case at least, his ability to train freely and openly was compromised.

Ma Haitao, another senior athlete also in the process of trying to apply for university (after years of delays), also experienced large amounts of stress during my time researching.  One day he spoke to me at length about the way in which these difficulties were impacting on his mood and his ability to focus on training. ``It impacts me so strongly'' he said as we walked to the canteen after a training session.  ``I’ll probably need to have a long sleep or take a break from training before I can recover from these sort of frustrations---the endless process of getting this form and completing that form, its so frustrating.''






































































































\end{CJK}
