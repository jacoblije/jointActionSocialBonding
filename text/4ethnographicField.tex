


\chapter{\label{4ethnographicField}Ethnographic Study}

\minitoc
\section{Abstract}
In this chapter I present the results of ethnographic data collected with the Beijing Men's Rugby team between September 2015 and August 2016.  I find a range of evidence in support of the predictions set out initially in Chapters 2 and refined in Chapter 3 in light of a review of the contextual specificities of the research setting.  First, I describe the culturally specific terrain of social cognition, which is defined by a dominance of ``hierarchical relationalism.''  The dominance of this culturally specific mode of social cognition is identifiable at multiple levels of social life, from the level of the institutions in which athletes, coaches, and officials interact, to the level of group norms in which athletes and coaches participate, to the level of on-field processses of joint action and perception. Having described the details of this cultural terrain, the contours generalisable cognitive mechanisms relevant to the predictions of this dissertation become more visible. I identify three key categories of athlete experience---1) perceptions of performance in joint action, 2) feelings relating to ``team click'', and 3) processes of group membership---which I argue are relevant to the hypothesised relationship between joint action and social bonding, as well as the hypothesised mediating role of team click in this relationship. I also outline evidence for possible moderating variables of these relationships, such as technical competence, personality type, injury, and fatigue. I conclude the chapter by summarising and discussing the results, with a particular focus on how these observations could be operationalised in futher experimental studies with a larger sample of athletes beyond the Beijing men's team.



\section{Vignette}

On that Monday day that I first arrived at the Institute and rounded the Stadium to its main administrative building, almost two years had passed since the Beijing Women's rugby team's controversy at the 2013 National Games.  Fortunately, my first visit to the Institute ended up going as smoothly as I could have hoped.

I first met with Jenny, the Vice-Principal of the Institute who was responsible for rugby.  I sat down with Jenny in her office as she finished a conversation to someone on the phone. Jenny flowed with the personality and affectations that only a true Beijing local could embody---her word endings textured with coarse yet elegant ``erhuayin''(儿化音) suffixes, and she addressed the person to which she was speaking using the respectful version of the 2nd person pronoun ``nin3.''  Once she had finished with the call, Jenny welcomed me and tactfully explained to me that rugby at the Institute had indeed experienced a dramatic fall from grace.  Jenny indicated that the head coach ZPH and his assistant SY really had their work cut out for them, and that my presence as observer and occasional coach would benefit the team.  She agreed to organise a room in the rugby program dormitory, as well as access to the Institute's canteen, in exchange for my expertise.

Bouyed by this meeting, I made my way to where the rugby program's dormitary to meet head coach Zhu Peihou (see map X).  My connection to ZPH went back to CAU in 2008, where he had been a coach at the time. From Shandong originally, ZPH was a graduate of of the Shanghai Sports University, another prominent rugby program at the time.  ZPH had been recruited from Shanghai to CAU by ZHJ to coach so that he could continue to play for the Chinese national team after he had completed his undergraduate studies.  After we had discussed my research and he had provisionally approved my plan to spend the next period with the team, I asked him about the current situation with the Beijing team.  ZPH explained that he was quite frustrated that the group of athletes he was coaching lacked experience and maturity. I asked him exactly what areas of the team's performance, and he indicated that all areas were not great, suggesting that not enough players had found that ``feel'' for gameplay and very few were motivated to train hard.  We agreed that my first assignment should be to accompany the team to the final National Tournament in the coastal city of Qingdao in Shandong province, in a week's time.

As it turned out, the National Tournament in Qingdao was organised to take place immediately following a two-day Asia Sevens Tournament.  The Asia Sevens is an annual series of three regional rugby sevens tournaments featuring men's and women's national sevens teams.  The men's series has been held regularly since 2009, and the women's series was established in 2013. The series usually consists of two three annual Tournaments, alternating between various locations including Hong Kong, South Korea, Sri Lanka, China, Japan, Malaysia, India, Singapore, and Thailand.  In 2015, The Asia Sevens Series Tournaments were held in Colombo, Bangkok, and Qingdao. The Tournament would be played on the Saturday and Sunday, and then the National Tournament would follow immediately after on the Monday and Tuesday.

\subsection{Asia Sevens Tournament}
When I arrived at the stadium in time for the beginning of the Asia Sevens Series, I met a number of coaches and athletes in the stands who I knew from my time in Beijing in 2008 and then coaching in 2013. I sensed from my various interactions that there was an air of nervousness around the Tournament, particularly on behalf of the Chinese women's team.   The Qingdao tournament was the final tournament before the Olympic Qualification Tournaments, to be held in Hong Kong and Tokyo in November 2015.  The top ranked team from those two legs would qualify for the Rio Olympics in 2016.  The Chinese men's team were not in serious contention for Olympic qualification, given the clear superiority of the more established men's rugby programs like Japan and Hong Kong. The Chinese Sports Commission did, however, expect the Chinese women's team to qualify for the Olympics.

Since the Chinese women's team was first established in 2002, China had made great strides in women's rugby, in both Asia and globally.  Not including occasional losses to closest rivals Kazakstan, between 2002 and 2012 the Chinese women's sevens team was the dominant women's team in Asia, easily outcompeting Japan and Hong Kong, and at times was competitive against the world's best including New Zealand and Australia.  The main reason for China's dominance in the women's game during this period was that other traditional rugby nations, despite having developed professional infrastructure for the men's game, lacked almost entirely an equivalent infrastructure for the women's game. China`s state sponsored sport system, on the other hand, was relatively agnostic towards gender in sport. When it comes to the bare incentive structure of the Chinese sports system a gold medal is a gold medal, regardless of the gender of the recipient. (This is of course not to say that there are not distinct gender inequalities in relation to sport in China).  Indeed, beginning with the Chinese Women's Volleyball Team's gold medal victory at the LA Olympics in 1982, China has enjoyed a comparative advantage in women's sport due to the fact that the Chinese sport system was comparatively more supportive of women's sport.

Following his success at the National Games in 2013, Shandong head coach Lu Xiaohui was given the responsibility as coach of the Chinese women's program in 2014.  Essentially, this involved giving Lu and  Shandong province the responsiblity for the team.  The national team trained at Shandong's provincial training centre, and most of the athletes who represented China in 2014 were from Shandong.

Alarmingly for the Chinese women's rugby team's hopes of Olympic qualification, by 2014 it had become obvious that other more traditional rugby playing nations in Asia, namely Japan and Hong Kong, had begun to make up serious ground on China and Kazakstan in the women's game.  This naturally prompted nervousness among the Administration therefore and CRFA.  In mid 2015, it was decided by the the Administration that CRFA would enlist the services of a foreign coach to bolster their campaign for Olympic qualification.  According to sources close to CRFA, apparently the original plan was for the appointed foreign coach, Ben, to act as a consultant for Lu and his existing group of coaches.  By the time I arrived in Qingdao in early September, however, the initial arrangement had since transformed into a situation in which Ben was given 100\% control over the program as head coach, and LXH was more or less sidelined as coach. Not only had Ben took over, but he had also set about reorganising the starting team and also scouting for new talent outside the squad, which was predominantly made up of Shandong athletes.

%The complication was that before being dethroned, LXH preferred to use his own athletes.  Most of the starting team at the LDN7s in June 2015 were indeed Shandong athletes, many of the women who had won gold at the National Games in 2013.  When BG took over the reigns as head coach, as directed by the GAS and CRFA,

When I arrived in Qingdao it appeared that tensions between the old and new guard were at their peak.  Lu and many of his favoured athletes had been relegated to the sidelines, and some had been completely removed from the squad altogether.  There were still six weeks to go before the all important first Olympic qualification tournament in Hong Kong.  I sat and watched the first few games of the women's Tournament, and I was indeed surprised to see that the Chinese women's side was missing some of its usual stalwarts, and indeed appeared in my eyes to lack the flow and familiarity that I had come to expect in Lu's clan of Shandong athletes.  In the stands I came across a group of Shandong women, one of which, Qi Gaige, I knew quite well from when she toured to Australia with the Shandong team when I was still with the Australian rugby sevens team in early 2013.  I asked Gaige why she wasn't playing for China, and it turned out that she was injured, and so wasn't eligible for selection.  But a few of the other women around her, who were all wearing Chinese national team uniforms, were all part of LXH's clan who had been effectively stood down by Ben. ``How do you think they're playing?'' (你觉得她们打得怎么样?) She asked me, after we had exchanged pleasantries. ``Not great'' I commented, hesitating, not knowing how much I should prime her, but also feeling obliged to be honest: ``it feels like they’re not coordinating together very well at the moment.  What do you think?'' (一般吧。感觉她们的配合不太好,目前。你觉得呢?) I asked.  ``They’re out there playing as individuals, not playing as a team! They can't get it together; there's no shared goal.'' `(她们都在打个人的,不打团体提的。打不到一块儿去啊,没有共同的目标.)  ``Hmm. Yes it does look like that.'' (嗯嗯,看起来像是) ``Hey, Lijie...'' she asked me quietly, ``...don’t you think they’re not even playing as well as our Shandong team could play?'' (嘿,李杰,是不是她们现在打的没有我们山东队打的好,是吧?)

Later that day I bumped into an assistant coach of the Chinese women's side, who had been working under Lu and now Ben since 2014.  Li Sheng was a big and booming Qingdao local, a CAU graduate, and a member of the Beijing Men's team from 2010-2013.  I asked Li about the current situation with the Chinese women's side and their prospects 6 weeks out from the first Olympic qualifier.  ``Chinese athletes must see the (individual) benefits if they are going to go all out'' (中国的运动员必要看到个人的利益才会全力以赴的) he insisted, and he went on to explain why for these athletes, there were no obvious benefits available sufficient to motivate them.  There were indeed very few material benefits associated with representing China in rugby at the national level.  Athletes were payed a nominal USD100 per month on top of their provincial contracts when training and touring with the national program.  If athletes were injured while playing for China, CRFA at the time did not have access to sufficient health insurance to cover the costs of treatment, and athletes had no choice but to return to their provincial programs and seek treatment at the expense of the province.  The less tangible benefits of playing for China, for example, access to high quality coaching, or the pride of representing the country in a sport, or even the promise of a trip to the Olympics, were heavily outweighed by other less tangible costs: long stints of time away from family, the constant risk of falling out of favour with provincial programs.  In effect, the lack of incentives at the national level meant that athletes were by definition more committed to their provincial systems---the programs that provided athletes with the benefits that they were most interested in obtaining, such as tertiary education, future employment, and a modest---but compared to CRFA--a reasonable salary (most national level players were paid 3-8k RMB/month). ``Its no wonder these athletes aren't performing well,'' (难怪这个队伍的表现不好),Li exclaimed.

\subsection{National Tournament}

On the first day of the National Tournament I met Beijing head coach Zhu in the stands before their first game, and he instructed me to keep an eye on the games that Beijing would play, and offer any feedback about things they could work on.  Of course, eager to being my observation of the athletes I would go on to spend many months together with, I obliged, and sat down with a notebook and watched the days play.  In their first game, Beijing obviously lacked coherence in attack and defence.  Their basic skills, for exmaple, passing accuracy and work in contact needed more work, and it did not seem as though they had anyone who was performing the role of leader and providing the team with direction on the field.  I felt that they lacked an element of maturity and patience that would be required to win tight games of rugby sevens. It was clear that the team was made up of many inexperienced athletes, many of whom had barely played in serious official Tournaments such as this one.

Regardless, Beijing ended up winning both its games on Day 1. I couldn't help but be puzzled by the atmosphere When the team huddled together around head coach Zhu after their victory over the PLA in their final game of Day 1.  While the younger athletes were concentrated intently on Zhu's every word, it was apparent that some of the older players were less focussed, in fact many of them appeared to not even be listening to Zhu.  In particular, I noticed that Wang Chongyi, a former Beijing men's team representative and assistant men's coach at the time, was as physically withdrawn from the huddle as possible while still maintaining connection to it by holding losely on to the shirts of the athletes either side of him. Wang was seemingly uninterested in what Zhu was saying.  Some of the senior athletes also appeared less engaged with what Zhu was saying.  I was puzzled by this piece of team theatre and wondered if I was simply reading too much into it, and that perhaps the posturing and performance of team unity was less emphasised in this context than the contexts in which I had played rugby elsewhere. As was common on my journeys through the world of rugby in China, I often did not quite grasp all the details and pieces of the puzzle that contextualised the interactions I was having until after the fact.

Despite the obvious deficiencies in Beijing's performance that weekend, they did well enough over the two days to make the final of the Tournament and indeed win the National Tournament.  This achievement probably owed more to the relative weakness of the other provincial outfits, more than Beijing's out and out strength.  In an unfortunate twist of fate, however, the Beijing side was later disqualified from the Tournament, due to the fact that they fielded an underage player, who was only 15 at the time.  It was clear that Zhu and his team could not catch a break.  After reconvening back in Beijing, Zhu gave athletes the rest of September off training, and instructed all to reconvene at the Institute at the start of October to begin off season training.




\section{Introduction}

Theory and empirical research from the emerging field of the social cognition of joint action suggest that the affordances of particular cultual environments act to enable and constrain observable behaviour in patterned ways.  The sensitivity of joint action (or any cognitive process for that matter) to informational affordances provided by various layers of ecological and cultural context.  The cognitive inputs to joint action in real world settings are rarely limited to essentialised componnents administered in laboratory paradigms. Cognitive processes relevant to joint action are known to be distributed throughout brains, bodies, and the physical environment of the ecological niche in which it is situated.

In the present study, I address these knowledge gaps in evolutionary approaches to group exercise through an ethnographic study of the social cognition of joint action among professional Chinese rugby players.  Based on emerging research from the social cognition of joint action, in which it is increasingly understood that dynamical coupling of lower-cognitive mechanisms associated with movement regulation set the foundations for higher-cognitive processes such as social bonding and declarations of group membership, I interrogate the relationship between athletes' experience and perceptions of joint action and attitudes towards group membership.  Within this space I isolate the psychological construct of ``team click,'' which refers to athletes' tacit sense of quality joint action coordination.   I then outline the specific predictions of the ethnographic component of this dissertation, before describing the method via which I collected primary data.


In this ethnographic I identify evidence for 1) affordances specific to the sport of rugby and the cultural milieu of competetive sport in contemporary China, and 2) cognitive mechanisms relevant to relationship, hypothesised in this dissertation, between joint action and social cohesion.  In particular, I highlight athlete's perceptions of performance in joint action, their experiences of phenomena associated with team click, and processes of group membership as relevant to study predicitons.

Accounting for human behavioural phenomena requires the consideration of a number of biological, cognitive, and ecological mechanisms that interact via reciprocal feedback loops spanning multiple scales of time and space \citep{Fuentes2015}.
%justifications for ethnography
The proliferation of anthropological approaches to human behaviour in the last 50 years, while at times threatening the overall coherence of the discipline as a whole \citep{Beller2012}, has also produced diverse theoretical and methodological options for documenting human variation \citep{Fuentes2016a}.
Anthropology is thus well placed to expand upon accounts of group exercise, via methods ranging from ethnographic exploration capable of uncovering novel dimensions of behaviour and generating testable hypotheses, to quantitative techniques---e.g., experimental and mathematical simulation paradigms---capable of testing hypotheses \citep{Epstein2006,Fuentes2016}.



%culturally specific contours
First, ethnographic data reveal evidence for culturally specific contours of mechanisms relating to joint action, team click, and social bonding.  These contours have formed through an interaction between the specific history of rugby and modern sport in China and facets of an indigenous Chinese psychology.  Describing these contours is an important first step, because of the way in which they can act as culturally specific affordances---``hyper-priors'' or ``coordination smoothers''---that shape observable patterns of action and perception, group membership, and adherence to institutional norms.

action and perception
group membership
institutusions

Second, ethnographic data collected with the Beijing men's team reveal the operation of proximate mechanisms relating to joint action, team click, and social bonding.

%Performance
Athletes appear to develop strong subjective perceptions of individual and team performance, based on attention to specific technical components of rugby game play.   On the one hand, athletes display evidence of anxiety when expectations around individual and team performance are negatively violated.  When the same expectations concerning individual and team performance are positively violated, athletes are noticeably exhilarated, energised, and motivated.  Data collected in this ethnography suggest that more junior athletes appear more emotionally volatile when expressing perceptions of individual or team performance, such that both performance related anxiety and exhileration appears to be more pronounced in junior athletes than senior athletes.  More senior athletes, by contrast, display the ability to relate to perceptions of individual and team performance expectations with more personal agency and emotional control.  More senior athletes talk in terms of an ability to strategically mitigate the costs of physical exertion and injury risk of joint action in rugby, whereas more junior athletes talk in less personal agency about their experience of individual and team performance.  These results suggest that perceptions of team and individual performance in joint action are formulated in relation to specific components of team and individual performance in joint action, as well as more general team and individual expectations structured by the specific team context, social norms, and institutional constraints.

%team click & bonding
Athletes describe a number of phenomenological dimensions to optimal team performance (team click), including ``tacit understanding'' between teammates (\textit{moqi} 默契), ``team atmosphere,'' (\textit{qichang} 气场) and general order and structure to team coordination.  The data reveal within-group variation in the way athletes communicate team click: more junior athletes talk about team coordination in a more general and emotional way, whereas more senior athletes talk with a more restrained but more specific understanding of the dimensions and factors required in order to achieve peak team performance.  In regards to dimensions of social bonding, athletes appear to prioritise feelings of emotional support and a perception of a shared goal between teammates, as well as a level of personal connection to team roles and team identity.  Testimonies of team identity vary from the initial excitement and positivity of new arrivals towards, to the realisation of irreversible attachment to rugby of middle-rung athletes, to the out-and-out identity fusion between senior athletes and the practice of rugby, whereby athletes recognise that their self concept is inextricably linked with rugby and the Temple Institute.

%individual differences
Finally, contained within the collected data is evidence that individual differences may moderate relationships between perceptions of joint action, team click, and social bonding.  Namely, an athlete's experience and competence with the technical and social dimensions of joint action of rugby at the Temple Institute may impact upon the ways in which they formulate and respond to expectations surrounding and team performance.  Individual differences in technical competence may also have flow-on effects for the formulation of feelings concerning team click and social bonding. It appears that individual personality types may moderate perceptions of performance in joint action, as well as feelings of team click and social bonding. In addition, other factors such as an athlete's injury status, and the level of fatigue and exertion experienced in joint action scenarios of rugby may impact on perceptions of performance, feelings of team click, and social bonding.  These individual differences should be considered in the design of quantitative studies to test the relationship between joint action, team click, and social bonding.



%Contribution: (See Cohen2007 and CohenWhitehouse2012)
The first time ethnographic evidence has been provided for mechanisms of joint actions (Wacquant: body and soul), Body for China (Brownell1995)
This ethnographic theorises divergent cultural modes
Cultural specificities may act as ``factors of attraction'' \citep{Sperber2014} that constrain and direct the fixation of cultural variants within and between populations.














\section{Method}




\subsection{Research Setting and Participants}

I planned to conduct ethnographic research with athletes and coaches of the Beijing Provincial Rugby Sevens Program based at the Beijing Temple of God of Agriculture Sport Institute.  The Program consists of Men's and Women's rugby sevens teams, each with approximately 20-30 athletes and 2-4 coaches per team. Athletes train full time and live in dormitary accommodation at the Institute.  One of four vice-principals of the Institute is responsible for the administration of the program.  Permission to conduct research at the Institute was sought from the vice-principal responsible for rugby at the Institute and the head coach of the rugby Program prior to arriving in Beijing in 2015.  Permission to conduct research was sought directly from athletes at the beginning of the first research period in September 2015. The University of Oxford’s Central University Research Ethics Committee approved this study (SAME/CUREC1A/15-059).

\subsection{Materials}

  \subsubsection{Participant Observation}
  I planned to conduct a number of stretches of participant observation with the Program between September 2015 and September 2017.  During these stretches, I planned to live full-time at the Institute and attend training sessions, team meetings, meals, and participate in any other activities relevant to the rugby program.  I planned to record field notes using Evernote (Version X), an electronic note taking software that is synchronised accross my mobile and personal computer devices.

  \subsubsection{Interviews}
  I planned to conduct and record a combination of ad-hoc exploratory (unstructured) interviews with a range of research participants with knowledge of rugby in China as well as scheduled and directed (semi-structured) interviews with athletes in particular.  The script for semi-structured interviews would be designed based on a combination of existing theory and initial ethnographic observations, in order to understand the culturally specific contours and operation of mechanisms relating to joint action and social bonding.

  \subsubsection{Informal Surveys}
  In addition to conducting participant observation and interviews, I also planned to issue surveys to measure athletes' motivations for, and perceptions of joint action and group membership in the rugby Program.  The surveys would be designed based on a combination of existing theory and initial ethnographic observations.


\subsection{Procedure}

In May 2015, 4 months prior to beginning ethnographic research, I contacted the vice-principal responsible for the rugby Program and the head coach of the rugby Program to ask for permission to conduct research at the Institute.  Following affirmative responses from both, I made plans to conduct two periods of in-depth ethnographic research: 1) 6 months between September 2015 and March 2016, 2) 6 weeks during July-August 2016.

\subsubsection{Participant Observation}
Soon after arriving in Beijing at the end of August in 2015 to begin research, I met in person with the vice-principal and head coach of the rugby program to confirm permission for research and to discuss logistics.  I discovered that the Beijing Rugby Program was limited to a Men's Program.  At the time, the Women's rugby Program had yet to be resurrected after the humiliating ``match strike incident'' of the 2013 National Games (see Chapter 3 ~\ref{} for a detailed explanation). The Women's program would later be resurrected at the start of 2016, in time to participate in qualification tournaments for the 2017 National Games.  Due to this limitation, I decided to focus my attention on the Men's program, which consisted of 25-30 athletes and four coaches at any one time.

Both the vice-principal and head coach agreed to provide me with research access to the rugby program, a room in the Institute's dormitory, and access the the Institute's 1st level canteen, in exchange for assisting the Program with rugby knowledge and coaching.   The Institute had two canteens in which athletes and coaches ate all of their meals.  Athletes and coaches who had represented Beijing at national level competitions were entitled to eat at the 1st Level Canteen (\textit{yixian shitang} 一线食堂), whereas all other athletes at the Institute, or athletes who were on temporary trial at the Institute, ate at the 2nd Level Canteen (\textit{erxian shitang} 二线食堂).

During periods of ethnographic research I lived full time with the team at the Institute, attending training sessions, team meetings, meals, and other activities with the team.   All my interactions with research participants took place in Modern Standard Chinese (Mandarin or \textit{putonghua} 普通话).  I took notes using an electronic note taking application (Evernote, version 6.11) which automatically synced and stored notes created on either my mobile phone of personal laptop computer. I collated, summarised, and tagged these notes weekly or fortnightly.


  \subsubsection{Interviews}

Unstructured interviews were conducted with athletes and coaches on an ad-hoc basis, often when an informal discussion developed into a conversation relevant to my research questions. In such instances, I would interrupt discussion with the research participant and ask permission to record the remainder of the discussion using the digital audio recording feature on the Evernote application on my mobile phone.

Semi-structured interviews were conducted by appointment in my dormitory room at the Institute at a period 2 months in to my first stint of participant observation.  During semi-structured interviews, I asked athletes about their personal background (including their family situation), their motivations for adherence to rugby, perceived costs and benefits of adherence to rugby, perceptions of joint action, and group membership. For a detailed script of semi-structured interviews, see Appendix ~\ref{} Figure ~\ref{}.  Questions served only as a loose structure for conversation, and at times either the athlete or I departed from these questions to talk about other dimensions of experience associated with rugby at the Institute.  The order in which athletes participated in semi-structured interviews was randomised.

I conducted all interviews in Modern Standard Chinese (Mandarin) and interviews were recorded with participant consent using digital audio recording feature on the Evernote application on my mobile phone or laptop computer.  Once all interviews were recorded, interviews were transcribed into written Chinese by a native Chinese speaking research assistant using a ``verbatim'' method \citep[i.e., including an account of all verbal and important nonverbal (coughs, pauses, etc) utterances, see][269-70]{Poland2003}.  I checked each transcript for accuracy by comparing the script against the original audio recording during the first phase of open coding analysis (see Section Data Analysis below). I analysed interviews in Chinese and only translated into English data extracts that were included in the main analysis of this dissertation.

%    \subsubsection{Structured}

\subsubsection{Surveys}

 I conducted a number of informal surveys designed to measure athletes' experience of joint action and group membership in training sessions.

   \myparagraph{Post-interview surveys}
   Following semi-structured interviews, I asked each athlete to rank 10 different possible motivations for adherence to rugby from most important to least important. Possible motivations for rugby consisted of: \textit{to gain access to eduction}, \textit{to represent Beijing}, \textit{to do Family proud}, \textit{to gain respect from others}, \textit{for (the benefit of) teammates}, \textit{for employment opportunities}, \textit{for money}, \textit{for enjoyment}, \textit{to find a partner}. In addition, athletes were asked to report their 1) three closest friends in the team, 2) the three team members most willing to sacrifice on behalf of the team, and 3) three most competent athletes in the team (see Appendix ~\ref{} for full script). Athletes answered these questions using a pen and paper. I later collated and uploaded these responses to Evernote.


  \myparagraph{Post-training surveys}
  I conducted informal surveys following three training sessions: 1) a session in which (predominantly junior) athletes ran an aerobic fitness test involving continuous straight line ``shuttle runs''  at and above the aerobic threshold for approximately 25 minutes (known as the ``Beep Test''), and 2) two 90-minute training sessions spread one week apart involving training scenarios that emulated high-intensity match conditions.  After each of these sessions, I administered to each participating athlete via WeChat nine items selected from a Chinese version of the Flow State Scale 2 \citep{Liu2012} designed to measure the nine conceptual dimensions of the flow experience: challenge-skills balance, action-awareness merging, clear goals, unambiguous feedback, total concentration on the task at hand, sense of control, loss of self-consciousness, transformation of time, and autotelic experience \citep{Csikszentmihalyi1990}).  All survey items used a 7-point Likert scale. For full survey details, see Appendix ~\ref{} Section ~\ref{}.

  \myparagraph{General survey administered at two time points (longitudinal)}
  I asked athletes to comment on experiences of joint action and group membership at two points in time spread three months apart.  These survey items included experience of agency in the team (weak-strong), perceived role in the team (central-marginal), perceptions of individual performance (weak-strong), perceptions of team performance (weak-strong), training intensity (\textit{qiangdu})(light-heavy) and difficulty (easy-hard).  All survey items used a 7-point Likert scale.


\subsection{Data analysis}
Field notes from participant observation, interview scripts, and informal survey responses formed a corpus of ethnographic data that was subjected to a recursive process of ``thematic analysis'' \citep{Braun2006}.  As Braun and Clark \textcite[10]{Braun2006} explain, ``A theme captures something important about the data in relation to the research question, and represents some level of patterned response or meaning within the data set.'' Identification of recurring themes was guided by (but not limited to) the research question and theoretical predictions of this dissertation, outlined initially in Chapters 1 and Chapter 2 (and then refined in Chapter 3 in accordance with the specific research context of rugby in China).  Themes were identified on both explicit (semantic) and implicit (latent) levels of the data \citep{Boyatzis1998}. Theoretical predictions and relevant existing research concerning the social cognition of joint action helped direct analysis of the latent level of the data.

The thematic analysis involved three stages that unfolded in a recursive (rather than linear) fashion \citep{Braun2006}. In phase one, I familiarised myself with the each data set in the corpus (field notes, interview transcripts, and informal survey responses) and tagged relevant extracts with theoretically-guided ``codes.'' For example, upon encountering Hongwei's description of his position in the team in his interview transcript (cited in the Introduction ~\ref{}), I tagged this with codes such as ``group membership,'' ``mutual support,'' ``emotional support,'' ``knowledge of team roles,'' ``signalling commitment to team'' etc.  My coding system was thus directed by (but not limited to) pre-identified theoretical variables relating to 1) athlete perceptions and expectation violations surrounding joint action, 2) perceptions and feelings associated with the phenomenon of ``team click,'' 3) understandings of and feelings relating to ``group membership'' and social bonding, as well as 4) possible moderator variables of technical competence and personality type.  For each data set, I created a data frame using Microsoft Excel (Version 14.7.1) in which research participants formed the rows, and distinct codes formed individual columns. Data extracts from interviews and field notes were imputed into the matrix, with an emphasis on including data surrounding the code's target, in order to preserve context \citep[see][]{Bryman2001}.

In phase 2, I sorted the different codes into potential themes and collated all the relevant coded data extracts within the identified themes and judged on the dual criteria internal homogeneity of codes within themes (coherence) and heterogeneity of codes between themes (distinction) \citep{Patton1990}.  I then produced a master data-frame (participants x themes), in which data extracts from all data sets were included.  In phase 3, I generated a definition of each theme, and a refined list of data extracts capable of representing that theme in subsequent analysis \citep{Braun2006}.









\section{Results}

In this section I introduce the details of the research context, provide evidence for its culturally specific terrain, and then analyse study predictions in light of this terrain.

\subsection{Rugby at the Institute after 2013}
Having let the dust settle on the embarrassment of the women's program's widely publicised disqualificaton from the 2013 National Games, in 2014 the Institute decided to quietly continue with both the men's and women's programs, in preparation for the 2017 National Games.  In April 2014, more than six months after the National Games, the men's program was ressurected with the appointment of a new head coach. The junior athletes from the previous National Games cycle were recalled back to the Institute to resume training, and the new head coach Zhu Peihou was charged with finding new talent. A former Chinese representative and CAU coach, Zhu had previously coached the Anhui women's team during their 2013 National Games campaign.  Zhu appointed his close colleague and graduate from the Shanghai Institute of Sport's rugby program, Shi Yan, as assistant coach.  Both Zhu and Shi were were employed by the Institute on a contract basis, rather than becoming full-time official employees of the Institute.

The women's program was inactive for a full two years after 2013, and was only just starting to re-activate when I arrived in October 2015.  Former Beijing women's rugby representative (2010-2013) and Beijing local Ma Jiale was appointed as head coach, and former CAU graduate, Chinese National Team representative, and Beijing men's rugby representative athlete and Beijing local Wang Chongyi was appointed as assistant coach.

Thus, rugby was ressurected at the Institute, but was no longer in centre stage. The Beijing men's team endured a series of mediocre performances during the 2014 and 2015 seasons, and clearly lacked experience, talent, and institutional support from the Institute.  A handful of senior athletes who had played in the era of the 2013 National Games remained, and two in particular, Han Xiaolong and Lu Peng were promoted to a transitional athlete-coach status. Unlike Women's assistant coach and former athlete Wang Chongyi, however, both Han and Lu were originally from Shandong province and so did not automatically have Beijing residency required to make them eligible for full time empoyment at the Institute. As such, their future place at the Institute was uncertain, and as I found out from both during the course of my ethnographic research, their ability to stay at the Institute would depend on the result the team could achieve at the 2017 National Games.

It is important to note that both coaches appointed to the women's program in 2015, Ma and Wang were---unlike Zhu, Shi, Han, and Lu---full-time permanent employees of the Institute. As explained in the previous chapter, China's infamously rigid ``Hukou'' (户口)residency system means that it is still the case that only individuals with Beijing residency can hold permanent employment roles at government institutions such as the Institute (\textit{shiye danwei} 事业单位).  Chinese citizens born outside of Beijing can become Beijing residents if offered employment, but due to Beijing's swelling population, the eligibility criteria for this process of naturalisation has become more and more stringent, and fewer and fewer applications are successfully processed, particularly in industries like sport.\footnote{This links to the question of the ``Quality'' of athletes}

Despite being only a shell of its former glory, the rugby program at the Institute nonetheless offered attractive incentives to prospective athletes.  The difficulties of Han and Lu in gaining Beijing residency made it clear to more junior athletes that there was little promise a passage to official Beijing residency or full-time employment at the Institute, but the program did offer the much more realistic opportunity of attending the Beijing Sports University (BSU)---considered to be the country's most prestigious sports universities and one of China's top ``brand universities'' (\textit{mingpai daxue} 民牌大学).  The mass exodus of experienced senior athletes from the rugby program meant that junior athletes from the pre-2013 era were now in a position to represent Beijing at a national level, and therefore attain the official athletic standard of a ``Master Sportsperson'' (\textit{yundong jianjiang} 运动健将), and thus eligibility to attend BSU.


\subsection{Participants}
I analysed data on a total of 26 Athletes ($avg. age = 21.3, range = 18-27, SD = 2.96$) and four coaches, who were not included in the main analysis but provided important contextual information.  Athletes were included in data analysis if they participated in 1) a semi-structured interview, 2) at least one informal survey relating to experiences of rugby training and group membership, and 3) at least 2 months of training at the Institute.  See Table ~\ref{tab:ethnoDescriptivesTable} for a summary of athlete attributes, including team status, contract status, etc.

% latex table generated in R 3.5.0 by xtable 1.8-2 package
% Wed Jun 20 16:44:16 2018
\begin{table}[ht]
\centering
\begin{tabular}{ll}
  \hline
row & Overall \\ 
  \hline
n &    26 \\ 
  Age (mean (sd)) & 20.96 (3.17) \\ 
  ResearchCategory = Senior (\%) &    10 (38.5)  \\ 
  TrainingAge (mean (sd)) &  3.34 (2.02) \\ 
  YearsInTeam (mean (sd)) &  2.59 (1.80) \\ 
  AthleteStatus (\%) &     \\ 
     Master Sportsperson &    10 (47.6)  \\ 
     Level 1 &     6 (28.6)  \\ 
     Level 2 &     5 (23.8)  \\ 
  ContractStatus (\%) &     \\ 
     Permanent Employee &     1 ( 3.8)  \\ 
     Full Time Contract &     7 (26.9)  \\ 
     Training Contract &     5 (19.2)  \\ 
     Student Contract &     6 (23.1)  \\ 
     Trial &     7 (26.9)  \\ 
  EducationLevel (\%) &     \\ 
     Graduate &     1 ( 3.8)  \\ 
     Undergraduate &    12 (46.2)  \\ 
     High School &    10 (38.5)  \\ 
     Middle School &     3 (11.5)  \\ 
  HomeProvince (\%) &     \\ 
     Shandong &    11 (42.3)  \\ 
     Beijing &     6 (23.1)  \\ 
     Jiangsu &     3 (11.5)  \\ 
     Liaoning &     2 ( 7.7)  \\ 
     Hebei &     2 ( 7.7)  \\ 
     Heilongjiang &     1 ( 3.8)  \\ 
     Fujian &     1 ( 3.8)  \\ 
  PreviousSport (\%) &     \\ 
     Athletics &    16 (61.5)  \\ 
     None &     8 (30.8)  \\ 
     Basketball &     1 ( 3.8)  \\ 
     Football &     1 ( 3.8)  \\ 
   \hline
\end{tabular}
\caption{Athlete Desciptives (n = 26)} 
\label{tab:ethnoDescriptivesTable}
\end{table}


The team consisted of one full-time employee (Su Hailiang, by virtue of the fact that he was a Beijing resident), seven fully contracted athletes (\textit{xieyi}), five provisionally-contracted athletes \textit{shixun}, and six training team athletes \textit{erjiban} who do not receive a salary but received training, food, and board (see Table ~\ref{}).  The remaining athletes (seven) were classed as athletes in training \textit{jixun} and were effectively on a trial arrangement until they showed promise or else withdrew from the squad, either voluntarily or upon suggestion by the head coach.  Most athletes were from urban and rural areas of northern China (Shandong (11), Beijing (5), Jiangsu (3), Liaoning (2), Hebei (2), Anhui (1), Henan (1), Heilongjiang (1)). Athletes were, generally speaking, and from what I could gather, from relatively modest socio-economic backgrounds.

The average rugby training age (years spent playing and training in a rugby program) of Beijing athletes was 3.12 years ($range = 0.16 –-- 10 years$).  17 Athletes had a background in other sports (15 athletes from track and field, one from football, one from basketball), usually beginning part-time or full-time physical training at the age of 11-13.  Those who transferred to rugby from other sports did so either at the beginning of senior high school (16 years) or at university age (18 years).  The rest of the group (9) had no particular sporting background before starting training with the Program, and were scouted by the head coach of the Program or by school athletics coaches based on their basic athletic attributes (running speed, strength, coordination, and potential for physical growth).  Of the 26 athletes in the squad, three junior athletes who were part of the squad when I arrived in September 2015 since left, and three new athletes arrived during the time I performed research. This flux of non-contracted players in and out of the Program was quite common, the reasons for which I explain below (HYPERLINK).

%Contracted senior athletes ($average age = 24.3 years$) had trained for an average of 5.4 years, whereas the average training age of junior non-contracted athletes (average age = 19.3 years) was 1.7 years.


\subsubsection{Training schedule}

 The Beijing men's rugby team competed against other provinces in five national tournaments held in different locations across the country   every year between March and September. The period in which I conducted my first stint of ethnographic research (September 2015 –-- February 2016), therefore, constituted the off-seaon and pre-season components of the training year.  Due to cold weather in the north of China during winter and spring, teams from northern China (e.g. Beijing, Liaoniang, and Shandong procinves) often elected to train   at other domestic or international training locations depending on amount of program funding available and the training strategy of each program.  In 2015, before an unexpected change in coaching team at the end of December (explained below in Section HYPERLINK), the head coach of the Beijing Men's team had planned to travel to Yunnan province in early 2016 for one month of altitude training before moving closer to sea-level somewhere in the south of China for one month (February/March).  Following the coaching leadership change, the team did not leave Beijing until after Chinese New Year (25th February). Training during this period was therefore consistently stationed at the Institute in Beijing, and as such subject to occasional disruption due to Beijing's cold winter weather and air pollution.

 All athletes lived and trained 6 days a week at the Institute, and would occasionally attend university or high school classes as part of their ongoing education committments.  Below is a table of a typical weekly training schedule (see Figure ~\ref{tab:tournamentData}). A typical week consisted of 10 two and a half hour (150 minute) training sessions, seven of which were on-field rugby sessions, three of which were strength and conditioning sessions (not involving a rugby-specific skills).  In addition, two one hour evening skills sessions were also allocated for junior athletes to hone their basic skills of passing, catching, and game-play.  Athletes lived full-time on campus in the Institute's dormitory accommodation (usually 3 athletes per room), and were permitted to take overnight leave on the weekend after the conclusion of Saturday morning training.  Athletes from Beijing or with family in Beijing would usually take this leave, while the remaining athletes would spend weekends at the Institute.  Generally speaking, the rugby program would break at the end of the national season in September for two weeks, and occasionally around Chinese New Year for 7-10 days, unless New Year interrupts pre-season training plans, in which case training would continue in spite of this national holiday.

  \newgeometry{margin=0.5cm} % modify this if you need even more space
  \begin{landscape}
    \begin{table}[htpb]\caption{Weekly Training Schedule}
      \begin{center}
        \begin{small}
            \begin{tabular}{| c | c | c | c | c | c | c | c |}
              \hline
              & \bf M & \bf T & \bf W & \bf T & \bf F & \bf S & \bf S \\
              \hline
              0600 & Training &  &  & & & & \\
              \hline
              0900 &  & Training & Training & Training & Training & Training &  \\
                \hline
              1500 & Training & Training & & Training & Training & Training &  \\
                \hline
              1900 &  & Training (junior athletes) & & Training (junior athletes) & & & \\
                 \hline
            \end{tabular}
                \label{tab:tournamentData}
          \end{small}
        \end{center}
      \end{table}
  \end{landscape}
  \restoregeometry




\subsubsection{Participant Observation}

  \subsubsection{Interviews}

  In addition to 26 semi-structured interviews, I also conducted 6 unstructured interviews with three members of the Program (the head coach and the two most senior athletes) and two former coaches of the Institute.  These unstructured interviews were not included in the main analysis but provided important ethnographic context for the main analysis.

  \subsubsection{Informal Surveys}

  All 26 Athletes participated in the post-interview survey tasks.  For post-training surveys, 12 Athletes (age, etc) participated in the survey following the Beep Test training , 16 Athletes (age, etc) participated in the survey following the first match-like training session, and 15 Athletes (age, etc) participated in the second match-like training session a week later.

  A total of 23 Athletes (age, etc) participated in the general survey administered twice with a three month interval. This general survey occured either side of the change in coaching staff over the Christmas period of 2015.




\subsection{The Research Context}


\subsubsection{Team Factions}

After a short time at the Institute, my conversations with (predominantly senior) athletes and coaches helped me identify a number of factions in the team, which appeared to align not only with each athlete's position in the rugby program, but also by the incentives that each athlete was pursuiing through participation in the program.

\myparagraph{What was left of the Old Guard (2)}
At the top of the hierarchy, senior players Han and Lu were insterested predominantly in the prospect of gaining full time employment and Beijing residency through their continued involvement in the program.  Both athletes were employed by the Institute on full-time contracts (Xieyi), but neither were formal permanent employees (Zhengshi) of the Institute.  Lu had finished his undergraduate studies at CAU, while Han had failed to complete his studies at CAU, and had instead transferred to BSU to complete his undergraduate degree. Han was still completing the second year of his undergraduate studies when I arrived in 2015.

According to both Han and Lu, formal permanent employment and Beijing residency were part of the incentives that were originally tabled in contract negotiations in 2010 when both athletes were enticed to the Institute from CAU as young and promising athletes.  The Institute reportedly set performance targets for the programs of a Women's gold medal and a top-three finish for the men.  Given the chaos that transpired at the 2013 National Games, however, the Institute's promises did not materialise for athletes like Han and Lu, even though the men technically achieved their performance goal.
    \footnote{As it turned out, many of the senior athletes in the men's team pre-2013 were already Beijing residents, due to the fact that many were from CAU.  Thus, the Institute's promise of residency applied to only a small portion of athletes that they recruited.  After many athletes were enticed away from Beijing to the Tianjin rugby program in early 2014, the only first team athletes to remain at the Institute were Han and Lu.  Both expressed frustration to me about not receiving the incentives that both felt they had rightfully earned, and this sentiment motivated a lot of complaining and problematisation of the program.}
When I consulted outside observers on this situation, some suggested that the fact that both Han and Lu had not been granted the incentives that were promised to them was a deliberate mechanism, commonly employed by leaders of sports institutes to motivate committment to the team at least until the end of the 2017 National Games.  Suffice to say, both Han and Lu sat atop the athlete hierarchy of the men's team in an unfomfortable position in which they were not confident that they would achieve what they wanted from their involvement in the program, particularly given that hopes for success in 2017 were uncertain at best, due to the way in which the men's team had been decimated after the events of 2013. During my time at the Institute, out of all of the athletes, I interacted most directly and personally with Han and Lu, given that they were the most senior athletes and we had known each other previously from my time spent at CAU.

\myparagraph{The new senior athletes (8)}

Below Han and Lu were a collection of eight athletes who were now considered members of the ``first team'' (Zhuli).   These athletes were either on full-time contracts (Xieyi), or otherwise had been promised full-time contracts (in the case of one athlete who had transferred from Shandong, and another graduate from CAU). But unlike Han and Lu, none of these athletes were particularly central to the senior team pre-2013.  Many of these athletes were recruited between 2010-2013, from other programs at the Institute or from other sport programs such as football and athletics.  As such, very few of these new ``senior athletes'' expressed to me any strong short term aspirations for attaining permanent employment and residency (although some may have harboured such aspirations in the longer term). Five of these athletes were in the process of completing their studies at BSU, and the remaining three had already completed their undergraduate degrees at CAU.

%WW, MHT, WWX, WZF, SHL

\myparagraph{The unruly undergrads (5)}

Below the senior athletes were a group of five athletes who had been recruited into the program between 2010-2013, but had not subsequently progressed through the ranks to the first team. This group had, however, managed to qualify to study at BSU by virtue of their participation in national level tournaments in 2014 and 2015. As such, four of these athletes were beginning their first year of study at BSU, and the remaining athlete was preparing to do so. There was a clear distinction in technical competence between these athletes and the first team, and as such there was not a whole lot of competition between these athletes and the first team athletes for a spot in the first team.  Three of these athletes are on ``training contracts,'' while the remaining two are still on ``student'' contracts.  I call these athletes the unruly undergrads, as they are often criticised by coaches and more senior athletes for being constantly distracted by university life, and not having any motivation to commit to the rugby program, given that the key incentive available to them (university attendence) has already been awarded to them.


\myparagraph{The aspiring athletes from Chaoyang Sports School (5)}

Below the unruly undergrads are a group of five athletes who are also high school students and athletes at the rugby program at the Beijing Chaoyang Sports Institute, a city-level high school institute located in the east of Beijing.  These athletes had been playing rugby for one to three years, and were all attached to the Institute as ``second training class'' (\textit{erjiban} 二级班).  Second training class athletes received coaching, board, and food, but no form of remuneration.  All athletes are aspiring to transition to full time members at the Institute and representatives of the Beijing men's team.  In order to qualify for admission to the Institute as a contracted athlete, these athletes must first complete high school and attain the standard of a level 1 athlete (一级运动员), which can be achieved by representing the Beijing Youth Rugby team at a national level youth tournament.  Once these athletes become full members of the program at the Institute, they will then be able to pursue subsequent opportunities such as university attendance.

\myparagraph{The hopefuls on trial (6)}

Finally, the remaining group of six athletes were all at the Institute on a ``trial'' basis.  These athletes usually appeared via some connection to either the head coach of the Institute or via the relational network of one of the Principals at the Institute, as was the case with SHW (see Chapter 2 Vignette HYPERLINK). The position of these athletes at the Institute was deeply uncertain.  Most had transitioned from athletics or another mainstream sport in which they had achieved a minimum standard of performance in their event (Level 1) thus making them eligible to attend the Institute.  Most, however, had not played rugby before, and so had a large gap in technical competence that each was attempting to address.


\subsection{Daily Life at the Institute}
Head coach Zhu and assistant coach Shi looked after the day-to-day organisation of team schedules and training. Zhu and Shi were also assisted by a further two player-coaches, Han and Lu, who were in the gradual process of transitioning from athlete to coach status.  These player-coaches were included in the analysis as Athletes.  One of the Institute's four Principals, Jenny, was responsible for the management and administration of the rugby program, and as such was occasionally present at team meetings and national competitions.

%Athletes were either already contracted with the Program as either students, full-time contracts, or formal employees of the institute (in the case of only one athlete and the head coach) (n=13), or aspiring to become students, or contracted athletes (n=13).


\subsubsection{An abrupt leadership transition}
Zhu and Shi were ostensibly in charge of the team when I entered in 2015, despite their being signs that their authority may not have been rock solid.  If I had understood as much about the machinations of politics and power in Chinese sport then as I do now, then perhaps I would have understood Wang's disengagement from the post-Match team huddle was a clear signal that the writing was on the wall for Zhu, and that Wang---a Beijing resident and Asian Games Bronze medalist---would eventually take his place atop the Men's program.  Key senior players like Lu and Wei Wenxin complained to me constantly about Zhu and his approach to coaching and team management.  I could also tell the women's coach Ma was frustrated by Zhu and his communication style.  And then there was the interaction with being forced to take on SHW, and the lack of support from Institute leadership for off-season training plans overseas... It was also the case the Zhu had barely taken an on-field training session since I had arrived to do my research.  After a few weeks of me settling in and observing, Zhu would suggested that I take over half of the onfield sessions.  There was a period in which he barely took training that he announced to the team that he was focussing on the injured players instead...

In hindsight, it is quite clear that Zhu and Shi were perhaps only ever going to be a temporary stop-gap for the Institute's program; outsiders employed to come in and do the thankless job of ressurecting a sporting program rugby team from a shameful fall from grace.  But at the time, only four months into my first stint of ethnographic research and still largely unaquainted with (and without access to) the deep details of politics at the Institute, I was not prepared for what happened when I returned to Beijing from two weeks at home in Australia over Christmas.

When I returned to Beijing on the 31st December, I stayed the night at my friend Kai's place, and the next morning I messaged head coach Zhu to ask when would be best to return to the Instutute to resume research.  His reply was brief and somewhat odd: ``You should be able to [return to the Institute], you should probably make contact with Han or maybe Coach Wang.'' (应该可以,你联系一下小龙呗,或王导呗).  This reply was a little bit opaque, but I didn't challenge it as was the polite norm with someone like Zhu who was my senior.  Instead I contacted Han, as directed by Zhu, and asked if I could go to the Institute that evening and if the team had the afternoon off: ``Mmmm yeah.. we're off training today. I'm currently out (not at the Institute). Teacher Zhu and Coach Shi have left...now Wang is head coach.  You're still in Room 113 for the time being...'' (嗯嗯 休息 我在外面 朱老师跟石导走了. 现在由王重一主教练 你还是先住113).  I didn't really register what he was saying and replied:
``So Zhu and Shi have gone home to rest today too?'' (朱,石,回家休息是么)
``No, they are no longer coaching Beijing...When you get back we can meet and I will explain it all to you, ok? '' (不是的 是不在北京做教练了等你回来了 我们见面再好好跟你说说吧).

And like just like that, Zhu and Shi were out of the job, and Wang had assumed the role of head coach of the men's team.  Han and Lu became his assistant coaches, and some weeks later, Zhu Jing, a classmate of Wang from CAU who represented Xinjiang province in the 2013 National Games, joined Wang as assistant coach in Beijing.

In the aftermath of the change of leadership, I heard gossip that Zhu had pushed the Institute to provide more funding towards the team's off-season training program, which Zhu wanted to take either to Fiji in a best case scenario, otherwise to high altitude in Kunming (in China's southwest) and then down south to Guangzhou. After the Institute (presumably Jenny) refused to support these plans, Zhu allegedly made the call to resign from the role, saying that there was no way he could do his job properly.

Soon after the change, there appeared to be a renewed energy in the team.  Senior players Han and Lu had been promoted and appeared to respond to this promotion with positivity and enthusiasm.  Both Wang, Han, and Lu had all been coached by Zheng Hongjun, the old boss of Chinese rugby, and they all saw eye-to-eye about training techniques.  The training schedule immediately reverted to an older more familiar format, and the content of training was dedicated largely to basic skills, fitness, in order to support a style of play that Zheng pioneered in Chinese rugby.  Zhu had been attempting to lead the Beijing team towards a different style of play in the years that he had been coach.  I got on well with Wang (we had known each other from CAU days in 2008 and 2013), and so I also experienced an increase in positivity after he took control of the team.




\section{Analysis of Study Predictions}







  \subsection{Culturally specific contours}


    Prediction: Given the specific history of sport and rugby in China, and evidence concerning factors of indigenous psychology, I predicted that ethnographic evidence for the core predictions of this dissertation would be located in culturally specific terrain.

China - indigenous Chinese psychology for which theoretical generalisations around psychological self construal are useful.  History in which hierarcichal relationalism and draconian statecraft has dominated for centuries, but has also recently become problematised, and subsequently re-imagined via Marxist-Leninist state craft and revived neo-Confucian values.  (anti-Manchu and anti-Western reevaluation of Qing dynasty ressurected legitimacy of a Han  / Confucian system of social norms and state control).

The specificity of this terrain can be identified at multiple levels, at 1) a social institutional level, 2) group norms, and 3) action perception.

I will explain my observation of the cul specificity at each of these levels, before offering some interpretation.  The basic interpretation is that, in the case of rugby at XNT at least, there exists a tension between culturally dominant modes of hierarchical relationalism, and categorical modes of team membership, self-conduct and construal, and action and perception.







    \subsubsection{Institution as platform for activity of relational networks}

1. DESCRIBE:
P: Institution as platform for activity of relational networks

  - What is the institutional structure of XNT, the program, and the team?

One of Beijing's 4 sports institutes, home to 7 sports, compete at National Games, some athletes compete internationally.

A number of Principals overlook the programs, each being directly repsonsible for one of the programs.

Bureacratic backbone of regulations, administered by central government.  Eg, residency restrictions for employment, and so on.

Students come in, usually through guanxi connection to coach, to trial.  Known to be processes of bribes or sweateners, but of course this is within the realms of reason - the coach won't pick a kid that has no hope, because he must align with the coach's goals of being successful.



P: The strength of the coach clan as coordinator of power
  F: Coach clan dictates athletes who come in and out

  Interestingly, when I returned to Beijing after a 10 day travel break in February, two of the athletes on trial, at the Institute, XG and LJX, had since left.  When I asked senior athlete Wang Wei about these students, he suggested that ``they had interests with Old Zhu...Old Zhu said that he could solve their ``school problems'' [i.e. get them into university]. They were here to try to get in to University.'' (和老朱有利益关系,老朱说可以解决上学的问题 都是过来上学的).(This was probably true for XG; LJX's story was a bit more complex.  He had a tough run with injury, and had persisted for some time through a number of painful shoulder injuries, notivceably stressed by the situation).

  The day that I sat on the side of training talking to WW about Zhu and the trials connected to Zhu, a group of four new trialists from a high school in the neighboring province of Hebei, introduced to Wang through his circle, arrived at the Institute and joined training for the day. This coincidence was a telling indication of the power of the coach to coordinate the members and activities of the team.






  - Shunyi crowd arrived with WCY...
E:


P: But then the leaders also have ultimate sway over the coach,
  F: HXL with contract and residency permits
  F: ZPH when SHW came in above his control.
  F: the fat kid son of the ``leader'' participating in the final tournament






2. MY TAKE: tension of team and clan...

    %P: Institution as platforms with incentives and constraints, but not sacred in and  of themselves, less sanctimony around the institution than my intuitions prepared me for.

    %F: motivations for personal strategic life-course opportunities, more explicit than teammates
        %Rugby/Employment: Old heads (1st team) and undergrads (BYH, MXK etc)
        %junior: Young bucks (Chaoyang crew) and freshers (SHW etc)
          %family strong, team weaker but still present; coach??
        %F: CSC: not motivated, kicking the bucket and conversation with ZPH
        %F: MHT explaining connection between his mood and study

    %E: Any sanctity around the team appears secondary to more immediate strategic goals


Example: Last fitness session ``一团一团!''
Performed for me, for the cameras...


    %P:coaches organise clans :
    %F: ZPH --> WCY transition:
       %WWX criticism of ZPH; subsequent alignment with WCY & ZJ
       %F: When ZPH left, he took with him XG and LJX (LJX stayed committed to the institution of BJM, present at National Games 2017).
       %F: WCY's clan from Shunyi, Hebei
    %E:

    %P: Power of regulation:
    %F: BYH, SHL, and WCY privileged position: contrast with HXL and LP (and ZPH).
    %E:Setting up regulations in institutions to incentivise and constrain, but not something that generates trust in the category of the institution itself.



    \subsubsection{Social norms}

1. DESCRIBE:


Hierarchical relationism plays out in the program:

Coach at the apex:
``should have the coach as one motivation category'' - (Shanghai)
- Coordinator of resources and opportunities for each





2. MY TAKE:
    There's no I in team:
    Tension between encouraging self-determination, but then needing the ``team'' to move

P: Self-promotion as ultimate prosocial
  % HXL team Dinner speech and Johnny Zhang's Kids... what is going on here?
  % Team as team, but also as family (Team meeting, WCY, etc)
          % Self-determination in relational system:
                  %Younger players: Coach agency but my choice (figures?)


                  %Younger player aware of team roles
                  %Senior players: WW, CSC aware of cycle from young to old, whereas others are very critical of younger players not motivated.

P: The relational synchrony needed to enable/motivation:
          % ``Its all very complicated (in China)''

          %ZJ: can't move without leadership support (ZJ) conversation

          %HXL: return to his period of absence from the team after final tournament.




    \subsubsection{Action-Perception}


%can these processes be located in action-perception (active inference)?

          % Action scrutiny:
            %reference to BBall:

          % distributed dissonance:
            %training would fall apart when senior players absent
            % Role of senior players to arouse dissonance / motivation on behalf of the group, rather than that motivation being located within the individual
            %Football games:
            %FC: I got angry at him, he said he was grateful, felt the dissonance, and will use it as a marker for futher improvement...


SUMMARY:


















  \subsection{Contours of Generalisable Mechanisms}

    \subsubsection{Performance in Joint Action}

      \myparagraph{Performance related anxiety (predominantly Junior)}


          \subparagraph{Team Awareness}

          Many newcomers are transitioning from individual sports over to team sports...as such there is an unfamiliarity with not only the specific joint action demands of rugby, but also the norms of group membership.. an ``awareness'' of the norms and expectations of group membership at the rugby program at XNT

        P1: %Anxiety around complexity of rugby, and the ``awareness'' required to execute team based joint-action:
        Number of athletes quoted team awareness as the hardest thing?

        P2: %Anxiety included reference to specific components of performance
        %TEAM: attack, defence, support play, communication
        Some examples from interviews?


        \subparagraph{Individual Social Shame (negative violations)}
              %Anxiety around individual letting the team down due to individual mistakes

              %Anxiety included reference to specific components of performance
                %IND: contact, tackle, passing, decision making, support play in attack






  \myparagraph{Strategy in individual performance and deflection of responsibility (more predominantly Senior athletes)}

          \subparagraph{Team Awareness: Agency over, deflection of responsibility}
       %Deflection of responsibility towards junior athletes: criticism of junior athletes for not committing, playing computer games, being complacent.  (Irony that LP was one of the most vocal when discussing computer games at the dinner table).
       %Others more generous and circumspect: WW and CSC realise that its a progression, and that individuals
       %EXPLAIN: reduction of dissonance via deflection

        \subparagraph{Individual performance: strategy}
       %MHT, CSC, HXL, WWX, MC use of experience to strategically avoid over-exertion and injury risk.







\myparagraph{Survey Results}

      SURVEY RESULTS: relatively equal levels of flow overall, but higher performance related anxiety in junior athletes, particularly in the scratch matches, which required high levels of technical competence and joint action coordination.

      %EXPLAIN: perception of performance in relation to social expectations of the team: joint action participation, as well as individual responsibilities (team member, self-determination)

      %Senior athletes talk with more composure regarding performance, any deflect responsibility for being the agent of team performance, and show strategy regarding how to regulate energy expenditure.





\myparagraph{Potential Mechanism: Positive Violation of expectations Performance related exhilaration and generalised emotionality (predominantly Junior)}

              %SWH story: the buzz and glow from acquisition of skill

          \subparagraph{Ind components}
              %WZF: explanation of the first side-step
              %YC: Fending - the feeling of picking up
        \subparagraph{Team performance}
              %WZF: likens team performance to side-step
              %HXL: rugby wasn't interesting (wasn't motivated until he started to pick up the team dimensions)

        \subparagraph{Individual Social Shame (negative violations)}







      \subsubsection{Team Click}


Junior Athletes:

Either:
        \myparagraph{Generalised emotionality, equating click with bonding (Junior)}

unspoken understanding linked with cohesion (tuanjie)

Or:
        \myparagraph{inability to conceive of click, not qualified to talk about it}


Either:
        \myparagraph{Familiarity and granularity/precision (Senior)}
        %HXL: Aura/atmosphere of the teams
        %WW: flow description
        %WZF: consideration of all the individuals
        %CSC:
        \myparagraph{Problematised}
          %Juniors for not being committed (Undergrad loafers)
          %Coach/Leadership for not supporting (LP)
          %China:
            %the system - No initiative, Chinese society too complex
            %the culture - Confucian education

            %Nostalgia for old regime pre 2013: HXL, LP,
          %E: a cognitive strategy in reorganising information in a way that maintains coherence (Nowak 2017)



  \subsubsection{Social Bonding}


    \myparagraph{Emotional Support}
    %MC, LZS, SHW, GJP



    \myparagraph{Shared Goal}
    %WW:



    \myparagraph{Identity Fusion}
    %Fun/interesting/compelling (juniors)
    %Attachment: MXK, LZS,
    %Fusion: HXL, LP,




  \subsubsection{Moderator Variables}
      \myparagraph{Technical Competence}


      \myparagraph{Personality}


      \myparagraph{Injury}


      \myparagraph{Fatigue}
























\section{Discussion}



The cultural hyper-priors of JA - Team Click - SB <- factors of attraction for cultural evolution.
Relational and Categorical modes in interaction and flux in Chinese sport.






\section{Conclusion/Contribution}
Cross-cultural psychology / anthropology, CAT

recap
