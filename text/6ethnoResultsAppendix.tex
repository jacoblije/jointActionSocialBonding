
\chapter{\label{app6:ethnoResults}Ethnographic Results}

                                                \begin{CJK}{UTF8}{gbsn}




\section{Analysis}
I collated all ethnographic data into a corpus that was subjected to a process of ``thematic analysis'' \citep{Braun2006}.  As Braun and Clark \textcite[10]{Braun2006} explain, ``A theme captures something important about the data in relation to the research question, and represents some level of patterned response or meaning within the data set.''  Identification of recurring themes was guided by (but not limited to) the research questions and research hypotheses outlined in Chapter 2.  Themes were identified on both explicit and implicit levels of the data \citep{Boyatzis1998}.  \footnote{The story of Sun Hongwei (see the opening vignette of Chapter~\ref{sect:SHW}) serves as a good example of the contrast between implicit and explicit levels of data.  In our interview, Hongwei was able to articulate many conceptions relating to joint action and social bonding, all of which were recorded as explicit declarations.  At the same time, I observed in training that Hongwei, while fluent in his declarations in interview, was far form fluent in his more embodied declarations on the field.  The combination of implicit and explicit data offered points of contrast and comparison that were fruitful for richer analysis.}
1. Thematic analysis involved three stages that unfolded in a recursive (rather than linear) fashion \citep{Braun2006}. In phase one, I familiarised myself with the each data source in the corpus and tagged relevant extracts with theoretically-guided ``codes.'' For example, upon encountering Hongwei's description of his position in the team in his interview transcript (cited in Chapter~\ref{sect:SHW}), I tagged this with codes such as ``group membership,'' ``mutual support,'' ``emotional support,'' ``knowledge of team roles,'' ``signalling commitment to team'' etc.  My coding system was thus directed by (but not limited to) pre-identified research questions rendered as open and investigative questions, i.e.,
  1. 1) how did athletes athlete experience of joint action,
  2. 2) perceptions associated with the phenomenon of team click, and
  3. 3) perceptions relating to social bonding.In addition, I looked for evidence for
  4. 4) possible moderator variables, such as technical competence and personality type, as well as the relationship between physiological exertion, fatigue, and injury and joint action, team click, and social bonding.
2. For each data set, I created a data frame using Microsoft Excel (Version 14.7.1) in which research participants formed the rows, and distinct codes formed individual columns. Data extracts from interviews and field notes were imputed into the cells of this matrix, with an emphasis on including data surrounding each text extract, in order to preserve context \citep[see][]{Bryman2001}.
1. In phase two, I sorted the different codes into potential themes and collated all the relevant coded data extracts within the identified themes and judged on the dual criteria internal homogeneity of codes within themes (coherence) and heterogeneity of codes between themes (distinction) \citep{Patton1990}.  I then produced a master data-frame ($ participants \times themes$), in the cells of which data extracts from all data sets were included.  In phase three, I generated a definition of each theme, and a refined list of data extracts capable of representing that theme in subsequent analysis \citep{Braun2006}.











\myparagraph{Computer games\label{sect:computerGames}}
Guo's utilisation of computer games as an analogy for the team sport of rugby requires further explanation. Soon after arriving at the Institute in 2015, I discovered that computer games were extremely popular among all athletes.  At the time, from what I could gather by way of conversations, the most popular came among the athletes was a game called League of Legends (LOL). LOL is a is a multiplayer online battle arena video game, in which small teams of individuals team up to fight other teams \citep{Wikipedia2018}.  I attended many meals with the team at the canteen during which discussions of computer games dominated. I was usually a passive listener in these conversations, but sometimes involved myself in an attempt to learn more.  ``Actually,'' said Lu Zhongsheng, turning to me during one such conversation:

  \begin{quote}
    ...there are so many similarities between LOL and rugby.  Different characters have different strengths, 1on1 battles are the basis, but then it also relies on team work and team formations and strategy.  So its no wonder that we love playing these games.
  \end{quote}

  \begin{quote}
    LOL和橄榄球有很多相似的地方, 不同的角色有不同的优势,一对一的战斗为基础,但是它也依赖于团队合作和团队组建以及战略。 难怪我们喜欢玩它们。
  \end{quote}

  Lu's comments were consistent with research conducted by Justin Hornbeck on gamers in China, which suggests that explanations for adherence to gaming in China should not be limited to the fact that they promote hedonic experiences surrounding competitive and violent behavioural tendencies \citep{Hornbeck2012}.  Hornbeck reports ethnographic and survey evidence that participation in multis-player computer games like LOL also activates moral and social cognitions. One key dimension of LOL game-play, for example is interdependence and belonging to a supportive (virtual) community \citep{Hornbeck2015}.  Han continued the conversation, making a direct comparison between gaming and sport:

    \begin{quote}
          The thing with gaming is that there are such low entry costs.  Anyone can play, as long as you can do this (he motioned the clicking of a computer mouse with his thumb, index and middle fingers).  Sport isn’t the same, if you can’t run or pass or kick then there’s no way you can participate.
    \end{quote}

    \begin{quote}
          玩游戏的入门成本太低了。 你可能不是那么好,任何人都可以玩, 只要你会这个\textellipsis运动不一样,如果你不会跑或不会传或不会踢,那么你就无法参与了。
    \end{quote}

This led Wang Wei to start talking about a gamer who only has one arm, and so uses his leg and foot to control the keyboard, and his arm to control the mouse.  Everyone laughed but also expressed respect and wonderment.  It served as a conclusion to the conversation that online games are in fact a huge phenomenon.  This conversation helped explain the reason why Guo Junping chose to use computer games as an analogue for rugby.  The sense of team, and of belonging to a team appear to be a core reason for the comparison.

Han's comparison drew attention to the importance of technical competence in gaining access to participation and group membership in team sport.  A sport like rugby requires intense levels of psychophysiological commitment over long periods of time in order to master the suite of game-specific skills.  In addition, Han indirectly explained why so many Beijing athletes were seemingly more enamoured with LOL than they were with rugby on a day-to-day basis.  It was much easier to master the click of the mouse than it was to master the click of joint action in rugby.
