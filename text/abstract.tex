The key assertion of this dissertation is that a scientific explanation of the cross-cultural prevalence of group exercise in human sociality must pay greater attention to the component mechanisms and system dynamics of joint action. To explore and substantiate this assertion, I conduct research with professional Chinese athletes who participate in the competitive interactional team sport of rugby union.  I utilise ethnographic and field-experimental methods in order to examine the specific relationship between joint action and social bonding.  Results provide evidence for a relationship between athletes' perceptions of success in joint action and social bonding.  Interestingly, this relationship appears to be mediated by ``team click'' ---a novel construct, theoretically grounded and ethnographically substantiated, which pertains to the subjective feeling associated with optimal interpersonal coordination.  Considered in light of existing debates on the process mechanisms of social cohesion, these results suggest that processes of psychophysiological alignment achieved through interpersonal movement regulation and coordination could be central to the formation of durable social bonds and the transmission of cultural practices between individuals and within groups.
