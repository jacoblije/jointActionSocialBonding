





\subsection{Implications for the present study}
The literature reviewed above potentially poses an interesting challenge to the way in which social bonding is understood within cognitive and evolutionary anthropology.  The professional Chinese rugby players that form the focus of this dissertation are young men and women predominantly from rural areas of China's northeast, and are therefore likely subject to relational modes of group membership made predominant and durable by persistent cultural and linguistic processes associated with group membership in contemporary China \citep{Liu2009}.  In addition, athletes are members of a relatively small and stable team environment, for which access to benefits should require attention to the maintenance of productive intragroup relationships, more so than processes of intergroup mobility or comparison \citep{Schug2010}, even though intergroup comparison is an inherent component of competitive interactional team sport. However, given the fact that rugby is an imported team-based interactive sport, for which categorical modes of group membership are required and celebrated, I also expect athletes to exhibit a categorical mode of group membership.
In addition, I expect the metaphors of family to be prominent in the scaffolding of team processes, and I also expect to find a tension between loyalties to the team, and loyalties to an athlete's pre-existing relational networks of family and friends\citep{Yang1994}.  Thus, the specific cultural setting is such that I do not necessarily expect to encounter the type of public representation or testimony of group membership common in Western sporting parlance, more indigenous to the rugby pitches and boathouses of Oxford or any high school American Football movie produced in the 1990s.  Instead, I expect to find evidence of a link between joint action, team click, and social bonding expressed and embodied in cultural representations that may vary distinctly in structure and content from Western intuitions based on a predominantly categorical mode of social cognition.




















\section{Research methods}
Ethnographic data included: unstructured and semi-structured interviews with athletes and coaches (yet to be analysed in-depth), general and activity-specific surveys, and field notes based on participant observation of daily activities of the team.
