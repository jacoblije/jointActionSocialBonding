\chapter{\label{researchSetting}Research Setting}


\minitoc


%In the case of group exercise contexts, therefore, it is essential to thoroughly consider both 1) the local parameters of joint action associated with the group exercise context, as well as 2) the global ecological and cultural frames in which the group exercise context is situated.  In this dissertation I concentrate on the empirical case of rugby union in contemporary China.  In order to provide a sufficient contextual grounding for the three empirical studies that follow, in this Chapter I introduce 1) the parameters of joint action associated with rugby union, 2) important psychological and cultural factors relevant to social cognition of joint action in China, and 3) a history of physical activity and sport in China, from which rugby in China has emerged and in which it remains situated.


\section{Abstract}

In this chapter, I introduce the specific group exercise context in which I test theoretical predictions relating to joint action, social bonding, and the mediating role of team click, as outlined in Chapter 2.  Rugby union in the People's Republic of China requires thorough contextualisation in order to identify evidence of variables of interest.  In this chapter, therefore, I outline the parameters of joint action, team click, and social bonding relevant the sport of rugby union, as well as the macro-cultural context of contemporary China.  I then outline the history of physical activity and sport in China, from which the specific group exercise setting of rugby in China is situated.   In addition, I outline the (unique) combination of qualifications that enabled me to conduct research in this specific context.  I conclude by updating the theoretical predictions of this dissertation in light of the specificities of the research context of rugby in China.

                                          \begin{CJK}{UTF8}{gbsn}

\section{Vignette: The Temple of the God of Agriculture Sports Institute}
I first visited the Beijing Temple of the God of Agriculture Sports Technology Institute (\textit{Beijingshi xiannongtan tiyujishu yundong xuexiao} 北京市先农坛体育技术运动学校,
hereafter the Institute) first thing in the morning on my first Monday in Beijing.  I entered via the main entrance in the south and made my way west to the main administration building by hugging the southwest perimeter of the 30,000 seat capacity multi-purpose sport stadium that spatially dominates the Institute's campus (see map ~\ref{fig:beijingXNT}). My aim that morning was to confirm the details of my proposed research with the vice principal of the Institute who was responsible for the rugby program, as well as the head coach of the rugby program.  I knew both the vice principal and the head coach from previous times spent in China studying (2006 and 2008) and coaching and coaching rugby (2013) prior to my doctoral research, and I had already received positive responses from both during the planning stages of my research.  But I still needed to confirm arrangements for research face-to-face.
%On that first morning I had two meetings scheduled: one with the vice-principal responsible for the administration of the rugby program, and the other with ZPH, the head coach of the Beijing rugby program.  Jenny I knew from interactions with the Beijing rugby team prior to the National Games in 2013, and I had originally met ZPH in 2008 when he was an assistant rugby coach and Chinese National team representative at CAU.  I hoped that Jenny and ZPH would both grant me the permission I needed to conduct research with the Beijing rugby team.

When I announced to friends within the Chinese rugby community in Beijing that I was preparing to conduct research with the Beijing rugby team at the Institute, a few who were well acquainted with the politics of rugby, Beijing, and sport more generally in China warned me about becoming too involved, perhaps worried that I might suffer a similar fate to the group of coaches and athletes involved in the National Games in 2013.  One friend Adrian, one of the first captains of the Chinese national team (he captained in 1992, two years after the first cohort of CAU players were recruited) warned me (half) jokingly to be careful, citing that the Institute was riddled with ghosts: ``There really are ghosts there, I'm telling you Lijie, you should keep your distance'' (真的有鬼啊,我告诉你李杰,你最好离远点吧).  I naturally scoffed at these predominantly tongue-in-cheek warnings, although I was intrigued to learn more about the cultural, political, and psychological (as well as ancient cosmological) principles that lay beneath the history of the Institute.

The Institute is centrally located in Beijing, just to the west of Yongding gate in on the South 2nd Ring Road. As one can imagine, given Beijing's 3000-year history, the land on which the Institute sits was not always home to sport facilities and professional athletes.  The Institute takes its name from the temple that was built on the land in during the Ming Dynasty in the 15th century.  Yongding gate marks the southern end of the city's ancient north-south axis, which also includes, from south to north, Tiananmen Square, the Forbidden City, Jingshan Park, and the Drum and Bell Towers (see Figure ~\ref{fig:beijingTemplesXNT}). Chinese cities have for thousands of years, been laid out on a north-south axis according to the principles of feng shui. The auspicious power (\textit{qi} 气) of each monument along this axis is believed to flow upward from the south---south being the most auspicious and therefore important of the Four Directions in Chinese cosmology.

Dynastic powers also sought to reinforce the cosmological order temporally, through regular performances of a system of Grand, Middle, and Common Sacrifices.  In the Ming and Qing Dynasties, Emperors (or their commissioned representatives) used the Temple of the God of Agriculture (hereafter the Temple of Agriculture) during the middle month of Autumn and Spring (according to the Chinese lunar calendar), to perform Middle Sacrifices (in a system of Grand, Middle and Common Sacrifices) in honour of the Harvest (\textit{nong} 农)}---one of the four main cosmological principles, in addition Heaven (\textit{tian} 天), Earth (\textit{di} 地), and Ancestors (\textit{zu} 祖)\citep[98]{Brownell2008}.  When the Qing Dynasty (1644---1912) finally buckled under the pressure of Western Imperial occupation and popular revolutionary political movements in 1912, however, Confucian Sacrifices in Beijing's various temples ceased.  But as one form of ritual expired, another form began.

The embrace of the practice and spectacle of modern sport by the Republic of China (ROC, 1912-1948) and the People's Republic of China (PRC, 1949-present) has been such that sport stadiums (and the sporting events for which these stadiums were specifically constructed) have supplanted ancient temples and rites along Beijing's sacred north-south axis as a medium of communicating state order \citep{Brownell1995}.  The flow of auspicious cosmological energy now begins with the Temple of Agriculture Sports Institute (founded in 1930) in the south, and ends 21km to the north where the National Stadium, National Aquatic Centre, and the Olympic Green---the iconic monuments of the Beijing 2008 Summer Olympics---are situated.  Since re-joining the International Olympic Committee in 1979, China's athletes have participated at every Summer and Winter Olympics (except for the 1980 Moscow Summer Olympics, which China joined 65 other nations in boycotting following the Soviet Union's invasion of Afghanistan in 1979), winning over 600 medals in 28 different sports.  China's elite performance on the international stage is facilitated by a now enormous state-sponsored competitive sport system (\textit{jingji tiyu tizhi} 竞技体育体制), which consists of thousands of sport programs housed in secondary and tertiary level education institutions and specialist sport institutes throughout China's 34 provincial level regions.  According to China's National Bureau of Statistics, in 2016 China's sport industry reached a total scale of RMB 1.9 trillion yuan (USD300 billion), and has been growing by an average of 18.2\% per year since 2014, when the central government first declared plans for sport to become a ``pillar industry'' of China's modern economy, setting a target total scale of 5 trillion yuan (USD800 billion) or 2\% of GDP by 2025.

Since the period in history when sport first became associated with the Temple of the God of Agriculture, in the form of a horseracing track at the southern gate of the Temple at the end of the Qing dynasty, not all of the energy that has flowed from the sacred site has been auspicious, however.  As I will explain below, the arrival of rugby to the Institute has been the predominant source of much of the most recent cosmological turbulence.  Thus, although I was personally captivated by the fact that the Institute was located in such a traditionally sacred location in Beijing, I was also slightly apprehensive about the state in which I was entering the Program and the Institute. It was with a slight pang of nervousness, therefore, that I made my way from the southern gate around the stadium towards the main office building to meet the Institute's vice-principal.

\begin{figure}[htbp]
  \includegraphics[width = \linewidth]{images/beijingXNT.png}
  \caption{Location of the Temple of the God of Agriculture Sports Technology Institute  (Source: Google Maps)}
  \label{fig:beijingXNT}
\end{figure}


\begin{figure}[htbp]
  \includegraphics[width = \linewidth]{images/beijingTemplesXNT.png}
  \caption{Locations of former Qing dynasty temples (Brownell 2008)}
  \label{fig:beijingTemplesXNT}
\end{figure}




\section{Introduction}

A core assertion of this dissertation is that a scientific explanation of the cross-cultural prevalence of group exercise in human sociality must pay greater attention to the component mechanisms and system dynamics of joint action.  Existing empirical evidence linking group exercise and social cohesion is derived predominantly from laboratory experiments, in which essentialisations of group exercise---namely, behavioural synchrony and physiological exertion---are independently manipulated within dyads or small groups.  Results of these studies indicate that various forms of controlled and tightly coupled  behavioural synchronisation (from finger tapping to walking to rowing to talking) and certain intensities and durations of physical exertion are independently responsible for generating pro-social behaviour \citep{Tarr2015}.  There is also some evidence to suggest that synchrony and exertion may interact to produce social bonding \citep{Lewis2018}.  Common to many group exercise contexts appears to be the creation of a psychophysiological environment---a ``social high''--- conducive to social affiliation and in-group cohesion \citep{Davis2015}. Researchers have coordinated evidence from social psychology and neuroscience to suggest that both 1) cognitive mechanisms associated with social interaction and 2) neuropharmacological mechanisms associated with pain and reward could mediate the social high, and these mechanisms may be moderated by the level of synchronisation, exertion, or group size of the specific activity \citep{Mogan2017}. There is strong evidence to suggest that these mechanisms independently (and in concert) promote a range of prosocial behaviours, including perceptions of emotional closeness, spontaneous helping behaviour, and cooperation in economic games. The theory of cultural group selection provides a theoretical framework for explaining how the proximate mechanisms of the social high of group exercise could be relevant to the observable cross-cultural prevalence of group exercise in the contemporary and historical record \citep{Dunbar2010,Whitehouse2014a}.

The current empirical inputs to the social high theory of group exercise and social bonding are limited largely evidence from laboratory studies that tend to essentialise the components of group exercise.

In this dissertation, I argue that the social high theory of group exercise and social bonding lacks a sufficient understanding of the component mechanisms and coordination dynamics of joint action.  These component mechanisms and their coordination are observable in various real world settings, and are beginning to be theorised and quantified in emerging paradigms of cognitive science and social psychology \citep{Marsh2009}.  Anecdote from highly skilled joint action practitioners, and emerging theory evidence from the social cognition of joint action supports the possibility that the proximate mechanisms of group exercise may require dynamic coordination of a certain perceivable \textit{quality} in order to generate pro-social behavioural outcomes.  The social high of group exercise may in some instances depend crucially on the perception of ``click'' of coordination in joint action.

Sporting anecdote has for some time pointed to the importance of quality of coordination in processes of affiliation and team cohesion.  In the case of highly skilled practitioners in particular, the ecstasy of group activity appears to be contingent not on mere participation in joint activity, but on the perceived achievement of fine grained alignment of behaviour with co-participants.  The psychology of flow offers support for this dimension of reward associated with peak athletic performance \citep{Jackson1995}. Scientific knowledge of flow states, however, remains heavily restricted to individual level neurocognitive and psychological processes, and thus lacks insight into specific mechanisms and coordination dynamics of joint action.  Promisingly, a combination of 1) a recent paradigm-shift in cognitive science away from static input-output computational models of information processing towards ``active'' inferential models based on mechanisms of prediction and free-energy minimisation \citep{Friston2010,Clark2013}, and 2) empirical research programs dedicated to broadening the frame of analysis of social cognition to include interactions involving the affordances of distributed throughout the brain, body, and the social and physical environment \citep{Sebanz2006,Semin2008}, have begun to confirm elements of intuition and anecdote related to the phenomenon of ``team click.''  This dissertation offers a novel theoretical framework as well as a multi-method and step-wise empirical investigation into the relationship between coordination in joint action and social cohesion in-group exercise.

Current research indicates that the perception of tacit understanding, team atmosphere, and general order in alignment of behaviours with co-participants (the dimensions of team click outlined in Chapter 2) could be due to the activity of a suite of cognitive mechanisms associated with 1) lower level sensory perception and movement regulation processes, 2) different levels of joint action representation and execution \citep{Semin2008,Frith2008,Pesquita2017}.
Importantly, skilled practitioners in particular demonstrate a capacity to co-regulate movement in joint action such that joint action resembles a functional interpersonal synergy (SOURCE).  Functional interpersonal synergies are coordinative structures capable of reducing informational uncertainty in a cognitive system \citep{Riley2011}. In this way, interpersonal synergies can be understood as extra-neural mechanism for minimising cognitive free energy in complex interactional environments \citep{Clark2013}. There is evidence to suggest that the formation and successful regulation of functional interpersonal synergies is underwritten by a regime of neuropharmacological reward \citep{Ross2013}. This evidence indicates that humans are endowed with the capacity to discriminate on a spectrum of low to high quality coordination, and have a psychophysiological preference for the latter end of the spectrum \citep{Marsh2009}.  By examining in detail the phenomenon of team click in a joint action context involving highly skilled practitioners, this dissertation seeks to illuminate the way in which component mechanism and coordination dynamics of joint action typical of group exercise contexts may be relevant to processes of social bonding and group cohesion.

%Evidence suggests that joint action specific processes bootstrap on lower-cognitive mechanisms associated with movement regulation.
%The perceived click of joint action with co-participants in group exercise offers a way in to quantifying this coordination in real world settings.
% (repetition) In sum, it appears that in some instances, what is crucial to the social high of group exercise is not mere activation of mechanisms of behavioural synchrony and exertion, but the \textit{coordination} and \textit{quality} of the activation of these mechanisms.



%Contribution: (See Cohen2007 and CohenWhitehouse2012)
The first time ethnographic evidence has been provided for mechanisms of joint actions (Wacquant: body and soul), Body for China (Brownell1995)
This ethnographic theorises divergent cultural modes
Cultural specificities may act as ``factors of attraction'' \citep{Sperber2014} that constrain and direct the fixation of cultural variants within and between populations.







\subsection{The affordance of culture}
A key insight overlooked by the existing social high account of group exercise and social cohesion, but revealed by the paradigm shift surrounding the social cognition of joint action, is the sensitivity of joint action (or any cognitive process for that matter) to informational affordances provided by various layers of ecological and cultural context.  The cognitive inputs to joint action in real world settings are rarely limited to essentialise components administered in laboratory paradigms. It is known that cognitive processes relevant to joint action are distributed throughout brains, bodies, and the physical environment of the ecological niche in which it is situated.  There is evidence to suggest that joint action co-participants rely on a series ``frames of reference'' for joint action execution \citep{Ray2018}, and it appears that social interaction functions best in situations where there is a snug fit between individuals' implicit cultural expectations and explicit rules for engagement \citep{Vollan2017}.  Indeed, widespread attention to the ``reproducibility crisis'' in social science \citep{Earp2015,Rathmacher2017}, suggests that even the most controlled spaces and procedures of experimental studies involving human subjects are laden with unquantified and often culturally specific cognitive affordances that constrain and enable quantified outcomes.
%The experimental approach to science requires re-evaluation in light of an emerging paradigm emphasising distribution of cognitive processes throughout an array of coordinated informational affordances.

Discussed as part of the theoretical framework outlined in Chapter 2, shared cultural knowledge can act as a ``coordination smoother'' \citep{Vesper2017} for joint action, enhancing the effectiveness and efficiency of joint action between co-participants who share a similar informational framework.  In the predictive coding paradigm, cultural habits and frames of reference act as ``hyper-priors'' that set the macro-contextual coordinates for joint action\citep{Clark2013}.  Contextual affordances for joint action appear to be dictated by processes operating at multiple conceptual levels---from the micro-level predictive processes associated with movement action and perception, to the macro-level predictive frames offered by specific cultural and contextual niches---interact in complex processes of reciprocal causation to shape joint action (SOURCE).  Conceptualisation of the causal complexity of cognitive processes relevant to joint action in this way echoes a broader reconceptualisation of the causal complexity associated with change on an evolutionary timescale, which recognises that human behavioural phenomena is the result of a number of biological, cognitive, and ecological mechanisms that interact via reciprocal feedback loops spanning varying scales of time and space \citep{Fuentes2015}.

%The Role of Culture: CAT (Sperber2014, )
%Link joint action and social bonding to information transfer and therefore cultural evolution:

This theoretical approach to social cognition of joint action (and human evolution more broadly) accords neatly with anthropology's long-standing concern for attending to the distinctiveness of cultural trajectories, and offers a space for reconciliation between anthropology and cognitive and evolutionary approaches to human behaviour \citep{Whitehouse2012}.  Prior to appropriate acknowledgement of the complexity of cognitive processes and psychological phenomena, researchers within the human cognitive, behavioural, and evolutionary sciences (for example, cultural and developmental psychology, see SOURCE) have been prone to overlooking local cultural specificity and causal complexity when seeking to generalise to the human species results of studies conducted with mainly Western subjects and methods \citep{Henrich2010d}.  Research agendas and the specific experimental designs to which they give rise are shaped by the historically and culturally contingent assumptions and priorities---predominantly of ‘‘WEIRD’’ (Western, Educated, Industrial, Rich, and Democratic) societies and experimental samples.  It is now clear that the complexity of observable behavioural phenomena can only be addressed by embracing a plurality of methodologies to systematically document variation within---and not simply between---cultural niches \citep{Fuentes2016}.  Anthropology is thus well placed to expand upon accounts of group exercise, via methods ranging from ethnographic exploration capable of uncovering novel dimensions of behaviour and generating testable hypotheses, to quantitative techniques---e.g., experimental and mathematical simulation paradigms---capable of testing hypotheses

In honour of the capacity of cultural and ecological trajectories to shape and direct observable behaviour in distinctive ways, the three empirical components of this dissertation are confined to one specific research setting---professional rugby in the People's Republic of China. Each study builds on the previous study in a step-wise manner, and as such the cultural and ecological affordances associated with the group exercise context can be identified and held relatively constant.  In order to explore the validity of theoretical predictions (formulated in Chapter 2) in a real world group exercise setting,  I begin with an in-depth ethnographic study of the Beijing Men's Rugby Team.  I then refine my theoretical predictions based on the results of ethnographic analysis, and test these in an \textit{in-situ} survey study of a more representative sample of professional Chinese athletes before, during and after a National Championship Tournament.  These two studies provide the necessary grounding for a controlled field experiment designed to interrogate specific mechanisms hypothesised to underpin the phenomenology of team click and social bonding in joint action.  The empirical findings of this dissertation confirm 1) the relevance of component mechanisms and coordination dynamics of joint action to processes of social bonding and group cohesion, and 2) the importance of a diverse and coordinated methodological approach to measuring human behaviour---one that honours the contextual complexity of informational affordances in which cognitive processes are distributed and grounded.

In order to provide a sufficient contextual grounding for the three empirical studies that follow in this dissertation, in this Chapter I introduce 1) the parameters of joint action associated with rugby union, 2) important psychological and cultural factors relevant to social cognition of joint action in China, and 3) a history of physical activity and sport in China, from which rugby in China has emerged and in which it remains situated. I also outline the (unique) combination of qualifications that enabled me to conduct research in this specific context.  I conclude the chapter with an update to theoretical predictions of this dissertation in light of the specificities of the research context of rugby in China.

%In order to set the scene for the empirical studies of this dissertation, therefore, in this chapter I introduce the specific group exercise context of professional rugby in China.  I identify in the activity of rugby union and the cultural trajectory of China evidence for the social relevance of joint action, including evidence relating to the phenomenon of team click in both contexts.   I then outline the history of physical activity and sport in China from which my specific instance of rugby in China emanates.  I then outline of the (unique) combination of qualifications that enabled me to conduct research in this specific context.  I conclude the chapter with an update to theoretical predictions of this dissertation in light of the specificities of the research context of rugby in China.

\section{The Group Exercise Context: Rugby union in China}
The task of introducing the specific group exercise context of this dissertation requires a consideration of 1) the micro-level components and dynamics of joint action typical of rugby union (specifically rugby 7s), and 2) a description of the way in which these micro-level processes interact with, and are shaped by, the specific historical and cultural trajectory in which they are embedded---namely, sport in the People's Republic of China (PRC, or simply China).  The experience of professional rugby players in China is nested within various layers of China's modern history, defined by projects of governance, state building and participation in the international community.  The cultural affordances recruited to facilitate these political and social activities---of which sport is central---emanate from a range of sources both foreign and indigenous to China, and have evolved as the result of China's modern history of confluence and antagonism with the world beyond China's sovereign borders.  As such, while the prescribed rules of rugby union are legislated by an international governing body and therefore in theory relatively consistent across the globe, the rugby played in China exhibits a particular style and quality that relates to the institutions, social norms, and culturally-shaped tendencies of action and perception specific to China. In this section, I note the relevant components of joint action typical in rugby contexts, and I outline the relevant layers of context in which this joint action is enmeshed, up to the present moment in China.

  \subsection{Rugby Union}

The parameters of joint action typical in rugby union make the sport highly suited to test the theories prescribed in Chapter 2, namely, the relevance of mechanisms and coordination dynamics of joint action to processes of social cohesion.  Rugby union entails high levels of both physiological exertion and socially coordinated movement, and is anecdotally and colloquially associated with social bonding.  In this section, I introduce the specific history and dimensions of the group exercise context of rugby union.

Rugby Union (hereafter rugby) is an interactional team sport played on a rectangular field (100m x 70m), by two teams, usually of 15 players, who physically contest possession of an egg-shaped ball that can be used to score points \citep{IRB2014}.  Descending from a variety of locally-specific folk games played in pre-industrial England, all loosely grouped as ``football,'' rugby developed within the elite public school system as a deliberate physical activity arbitrated by rules and regulations, before circulating through the arteries of England's colonial empire as a leisurely pastime—--a ``sport'' \citep{Dunning2005}.  In 1996, rugby became a professional sport and is played as such in Western Europe and in the Southern hemisphere. Rugby sevens---the specific focus of this dissertation---is a modified version of the conventional 15-a-side game involving teams of 7-a-side (rather than the conventional 15), and 14-minute games played in a Tournament structure over two or more days (rather than a one-off 80-minute match between two teams).  Rugby sevens has grown in popularity more recently, particularly since its introduction to the Olympics for the 2016 Games in Rio de Janeiro.  More so than the traditional version of the game, rugby sevens is played by countries all over the world, and attracts more balanced participation by men and women.

Rugby is a highly interactive and physiologically demanding sport in all forms and at all levels at which the game is currently played. Rugby requires players to participate in frequent bouts of intense activity at and above the aerobic threshold such as sprinting, physical collisions, tackles, and grappling, separated by short bouts of low intensity activity such as walking and jogging.  Rugby also requires high levels of behavioural interdependence between team members due to the uncertainty and complexity of interactive coordination tasks.  At the elite level in particular, the physiological costs and complexity of joint action requirements of rugby are amplified. From this point forward, I use ``rugby'' to refer to the version of the game that is the focus of this dissertation, i.e., rugby sevens.


\subsubsection{Joint action in rugby}

Rugby, like many equivalent team sports in which a single ball (or similar object) is contested, such as basketball, association football, and ice hockey, is made up of a series of sub-phases involving attacking and defending sub-units \citep{Passos2011}. This structure of play requires athletes of these interactional team sports to perform and continually repeat similar joint actions with teammates.   The highest order of joint action in rugby consists of 14 athletes (7 per team) who coordinate around the shared goal of completing a 14-minute game in which one team competes against the other team for victory. Lower order goal-directed joint actions are nested within this overarching frame.  Depending on which team is in possession of the ball, players coordinate their movements around shared goals of attack or defence.  Completing the goals of attack and defence usually require coordination between sub-units of 2-4 athletes per team.  The goal of attacking subunits is to penetrate the defensive line or to at least advance towards the opposition's try line by securing possession at the ``breakdown'' (the contest for possession that occurs after a ball-carrier is tackled and brought to ground) in order to score points during a later sub-phase of attack.  The goal of the defensive subunit is to halt the ball carrier and subsequently successfully contest possession of the ball at the breakdown.

% P: Cognitive threshold for maintaining social bonds
Dunbar \textcite{Dunbar1992} proposes that the ratio of human neocortex size to total brain volume imposes an upper cognitive limit on real-time coordination of behaviour of 4-5 individuals.  The group size of joint action sub-phases and sub-units in rugby sevens (2-4) falls within this upper limit, or just above the upper limit if attacking and defending subunits are grouped together (4-8). Each team of 7 is complemented by a further 5 reserves to make up a total squad of 12 who compete in a tournament setting.  In addition, the size of squads that train together outside of tournaments can range anywhere from 16 to 28.  These group sizes also exist within the cognitive limits for maintaining face-to-face intimate relationships, thought to be in the realm of 15-25 \citep{Dunbar1992,Dunbar2010}. Thus, in the case of rugby sevens, it is possible that neurocognitive mechanisms that appear to be responsible for tracking and monitoring coordination between co-actors in dyads or small groups and establishing interpersonal synergies between co actors will be particularly relevant
to rugby's specific joint action requirements, while neuropharmacological mechanisms may be more relevant to promoting more generalised social bonding to the larger group of in group members, irrespective of specific coordination dynamics.

Like many other Anglo-American interactive team sports, rugby is a competitive activity in which one team of athletes attempts to outplay another team.  The conflict inherent in this structure of activity has important implications for the level of uncertainty in the joint action environment.  Essentially, while one team of athletes is attempting to coordinate behaviours in joint action, the opposing team is attempting---at the exact same moment---to foil and disrupt this coordination.  From a cognitive perspective, the types of scenarios orchestrated in competitive team sport require coordination in joint action under conditions of extreme uncertainty.  Essentially, when taking the field in competitive joint action scenarios, athletes engage in a series of relatively low-probability predictions concerning joint action with co-participants.  In this context, it is conceivable that technical competence and increased understanding of the predictions and behavioural tendencies of co-actors would buffer against the uncertainty inherent in the interaction environment.  In this sense, while the bets that athletes make on the field are high stakes and relatively low-probability, they are, in an ideal situation, at least, educated bets, based on a level of trust in the viability of individual and joint capacity for movement execution and coordination.  Considered from a predictive coding paradigm, this type of scenario creates the conditions for the activation of heightened neurocognitive reward, particularly when the high stakes bets come off.

There is evidence to suggest that dynamic coupling occurs between dyads and sub-units of attack and defence\citep{Passos2011,Correia2014}.  Passos and colleagues \textcite{Passos2011} for example find that functional coupling tendencies emerge between attacking dyads and adapt to specificities of the task environment.  Correia and colleagues \textcite{Correia2014} show that coupling tendencies also emerge between co-actors of opposing teams in rugby union, for example, in a 1-on-1 attacker/defender sub-phase.  These results are confirmed in similar joint action contexts in other equivalent sports such as basketball and association football \citep{Duarte2013}. There is evidence to suggest that the establishment and maintenance of functional interpersonal synergies in rugby joint action depend on an athlete's perception of affordances of the task-specific cognitive system made up of constraints including other athletes, the physical environment, and the rules of the game \citep{Passos2012}.

\subsubsection{Social bonding in rugby}
There is very little direct empirical evidence specific to rugby union that can be used to substantiate a link between joint action and team click, and team click and social bonding.  Despite this, rugby is a sport heavily associated with the popular interpretation of ``social bonding,'' particularly in all-male social organisation common in the elite educational spaces of England and Commonwealth countries in which rugby originally developed \citep{Dunning2005,Richards2007,Collins2008}.\footnote{Recently, rugby union has been the site of much criticism due to the fact all-male social spaces cultivated by rugby appear to support and sustain hyper-masculine and hyper-normative behaviours, including gender-related violence \citep{Cosslett2014}.}

``Rugby is a game for barbarians played by gentlemen,'' or so the saying goes.\footnote{The origins of this oft-cited Rugby adage is unclear.  The phrase is supposedly the adopted motto of the British Barbarians Football Club, established in 1890 \citep[34]{Dunning2005}.  The complete phrase reads ``Rugby is a game for barbarians played by gentlemen, football is a game for gentlemen played by barbarians.''  However, official club history cites its original motto as, ‘Rugby Football is a game for gentlemen in all classes, but for no bad sportsman in any class' \citep[vii]{Starmer-Smith1977}.  Some sources attribute the saying to British writer and poet Oscar Wilde (1854-1900) \citep{Fleenor2015}}. Different inflections on this adage have been reproduced by people in all parts of the world that rugby has reached (including China \cite{}), presumably as a way of linking the nature of rugby's physical requirements with social virtues of fair play, cooperation, and moral integrity. For example, the current slogan of World Rugby, the international governing body of rugby union, is ``Building character since 1886'' \citep{WorldRugby2017}, referring to the moral character that can be generated through participation in rugby.

As discussed in this section, the physiological demands, joint action complexity, and social-historical trajectory of rugby suggests that it is extremely suited to an investigation of the social bonding effects of joint action in group exercise.  Rugby involves various types of complex behavioural coordination between team mates and opposition.  Joint action tasks involve sub-phases ranging from dyadic interaction, to interaction of small groups (3 - 8), to entire teams (7 - 14), and squads.  The coordinated activity at each of these scales could have important implications for social bonding.  In this dissertation I focus in particular on joint action that takes place on the playing field, involving a maximum of 14 athletes at any one time, but usually involving smaller sub-phases of attackers and defenders (2-6).  Analysis of this level of activity draws attention to the neurocognitive mechanisms and coordination dynamics associated with establishing and maintaining interpersonal coordinative relationships between co-actors.  Competitive team sports are unique in their ability to orchestrate a joint action environment of high informational uncertainty due to the unconstrained actors in the cognitive system (in the form of athletes of the opposing team).  It is conceivable that technical competence and shared understanding would buffer against high uncertainty, allowing athletes to make high stakes (low-probability) but educated predictions about the outcome of joint action.  The ecstasy of the ``click'' of joint action in team sports, therefore, could arise from successful formation of functional interpersonal synergies in environments of high uncertainty on the playing field.





  \subsection{Joint action, team click, and social bonding in China}

In addition to the parameters of joint action specific to rugby, the various cultural contexts in which rugby is played also have meaningful implications for behaviour observable in these contexts.  Sporting anecdote indicates that different teams from different places and times appear to play the same game in very different ways, often appearing to embody different ``styles'' of play \citep{Bourdieu1990,Taylor2010}.  It has been demonstrated that shared cultural knowledge can serve to structure joint action scenarios in ways that help ``smooth'' coordination by providing pre-loaded expectations between co-participants \citep{Vesper2017}.  Theory from the predictive coding paradigm suggests that team click may not necessarily be limited to coordination of the most proximate joint action parameters of a particular group exercise setting such as rugby, but could rather be contingent on a snug fit between the specific demands of joint action and a whole assemblage of hierarchically ordered expectations and affordances pertaining to personal, cultural, and ecological trajectories \citep{Clark2013}.  Thus, a careful consideration of the culturally specific affordances relevant to coordination of joint action and group membership in China is crucial to subsequent empirical analyses concerning professional rugby union in China. In the following section, I outline the cultural contours of the social cognition of joint action in contemporary China.

% Vollan2017 - cooperation under authoritarian conditions

There is very little direct empirical evidence linking movement coordination and social bonding within the Chinese cultural context. There is, however, extensive indirect evidence from historical and contemporary psychological literatures to suggest ways in which a relationship between joint action and social processes of group formation and cohesion is uniquely articulated in the Chinese context \citep{Weed2011}.  Considered together, evidence suggests that China's specific cultural trajectory contains a unique combination of affordances that resource observable social behaviour.  As such, while the component mechanisms and coordination dynamics of joint action and social cohesion that form the focus of this dissertation are thought to be generalisable across the continuum of human cultural groups, it is important to consider the culturally specific contours of these mechanisms and dynamics, and the way in which they are uniquely expressed in specific environmental niches.  In this section, I outline evidence for the way in which an ``indigenous Chinese psychology,'' a construct generated through the product of a distinct historical trajectory and facilitated by specific socio-cultural institutions---uniquely shapes social cognitive processes of action and perception, self-construal and group membership, and institutional norms \citep{Liu2009}.


\subsection{Relational and categorical modes of group membership}

A core finding from cross-cultural psychology concerns the correlation between variation in modes of social group formation and variation in cultural niches.  Anthropologists have for some time emphasised meaningful cultural variation in processes of social group formation \citep{Strodtbeck1961,Kluckhohn1961,Mead1967,Fei1992}, and more recently cultural psychologists have sought to demonstrate this variation in experimental paradigms \citep{Markus1991,Nisbett2001}.
This research has produced a theoretical spectrum of group membership processes, the two poles of which are usually described as ``categorical'' and ``relational'' \citep{Hofstede1980,Brewer2007}.  It appears that in the case of some cultural niches, traditionally samples from modern and industrialised ``Western'' societies such as the USA, for example, the dominant core of self-definition is based on individual autonomy and separation from others.  By contrast, the self-concept of relational societies, many examples of which can be identified in East Asia (Japan, China, Korea, etc.), is defined primarily based on social embeddedness and interdependence with others comprising their in-groups\citep{Leung2012}.

Much of Anglo-American social psychology of the 20th Century is rooted in a categorical mode of representing and measuring group membership (which is of course unsurprising given that this conception of social psychology is congruent with a North American cultural and historical trajectory).  The canonical self-categorisation paradigm of social psychology \citep{Turner1987}, for example, requires that an individual make an identification between abstract categories of the self and the in-group or out-group. In this ``social identification'' paradigm, group membership is achieved when the perceived differences between the self and other in-group members are smaller than the differences between in-group and out-group members \citep{Yuki2014}.  In distinction to a categorical mode of group membership, relational group membership involves attention to maintaining and harmonising intragroup relationships, rather than engaging in intergroup categorical comparisons \citep{Yuki2003}.  In a relational mode of group membership, social identity is less a function of distance between abstract categories of self and in-group, and more a degree of commitment to cultivating a network of hierarchically structured---but more or less self-centred and self-enhancing---relationships \citep{Liu2009,Nisbett2003}.

Experimental evidence suggests that categorical group processes facilitate fast and effective identification with arbitrary minimal groups \citep{Diehl1990,VanBavel2014}, the arousal of intrapersonal cognitive dissonance between the self and experimentally constructed in-group \citep{Festinger1957, Stone2001}, higher levels of cooperation with categorically similar strangers in economic games \citep{Yuki2005,Yuki2003}, and greater attention to and memory recall \citep{Buchan2006,Ng2016}.  In cultural environments where relational processes of group membership are more prominent or salient, on the other hand, the inverse is usually observed. It has been noted, for instance, that minimal group experimental paradigms have had very little (if any) success in East Asian (particularly Japanese) contexts \citep[586]{Liu2009}.  Relational group processes, on the other hand, allow for the arousal of cognitive dissonance only when it is constructed interpersonally (as opposed to intrapersonally) between an individual and specific individuals to which that individual is connected by a meaningful social relationship \citep{Hoshino-Browne2005}.  Likewise, individuals with predominantly relational group awareness are more willing to cooperate with and attend to strangers with whom they share relational rather than categorical ties \citep{Ng2016,Yuki2005}.

Both modes of group membership have been shown to shape attention, cognition, and social behaviour \citep{Nisbett2003}, and as such could have important implications for the task of identifying and measuring generalisable mechanisms relevant to the hypothesised relationship between joint action and social bonding.  Nisbett and colleagues suggest that, as a general rule, East Asians subjects tend to be holistic, ``attending to the entire field and assigning causality to it, making relatively little use of categories and formal logic, and relying on ‘dialectical’ reasoning...whereas Westerners tend to be more analytic, paying attention primarily to the object and the categories to which it belongs and using rules, including formal logic, to understand its behavior'' \citep[291]{Nisbett2001}.

The prominence of one mode of membership over another appears to be
associated with the durable persistence of cultural and linguistic institutions.  Variation is not limited to broad ethnic or cultural groups, however. Indeed, modes of group membership have been shown to vary not only across cultures (i.e. East Asian versus Western European or North American), but also within cultures \citep{Henrich2014}, within social groups (according to sex and personality differences \citep{Yuki2014}), and even within individuals (depending on contextual and situational primes, \citep{Lee2014,Wong2005}). Some researchers suggests that divergent modes of group membership may also be mediated by context-specific socio-ecological factors such as the level of relational mobility in any given environment \citep{Oishi2010,Takagishi2014,Yuki2005}.

There is preliminary evidence to suggest that East Asian subjects are under certain circumstances capable of behaving according to the tenets of a categorical mode of group membership when (experimental) conditions make such an identity adaptive \citep{Hong2000}.




\subsubsection{Tenets of an indigenous Chinese psychology}
%indigenous Chinese psychology
Understanding the cultural contours of the social cognition of joint action in China requires an engagement not only with contemporary findings from cross-cultural social psychology, but also with what social psychologist James Liu terms an ``indigenous Chinese psychology'' \citep{Liu2009}. For Liu, the construct of an Indigenous Chinese psychology is important for the purposes of deepening theorisations of observable social behaviour in China beyond the globally dominant Western modes of scientific knowledge production (including productions within human sciences such anthropology, psychology, and cognitive science).  As outlined above, a considerable amount of evidence has amassed within cross-cultural and social psychology to suggest that the existence of meaningful contrasts between samples of ``East Asian'' populations (predominantly undergraduates of Japanese, Chinese, and South Korean universities) and ``Western'' populations (predominantly undergraduates of North American and Western European universities) in domains of attention and perception \citep{Peng1997,Nisbett2003}, psychological construal of social categories of self and group \citep{Markus1991}, behavioural tendencies in social interaction \citep{Yuki2003}, and institutional norms \citep{Liu2017}.  As Liu argues, however, theoretical generalisations based on this evidence alone run the risk of being frail to the behavioural diversity observable both between the East Asian nations (Japan, China, Korea among others), and within each individual nation itself (for example, the vast internal cultural variation in China between North and South; East and West \citep[see, for example,][]{Henrich2014}).

%multi-ethnic dynamics??
Liu and colleagues thus argue that conventional social and cultural psychological approaches require bolstering through the utilisation of a ``representational'' account of social psychology \citep{Liu2005}, in which it is understood that socially shared representations of history are central to creating, maintaining, and changing psychological identity and patterns of social interaction.  This account is very much in line with the theoretical framework utilised in this dissertation, in which cultural representations are understood as causally relevant to social interaction \citep{Vesper2017}, and perhaps even processes of cultural transmission and evolution \citep{Claidiere2014}.  Although the cultural productions emanating from China's recorded history are infinitely diverse and varied, the contemporary curating of these representations in both Chinese and Western institutions of knowledge production (academia, media, and so on) generally converge on the understanding that an indigenous Chinese psychology is the product of an ongoing historical interaction between two millennia of cultural continuity associated with the ancient development of a singularly successful, multilingual Chinese civilization, and a more recent engagement with---including, importantly, perceived sufferings and failings at the hands of, and hopes of rejuvenation within---global activities of commerce, governance, knowledge production, nation-building, and international relations \citep{Liu2009,Barme2009}.  These two grand historical processes have produced a number of indigenous socio-cultural institutions that support the representational affordances and social behavioural tendencies of an indigenous Chinese psychology, most notably an ethically-prescribed Confucianism, formidable legal and bureaucratic systems of state governance, and the modern creation of Chinese nationalism, fuelled by the revolutionary spirit of both Marxist/Leninist dialectical materialism scientific and technological advancement.

\myparagraph{Confucianism}
Ethically prescribed Confucian values pervade all levels of social interaction in Contemporary China. What is understood as Confucianism in contemporary discourse emerges from deep historical roots in folk-cultural axioms \citep{Wang2009}, agricultural modes of production \citep{Talhelm2014,Fei1992}, dynastic rule, and modern reinventions of these cultural forms by processes of governance and knowledge production by the nation-state\citep{Hwang1999,Liu2014}.  Confucian philosophy provides a number of resources for directing and harmonising social interaction, all of which stem from an epistemology of holism.


``Hierarchical relationism'' (a guide for managing hierarchically organised familial and social networks, known as ``guanxi'' \textit{guanxi} 关系), holistic and dialectical reasoning (employed to harmonise social relationships and avoid interpersonal conflict, often via the ``middle way'' \textit{zhongyong zhidao} 中庸之道), and practices of personal cultivation (used as a means of accumulating ethical virtue, or ``human heartedness'' \textit{ren} 仁).   These three core tenets of Confucianism pervade all spaces of contemporary Chinese social life, and are united by the understanding that social harmony results in part from every individual knowing his or her place in the natural order, and playing his or her part well.

Holism is an epistemology that originated in Daoism before being reappropriated originally by Confucius (551–479 BCE) and then later by a lineage of Confucian scholars during the Warring States Period.  Holism involves less emphasis on Reason in the Western epistemological sense (i.e., the search for ultimate knowledge via reduction), and instead suggests that manifest and latent aspects of reality come in and out of being through an interaction between the Receptive (\textit{Yin} 阴) and Creative (\textit{Yang} 阳) principles inherent in the universe.  The interaction of these two fundamental principles (rather than the deterministic, causal principle of a single, omnipotent God), give birth to ever-changing phenomena---including humanity.  The postulate of holism means that in Confucian thinking, ``it is the dynamics among the elements, rather than the elements themselves, that serve as the primary units of analysis'' \citep[156]{Ji2010}. In the case of knowledge of the social, it is not an unchanging human biology that should be the focal point of interest, but dynamic and evolving patterns of family, groups, society, and culture.

Hierarchical relationism or guanxi builds on this dialectical reasoning by providing cues and directives for managing reciprocity and responsibility in particular kinship and extra-kin social relationships (\textit{renqing} 人情), including those involving the individual and the state \citep{Maehr1980}.  The individual stands simultaneously in several different relationships with different people: as a junior in relation to parents and elders, and as a senior in relation to younger siblings, students, and others. While juniors are considered in Confucianism to owe their seniors reverence, seniors also have duties of benevolence and concern toward juniors. The Five Relationships are: 1) ruler to ruled, 2) father to son, 3) husband to wife, 4) elder brother to younger brother, and 5) friend to friend. Four of the five key relationships concern unequal hierarchical relationships, the only exception being friendship, in which reciprocity and respect is emphasised.  Notably, none of the prescriptions concern strangers.

This prescription for social interaction amounts to a strong contrast to the practical ethics dominant in Western Europe and North America, which blends Enlightenment philosophy and Christian values to emphasise categorical (egalitarian) concerns such respect for thy neighbour and Good Samaritanism. In a morally centred Confucian worldview, deeds done in the context of a father-son relationship, as compared to that between siblings, or between ruler- ministers are interpreted through different moral lenses (Liu, 2011).  Thus, a universal morality applying to all situations and individuals is not only impossible (e.g. Kant’s categorical imperative), but also undesirable.

Beyond the specific mandates of the five relationships, the metaphor of family dominates many extra-kin social relationships between individuals, and between the individual and the state. The state is represented in Confucianism as family and other particularistic social relations writ large \citep[579]{Liu2009}.  Traditionally, the father-son kinship relationship is utilised to naturalise the socially constructed relationship between lord and minister: ``Parents naturally love their children, and children naturally love and respect their parents, and they both know that they're stuck with one another no matter what...'' \citep[178]{Slingerland2014}. Likewise, the metaphor of family is often utilised to naturalise the notion of an extra-kin social group, for example a company or a sporting team \citep{Brownell2008}.
It is important to emphasise that, whereas traditionally Western In this sense, whereas (Anglo-American) notions of group membership generally hinge upon a psychological representation of egalitarian equivalence between categories of self and group, the utilisation of the family metaphor for group processes relies on the individual identifying group membership through a participation in hierarchically structured (familial-like) \textit{relationships} \citep{Fei1992}.

%Confucian holism is carried through in the form of Marxist dialectical materialism

%Self-cultivation:
Self management and responsibility as pro-social expression.  (sweep your own path of snow before you worry about others); Confucius had deep scepticism of the ``social.''


Historical evidence suggests that the dynamics of joint action in particular have the focus of ethical considerations, namely resolution of social tensions inherent in collective action problems \citep{Slingerland2014}.

%\subsection{the social significance of joint action in China}
It is within these processes of extending familial metaphors to formal social relationships that Slingerland (2014) locates a key role for joint action. Inherent in the scaling up of familial relationships to extra-kin social relationships is the tension common to many collective action problems, that of vulnerability to free-riding and defection \citep{Cosmides2013}.
As Slingerland explains, the ``logic of civilised life makes us very keen to distinguish reliable cooperators from unreliable defectors...what we want then is a particular type of desirable...behaviour where there is absolutely no gap between action and motivation'' \citep[192]{Slingerland2014}. The paradoxical Chinese idiom of \textit{wei wuwei}, translated awkwardly into English as ``trying not to try'' or ``effortless action,'' is promoted in many ancient Chinese philosophical corpuses as a behavioural solution to conflicting incentives involved in coordination problems: effortlessness in performing socially desirable actions (virtues) communicates a level of mutual trust and commitment that is ``hard-to-fake.''  The phenomenology of effortless action is described at length in many of these corpuses, and is a key dimension of many traditional Chinese martial arts such as Taichi and Wushu \citep{Morris1998}. While mass participation in Anglo-American competitive sports is only a recent phenomenon in China \citep{Brownell2008}, the social valorisation of certain types and styles of movement, particularly the phenomenology of flow in (joint) action appears to enjoy a long history.


\myparagraph{Civil institutions}
A second, and less emphasised instrument responsible for shaping social interaction is  formidable legal and bureaucratic institutions that have served to administer state control and power over the particularistic social interactions in both dynastic and modern history \citep[]{Liu2017}.  In addition, China's interaction with the

The three layers of treatment in social Relationships (Liu)

* Hwang’s (1987) work on face and favour hypothesizes that Chinese have a precise set of rules to govern resource allocation/exchange:
    * close, affective ties are governed by a need rule,
    * mixed ties by a ‘renqing’ or human relations rule, and
    * instrumental, or distant relationships are governed by a fairness rule
* Such a set of rules for resource exchange in social relations would be highly adaptive not only for maintaining committed social relationships (what you need I will give you, and what I need you will give me), but also for using a tit-for-tat type of fairness rule to govern the expansion of the social network to accommodate new members, initially by an instrumental fairness rule, and then over time to a renqinq or human relations rule

* In practice, ideals of benevolence and morality were accompanied by the yin of a comprehensive, well-articulated, but also draconian legal system (see Chien, 1976; Fitzgerald, 1961 for overviews of Chinese history).
* Ordinary people feared magistrates and the law, regarding legal institutions more as a system of last resort than a method for ordering daily social life.
* The ethics of social relations were embedded within rule-based bureaucratic practices.

Chinese society was rare and distinctive among ancient civilizations in practicing bureaucratic meritocracy, with examinations that allowed ordinary people to become officials of state (thereby increasing the fortunes of their entire clan).

Nevertheless, meritocracy made Chinese high culture esteemed, and created a cultural psychology where education was cherished as a primary marker of Chinese identity, and a route of upward mobility


* ...Unlike practices in ancient Rome, where citizenship, as the official marker of Roman identity was a legal matter of heredity and wealth, there was no formal categorical demarcation between citizens and non-citizens in ancient China.
* Rather, there were degrees of power and practice between the high culture of the Confucian gentleman and the low culture of the peasant in the fields.
* More than this environmental support, however, there was an ‘imagined community’ (in Anderson’s, 1991, terms) of China as a singular centre of civilization that was more capable of regeneration than was Rome.

Institution as a platform for the performance of relational social processes, with limit incentives and boundaries to shape behaviour

But no inherent sanctimony around the institution itself, as a category of identity...



\myparagraph{Chinese Nationalism and categorical modes of group membership}
While a relational mode of group membership may tend to dominate cultural niches in East Asian countries, all of these nations have also, by definition of their status as nations, developed indigenous categorical modes of psychological identity through varying levels of engagement with global economic and geopolitical activities. It is thus important to avoid binary generalisations concerning vague notions of ``East'' and ``West,'' as well as one-dimensional theorisations of social behaviour based on suggestions that one mode of group membership is more culturally dominant or salient than another in a particular cultural setting.  Predictions concerning social behaviour must be generated in consultation with information regarding the inventory of cultural affordances that are part of the representational schema of the particular cultural niche in question.


Although the psychological consequences of this history are touched upon in contemporary theoretical generalisations emanating from comparative projects of cross-cultural psychology, as I discuss in further detail below, these generalisations often fall short of capturing the richness of cultural affordances that are relevant to the shaping of social interaction.  Serious engagement with the cultural specificities of contemporary China, in the form of an indigenous Chinese psychology, allows for a more accurate identification of the contours of universally generalisable mechanisms responsible for the hypothesised relationship between joint action and social cohesion.




  \subsubsection{Predictions for joint action and social bonding in a Chinese cultural context}


  Based on findings from cultural psychology, it is possible to predict that Chinese subjects, when a relational mode of group membership is made salient, dominant, or adaptive, will attend to the management of social relationships as a behavioural priority.  In the case of pro-social behaviour, for example, this may involve attempts by an individual to harmonise relationships with in-group members according to Confucian principles of hierarchical relationism (guanxi).  Beyond this level of prediction, however, there is no prescriptions within cultural psychology for how pro-social behaviour will likely manifest in a context in which a relational mode of group membership is dominant.  It is important at this juncture not to fill in this void with Western (i.e., dominantly ``categorical'') intuitions surrounding group membership, for example the assumption that harmonisation of in-group social relationships may naturally involve some form of symbolic or material sacrifice to the category of the (individually bounded) self for the symbolic or material benefit of the category of the in-group (family, organisation, team, nation).

  As I discuss in more detail in the ethnographic results chapter of this dissertations (Chapter 4), I observe various instances of what appeared to me to be self-promotional and self-enhancing (and therefore anti- rather than pro-social) behaviours from members in contexts in which relational modes of group membership were salient (see Chapter 4, section X).  It was only later through a more thorough consultation of components of an indigenous Chinese psychology, particularly the details of the merger of ethical prescriptions of Confucianism with late-industrial processes of self-distinction, that I was able to understand these behaviours as carrying a deeply pro-social signal.  It is somewhat paradoxical, when viewed from a Western standpoint, even more particularly an Australian standpoint (known for an emphasis on egalitarian modes of group membership), at least, that self-cultivation, self-enhancement, and self-promotion can be deeply pro-social expressions.  But this appears to be a viable option in the behavioural repertoire of contemporary Chinese subjects in social settings in which the categories of individual self and team are de-emphasised, and instead fused together holistically, such that self-promotion is team promotion, and vice-versa.

  Evidence that higher levels of cooperation in Chinese subjects under an authoritarian leadership model, particularly when subjects self-report belief in the efficacy of authoritative over democratic leadership models (Vollan2017)

  %The metaphor of the family is particularly salient and pervasive in modern public representations.  On almost every scale of social organisation, from dyadic extra-kin friendships through to workplace interactions and macro-social organisations (cities, provinces, nation), the family and its relational priorities are consistently invoked to aid coordination and coerce participation in collection action.

    %Joint action and social cohesion in China:
    %1) action/perception,
    %2) self-construal and social norms of group membership,
    %3) social institutions (guanxi and the state).












\section{Historical Context of rugby in China}

\subsection{Physical activity in Chinese modern history}
The history of Sport and exercise in China is in many fascinating ways emblematic of broader---often fraught and turbulent---processes associated with China's modernisation.  Beyond physical cultures indigenous to China, or practices imported much earlier in history from south Asian religious traditions (e.g. Buddhism and Hinduism), the emergence of Anglo-American interactional team sports and Northern European calisthenics in China is entangled with a history of interaction and conflict with, and embrace of foreign influence during the 19th, 20th, and 21st centuries.

  \subsubsection{Introduction, rejection, and embrace of sport and exercise in China 1842-1912}

The history of modern competitive sport in China begins with the history of trade and mercantilism between China and Western Europe.
By the beginning of the 19th century, China had established flourishing international trade relationships with Western empires.  Generally speaking, China traded in tea, silk, and porcelain, meeting high demand in emerging middle class households of newly industrialised nations of England, France, and Germany.  In return for these goods, merchants from Western empires initially paid in silver, until it became clear that a trade imbalance was emerging between China and the West, owing to the fact that tea, silk, and porcelain where much more renewable than silver \citep{Fay2005}.  Thus, by the late 19th century Western powers increasingly sought alternatives to silver, for which opium provided a much more sustainable (and addictive) solution.  By the mid-19th century, however, a series of conflicts arose between Qing dynasty rulers and the Western empires over the regulation and trade of opium, leading to the first Opium War (1939-1942).  The resulting Treaty of Nanking (1842) initiated a period of colonial occupation of China that drastically weakened the political power and legitimacy of the Qing Dynasty and increased the influence of foreign powers within China's major inland and port cities.

Increased foreign influence in China during this period brought with it the introduction of an array of ideas and practices to China's urban elite ruling classes, including novel physical cultures of dress, adornment, leisure, and physical exercise.  Increasingly popular at the time in Europe and North America was the belief that sport and exercise were important pedagogical tools in the development of physically and mentally strong subjects of post-Enlightenment modernisation \citep{Elias1986}.  This belief largely gelled with the values of a nationally (and internationally) motivated Chinese urban elite, and subsequently manifested in China in the promotion of Anglo-American competitive sports by North American Christian missionary organisations such as the YMCA (Young Men’s Christian Association), and the incorporation of calisthenics routines into military exercises \citep[240]{Morris2004}.  Such techniques were soon popularised within elite intellectual communities as pedagogical tools designed to foster an explicit link between the strength of the physical body and the strength of the Chinese nation \cites[32]{Morris2004}[49]{Brownell1995}.

The introduction of novel political ideas to China in the late 19th century involved importation \textit{en masse} of novel linguistic, cultural, and social categories and practices from the West and Meiji Japan (1868-1912) \citep{Liu1995}. \textit{Tiyu} (体育), the term in modern Chinese that most closely translates to sport, was one of many neologisms inherited from the Western social sciences via its Japanese translation.  Importantly, tiyu encompasses more than just the modern Anglo-American competitive sports (roughly translatable to ``yundong'' (运动) that the English word connotes.  Instead, the modern Chinese notion of sport refers to an entire culture and discipline of the body that is deeply intertwined with the political project of Chinese modernisation and advancement \citep{Morris2004}.\footnote{As Lydia Liu (1995: 58) points out, even the notion of ``China'' itself as a term linked to a national imaginary, only began to emerge as such through interaction with Western missionary discourses concerning ``China'' during the 2nd half of the 19th century.}

The two-character phrase \textit{tiyu} is a contraction of a longer four-character phrase \textit{shenti jiaoyu} (身体教育), a direct translation of Herbert Spencer’s notion of  ``a physical education'' that first appeared in Chinese reformist intellectual Yan Fu’s ``On Strength'' 1895 \citep[9-10]{Morris2004}.  Now naturalised within the modern Chinese vernacular, compound words such as the ``body'' (\textit{shenti} 身体) and ``education'' (\textit{jiaoyu} 教育), as well as other fundamental conceptual social categories such as ``society'' (\textit{shehui} 社会) and ``culture'' (\textit{wenhua} 文化) all first appeared in their modern form as an arsenal of translated neologisms made popular by Chinese intellectuals in the late 19th and early 20th century who were grappling with ways to transform a dynastic realm crippled by colonial occupation and feudal backwardness into a strong nation in a system of modern nations \citep{Pusey1983;Schwartz1964;Liu 1995;Huters2005}.   Convinced that the body, \textit{shenti}, was crucial to realising this transformation, intellectuals rigorously subscribed to the Spenserian ideal of a physical education, which combined with a moral and intellectual education, as a ``cultivation of the whole moral, intellectual, physical, and aesthetic self'' (\textit{dezhitimei quanmian fazhan} 德智体美全面发展) \citep[10]{Morris2004}.

%\subsubsection{The Boxer Rebellion}
Novel regimes of sport and exercise were not uniformly embraced in China during this period, however.  The most organised movement of anti-foreign, anti-colonial, and anti-Manchurian resistance during the period 1842-1912 came in the form of an army of Han Chinese subjects from rural areas of Shandong was known as the ``Militia United in Righteousness'' (\textit{yihetuan} 义和团).  The Militia trained in martial arts and meditation and explicitly rejected Western forms of physical culture such as sport and exercise \citep{Brownell2008}.  The Militia's first target in the ``Boxer Rebellion'' (1900-1901, members of the Militia were known in English as the ``Boxers,'' due to the fact that they had been practitioners of martial arts that included boxing) was the Horse Racing track in Beijing situated at the south gate of the Temple of God of Agriculture---a site that had become heavily associated with the activity of Manchurian ruling class and the influence of foreign legions.

In a somewhat harsh twist of fate for the Militia, the influence of modern sport infiltrated Beijing's sacred sites even further in 1901, when the troops of the Eight-Nation Alliance (United Kingdom, The United States of America, Germany, Japan, Russia, Italy, and Austro-Hungary) arrived in Beijing to relieve the besieged international legions.  During this period, troops from the UK occupied the Temple of Heaven to the east of Yongding gate, while US troops occupied the Temple of Agriculture to the west \citep{Brownell2008}. Compared to many other open public spaces in Beijing, the flat, open spaces of Beijing's temple grounds were conducive to the playing of Western sports common in garrison life such as field hockey and association football.  Throughout the final years of the Qing Dynasty, the sporting activities organised within the grounds of the Temple of Heaven and Temple of Agriculture attracted the participation of troops from other legions, as well as Western missionary organisations such as the YMCA, and local Chinese elites and revolutionaries \citep{Steel1985}.

\subsubsection{Republican Era (1912-1949)}
The combination of a weakened dynastic regime and the influx and development of new political and social ideas among China's urban intellectual elite led eventually to the Chinese revolution in 1911, and the establishment of the Republic of China (ROC) in 1912 \citep{Mitter2008}. Sport in China began to develop among urban elite along two main strands—--``competitive sports'' (\textit{jingji tiyu} 竞技体育) and ``games and calisthenics'' (\textit{ticao} 体操).  Traditional Chinese martial arts also made a resurgence during this period.  Although initially delegitimised within the New Culture Movement (\textit{xinwenhua yundong} 新文化运动)as elements of feudal superstition, members of the National Essence Movement (\textit{guocui yundong} 国粹运动) reappropriated martial arts by deliberately aligning these practices with rational modernist ideas about sport and the body, at a time when sport became more significant for the articulation of an emerging national identity in the face of imperial powers \citep[38]{Brownell1995}\citep[45]{Morris2004}.  A significant aspect of competitive sports was the public spectacle of the ``games meets'' (\textit{yundonghui} 运动会), in which the performance of emerging national and international political identities could take place.  As early as 1908, the Chinese sport community enshrined the Modern Olympic Games (\textit{Aolinpike yundonghui} 奥林匹克运动会) as the pinnacle of participation in an international community of nations, and as such, the quadrennial global ritual has since preoccupied a collective Chinese sporting consciousness, and a Chinese national consciousness more broadly (\citep{Burnett2009;Barme2009;Brownell2008;Morris2004;Xu2008}.

The most southern point of Beijing's sacred north-south axis continued to feature prominently in the development of sport in the ROC.  In 1914, the YMCA organised the first ever multi-sport event within the walls of the Temple of Heaven in Beijing, which the ROC nationalist government later labelled the Second National Games (the label of the First National Games of the Republican era was retrospectively assigned to the ``First National Athletic Alliance of Regional Student Teams'' multi-sport event hosted by the YMCA in Nanjing in October 1910) \citep[441]{Li2015}. The active reappropriation of the sacred spaces of the Qing Dynasty through activities such as team sport, allowed the foreign legions and local Chinese elites to perceive their occupation of these sites as a demonstration of superiority over the Qing court, and facilitated the ROC's priority of displacing traditional reverence of the throne \citep{Hevia1990}.

Beijing lost its importance as the centre of state power when the ROC moved the capital to Nanjing between 1928-1948.  In 1937, however, after the space surrounding the Temple of Agriculture had been repurposed for a range of leisure and amusement activities (including sport), a large sports stadium was constructed on the land directly to the south east of the main altar.  With a capacity of 10,000 people and a dirt association football field in the middle, the Beiping Public Stadium (as it was originally named) was the second ever modern sports stadium to be built in the ROC---the first being built two years previously in Shanghai for the hosting of the National Games in 1935.  The grounds immediately surrounding the stadium subsequently became the site for the Beijing Municipal Elite Sport Training facility, the predecessor to the current Institute.


\subsubsection{Sport in the People's Republic of China (1949-1976)}
Sport was transformed and politicised in radical ways following the establishment of the People’s Republic of China in 1949 (hereafter PRC).  As a key member in the New Culture Movement and the May Fourth Student Revolution, CCP Chairman and first President of the PRC Mao Zedong was a proponent of the Spenserian logic of physical education.  When the CCP took power in 1949, the proletarian body and the propagation of an ideology of active cultivation of the physical body was centralised in CCP propaganda \citep[58]{Brownell1995}.  The body of the worker (\textit{gongren} 工人), peasant (\textit{nongren} 农), and soldier (\textit{bingyuan} 兵员), as well as the body of the athlete (\textit{yundongyuan} 运动员), were glorified for their ``capacity for manual labour'' (\textit{laodongli} 劳动力 )---the ideological foundation for the ``proletarian revolution'' (\textit{wuchanjieji dageming} 无产阶级革命) (Ge and G. 2005: 91).  The ``emancipation ''(\textit{fanshen} 翻身) and glorification of the physical, labouring body is particularly explicit in the propaganda posters of the early Mao era \citep[87]{Ge2005} (see Figure ~\ref{fig:motherlandStrength}).

    \begin{figure}[htbp]
      \includegraphics[width = \linewidth]{images/motherlandStrength.png}
      \caption{Strengthen Physique to Defend Motherland (1950)}
      \label{fig:motherlandStrength}
    \end{figure}

Along with many other facets of society after 1949, sport was institutionalised in line with Soviet bureaucratic models of governance.  In 1952 the ``State Sports (and Physical Culture) Commission'' (\textit{guojia tiyu yundon wieyuanhui} 国家体育运动委员会) (hereafter the Sports Commission) was established, which acted as the central State organ responsible for the administration of ``sport for the masses'' (\textit{qunzhong tiyu} 群众体育), ``physical culture education'' (\textit{tiyujiaoyu} 体育教育), as well as an elite competitive sport (\textit{jingji tiyu tixi} 竞技体育).  The competitive sport system was designed with the intention of creating a fast track for the development of world class athletic talent, in lieu of a sports system as advanced as other (predominantly Western) nations, whose development pathways for athletes were more organically embedded within existing social and educational institutions \citep{Brownell2008}.  By creating model athletes capable of performing and advocating the healthy, egalitarian and militaristic body promoted by the Party, competitive sport was designed to kick start more widespread engagement in ``sport for the masses'' and ``sport education''\citep[56]{Brownell1995}.

\begin{figure}[htbp]
  \includegraphics[width = \linewidth]{images/maoXNT.jgp}
  \caption{Mao Zedong congratulating members of the St Petersburg Zenit FC following a fixture against China in 1952}
  \label{fig:maoXNT}
\end{figure}

The reinstatement of Beijing as China's capital immediately following the establishment of the People's Republic of China in 1949, had immediate implications for the Institute situated at the Temple of God of Agriculture.  The existing stadium was enlarged to a capacity of nearly 30,000, and lights were added to enable hosting training and events at night.  The stadium was host to many important sporting and political events between the years of 1949 to 1976, including a number of International football matches attended by high profile CCP members, including Mao Zedong and Zhou Enlai (see Figure ~\ref{fig:maoXNT}).

Despite becoming heavily entwined with political processes of the PRC during 1949-1976, sport and its development also became severely hampered by many factors during this period.  On the one hand, the internal political, social, and economic chaos of The Great Leap Forward (1955-58) and the Cultural Revolution (1966-77) detracted from a focus on development of sporting infrastructure and sporting development.  On the other hand, PRC's exclusion from membership in the International Olympic Committee (IOC) and thus participation in the Olympic games, limited China's ability to participate in sporting events on the International stage.






\subsubsection{Reform era tiyu (1976-2000)}
The death of Mao and the end of the Cultural Revolution in 1976 signalled the beginning of widespread social and economic transformations in China, in which the development of sport was heavily implicated.  American cultural Anthropologist Susan Brownell’s work, ``Training the Body for China'' (1995) was the first and most comprehensive attempt at an anthropology of sport in China. The research that forms the basis of Brownell's monograph was conducted during the mid 1980s at a time when China was only just beginning to interact politically and economically with an international community.

In reference to the unprecedented success of the Chinese women’s volleyball team in the 1980s, including winning China's first ever gold medal in a team event at the LA Olympics in 1984, Brownell (1995: 86) explains how elite level sport functioned as a crucial symbolic practice for China in the process of ``re-joining the world.''  As a participant in the sports system as a student-athlete herself, Brownell draws on first-hand ethnographic experience of training and existing as subject to the state-administered ``microtechniques of power'' (citing \cite{Foucault1977}) designed to cultivate athletes in post-Mao China.  Brownell interrogates the role of the athlete in the perpetuation of a hyper-visible and generalisable moral order, cast in official terms as a ``socialist spiritual civilisation'' (\textit{jingshen wenming} 精神文明) (1995: 156).  Brownell explains that the position of the athlete in reform era China was one characterised by the tensions and shifts of an ever-transforming social terrain structured by contradictory forces of the state and the emerging logic of the market.

One of the most immediate transformations to effect the Chinese sports system after the death of Mao was the restoration of the High School University Entrance Examination (\textit{Gaokao} 高考, hereafter Gaokao), following the end of the Cultural Revolution in 1976 \citep[198]{Brownell1995}.  School curricula were immediately redesigned around the Gaokao, and as a result, schools quickly reduced emphases on sport programs as they were seen to draw student’s attention and energy away from academic study.  A situation thus emerged where the only option for prospective athletes was to attend a specialist sports boarding school in which a scholastic education was not emphasised or was abandoned all together in favour of intense physical training.

China's sporting success on an international stage in the early 1980s delayed public scrutiny of this widening gap between education and sport. The PRC won a total of 32 medals at the 1984 Los Angeles Olympics---its first official appearance at the Olympics since it boycotted the games in 1952 due to a dispute with the Republic of China (now Chinese Taipei) over the use of ``China.''  Importantly, 15 of these 32 medals were gold, and this powerful display of strength on the international stage was an enormous moment for modern Chinese nationalism in the reform era \citep{Brownell2008}.  When China produced a much less impressive performance in the summer Seoul Olympics in 1988, winning only 5 gold medals (and a total of 28), latent public criticism of way in which reform era sport had become isolated from society readily surfaced and a ``crisis in Chinese sports'' was declared \citep[199]{Brownell1995}.  Amidst broader social anxieties concerning not only the alarming quantity of the Chinese population (\textit{renkou guoduo} 人口过多问题), but also the problem of population \textit{quality} (\textit{renkou suzhi} 人口素质问题), the athlete in China was problematised as lacking sufficient ``cultural quality'' (\textit{wenhua suzhi} 文化素质) in accordance with his or her elevated social status as a ``representative'' (\textit{daibiao} 代表) of the Chinese nation on an ever-expanding international stage (General Administration of Sport 2009a; Brownell 1995: 95).

In response to this public sentiment, in 1988 former army general Wu Shaozu (伍绍祖) was appointed head of national sports commission and tasked with implementing reform measures that would help the ``societisation'' of the Chinese sport system.  In 1989 the Sports Commission adopted a policy modelled on the US college sports system, of ``combining sport and education'' (\textit{tijiao jiehe} 体教结合).  In an attempt to move away from a reliance on sport boarding schools and full-time sports training centres for the development of athletic talent, ``high level tiyu programs'' (\textit{gaoji tiyu xiangmu} 高级体育项目) were embedded within existing stand alone high schools and universities so as to ensure the ``all-round development''(\textit{quanmian fazhan} 全面发展) of the athlete \citep[203]{Brownell1995}.  As part of an emphasis on a broader range of sports and their perceived potential to facilitate community engagement, international relations, as well as commercial opportunities, various sports programs, including many non-Olympic sports such as rugby, were inducted into the Chinese sports system for the first time\citep[70]{Knuttgen1990}.  Above all, the  democratisation of sports programs to include non-Olympic sports was driven by a persistent faith---built-in to the logic of `\textit{tiyu} form its inception in China---in the ability of sport to produce citizens of a certain \textit{quality} \citep[7]{Woronov2003}.

Reform measures in sport continued into the mid 1990s before being interrupted by China's first chance at hosting the Olympics.  In an attempt to reduce the monopolisation of power and resources in the sports system, in 1993 Wu Shaozu broke up the six major sporting bodies of the Sports Commission into 23 sports management centres, with the ultimate goal of placing every sport under the management of an independent sporting association.  In 1994, the first professional Chinese Football League was established, followed soon after by the professional Chinese Basketball League in 1995.  In 1998, the Sports Commission rebranded as the General Administration of Sport (hereafter GAS) to accord with this direction of institutional reform.  But, one year earlier in 1997, it was decided (by the powers above in the CCP) that Beijing would bid for the 2008 Olympics, and as such a subtle shift in focus occurred in Chinese sport that altered the course of reform.


\subsubsection{Beijing Olympics and the lost decade of sport reform (2002-2012)}

Many political and social commentators within China refer to the period under the leadership of President Hu Jintao and Premier Wen Jiabao as the ``lost decade'' (\textit{shiqu de shinian} 失去的十年) in the PRC's modern development (SOURCE). The main criticism by commentators is that despite China's economic rise, social and political reform during this period stagnated in comparison, leading to a deepening of problems ranging from corruption to degradation of the natural environment, to structural imbalances in Chinese economic and financial systems \citep{Barme2014}.  Many in the Chinese sport community adopt the same stance in regards to the Chinese sports industry \citep{News2017}.

Indeed, once the bid for the Olympics was announced as successful in 2001, sport reform ground to a halt as priority shifted to winning as many gold medals as possible \citep{News2017}.  Wu Shaozu left GAS in 2000, and his two successors Yuan Weimin (袁伟民, 2000-2004) Liu Peng (刘鹏, 2004-2016) did not actively return to the project of reform, continuing to invest in Olympic performance.  Even though many sports had since established independent associations, these associations had to be directly affiliated with one of the 23 GAS has 23 sport management centres. Rugby, for example, was affiliated with the ``Management Centre for Small Ball Sports'' (\textit{xiaoqiu guanli zhongxin} 小球管理中心),which was also home to sports such as Golf and Ten-pin Bowling.  The structure of the sports system exists today largely unchanged, although its name changed in 1998 to ``The General Administration of Sport in China'' (\textit{Guojia tiyu zongju} 国家体育总局, hereafter GAS).

The combination of high profile scandals and corruption (match fixing, doping reports, etc.) and the persistence of China's narrow performance-focussed sports system produced palpable public discontent during the lost decade of Chinese sport. ``Low investment in public team sport, and deteriorating public health, contributed to lack of public participation, and the industry’s low value'' (Wang Qi). Sports management centres swelled into sovereign entities managing large accounts and thus became vulnerable to corruption and graft and served to hinder reform despite their original design as reform facilitators. The CBA and CFA suffered year after year of losses, and sports such as table tennis and volleyball struggled to secure a sturdy commercial foundation. Critical murmurings remained sufficiently muffled in public discourse, however, by the strong performances of Chinese Olympic athlete delegations in Sydney 2000 (28 gold, 58 total) and Athens 2004 (32 gold, 63 total). The choreography of Chinese sporting might on the world stage reached its pinnacle when Beijing hosted the 2008 Olympics, with the Chinese athlete delegation winning 48 gold medals in a total haul of 100 medals.  It appears that the highly visible success on the world stage distracted from much needed reforms and further institutionalised an incentive structure revolving around success in Olympic sports.

\subsubsection{the impact of competitive Olympic logic on team sports}







\subsubsection{Xi Jinping Era Sport reform and Beijing 2022 Winter Olympics (2013 - present)}

Unlike his predecessors in the post-Mao era of the PRC, Chinese President Xi Jinping wasted no time in signaling strong intentions for major political reform as part of his tenure when he came to power at the end of 2013, and this assertiveness had implications for sport.  In the second half of 2014, sport was earmarked to become a ``major pillar'' of domestic economic consumption. In 2015, the GAS released a number of policies charting a course for the transformation of the sports industry from an industry dominated by manufacturing (55\%) to a services dominated industry with a total scale of USD 800 billion by 2025 (for context, the total scale of the comparatively mature US sport industry was USD 400 billion in 2016).  In 2016, CCP Standing Committee Member, Deputy Party Secretary and former Vice Major of Beijing Gou Zhongwen was appointed Director of GAS.  Gou's appointment appeared to be a surprise move, given that Gou had no previous experience in sport. Gou is however reported to be one of Xi's trusted CCP colleagues and therefore his appointment signalled a seriousness in relation to finally executing long-awaited structural reform in Chinese sport.  Signals of this reform process include the appointment of Chinese Basketball deity Yao Ming as director of the Chinese Basketball Association (the first time a non-Bureaucrat athlete had been appointed to a top sports administration position), the removal of coaching cliques (such as the removal of Liu Guoliang as head of table tennis), and the establishment of new high performance model called ``Team China'' (\textit{beizhanban} 备战办) to kick-start elite performance ahead of the Tokyo 2020 Olympics and the Beijing Winter Olympics in 2022.

%Late 2016:  GZW blindsided sports centre managements, calling them monopolies and saying too much power rests in the hands of their chiefs.




\section{The structure, regulations, and incentives of the Chinese Sports System}

As explained above, the Chinese Sports system has reformed gradually since it was erected at the beginning of the PRC in the 1950s. The organisational structure, and the ways in which its athletes, coaches, officials, and administrators are incentivised, however, have not changed significantly since the their inception.

\subsubsection{Organisational structure of the Sports System}
At the bottom of the hierarchical structure of the Sports Commission are local sports commissions (county, township and city), above which are the provincial and municipal sports commissions; and at the top is the National Sports Commission, located in Beijing \citep[59]{Brownell1995}.  The Sports Commission was responsible for all sports training centres and sports programs, of which there were many types.  On one extreme, the elite professional arm of the Sports Commission , a ``national (sport) system'' (\textit{juguo tizhi} 举国体制), which presides over all full-time professional sports teams (\textit{tigongdui} 体工队) that exist at national and provincial/municipal levels.  The main objective of this national system is to cultivate elite athletes to compete on a national and international level, in events such as the National Games (\textit{Quanguo yundonghui} 全国运动会), the Asian Games (\textit{Yazhou yundonghui} 亚洲运动会), and most importantly, the Olympic games (\textit{Aolinpike yundonghui} 奥林匹克运动会).  Due to an overwhelming Olympic-focus, all sports under the umbrella of professional arm of the Sports Commission are either Olympic sports, or Chinese martial arts (Guojia tiyu zongju 2009a).  Outside of this professional arm, elite sport programs are embedded within secondary and tertiary education institutions in a number of different ways, under the banner of the ``high school and university sport system'' (\textit{gaoxiaotizhi} 高校体制) (Guojia tiyu zongju 2009a).  At a high school level, elite sports programs are offered at ``extracurricular sports schools'' (\textit{yeyu tixiao} 业余体校), as well as regular high schools that focus on one or two sport programs in particular \citep[59]{Brownell1995}. At a tertiary level, a number of specialist sport colleges operate at national, provincial/municipal and local levels.


\subsubsection{Athlete Incentives}

Despite proletarian valorisation in the Mao era,

A Han majority saw sport as a hard career
Giving up your ``youth spring'' to be an athlete (women, Brownell).  Train outside in the hot sun, threaten femininity and fertility, marriage prospects, etc.

While there was some celebration of the nation's top athletes, it was well understood that an athlete's career inevitably involved considerable sacrifice, and that ``the love of the sport'' wasn't strong enough to attract

For many in rural areas, sport is an opportunity for social mobility; as MHT told me in an interview, for a young boy like me in my village in Shandong, it was either military or sport.  These were the ways to get an education, the only way to go up.

Pursuing sport in China has in the past, and still does today, offer two main life-course opportunities: education and employment.  With employment also came residency: the ability live in China's capital city---access to the country's best education, healthcare, and employment opportunities.

Education is for reputation and network; employment is for stable workplace (Danwei), from which other business can flow flexibly.












\subparagraph{National Athlete Technical Standards}

Hundreds of thousands of athletes, coaches, officials, and administrators are incentivised towards

 based at a number of institutions, including provincial and city level professional sports institutes (like the Temple of the God of Agriculture Institute in Beijing), universities, and high schools.  This enormous system is regulated by a series of national standards for athletes, coaches, and officials, set out by the National governing body of each sport.  Athletes



 Institutions adhere to these standards as admission criteria.  As of 2010,



Level 2 sports:
Athletics, Football, Basketball, Volleyball, Table Tennis, Chinese Martial Arts, Swimming,

Level 1 athletes can attend XNT

once they had attained the official athletic standard of a ``Master Sportsperson'' (\textit{yundong jianjiang} 运动健将), which could be done by representing Beijing at a national level tournament (竞体司2014, the full system of athlete standards will be explained fully below HYPERLINK).






\subsubsection{Conclusion: the logic of the Chinese Sports System}

Sport in China began as a deliberate social project driven by revolutionary goals of modernisation against the backdrop of what was perceived to be a corrupted dynastic rule.  With the establishment of the PRC, sport became a central ideological piece in China's participation with the world, heavily geared around participation in international multisport events such as the Olympics and the Asian Games. This gearing has ultimately had implications for the ability of sport in China to move beyond a fixation on Olympic performance, as is demonstrated by the problematisation of the ``lost decade'' of Chinese sport and the continual challenge of developing grass-roots participation.  Despite aggressive reforms in recent years, the challenge of reform continues, and it has become clear that the founding logic of the sports system runs deep.










  \subsection{The history of rugby union in China}



    \subsection{Rugby Union in China (1992-2009)}
Although reportedly existing in China within colonial and expatriate circles for more than a century (Reason and Carwyn 1979: 210; HKRFU 2009), and as a modified military exercise as early as the 1930s \citep[135]{Morris2004}, rugby was a late entrant into the Chinese sport system, established as a ``high level sport program'' in 1990.  The advent of rugby in China can thus be understood in terms of the processes of sport ``democratisation'' explained above.  The introduction of rugby into China was initiated originally at CAU by professor Shi Zhengsheng (施振声) who was introduced to rugby by his
supervising professor while completing vocational studies at Azabu University in Japan 1987-1989.  Following Shi’s return to CAU, an exchange relationship was set up between the two universities, and throughout 1990, coaches and referees from Azabu University came to CAU to help set up the necessary infrastructure required for a rugby program.  On the 12th December 1990, China’s first rugby union team was created.  The program was originally made up of existing CAU students who expressed interest in the novel activity, but by its second year, the program earned status as a High Level Sport Program and was subsequently advertised to student-athletes across the country \citep[2]{Xu2010}.  Between 1990 and 2009, rugby programs based on this original CAU model were established within over 30 regular universities and specialist sports colleges in cities throughout China.  There are also a number of social rugby clubs (\textit{julebu} 俱乐部), organisations completely independent of the state sports system, in major cities with high expat populations (Shanghai, Beijing, Chengdu, Qingdao).

The Chinese Rugby Football Association (CRFA) was established in 1997. Both the men and women’s national teams, made up of players predominantly from CAU, but also from other well-established programs based at the Shanghai Sports University, Shenyang Sports College, and the People’s Liberation Army Sports College. China consistently competes against other nations in the Asia Pacific region (most notably in the Asian Games and the East Asian Games), and is also occasionally involved in top-tier international tournaments such as the International Hong Kong Sevens.

Before rugby became a professional sport in 2010, it had existed as a non-professional university sport for 20 years.  First established in 1990 at CAU, Beijing, rugby was and still is part of a large collection of ``cold-gate'' sports (\textit{lengmen xiangmu}, a term that refers to a profession, trade or branch of learning that receives little attention) in China, with a relatively small participation base compared to other interactive team sports like basketball or football.  While football and basketball have matured as standalone enterprises with supporting market-based consumer industries, most other sports in China (i.e., all other Olympic events, including rugby) exist primarily due to the support of the enormous state-sponsored sport system.  Whereas the commercial basketball and football industries might offer a small percentage of prospective athletes incentives of fame and fortune, the benefits of a state-sponsored sports programs like rugby are more modest.  Chinese youth either gravitate or are ushered by their parents towards sporting careers primarily due to potential life-course opportunities such as access to tertiary education and post-athletic career employment (in the government sports system).  The extent to which an athlete is able to maximise these potential benefits depends on the strength of an athlete's results (\textit{chengji}).

However, due to the persistent Olympic focus of the Soviet-modelled Chinese competitive professional sports system (\textit{juguo tizhi}), rugby's recently acquired Olympic status means that it is one of 33 sports featured in the all important quadrennial National Games.  Ten of China's collection of 32 provinces and municipalities that participate in the National Games have full time men's and women's rugby programs.  In Olympic sports, the most important measure of a Province's success is its results in the National Games (\textit{quanguo yundonghui} 全国运动会), a quadrennial multi-sport event hosted on rotation by provincial capital cities \citep{Hong2002}.  The amount of funding a province and its provincial sporting institutes and programs receive is decided to a large extent by results at the national games.


\subsection{Rugby in China 2010 - 2013}
Olympic status transformed rugby almost overnight from its former position as an amateur sport played at university level by a handful of universities.  In 2010, rugby (in its seven-a-side version of rugby sevens) was inducted into the provincial level competitive sports system, which allowed provinces to set up professional rugby programs at provincial sports institutes, to compete at the National Games in 2013.  With professional provincial sports programs in place as an apex, city level rugby programs at city sport institutes, catering for high school aged athletes (10 - 16 years), could follow.  In this way, a previously non-existent development pathway for athletes, coaches, and officials began to emerge in provinces interested in investing in the sport.

When rugby union was officially inducted into the state sponsored sports system in 2010, a total of five full time Men's (Beijing, Shandong, People's Liberation Army (PLA), Liaoning, and Shanghai) and six Women's (Beijing, Shandong, Anhui, Liaoning, Shanghai, and Jiangsu) provincial programs were established, signalling a intention to invest in the sport for the long term.  Full time programs could draw fully on the institutional resources of their respective provinces to offer athletes a range of attractive life-course opportunities relating to education, employment, and permanent residency.  In the case of the Men's and Women's programs based at the Temple of the God of Agriculture Sports Institute, for example, athletes were attracted by the offer of the much sought after Beijing permanent residency and the opportunity to attend a well-renowned Beijing university, and the chance to remain at the Institute as an employee after their career as an athlete.  All of these opportunities were of course conditional on various hurdles of measurable performance.  Resources of provincial sport institutes also included athletes from adjacent programs such as athletics or association football.  Full time rugby programs soon began to attract transition athletes from these sports, which were often overcrowded due to their traditional popularity.  In addition to these full time provincial programs, three part-time Men's (Inner Mongolia, Heilongjiang, and Xinjiang) and two part-time Women's (Sichuan and Xingjiang) programs were established, in which these provinces temporarily employed rugby athletes from university programs. In addition, Hong Kong fielded both a Men's and a Women's side, bringing the total of Men's and Women's teams eligible to compete in the National Games to nine and 10, respectively.

A few provinces in particular identified an opportunity to achieve a beneficial result at the National Games by heavily investing in this debutant sport.  The Beijing Men's and Women's programs (based at the Temple of the God of Agriculture Sports Institute) managed to attract a large amount of China's existing rugby talent from where it was previously based at the Chinese Agricultural University, Beijing.  Importantly, among Beijing's recruits was the unofficially touted ``Boss''  (\textit{Laoda}) of Chinese rugby, Chinese national coach Zheng Hongjun.  Meanwhile, Shandong province, a powerhouse in other provincial sports, succeeded in attracting the majority of the remaining rugby talent. The pull to Shandong was strong for a large majority of rugby players in China at the time, many of which were originally from Shandong.\footnote{(Indeed, a large proportion of athletes more generally are from Shandong, see Taylor2010.}  Importantly, the talent transferred to Shandong province also included coaching staff, namely Zheng Hongjun's student and soon to be rival, former Chinese Women's Team coach, Lu Xiaohui.  Besides Beijing and Shandong, Jiangsu and Anhui province were strong contenders for the Women's gold medal, while the People's Liberation Army (PLA) and Hong Kong in particular were strong contenders for top spot in the men's competition.



\subsubsection{The National Games 2013}
Beijing's results in the two years leading into the 2013 national games were strongest overall across the men's and women's teams, however the traditionally strong Hong Kong men's and women's teams had only occasionally participated in these tournaments due to conflicting international tournaments.  In the semi-finals of the National Games, held in Shenyang at the beginning of September, the Beijing men came up against Hong Kong, while the Shandong men played off against the PLA.  Beijing lost to their stronger and more favoured opponents, and Shandong beat the PLA.  Meanwhile in the women's tournament, both Beijing and Shandong advanced to the final without faltering.  The stage was set: the traditional favourites, Beijing, led by the reigning Boss of Chinese rugby, would face Shandong, the underdogs lead by the Boss's cunning apprentice come challenger.

The men's final was played first, and in somewhat of an upset, Shandong edged out Hong Kong to win the gold medal by one try.  In the women's final, scores were level until early in the 2nd half when Shandong went ahead by two tries to nil.  At that point, the Beijing women's team, under instruction from their coach Zheng Hongjun, suddenly stopped playing.  After being asked by the referee and match officials to continue, the Beijing women stood firm and refused to play on, forming a huddle on their side of half-way in the middle of the field. Shandong had no choice but to continue to play out the rest of the 2nd half, running in try after try, until the final score at full time was a farcical 71-0.  Shandong was declared victorious, while Beijing called foul play, claiming that the Spanish referee had been unfairly adjudicating the match in Shandong's favour.  The details and dramas of this now well-known story in China's sporting history (known as ``Match-striking-gate'') require more detailed development in a format that exists beyond the scope of this particular dissertation. Suffice to say, the repercussions of this incident for the Beijing provincial rugby program were extremely costly.




\subsection{Temple of the God of Agriculture Sports Institute}
The rugby match-striking-gate of 2013 led to a sudden fall from grace for the Beijing rugby programs.  Between 2010 and 2013, the rugby programs at the Institute were immediately elevated to top-priority status, receiving unrivalled institutional and financial support in the hope that both teams would be crowned National champions---what would have been the Institute's first National Games gold medals since 2004 (SOURCE).  During this period, the rugby program attracted a high profile sponsorship deal from Beijing Steel, which enabled the Institute to invest in a team of foreign coaches from New Zealand to come to Beijing on a periodic basis to consult on training and preparation, and both teams travelled twice to New Zealand for two three-month stints of off-season training and competitions.  Between 2010-2013, the rugby team lived in the Institute's best available accommodation, and ate at the Institute's highest level canteen, reserved for National-level champions.  Up until 2013, the men's and women's teams had met the high expectations placed on them, winning all but one of seven national tournaments each. All indications were positive for Beijing to take home two gold medals. As explained above, however, the National Games in Shenyang in September 2013 did not transpire as Beijing would have hoped.

In the end, Beijing came away with one bronze medal (men's team) and one face-destroying disqualification for the women.  The assistant coach of the Beijing men's team, Shi Yan, told me quietly one evening that the Beijing women's rugby team was the first Beijing team in the 48-year history of the National Games not to receive the ``medal for civilised spirit''  (awarded by the Beijing Mayor to all Beijing representatives in the National Games) (SOURCE).  All rugby coaches and many senior athletes of the 2013 National Games campaign have since left the Temple of Agriculture, either retiring or moving to other provinces.  The rugby program was all but abandoned at the end of 2013, with athletes from both teams being told to take a break for an undetermined length of time.  It wasn't until April 2014 that the men's program was resurrected with the appointment of a new head coach.  It was in this context that I entered the Institute and began ethnographic research.



\section{Conclusion}

                                                          \end{CJK}
