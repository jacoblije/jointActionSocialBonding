
%\begin{savequote}[8cm]
%By long years of military experience he knew, and with the wisdom of age understood, that it is impossible for one man to direct hundreds of thousands of others struggling with death, and he knew that the result of a battle is decided not by the orders of a commander-in-chief, nor the place where the troops are stationed, nor by the number of canons or of slaughtered men, but by that intangible force called the spirit of the army, and he watched this force and guided it in as far as that was in his power\dots
%  \qauthor{--- Leo Tolstoy \cite{Tolstoy1956}}
%\end{savequote}

\chapter{\label{generalDiscussion}General Discussion}


\section{Overview}

 - what this thesis was all about and set out to do, and
 - a brief summary of what each chapter did (ie. what it was dealing with)
  ...this is basically the 'abstract' for the final Discussion chapter, and should be <1 page...dont talk about results just the overall remit of the whole thesis, though you can refer back to the theory as outline in Chapter xyz)

  \subsection{Summary of findings}
    Moving through chapter by chapter (of those that were not theoretical), summarise what your overall findings were for each...
      -1 paragraph per empirical chapter.
        -You could do this for the ethnographic section too...
        -to write this, you can basically paraphrase/copy things you have said in each chapter's abstract.
        - For each paragraph, as you move through chapter by chapter, include a page reference to where that chapter begins...this way the reader can easily flip back to the chapter to re-familiarise themselves with any specific elements

  \subsection{Contributions to the literature}
Contributions to the literature (arrange this section around the core questions that your thesis centred around...so this section is not about chapter by chapter results, but cutting across the whole thesis,
    -what were the overall contributions to the literature that your thesis hereby provides?
    - Mention which chapters' results fed into each contribution.
    - This section is a genuine synthesis, across the chapters, of all your findings...it is very important for the examiner to see that you can zoom out and name and summarise the 'golden threads' of the thesis contributions.
    - These will also be mentioned in your Chapter 1 of thesis, preemptively, and you can signpost your reader to this last chapter's section for them to read about them in more detail.
    - Each contribution can be maybe 1-2 pages long, but it depends really on how you are partitioning the material.

\section{Limitations and directions for future research}


(brief paragraph saying that you will explore some important considerations regarding the interpretation of your results, limitations of your studies (if you can identify a series of main limitations, list them here, then go into them one by one in the next section) and avenues for future research.

\subsection{Limitations}
2.1 Limitations of the studies

\subsection{Generalisability of results and outstanding research questions}
 - Generalisability is to do with, for example, to what extent your findings may be specific to China, or more generalisable to other cultural contexts etc.
 - Outstanding research questions is to do with what you think future research should focus on...
 - you could split these two sections up, if you see fit, and change the order in which you include them, up to you).

\section{General implications of findings}
(what does this all have to do with bigger picture (e.g. evolutionary) theories about bonding etc... basically this section, like in your Intro chapter, should be a massive zooming out to bigger picture stuff..1-2 pages)

\section{Final conclusion}
 (like...1 page...the final word!!)
