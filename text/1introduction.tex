
\begin{savequote}[8cm]

    We hear a lot these days about genes and molecules, but how does one human brain, or one human being coordinate, or entrain, or resonate with another?  We may not realise it but we live in a world of coordination, at every level and every scale of endeavour.

  \qauthor{--- J. A. S. Kelso  \textit{Coordination and the Complimentary Nature} Presentation to the The New York Academy of Sciences - May 12, 2010}
\end{savequote}

\chapter{\label{introduction}Introduction}



\minitoc

Introduction

Sun Hongwei arrived, escorted by his high school athletics coach, to the Beijing Agricultural Temple Institute of Sport (hereafter the Institute) soon after I began my fieldwork in August 2015.  An 18-year-old with a slight build and timid demeanour---his gaze remained diverted to the ground during his first few months at the Institute---Hongwei had never seen a rugby ball before that day he arrived.

Hongwei was from Hebei province, immediately surrounding the special prefecture of Beijing, China's capital.  As is relatively common practice in professional sport programs such as this one at the Institute, Hongwei's coach had organised a trial for Hongwei with the Beijing Provincial Men’s and Women's Rugby Program (hereafter the Rugby Program) by calling upon social connections to the leadership of the Institute.

Athletes come to the Rugby Program from all over the country.  Representing Beijing at a provincial level in a sport like rugby can translate into the opportunity to gain entrance---via a designated ``specialist athlete'' (\textit{tiyu techan} 体育特产) pathway---to one of China's top universities (Beijing Sports University, in this case) and enhanced career employment opportunities thereafter.  Rugby is not a popular sport in China. It is referred to as a neglected branch of the Chinese sport system, or a ``cold-gate'' (\textit{lengmen xiangmu} 冷门项目, referring to a profession, trade or branch of learning that receives little attention). Despite its minnow status in China in terms of its popularity, rugby's recent inclusion in the Olympics (in the form of the modified seven-a-side version of ``rugby sevens'') means that it now occupies a prominent place in the Chinese sport system, which has been defined since its inception by a strong Olympic logic \citep{Brownell2008}.  If a sport is a current member of the summer or winter Olympic Games, then the sport is included in the roster of sports played at the quadrennial China National Games (\textit{quanhunhui}). As such, rugby programs such as the one at the Institute now exist in twelve of China's 34 provincial level regions, either embedded within or somehow associated with tertiary education institutions. Thus, although rugby and China are two words that have not historically featured within the same utterance, rugby in China now affords athletes a rare and under-capitalised opportunity to pursue attractive life-course opportunities of education and employment in an intensely competitive sport system.

Almost without exception, the athletes who arrive at the Institute to join the rugby team have not spent their childhoods playing rugby in their schoolyards or watching professional rugby on television. Many who come to rugby transition from other more popular sports such as athletics, basketball, or association football, and often---like Hongwei---have never seen a rugby ball before they arrive.  Most ``start from scratch,'' so to speak, in terms of their grasp of the requirements of the highly interactive and technically complex team sport. In addition to complex patterns of movement coordination, rugby also involves unrestrained body-on-body collisions and intense bouts of high physiological exertion, requiring speed, strength, agility, and endurance to perform all of rugby's technical requirements successfully.  Learning the game of rugby from a baseline of essentially zero, while also navigating the inevitably demanding  social and political dynamics within the team and the Institute, was clearly going to be a daunting task for Hongwei.
\begin{center}
  * * *
\end{center}

\section{Thesis overview}

Competitive team sports, whirling Sufi dervishes, late-night electronic music raves, Maasi ceremonial dances, or the fitness cults of Cross-fit and Soul Cycle---endless examples can be plucked from across cultures and throughout time to exemplify the human compulsion to come together and move together.  How can we explain the prevalence of physiologically exertive and socially coordinated movement in the human record?

%Discussion:
A scientific answer to this type of research question conventionally involves a partition between two complimentary explanatory accounts.  In this case, the phenomenon of physiologically exertive and socially coordinated movement (hereafter simply ``group exercise'') can be explained in terms of 1) an account of its immediate physiological, cognitive, and social mechanisms (proximate explanation) and 2) an account of its evolutionary history (ultimate explanation) has proven extremely effective in generating testable theories of biological life, from bacteria to brain cells.
%Discussion:
As I demonstrate in this dissertation, however, there is an unmistakably ``embodied'' dimension to group exercise for which conventional evolutionary approaches have not yet taken into account. In particular, assumptions and conventions associated with existing paradigms of evolutionary theory serve to occlude dynamical properties of human movement observable in many group exercise contexts, thus restricting the capacity of such properties to contribute to a causal account of the ubiquity of group exercise in the human evolutionary trajectory.  Indeed, diverse evidence from the biological and human sciences corroborates the claim that spontaneous, non-linear dynamics of physical movement must be more precisely quantified and theoretically integrated into existing explanations of biological life---bacteria, brain cells, and human behaviour included \citep{Nowak2006,Kelso2016,Laland2011,Laland2015,Yufik2017a,Nowak2017}.

\subsection{Team Click and the social cognition of joint action}
In this dissertation, I address the under-theorisation of dynamical properties of movement in group exercise contexts through a close study of professional Chinese rugby players.  I draw upon emerging evidence from the social cognition of joint action, to analyse the psychological phenomenon of what I term ``team click''---a subjective perception of the tacit quality of coordination in joint action among athletes.  Similar in many respects to psychological states associated with ``flow'' and peak performance \citep{Csikszentmihalyi1992}, team click specifically delineates perceptions of joint action from individual action, and therefore implicates physiological, cognitive, and social mechanisms unique to joint action, as well as nonlinear systems dynamics associated with participating in a socially-coordinated, multi-agent system of physical movement \citep{Kelso2009}. As I will discuss in Chapter 2, team click is anecdotally present in a wide range of joint action contexts, and is often associated in these contexts with psychological processes of positive affect and wellbeing, as well as personal agency, social affiliation, and group membership \citep{Jackson1995,Marsh2009,Wheatley2012,Slingerland2014}.

While the contextual antecedents\citep{Fong2015}, neuropharmacological \citep{Boecker2008} and neurocognitive mechanisms \citep{Dietrich2004,Dietrich2011,Cheron2016}, and psychological consequences \citep{Wheatley2012} of flow and related states have been well researched, very few direct attempts have been made to incorporate non-linear systems dynamics of joint action into these accounts \citep[but see][]{Marsh2009}.  Further still, very few researchers have attempted to situate flow and related group-level psychological phenomena within a broader evolutionary framework \citep[but, for a general theoretical proposal, see][]{Slingerland2014}).  Reasons for these knowledge gaps within cognitive and evolutionary approaches to flow and joint action can be explained in part by the theoretical occlusion of dynamical properties associated with established theoretical paradigms, and in part due to methodological difficulty in quantifying non-linear dynamics of human movement \citep{Kelso2009}.

Recent advances in neuroimaging technologies \citep{Frith2007}, emerging neurocomputational theories of brain function \citep{Friston2010,Frith2010,Clark2013}, and constructive attempts to extend the theoretical paradigm of human social cognition to account for inter-individual processes of interaction and coordination \citep{Sebanz2006,Dale2014}, now provide an opportunity to address these knowledge gaps within the research domain of the social cognition of joint action.  Owing to innovative research within this domain, it is now more clearly understood that basal human capacities for physical movement regulation and coordination set the foundation for social cognitive systems whose resources are distributed between brains, bodies, and physical features of task-specific environments \citep{Hutchins2000,Kirsh2006,Semin2008,Semin2012,Coey2012}.
Furthermore, it has been shown that the quality of coordinated movement within these cognitive systems has implications for psychophysiological health and subjective well-being \citep{Wheatley2012}, and is relevant to the effective function of more complex goal oriented social activities, including the large-scale reproduction and transmission of shared cultural practices \citep{Dunbar2012,Roepstorff2010,Claidiere2014,Launay2016}. Thus, there is evidence to suggest that the ``embodied'' dimension of group exercise, traditionally occluded by theoretical convention and methodological challenges, can now be accessed by  interrogating the ways in which cognitive mechanisms and system dynamics of joint action are responsible for processes of social cohesion.

The core prediction of this dissertation is thus that the phenomenon of team click mediates a relationship between joint action and processes of social cohesion. I present ethnographic and field-experimental evidence that tests and confirms this core prediction, and I evaluate these results in terms of their implications for understanding the proximate cognitive mechanisms, ecological system dynamics, and ultimate evolutionary processes relevant to the anthropology of group exercise.

In the following sections of this introduction, I will 1) review the adequacy of existing cognitive and evolutionary theories of human behaviour for explaining the ubiquity of group exercise in human sociality, 2) introduce the research domain of the social cognition of joint action, and draw attention to ways in which a more sophisticated theoretical appreciation of coordination dynamics of human movement could prove useful in formulating testable research hypotheses, and 3) formulate the specific predictions of my thesis and introduce the empirical research designed to test them.  This chapter concludes with an outline of the empirical contributions of this dissertation and suggestions of the theoretical and methodological considerations for the cognitive and evolutionary anthropology of group exercise, which are developed fully in the General Discussion (HYPERLINK Chapter 8).

Before continuing, I return first to a series of ethnographic vignettes concerning rugby's newest recruit, Hongwei. The vignettes below are designed to highlight central focus of this dissertation, namely the relationship between mechanisms of joint action and processes of social cohesion, as well as the power of ethnographic observation to shed light on dimensions of this relationship that are otherwise less accessible.  Hongwei's story is emblematic of various other athletes who I followed closely between August 2015 and September 2017, and it captures covariation among key variables of interest in this dissertation, specifically perceptions of joint action, feelings of team click, and attitudes towards group membership.

\section{Sun Hongwei, rugby's newest recruit}

\subsection{Vignette 1}
Hongwei was the first of the Program’s new recruits that I followed closely. Even compared to other newly arrived junior athletes, he was noticeably timid and shy, particularly in his interactions with the coaches (myself included) and senior players. Nevertheless, Hongwei clearly signalled diligence and commitment through his participation in team activities, arriving early to each training session, and carrying more than his fair share of the training equipment---a task shared by the most junior members of the team.  Each time I passed Hongwei in the corridors of the Institute he would greet me with a polite bow and greeting, ``Hello Coach'' (\textit{jiaolian hao}).  In these instances, Hongwei would coordinate his greeting with a moment's eye contact, only to return his gaze to the floor and continue walking.

Due to his initial lack of a grasp of the basic techniques of rugby, Hongwei was unable to properly participate in normal training with the rest of the team. Instead, during the first month or so, Hongwei stood on the sidelines and practiced the basics with other athletes who were unable to fully participate in training due to injury: learning how to pass and catch the rugby ball, both stationary and in-motion. In my eyes at least---those of an observer accustomed to instinctual grasp of these movements from a young age---Hongwei's attempts to accustom himself with the skills of rugby were jarring.  The bizarre idiosyncrasies of rugby's ovular ball often foiled him. I would regularly see Hongwei chasing after a ball he'd just fumbled, as if he was chasing in vein after a scurrying rabbit tactfully evading his pursuit.

Whilst coaching and playing rugby in China, I watched many start exactly as did Hongwei: on the sidelines of training, learning how to pass the ball. But for some reason I found Hongwei’s attempts to learn particularly unusual.  His actions appeared so mechanical that it was almost as if he was deliberately (over)imitating the required actions of passing, catching, and running as a signal of diligence and commitment (at the expense of any personal reckoning or negotiation with these prescribed techniques).
%---> I will work on re-wording this


A few weeks into Hongwei’s time at the Institute, I asked head coach Zhu Peihou about his newest recruit.  He immediately shook his head and scrunched up his face dismissively, adding in a disappointing whisper, ``no good'' (\textit{buxing} 不行).

Chinese rugby coaches are well acquainted with athletes starting from scratch with the technical requirements of rugby---they were used to things looking awkward and ugly at the start. Coaches are often more interested in the physical raw materials that enable athletes to develop into rugby players over time.  Often this means that coaches have a habit of fixating on an athlete's baseline characteristics (e.g. height and body frame) as an indication of his or her capacity to develop the physical speed, size and strength deemed crucial for elite performance.  Also important, but less crucial than physical attributes, are an athlete’s baseline ball-handling skills and ``game sense'' (which coaches often assess by observing new recruits participating in analogous interactive team sports, like basketball or association football).

Hongwei was still relatively young and physically undeveloped when he arrived, but it was already clear that he was not endowed with a big physical frame; nor was he noticeably fast or agile compared to other athletes.  For these reasons, head coach Peihou couldn’t help but let on to me that he was not particularly excited about Hongwei's future prospects in the Program.  In fact, I got the sense that Peihou's reaction to my question contained an element of annoyance or frustration with the terms under which Hongwei had arrived to the Institute, i.e., via the arrangements of Hongwei’s athletics coach.  According to Peihou's own assessment of Hongwei's ability and future potential, the head coach had perhaps conceded that he had been forced to accept an athletic ``dud'' into the team.

\subsection{Vignette 2}
I first interviewed Hongwei approximately six weeks after he arrived at the Institute.   Hongwei's demeanour during the interview was consistent with the timid and shy one that he presented publicly at training. He did show some signs of captivation with his new sport and social environment.  When I asked about his initial impressions of the on-field demands of rugby, however, Hongwei was quick to confess that he felt utterly unacquainted:

\begin{quotation}
  Hongwei: I still haven’t really started to practice any of the team plays or anything; all I can do so far is pass and run a little bit...(but) it's quite fun! \\
  JT: What do you think is the most difficult component of rugby? \\
  Hongwei: Um\textellipsis well, coordinating with teammates [on the field], particularly coordination in attack.  Because I can't figure it out. When I first arrived, I didn’t even know what a ``switch play'' or a ``blocker play'' was.
\end{quotation}

\begin{CJK}{UTF8}{gbsn}
  \begin{quotation}
    :战术没怎么接触,就是像传球啊、跑动什么的会一点了 \\
    ?感觉怎么样?\\
    :挺好玩的!\\
    ?你认为橄榄球最难的一部分是什么? \\
    :...打配合,进攻的配合,因为搞不明白,刚来的时候也不知道什么交叉,后插什么的 \\
  \end{quotation}
\end{CJK}

When discussing the technical demands of rugby, Hongwei was bashful in his confession regarding his minimal grasp of these requirements. This was indicative of a broader trend, as I will explain in more detail in the ethnographic sections of this dissertation (Chapters 3 and 4). When athletes were asked in interviews about the most difficult aspect of rugby, on-field coordination with teammates was by far the most common answer, particularly among junior athletes (rugby training age < 3 years).

As I directed the interview towards topics beyond the on-field technical demands of rugby, Hongwei was more positive, framing rugby as an exciting new opportunity, and commenting that his friends and family were in awe of the fact that he is playing such an impressively ``strong'' physical sport like rugby. When I asked what was new about rugby that he hadn't experienced before, Hongwei automatically responded by emphasising the social dimensions of his experience at the Institute:

\begin{quotation}
  ``...I think it's mainly this thing of having teammates. Before, when I was training for an individual sport, it was just me training by myself. [In that environment] it was a case of whoever trained well was successful.  But now with this team of brothers, elder teammates will take care of younger teammates. We all train together, and if you can’t do something, you can always ask your elder teammates...[Rugby] is so much better, because in an individual sport, if you can't master something, you have to go to your coach for help. Other athletes don't want to teach you, because if you surpass other people, then they have to work even harder to keep up... I have had to learn about helping each other, because rugby is not like an individual sport, where you look after your own performance and that's it.  In a team sport, if you don't do well, there's no need to get too frustrated or upset, because other athletes will help you out, and I will also help others out, that type of collaboration with each other.''
\end{quotation}

\begin{CJK}{UTF8}{gbsn}
  \begin{quotation}
    :我觉得主要是师哥师弟的这一块儿,原来练个体项目都是自己练自己的,谁练好了谁厉害,但是现在师哥师弟,有师哥照顾师弟带着,互相练,我不会我可以问师哥
    :好多了,因为个人项目你不会就必须要找教练,但是别人不愿意教你,因为你把别人超越了,那别人还还得努力。(3) :学到互相帮助,因为向个体项目自己成绩自己来拿就行,而像团体项目,即使自己做不好,也不用太泄气太沮丧,因为别人会帮你做好,我也会帮别人做好,互相协作的那种.
  \end{quotation}
\end{CJK}

Hongwei’s background was as a pentathlete (track and field / athletics---an individual sport), and through his explicit reference to the fraternity of the Program, and his position as junior member, he highlights that the technical skills of rugby were not the only novelties of importance to Hongwei. As I listened to his experiences associating rugby and group membership, I could not help but associate the quality of these declarations of prosociality with his overly mechanical imitation of rugby's foundational techniques.  It was as if both were equivalently diligent (albeit somewhat unsophisticated) signals of team commitment.

\subsection{Vignette 3}
A few months passed, and Hongwei continued to train. He was as eager and committed as when he began, and I did notice some gradual improvement in his grasp of rugby's basic skills.  But he also remained extremely reserved, keeping his head low at all times in team settings, unless addressed by senior players or coaches.  Then, one evening when I had returned to my room in the Institute dormitory from a three-week hiatus in Australia over Christmas of 2015, I heard a knock on my open door, and to my surprise Hongwei took an assertive stride into my room, carrying in two arms a draw-string bag containing rugby balls (which were in need of more air before the next day's training session).  Hongwei had never ventured into my room before on his own accord, apart from our first interview, two months earlier.  Remarkably, Hongwei looked me straight in the eyes with his head held high and energy beaming from his face and chest. I couldn't help but smile and ask, with genuine intrigue, ``How has training been recently?''
``Very good'' he said, assertively and excitedly.  ``Much better than before.  At least now I know what’s going on at training, I can keep up with the plays!''  A big smile grew on his face as he continued to hold my gaze.  ``Oh good!'' I said. I congratulated him for his hard work in training while I had been away, and encouraged him to keep at it. Wow, I remember thinking to myself, ``the force'' was in him.  Somehow, Hongwei, rugby, and the team in which he was by now enmeshed had clicked into place to instil him with a visceral sense of agency.
%(I felt like giving him a pat on the back and suggesting that he try to relax and take the weight off his shoulders, thinking that his anxiety about fitting in may be getting in the way of the ultimate goal of fast-tracking skill acquisition!)
%Hongwei had begun to develop an innate feel for the game.

\subsection{Vignette 4}
A few weeks later, the new head coach of the Program, Chongyi (who took over from Peihou, who abruptly resigned while I was away in Australia), told me that he had decided to take Hongwei to pre-season training in Guangzhou, for a month starting in March 2016.  Chongyi admitted that while Hongwei was perhaps not the most promising of the junior athletes, his attitude was very good:

\begin{CJK}{UTF8}{gbsn}
  \begin{quotation}
``He [Hongwei] likes to train, and he is very diligent. I want to take his positivity with us [to Guangzhou]'' \\
他爱练,而且很用心,带上他的积极性过去
  \end{quotation}
\end{CJK}

\begin{center}
  * * *
\end{center}


These ethnographic observations relating to Hongwei's first four months at the Institute highlight key themes of this dissertation.  As I discuss in more detail in Chapter 2, existing research suggests that successful coordination in joint action requires alignment and maintenance of equivalent expectations between co-actors \citep{Sebanz2006,Vesper2017,Pesquita2017}.  Importantly, evidence also suggests that violation of expectations in joint action has strong affective consequences \citep{Chetverikov2016}.  The fact that Hongwei's familiarity with the technical requirements of rugby appeared over time to co-vary with aspects of his personal demeanour suggests a relationship worthy of further investigation.  As I explore in Chapters 3 and 4, my ethnographic observations reveal broader patterns of within-group variation between perceptions of joint action performance, team click, and feelings of social bonding.  These observations, coupled with predictions from existing literature within the social cognition of joint action, set the foundations for subsequent field-experimental studies in which I test specific relationships with a broader subset of Chinese professional rugby players beyond the Program at the Institute.


\section{Existing cognitive and evolutionary theories of group exercise}

How is it possible to scientifically account for the unmistakably ``visceral'' quality of Hongwei's transformation from timid newcomer to budding Beijing rugby player? How can these ethnographic observations be explained in terms of generalisable cognitive mechanisms and systems dynamics of physical movement?  Finally, and ultimately, how do answers to these questions improve our ability to comprehend the evolutionary significance of group exercise in the human record?

Because physical movement is a metabolically expensive endeavour for all biological organisms, it is justifiable, in an evolutionary sense, only if the benefits somehow outweigh the costs. It is easy to imagine how physiologically exertive and coordinated group activities would have served important survival functions in our ancestral past, such as hunting, travel, communication, and defence \citep{Sands2010}. However, in more recent domains of human history---at least since the late Pleistocene era (approx. 500ka), and particularly since the Holocene transition (approx. 11ka) from hunter-gatherer to agricultural, and later industrial and post-industrial, societies---the task of explaining the persistent recurrence of group exercise is much more complicated.  Cross-cutting shared cultural practices as varied as religion, organised warfare, music, dance, play, and sport, the fitness-relevant benefits of group exercise are not so immediate or obvious.  Instead, causal explanations for the prevalence of group exercise are intertwined with the causal processes of a species-unique evolutionary trajectory, defined by increasingly complex cognitive and cultural capacities, including technical manipulation of extra-somatic materials and ecologies; advanced theory of mind; and information-rich, malleable, and scaleable communication systems \citep{Fuentes2016}.  A theory capable of satisfactorily explaining group exercise within these distinctive evolutionary parameters is yet to be fully formulated \citep{Cohen2017}.

Interpersonal coordination and physical exercise are both independently adaptive behaviours from which humans derive numerous benefits \citep{Tomasello2014}. It is therefore plausible that we have evolved physiological, cognitive, and social mechanisms that reward group exercise.  Current cognitive and evolutionary understandings of group exercise tend to converge around this reasoning, i.e., that group exercise is an adaptive behaviour that has been subject to mulltilevel selection pressures in humans' ancestral (and perhaps even more recent) past \citep{Sands2012,Dunbar2010,Cohen2017}. (Current?) evolutionary accounts of group exercise tend to remain true to the conventions of the Modern Synthesis (MS), whereby proximate mechanisms are investigated and adaptive value is inferred from the dependent variables of these studies in order to hypothesise about ultimate evolutionary explanation of the behaviour.  In addition to the obvious health and wellbeing benefits of physical exercise that appear to suggest the adaptive function of physical activity on the level of the individual, evolutionary accounts also emphasise evidence of positive emotional and social effects of \textit{group} exercise as support for the hypothesis that group exercise and the cultural activities in which group exercise commonly features (music, dance, ritual, sport) have been subject to processes of cultural group selection \citep{Dunbar2010,Whitehouse2004,Atkinson2011a}.  As I will develop in the sections that follow, this prevailing evolutionary account of group exercise---while convincing and coherent---threatens to occlude key dimensions of group exercise context, particularly those relating to the nonlinear self-organsing dynamics of human movement systems.  The current evolutionary account, in which group exercise is tethered to processes of social cohesion via evolved cognitive and neuropharmacological mechanisms is currently unable to account for subjective and distributed cognitive processes of joint action common in interactive team sport scenarios. I explain that constraints upon theories of group exercise are part theoretical, part methodological.  This dissertation makes contributions that work towards relieving both constraints.

\subsection{The Modern Synthesis}
Generally speaking, rigorous and scientifically testable accounts of human behaviour have emerged in the last ~70 years, facilitated by 1) the gradual refinement of evolutionary theory over the last 200 years now known as the ``modern synthesis,'' as well as 2) the ``cognitive revolution'' of the 1950s and 60s, in which mechanisms associated with information theory, cybernetics, and computation provided useful conceptual metaphors for understanding population-level transmission and fixation of biological and cultural variants. Below, I provide a brief overview of the main assumptions, protocols, and critiques of the modern synthesis, identifying knowledge gaps and research opportunities along the way.

The modern synthesis (also known as ``neo-Darwinism'', hereafter simply MS) refers generally to the gradual maturation of evolutionary theory in the last 200 years, and specifically to the unification of the theory of evolution by natural selection (attributed to Darwin and Wallace in the second half of the 19th century) with a theory of genetic inheritance (replacing a previously popular theory of blended inheritance).  The MS was first proposed by Huxley in 1942, following successful mathematical formalisations performed by population geneticists between 1930 and 1947 (e.g., Fisher (1930) and Haldane (1932)).  Subsequent advances in molecular biology and genetics, including the verification of the structure of the DNA molecule by Wallace and Crick in 1954, paved the way for a definition of biological evolution as changes in the frequency of heritable DNA sequences in a population due to selection pressures exerted at the level of the phenotype \citep{Dawkins1976,Grafen1984}.  Shown to be mathematically plausible, the mechanism of genetic inheritance served to explain observable intra-species phenotypic variation (for which the preceding theory of blended inheritance failed to account), and confirmed Darwin's original insight that organismic change occurs via gradual population-level accumulation of adaptive traits over evolutionary time. The MS and its associated methodological innovations and empirical findings have collectively transformed scientific knowledge of the historical origins and function of biological (including human) life. This synthesis has been widely touted as the second most successful theory in the history of science after modern quantum physics in its ability to explain and predict the world in which we live \citep{Dunbar1996}.

The power of the MS to account for observable biological phenomena hinges on two interrelated assumptions: (1) evolutionary explanations of a biological trait are solely determined by the process of natural selection, and (2) the consequences of the process of natural selection accumulate in the germ-line in the form of statistical frequencies of programs (alleles) for specific phenotypic traits.  As Mayr first suggested in 1961, these two interrelated assumptions create conceptual space for two distinct but complimentary explanations of observable biological phenomena---one evolutionary or ``ultimate'' level, and one (immediate causal) developmental or ``proximate'' level \citep{Mayr1961}.  This ``proximate-ultimate'' distinction is a core epistemological pillar of the MS.

Assuming that natural selection is the primary agent of evolutionary change, biologists interested in an evolutionary explanation can largely set to one side immediate processes that contribute to the development or causation of a biological trait, and instead focus on the adaptive value of a trait and its phylogenetic history. \citep{Mayr1961,Tinbergen1963}.  Proximate causal mechanisms and developmental processes are not passed on in the germ-line and are thus relatively inconsequential to macro-scale processes of evolutionary change (\citep{Dawkins1982,Grafen1991,Svensson2017} but see \citep{Laland2013,Laland2015}.  Meanwhile, biologists interested in explaining the developmental processes and immediate causal mechanisms associated with a trait can proceed under the assumption that an observable phenotypic trait is the product of interaction between the organism's internal evolved biological programs (genes) and external environmental triggers or cultural capacities emanating from gene-environment interactions within a specific phenotypic niche (SOURCE).  Importantly, proximate and ultimate levels of causation must be considered together, as distinct but complimentary components of an overall explanation of the biological phenomena\citep{Mayr1961,Tinbergen1963,Scott-Phillips2011}.

%Something on Aristotle's four questions here...philosophical/epistemological groundings
%if natural selection is the primary agent of shaping biological information (assumption 1), then it can be predicted that the programs for subsequent life contained in the germ-line (assumption 2) have been subject to gradual historical processes of selection.
%Application of the MS to biological phenomena has spawned enormously productive and wide ranging research programs, which have contributed to communication between previously disparate strands of biology (structural, functional, evolutionary) \citep{Svensson2017}.

In the case of the commonly cited example of a human infant crying \citep[taken from][]{Scott-Phillips2011,Nettle2009}), a proximate explanation of this behaviour would require an account of both the external (e.g., physical separation from the caregiver, lack of food, cold) and internal (e.g., activity of the limbic system to initiate crying or the role of endogenous opioids in the cessation of crying) factors responsible for the crying behaviour \citep[38]{Scott-Phillips2011}. An ultimate causal explanation for human infants crying includes both a description of the adaptive value of crying (e.g., crying elicits support and defence from mothers and other care-givers; infants that do not cry when in need of assistance are less likely to survive), as well as an account of its phylogenetic history \citep{Mayr1961,Tinbergen1963}.  Any explanation of biological phenomena requires an account of both proximate and ultimate levels of causation---simply knowing \textit{how} it is that an infant cries (proximate), or alternatively only knowing \textit{why} it is present in the phenotypic repertoire (ultimate) is not enough to satisfactorily account for the observed phenomena \citep[38]{Scott-Phillips2011}.  In effect, the MS protocol of distinguishing between proximate-ultimate levels of explanation constructs two simple and testable causal models on complimentary levels of biological time---one being the phenotypic lifespan of the organism (proximate), the other being the phylogenetic (evolutionary) trajectory of the species. In so doing, the proximate-ultimate distinction facilitates an efficient division of labour in the dauntingly complicated task of wrangling into a theoretically consistent and empirically substantiated scientific domain, the complex, continuous, and historical processes of biological life \citep{Mayr1961}. In this regard, the MS has facilitated communication and integration between what were previously disparate strands of biology around a single explanatory project \citep{Svensson2017}.

Ever since the first declarations of the viability of a theoretical synthesis between Darwinian natural selection and Mendelian genetic inheritance, momentum in favour of the simplifying assumptions and protocols of the MS outlined above has been coupled with theoretical undercurrents, in the form of periodical critiques, revisions, and proposed extensions to the MS that call these assumptions and protocols into question \citep[see, for example][]{Waddington1950,Gould1980,Levins1985,Ingold1990,Ingold1995,Odling-Schmee2003,Piggliuci2007}. Generally speaking, critiques boil down to the claim that the simplifying assumptions and protocols of the MS (such as the proximate-ultimate distinction) do not satisfactorily account for the causal complexity of evolutionary processes \citep{Laland2011}. In the most recent calls for an ``extended evolutionary synthesis,'' \citep[EES, see][]{Pigliucci2007}, for example, researchers claim that the dichotomy between proximate and ultimate causal models impedes adequate theorisation and measurement of the ways in which proximate-level processes are linked with ultimate-level evolutionary processes via positive and negative feedback loops of reciprocal causation \citep{Pigliucci2007,Laland2011,Laland2013,Mesoudi2013,Laland2015}. Evidence within developmental biology suggests that processes of ontogeny typically understood as proximate (and therefore inconsequential to evolutionary change) can shape and co-direct evolutionary trajectories through reciprocal processes that ``reverberate'' through the organism \citep{Laland2013}. Proponents of the EES outline key domains such as developmental bias, extra-genetic forms of inheritance (i.e., epigenetic, parental, or cultural systems), niche construction, and phenotypic plasticity, which can be responsible for initiating evolutionary episodes and shaping evolutionary outcomes within specific evolutionary niches \citep{Laland2015}.  The EES thus challenges the exclusive position of natural selection as the sole director of evolutionary change.  Instead, it has been suggested that the emphasis on natural selection in the evolutionary landscape should be relaxed to make room for modelling other demographic, social, and spatial factors that are hypothesised be responsible for the influence of population-level distributions of genetic and phenotypic variants \citep{Mesoudi2013}.  The key methodological and empirical challenge to this suggestion is that these factors often spontaneously (self-organising) emerge as system dynamics, making them harder to manipulate and measure in traditional ``snapshot'' experimental paradigms \citep{Svensson2017}.

The key claim of the EES is captured by developmental biologist and leading proponent of the EES, Keven Laland, and his inversion of E.O. Wilson's original proclamation made in 1978 against the prevailing behaviourist paradigm common within the human sciences at the time: that human culture is held on a genetic ``leash'' \citep{Wilson1978}.  In 2017, Laland claimed that genes should be more accurately depicted as dog-walkers struggling to retain control of a number of unruly dogs (niche construction, developmental plasticity, developmental biases, non-genetic inheritance, etc.) pulling in different directions at different intensities \citep{West-Eberhard2003,Laland2017}.  Evolution, in this image, is depicted by the outcome of the struggle between dog-walker and dogs.  As Laland \citep[723][]{Laland2013} and colleagues suggest, ``much of adaptive evolutionary change may have its origin in plastic responses to novel environments, later followed by genetic changes that stabilize and fine-tune those phenotypes, rather than the other way around.'' Only by modelling and quantifying the dynamic coupling (reciprocal relationship) of the dogs (developmental processes) to their owner (genes) over time can the various contributors to evolutionary change be sufficiently represented \citep{Laland2013,Laland2015}.

It is worth noting that despite the most recent calls for an EES, and beneath the highest profile debates between opposite ends of the MS theoretical spectrum (for example, the popularised debate between E.O. Wilson and Richard Dawkins over the generalisability of kin selection and/or multilevel (group) selection), consistent and productive attempts have been made by evolutionary biologists to integrate the system dynamics of evolutionary processes into empirical research programs \citep{Wray2014,Svensson2017}.  Beginning with Fisher's initial attempts to account for the reciprocal dynamics of sexual selection on gene frequencies in initial mathematical formulations of the MS \citep{Fisher1930}, there have been numerous attempts to either empirically measure or mathematically model bidirectional causation in evolutionary processes.  Such attempts can identified in areas such as coevolution (SOURCE), frequency-dependent selection \citep{Prum2010}, sexual selection \citep{Svensson2009}, speciation \citep{Mayr1965}, and canalisation \citep{Waddington1950}.  As I will discuss in more detail below, the bidirectional causation of evolutionary processes is particularly relevant to human evolution, and has been modelled in relation to human social evolution in coalition formation and cooperative networks \citep{Gavrilets2008}, as well as in approaches that attempt to model the interaction between biological and cultural systems of transmission \citep{Cavalli-Sforza1981,Cavalli-Sforza1989,Boyd1988,Henrich2007,Claidiere2007}.  Reciprocal causation is also central to the field of ``eco-evolutionary dynamics,'' in which the bidirectional feedbacks between ecological (demographic, social, spatial) and evolutionary processes (genetic change within populations) are modelled, particularly in instances in which ecological and evolutionary timescales converge \citep{Hendry2017}.
While the EES represents the most recent and vocal critique of the narrowest assumptions and protocols of the MS, it is important to keep in mind that consistent empirical and theoretical progress has been made within the existing parameters of the MS.  Indeed, many evolutionary biologists insist that the predictions of the MS are robust and resilient enough to endure continued innovation and modification, including the incorporation of nonlinear dynamics, without the need for an ``extension'' \citep{Wray2014,Lewens2017,Svensson2017}.
%epistemological TENSION here between line and circle.

Despite these areas of theoretical contentions, almost all evolutionary theorists now agree that the only way to progress evolutionary approaches to knowledge beyond theoretical advocacy is to initiate methodologically innovative empirical research programs designed to test and expand the predictions of both the MS and EES, whether or not they constitute the same or separate scientific paradigms.  In particular, this includes utilising methods capable of quantifying (and qualifying) reciprocal causation in both micro and macro evolutionary processes  \citep{Wray2014,Laland2014,Laland2015,Svensson2017}.  As I will explore in the next section, the social cognition of joint action is a research domain rich in methodological and empirical opportunities, and in which key knowledge gaps in evolutionary approaches to behaviour (particularly approaches to human behaviour) can be addressed.


  \subsection{MS applied to human behaviour}
In this section, I review the application of MS approaches to human behaviour in order to assess their usefulness in an explanation of group exercise.  The phenotypic niches of many organisms involve interactions between complex social, ecological, and spatial dimensions made up of social partners, other species, and environmental fluctuations over various timescales. The human evolutionary niche appears to be distinct, however, in that it entails species-unique capacities for complex manipulation of extra-somatic materials (producing artefacts ranging from stone tools to quantum computing); information rich, malleable and scalable communication systems (producing syntactically complex language); and physiological, psychological, social affordances of shared cultural practices (producing phenomena such as ritual and religious activities). Importantly, the distinctiveness of the human evolutionary trajectory , (identifiable in its various biological, technical, and cultural productions,) appears to be fundamentally contingent upon processes of bidirectional causation between proximate developmental processes and ultimate evolutionary consequences.  As I will outline below, however,
most applications of MS to human behaviour remain constrained by evolutionary models that adhere relatively closely to a narrow neo-Darwinian conception of evolutionary processes \citep{Claidiere2014,Mesoudi2017}.  The theoretical tensions and methodological challenges of the MS paradigm outlined above are thus particularly salient when dealing with evolutionary explanations of human behaviour.

% perhaps start section here (BT)

 Initial attempts to use the MS paradigm to account for the complexity of the human evolutionary niche fixated on explaining humans' species-unique behavioural capacities---what is traditionally understood as ``culture'' (broadly construed)---in one of two ways.  First, within the strictest Darwinian interpretations of the MS, human cultural capacities are understood as purely proximate mechanisms of the phenotype, in Dawkins' parlance, part of an ``extended phenotype'' that provides genes with adaptive affordances (organisms and organism-curated environments) conducive to replication \citep{Dawkins1982}. Second, strict MS adherents also tend to endorse that human culture resembles an evolutionary system in its own right---distinct from (albeit contingent upon) biological evolution, and should therefore be modelled as such. Initial MS approaches to human behaviour thus began with the recognition that the processes associated with population-level transmission, accumulation, and fixation of cultural variants (cultural practices, norms, languages, etc.) were similar in many ways to processes of biological evolution. The strictest versions of MS approaches to human cultural evolution propose the ``meme'' as a gene-like unit of cultural information, which, like a gene, is subject to selection pressures of replication.  Memes that successfully replicate will successfully populate \citep{Dawkins1976}.  In both these interpretations, explanations offer very little room for considering the causal determinacy of human cultural capacities. Instead, culture is a proximate mechanism of the phenotype, or an evolutionary system in its own right, distinct from the replication dynamics of genes.

Beyond the strict meme approach to cultural transmission, models of cultural evolution adjust population genetic models to take into account the observable differences between cultural and genetic information, such as culture's capacity to support one-to-many transmission, the blending of cultural variants, and non-randomly guided variation \citep{Cavalli-Sforza1981,Boyd1988}.  These adjustments are part of the concession that cultural variants are not as dependent on high fidelity replication as their genetic cousins, but instead are shaped by evolved cognitive biases that favour the acquisition and transmission of some cultural variants over others due to their memorability or effectiveness \citep{Henrich2007}.

Theories of cultural evolution thus direct attention towards the role of cognitive mechanisms of imitation, teaching, and memory in enabling high fidelity copying (with occasional mutation-like errors) of cultural variants between individuals and throughout populations, with distributions stable enough for selection to operate.  Models indicate that for social learning to actually enhance population fitness, it must be cumulative throughout generations, i.e., individuals must be able to socially learn what they could not learn on their own \citep{Boyd1995}.  Thus, particular attention has been paid to the mechanisms that could be responsible for facilitating species-unique \textit{cumulative} culture \citep{Tomasello2008}.  Evidence from comparative and developmental psychology indeed appears to confirm that a precocious and species-unique tendency to accurately imitate the actions of trusted or authoritative others (even when the goal of the action is unclear) sets the cognitive foundation for the transmission of cultural representations \citep{Tomasello2014a}.

Analyses of microevolutionary processes of cultural transmission have been enhanced by supplementing macroevolutionary processes, also known as cultural phylogenetics \citep{Mace1994}.  The phylogenetic comparative method seeks to understand long-term cultural change at or above the level of the society by 1) reconstructing the cultural evolutionary history of a particular trait or set of traits and 2) testing functional hypotheses concerning the spread or distribution of cultural variation across societies while controlling for evolutionary history.  The combination of these micro- and macroevolutionary approaches to human behaviour supports the theory that  ``dual-inheritance'' or ``co-evolution'' of genetic and cultural information in humans over time has led to a diverse range of capacities ranging from behaviours as lactose tolerance in adult populations \citep{Feldman1989}, to a ``norm-psychology'' thought to underwrite prosocial behaviours and institutions that facilitate collective adherence to shared cultural practices \citep{Richerson2008,Chudek2011}.  While these adjusted models help to nuance the story surrounding dual (co-dependent and mutually reinforcing) systems of inheritance, they nonetheless adhere to relatively strict logic of selection as the key determinant of population-level frequencies of biological and cultural variants.


%In contrast to the informal, idiosyncratic, and subjective schemas of historical linguistics, archaeology, and social and cultural anthropology, these micro and macro theories of human cultural evolution are explicit in their assumptions, repeatable and extendable by others, and easily scaled up to large datasets \citep{Mesoudi2017}. The fact that many details of human social interaction still remain open scientific questions is a necessary part of the trade-off involved in building a scientific formulation of cultural evolutionary processes.

\subsubsection{Cultural Attraction}
Recent contributions have begun to challenge the dominance of selection in the evolutionary landscape of human cultural variants.  There is evidence to suggest that, beyond microevolutionary mechanisms of transmission, other factors may also have an important causal impact on the accumulation and distribution of cultural variants. Whiten \textcite{Whiten2000} demonstrates in the case of non-human primates that cultural knowledge may not necessarily accumulate via strict imitation or copying, but also from affordances of social-ecological niches.  In this sense, the ``ratchet'' of human cultural learning may rely on contingencies that have accumulated within the evolutionary trajectory of the species over long periods of time, associated with processes such as niche construction \citep{Odling-Smee2003}, canalisation of selection initiated by phenotypic plasticity \citep{Godfrey-Smith2017}, and other forms of non-genetic inheritance \citep{Lewens2017}.  Demographic factors such as population size, structure, and interconnectedness have been shown to determine cultural complexity (variation) in hunter gatherer populations, with adaptive implications \citep{Henrich2004}. It has also been suggested that variation in prosociality, social bonding, and social cohesion could have an important bearing on information transfer between individuals and within groups \citep{Heyes2011,Whitehouse2014,Wheatley2016}.  As researchers in comparative and social psychology have pointed out, humans do not merely aggregate, but rather actively congregate around shared cultural practices, seemingly driven by species-unique affective and motivational mechanisms \citep{Dunbar2010,Tomasello2005a}.  In addition, there is evidence of cross-cultural variation in the microevolutionary dynamics of cultural evolution, for example, with specifically higher social learning in collectivistic East Asian societies than in individualistic Western societies \citep{Mesoudi2015,DiYanni2015}.

In light of this collection of evidence, researchers have sought to broaden the scope of cultural evolution by relaxing the strictly selectional logic of memetic and dual-inheritance models, instead suggesting that cultural variants will tend locally towards certain ``attractor points'' depending on the diverse cognitive, demographic, ecological factors of attraction to which they are subjected \citep{Sperber1996}.  Rather than explaining patterns of cultural diversity, stability, and change in terms of the differential selection of certain cultural variants (e.g., content biases) or differential copying of certain individuals (e.g., success bias, prestige bias),  ``cultural attraction theory'' (CAT) focuses on how cultural variants are systematically \textit{re-produced} by a combination of frequency dependent (i.e. conformity) and context sensitive (i.e. prestige) transmission biases, as well as the biophysical, psychological, historical, and ecological dynamics by which these biases are constrained and directed \citep{Claidiere2014}.  It has been pointed out that in contrast to genetic evolution, the mechanisms responsible for transmitting cultural information in humans (imitation, learning, and memory) cannot alone explain population level stabilisation of cultural variants, because they are not faithful enough to stabilise distributions of cultural variants on which selection can operate \cite{Claidiere2014}. CAT suggests instead that population level cultural variation is produced by processes that are partly preservative (i.e., occur via mechanisms of transmission), and partly re-constructive---the combination of which will result in cultural variants that tendentially converge upon particular types, called attractor points.
%metastable system is the unit of selection

%\subsubsection{Summary of knowledge gaps in MS approaches}
The CAT framework offers one of the most complete evolutionary models to challenge the strict neo-Darwinian assumptions of the MS approach.  The model is, in name at least, non-linear and dynamical, using language of ``attractor points'' and tendentiality.  While CAT signals a constructive attempt to develop ideas of evolution beyond the strict priorities of the MS, it is also important to interrogate to what extent CAT remains constrained by MS assumptions.  CAT relaxes the selective environment to include other non-Darwinian ``factors of attraction,'' but it does not alter the unit of transmission, i.e., the cultural ``representation.''  As emerging strands of evidence from the social cognition of joint action suggests, the unit of selection may not be the representation, but the functional assemblages that reliably give rise to recognisable cultural forms \citep{Kelso2016,Yufik2013,Corning2013,Nowak2017}.  This line of argument resembles the image of the dog-owner coupled to forces of development, and evolution as the product of the struggle between the two--the path that the entire assemblage travels.


\subsection{MS applied to group exercise}
Existing cognitive and evolutionary approaches to group exercise adhere closely to the conventions of the MS, particularly the distinction between proximate and ultimate explanations.  On the one hand, considerable research has shed light on to the immediate physiological, cognitive, and social mechanisms hypothesised to play a casual role in group exercise.  On the other hand, researchers have also considered the adaptive value of cultural activities and social organisation in which group exercise commonly features.  While the physiological, psychological, and social processes that combine in instances of exerted, coordinated movement are rich and varied, existing evidence pertaining to the immediate causal triggers of group exercise suggest a link between group exercise and social cohesion \citep{Davis2015,Cohen2017}.

Given the constraints of MS assumptions and protocols, and the methodological challenges associated with deriving quantities for dynamical properties of physical movement in group exercise contexts, many dimensions of group exercise have been occluded and overlooked.  In particular, I identify the research domain of the social cognition of joint action as an area in which progress can be made in filling theoretical and empirical knowledge gaps, and thus expanding an account of group exercise in the human evolutionary trajectory.

%Physical activity, exercise, and sport have well-known positive effects on psychological and physical health (Ekkekakis, 2003; Fiuza-Luces, Garatachea, Berger, \& Lucia, 2013).Social scientists have also long speculated about the benefits of energetic group activities such as ritual, music, and dance for social cohesion (Durkheim, 1965).Group exercise contexts typically require the coordination of both movement and intentions (Reddish, Fischer, \& Bulbulia, 2013), which together activate neurobiological mechanisms implicated in social reward (Dunbar, 2010; Eisenberger, 2012), as well as those involved in enduring the pain and discomfort of physiological exertion, i.e., the ``runner’s high'' (Boecker et al., 2008; Dietrich \& McDaniel, 2004; Sullivan, Rickers, \& Gammage, 2014).

\subsection{Proximate mechanisms of group exercise: physiological exertion and behavioural synchrony}
Research that sheds light on the proximate mechanisms of group exercise tends to relate to one of two dimensions of group exercise: 1) the level of physiological exertion and 2) the level of coordination of movement between co-actors.  It is now understood that strenuous and prolonged physical exercise is modulated by the same neuropharmacological systems (namely, the opioidergic and endocannabinoid systems) responsible for regulating pain, fatigue, and reward \citep{Boecker2008,Raichlen2013}.  Exercise-specific activity of these systems offers a plausible neurobiological explanation for commonly reported sensations of positive affect, anxiety reduction, and improved subjective well-being during and following exercise---extremes of which are popularly referred to as the ``runner's high'' (Dietrich  McDaniel, 2004; Boecker et al. 2008; Raichlen, Foster, Gerdeman, Seillier,  Giuffrida, 2012).

Meanwhile, research in social psychology focusing on the relationship between time-locked behavioural synchrony and processes of self-other merging, social alignment, and affiliation has shed light on the social and affective significance of interactive and coordinated movement typical of many group exercise contexts \cite{Wiltermuth2009,Kirschner2010,Reddish2013,Tuncgenc2016}. Experimental evidence suggests that time-locked coordination of behaviour between two or more individuals in the stable attractor/equilibrium states of either in-phase or anti-phase synchrony is conducive to psychological processes of self-other merging, liking, trust and affiliation.  It is believed that lower cognitive processes of joint attention mediate the link between synchrony and social bonding, with synchronised activity (common in music, dance, and some sports) providing a shared spatio-temporal (and often haptic) referent around which to coordinate attention and behaviour \cite{Launay2016,Wolf2015}.

Studies linking synchrony with social bonding and cooperation are supported by a literature than connects nonconscious mimicry with liking and affiliation\citep{VanBaaren2009}.  The experimental studies above predominantly refer to dyadic synchronisation of behaviour.   There is also experimental evidence to suggest that exertive and social or coordinated dimensions of group exercise interact to produce social effects.  Recent experimental evidence suggests that social features of the exercise environment (for example, perceived social support, level and quality of behavioural synchrony, etc.) modulate exercise-induced mechanisms of pain, and reward \citep{Cohen2009,Sullivan2014,Tarr2015,Davis2015,Weinstein2016}. This work is bolstered by existing literature on the social modulation of pain \citep{Eisenberger2012a} and links between pain and prosociality \citep{Bastian2014a}.

%The social and psychological effects of group level synchronisation have been harder to induce and measure in experimental settings. However, in addition to in- and anti-phase behavioural matching, group synchronisation may be subject to more complex and dynamical processes of coupling, which could entail specific psychological consequences. This also appears to be true in cases of joint---but not necessarily explicitly synchronised---action, whereby implicit processes of movement regulation link two or more individuals in a complex and dynamic coupling. The variation and stabilisation of such dynamic couplings could have psychological effects (see \citep{Schmidt2008,Marsh2009a}).

% BT: Before you launch into grooming story, include here (e.g. at end of this sentence) something that indicates where you are going with this (e.g. the ‘grooming at a distance’ concept).
\subsection{Ultimate explanations: social cohesion and cultural group selection}
The neuropharmacological account of group exercise and social cohesion has its roots in studies of social grooming in non-human primates.  Dunbar and colleagues propose a neuropharmacologically mediated affective mechanism linking dyadic grooming practices with group-size maintenance \citep{Machin2011}.  The capacity for social cohesion is thought to have arisen in primates as an adaptive response to the pressures of group living.  Aggregating in groups serves to reduce threat from predation.  At the same time, it can be individually costly due to stress arising from interactions at close proximity and conflict over resources among genetically unrelated individuals.  These pressures are hypothesised to have led to selection for social bonding (e.g., via dyadic grooming). Resulting coalitional alliances among close partners allow for the maintenance of the group by buffering the stresses of group living.  Primate social grooming, for example, is associated with the release of endorphins, presumably leading to sustained rewarding and relaxing effects.  While other neurotransmitters such as dopamine, oxytocin, or vasopressin may also be important in facilitating social interaction, endorphins are argued to reinforce individuals (who are not related or mating) to interact with each other long enough to build ``cognitive relationships of trust and obligation'' \citep[1839]{Dunbar2012}.  It is thought that, as the homo genus evolved more complex collaborative capacities for survival in interdependent group contexts, grooming-like behaviours sustained social bonding in larger groups where dyadic grooming would cumulatively take too much time \citep{Dunbar2012}.  Experimental studies suggest that neurophysiological mechanisms activated by activities that involve physical exertion and coordinated movement, such as group laughter, dance and music-making, exercise, and group ritual can bring groups closer together, mediated by the psychological effects of endogenous opioid and endocannabinoid release \citep{Cohen2009,Fischer2014a,Fischer2014,Sullivan2014,Tarr2016,Tarr2015}.

%explanation of ritual practices: Whitehouse, Fisher & Xygalatas
%Unifying theories: WHITEHOUSE ritual practices
Recently, anthropologists have attempted to integrate analyses of ethnographic, archaeological, and phylogenetic information in order to develop broader theories of social cohesion. Drawing initially from ethnographic observations of ritual practice in the Papua New Guinean Highlands, Harvey Whitehouse developed a general theory of human social cohesion based on two divergent modes of ritual practice and their associated psychological and sociopolitical effects \citep{Whitehouse1996}. High-frequency, low-arousal religious rituals (weekly attendance at church sermon, praying, etc.) are associated with identification with the prototypical features of the group (“group identification”) whereas low-frequency, high-arousal rituals (tribal initiation rituals, dysphoric experiences such as frontline combat) generate “identity fusion”—--a visceral feeling of oneness with the group.

Whereas group identification can be understood as a psychological adaptation deriving from a norm coalitional cooperative mechanisms, identity fusion arises from the generalization of kin-detection mechanisms, whereby individuals recognize others with whom they have shared core self-defining experiences as ``fictive kin.'' Whitehouse and colleagues (2014) argue that these two distinct psychological states, and the ritual practices that reliably generate such states, represent ``attractor positions'' (Sperber 1996) in the cultural evolution of religion and human sociality. In a return to Durkheim’s original outlay for the study of social cohesion, the modes theory incorporates the two key underpinning mechanisms of social cohesion (i.e., cumulative culture and its interaction with human capacities for social bonding) by accounting for variance both in the modes of shared cultural practices and in the emotional quality of group-level commitment.

In his study of religion, prosociality, and extreme altruistic behaviour, Scott Atran similarly insists on the need to carefully consider the interaction between cultural, cognitive, and affective mechanisms, in particular the role of communal rituals in “rhythmically coordinat[ing] emotional validation of, and commitment to, moral truths” (Atran and Norenzayan 2004, 714). Atran suggests that extreme altruism can be explained only by considering the codependent relationship between the affective motivational processes (of arousal, pain, and reward) and the cultural representations (“sacred values” of the group) with which these psychophysiological mechanisms interact and become associated.

Relatedly, recent empirical studies of extreme ritual practice present evidence of a positive relationship between pain experienced during high ordeal rituals (e.g., walking on hot coals) and subsequent expressions of parochial prosociality. Interestingly, in contrast to the prevailing adherence to multilevel selection and cultural-group selection theories of social cohesion, these researchers speculate that the selective advantages available in extreme group ritual are afforded primarily to the individual (e.g., moral cleansing, improvements in social standing), with group-level benefits arising only secondarily (Fischer and Xygalatas 2014). In so doing, Fischer and Xygalatas, for example, centralize the explanatory role of proximate psychological mechanisms (of arousal, pain, and reward) in establishing and maintaining social cohesion.
<Signpost here: where are we, where are you going with this, what’s next in the review?>


%Most encouraging is evidence that manages to integrate the social and neurophysiological dimensions of group exercise.  Recent experimental evidence suggests that social features of the exercise environment (for example, perceived social support, level and quality of behavioural synchrony, etc.) modulate exercise-induced mechanisms of pain, and reward \citep{Cohen2009,Sullivan2014,Tarr2015,Davis2015,Weinstein2016}. This work is bolstered by existing literature on the social modulation of pain \citep{Eisenberger2012a} and links between pain and prosociality \citep{Bastian2014a}.

%The social and psychological effects of group level synchronisation have been harder to induce and measure in experimental settings. However, in addition to in- and anti-phase behavioural matching, group synchronisation may be subject to more complex and dynamical processes of coupling, which could entail specific psychological consequences. This also appears to be true in cases of joint---but not necessarily explicitly synchronised---action, whereby implicit processes of movement regulation link two or more individuals in a complex and dynamic coupling. The variation and stabilisation of such dynamic couplings could have psychological effects (see \citep{Schmidt2008,Marsh2009a}).

\section{Knowledge gaps in evolutionary approaches to group exercise}

Current evolutionary accounts of group exercise emphasise the role of proximate cognitive and neuropharmacological mechanisms capable of inducing a psychophysiological environment conducive to forging affiliative social bonds \citep{Dunbar2010,Cohen2017}. Under the assumption that social bonding contributes to processes of social cohesion, researchers propose the evolutionary story that group exercise has become implicated in processes of cultural group selection.  This account of group exercise is compelling and testable, but it also fails to integrate the unmistakably ``embodied'' dimensions of joint action and movement coordination in group exercise contexts.  In this dissertation, I argue that this embodied dimension can be quantified via the science of coordination dynamics, and should be incorporated into existing accounts of group exercise.
%BT: Specifically I will discuss this with regards to (insert sub-points here: flow, X, Y Z) I.e. provide some signposting for this section…what is the purpose of this section?
% But how does this relate to what you have just reviewed, specifically? You need to highlight how the sense of fusion etc described in above section is different to what you mean here… otherwise this just seems like a jump from previous section.
%MEANING & FLOW
<Flow>
Evidence that there is more to group exercise than existing theories currently explain lies in observation and anecdote relating to the profound meaning derived from participation in group exercise contexts.  In addition to reports of exercise-induced euphoria and positive affect, adherents to (group) exercise and other activities---particularly highly skilled practitioners---also commonly report experiencing states of ``optimal'' or ``peak'' performance, which include feelings of heightened focus, personal transcendence, time-warp (the experience of time either speeding up or slowing down), spontaneity, creativity, and effortlessness \citep{Jackson1995a}.  ``Flow,'' as this particular cluster of states has commonly been referred to, is a powerful, autotelic and embodied experience, which combines components of both ``hedonic'' (sensation-centred, see \citep{Huta2010}) and ``eudaimonic'' (meaning-centred, see \cite{Ryff1989,Ryff2015}) dimensions of subjective well-being, and is theorised to emerge when activity strikes a balance for the individual between challenge and skill requirements \citep{Csikszentmihalyi1990,Abuhamdeh2012}.  At the level of the group, the ``team click'' and ``group flow'' are highly elusive possibilities, coveted by athletes, coaches, and fans alike \citep{Novak1993,Sawyer2006}.  While the experience of flow associated with prolonged exercise may be in part neuropharmacologically mediated by the opioidergic and endocannabinoidergic systems, phenomenological accounts suggest that there is something distinct about the experience of flow in exercise that requires a more complete cognitive and social explanation \citep{Dietrich2006,Dietrich2011}.

<Meaning>
It is clear that many people do not engage in exercise just for health, enjoyment, or a ``social high''; rather, in some contexts it forms part of a life of meaning, purpose, and self-discovery (see for example Jackson, 1995; Jones, 2004; White \& Murphy, 2011). Modern sport has always been much more than ``just a game'', and instead offers an arena in which virtues and vices are learned, and the ``morality plays''—--of community, national, and globe—--thus performed (Elias \& Dunning, 1986; McNamee, 2008).  Ethnographic perspectives on group exercise contexts emphasise the ethical and moral dimensions of athletes’ experiences, and contextualise these experiences within political processes relating to the construction of the self, community, and nation-state (Alter, 1993; Brownell, 1995; Downey, 2005b; Wacquant, 2004).

Activities such as music-making, dance, and sport depend upon highly complex coordination of behaviours between individuals, in which the movements of one individual must align in time and space with the movements of another. This is a highly complex movement coordination problem involving high levels of uncertainty and ``free energy'' (degrees of freedom). Highly skilled practitioners who develop a fine-grained sensitivity concerning the perceived outcome of joint action, often report that the ecstasy of group activity is contingent not just on participation, but on the extent to which joint action with co-participants ``clicks.'' To date, however, very little research has dealt directly with the relationship between perceptions of \textit{quality} joint action and processes of social bonding and group formation \citep[but see][]{Marsh2009}.
In order to explain the social and evolutionary significance of the observable phenomenology of team click, this dissertation is grounded in the current state of the art of social cognition of joint action.

\subsection{Social cognition of joint action}
A combination of advances in neuroimaging technologies \citep{Frith2007}, emerging neurocomputational theories of brain function \citep{Friston2010,Frith2010,Clark2013}, and constructive attempts to extend the theoretical paradigm of human social cognition to account for inter-individual processes of interaction and coordination \citep{Sebanz2006,Dale2014}, has created an opportunity to empirically examine the relationship between coordinated and exertive group activities and social cohesion.  It is now more clearly understood that basal human capacities for physical movement regulation and coordination set the foundation for social cognitive systems whose resources are distributed between brains, bodies, and physical features of task-specific environments \citep{Hutchins2000,Kirsh2006,Semin2008,Semin2012,Coey2012}.  Furthermore, it has been shown that the quality of coordinated movement within these cognitive systems has implications for psychophysiological health and subjective well-being \citep{Wheatley2012}, and is relevant to the effective function of more complex goal oriented social activities, including the large-scale reproduction and transmission of shared cultural practices \citep{Dunbar2012,Roepstorff2010,Claidiere2014,Launay2016}.
Establishing functional interpersonal synergies is an adaptive response to the ``degrees-of-freedom problem'' encountered by the nervous system in social interactions involving many moving parts.  Synergies have been shown to act as an extra-neural basis for prediction error minimisation, aid reciprocal information sharing throughout system nodes, and increase individual cognitive performance \citep{Schmidt2016}.  The link between interpersonal coordination and social bonding has been addressed in the behavioural mimicry and synchrony literatures \citep[e.g.,][]{Wheatley2012,Launay2016,Mogan2017}, but there is less substantive evidence in relation to dynamic interpersonal coordination in natural joint action settings such as those found in group exercise contexts \citep{Marsh2009,Miles2009,Lumsden2012}.

Literature suggests that successful joint action in humans is contingent on the ability to share functionally equivalent task representations. Considering the cognitive principles of ``active inference'' referenced above, shared task representation amounts to minimising prediction error in social cognitive systems involving two or more co-actors \citep{Semin2008,Frith2010}.  Humans appear to employ an array of explicit and implicit behavioural strategies in order to achieve this.  The ways in which co-actors ``close the loop'' \citep{Frith2007} on joint action through deliberate ostensive communication has been the traditional focus of developmental, comparative \cite{Tomasello2005a}, and social psychologists \citep{Sebanz2006}.
More recently, however, analysis of dynamic coupling of co-actors in joint action scenarios reveals that synchronised movement implicates an array of implicit and pre-perceptual cognitive processes of alignment and prediction error minimisation \citep{Schmidt2011}, which, in addition to more explicit forms of communication, could be central to the generation of feelings of self-other merging, self-other distinction, and perceived reliability and trust associated with social bonding \citep{Marsh2009}.

A key facet of the paradigm shift in social cognition outlined above is the overhaul of traditional individual-centred computational models of information processing (originally inspired by the mechanics of the electronic computer), which tend to render movement as the final product of a linear sequence of sensory perception, amodal mental representation, and action selection \citep{Lewis2005}.  By contrast, the prevailing neurocomputational paradigm of ``predictive processing,'' also known as ``predictive coding,'' \citep[see][]{Frith2007,Kilner2009,Clark2013} offers a model of brain function in which perception, representation, emotion, and action are functionally and temporally integrated in the service of informational processes.  Perception, cognition, and action work closely together to minimise sensory prediction errors by selectively sampling, and actively sculpting, the stimulus array. <link to next section>

\subsection{Functional Interpersonal Synergies}
For a brain in the business of prediction error minimisation, interpersonal interactions with other agents present a computational challenge due to the moving parts.  The large number of independently controllable movement system degrees of freedom of multiple agents places computational burden to the central nervous system, dubbed by Bernstein \textcite{Bernstein1967} as the ``degrees-of-freedom problem'' \citep[see also][]{Turvey1982,Turvey1990}.  From a system dynamics perspective, a solution to this problem is to work in such a way that the movement system degrees of freedom residing in different actors and environmental features are coupled to form low-dimensional, reciprocally compensating synergies, known as ``functional interpersonal synergies'' \citep{Riley2011}.

Much like intrapersonal coordination, functional interpersonal synergies are temporarily assembled, task-specific, functional couplings between a system's componential degrees of freedom, such that one component of a synergy reacts to changes in the others \citep{Kelso2009}.  Once coordinated to behave as a functional unit, the individual degrees-of-freedom do not need to be controlled independently of one another, and perturbations applied to a component are automatically compensated by the coupled components \citep{Kelso1984,Latash2002,Riley2011}, making them adaptive responses to environments of high computational uncertainty.

Experimental evidence has shown that functional interpersonal synergies, for example in-phase or anti-phase coordination of hand movements, facilitates memory recall of incidental information concerning co-actors \citep{Miles2010}. It has also been shown that interpersonal synergies more generally facilitate performance of social cognitive or linguistic tasks, such as gaze coordination and turn taking in conversation \citep{Richardson2005,Shockley2009}.  Richardson and Dale (2005), for example, show that more tightly coupled gaze between conversing dyads leads to higher discourse comprehension.  Conversely, being psychologically distanced from another individual can inhibit the emergence of interpersonal synergies \citep{Miles2010}.  The ways in which functional interpersonal synergies facilitate adaptive information transfer between individuals and within groups suggests that psychological mechanisms and cultural practices responsible for generating these synergies could have been subject to cultural evolutionary forces of selection and attraction \citep{Claidiere2014,Mesoudi2016a}.

Considered form within the PP and Free Energy principle paradigms, interpersonal synergies represent one way in which error minimisation can be achieved via extra-neural mechanisms.  In addition to the behavioural mimicry and synchrony literatures mentioned above, research into the coordination dynamics of natural joint actions has shown evidence of dynamic coupling (synchronisation) in joint-action tasks, such as dancing, martial arts, moving objects like furniture, etc. In these studies, specific component degrees of freedom are modelled as coupled oscillators (using the HKB model \citep{Haken1985,Kelso1986}, which describes the change in the relative phase between two oscillatory components). Models are analysed for non-random fluctuations in relative phase over multiple time scales.  This type of synchronisation is said to be of a fractal or semi-fractal organisation, also known as 1/f scaling or ``pink noise'' \citep{Caron2017}. According to Anderson and colleagues \citep{Anderson2012}, 1⁄f scaling is ubiquitous in smooth cognitive activity, and indicates a self-similar structure in the fluctuations that occur over time (within a time series of measurements). 1⁄f scaling indicates that the connections among the cognitive system's components are highly nonlinear \citep{Ding2002,Holden2013,Kello2010,Riley2011,VanOrden2003,VanOrden2005}. Pink noise has been measured beyond dyadic synchronisation, in the analysis of sub-phases of team sports \citep{Passos2014,Duarte2012} and group dancing \citep{Chauvigne2017}.\footnote{1⁄f scaling is temporal long-range dependencies in the fluctuations of a repeatedly measured behaviour or activity. Analogous to spatial fractals, 1⁄ f scaling denotes a fractal or self-similar structure in the fluctuations that occur over time. That is, higher frequency, lower amplitude fluctuations are nested within lower frequency, higher amplitude fluctuations as one moves from finer to courser grains of analysis \cites(for a more detailed description see, for example)(){Holden2005}{Kello2009}}
% Do you need all of this? It doesn’t seem to directly inform what comes next, so I’d say, CUT!

In studies involving skilled versus non-skilled practitioners in dyadic interactions, it has been shown that more skilled practitioners create stronger dynamical coupling through flexibly modulating their actions with others \citep{Schmidt2011, Caron2017}. These findings are corroborated by other studies that find that professional footballers (versus novice controls) are able to more accurately predict the direction of a kick from another player's body kinematics (\cite{Tomeo2012}, see also \cite{Aglioti2008,Mulligan2016} for similar results with basketball and dart players). Interestingly, when analysing co-regulation between members of basketball teams, it was shown by Bourbousson \textcite{Bourbousson2015} that more expert teams made fewer mutual adjustments (at the level of the activity that was meaningful for co-actors), suggesting an enhanced capability of expert social systems to achieve and maintain an optimal level of awareness during the unfolding activity, potentially implicating down-regulation of prediction error management processes.


\section{The present study}
The empirical content of this dissertation draws from one contemporary instance of group exercise, in one geographic region.  Rugby union football and the People's Republic of China are subjects that do not commonly appear in the same sentence, but nonetheless, the Olympic status of rugby union, and the deep Olympic logic of the state-sponsored Chinese sports system, means that today hundreds of professional Chinese rugby players are meaningfully engaged in one of the world's most physiologically strenuous interactive team sports.  During a two year period between August 2015 and September 2017, I spent three separate periods in China during which I conducted a total of 10 months of ethnographic research with the Beijing Men's Provincial Rugby Team. I then extended this ethnographic analysis by conducting as two field studies, for which I sampled from a broader population of professional Chinese rugby players from 9 different provinces.

Between August 2015 --- March 2016, I spent seven months in Beijing engaged in participant observation and conducting unstructured and semi-structured interviews, and informal surveys with the Beijing Men's Rugby Team. Between July --- August 2016, I returned to China for a further two months, during which time I continued ethnographic observations of the Beijing team, while also conducting two pseudo-experimental field studies spanning two other locations, Hebei province and Shandong province. Finally, I spent one month in Beijing and Tianjin between August --- September 2017 during which time I conducted follow-up structured interviews the with 10 athletes who participated in the Chinese National Games, as well as follow up informal interviews with athletes from the Beijing Men's Rugby Team.


\subsection{Predictions}

The overarching prediction of this thesis is that the psychological phenomenon of team click mediates a relationship between joint action and social bonding.

Within this main hypothesis, I also formulate the following sub-hypotheses:
1)	Athletes who perceive greater success in joint action will experience higher levels of felt ``team click.'' I predict that relevant perceptions of joint action success will relate to athlete perceptions of:
1.a) a combination of specific technical components; or
1.b) an overall perception of team performance relative to prior expectations; or
1.c) an interaction between these two dimensions of team performance.
2)	Athletes who experience higher levels of team click will report higher levels of social bonding.
3)	More positive perceptions of joint action success will predict higher levels of social bonding, driven by more positive:
3.a) perceptions of components of team performance; or
3.b) violation of team performance expectations; or
3.c) an interaction between the two predictors.



\subsection{Results}

% You shouldn’t summarise your results in the introduction (only in the 1-page abstract, right at the front of the thesis)
\section{Thesis Contributions}

\section{Chapter Summary}

% A brief chapter summary of what we have just read would be useful. And a bridge to the next chapter.
