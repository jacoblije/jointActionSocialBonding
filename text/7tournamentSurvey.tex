  %\documentclass[12pt]{report}
%\usepackage[utf8]{inputenc}
%\usepackage{graphicx}
%\graphicspath{{images/}}
%\usepackage{pdfpages}
%\usepackage{booktabs}
%\usepackage[british]{babel}
%\usepackage{csquotes}
%\usepackage[backend=biber,style=apa,sorting=nyt,natbib=true]{biblatex}
%\DeclareLanguageMapping{british}{british-apa}
%\addbibresource{references.bib}
%\usepackage{cleveref}
%\newcommand{\crefrangeconjunction}{ to~}
%\usepackage{enumitem}
%\setlist[description]{leftmargin=\parindent,labelindent=\parindent}
%\newcommand{\myparagraph}[1]{\paragraph{#1}\mbox{}\\}
%\usepackage{geometry}
%\usepackage{pdflscape}
%\usepackage{amsmath}

%\title{
%{National Tournament Survey Study}\\
%{Qianan, Hebei Province, July 16-17 2016}
%}
%\author{Jacob Taylor}



%\begin{document}

%\maketitle{}


%\tableofcontents
%\clearpage


\chapter{\label{tournamentSurvey}National Tournament Survey Study}


\section{Abstract}
The following study was designed to analyse the hypothesised relationship between joint action, team click, and social bonding between professional Chinese rugby players in a naturalistic real-world setting.  Specifically, the study aimed to analyse the hypothesis that feelings of ``team click'' mediate a relationship between perceptions of joint-action success and social bonding.  The study took place in the context of a two-day National Rugby Sevens Tournament in Qianan, Hebei Province, China July 16-17 2016 (hereafter the Tournament).  Self-report survey measures and data gleaned from official Tournament performance records were collected at various time points before, during, and after the Tournament.  Athletes were asked to respond to survey questions designed to measure perceptions of individual and team performance, feelings concerning team click, social bonding, as well as items designed to measure technical competence, personality type, feelings of exertion and fatigue, and injury status (among other items). When controlling for perceptions of individual performance and objective measures of individual and team success in the Tournament, results revealed significant statistical associations between 1) perceptions of joint action success and feelings of team click, 2) feelings of team click and feelings of social bonding, and 3) perceptions of joint action success and feeling of social bonding.  In addition, a mediation analysis revealed that the relationship between perceptions of joint action success and social bonding was fully mediated by feelings of team click, suggesting that athletes felt more bonded to their teammates when they felt the ``click'' of successful joint action.  A secondary analysis revealed that more positive violations of expectations around team performance also predicts feelings of team click and feelings of social bonding.  These results suggest that violations of expectations around joint action could be an important mechanism in the hypothesised relationship between joint action and social bonding.

%($male = 93, M = 21.67, SD = 3.67, range = 17-32$)
%Data were analysed according to predictions derived from existing theory of the social cognition of joint action and social bonding, and ethnographic analysis presented in Part A.

\section{Introduction}
When athletes coordinate their behaviours they do so by making predictions about the state of the world.  In the case of joint action, in which there are many autonomous moving parts, the accuracy of predictions concerning joint action rely on the behaviours of others for their accuracy.  Annecdote and observation in team sport and other joint activities involving complex realtime coordination of behaviours around technically demanding tasks suggests that the phenomenology associated with co-actors' perception of joint action ``clicking into place'' is an extremely powerful social bonding agent.  Feelings of team click appear to be associated with feelings of extended agency, and may function as a strong signal of co-actor reliability.  Thus, athletes' perception of joint action success, especially the feeling associated with the ``click'' of joint action, could be an important facet of the explanation for why when humans move together, we bond together.

Perceptions of successful synchronisation of behaviour in joint action appears to have positive implications for individual psychophysiological function, health, and subjective well being \citep{Wheatley2012}.  Likewise, there is well-documented evidence of a link between psycho-social isolation and ill-health and developmental and neurocognitive deficits in behaviours key to dynamic interpersonal interaction \citep[e.g.][]{Blakemore2005,Baron-Cohen1991}.

Lit Review:

Joint Action framework:

Experimental evidence from the behavioural synchrony and mimicry literatures suggests that high quality coordination of movement between co-actors in joint action may be a powerful source of positive affect, blurring of self-other agency, pro-sociality, and cooperation \citep{Mogan2017}.

Beyond these literatures, the relationship between less tightly coupled joint action and social bonding is yet to be thoroughly tested.

Anecdotal and observational evidence from anthropology and psychology---particularly the psychology of ``flow'' \citep{Csikszentmihalyi1992,Jackson1999}---suggests that perceptions of joint action success may set the psychological foundation for processes of affiliation and cohesion.  Various neurological, cognitive, and sociological strands of evidence support this proposal.



Closing (predictions & how?)



%In this study, professional rugby players were surveyed before, during, and after a high-intensity, high-stakes National professional rugby tournament.

\subsection{Predictions}
It should be predicted that 1) athletes who perceive greater success in joint action will experience higher levels of felt  ``team click.'' I predict that relevant perceptions of joint action success will relate to athlete perceptions of either 1.a) a combination of specific technical components, 1.b) an overall perception of team performance relative to prior expectations, or 1.c) an interaction between these two dimensions of team performance.
I also predict that 2) athletes who experience higher levels of team click will report higher levels of social bonding. In addition, I predict that 3) more positive perceptions of joint action success will predict higher levels of social bonding, driven by more positive 3.a) perceptions of components of team performance, or 3.b) violation of team performance expectations, or 3.b) an interaction between the two predictors.  Finally, I predict that 4) Team Click will mediate a direct relationship between percpetions of joint action and social bonding.





\section{Method}

\subsection{Participants}
174 Chinese professional adult rugby players from 8 men’s provincial teams and 7 women’s provincial teams were surveyed  (mean athletes per team = 11.6 ($SD =1.06$), $male = 93$, $M(age) = 21.67$ ($SD = 3.67$, $range = 16 - 32$))---once before (2-4 days before, $n = 120$), twice during (once each day of the two day tournament, following the 2nd or 3rd game of each day, $n = 164$), and once after the Tournament ($n = 118$).  The University of Oxford’s Central University Research Ethics Committee approved this study (SAME/CUREC1A/15-059).


\subsection{Materials}

\subsubsection{The Tournament}
``The Chinese National Rugby Sevens Championship'' was held 16-17th July 2016 in Qianan, Hebei Province, China. The Tournament was the most important of the four national-level tournaments held in 2016 because results in the Tournament decided the title of overall men's and women’s national champion. The rugby players surveyed in the Tournament represent the top level of current professional rugby playing athletes in China.

A rugby sevens tournament generally requires two full days to complete, and most teams play an average of 5 or 6 games in total. Within each tournament, participating teams are first divided into smaller pools, and spend the first day of the tournament playing a 14-minute game against every team in their pool. On the second day of the tournament, teams are re-grouped according to Day 1 results, and play in a three-round knock out phase to decide overall placings (usually quarter-, semi- and grand-final structure). On day 2, the top 8 teams of the tournament compete for overall championship in a quarter-semi-grand final format, and any remaining teams compete for the remaining placings below these 8 teams. The playing time for the grand-final is extended to 20 minutes (10 minutes per half, as opposed to the usual 7 minutes).

In this particular Tournament, seven women's teams and eight men's teams participated.  The men’s competition was split into two pools of four teams each, and the women’s competition was split into one pool of four and one pool of three teams (see Table~\ref{tab:poolStructure}). \\


\begin{table}[htpb]\caption{Tournament Pool Structure}
  \begin{center}
    \begin{small}
        \begin{tabular}{| c | c || c | c |}
          \hline
          \bf Women's Pool A & \bf Women's Pool B &  \bf Men's Pool A & \bf  Men's Pool B \\
          \hline
          Jiangsu & Anhui & Shandong & Tianjin \\
          Shandong & Shanghai & Beijing & PLA\superscript{*} \\
          Tianjin & Beijing & Hebei & Anhui \\
             & Fujian & Shanghai & Fujian \\
             \hline
        \end{tabular}
          \end{small}
        \end{center}

        \begin{footnotesize}
          $^*$People's Liberation Army
        \end{footnotesize}

  \label{tab:poolStructure}
    \end{table}



\subsection{Surveys}
Surveys were generated using Qualtrics software (Qualtrics version 9, Provo, UT). Surveys were translated into Chinese and then back translated by two independent native Chinese speaking translators from Beijing Sports University.  Pre- and post-Tournament surveys were administered online using a social networking software called WeChat. WeChat is an online messaging and social networking platform that has become a near-universal means of electronic communication in Mainland China (an English language/Western equivalent would be something of a cross between Facebook and WhatsApp Messenger). The surveys administered before and after the Tournament were completed by athletes within the WeChat application, using their personal mobile phone devices and Internet access. Due to the constraints of the Tournament setting, in particular athletes’ lack of access to mobile phones and Internet immediately following games, meant that hard copy (paper) surveys rather than electronic surveys were administered mid-Tournament.

Surveys were designed according to theoretically and ethnographically derived predictions about athlete experience of high-intensity competition.  In the pre-Tournament, mid-Tournament, and post-Tournament surveys, athletes were asked to report their subjective experience of individual and team performance, overall feeling about the quality of team coordination or \textbf{``team click''}, and feelings of social bonding.  In addition, athletes were asked about feelings of exertion, fatigue, injury, as well as objective and subjective measures of technical competence, perceptions of team discipline, and individual personality type.

%% Summary table of key variables, ned to adapt from Ch6
%<<surveyMeasureSummaryTable, eval=T, echo=F>>=
%  #create all columns:
%  Items <- c("Performance(Ind)", "Performance(Group)", "Performance(Team)",
%             "TeamClick(Group)", "TeamClick(Team)", "SocialBonding(Group)", %"SocialBonding(Team)", "Arousal", "Exertion")
%  Baseline <- c("*","","*","","*","","*","","")
%  Pre <- c("*","*","","*","","*","","*","")
%  Post <- c("*","*","","*","*","*","*","*","*")

%  surveyMeasureSummary <- data.frame(Items, Baseline, Pre, Post, stringsAsFactors = FALSE)

%  summaryTable <- xtable(surveyMeasureSummary,
%                                caption = "Survey items measured at each time point"
%                                label = "tab:surveyItemsByTime")
%  align(summaryTable) <- "llccc"
%  print(summaryTable, file="surveyItemsByTime.tex")
%@

  \subsubsection{\label{Section:Tournament Survey Items}Tournament Survey Items}


    \myparagraph{Athlete perceptions of performance}

Athletes reported their perceptions of two dimensions of performance: 1) components of team and individual performance in rugby sevens, and 2) the overall quality of team and individual performance relative to prior expectations.  \textit{Components of team and individual performance} were selected according to aspects of team and  individual performance commonly scrutinised in the researcher's ethnographic setting (see Chapters 5 and 6).  Components of team performance included 1) coordination of the defensive line, 2) coordination of the attacking line, 3) support play, and 4) on-field communication, whereas Individual components of performance included 1) passing technique, 2) support play in attack, 3) one-on-one defence, 4) effectiveness in contact, and 5) decision making in game play.  In the pre-Tournament survey, athletes were asked about their impression of team or individual performance in the past month, for example: ``how do you feel about your team's coordination of the defensive line over the past month?''  In the post-Tournament survey athletes were asked about their impressions of the same components of performance as they were perceived in the Tournament, e.g. ``How do you feel about your passing technique during the Tournament?'' Athletes responded to each item by moving a toggle left or right from its default centre position on a continuous 100-point scale: 0 - ``Extremely poor'', 100 - ``Extremely good.''  Given the time constraints associated with delivering surveys during the Tournament itself (immediately following individual games), athlete perceptions of components of team and individual performance were only included in the pre- and post-Tournament surveys.

Items designed to measure athlete perceptions of overall quality of team and individual performance \textit{relative to prior expectations} were included each mid-Tournament survey, and in the post-Tournament survey.  In the case of the mid-Tournament survey, for example, athletes were asked: ``Overall, how do you feel about your individual performance/the performance of the team in this game?'' (100 point continuous scale centred at zero: -50 = ``Much worse than expected'', 0 =  ``As expected'', 50 =  ``Much better than expected''). In the post-Tournament survey, athletes answered the same questions about individual and team performance, but in relation to the Tournament as a whole. In the pre-Tournament survey, athletes answered questions relating to overall team and individual performance in relation to the month of training and competition prior to date of the survey (``Overall, how well do you feel you/your team has been performing in training and competition over the past month?'').  Without a specific or immediate focus for perceptions of overall performance, it was unnatural to frame these items in terms of prior expectations.  Instead, overall performance was rated on a continuous scale (0 = ``Extremely poor,'' 100 = ``Extremely well'').  While these items did not provide a measure of performance in relation to prior expectations \textit{per se}, they provided a baseline control measure of attitudes towards performance.  In addition, Athletes were also asked to report the extent to which 1) the quality of recent individual performances influences their mood and 2) the extent to which recent performance influences their confidence for future performance (see appendix X for a full description).


  \myparagraph{Team Click}
Athletes were asked about feelings relating to the phenomenon of ``team click.'' Survey items pertaining to team click were generated by utilising phrasing commonly encountered by the researcher during ethnographic observations of the Beijing provincial team and other Chinese provincial teams, and further afield (see Chapter 6 for a full explanation).  Team click items included: 1) the feeling of ``unspoken understanding'' (\textit{moqi}) between team mates (Unspoken Understanding), 2) the ``general atmosphere'' of the team (General Atmosphere), 3) the reliability of athletes' teammates to perform their roles on the field (Reliability of Others), 4) the reliability of individual athletes to perform their own on-field roles for teammates (Reliability For Others), and 5) the extent to which athletes felt that their individual abilities were extended by their team mates (Ability Extended).  Athletes responded to each item by moving a toggle left or right from its default centre position on a continuous 100-point scale (0 = ``Extremely weak,'' 100 = ``Extremely strong''). In addition to these items, athletes also responded to a novel visual item with five responses, ranging from less to more coordinated arrangements of dots, designed to represent the coordination of the team on the field (Click Pictorial, see Appendix ~\ref{app5:tournamentSurvey} Section ~\ref{app5:clickPre} for a detailed explanation of all pre-Tournament survey items relevant to team click).  Given time constraints, only three items pertaining to team click were included in the mid-Tournament survey(``unspoken understanding,'' ``general atmosphere,'' and the ``Click Pictorial'' measure).

  \myparagraph{Social Bonding}
Athletes answered items relating to feelings of social bonding to their team and teammates.  Athletes were asked to report the extent to which they felt 1) emotional support from teammates (``Emotional Support''), and 2) a sense of shared goal with their team mates (``Shared Goal''). Both items used 100-point continuous scales, 0 - ``Extremely weak'', 100 - ``Extremely strong'').  In addition, Athletes completed two pre-validated multi-item scales designed to measure 1) an individual's personal identification with the stereotypical features of the in-group \citep[Group Identification Verbal Scale, see][]{Mael1992} and 2) an individual's fusion or ``feeling of oneness with the group'' \citep[Identity Fusion Verbal Scale, see][]{Swann2009}.
A visual scale designed to measure construct of Identity Fusion to the target in-group \citep[Identity Fusion Pictorial][]{Swann2009} was also included, for three targets: the athletes' team, family, and nationality (``China'').  Finally,  Athletes ranked their level of Identity Fusion to team, family, and country \citep[Identity Fusion Pictorial Rank, replicating][]{Whitehouse2014}.  See Appendix ~\ref{app5:tournamentSurvey} Section ~\ref{app5:bondingPre} for a detailed explanation of all pre-Tournament items relevant to social bonding.  Given time constraints, only three items pertaining to social bonding were included in the mid-Tournament survey (Emotional Support, Shared Goal, and the ``Identity Fusion Pictorial (team)'' measure).

  \myparagraph{Moderator Variables}
\textit{Fatigue and Exertion:} Following each mid-Tournament survey and in the post-Tournament survey, athletes were asked about their mood (including dimensions of  of arousal, excitement, and nervousness), feelings of fatigue (``How fatigued do you feel as a result of the game/tournament?''), perceived physical exertion (Borg RPE scale, \citep{Borg1990}) and perceived mental exertion \citep[see][ ]{Noakes2012a}.  \textit{Technical Competence:} Athletes were asked to report their perceptions of individual technical competence relative to 1) other teammates, 2 other current professional Chinese rugby players, and  3) Professional rugby players form other countries.  Athletes also provided information on rugby-related attributes, which also provided a more objective indicator of technical competence.  These measures included: 1) rugby training age (number of years of experience training for rugby, to the nearest number of years), 2) the number of years spent training with the provincial teams (to the nearest year) and 3) whether the athlete is a usual member of the provincial program's starting team or the reserves.  See Appendix ~\ref{app5:tournamentSurvey} Section ~\ref{app5:technicalCompetence} for a detailed explanation of all items relating to technical competence.  \textit{Personality:} Due to the hypothesised links between personality type and dispositional tendencies towards interpersonal coordination strategies \citep[see for example][]{Marsh2009}, and inspired by the prevalence of athlete testimonies in which coordination problems were explained with reference to individual differences (such as personality),  a ten-item personality measure was included in the pre-Tournament survey (Ten Item Personality Index - TIPI)\citep{Gosling2003}. See Appendix ~\ref{app5:tournamentSurvey} Section ~\ref{app5:TIPI} for a detailed explanation.  Moderator variables were included only in the pre-Tournament survey, and were not repeated in mid- or post-Tournament surveys.

  \myparagraph{Additional Items}
In addition to performance, team click, social bonding, and moderator variables outlined above (technical competence and personality), athletes were also asked (in the pre-Tournament survey) about their perceptions of components of team discipline, their current injury status (all surveys), and other basic identification variables (athlete name, date of birth, team membership, etc).  See Table ~\ref{tab:surveyVariablesIncluded} for a summary of survey items included in each survey.  For a detailed description of additional items included in the pre-Tournament survey, see Appendix ~\ref{app5:tournamentSurvey} Section ~\ref{app5:additionalItems}.


  \begin{table}[htpb]\caption{Tournament Survey Items}
    \begin{center}
      \begin{small}
          \begin{tabular}{ r | c  c  c  c }
            \bf Variable & \bf Pre &  \bf Day 1 & \bf Day 2 & \bf  Post \\
            \hline
            \bf Performance & & & & \\
            Joint Action Success & \cmark &   &   & \cmark\\
            Ind Performance Success & \cmark &   &   & \cmark \\
            Team Performance Exp &  & \cmark & \cmark & \cmark \\
            Ind Performance Exp &  & \cmark & \cmark & \cmark \\
             \bf Team Click & & & & \\
            6 item factor & \cmark &   &   & \cmark \\
            3 item factor & \cmark & \cmark & \cmark & \cmark \\
             \bf Social Bonding & & & & \\
            6 item factor & \cmark &   &   & \cmark \\
            3 item factor & \cmark & \cmark & \cmark & \cmark \\
             \bf Moderator variables & & & & \\
            Fatigue &   & \cmark & \cmark & \cmark \\
            Injury & \cmark  & \cmark & \cmark & \cmark \\
            Objective Competence & \cmark &   &   &   \\
            Subjective Competence & \cmark &   &   &   \\
            Personality Type & \cmark &   &   &   \\
            \bf ID Variables & \cmark &   &   &   \\
        \end{tabular}
      \end{small}
    \end{center}
      \label{tab:surveyVariablesIncluded}
  \end{table}


\subsubsection{\label{app5:objectivePerformance}Objective Performance Measures}
Following the completion of the Tournament, CRFA provided the researcher with performance data in electronic format. These data included results for each game, minutes played and points scored by individual athletes in each game, substitutions made during each game, and video footage of every game played during the Tournament.  Based on the data provided, a number of objective performance variables were created for use as statistical controls, including the final rank of each team in their respective competition (men's and women's, reverse coded so that the top ranked team was awarded the highest value), a team's total number of wins minus total number of losses, each individual athlete's total number of minutes played throughout the course of the Tournament, each individual athlete's total number of points scored in the Tournament, the average number of times an individual athlete was part of the starting team throughout the Tournament. For full details on these objective performance measures, see Appendix ~\ref{app5:tournamentSurvey} Section ~\ref{app6:objectivePerformance}.




\subsection{Procedure}

\subsubsection{Pre-Tournament Survey}
Two months prior to the Tournament, I contacted the head coach of each provincial team and officials from the Chinese Rugby Football Association (CRFA) to seek permission for the study.  After receiving the permission from all participating teams and CRFA, five days prior to the Tournament I asked the coach or manager of each team to create a virtual message group on WeChat containing only athletes who would be participating in the Tournament. The WeChat group could be accessed by the athletes on their personal mobile phone devices with Internet connection. The WeChat group was populated by the coach/manager of the team, the athletes competing in the Tournament, and the researcher. Once the WeChat group was set up, I posted a standard message in each group in which I introduced the study and provided the link to the Qualtrics survey for the athletes to complete in their own time (for English and Chinese versions of the script, see Appendix ~\ref{app5:tournamentSurvey} Section ~\ref{app5:studyIntro}).

Upon opening the link to the survey, Athletes read a detailed brief about the survey, provided consent, and demonstrated their ability to answer the survey questions by changing the position of a virtual sliding bar toggle that would feature in many of the survey questions.  Athletes were then asked a number of questions grouped by the following categories: perceptions of individual and team performance, feelings about the overall quality of recent team coordination (team click), social bonding, technical competence, and personality type. The order in which each item appeared within these categories was randomised for each survey participant. At the end of the survey, athletes were asked to provide basic identification variables such as age, sex, team, position, and injury status.  The pre-Tournament survey took approximately 15 minutes to complete.  Once the data collection window for the pre-Tournament survey had ceased, survey responses were collated in Qualtrics and then imported into RStudio (Version 1.0.136) for cleaning and statistical analysis.

\subsubsection{Mid-Tournament Surveys}
Copies of the mid-Tournament survey were printed on A4 paper for athletes to complete with a pen or pencil within 30 minutes of completion of the second or third game of each day. All surveys were administered and collected by the researcher, occasionally with some minimal assistance from team staff who handed out surveys or pens. After receiving permission from the team coach or manager, I approached each team approximately 10-20 minutes following the completion of the game, and administered a hard copy of the mid-Tournament survey to each athlete.  Data collection occurred on the side of the Tournament field after athletes had completed their cool-down routines.  The mid-Tournament took approximately 3 minutes to complete. Completed surveys were collected by the researcher and sealed in envelopes labelled by team. Athletes were surveyed following the second game of each day (or in the case of two of the teams, following the third game of Day 1, and the second game of Day 2).  Survey responses were later manually collated and data were imputed into a .csv file using Microsoft Excel (Version 14.7.1). Collated data were then combined with other survey and performance data to be analysed in RStudio.
\subsubsection{Post-Tournament Survey}
The post-Tournament survey was administered via the same WeChat group that was set up for the pre-Tournament survey. Data collection for the post-Tournament survey began the day after the completion of the Tournament, and finished four days later. Athletes were asked to respond to survey items framed in terms of their experience of the Tournament. Survey responses were collated in Qualtrics and then imported into RStudio, where they were cleaned and combined with pre-Tournament and mid-Tournament survey responses for statistical analysis.
\subsubsection{Tournament performance data}
Following the completion of the Tournament, game-by-game performance data were collected from the CRFA Tournament statistician. Data were stored on an encrypted external hard disk. These data were later manually imputed into a data frame in Microsoft Excel, before being imported into RStudio to be merged with other survey data (pre-tournament survey, mid-Tournament surveys, post-Tournament surveys, and post-Tournament survey) for statistical analysis.


















\section{Results}



\subsection{Descriptive Statistics}

  \subsubsection{Participants}

Data were collected for a total of 174 adult rugby playing athletes ($male = 93, M = 21.67, SD = 3.67, range = 17-32$) a total of eight time points. Tournament performance data were collected after each of the six games for all 174 athletes who participated in the Tournament;  Survey responses were recorded for a total of 165 unique athletes at four different time points: once pre-Tournament, twice during the Tournament, and once post-Tournament (see Table ~\ref{tab:tournamentData}).

      \begin{table}[htbp]\caption{Data collected during the Tournament}
        \begin{center}
          \begin{small}
            \begin{tabular}{c|c c c}
                \bf Time & \bf Phase & \bf Survey Data (Men) & \bf Performance Data (Men) \\
                \hline
                1 & pre-Tournament & \bf 120 (68) & - \\
                & & & \\
                & \multicolumn{1}{l}{Day 1} & & \\
                2 & Game 1 & - & 174 (93) \\
                3 & Game 2 & \bf129 (60) & 174 (93) \\
                4 & Game 3 & \bf22 (8) & 174 (93) \\
                & \multicolumn{1}{l}{Day 2} & & \\
                5 & Game 4 & - & 174 (93) \\
                6 & Game 5 & \bf 163 (91) & 174 (93) \\
                7 & Game 6 & - & 174 (93) \\
                & & & \\
                8 & post-Tournament & \bf118 (65) & - \\
            \end{tabular}
          \end{small}
        \end{center}
        \label{tab:tournamentData}
      \end{table}

120 of a total of 174 athletes competing in the Tournament ($male = 68$) were surveyed during a four day period before the Tournament, which represented 69\% of the total sample. On Day 1 of the Tournament, a total of 151 athletes (87\% of sample, male = 68) were surveyed: 129 athletes (male = 60) in 11 teams were surveyed after their 2nd game of the day, and 22 athletes (male = 8) in 2 teams were surveyed after their 3rd game. 2 of 11 teams (Hebei men’s and Fujian men’s) were not surveyed due to timing and logistical constraints experienced by the researcher during data collection on Day 1. On Day 2 of the Tournament, a total of 163 athletes (94\% of sample, male = 91) in 14 teams were surveyed after their second game of the day. One team (Shanghai Women’s) was not surveyed due to timing and logistical constraints experienced by the researcher. A total of 100 athletes (57\% of the sample, male = 59) completed both the pre- and post-Tournament surveys, and a total of 99 athletes completed all four surveys (57\% of the sample, male = 59). Logistical challenges relating to data collection meant that observations were missing for athletes across the four survey time points. Missingness in the survey data ranged from 15-19\% at any one of the four survey time points.\\



%\subsubsection{Overall Tournament Descriptive Statistics}
Tables~\ref{tab:performanceOverallSummary}\nobreakdash--\ref{tab:fatigueOverallSummary} display basic summary statistics (mean and standard deviation) for variables of interest at each recorded time point. The central tendency of variables within the categories of performance, team click, social bonding, and fatigue and exertion are above the mid-point of each scale. In addition, mid-Tournament measurements tend to be lower than pre- or post-Tournament measures for each category. In relation to performance measures, for example, athletes appear on average to be more critical of their own and their team’s performance (relative to prior expectations) when surveyed immediately after games on day 1 and 2 than they were following the completion of the Tournament (note, however, that survey items relating to individual and team performance administered pre-Tournament were not posed in relation to athlete expectations, and thus could not be directly compared to subsequent mid- and post-Tournament measures). The same pattern was identifiable in team click variables, with the central tendency of mid-Tournament measures of Unspoken Understanding and General Atmosphere 10-15\% lower than pre- or post-Tournament measures. This is also the case for variables related to social bonding: Emotional Support and Shared Goal in particular showing a steep increase form mid-Tournament measurements to the post-Tournament measurement. The same pattern was identifiable for variables relevant to fatigue (see Table ~\ref{tab:fatigueOverallSummary}).
Pre-, mid-, and post-Tournament survey items are reviewed in more detail in Appendix ~\ref{app5:tournamentSurvey}, Sections ~\ref{app6:descriptivesPre}\nobreakdash~\ref{app6:descriptivesPost}.


% Table created by stargazer v.5.2 by Marek Hlavac, Harvard University. E-mail: hlavac at fas.harvard.edu
% Date and time: Mon, Jun 26, 2017 - 09:48:21
\begin{table}[!htbp] \centering 
  \caption{Overall Tournament Performance Summary Statistics} 
  \label{tab:performanceOverallSummary} 
\scriptsize 
\begin{tabular}{@{\extracolsep{5pt}} ccccccccc} 
\\[-1.8ex]\hline 
\hline \\[-1.8ex] 
time & teamPerf & tP.sd & indPerf & iP.sd & teamPerfExp & tPE.sd & indPerfExp & iP.sd.1 \\ 
\hline \\[-1.8ex] 
pre-Tournament & $71.11$ & $21.87$ & $68.92$ & $21.32$ & $$ & $$ & $$ & $$ \\ 
day1 & $$ & $$ & $$ & $$ & $54.43$ & $30.63$ & $39.43$ & $27.25$ \\ 
day2 & $$ & $$ & $$ & $$ & $53.19$ & $32.98$ & $40.92$ & $27.93$ \\ 
post-Tournament & $$ & $$ & $$ & $$ & $64.36$ & $23.61$ & $56.36$ & $23.47$ \\ 
\hline \\[-1.8ex] 
\end{tabular} 
\end{table} 


% Table created by stargazer v.5.2 by Marek Hlavac, Harvard University. E-mail: hlavac at fas.harvard.edu
% Date and time: Mon, Jun 26, 2017 - 09:21:29
\begin{table}[!htbp] \centering 
  \caption{Team Click Overall Tournament Summary Statistics} 
  \label{tab:clickOverallSummary} 
\scriptsize 
\begin{tabular}{@{\extracolsep{5pt}} ccccccc} 
\\[-1.8ex]\hline 
\hline \\[-1.8ex] 
time & unspUnd & uU.sd & genAt & gA.sd & clickPic & cP.sd \\ 
\hline \\[-1.8ex] 
pre-Tournament & $71.58$ & $20.77$ & $75.51$ & $23.27$ & $3.87$ & $1.24$ \\ 
day1 & $55.92$ & $26.88$ & $65.74$ & $31.95$ & $3.46$ & $1.49$ \\ 
day2 & $55.30$ & $29.43$ & $64.32$ & $33.39$ & $3.33$ & $1.70$ \\ 
post-Tournament & $72.72$ & $19.95$ & $78.45$ & $21.34$ & $3.93$ & $1.04$ \\ 
\hline \\[-1.8ex] 
\end{tabular} 
\end{table} 


% Table created by stargazer v.5.2 by Marek Hlavac, Harvard University. E-mail: hlavac at fas.harvard.edu
% Date and time: Mon, Jun 26, 2017 - 09:22:14
\begin{table}[!htbp] \centering 
  \caption{Social Bonding Overall Tournament Summary Statistics} 
  \label{tab:bondingOverallSummary} 
\scriptsize 
\begin{tabular}{@{\extracolsep{5pt}} ccccccc} 
\\[-1.8ex]\hline 
\hline \\[-1.8ex] 
time & emoSup & eS.sd & sharedGoal & sG.sd & fusionPic & fP.sd \\ 
\hline \\[-1.8ex] 
pre-Tournament & $70.12$ & $26.21$ & $77.66$ & $24.28$ & $4.26$ & $1.25$ \\ 
day1 & $67.29$ & $30.56$ & $76.34$ & $30.50$ & $4.06$ & $1.47$ \\ 
day2 & $67.53$ & $32.55$ & $71.42$ & $35.47$ & $3.85$ & $1.69$ \\ 
post-Tournament & $79.67$ & $18.84$ & $86$ & $15.56$ & $4.33$ & $1.19$ \\ 
\hline \\[-1.8ex] 
\end{tabular} 
\end{table} 


% Table created by stargazer v.5.2 by Marek Hlavac, Harvard University. E-mail: hlavac at fas.harvard.edu
% Date and time: Mon, Jun 26, 2017 - 09:23:43
\begin{table}[!htbp] \centering 
  \caption{Fatigue Overall Tournament Summary Statistics} 
  \label{tab:fatigueOverallSummary} 
\scriptsize 
\begin{tabular}{@{\extracolsep{5pt}} ccccccccc} 
\\[-1.8ex]\hline 
\hline \\[-1.8ex] 
time & fat & f.sd & prpe & p.sd & mental & m.sd & inj.mu & inj.sd \\ 
\hline \\[-1.8ex] 
pre-Tournament & $$ & $$ & $$ & $$ & $$ & $$ & $18.73$ & $23.36$ \\ 
day1 & $50.14$ & $28.80$ & $12.45$ & $4.52$ & $4.09$ & $2.83$ & $29.91$ & $33.15$ \\ 
day2 & $53.65$ & $31.03$ & $12.60$ & $5.53$ & $4.13$ & $3.17$ & $37.14$ & $37.66$ \\ 
post-Tournament & $69.27$ & $21.24$ & $14.97$ & $2.66$ & $6.08$ & $2.47$ & $23.86$ & $26.91$ \\ 
\hline \\[-1.8ex] 
\end{tabular} 
\end{table} 



\subsubsection{Pre-post Tournament differences in variables of interest}
Pre- and post-Tournament measures of variables of interest are displayed in tables below.  Three out of five variables designed to measure athletes' perceived success in components of individual performance appeared to drop from pre- to post-Tournament (see Table ~\ref{tab:indPerformancePrePostSummary}).  Paired samples t-tests revealed significant negative mean differences between pre- and post-Tournament measures of passing technique ($M = -7.48 (-13.17, -1.80)$, $t(98)= -2.61,, p = .01$), support play ($M = -7.32 (-12.18, -2.47)$, $t(98)= -2.99,, p = .003$), and decision making in attack ($M = -5.19 ( -9.73, -0.65)$, $t(98)= -2.27,, p = .03$), while individual defence ($M = -3.42 (-9.01, 2.16)$, $t(98)= -1.22,, p = .23$), and effectiveness in contact ($M = -4.51 (-10.25, 1.24)$, $t(98)= -1.56,, p = .12$) did not decrease significantly between pre- and post-Tournament measures.

By contrast, all four variables designed to measure team success in components of joint action did not differ significantly between pre- and post-Tournament measurements (see Table ~\ref{tab:teamPerformancePrePostSummary} for results).  Similarly, variables designed to measure team click also did not vary significantly.  These results suggest that athlete perceptions of team joint action success and team click on average remained stable throughout the Tournament.  Perceptions of success in individual component performance, by contrast, decreased significantly, indicating that following the Tournament athletes were on average more critical of their performance following the Tournament.

In terms of Social Bonding, both feelings of emotional support and feelings of shared goal with the team differed significantly between pre- and post-Tournament measurements (see Table ~\ref{tab:bondingPrePostSummary}). Both measures increased significantly in the post-Tournament measure (emotional support: $M = 9.30 (4.16, 14.45)$, $t(98)= 3.59,, p = .0005$; shared goal $M = 7.12 (3.06, 11.19)$, $t(98)= 3.48,, p = .0007$).  All other variables did not vary significantly from pre-Tournament measures, except for the pictorial measure of fusion to \textit{family}, which increased significantly, $M = .33 (0.03, 0.64)$, $t(98)= 2.18, p = .03$.  Bonding variables for which measures were 5-point Likert may have suffered from ceiling effects: the mean and median of each Fusion Pictorial scale was $> 4$.  In general, it appeared that bonding to the team (and to one measure of Family) increased when measured following the Tournament.

In addition, variables designed to measure feelings of exertion and fatigue also varied significantly between pre- and post-Tournament measurements.  Fatigue ($M = 21.61 (15.02 28.20)$, $t(116)= 6.49,, p < .0001$), physical perceived exertion ($M = 2.84 (1.89, 3.79)$, $t(116)= 5.93, p < .0001$), and mental perceived exertion ($M = 2.84 (1.89, 3.79)$, $t(116)= 5.93, p < .0001$) all increased significantly from the point of first measurement immediately following the 2nd game on day 1 of the Tournament (see Table ~\ref{tab:fatiguePrePostSummary}).  Injury status post-Tournament did not differ significantly from pre-Tournament measures when tested using a paired-samples t-test, although the overall group mean difference between pre- and post-Tournament injury did show marginally significant signs of increasing, $M = 2.84 (-10.63, 0.59)$, $t(98)= 1.78, p = .08$.
These results indicate that on average, athletes felt higher levels of exertion and fatigue after the Tournament, than they did before or after the first game of the Tournament. This is to be expected, given the intensity of the rugby sevens Tournament format. It provides confirmation for the strenuous nature of the activity of the present study.





\clearpage


\newgeometry{margin=0.5cm} % modify this if you need even more space
\begin{landscape}

% Table created by stargazer v.5.2 by Marek Hlavac, Harvard University. E-mail: hlavac at fas.harvard.edu
% Date and time: Sun, Jun 25, 2017 - 22:44:09
\begin{table}[!htbp] \centering 
  \caption{Individual Performance change pre-post Tournament} 
  \label{tab:indPerformancePrePostSummary} 
\footnotesize 
\begin{tabular}{@{\extracolsep{5pt}} cccccccccc} 
\\[-1.8ex]\hline 
\hline \\[-1.8ex] 
 & Variable & Mean.pre & SD.pre & Mean.post & SD.post & MeanDiffernece & MeanPairedDifference & t-test & p-value \\ 
\hline \\[-1.8ex] 
1 & passingTechnique & $65.74$ & $24.94$ & $58.41$ & $24.25$ & $$-$7.33$ & $$-$7.48$ & $$-$2.61$ & $0.01$ \\ 
2 & supportPlay & $69.54$ & $23.23$ & $62.62$ & $22.70$ & $$-$6.92$ & $$-$7.32$ & $$-$2.99$ & $0.003$ \\ 
3 & individualDefense & $59.22$ & $26.64$ & $57.64$ & $23.57$ & $$-$1.57$ & $$-$3.42$ & $$-$1.22$ & $0.23$ \\ 
4 & effectivenessContact & $65.66$ & $24.85$ & $62.15$ & $24.81$ & $$-$3.51$ & $$-$4.51$ & $$-$1.56$ & $0.12$ \\ 
5 & decisionMakingAttack & $65.48$ & $23.80$ & $61.22$ & $21.43$ & $$-$4.26$ & $$-$5.19$ & $$-$2.27$ & $0.03$ \\ 
\hline \\[-1.8ex] 
\end{tabular} 
\end{table} 


% Table created by stargazer v.5.2 by Marek Hlavac, Harvard University. E-mail: hlavac at fas.harvard.edu
% Date and time: Sun, Jun 25, 2017 - 22:47:06
\begin{table}[!htbp] \centering 
  \caption{Team Performance change pre-post Tournament} 
  \label{tab:teamPerformancePrePostSummary} 
\footnotesize 
\begin{tabular}{@{\extracolsep{5pt}} cccccccccc} 
\\[-1.8ex]\hline 
\hline \\[-1.8ex] 
 & Variable & Mean.pre & SD.pre & Mean.post & SD.post & MeanDiffernece & MeanPairedDifference & t-test & p-value \\ 
\hline \\[-1.8ex] 
1 & teamDefense & $65.06$ & $22.85$ & $62.42$ & $22.50$ & $$-$2.63$ & $$-$4.92$ & $$-$1.57$ & $0.12$ \\ 
2 & teamAttack & $66.01$ & $24.95$ & $65.33$ & $20.26$ & $$-$0.68$ & $$-$1.59$ & $$-$0.55$ & $0.58$ \\ 
3 & teamSupportPlay & $64.36$ & $23.53$ & $65.75$ & $19.72$ & $1.40$ & $0.90$ & $0.35$ & $0.73$ \\ 
4 & teamCommunication & $62.37$ & $24.21$ & $65.25$ & $21.26$ & $2.89$ & $$-$0.07$ & $$-$0.03$ & $0.98$ \\ 
\hline \\[-1.8ex] 
\end{tabular} 
\end{table} 


% Table created by stargazer v.5.2 by Marek Hlavac, Harvard University. E-mail: hlavac at fas.harvard.edu
% Date and time: Sat, Jun 03, 2017 - 15:44:29
\begin{table}[!htbp] \centering 
  \caption{click change pre-post Tournament} 
  \label{} 
\footnotesize 
\begin{tabular}{@{\extracolsep{5pt}} cccccccccc} 
\\[-1.8ex]\hline 
\hline \\[-1.8ex] 
 & Variable & Mean.pre & SD.pre & Mean.post & SD.post & MeanDiffernece & MeanPairedDifference & t-test & p-value \\ 
\hline \\[-1.8ex] 
1 & unspokenUnderstanding & $71.58$ & $20.77$ & $72.72$ & $19.95$ & $1.14$ & $1.33$ & $0.58$ & $0.56$ \\ 
2 & generalAtmosphere & $75.51$ & $23.27$ & $78.45$ & $21.34$ & $2.94$ & $1.05$ & $0.47$ & $0.64$ \\ 
3 & clickPictorial & $3.87$ & $1.24$ & $3.93$ & $1.04$ & $0.07$ & $0.03$ & $0.21$ & $0.84$ \\ 
4 & reliabilityOfOthers & $67.43$ & $28.01$ & $68$ & $23.09$ & $0.57$ & $1.21$ & $0.40$ & $0.69$ \\ 
5 & reliabilityForOthers & $62.38$ & $25.40$ & $63.45$ & $25.80$ & $1.07$ & $1.06$ & $0.38$ & $0.70$ \\ 
6 & abilityExtended & $70.45$ & $25.83$ & $72.25$ & $19.27$ & $1.80$ & $1.43$ & $0.53$ & $0.60$ \\ 
\hline \\[-1.8ex] 
\end{tabular} 
\end{table} 


% Table created by stargazer v.5.2 by Marek Hlavac, Harvard University. E-mail: hlavac at fas.harvard.edu
% Date and time: Sun, Jun 25, 2017 - 22:51:01
\begin{table}[!htbp] \centering 
  \caption{bonding change pre-post Tournament} 
  \label{tab:bondingPrePostSummary} 
\footnotesize 
\begin{tabular}{@{\extracolsep{5pt}} cccccccccc} 
\\[-1.8ex]\hline 
\hline \\[-1.8ex] 
 & Variable & Mean.pre & SD.pre & Mean.post & SD.post & MeanDiffernece & MeanPairedDifference & t-test & p-value \\ 
\hline \\[-1.8ex] 
1 & emotionalSupport & $70.12$ & $26.21$ & $79.67$ & $18.84$ & $9.55$ & $9.30$ & $3.59$ & $0.001$ \\ 
2 & sharedGoal & $77.66$ & $24.28$ & $86$ & $15.56$ & $8.34$ & $7.12$ & $3.48$ & $0.001$ \\ 
3 & groupId & $4.34$ & $0.78$ & $4.29$ & $0.67$ & $$-$0.04$ & $0.06$ & $0.78$ & $0.44$ \\ 
4 & fusionVerbal & $4.03$ & $0.74$ & $4.00$ & $0.71$ & $$-$0.03$ & $0$ & $0$ & $1$ \\ 
5 & fusionPictorialTeam & $4.26$ & $1.25$ & $4.33$ & $1.19$ & $0.07$ & $0.04$ & $0.30$ & $0.77$ \\ 
6 & fusionPictorialCountry & $4.01$ & $1.58$ & $4.03$ & $1.39$ & $0.02$ & $$-$0.04$ & $$-$0.22$ & $0.83$ \\ 
7 & fusionPictorialFamily & $4.10$ & $1.50$ & $4.51$ & $0.96$ & $0.41$ & $0.33$ & $2.18$ & $0.03$ \\ 
8 & rankTeam & $1.49$ & $0.79$ & $1.61$ & $0.81$ & $0.12$ & $0.09$ & $1.15$ & $0.25$ \\ 
9 & rankCountry & $1.92$ & $1.06$ & $1.93$ & $1.03$ & $$ & $$-$0.04$ & $$-$0.44$ & $0.66$ \\ 
10 & rankFamily & $1.41$ & $0.88$ & $1.42$ & $0.86$ & $0.01$ & $0$ & $0$ & $1$ \\ 
\hline \\[-1.8ex] 
\end{tabular} 
\end{table} 


% Table created by stargazer v.5.2 by Marek Hlavac, Harvard University. E-mail: hlavac at fas.harvard.edu
% Date and time: Sun, Jun 25, 2017 - 23:08:36
\begin{table}[!htbp] \centering 
  \caption{Fatigue change pre-post Tournament} 
  \label{fatiguePrePostSummary} 
\footnotesize 
\begin{tabular}{@{\extracolsep{5pt}} cccccccccc} 
\\[-1.8ex]\hline 
\hline \\[-1.8ex] 
 & Variable & Mean.pre & SD.pre & Mean.post & SD.post & MeanDiffernece & MeanPairedDifference & t-test & p-value \\ 
\hline \\[-1.8ex] 
1 & fatigue & $50.14$ & $28.80$ & $69.27$ & $21.24$ & $19.13$ & $21.61$ & $6.49$ & $0$ \\ 
2 & prpe & $12.45$ & $4.52$ & $14.97$ & $2.66$ & $2.52$ & $2.84$ & $5.93$ & $0.0000$ \\ 
3 & mental & $4.09$ & $2.83$ & $6.08$ & $2.47$ & $1.99$ & $2.30$ & $6.85$ & $0$ \\ 
4 & injury & $81.28$ & $23.36$ & $76.14$ & $26.91$ & $$-$5.13$ & $5.02$ & $1.78$ & $0.08$ \\ 
\hline \\[-1.8ex] 
\end{tabular} 
\end{table} 


\end{landscape}
\restoregeometry






\subsection{\label{Ch5:dataReduction}Data Reduction}
Data reduction was required in order to make analysis of predicted relationships more tractable and parsimonious. Data reduction allows for a reduction in multicolinearity between predictor variables of interest while retaining as much variance as possible in the observed data \citep{Yong2013}.  Given that the survey items of this study were designed to collectively access more latent psychological constructs, particularly in the case of outcome variables related to concepts such as team click, social bonding, and fatigue (but also for performance variables), a data reduction technique capable of modelling the theoretical structure of these data was preferred over a procedure that merely reduced the collected data to its common variance.  Exploratory Factor Analysis (EFA) was thus chosen as the most suitable data reduction technique over Principal Component Analysis (for a full explanation of EFA, and a more detailed justification for EFA over other available techniques, see Appendix ~\ref{app5:tournamentSurvey} Section ~\ref{app5:EFA}).

For each subset of the data (post-Tournament, pre- to post-Tournament change, and overall Tournament), an EFAs were performed in order to reduce data to key variables of interest: components of team and individual performance, team click, social bonding, fatigue, and technical competence (objective and subjective measures). Prior to factors being extracted, correlation matrices of each group of variables were subjected to two common sampling adequacy measures: the Kaiser-Meyer-Olkin (KMO) index and Bartlett’s test of sphericity.  Factor loadings of $> .3$ were considered adequate, and only items that loaded on one factor were accepted \citep{Field2012}. Sum of Squares Loadings (SS Loadings) for each factor were also reported\citep{Dziuban1974}.  Finally, two reliability measures (Guttman's lambda3 and Cronbach's Alpha) were also reported as an indication of whether or not the average correlation of each subset of variables is an accurate estimate of the average correlation of all items that could pertain to the underlying construct.  Factor scores were calculated as standardised z-scores: zero-centred, with a standard deviation of 1.



\subsubsection{Joint Action Success and Individual Performance Success}
Items related to athlete perception of components of performance were isolated from overall perceptions of performance relative to prior individual expectations for further data reduction. Given that the theoretical predictions of this dissertation concentrate in particular on athlete perceptions of \textit{joint action} (and not simply individual action), perceptions of individual and team components of performance were analysed separately. This separation allowed for the testing the differential effects of perceptions of individual and team performance on team click and social bonding. In addition, perceptions of success in individual performance components were used as a statistical control for perceptions of success in joint action.

\myparagraph{Post-Tournament}
Items concerning team components of performance (team defence, team attack, team support play, and on field communication), and individual components of performance (passing technique, support play in attack, 1on1 defence, and decision making in attack)) were subjected to EFAs (with oblique ``promax'' rotation).  Correlations between team component performance items was very high (all $r's > .5$), which suggested that one factor would be appropriate (see Appendix ~\ref{app5:tournamentSurvey} Table ~\ref{tab:22teamPerformancePostCorr}). The KMO index and Bartlett's test both suggested high sampling adequacy, ($KMO = 0.79$, $\chi^2(6, N = 118) = 342.14$, $p < .001$).  One factor, labelled ``Joint Action Success'' was imposed on the data, which explained 72.8\% of the overall variance (SS Loading = 2.91). $Guttman's \lambda =.90$ and $Cronbach's\alpha = .91$ indicated that the data reduction was appropriate and reliable. Items relating to individual component performance  were subjected to an EFA.  Correlations between individual component performance items were also very high (all $r's > .5$, see Appendix ~\ref{app5:tournamentSurvey} Table ~\ref{tab:21indPerformancePostCorr}), which suggested that one factor would be sufficient (confirmed by sampling adequacy tests, $KMO =  0.84$, $\chi^2(10, N = 118) =  326.38$, $p < .001$).  One factor was extracted and labelled ``Individual Performance Success'', which explained 62.1\% of the overall variance (SS Loading = 3.10).
$Guttman's\lambda =.88$ and $Cronbach's \alpha = .89$ both indicated that the data reduction was appropriate.  To confirm that the theoretically motivated separation of Joint Action Success from Individual Performance Success was appropriate for the collected data, a follow up EFA was conducted, in which team and individual performance component variables were combined in one matrix (see Appendix ~\ref{app5:tournamentSurvey} Section ~\ref{app5:performanceDataReduction} for full details).
%$\chi^2 (df=5) = 15.47$, $p < .01$

\myparagraph{Pre- to post-Tournament}
Correlations between team component performance items were very high (all $r's > .65$), and the suitability of the correlation matrix for factor extraction was confirmed by sampling adequacy tests ($KMO = 0.82$, $\chi^2(6, N = 238) = 717.55$, $p < .001$).  As such, one factor (``Joint Action Success Pre Post'') was imposed on items concerning team component performance, which explained 74.5\% of the overall variance in items relating to team performance (SS Loading = 2.97). $Guttman's \lambda =.90$ and $Cronbach's \alpha = (.92)$ indicated that the data reduction was appropriate.  Correlations between five items relating to individual performance were also sufficiently high (all $r's > .45$, $KMO = 0.83$, corrtest Bartlett: $\chi^2(10, N = 238) = 633.82$, $p < .001$), suggesting that one factor would be appropriate for individual performance (see Table ~\ref{X}).  One factor (``Ind Performance Success Pre Post'') was imposed, which explained 59.9\% of the overall variance (SS Loading = 2.99).  $Guttman's\lambda =.87$ and $Cronbach's \alpha = .88$ indicated that the data reduction was appropriate.

Once again, to confirm that the theoretically motivated separation of Joint Action Success from Ind Performance Success was appropriate for the collected data, a follow up EFA was conducted, in which team and individual performance component variables were combined in one matrix. Sampling adequacy measures indicated high suitability ($KMO = 0.83$, $\chi^2(36, N = 118) = 726.60$, $p < .0001$).  As expected, an EFA extracted two factors, with Individual performance measures loaded on one factor (proportion of variance = .34, $SS Loading = 3.09$), and team performance measures loading on a second factor (proportion of variance = .32, $SS Loading = 2.90$). $Guttman's \lambda =.93$ and $Cronbach's \alpha = .90$ indicated that the data reduction was appropriate.

%$\chi^2 (df=2) = 20.27$, $p < .001$,
%$\chi^2 (df=5) = 47.29$, $p < .001$,

\subsubsection{Tournament Survey Items}

\myparagraph{Overall Tournament}
\textit{As explained in Section ~\ref{Section:Tournament Survey Items} above, athlete perceptions of components of performance were not administered in the mid-Tournament surveys, and thus were not eligible for overall Tournament analysis}


\subsubsection{Perceptions of team performance relative to prior expectations}
One of the two main predictor variables of interest, perceptions of team performance relative to prior expectations, was a single item measure, and did not require data reduction.  Given that many of the other outcome variables of interest were transformed, through Factor Analysis procedures, into standardized z-scores (mean = 0, SD = 1), perceptions of team performance relative to prior expectation was also standardised (mean = 0, SD = 1) for consistency and accurate generation of estimates within subsequent linear mixed effects models. Perceptions of individual and team performance relative to prior expectations was included in only the mid- and post-Tournament surveys.


\subsubsection{Team Click}
Variables associated with team click within each subset of data were subjected to EFAs.

\myparagraph{Post-Tournament}
All six items relevant to team click were included in an EFA: Unspoken Understanding, General Atmosphere, Team Click Pictorial, Reliability Of Others, Reliability For Others, Abilities Extended.  Strong correlations between variables of interest ($r's > .3$, see Table ~\ref{tab:3clickPostCorr}) and sampling adequacy measures suggested that imposing one factor was appropriate, $KMO =  0.69$, $\chi^2(15, N = 118) = 182.73$, $p < .001$.  One factor labelled ``Team Click'' was extracted from the data, which explained 34.5\% of the overall variance ($SS Loading = 2.07$).  Guttman's $\lambda =.76$ and Cronbach's $\alpha = .75$ indicated that the data reduction was appropriate.  These reliability statistics provided confidence that the novel pictorial click measure related strongly to the ethnographically derived click items (for example, Unspoken Understanding and General Atmosphere).

\myparagraph{Pre- to post-Tournament}
Pre- and post-Tournament measures of the same six team click items were collated and subjected to EFA. Correlations between variables was high, indicating that one factor was suitable (all $r's > .1$, $KMO = 0.78$, corrtest Bartlett: $\chi^2(15, N = 238) = 336.41$, $p < .001$).  One factor, ``Team Click Pre Post'', was imposed, which explained 37.6\% of the overall variance (all items loadings $> .4$, SS Loading = 2.26).  $Guttman's \lambda =.76$ and $Cronbach's \alpha = .76$ indicated that the data reduction was appropriate.
%$\chi^2(9, N = 238) = 31.52 $, $p < .001$
\myparagraph{Overall Tournament}
Unspoken Understanding, General Atmosphere, and Click Pictorial---the three team click items that were included in all 4 surveys---were subjected to EFA.  Correlations between variables were high, suggesting that imposing one factor was appropriate (all $r's > .62$, $KMO = 0.7$, $\chi^2(3, N = 440) = 723.67$).  The factor ``Team Click Tournament'' explained 70.3\% of the variance (SS Loadings = 2.11).  $Guttman's \lambda =.83$ and Cronbach's $\alpha = .87$ indicated that the data reduction was appropriate.

%, and a Chi-squared test indicated that one factor was sufficient to account for the variance of these items ($\chi^2 (9, ) = 46.36$, $p < .001$).

%Interestingly, when click and bonding measures were analysed together, the following loading were observed:
%Loadings:
%                       Factor1 Factor2
%unspokenUnderstanding7  0.710
%generalAtmosphere7      0.713
%clickPictorial7         0.672  -0.106
%reliabilityOfOthers7    0.128   0.539
%reliabilityForOthers7           0.424
%abilityExtended7       -0.110   0.956
%emotionalSupport7       0.656   0.131
%sharedGoal7             0.838
%fusionPictorialTeam7    0.424
%fusionVerbal7           0.138   0.281


\subsubsection{Social Bonding}
Five survey items related to feelings of social bonding to the team and to teammates, were separately analysed for the purposes of data reduction.  These items included: Emotional Support, Shared Goal, Identity Fusion Verbal Scale, Group Identification Verbal Scale, and the Identity Fusion Pictorial (Team, Family, Country).

\myparagraph{Post-Tournament}
A correlation matrix of all five social bonding variables (~\ref{tab:4bondingPostCorr}) indicated that Group Identification Verbal Scale did not share common variance with other variables (all correlations were $<.1$, except for the verbal measures of Identity Fusion ($r =.358$)). As such, Group Identification was excluded from analysis.  Sampling adequacy variables suggested that the remaining subset of variables were appropriate for analysis, $KMO = 0.65$, $\chi^2(10, N = 118) = 108.22$, $p < .001$.  EFA was performed on 4 remaining items, imposing one factor labelled ``Social Bonding'', which explained 34.5\% of the overall variance ($SS Loading = 1.60$, $Guttman's \lambda =.66$ and $Cronbach's \alpha = .65$).

\myparagraph{Pre- to post-Tournament}
The same four survey items related to feelings of social bonding were subjected to EFA. Correlations between variables were generally high (all $r's > .3$, $KMO = 0.66$, corrtest Bartlett: $\chi^2(6, N = 238) = 218.95$, $p < .001$), indicating that one factor would be appropriate.  One factor, labelled ``Social Bonding Pre Post'' was extracted, which explained 41.9\% of the overall variance (SS Loading = 1.68).  $Guttman's \lambda =.68$ and $Cronbach's \alpha = .71$ indicated that the data reduction was appropriate.
% $\chi^2 (df=2) = 10.3 $, $p < .01$,

\myparagraph{Overall Tournament}
The three social bonding variables that featured consistently in all surveys (Emotional Support, Shared Goal, and Fusion Pictorial) were subjected to EFA. Correlations between variables were high, suggesting that imposing one factor was appropriate (all $r's > .63$, $KMO = 0.71$, $\chi^2(3, N = 440) =  759.30$, $p < .001$).  The factor imposed for Social Bonding (``Social Bonding Tournament'') explained 71.7\% of the variance ($SS Loadings =  2.15$), and $Guttman's \lambda =.84$ and $Cronbach's \alpha= .88$ indicated that the data reduction was reliable.

\subsubsection{Data Reduction for moderator variables}
Data reduction on survey items pertaining to moderator variables of 1) Fatigue and Exertion (hereafter Fatigue) and Technical Competence (two factors, Subjective Competence and Objective Competence), are reported in full detail in Appendix \ref{app5:tournamentSurvey} Section ~\ref{app:moderatorVarsEFA}.



 \subsection{Pre-post Tournament differences in latent factors}
 Following data reduction, paired samples t-tests were used to compare pre- and post-Tournament measures of extracted factors relating to team and individual performance, team click, social bonding, and fatigue. Results are displayed in Table ~\ref{}.  Mean difference between athlete scores of team component performance (Joint Action Success Pre Post) did not vary significantly from pre- to post-Tournament, $M = -.06 (-0.30  0.18)$, $t(98)= -.48, p = .63$, nor did team click, $M = .06 (-0.12, 0.25)$, $t(98)= .69, p = .49$. Athlete mean differences in perceptions of components of individual performance (Individual Performance Success Pre Post) significantly decreased following the Tournament, $M = -0.28 (-0.47, -0.10)$, $t(98)= -2.99, p = .003$, suggesting that athletes were on average more critical of their individual performance than their team performance at the completion of the Tournament.

 Tests revealed that the there was a significant difference in athlete reports of Social Bonding between pre- and post-Tournament surveys, $M = 0.34 (0.16, 0.52)$, $t(98)= 3.73, p = .0003$; Athletes' feelings of bonding on average increased from pre- to post-Tournament.  Average ratings of Fatigue also significantly increased from measurement following athletes' second game on Day 1 to ratings following the Tournament, $M =  0.77 (0.55, 0.99)$, $t(116)= 7.03, p < .0001$ (see Table ~\ref{tab:factorsPrePostSummary}).

 \newgeometry{margin=0.5cm} % modify this if you need even more space
 \begin{landscape}
 
% Table created by stargazer v.5.2 by Marek Hlavac, Harvard University. E-mail: hlavac at fas.harvard.edu
% Date and time: Sat, Jun 03, 2017 - 17:41:41
\begin{table}[!htbp] \centering 
  \caption{factors change pre-post Tournament} 
  \label{} 
\footnotesize 
\begin{tabular}{@{\extracolsep{5pt}} cccccccccc} 
\\[-1.8ex]\hline 
\hline \\[-1.8ex] 
 & Variable & Mean.pre & SD.pre & Mean.post & SD.post & MeanDiffernece & MeanPairedDifference & t-test & p-value \\ 
\hline \\[-1.8ex] 
1 & teamPerformanceFactorPrePost & $$-$0.01$ & $1.04$ & $0.01$ & $0.88$ & $0.01$ & $$-$0.06$ & $$-$0.48$ & $0.63$ \\ 
2 & indPerformanceFactorPrePost & $0.12$ & $0.95$ & $$-$0.13$ & $0.93$ & $$-$0.25$ & $$-$0.28$ & $$-$2.99$ & $0.004$ \\ 
3 & clickFactorPrePost & $$-$0.04$ & $0.98$ & $0.04$ & $0.85$ & $0.09$ & $0.06$ & $0.69$ & $0.84$ \\ 
4 & bondingFactorPrePost & $$-$0.18$ & $1.03$ & $0.19$ & $0.71$ & $0.37$ & $0.34$ & $3.74$ & $0.0003$ \\ 
5 & fatigueFactorPrePost & $$-$0.29$ & $1.01$ & $0.40$ & $0.65$ & $0.68$ & $0.77$ & $7.03$ & $0$ \\ 
\hline \\[-1.8ex] 
\end{tabular} 
\end{table} 

 \restoregeometry



\subsection{Roadmap for analysis of study predictions}
Three distinct subsets of the collected data were analysed in order to test study predictions. First, post-Tournament survey data were analysed, in order to assess the extent to which there was a statistically meaningful relationship between perceptions of team performance, team click, and social bonding following a high-intensity, high stakes professional rugby Tournament.  Second, pre- to post-Tournament change in variables of interest was analysed in order to infer whether or not within-subject variation in key outcome variables (Team Click and Social Bonding) was driven by variation in hypothesised predictor variables (Joint Action Success and Team Performance Vs Expectations). Third, the entire dataset (i.e., pre-, mid-, and post-Tournament surveys) was analysed to ascertain whether the predicted relationships between variables of interest were consistent throughout the Tournament.  By analysing multiple observations for the same athlete over a number of time points, it was possible to better account for intra- and inter-individual variation in the collected data, enabling more robust inferences regarding study predictions.

\subsubsection{Data Structure and Model Selection}
The \textit{in situ} nature of this study meant that the collected data contained multiple levels of dependency: Observations from multiple time points were nested within the same individual athlete; individual athletes were nested within their respective teams; teams were nested within the men’s and women’s competitions, and so on.  In addition, the data were unbalanced, meaning that there were an uneven number of observations recorded for each of the 15 teams and 8 separate time points. Linear mixed-effects regression (LMER) models were thus used in order to avoid the violation of the assumptions of independence and equality of variance due to dependencies, and at the same time deal with the problem of missing data \citep{Quene2004,Field2012}.  The LMER can be expressed in notation form as follows:

  \begin{equation}
    \begin{align*}
      Y_ij & \sim  (\beta_0 + u_0j) + (\beta_1 + u_1j)X_ij + \varepsilon_ij\\
           & \varepsilon_ij \sim \mathcal{N}(0,\sigma^{2})
    \end{align*}
  \end{equation}

Where $Y_ij$ denotes the $i^t^h$ observation for group $j$, $(\beta_0 + u_0j)$ denotes the fixed and random intercept, $(\beta_1 + u_1j)$ the random and fixed slope, and $\varepsilon_ij$ denotes the error term.  Errors are assumed to be normally distributed with mean of zero.  The random structure(s) of the model (j) was determined by assessing  group-level (i.e., team, competition, or athlete (in the case of repeated measures analyses) Intra Class Correlations (ICC) values for main variables of interest.

\subsubsection{post-Tournament ICC Values}
Computing ICC values for the post-Tournament factors of interest identified team-level clustering in the data for Joint Action Success (ICC = .374), Individual Performance Success (ICC = .223) Team Click (ICC = .299), Social Bonding (ICC = .098) and objective measures of technical competence (ICC = .356). Although the ICC value for Social Bonding was relatively low, a group-wise means difference test (computed using a linear regression model) revealed significant team level differences in mean Social Bonding, $F(14, 103) = 1.84, p = .04, R^2 = 0.09$). ICC values indicated only trivial clustering according to Tournament competition (men's or women's, all ICC values were less than .10, except for the ICC for Individual Performance Success, which was .108).  As such, team (but not Competition/sex)) was included in subsequent models as a random effect (for a detailed assessment of dependencies in the collected data, see Appendix ~\ref{app5:tournamentSurvey} Section ~\ref{app5:modelSelection}).


\subsubsection{Pre- to Post-Tournament ICC Values}

\subsubsection{Overall Tournament ICC Values}





\section{Analysis of study predictions}
For the following analyses, multilevel linear models were fit with Maximum Likelihood parameter estimation method using the lme4 package (Bates and Sarkar, 2006) in the R environment (R Development Core Team, 2006).  Team was included as a random effect, which allowed the intercepts and slopes of the response to vary according to team (i.e., random intercept model, see \citep{Pinheiro2000}; for an application, see \citep{Oberauer2006}). This helped statistically account for clustering of observations within teams. Control and moderator variables were introduced into each model as fixed effects in a stepwise fashion, and model fit was judged by comparing the Akaike Information Criteria (AIC) and Bayesian Information Criteria (BIC) using a chi-squared test based on -2Log Likelihood score when possible.\footnote{The chi-squared test based on -2Log Likelihood scores was not possible when comparing models with different sample sizes.
This created difficulty when introducing covariates into the model, and subsequently reducing the number of observations} In addition, marginal and conditional $R^2$ values of equivalent models were compared and used as an indication of model effect size.\footnote{Nakagawa and Schielzeth (2013) have proposed a formula for calculating the proportion of variance explained by the fixed factor of the model (marginal $R^2$, compared to an intercept-only or null model), and the proportion of variance explained by the combination of the fixed factor plus the random factor, (conditional $R^2$, compared to the null)}

\subsection{Prediction 1: more positive perceptions of team performance predict higher levels of team click}

\subsubsection{Prediction 1.a: Perceptions of success in components of joint action predicts Team Click}

The prediction that more positive perceptions of success in components of joint action will correlate with Athletes' feelings of Team Click was analysed using the post-Tournament and pre- to post-Tournament subsets (Joint Action Success was only measured in the pre- and post-Tournament surveys).

\myparagraph{Post-Tournament Survey}
A scatterplot depicting the relationship between perceptions of components of team performance (Joint Action Success) and Team Click in the post-Tournament survey data is shown in Figure ~\ref{fig:jasClickBasicXY}. The plot indicates a strong positive relationship between Joint Action Success and Social Bonding.

\begin{figure}[htbp]
\includegraphics[width = \linewidth]{images/jasClickBasicXY.pdf}
  \caption{Joint Action Success predicts Team Click (n = 118)}
  \label{fig:jasClickBasicXY}
\end{figure}


To test the prediction that athlete perceptions of joint action are positively
related to team click in the post-Tournament data, the following model was constructed:

\begin{equation}
  \begin{align*}
    Team Click =  & Joint Action Success\\
              & + Individual Performance Success \\
              & + Objective Competence + Subjective Competence\\
              & + TournamentPerformanceMeasures \\
  \end{align*}
\end{equation}
\bigskip

Predictor variable (Joint Action Success), moderator variables (Objective Competence and Subjective Competence), and controls (Individual Performance Success and Tournament performance measures) were introduced as fixed effects.   Team was specified as a random effect, which allowed the slope and intercept of team click responses to vary by team.\footnote{To avoid issues of multicolinearity, Total Win Loss was excluded from the model, given the high pairwise correlation with Final Rank (.90). Pairwise correlation between all other predictor variables in the model was below .80}

The model revealed a significant positive effect of Joint Action Success on Team Click, $\beta = .69$ ($95\% CI =  0.46, 0.89$), $SE = 0.11$, $t(13.23) = 6.20$, $p < .01$, $marginal R^2 = .53$, $conditional R^2 = .60$. All other fixed effects did not significantly predict team click, but the inclusion of these fixed effects in the model did significantly improve the overall model fit, as indicated by a comparison of AIC/BIC values between interactions of the model (see Table ~\ref{tab:MLM1aJointActionSuccessClick}).  Model residuals were normally distributed around zero ($W = 0.97, p = .03$), and individual cases had low influence on the model (Cook's Distances all < .25, see Appendix ~\ref{app5:tournamentSurvey} Figure ~\ref{fig:MLM1aAssumptions}). Results of this model support the prediction that athletes' perceptions of joint-action success correlate positively with feelings of team click.  The model-adjusted slope is included in scatterplot ~\ref{fig:jasClickModelSLope}.

%Individual Performance Success ($\beta = -.03$, $SE = .09$, $t(97) = -.33$, $p = .74$), Objective ($\beta = .05$, $SE = .08$, $t(84.41) = .61$, $p = .54$) and Subjective competence ($\beta = .11$, $SE = .07$, $t(87.76) = 1.57$, $p = .12$), and Tournament performance measures (Final Rank: $\beta = .02$, $SE = .04$, $t(92.85) = .58$, $p = .57$; Total Minutes: $\beta = .01$, $SE = .003$, $t(88.68) = 1.84$, $p = .07$), and Total Points: $\beta = .004$, $SE = .005$, $t(88.39) = .89$, $p = .38$)) did not significantly predict team click.


% Table created by stargazer v.5.2 by Marek Hlavac, Harvard University. E-mail: hlavac at fas.harvard.edu
% Date and time: Mon, Jun 26, 2017 - 18:43:32
\begin{table}[!htbp] \centering 
  \caption{M1.a Team Click = Joint Action Success} 
  \label{tab:MLM1aJointActionSuccessClick} 
\footnotesize 
\begin{tabular}{@{\extracolsep{5pt}}lccc} 
\\[-1.8ex]\hline 
\hline \\[-1.8ex] 
 & \multicolumn{3}{c}{\textit{Dependent variable:}} \\ 
\cline{2-4} 
\\[-1.8ex] & \multicolumn{3}{c}{Team Click} \\ 
\\[-1.8ex] & (1) & (2) & (3)\\ 
\hline \\[-1.8ex] 
 (constant) & $-$0.04 & 0.02 & $-$0.45 \\ 
  & (0.15) & (0.07) & (0.24) \\ 
  & & & \\ 
 jointActionSuccess &  & 0.65$^{***}$ & 0.67$^{***}$ \\ 
  &  & (0.10) & (0.11) \\ 
  & & & \\ 
 indPerformanceSuccess &  &  & $-$0.03 \\ 
  &  &  & (0.09) \\ 
  & & & \\ 
 objectiveCompetence &  &  & 0.05 \\ 
  &  &  & (0.08) \\ 
  & & & \\ 
 subjectiveCompetence &  &  & 0.11 \\ 
  &  &  & (0.07) \\ 
  & & & \\ 
 finalRank &  &  & 0.02 \\ 
  &  &  & (0.04) \\ 
  & & & \\ 
 minutesTotal &  &  & 0.01 \\ 
  &  &  & (0.003) \\ 
  & & & \\ 
 pointsTotal &  &  & 0.005 \\ 
  &  &  & (0.01) \\ 
  & & & \\ 
\hline \\[-1.8ex] 
Marginal R-squared &  &  & .53 \\ 
Conditional R-squared &  &  & .60 \\ 
Observations & 118 & 118 & 174 \\ 
Log Likelihood & $-$146.53 & $-$112.76 & $-$92.74 \\ 
Akaike Inf. Crit. & 299.06 & 237.51 & 209.48 \\ 
Bayesian Inf. Crit. & 307.37 & 254.14 & 240.37 \\ 
\hline 
\hline \\[-1.8ex] 
\textit{Note:}  & \multicolumn{3}{r}{$^{*}$p$<$0.05; $^{**}$p$<$0.01; $^{***}$p$<$0.001} \\ 
\end{tabular} 
\end{table} 



\begin{figure}[htbp]
\includegraphics[width = \linewidth]{images/jasClickModelSlope}
  \caption{Joint Action Success predicts Team Click. The blue slope is the original line of best fit, the red slope has been adjusted according to the predictions of the linear model}
  \label{fig:jasClickModelSLope}
\end{figure}



\myparagraph{Pre- to Post-Tournament change}
To further explore the hypothesised relationship between Joint Action Success and Team Click, pre- to post-Tournament change in perceptions of Joint Action Success and Team Click was analysed.  Figure ~\ref{fig:jasClickDeltaBasicXY} displays a scatterplot depicting a positive relationship between change in perceptions of Joint Action Success and change in feelings of team click between the pre- and post-Tournament surveys.

\begin{figure}[htbp]
\includegraphics[width = \linewidth]{images/jasClickDeltaBasicXY}
  \caption{Joint Action Success predicts Team Click. The blue slope is the original line of best fit, the red slope has been adjusted according to the predictions of the linear model}
  \label{fig:jasClickDeltaBasicXY}
\end{figure}

Controlling for Tournament performance, measures and objective and subjective competence (fixed effects), and team-level variance (random effect), the model revealed a significant positive relationship between change in perceptions of joint-action success and change in feelings of team click, $\beta = .52$ ($95\% CI =  .34, .71$), $SE = 0.09$, $t(11.32) = 5.55$, $p < .001$, $marginal R^2 = .40$, $conditional R^2 = .47$ (see Table ~\ref{tab:MLM21aJointActionSuccesscClick} for results of all iterations of the model).  Model residuals were normally distributed around zero ($W = 0.98228, p = .22$), and individual cases had low influence on the model (Cook's Distances all < .20, see Appendix Figure \ref{fig:MLM21aAssumptions}).  This model supported the prediction that more positive perceptions of Joint Action Success correlate with stronger feelings of team click, by showing that athletes who experienced a positive increase in perceptions of joint-action success also reported a positive increase in feelings of team click throughout the Tournament.
The model-adjusted slope is included in scatterplot ~\ref{fig:jasClickDeltaModelSLope}.


% Table created by stargazer v.5.2 by Marek Hlavac, Harvard University. E-mail: hlavac at fas.harvard.edu
% Date and time: Tue, Jun 27, 2017 - 09:10:21
\begin{table}[!htbp] \centering 
  \caption{cTeamClick ~ cJointActionSuccess} 
  \label{tab:MLM21aJointActionSuccesscClick} 
\begin{tabular}{@{\extracolsep{5pt}}lccc} 
\\[-1.8ex]\hline 
\hline \\[-1.8ex] 
 & \multicolumn{3}{c}{\textit{Dependent variable:}} \\ 
\cline{2-4} 
\\[-1.8ex] & \multicolumn{3}{c}{cTeamClick} \\ 
\\[-1.8ex] & (1) & (2) & (3)\\ 
\hline \\[-1.8ex] 
 (constant) & 0.10 & 0.10 & $-$0.10 \\ 
  & (0.13) & (0.08) & (0.28) \\ 
  & & & \\ 
 cJointActionSuccess &  & 0.47$^{***}$ & 0.52$^{***}$ \\ 
  &  & (0.08) & (0.09) \\ 
  & & & \\ 
 cIndPerformanceSuccess &  &  & $-$0.04 \\ 
  &  &  & (0.09) \\ 
  & & & \\ 
 objectiveCompetence &  &  & 0.03 \\ 
  &  &  & (0.09) \\ 
  & & & \\ 
 subjectiveCompetence &  &  & $-$0.01 \\ 
  &  &  & (0.08) \\ 
  & & & \\ 
 finalRank &  &  & $-$0.02 \\ 
  &  &  & (0.04) \\ 
  & & & \\ 
 minutesTotal &  &  & 0.004 \\ 
  &  &  & (0.004) \\ 
  & & & \\ 
 pointsTotal &  &  & 0.01 \\ 
  &  &  & (0.01) \\ 
  & & & \\ 
\hline \\[-1.8ex] 
Marginal R-squared &  &  & .40 \\ 
Conditional R-squared &  &  & .47 \\ 
Observations & 99 & 99 & 97 \\ 
Log Likelihood & $-$130.59 & $-$107.75 & $-$104.44 \\ 
Akaike Inf. Crit. & 267.19 & 227.50 & 232.88 \\ 
Bayesian Inf. Crit. & 274.97 & 243.07 & 263.77 \\ 
\hline 
\hline \\[-1.8ex] 
\textit{Note:}  & \multicolumn{3}{r}{$^{*}$p$<$0.05; $^{**}$p$<$0.01; $^{***}$p$<$0.001} \\ 
\end{tabular} 
\end{table} 


\begin{figure}[htbp]
\includegraphics[width = \linewidth]{images/jasClickDeltaModelSlope}
  \caption{Change in Joint Action Success predicts change in Team Click. The blue slope is the original line of best fit, the red slope has been adjusted according to the predictions of the linear model.}
  \label{fig:jasClickDeltaModelSLope}
\end{figure}

\myparagraph{Prediction 1.a: Summary of results}
Results from the analysis of Prediction 1.a suggest that following the Tournament, more positive perceptions of technical components of team performance correlate with higher levels of self-reported team click.  In addition, a positive increase in perceptions of Joint Action Success predicts a positive increase in feelings of Team Click between pre- and post-Tournament surveys, suggesting that athletes who perceived an increase in success of components of team performance also experienced an increase in feelings of team click.






\subsubsection{Prediction 1.b: Team Performance Expectations predicts Team Click}

The prediction that more positive violations of team performance expectations (Team Performance Expectations) will correlate with higher feelings of Team Click was analysed using the 1) post-Tournament, 2) pre- to post-Tournament, and 3) overall Tournament subsets (using the three-item Team Click Factor).

\myparagraph{Post-Tournament}
A scatterplot depicting the relationship between Team Performance Expectations and Team Click in the post-Tournament survey data is shown in Figure ~\ref{fig:teamPerfClickBasicXY}. The plot indicates a strong positive relationship between Joint Action Success and Social Bonding.

\begin{figure}[htbp]
\includegraphics[width = \linewidth]{images/teamPerfClickBasicXY.pdf}
  \caption{Team Performance Expectations predicts Team Click (n = 118)}
  \label{fig:teamPerfClickBasicXY}
\end{figure}

To assess the prediction that more positive violation of expectations around team performance correlates with higher levels of team click, the following model was constructed:

  \begin{equation}
    \begin{align*}
      Team Click =  & Team Performance Expectations \\
                &+ Individual Performance Expectations \\
                &+ Objective Competence + Subjective Competence \\
                &+ Tournament performance measures \\
    \end{align*}
  \end{equation}

Expectations around individual performance, objective and subjective competence, and Tournament performance measures were introduced to the model as controls (fixed effects), while team was introduced as a random (level 2) effect.  Results of the model revealed a significant positive relationship between Team Performance Expectations and Team Click, $\beta = .52$ ($95\% CI =  0.28, 0.73$), $SE = 0.12$, $t(4.23) = 4.23$, $p < .001$, $marginal R^2 = .40$, $conditional R^2 = .56$.  All other fixed effects did not significantly predict team click, but did improve the model fit, as indicated by the reduced AIC and increased marginal $R^2$ value (see Appendix Table ~\ref{tab:MLM1bteamExpectationsClick} for the various iterations of the model).   Model residuals were normally distributed around zero ($W = 0.98, p = .30$), and individual cases had low influence on the model (Cook's Distances all < .25, see Appendix ~\ref{app5:tournamentSurvey} Figure ~\ref{fig:MLM1bAssumptions}).  These results provide support for the prediction that more positive violations of expectations around team performance predict higher feelings of team click.  The slope of the linear model is included in scatterplot ~\ref{}

%Expectations around individual performance ($\beta = .04$, $SE = .08$, $t(86.26) = .46$, $p = .65$), objective ($\beta = .01$, $SE = .09$, $t(84.89) = .13$, $p = .89$) and subjective competence ($\beta = .12$, $SE = .07$, $t(91) = 1.62$, $p = .11$), and Tournament performance measures (Final Rank: $\beta = .08$, $SE = .05$, $t(44.33) = 1.72$, $p = .09$; Total Minutes: $\beta = .004$, $SE = .003$, $t(87.05) = 1.00$, $p = .32$); Total Points: $\beta = .001$, $SE = .005$, $t(84.94) = .21$, $p = .84$)) did not significantly predict team click.

  
% Table created by stargazer v.5.2 by Marek Hlavac, Harvard University. E-mail: hlavac at fas.harvard.edu
% Date and time: Mon, Jun 26, 2017 - 18:44:39
\begin{table}[!htbp] \centering 
  \caption{M1.b teamClick = teamPerformanceExpectations} 
  \label{tab:MLM1bteamExpectationsClick} 
\footnotesize 
\begin{tabular}{@{\extracolsep{5pt}}lccc} 
\\[-1.8ex]\hline 
\hline \\[-1.8ex] 
 & \multicolumn{3}{c}{\textit{Dependent variable:}} \\ 
\cline{2-4} 
\\[-1.8ex] & \multicolumn{3}{c}{teamClick} \\ 
\\[-1.8ex] & (1) & (2) & (3)\\ 
\hline \\[-1.8ex] 
 (constant) & $-$0.04 & $-$1.36$^{***}$ & $-$2.12$^{***}$ \\ 
  & (0.15) & (0.36) & (0.44) \\ 
  & & & \\ 
 teamPerformanceExpectations &  & 0.02$^{***}$ & 0.02$^{***}$ \\ 
  &  & (0.005) & (0.01) \\ 
  & & & \\ 
 individualPerformanceExpectations &  &  & 0.002 \\ 
  &  &  & (0.003) \\ 
  & & & \\ 
 objectiveCompetence &  &  & 0.01 \\ 
  &  &  & (0.09) \\ 
  & & & \\ 
 subjectiveCompetence &  &  & 0.12 \\ 
  &  &  & (0.07) \\ 
  & & & \\ 
 finalRank &  &  & 0.08 \\ 
  &  &  & (0.05) \\ 
  & & & \\ 
 minutesTotal &  &  & 0.004 \\ 
  &  &  & (0.004) \\ 
  & & & \\ 
 pointsTotal &  &  & 0.001 \\ 
  &  &  & (0.01) \\ 
  & & & \\ 
\hline \\[-1.8ex] 
Marginal R-squared &  &  & .40 \\ 
Conditional R-squared &  &  & .56 \\ 
Observations & 118 & 118 & 97 \\ 
Log Likelihood & $-$146.53 & $-$128.22 & $-$102.82 \\ 
Akaike Inf. Crit. & 299.06 & 268.43 & 229.63 \\ 
Bayesian Inf. Crit. & 307.37 & 285.06 & 260.53 \\ 
\hline 
\hline \\[-1.8ex] 
\textit{Note:}  & \multicolumn{3}{r}{$^{*}$p$<$0.05; $^{**}$p$<$0.01; $^{***}$p$<$0.001} \\ 
\end{tabular} 
\end{table} 


  \begin{figure}[htbp]
  \includegraphics[width = \linewidth]{images/teamPerfClickModelSlope.pdf}
    \caption{Team Performance Expectations predicts Team Click. The blue slope is the original line of best fit, the red slope has been adjusted according to the predictions of the linear model.}
    \label{fig:teamPerfClickModelSlope}
  \end{figure}



  \myparagraph{Pre- to Post-Tournament change}
Given that the item measuring Athlete perceptions of overall team performance in the pre-Tournament survey was not framed in terms of expectation violation (but instead in terms of more or less ``good'' or ``poor''), it was not possible to directly compare the pre- and post-Tournament measures.  Instead, the post-Tournament measure of Team Performance Expectation violation was analysed in terms of its relationship to a change in feelings of team click between pre- and post-Tournament measurements. Did more positive violations of team performance expectations following the Tournament predict an increase in feelings of team click? A scatterplot depicting the relationship between Team Performance Expectations and change in Team Click is shown in Figure ~\ref{fig:teamPerfClickDeltaBasicXY}. The plot indicates a positive (albeit less pronounced) relationship between Team Performance Expectations and change in Team Click.
    %\subsubsection{2.1.b $\Delta$Team Click $\sim$ Team Performance Expectations}

The relationship between team performance relative to prior expectations (recorded post-Tournament) and Team Click was analysed using a linear mixed effects model. The model revealed a significant positive relationship between Team Performance Expectations and change in Team Click,  $\beta = .27$ ($95\% CI =  .05, .49$), $SE = 0.11$, $t(32.97) = 2.36$, $p = .02$, $marginal R^2 = .12$, $conditional R^2 = .22$, indicating that athletes who were more positive about their appraisals of team performance following the Tournament on average experienced an increase in feelings of team click.

Examination of model residuals revealed that they were not normally distributed around zero ($W = 0.95, p = .001$), due to positive skew ($.85$) (see Appendix Figure ~\ref{fig:MLM21bAssumptions}).  Non-normally distributed residuals are problematic as they may influence the model's ability to generate accurate parameter estimates . Two data manipulation techniques were considered in order to normalise residuals and therefore preserve the estimates of the model: exclusion of outliers and transformation of the outcome variable.  Transformation of the outcome variables was the preferred method over outlier exclusion, due to the cost involved in removing observations that may be of potential theoretical relevance to the scientific investigation \citep{Rousseeuw2011}. Nonetheless, for this first phase of analysis, both outlier exclusion and transformation manipulations of the data were performed and reported.

Log-transformation of the outcome variable did not markedly improve the non-normality of residuals, ($Shapiro-Wilk = 0.96, p = .008$).  Instead, exclusion of outliers according to Tukey's method (observations above and below 1.5x the Inter Quartile Range (IQR), see \citep{Tukey1977}) improved the model fit such that residuals were normally distributed, ($Shapiro-Wilk = 0.99, p = .72$), and individual cases had low influence on the model (Cook's Distances all < .10, see Appendix Figure ~\ref{fig:MLM21bOutAssumptions}).
The adjusted model supported the original significant positive effect of Team Performance Expectations on Team Click, $\beta = .36$ ($95\% CI =  .16, .55$), $SE = 0.10$, $t(9.78) = 3.66$, $p < .01$, $marginal R^2 = .16$, $conditional R^2 = .55$.
See Appendix ~\ref{app5:tournamentSurvey} Table ~\ref{tab:MLM21bOutLogComparison} for a comparison of adjusted models. These results confirm the prediction that those who experience more positive violations around team performance also experience an higher levels of Team Click---shown in this case by an increase in feelings of Team Click between pre- and post-Tournament measurements.  The slope of the adjusted model is included in the scatterplot of figure ~\ref{fig:teamPerfClickDeltaModelSlope} purposes of comparison

%The adjusted model also revealed a significant negative effect of violations of expectations around \textit{individual} performance on team click, $\beta = -.008 (95\% CI =  -.01, .001), SE = 0.004, t(91.79) = -2.28, p = .03$,  which could suggest that athletes who experienced more positive violations about their own performance did not feel the team click as strongly.

    
% Table created by stargazer v.5.2 by Marek Hlavac, Harvard University. E-mail: hlavac at fas.harvard.edu
% Date and time: Tue, Jun 27, 2017 - 09:21:13
\begin{table}[!htbp] \centering 
  \caption{M2.1b cTeamClick ~ cPerformanceExpectations} 
  \label{tab:MLM21bcTeamPerfExpcClick} 
\begin{tabular}{@{\extracolsep{5pt}}lccc} 
\\[-1.8ex]\hline 
\hline \\[-1.8ex] 
 & \multicolumn{3}{c}{\textit{Dependent variable:}} \\ 
\cline{2-4} 
\\[-1.8ex] & \multicolumn{3}{c}{cTeamClick} \\ 
\\[-1.8ex] & (1) & (2) & (3)\\ 
\hline \\[-1.8ex] 
 (constant) & 0.10 & $-$0.57 & $-$0.64 \\ 
  & (0.13) & (0.30) & (0.44) \\ 
  & & & \\ 
 cTeamPerformanceExpectations &  & 0.01$^{*}$ & 0.01$^{*}$ \\ 
  &  & (0.004) & (0.005) \\ 
  & & & \\ 
 cIndPerformanceExpectations &  &  & $-$0.003 \\ 
  &  &  & (0.004) \\ 
  & & & \\ 
 objectiveCompetence &  &  & $-$0.13 \\ 
  &  &  & (0.11) \\ 
  & & & \\ 
 subjectiveCompetence &  &  & $-$0.16 \\ 
  &  &  & (0.09) \\ 
  & & & \\ 
 finalRank &  &  & 0.02 \\ 
  &  &  & (0.06) \\ 
  & & & \\ 
 minutesTotal &  &  & 0.003 \\ 
  &  &  & (0.005) \\ 
  & & & \\ 
 pointsTotal &  &  & 0.003 \\ 
  &  &  & (0.01) \\ 
  & & & \\ 
\hline \\[-1.8ex] 
Marginal R-squared &  &  & .40 \\ 
Conditional R-squared &  &  & .47 \\ 
Observations & 99 & 99 & 97 \\ 
Log Likelihood & $-$130.59 & $-$127.00 & $-$123.15 \\ 
Akaike Inf. Crit. & 267.19 & 266.01 & 270.30 \\ 
Bayesian Inf. Crit. & 274.97 & 281.58 & 301.20 \\ 
\hline 
\hline \\[-1.8ex] 
\textit{Note:}  & \multicolumn{3}{r}{$^{*}$p$<$0.05; $^{**}$p$<$0.01; $^{***}$p$<$0.001} \\ 
\end{tabular} 
\end{table} 

    
% Table created by stargazer v.5.2 by Marek Hlavac, Harvard University. E-mail: hlavac at fas.harvard.edu
% Date and time: Tue, Jun 27, 2017 - 09:21:15
\begin{table}[!htbp] \centering 
  \caption{M2.1b cTeamClick ~ cPerformanceExpectations (adjusted models)} 
  \label{tab:MLM21bOutLogComparison} 
\begin{tabular}{@{\extracolsep{5pt}}lcc} 
\\[-1.8ex]\hline 
\hline \\[-1.8ex] 
 & \multicolumn{2}{c}{\textit{Dependent variable:}} \\ 
\cline{2-3} 
\\[-1.8ex] & cTeamClick & clickFactorChangePrePostOut \\ 
 & log-transformed & outliers removed \\ 
\\[-1.8ex] & (1) & (2)\\ 
\hline \\[-1.8ex] 
 (constant) & 0.91$^{***}$ & $-$0.14 \\ 
  & (0.15) & (0.34) \\ 
  & & \\ 
 cTeamPerformanceExpectations & 0.004$^{*}$ & 0.02$^{***}$ \\ 
  & (0.002) & (0.004) \\ 
  & & \\ 
 cIndPerformanceExpectations & $-$0.001 & $-$0.01$^{*}$ \\ 
  & (0.001) & (0.004) \\ 
  & & \\ 
 objectiveCompetence & $-$0.04 & 0.07 \\ 
  & (0.04) & (0.09) \\ 
  & & \\ 
 subjectiveCompetence & $-$0.05 & $-$0.02 \\ 
  & (0.03) & (0.08) \\ 
  & & \\ 
 finalRank & 0.003 & $-$0.08 \\ 
  & (0.02) & (0.04) \\ 
  & & \\ 
 minutesTotal & 0.001 & 0.001 \\ 
  & (0.002) & (0.004) \\ 
  & & \\ 
 pointsTotal & 0.001 & 0.003 \\ 
  & (0.002) & (0.01) \\ 
  & & \\ 
\hline \\[-1.8ex] 
Marginal R-squared & .14 & .17 \\ 
Conditional R-squared & .25 & .20 \\ 
Observations & 97 & 93 \\ 
Log Likelihood & $-$12.38 & $-$98.81 \\ 
Akaike Inf. Crit. & 48.75 & 221.62 \\ 
Bayesian Inf. Crit. & 79.65 & 252.01 \\ 
\hline 
\hline \\[-1.8ex] 
\textit{Note:}  & \multicolumn{2}{r}{$^{*}$p$<$0.05; $^{**}$p$<$0.01; $^{***}$p$<$0.001} \\ 
\end{tabular} 
\end{table} 



    \begin{figure}[htbp]
    \includegraphics[width = \linewidth]{images/teamPerfClickDeltaModelSlope.pdf}
      \caption{Team Performance Expectations predicts change in Team Click. The blue slope is the original line of best fit, the red slope has been adjusted according to the predictions of the linear model}
      \label{fig:teamPerfClickDeltaModelSlope}
    \end{figure}




    %Low values for both of these regression coefficients...
  \myparagraph{Overall Tournament}
The overall relationship between Team Performance Expectations and Team Click was analysed using a subset of data that included observations from the two mid-Tournament surveys as well as the post-Tournament survey (Team Performance Expectation was not administered to athletes in the pre-Tournament survey).  A scatterplot depicting the relationship between Team Performance Expectations and Team Click is displayed in Figure ~\ref{fig:teamPerfClickOverallBasicXY} The three-item Team Click factor, consistent throughout all four surveys, was used in place of the six-item Team Click measure used in the analysis of post- and pre- to post-Tournament subsets

\begin{figure}[htbp]
\includegraphics[width = \linewidth]{images/teamPerfClickOverallBasicXY.pdf}
  \caption{Team Performance Expectations predicts Team Click in Overall Tournament subset}
  \label{fig:teamPerfClickOverallBasicXY}
\end{figure}

A LMER model was used to predict the relationship between joint action and team click using the overall Tournament subset survey data.  To account for multiple repeated measurements per athlete, athlete was included in the model as a random effect, in addition to the random effect of team.  The model structure involved three hierarchical levels: individual observations (level 1) were nested within individual athletes (level 2), which were nested within teams (level 3).

     \begin{equation}
       \begin{align*}
         Team Click =  & Team Performance Expectations  \\
                   &+ Individual Performance Expectations   \\
                   &+ Objective Competence + Subjective Competence \\
                   &+ TournamentPerformanceMeasures  \\
       \end{align*}
     \end{equation}
     \bigskip

The model revealed a significant relationship between team performance expectation violation and team click, $\beta = .71$ ($95\% CI =  .0.62, .80$), $SE = .001$, $t(193.20) = 16.29$, $p < .0001$, $marginal R^2 = .63$, $conditional R^2 = .69$.
The model also indicated that individual performance expectation violation significantly predicted team click, $\beta = .004$ ($95\% CI =  .04, .20$), $SE = .04$, $t(311.7) = 2.81$, $p < .01$, as did Subjective Competence, $\beta = .08$ ($95\% CI =  .002, .15$), $SE = .04$, $t(292) = 2.00$, $p = .02$  (see Table ~\ref{tab:MLM31ateamPerfClickTournament} for full description of results).
The residuals of the model were normally distributed around zero, ($W = 0.99, p = .11$), and individual cases had low influence on the model (Cook's Distances all < .05) (see Appendix ~\ref{app5:tournamentSurvey} Figure ~\ref{fig:MLM31aTeamPerfExpClick}).
Results of the model suggest that, when controlling for individual performance, measures of objective and subjective competence, as well as Tournament performance outcomes, athletes whose expectations around team performance were more positively violated also experienced stronger feelings of team click. Overall, it also appears that more positive expectations about individual performance, and higher levels of self-reported technical competence predicted higher levels of team click.  The slope of the LMER model is added to the scatter plot for comparison (see Figure ~\ref{fig:teamPerfClickOverallModelSlope}).

   
% Table created by stargazer v.5.2 by Marek Hlavac, Harvard University. E-mail: hlavac at fas.harvard.edu
% Date and time: Thu, Sep 14, 2017 - 09:28:54
\begin{table}[!htbp] \centering 
  \caption{teamClickTournament ~ teamPerformanceExpectationsTournament} 
  \label{tab:MLM31ateamPerfClickTournament} 
\footnotesize 
\begin{tabular}{@{\extracolsep{5pt}}lccc} 
\\[-1.8ex]\hline 
\hline \\[-1.8ex] 
 & \multicolumn{3}{c}{\textit{Dependent variable:}} \\ 
\cline{2-4} 
\\[-1.8ex] & \multicolumn{3}{c}{teamClick} \\ 
\\[-1.8ex] & (1) & (2) & (3)\\ 
\hline \\[-1.8ex] 
 (constant) & $-$0.00 & $-$1.46$^{***}$ & $-$1.59$^{***}$ \\ 
  & (0.04) & (0.08) & (0.15) \\ 
  & & & \\ 
 teamPerformanceExpectations &  & 0.02$^{***}$ & 0.02$^{***}$ \\ 
  &  & (0.001) & (0.001) \\ 
  & & & \\ 
 indPerformanceExpectations &  &  & 0.004$^{***}$ \\ 
  &  &  & (0.002) \\ 
  & & & \\ 
 objectiveCompetence &  &  & 0.02 \\ 
  &  &  & (0.04) \\ 
  & & & \\ 
 subjectiveCompetence &  &  & 0.08$^{**}$ \\ 
  &  &  & (0.04) \\ 
  & & & \\ 
 finalRank &  &  & $-$0.0002 \\ 
  &  &  & (0.02) \\ 
  & & & \\ 
 minutesTotal &  &  & $-$0.001 \\ 
  &  &  & (0.002) \\ 
  & & & \\ 
 pointsTotal &  &  & 0.003 \\ 
  &  &  & (0.003) \\ 
  & & & \\ 
\hline \\[-1.8ex] 
Marginal R-squared & .58 & .65 &  \\ 
Conditional R-squared & .67 & .72 &  \\ 
Observations & 564 & 444 & 331 \\ 
Log Likelihood & $-$772.35 & $-$412.33 & $-$303.82 \\ 
Akaike Inf. Crit. & 1,552.70 & 842.67 & 637.65 \\ 
Bayesian Inf. Crit. & 1,570.04 & 879.53 & 694.68 \\ 
\hline 
\hline \\[-1.8ex] 
\textit{Note:}  & \multicolumn{3}{r}{$^{*}$p$<$0.1; $^{**}$p$<$0.05; $^{***}$p$<$0.01} \\ 
\end{tabular} 
\end{table} 

% Table created by stargazer v.5.2 by Marek Hlavac, Harvard University. E-mail: hlavac at fas.harvard.edu
% Date and time: Thu, Sep 14, 2017 - 09:28:54
\begin{table}[!htbp] \centering 
  \caption{teamClickTournament ~ teamPerformanceExpectationsTournament} 
  \label{tab:MLM31ateamPerfClickTournament} 
\footnotesize 
\begin{tabular}{@{\extracolsep{5pt}} ccc} 
\\[-1.8ex]\hline 
\hline \\[-1.8ex] 
$0.05$ & $0.01$ & $0.001$ \\ 
\hline \\[-1.8ex] 
\end{tabular} 
\end{table} 



   \begin{figure}[htbp]
   \includegraphics[width = \linewidth]{images/teamPerfClickOverallBasicXY.pdf}
     \caption{Team Performance Expectations predicts Team Click in Overall Tournament subset. The blue slope is the original line of best fit, the red slope has been adjusted according to the predictions of the linear model}
     \label{fig:teamPerfClickOverallModelSlope}
   \end{figure}

\myparagraph{Prediction 1.b: Summary of results}
Results from the analysis of Prediction 1.b suggest that following the Tournament, more positive violation of expectations surrounding team performance correlate with higher levels of self-reported team click.  In addition, more positive violations of expectations following the Tournament are associated with an increase in feelings of Team Click between pre- and post-Tournament surveys, suggesting that athletes who perceived an increase in success of components of team performance also experienced an increase in feelings of team click.  Finally, the positive relationship between Overall Tournament measures of team performance expectations and team click was also significant.




\subsection{Prediction 1.c: Joint Action Success predicts Team Click, moderated by Team Performance Expectations}

If both perceptions of success and positive violations of team performance expectations predict team click, do these two predictors interact to predict team click?  Is team click higher for individuals whose perceived success in joint action was also a more positive violation of performance expectation?  This possibility was tested by including the interaction term (Joint Action Success \times Team Performance Expectations) in the LMER models used for the post-Tournament and pre- to post-Tournament subsets as follows:

  \begin{equation}
    \begin{align*}
      Team Click =  & Joint Action Success \times Team Performance Expectations \\
                &+ Individual Performance Success \\
                &+ Individual Performance Expectations \\
                &+ Objective Competence + Subjective Competence  \\
                &+ Tournament performance measures \\
    \end{align*}
  \end{equation}
  \bigskip

  \subsubsection{post-Tournamnet}
  In the case of the post-Tournament subset, the inclusion of the interaction term failed to improve upon the fit of the previous model, judging by the relative goodness of fit, $AIC(1.c) = 217.91$ compared to $AIC(1.a) = 209.48$, $SD = .52 $, $\chi^2(18, N = 97) = 3.56$, $ p =.74$.
  Results revealed that the interaction between Joint Action Success and Team Performance Expectations was not significant, $\beta = .002$ ($95\% CI =  -0.0062, 0.0063$), $SE = 0.08$, $t(39.7) = .026$, $p = .98$, $marginal R^2 = .56$, $conditional R^2 = .65$ (see Appendix ~\ref{app5:tournamentSurvey}, Table ~\ref{tab:MLM1cPerformanceClickInteraction}).


  %\textit{See appendix A for details about model assumptions}.
  %\newpage
  %\includegraphics[scale =.4]{../images/MLM1cHist.pdf}
  %\includegraphics[scale =.4]{../images/MLM1cScatter.pdf}
  %\includegraphics[scale =.4]{../images/MLM1cQQNorm.pdf}
  %\includegraphics[scale =.4]{../images/MLM1cCooksD.pdf}

  %\textit{problem here: increase in marginal and conditional R^2 values, perhaps due to adding additional variables into the model}
  %The interaction did not improve the model fit,
  %Nor did adding teamPerformance7 on its own into the model...
  %But an interaction alone was significant

  \subsubsection{Pre- to Post-Tournament change}
  %\subsubsection{2.1.c $\Delta$Team Click $\sim$ $\Delta$Joint Action Success$\times$ Team Performance Expectations}
  Introduction of the interaction effect of change in perceptions of joint action success and expectations around team performance on change in Team Click was not significant, $\beta = .004$ ($95\% CI =  -.002, .01$), $SE = 0.003$, $t(30.14) = 1.42$, $p = .17$, $marginal R^2 = .40$, $conditional R^2 = .49$, and failed to improve the model fit, $\chi^2 = 2.72$, $ p = .44$ (see Table ~\ref{tab:MLM21ccTeamPerfExpcClickInt}).  This result indicates that the extent to which team performance violations were more or less positive did not have an additive effect on the positive relationship between perceptions of Joint Action Success and Team Click.

  
% Table created by stargazer v.5.2 by Marek Hlavac, Harvard University. E-mail: hlavac at fas.harvard.edu
% Date and time: Tue, Jun 27, 2017 - 09:42:53
\begin{table}[!htbp] \centering 
  \caption{cSocialBonding = cTeamClick} 
  \label{tab:MLM22acClickcBonding} 
\begin{tabular}{@{\extracolsep{5pt}}lccc} 
\\[-1.8ex]\hline 
\hline \\[-1.8ex] 
 & \multicolumn{3}{c}{\textit{Dependent variable:}} \\ 
\cline{2-4} 
\\[-1.8ex] & \multicolumn{3}{c}{cSocialBonding} \\ 
\\[-1.8ex] & (1) & (2) & (3)\\ 
\hline \\[-1.8ex] 
 (constant) & 0.34$^{***}$ & 0.31$^{***}$ & 0.22 \\ 
  & (0.09) & (0.08) & (0.31) \\ 
  & & & \\ 
 cTeamClick &  & 0.39$^{***}$ & 0.40$^{***}$ \\ 
  &  & (0.11) & (0.12) \\ 
  & & & \\ 
 objectiveCompetence &  &  & 0.04 \\ 
  &  &  & (0.10) \\ 
  & & & \\ 
 subjectiveCompetence &  &  & $-$0.003 \\ 
  &  &  & (0.09) \\ 
  & & & \\ 
 finalRank &  &  & $-$0.01 \\ 
  &  &  & (0.05) \\ 
  & & & \\ 
 minutesTotal &  &  & 0.003 \\ 
  &  &  & (0.005) \\ 
  & & & \\ 
 pointsTotal &  &  & 0.001 \\ 
  &  &  & (0.01) \\ 
  & & & \\ 
\hline \\[-1.8ex] 
Marginal R-squared &  & .16 & .17 \\ 
Conditional R-squared &  & .20 & .23 \\ 
Observations & 99 & 99 & 97 \\ 
Log Likelihood & $-$129.78 & $-$122.27 & $-$119.44 \\ 
Akaike Inf. Crit. & 265.55 & 256.54 & 260.88 \\ 
Bayesian Inf. Crit. & 273.34 & 272.11 & 289.20 \\ 
\hline 
\hline \\[-1.8ex] 
\textit{Note:}  & \multicolumn{3}{r}{$^{*}$p$<$0.05; $^{**}$p$<$0.01; $^{***}$p$<$0.001} \\ 
\end{tabular} 
\end{table} 


  %Exclusion of outliers according to Tukey's method (observations above and below 1.5 \times Inter Quartile Range (IQR)) appeared to improve the model fit, $\beta = .25$ ($95\% CI =  .11, .48$), $SE = .10$, $t(6.58) = 3.06$, $p < .02$, $marginal R^2 = .18$, $conditional R^2 = .25$. Model residuals were normally distributed, ($Shapiro-Wilk = 0.98, p = .14$),

%\subsubsection{1.c Overall Tournament}
%    NA
\myparagraph{Prediction 1.c: Summary of results}
Results from the analysis of Prediction 1.c fail to provide evidence for an interaction effect of Joint Action Success and Team Performance Expectations on social bonding.  These results suggest that the extent to which overall team performance is experienced as a violation of prior expectations does not have an additive effect on the relationship between perceptions of success in joint action and feelings of team click.

















\subsection{Prediction 2: Team Click predicts Social Bonding}
The second prediction of this study, that higher levels of team click will predict higher levels of social bonding is designed to test the intuition that athletes will feel more bonded to the team and their teammates when they feel the ``click'' successful joint action.  Prediction 2 was tested in the post-Tournament, pre- to post-Tournament, and Overall Tournament survey subsets, using the following model:

  \begin{equation}
    \begin{align*}
      Social Bonding   =& Team Click\\
                      &+ Objective Competence + Subjective Competence  \\
                      &+ Tournament performance measures \\
    \end{align*}
  \end{equation}
  \bigskip

  \myparagraph{Post-Tournament}
The scatterplot in Figure ~\ref{fig:clickBondBasicXY} depicts a strong relationship between Team Click and Social Bonding in the post-Tournament survey.

\begin{figure}[htbp]
\includegraphics[width = \linewidth]{images/clickBondBasicXY.pdf}
  \caption{Team Click predicts Social Bonding in post-Tournament survey data}
  \label{fig:clickBondBasicXY}
\end{figure}

 A LMER model with team included as a random effect and controlling for Tournament performance measures and objective and subjective competence, revealed a significant relationship between team click and social bonding, $\beta = .67$ ($95\% CI =  .51, .84$), $SE = 0.08$, $t(16.01) = 8.22$, $p < .0001$, $marginal R^2 = .49$, $conditional R^2 = .51$ (see Table ~\ref{tab:MLM2aTeamClickBonding}).  Model residuals were normally distributed around zero ($W = 0.98, p = .15$), and individual cases had low influence on the model (Cook's Distances all < .15, see Appendix ~\ref{app5:tournamentSurvey} Figure ~\ref{fig:MKM2aAssumptions} for a full report of model assumptions).  The model supported the prediction that higher levels of team click are associated with higher levels of social bonding.
 The slope of the model is added to scatterplot for comparison with the original line of best fit (see ~\ref{fig:clickBondModelSlope})

  
% Table created by stargazer v.5.2 by Marek Hlavac, Harvard University. E-mail: hlavac at fas.harvard.edu
% Date and time: Mon, Jun 26, 2017 - 20:48:41
\begin{table}[!htbp] \centering 
  \caption{socialBonding = teamClick} 
  \label{tab:MLM2aTeamClickBonding} 
\begin{tabular}{@{\extracolsep{5pt}}lccc} 
\\[-1.8ex]\hline 
\hline \\[-1.8ex] 
 & \multicolumn{3}{c}{\textit{Dependent variable:}} \\ 
\cline{2-4} 
\\[-1.8ex] & \multicolumn{3}{c}{socialBonding} \\ 
\\[-1.8ex] & (1) & (2) & (3)\\ 
\hline \\[-1.8ex] 
 (constant) & $-$0.01 & $-$0.0002 & 0.21 \\ 
  & (0.10) & (0.07) & (0.27) \\ 
  & & & \\ 
 teamClick &  & 0.64$^{***}$ & 0.67$^{***}$ \\ 
  &  & (0.08) & (0.08) \\ 
  & & & \\ 
 objectiveCompetence &  &  & 0.04 \\ 
  &  &  & (0.08) \\ 
  & & & \\ 
 subjectiveCompetence &  &  & 0.12 \\ 
  &  &  & (0.07) \\ 
  & & & \\ 
 finalRank &  &  & $-$0.01 \\ 
  &  &  & (0.04) \\ 
  & & & \\ 
 minutesTotal &  &  & $-$0.003 \\ 
  &  &  & (0.004) \\ 
  & & & \\ 
 pointsTotal &  &  & $-$0.002 \\ 
  &  &  & (0.01) \\ 
  & & & \\ 
\hline \\[-1.8ex] 
Marginal R-squared &  &  & .49 \\ 
Conditional R-squared &  &  & .51 \\ 
Observations & 118 & 118 & 97 \\ 
Log Likelihood & $-$151.95 & $-$118.76 & $-$97.75 \\ 
Akaike Inf. Crit. & 309.90 & 249.53 & 217.50 \\ 
Bayesian Inf. Crit. & 318.21 & 266.15 & 245.82 \\ 
\hline 
\hline \\[-1.8ex] 
\textit{Note:}  & \multicolumn{3}{r}{$^{*}$p$<$0.05; $^{**}$p$<$0.01; $^{***}$p$<$0.001} \\ 
\end{tabular} 
\end{table} 


  \begin{figure}[htbp]
  \includegraphics[width = \linewidth]{images/clickBondModelSlope.pdf}
    \caption{Team Click predicts Social Bonding in post-Tournament survey data. The blue slope is the original line of best fit, the red slope has been adjusted according to the predictions of the linear model.}
    \label{fig:clickBondModelSlope}
  \end{figure}



  \myparagraph{Pre- to Post-Tournament change}
  %\subsubsection{2.2.a $\Delta$Social Bonding$\sim$ $\Delta$Team Click}
To further test prediction 2, change in feelings of social bonding observed in Athletes pre- to post-Tournament was analysed as a function of change in feelings of Team Click. The scatterplot in Figure ~\ref{fig:clickBondDeltaBasicXY} depicts a positive relationship between Change in Team Click and Change in Social Bonding.

    \begin{figure}[htbp]
    \includegraphics[width = \linewidth]{images/clickBondDeltaBasicXY.pdf}
      \caption{Change in Team Click predicts Change in Social Bonding}
      \label{fig:clickBondDeltaBasicXY}
    \end{figure}

A LMER model revealed a significant positive effect of change in team click on change in social bonding, $\beta = .40$ ($95\% CI =  .16, .64$), $SE = .12$, $t(7.02) = 3.31$, $p = .01$, $marginal R^2 = .17$, $conditional R^2 = .23$ (see Table ~\ref{tab:MLM22acClickcBonding} for a full description of model estimates).  Model residuals were normally distributed around zero ($Shapiro-Wilk = 0.98, p = .15$). and individual cases had low influence on the model (Cook's Distances all < .6, see Appendix Figure ~\ref{fig:MLM22aAssumptions} for details). This model suggests that athletes who experienced an increase in feelings of team click also experienced an increase in feelings of social bonding towards their team.  The slope of the model is added to scatterplot for comparison with the original line of best fit (see ~\ref{})

  \begin{figure}[htbp]
  \includegraphics[width = \linewidth]{images/clickBondDeltaModelSlope.pdf}
    \caption{Change in Team Click predicts change in Social Bonding in the pre- to post-Tournament subset of survey data. The blue slope is the original line of best fit, the red slope has been adjusted according to the predictions of the linear model.}
    \label{fig:clickBondDeltaModelSlope}
  \end{figure}


\myparagraph{Overall Tournament}
  %\subsubsection{3.2.a Social Bonding $\sim$ Team Click}
   The relationship between team click and social bonding was tested in the Overall Tournament subset of survey responses. The scatterplot in Figure ~\ref{fig:clickBondDeltaBasicXY} depicts a positive relationship between Change in Team Click and Change in Social Bonding.

        \begin{figure}[htbp]
        \includegraphics[width = \linewidth]{images/clickBondOverallBasicXY.pdf}
          \caption{Change in Team Click predicts Change in Social Bonding}
          \label{fig:clickBondOverallBasicXY}
        \end{figure}

   Controlling for measures of objective and subjective competence and Tournament performance, the model revealed a significant positive relationship between feelings of team click and feelings of social bonding, $\beta = .64$ ($95\% CI = .56, .74$), $SE = .05$, $t(87.4) = 13.84$, $p < .0001$, $marginal R^2 = .49$, $conditional R^2 = .66$ (see Table ~\ref{tab:MLM31bclickBondingTournament} for complete description of model estimates). Model residuals were not normally distributed around zero, ($W = 0.92, p < .00001$), owing to high negative skew ($-1.24$) and high kurtosis (5.12) (see Appendix ~\ref{app5:tournamentSurvey} Figure ~\ref{fig:MLM31bAssumptions}).
   Both log transformation ($W = 0.95, p < .00001$) and outlier removal ($W = 0.96, p < .00001$) procedures improved the model fit marginally, but not within the bounds of normality.  To resolve this assumption violation, the outcome variable was first subjected to outlier-removal, and then subsequently log-transformed, which appeared to improve the distribution of residuals somewhat, ($W = 0.98, p = .0002$, see Table ~\ref{MLM31bclickBondingTournamentModelComparison} for adjusted model comparisons and Appendix Figure ~\ref{fig:MLM31bLogOutAssumptions} for adjusted model residuals).  The failure of this model to converge indicated that it was not a robust model from which to make statistical inferences.

   %confirmed a significant positive effect of team click on social bonding over the course of the Tournament, $\beta = .19$ ($95\% CI =  .15, .23$), $SE = .05$, $t() = 9.89$, $p < .001$, $marginal R^2 = .30$, $conditional R^2 = .43$.

   
% Table created by stargazer v.5.2 by Marek Hlavac, Harvard University. E-mail: hlavac at fas.harvard.edu
% Date and time: Thu, Sep 14, 2017 - 09:39:55
\begin{table}[!htbp] \centering 
  \caption{bondingTournament ~ teamClickTournament} 
  \label{tab:MLM31bclickBondingTournament} 
\footnotesize 
\begin{tabular}{@{\extracolsep{5pt}}lccc} 
\\[-1.8ex]\hline 
\hline \\[-1.8ex] 
 & \multicolumn{3}{c}{\textit{Dependent variable:}} \\ 
\cline{2-4} 
\\[-1.8ex] & (1) & (2) & (3)\\ 
\hline \\[-1.8ex] 
 (constant) & $-$0.00 & $-$1.46$^{***}$ & $-$1.59$^{***}$ \\ 
  & (0.04) & (0.08) & (0.15) \\ 
  & & & \\ 
 teamPerformanceExpectations &  & 0.02$^{***}$ & 0.02$^{***}$ \\ 
  &  & (0.001) & (0.001) \\ 
  & & & \\ 
 indPerformanceExpectations &  &  & 0.004$^{**}$ \\ 
  &  &  & (0.002) \\ 
  & & & \\ 
 objectiveCompetence &  &  & 0.02 \\ 
  &  &  & (0.04) \\ 
  & & & \\ 
 subjectiveCompetence &  &  & 0.08$^{*}$ \\ 
  &  &  & (0.04) \\ 
  & & & \\ 
 finalRank &  &  & $-$0.0002 \\ 
  &  &  & (0.02) \\ 
  & & & \\ 
 minutesTotal &  &  & $-$0.001 \\ 
  &  &  & (0.002) \\ 
  & & & \\ 
 pointsTotal &  &  & 0.003 \\ 
  &  &  & (0.003) \\ 
  & & & \\ 
\hline \\[-1.8ex] 
Marginal R-squared & .47 & .49 &  \\ 
Conditional R-squared & .66 & .66 &  \\ 
Observations & 564 & 444 & 331 \\ 
Log Likelihood & $-$772.35 & $-$412.33 & $-$303.82 \\ 
Akaike Inf. Crit. & 1,552.70 & 842.67 & 637.65 \\ 
Bayesian Inf. Crit. & 1,570.04 & 879.53 & 694.68 \\ 
\hline 
\hline \\[-1.8ex] 
\textit{Note:}  & \multicolumn{3}{r}{$^{*}$p$<$0.05; $^{**}$p$<$0.01; $^{***}$p$<$0.001} \\ 
\end{tabular} 
\end{table} 


   
% Table created by stargazer v.5.2 by Marek Hlavac, Harvard University. E-mail: hlavac at fas.harvard.edu
% Date and time: Thu, Sep 14, 2017 - 09:39:58
\begin{table}[!htbp] \centering 
  \caption{Model Adjustment Comparison:bondingTournament ~ teamClickTournament} 
  \label{MLM31bclickBondingTournamentModelComparison} 
\tiny 
\begin{tabular}{@{\extracolsep{5pt}}lcccc} 
\\[-1.8ex]\hline 
\hline \\[-1.8ex] 
 & \multicolumn{4}{c}{\textit{Dependent variable:}} \\ 
\cline{2-5} 
 & model & log-transformed & outliers removed & outliers + log-transformed \\ 
\\[-1.8ex] & (1) & (2) & (3) & (4)\\ 
\hline \\[-1.8ex] 
 (constant) & $-$1.59$^{***}$ & 1.63$^{***}$ & 0.31$^{***}$ & 1.50$^{***}$ \\ 
  & (0.15) & (0.02) & (0.08) & (0.04) \\ 
  & & & & \\ 
 teamPerformanceExpectations & 0.02$^{***}$ &  &  &  \\ 
  & (0.001) &  &  &  \\ 
  & & & & \\ 
 indPerformanceExpectations & 0.004$^{**}$ &  &  &  \\ 
  & (0.002) &  &  &  \\ 
  & & & & \\ 
 objectiveCompetence &  & 0.16$^{***}$ & 0.42$^{***}$ & 0.19$^{***}$ \\ 
  &  & (0.01) & (0.04) & (0.02) \\ 
  & & & & \\ 
 subjectiveCompetence & 0.02 & 0.01 & 0.01 & 0.01 \\ 
  & (0.04) & (0.01) & (0.03) & (0.01) \\ 
  & & & & \\ 
 finalRank & 0.08$^{*}$ & 0.01 & 0.03 & 0.02 \\ 
  & (0.04) & (0.01) & (0.02) & (0.01) \\ 
  & & & & \\ 
 minutesTotal & $-$0.0002 & $-$0.004 & $-$0.01 & $-$0.01 \\ 
  & (0.02) & (0.003) & (0.01) & (0.005) \\ 
  & & & & \\ 
 pointsTotal & $-$0.001 & $-$0.001 & $-$0.002 & $-$0.001 \\ 
  & (0.002) & (0.0004) & (0.001) & (0.001) \\ 
  & & & & \\ 
 pointsTotal & 0.003 & $-$0.001 & $-$0.001 & $-$0.001 \\ 
  & (0.003) & (0.001) & (0.002) & (0.001) \\ 
  & & & & \\ 
\hline \\[-1.8ex] 
Marginal R-squared & .49 & .50 & .23 & .23 \\ 
Conditional R-squared & .66 & .61 & .35 & .34 \\ 
Shapiro-Wilk Test (p-value) & .92(<.00000000001) & .95(<.000001) & .96(<.00001) & .98(.0002) \\ 
Observations & 331 & 449 & 405 & 405 \\ 
Log Likelihood & $-$303.82 & 225.59 & $-$241.97 & 79.71 \\ 
Akaike Inf. Crit. & 637.65 & $-$423.18 & 511.94 & $-$131.42 \\ 
Bayesian Inf. Crit. & 694.68 & $-$365.69 & 567.99 & $-$75.37 \\ 
\hline 
\hline \\[-1.8ex] 
\textit{Note:}  & \multicolumn{4}{r}{$^{*}$p$<$0.05; $^{**}$p$<$0.01; $^{***}$p$<$0.001} \\ 
\end{tabular} 
\end{table} 



\myparagraph{Summary of results}
   Results from the analysis of Prediction 1.b suggest that following the Tournament, more positive violation of expectations surrounding team performance correlate with higher levels of self-reported team click.  In addition, more positive violations of expectations following the Tournament are associated with an increase in feelings of Team Click between pre- and post-Tournament surveys, suggesting that athletes who perceived an increase in success of components of team performance also experienced an increase in feelings of team click.  Finally, the positive relationship between Overall Tournament measures of team performance expectations and team click was also significant.






\subsection{Prediction 3: More positive perceptions of joint action success will predict higher levels of social bonding}



\subsubsection{Prediction 3.a: More positive perceptions of technical components of team performance will predict higher levels of social bonding }
In light of evidence for significant relationships between perceptions of joint action success and team click, and feelings of team click and social bonding, the predicted direct relationship between joint action success and social bonding was assessed in the post-Tournament and pre- to post-Tournament survey subsets.

\myparagraph{post-Tournament}
A scatterplot depicting the relationship between perceptions of components of team performance (Joint Action Success) and Social Bonding in the post-Tournament survey data is shown in Figure ~\ref{fig:jasClickBasicXY}. The plot indicates a strong positive relationship between Joint Action Success and Social Bonding.

\begin{figure}[htbp]
\includegraphics[width = \linewidth]{images/jasBondBasicXY.pdf}
  \caption{Joint Action Success predicts Social Bonding (post-Tournament)}
  \label{fig:jasBondBasicXY}
\end{figure}

Controlling for perceptions of success in individual performance, objective and subjective competence, and Tournament performance measures, the model revealed a significant effect of joint-action success on social bonding, $\beta = .45$ ($95\% CI =  .17, .73$), $SE = .14$, $t(23.4) = 3.19$, $p < .01$, $marginal R^2 = .27$, $conditional R^2 = .42$.  The model also revealed a significant positive effect of subjective measures of competence on social bonding, $\beta = .18$ ($95\% CI =  .03, .35$), $SE = .08$, $t(90.86) = 2.26$, $p = .03$ (full results in Table ~\ref{tab:MLM3aJointActionSuccessBonding}).

Model residuals were non-normally distributed ($W = 0.95, p = .0007$), owing to relatively large negative skew (-.87) (see Appendix Figure ~\ref{fig:MLM3aAssumptions}).
Exclusion of outliers appeared to improve the model fit, $\beta = .26$ ($95\% CI =  .05, .47$), $SE = 0.004$, $t(9.78) = 3.66$, $p < .01$, $marginal R^2 = .18$, $conditional R^2 = .34$.  The distribution of model residuals appeared to improve, but still violated the assumption of normality ($Shapiro-Wilk = 0.96, p = .009$).  Individual cases had low influence on the model (Cook's Distances all < .10).

Due to the \textit{positive} skew of the model residuals, the outcome variable was transformed by taking the log of the reversed scores of the outcome variable, i.e. $log10(k - y)$, where $k$ is a constant value from which each score for $y$ is subtracted so that the distribution of the outcome variable is reversed\citep{Howell2012}.\footnote{Reversing the distribution of the outcome variable allows the logarithmic function to normalise the distribution of the variable, by pushing them from the left hand side of the distribution towards the centre}  Transformed values were then returned to their original direction for analysis\citep{Field2012}.  Rebuilding the model with a log-transformed outcome variable appeared to improve the fit of the model more than the outlier-removed model, and avoided the removal of any observations.

The distribution of residuals appeared more normal, ($W = 0.97, p = .04$), and the R-squared values for the model improved, $marginal R^2 = .27$, $conditional R^2 = .43$ (see Table ~\ref{tab:MLM3aJointActionSuccessBonding} for results of all three models, and Appendix ~\ref{app5:tournamentSurvey} Figures ~\ref{fig:MLM3aAssumptions}\nobreakdash~\ref{fig:MLM3aLogAssumptions}). Results of the log-transformed model were taken as satisfactory evidence for a significant positive relationship between perceptions of joint-action success and feelings of social bonding.  The a line of best fit adjusted according to (log-transformed) model predictions is included in Figure ~\ref{fig:jasBondModelSlope} for comparison.
%While Levene's Test for Equality of Variance indicated that the assumption of homoscedasticity was met at the group-level, $F(13,83) = .80, p = .66$,
  
% Table created by stargazer v.5.2 by Marek Hlavac, Harvard University. E-mail: hlavac at fas.harvard.edu
% Date and time: Mon, Jun 26, 2017 - 21:18:41
\begin{table}[!htbp] \centering 
  \caption{socialBonding = jointActionSuccess} 
  \label{tab:MLM3aJointActionSuccessBonding} 
\footnotesize 
\begin{tabular}{@{\extracolsep{5pt}}lccc} 
\\[-1.8ex]\hline 
\hline \\[-1.8ex] 
 & \multicolumn{3}{c}{\textit{Dependent variable:}} \\ 
\cline{2-4} 
\\[-1.8ex] & socialBonding & bondingPostFactorOut & bondingPostFactorLogReturned \\ 
 &  & outliers removed & log-transformed \\ 
\\[-1.8ex] & (1) & (2) & (3)\\ 
\hline \\[-1.8ex] 
 (constant) & $-$0.06 & $-$0.06 & 1.97$^{***}$ \\ 
  & (0.31) & (0.31) & (0.13) \\ 
  & & & \\ 
 jointActionSuccess & 0.45$^{**}$ & 0.45$^{**}$ & 0.20$^{***}$ \\ 
  & (0.14) & (0.14) & (0.06) \\ 
  & & & \\ 
 indPerformanceSuccess & 0.05 & 0.05 & $-$0.001 \\ 
  & (0.11) & (0.11) & (0.05) \\ 
  & & & \\ 
 objectiveCompetence & 0.07 & 0.07 & 0.04 \\ 
  & (0.10) & (0.10) & (0.04) \\ 
  & & & \\ 
 subjectiveCompetence & 0.19$^{*}$ & 0.19$^{*}$ & 0.09$^{*}$ \\ 
  & (0.08) & (0.08) & (0.03) \\ 
  & & & \\ 
 finalRank & $-$0.02 & $-$0.02 & $-$0.01 \\ 
  & (0.05) & (0.05) & (0.02) \\ 
  & & & \\ 
 minutesTotal & 0.002 & 0.002 & 0.001 \\ 
  & (0.004) & (0.004) & (0.002) \\ 
  & & & \\ 
 pointsTotal & $-$0.001 & $-$0.001 & $-$0.002 \\ 
  & (0.01) & (0.01) & (0.003) \\ 
  & & & \\ 
\hline \\[-1.8ex] 
Marginal R-squared & .27 & .18 & .28 \\ 
Conditional R-squared & .42 & .34 & .43 \\ 
Observations & 97 & 97 & 97 \\ 
Log Likelihood & $-$113.05 & $-$113.05 & $-$29.49 \\ 
Akaike Inf. Crit. & 250.10 & 250.10 & 82.98 \\ 
Bayesian Inf. Crit. & 281.00 & 281.00 & 113.87 \\ 
\hline 
\hline \\[-1.8ex] 
\textit{Note:}  & \multicolumn{3}{r}{$^{*}$p$<$0.05; $^{**}$p$<$0.01; $^{***}$p$<$0.001} \\ 
\end{tabular} 
\end{table} 



  \begin{figure}[htbp]
  \includegraphics[width = \linewidth]{images/jasBondModelSlope.pdf}
    \caption{Joint Action Success predicts Social Bonding in post-Tournament survey data. The blue slope is the original line of best fit, the red slope has been adjusted according to the predictions of the linear model.}
    \label{fig:jasBondModelSlope}
  \end{figure}








\myparagraph{Pre- to Post-Tournament change}
%\subsubsection{2.3.a $\Delta$Social Bonding $\sim$ $\Delta$Joint Action Success}
Next, change in social bonding pre-post Tournament as a function of change in Joint Action Success was tested. A scatterplot depicting the relationship between change in Joint Action Success and change in Social Bonding is shown in Figure ~\ref{fig:jasBondDeltaBasicXY}. The plot indicates a strong positive relationship between Joint Action Success and Social Bonding.

\begin{figure}[htbp]
\includegraphics[width = \linewidth]{images/jasBondDeltaBasicXY.pdf}
  \caption{Joint Action Success predicts Social Bonding (post-Tournament)}
  \label{fig:jasBondDeltaBasicXY}
\end{figure}



Results of a LMER model revealed a significant positive effect of change in Joint Action Success on changes in Social Bonding, $\beta = .38$ ($95\% CI =  .21, .56$), $SE = .09$, $t(97) = 4.35$, $p < .0001$, $marginal R^2 = .18$, $conditional R^2 = .18$, suggesting that on average, athletes who experienced an increase in positive perceptions of joint-action success as a result of the Tournament also experienced an increase in feelings of social bonding to their team and teammates.  Interestingly, the model also revealed a significant \textit{negative} effect of perceptions of success in individual component performance and social bonding, $\beta = -.23$ ($95\% CI =  -.44, -.02$), $SE = .11$, $t(97) = -2.12$, $p = .04$, indicating that athletes whose attitudes towards their own performance increased in positivity showed an average decrease in feelings of social bonding between pre- and post-Tournament measures. All other fixed effects were not significant, see Table ~\ref{tab:MLM23acJointActionSuccesscBonding}. Model residuals were normally distributed around zero ($W = 0.98, p = .19$), and individual cases had low influence on the model (Cook's Distances all < .5, see Appendix Figure ~\ref{fig:MLM23aAssumptions}.  This model supported the prediction that higher perceptions of joint action success would significantly relate to higher levels of social bonding, by showing that increases in perception of joint action success significantly predicted increases in feelings of social bonding before and after the Tournament.


  
% Table created by stargazer v.5.2 by Marek Hlavac, Harvard University. E-mail: hlavac at fas.harvard.edu
% Date and time: Tue, Jun 27, 2017 - 17:13:12
\begin{table}[!htbp] \centering 
  \caption{cSocialBonding ~ cJointActionSuccess} 
  \label{tab:MLM23acJointActionSuccesscBonding} 
\footnotesize 
\begin{tabular}{@{\extracolsep{5pt}}lccc} 
\\[-1.8ex]\hline 
\hline \\[-1.8ex] 
 & \multicolumn{3}{c}{\textit{Dependent variable:}} \\ 
\cline{2-4} 
\\[-1.8ex] & \multicolumn{3}{c}{cSocialBonding} \\ 
\\[-1.8ex] & (1) & (2) & (3)\\ 
\hline \\[-1.8ex] 
 (constant) & 0.34$^{***}$ & 0.35$^{***}$ & 0.13 \\ 
  & (0.09) & (0.09) & (0.31) \\ 
  & & & \\ 
 cJointActionSuccess &  & 0.24$^{***}$ & 0.38$^{***}$ \\ 
  &  & (0.07) & (0.09) \\ 
  & & & \\ 
 cIndPerformanceSuccess &  &  & $-$0.23$^{*}$ \\ 
  &  &  & (0.11) \\ 
  & & & \\ 
 objectiveCompetence &  &  & 0.14 \\ 
  &  &  & (0.11) \\ 
  & & & \\ 
 subjectiveCompetence &  &  & 0.03 \\ 
  &  &  & (0.09) \\ 
  & & & \\ 
 finalRank &  &  & $-$0.04 \\ 
  &  &  & (0.05) \\ 
  & & & \\ 
 minutesTotal &  &  & 0.01 \\ 
  &  &  & (0.005) \\ 
  & & & \\ 
 pointsTotal &  &  & 0.003 \\ 
  &  &  & (0.01) \\ 
  & & & \\ 
\hline \\[-1.8ex] 
Marginal R-squared &  & .10 & .18 \\ 
Conditional R-squared &  & .10 & .18 \\ 
Observations & 99 & 99 & 97 \\ 
Log Likelihood & $-$129.78 & $-$124.37 & $-$118.30 \\ 
Akaike Inf. Crit. & 265.55 & 260.75 & 260.60 \\ 
Bayesian Inf. Crit. & 273.34 & 276.32 & 291.50 \\ 
\hline 
\hline \\[-1.8ex] 
\textit{Note:}  & \multicolumn{3}{r}{$^{*}$p$<$0.05; $^{**}$p$<$0.01; $^{***}$p$<$0.001} \\ 
\end{tabular} 
\end{table} 




  \subsubsection{3.a Overall Tournament}
  NA



  3) more positive perceptions of joint action success will predict higher levels of social bonding, driven by more positive 3.a) perceptions of components of team performance, or 3.b) violation of team performance expectations, or 3.b) an interaction between the two predictors.
  Joint Action Success predicts Social Bonding}




\subsection{Prediction 3.b: Team Performance Expectations predicts Social Bonding}

  \subsubsection{3.b Post-Tournament}

  The direct relationship between expectations around team performance and social bonding was also tested.  The model revealed a significant positive relationship between Team Performance Expectations and Social Bonding, $\beta = .34$ ($95\% CI =  .06, .63$), $SE = .14$, $t(13.79) = 2.38$, $p = .03$, $marginal R^2 = .20$, $conditional R^2 = .40$.  The model also revealed a significant positive effect of Subjective Competence on Social Bonding, $\beta = .19$ ($95\% CI =  .02, .36$), $SE = .08$, $t(90.37) = 2.24$, $p = .03$, such that athletes who provided higher ratings of their own technical competence in rugby (measured before the Tournament) reported higher levels of social bonding.

  Model residuals were not normally distributed, ($W = 0.91, p < .0001$), owing to large negative skew ($-1.33$) and high kurtosis ($.49$). Re-running the model with a log-transformed outcome variable appeared to make the best improvement to model residuals more than the outlier-removed model (see Table ~\ref{tab:MLM3bExpectationsBonding}). In the log-transformed model, the distribution of model residuals appeared most normal,  ($W = 0.97, p = .02$), and the R-squared values for the model improved, $marginal R^2 = .23$, $conditional R^2 = .44$.  Individual cases had low influence on the model (Cook's Distances all < .15, see Appendix Figures ~\ref{fig:MLM3bAssumptions} and ~\ref{fig:MLM3bLogAssumptions} for a comparison of model assumptions between the original and log-transformed model). Owing to non-normally distributed residuals, this model does not provide robust support for the prediction that team performance expectation violations relates directly to social bonding.

  Results outlined above demonstrate significant relationships between Joint-Action Success and Team Click, Team Click and Social Bonding, and a direct relationship between Joint-Action Success and Social Bonding. Results of models linking Team Performance Expectations, Team Click, and Social Bonding were lest robust, and should be treated with caution.


  %Outlier exclusion
  %Exclusion of outliers improved the distribution of residuals, ($W = 0.96, p = .007$), and individual cases had low influence on the model (Cook's Distances all < .15).  The significant positive effect of teamPerformanceExpectations on socialBonding remained in the adjusted model, $\beta = .01$ ($95\% CI =  .003, .02$), $SE = .004$, $t() = 2.65$, $p = .03$, $marginal R^2 = .19$, $conditional R^2 = .40$.  The effect of Subjective Competence on socialBonding was no longer significant, $\beta = .12$ ($95\% CI =  -.0004, .24$), $SE = .06$, $t(90.44) = 1.95$, $p = .06$.

  \newgeometry{margin=0.5cm}
  
% Table created by stargazer v.5.2 by Marek Hlavac, Harvard University. E-mail: hlavac at fas.harvard.edu
% Date and time: Tue, Jun 27, 2017 - 17:01:54
\begin{table}[!htbp] \centering 
  \caption{M3.b socialBonding = teamPerformanceExpectations} 
  \label{tab:MLM3bExpectationsBonding} 
\scriptsize 
\begin{tabular}{@{\extracolsep{5pt}}lccc} 
\\[-1.8ex]\hline 
\hline \\[-1.8ex] 
 & \multicolumn{3}{c}{\textit{Dependent variable:}} \\ 
\cline{2-4} 
\\[-1.8ex] & socialBonding & bondingPostFactorOut & bondingPostFactorLogReturned \\ 
 &  & outliers removed & log-transformed \\ 
\\[-1.8ex] & (1) & (2) & (3)\\ 
\hline \\[-1.8ex] 
 (constant) & $-$1.67$^{*}$ & $-$1.18$^{*}$ & 1.19$^{***}$ \\ 
  & (0.73) & (0.53) & (0.31) \\ 
  & & & \\ 
 teamPerformanceExpectations & 0.01$^{*}$ & 0.01$^{**}$ & 0.01$^{**}$ \\ 
  & (0.01) & (0.004) & (0.003) \\ 
  & & & \\ 
 individualPerformanceExpectations & 0.004 & 0.001 & 0.001 \\ 
  & (0.004) & (0.003) & (0.002) \\ 
  & & & \\ 
 objectiveCompetence & 0.03 & 0.02 & 0.02 \\ 
  & (0.10) & (0.07) & (0.04) \\ 
  & & & \\ 
 subjectiveCompetence & 0.19$^{*}$ & 0.12 & 0.08$^{*}$ \\ 
  & (0.09) & (0.06) & (0.04) \\ 
  & & & \\ 
 finalRank & 0.08 & 0.10 & 0.04 \\ 
  & (0.13) & (0.09) & (0.05) \\ 
  & & & \\ 
 minutesTotal & 0.0003 & 0.001 & 0.0001 \\ 
  & (0.004) & (0.003) & (0.002) \\ 
  & & & \\ 
 pointsTotal & $-$0.04 & $-$0.06 & $-$0.03 \\ 
  & (0.09) & (0.06) & (0.04) \\ 
  & & & \\ 
 pointsTotal & $-$0.001 & $-$0.004 & $-$0.002 \\ 
  & (0.01) & (0.005) & (0.003) \\ 
  & & & \\ 
\hline \\[-1.8ex] 
Marginal R-squared & .20 & .19 & .23 \\ 
Conditional R-squared & .40 & .40 & .44 \\ 
Observations & 97 & 91 & 97 \\ 
Log Likelihood & $-$117.26 & $-$77.84 & $-$32.42 \\ 
Akaike Inf. Crit. & 260.52 & 181.67 & 90.85 \\ 
Bayesian Inf. Crit. & 293.99 & 214.31 & 124.32 \\ 
\hline 
\hline \\[-1.8ex] 
\textit{Note:}  & \multicolumn{3}{r}{$^{*}$p$<$0.05; $^{**}$p$<$0.01; $^{***}$p$<$0.001} \\ 
\end{tabular} 
\end{table} 

  \restoregeometry

  \subsubsection{3.b Pre- to Post-Tournament change}

  %\subsubsection{2.3.b $\Delta$Social Bonding $\sim$ Team Performance Expectations}
  A model designed to test the relationship between expectations around team performance and change in social bonding revealed a marginally significant main effect, $\beta = .20$ ($95\% CI =  -.0004, .02$), $SE = .11$, $t(97) = 1.87$, $p = .07$, $marginal R^2 = .05$, $conditional R^2 = .05$.  Examination of model residuals revealed that they were not normally distributed around zero ($Shapiro-Wilk = 0.96, p = .004$).

  Re-running the model with a log-transformed outcome variable and outliers excluded improved the normality of residuals, ($Shapiro-Wilk = 0.98, p = .24$), however the effect of Team Performance Expectations on Social Bonding was no longer significant, $\beta = .04$ ($95\% CI =  .02, .10$), $SE = .03$, $t()) = 1.29$, $p = .01$, $marginal R^2 = .03$, $conditional R^2 = .09$ (see Table ~\ref{tab:MLM23bcBondingteamPerfExp} for full description of results).  These results from adjusted models suggests that variation in Team Performance Expectations across pre- and post-Tournament measurements did not suitably explain observed variation in pre-post measures of Social Bonding.  As such, Team Performance Expectations was not considered in the subsequent mediation analysis.



  \newgeometry{margin=0.5cm}
  
% Table created by stargazer v.5.2 by Marek Hlavac, Harvard University. E-mail: hlavac at fas.harvard.edu
% Date and time: Tue, Jun 27, 2017 - 17:15:15
\begin{table}[!htbp] \centering 
  \caption{cSocialBonding ~ teamPerformanceExpectations} 
  \label{tab:MLM23bcBondingteamPerfExp} 
\scriptsize 
\begin{tabular}{@{\extracolsep{5pt}}lccccc} 
\\[-1.8ex]\hline 
\hline \\[-1.8ex] 
 & \multicolumn{5}{c}{\textit{Dependent variable:}} \\ 
\cline{2-6} 
\\[-1.8ex] & \multicolumn{3}{c}{cSocialBonding} & bondingPostFactorLogReturned & bondingFactorChangePrePostOut \\ 
\\[-1.8ex] & (1) & (2) & (3) & (4) & (5)\\ 
\hline \\[-1.8ex] 
 (constant) & 0.34$^{***}$ & $-$0.07 & 0.01 & 1.34$^{***}$ & $-$0.01 \\ 
  & (0.09) & (0.25) & (0.39) & (0.22) & (0.33) \\ 
  & & & & & \\ 
 teamPerformanceExpectations &  & 0.01$^{*}$ & 0.01$^{*}$ & 0.01$^{***}$ & 0.005 \\ 
  &  & (0.004) & (0.004) & (0.003) & (0.004) \\ 
  & & & & & \\ 
 indPerformanceExpectations &  &  & $-$0.004 & 0.001 & $-$0.0000 \\ 
  &  &  & (0.004) & (0.002) & (0.003) \\ 
  & & & & & \\ 
 objectiveCompetence &  &  & 0.03 & 0.01 & 0.04 \\ 
  &  &  & (0.11) & (0.04) & (0.08) \\ 
  & & & & & \\ 
 subjectiveCompetence &  &  & $-$0.06 & 0.09$^{**}$ & $-$0.03 \\ 
  &  &  & (0.10) & (0.04) & (0.07) \\ 
  & & & & & \\ 
 finalRank &  &  & $-$0.03 & 0.01 & $-$0.03 \\ 
  &  &  & (0.05) & (0.02) & (0.04) \\ 
  & & & & & \\ 
 minutesTotal &  &  & 0.004 & 0.0003 & 0.001 \\ 
  &  &  & (0.005) & (0.002) & (0.003) \\ 
  & & & & & \\ 
 pointsTotal &  &  & 0.002 & $-$0.002 & $-$0.0003 \\ 
  &  &  & (0.01) & (0.003) & (0.01) \\ 
  & & & & & \\ 
\hline \\[-1.8ex] 
Marginal R-squared &  & .03 & .05 & .23 &  \\ 
Conditional R-squared &  & .03 & .05 & .46 &  \\ 
Observations & 99 & 99 & 97 & 97 & 86 \\ 
Log Likelihood & $-$129.78 & $-$128.30 & $-$125.20 & $-$32.67 & $-$75.13 \\ 
Akaike Inf. Crit. & 265.55 & 268.60 & 274.39 & 89.34 & 174.25 \\ 
Bayesian Inf. Crit. & 273.34 & 284.17 & 305.29 & 120.24 & 203.70 \\ 
\hline 
\hline \\[-1.8ex] 
\textit{Note:}  & \multicolumn{5}{r}{$^{*}$p$<$0.1; $^{**}$p$<$0.05; $^{***}$p$<$0.01} \\ 
\end{tabular} 
\end{table} 

% Table created by stargazer v.5.2 by Marek Hlavac, Harvard University. E-mail: hlavac at fas.harvard.edu
% Date and time: Tue, Jun 27, 2017 - 17:15:15
\begin{table}[!htbp] \centering 
  \caption{cSocialBonding ~ teamPerformanceExpectations} 
  \label{tab:MLM23bcBondingteamPerfExp} 
\scriptsize 
\begin{tabular}{@{\extracolsep{5pt}} ccc} 
\\[-1.8ex]\hline 
\hline \\[-1.8ex] 
$0.05$ & $0.01$ & $0.001$ \\ 
\hline \\[-1.8ex] 
\end{tabular} 
\end{table} 

  \restoregeometry





  \subsubsection{3.b Overall Tournament}
  %\subsubsection{3.3.b Social Bonding $\sim$ Team Performance Expectations}
  The direct relationship between Team Performance Expectations and social bonding was tested.  The initial model failed to converge with the random slope and intercept model structure.  As such, the model was simplified to estimate only the random slope. This simplified model revealed a significant positive relationship between team performance expectation violation and social bonding, $\beta = .46$ ($95\% CI =  .35, .57$), $SE = .05$, $t(331.00) = 8.48$, $p < .0001$, $marginal R^2 = .40$, $conditional R^2 = .40$.
  The model also revealed a significant relationship between Individual Performance Expectations and social bonding, $\beta = .005$ ($95\% CI =  .002, .009$), $SE = .002$, $t(303.4) = 2.80$, $p < .01$, as well as subjective competence and social bonding, $\beta = .01$ ($95\% CI =  .01, .18$), $SE = .04$, $t(265.3) = 2.24$, $p = .03$ (see Table ~\ref{tab:MLM32ateamPerfBondingTournament} for full description of model estimates).

  Model residuals were non-normally distributed, ($W = 0.97, p < .00001$), with negative skew ($-.6$), and higher than normal kurtosis, see Appendix Figure ~\ref{fig:MLM32aAssumptions}.  A model in which the outcome was log-transformed following removal of outliers provided the best possible fit for the available data ~\ref{MLM32ateamPerfBondingTournamentModelComparison}. While the distribution of errors was still non-normal, ($W = 0.99, p = .03$),  error terms appear much more evenly distributed around zero than the original model, albeit with a slight negative skew ($-0.1$).
  The adjusted model confirmed the significant positive effect of team performance expectation violation on social bonding,  $\beta = .16$ ($95\% CI =  .12, .20$), $SE = .02$, $t(336.80) = 7.98$, $p < .0001$, $marginal R^2 = .36$, $conditional R^2 = .36$.

   
\begin{table}
\begin{center}
\begin{tabular}{l c c c c }
\toprule
 & Main effect & Controls & Log-transformed & Log and outliers \\
\midrule
(constant)                        & $-0.00$               & $\mathbf{0.51}^{*}$   & $\mathbf{1.71}^{***}$ & $\mathbf{1.64}^{***}$ \\
                                  & $(0.04)$              & $(0.26)$              & $(0.06)$              & $(0.08)$              \\
Team Performance Vs Expectations  & $\mathbf{0.59}^{***}$ & $\mathbf{0.47}^{***}$ & $\mathbf{0.11}^{***}$ & $\mathbf{0.07}^{***}$ \\
                                  & $(0.04)$              & $(0.06)$              & $(0.01)$              & $(0.02)$              \\
Ind Performance Vs Expectations   &                       & $\mathbf{0.25}^{***}$ & $\mathbf{0.05}^{***}$ & $0.01$                \\
                                  &                       & $(0.06)$              & $(0.01)$              & $(0.02)$              \\
Objective Competence              &                       & $0.07$                & $0.02$                & $0.01$                \\
                                  &                       & $(0.05)$              & $(0.01)$              & $(0.02)$              \\
Subjective Competence             &                       & $0.09$                & $\mathbf{0.02}^{*}$   & $\mathbf{0.03}^{*}$   \\
                                  &                       & $(0.05)$              & $(0.01)$              & $(0.01)$              \\
Final Rank                        &                       & $-0.03$               & $-0.01$               & $-0.01$               \\
                                  &                       & $(0.02)$              & $(0.01)$              & $(0.01)$              \\
Minutes Total                   &                       & $-0.01$  & $-0.00$  & $-0.00$               \\
                                  &                       & $(0.00)$              & $(0.00)$              & $(0.00)$              \\
Points Total                      &                       & $0.00$                & $0.00$                & $0.00$                \\
                                  &                       & $(0.00)$              & $(0.00)$              & $(0.00)$              \\
Fatigue                           &                       & $0.00$                & $0.00$                & $0.00$                \\
                                  &                       & $(0.00)$              & $(0.00)$              & $(0.00)$              \\
Extraverted                       &                       & $-0.04$               & $-0.01$               & $-0.02$               \\
                                  &                       & $(0.03)$              & $(0.01)$              & $(0.01)$              \\
\midrule
AIC                               & 1083.79               & 752.42                & -131.94               & -8.28                 \\
BIC                               & 1104.27               & 800.83                & -83.53                & 38.54                 \\
Log Likelihood                    & -536.89               & -363.21               & 78.97                 & 17.14                 \\
Num. obs.                         & 444                   & 306                   & 306                   & 271                   \\
\bottomrule
\multicolumn{5}{l}{\scriptsize{Coefficients with $p < 0.05$ in \textbf{bold}. Effect sizes of the Log and outlier model: Marginal $R^2 = .36$, Conditional $R^2 = .36$}}
\end{tabular}
\caption{Prediction 3: Team Performance Vs Expectations predicts Social Bonding in the Overall Tournament survey data (n = 90).}
\label{tab:MLM32ateamPerfBondingTournament}
\end{center}
\end{table}


   
% Table created by stargazer v.5.2 by Marek Hlavac, Harvard University. E-mail: hlavac at fas.harvard.edu
% Date and time: Thu, Sep 14, 2017 - 09:42:11
\begin{table}[!htbp] \centering 
  \caption{Model Comparison: M3.2a socialBondingTournament ~ teamPerformanceExpectationsTournament} 
  \label{tab:MLM32ateamPerfBondingTournamentModelComparison} 
\scriptsize 
\begin{tabular}{@{\extracolsep{5pt}}lccc} 
\\[-1.8ex]\hline 
\hline \\[-1.8ex] 
 & \multicolumn{3}{c}{\textit{Dependent variable:}} \\ 
\cline{2-4} 
 & model & log-transformed & outliers+log-transformed \\ 
\\[-1.8ex] & (1) & (2) & (3)\\ 
\hline \\[-1.8ex] 
 (constant) & $-$0.70$^{***}$ & 1.42$^{***}$ & 1.42$^{***}$ \\ 
  & (0.18) & (0.04) & (0.06) \\ 
  & & & \\ 
 teamPerformanceExpectations & 0.01$^{***}$ & 0.003$^{***}$ & 0.003$^{***}$ \\ 
  & (0.002) & (0.0005) & (0.001) \\ 
  & & & \\ 
 indPerformanceExpectations & 0.01$^{**}$ & 0.001$^{**}$ & 0.0001 \\ 
  & (0.002) & (0.0004) & (0.001) \\ 
  & & & \\ 
 objectiveCompetence & 0.04 & 0.01 & 0.01 \\ 
  & (0.05) & (0.01) & (0.02) \\ 
  & & & \\ 
 subjectiveCompetence & 0.10$^{*}$ & 0.03$^{*}$ & 0.03$^{*}$ \\ 
  & (0.04) & (0.01) & (0.01) \\ 
  & & & \\ 
 finalRank & $-$0.04 & $-$0.01 & $-$0.01 \\ 
  & (0.02) & (0.005) & (0.01) \\ 
  & & & \\ 
 minutesTotal & $-$0.003 & $-$0.001 & $-$0.001 \\ 
  & (0.002) & (0.001) & (0.001) \\ 
  & & & \\ 
 pointsTotal & 0.001 & 0.0001 & 0.0000 \\ 
  & (0.004) & (0.001) & (0.001) \\ 
  & & & \\ 
\hline \\[-1.8ex] 
Marginal R-squared & .33 & .11 & .09 \\ 
Conditional R-squared & .53 & .23 & .17 \\ 
Shapiro-Wilk Test (p-value) & .97($<$.00001) & .96($<$.00001) & .97($<$.00001) \\ 
Observations & 331 & 331 & 294 \\ 
Log Likelihood & $-$373.85 & 96.45 & 21.56 \\ 
Akaike Inf. Crit. & 777.70 & $-$162.91 & $-$13.13 \\ 
Bayesian Inf. Crit. & 834.73 & $-$105.88 & 42.13 \\ 
\hline 
\hline \\[-1.8ex] 
\textit{Note:}  & \multicolumn{3}{r}{$^{*}$p$<$0.05; $^{**}$p$<$0.01; $^{***}$p$<$0.001} \\ 
\end{tabular} 
\end{table} 







\subsection{Prediction 4.a: Team Click mediates the relationship between Joint Action Success and Social Bonding}

  \subsubsection{4.a Post-Tournament}
  Mediation analyses were conducted using linear mixed effects regressions in the Causal Mediation Analysis package in R (Version 4.4.5).  To make inferences concerning the average indirect and total effects, quasi-Bayesian Markov Chain Monte Carlo (MCMC) method based on normal approximation and 1000 simulations was used to estimate the 95\% Confidence Intervals \citep{Tofighi2016a,Imai2010}. MCMC estimation is a form of non-parametric bootstrapping whereby the sampling distribution for the effect of interest is not assumed to be normal but is instead simulated from the model estimates and their asymptotic variances and covariances \cite{Preacher2008}.

  Results of the mediation analysis revealed significant average indirect effect of Joint Action Success on Social Bonding attributable to Team Click, $\beta = .37, 95\% CI = 0.20 , 0.59, p < .001$.  When controlling for the effect of team click on social bonding, the average direct effect between Joint Action Success and Social Bonding was no longer significant, $\beta = -.006, 95\% CI = -.27 , .23, p = .96 $ (see Figure ~\ref{fig:MLM4aMediationAnalysis}). The direct effect diminished such that including Joint Action Success in the model produced a total effect that was marginally \textit{smaller} than the indirect effect alone, $\beta = .36, 95\% CI = .13 , .61, p = .01$. These results suggest that feelings of team click fully mediate the relationship between perceptions of joint-action success and social bonding.


  %Residuals of the mediation model were normally distributed around zero, ($W = 0.99, p < .28$, see Figure ~\ref{fig:MLM4aAssumptions} for model assumptions).

  \begin{figure}[htbp]
    \centering
    \includegraphics[scale = .5]{images/MLM4aMediationEffects.pdf}
    \caption{M4a Mediation Analysis}
    \label{fig:MLM4aMediationAnalysis}
  \end{figure}

  \subsubsection{4.a Pre- to Post-Tournament change}

  %\subsubsection{2.4.a Mediation Analysis: $\Delta$Joint Action Success $\rightarrow$ $\Delta$Team Click $\rightarrow$ $\Delta$Social Bonding}

  Results from the models reported above demonstrate significant relationships between 1) change in perceptions of joint-action success and changes in feelings of team click, 2) changes in feelings of team click and changes in feelings of social bonding, and a direct relationship between changes in joint-action success and changes in social bonding, but not a direct relationship between team performance expectations and changes in social bonding. As such, a mediation analysis was performed to formally test whether a change in feelings of team click over the course of the Tournament mediated a direct relationship between change in perceptions of joint-action success and changes in perception of social bonding.\\

  Results of the mediation analysis revealed that the average indirect effect of change in Joint Action Success on change in Social Bonding attributable to change in Team Click was not significant, $\beta = .14, 95\% CI = -.004 , .30, p = .06$, but trended in the predicted direction (see Figure ~\ref{fig:MLM24aMediationAnalysis}).  When controlling for the effect of team click on social bonding, the average direct effect between Joint Action Success and Social Bonding was, however, significant, $\beta = .27, 95\% CI = .07 , .48, p < .001$.  The total effect of the meditation was also significant, $\beta = .41, 95\% CI = .21 , .63, p < .001$.  These results suggest a marginally-significant partial mediation effect of team click on the relationship between joint-action success and social bonding.  While the indirect effect of joint-action success on social bonding (mediated by Team Click) was only marginally significant, this result does provide some support for the indirect effect observed in the post-Tournament data.  Considering the relatively small variation in change in team-click pre-post Tournament, to observe a marginally significant indirect effect of joint-action success on social bonding (mediated by team-click) in the pre-post-Tournament data is noteworthy.




\begin{figure}[htbp]
  \centering
  \includegraphics[width=\columnwidth]{images/MLM24aMediationAnalysis.pdf}
  \caption{M24a Mediation Analysis}
  \label{fig:MLM24aMediationAnalysis}
\end{figure}



  \subsubsection{4.a Overall Tournament}
NA









\subsection{Prediction 4.b: Team Click mediates the relationship between Team Performance Expectations and Social Bonding}

    \subsubsection{4.b Post-Tournament}
  Post-Tournament results also demonstrate a significant positive relationship between Team Performance Expectations and Team Click. The model generated for relationships between between Team Performance Expectations and Social Bonding, however, was not robust to the demands of model assumptions.   Nonetheless, mediation analysis was used to test the possibility that Team Click mediated the effects of Team Performance Expectations on Social Bonding.

  Results of the mediation analysis revealed significant average indirect effect of Team Performance Expectations on Social Bonding attributable to Team Click, $\beta = .36, 95\% CI = 0.17 , 0.62, p < .0001$.  When controlling for the effect of Team Click on Social Bonding, the average direct effect between Team Performance Expectations and Social Bonding was no longer significant, $\beta = -.07, 95\% CI = -.27 , .12, p = .84 $ (see Figure ~\ref{fig:MLM4bMediationAnalysis}). The total effect of the model was significant $\beta = .29, 95\% CI = .02 , .57, p = .04$.  This model, as with model 4.a above, indicated that Team Click fully mediated the relationship between Team Performance Expectations and Social Bonding.


  Again, the results of this mediation model should be treated with caution, given the fact that the mediation model is composed of a non-robust statistical model (i.e., path ```'c'').

  \begin{figure}[htbp]
    \centering
    \includegraphics[scale = .5]{images/MLM4bMediationEffects.pdf}
    \caption{M4b Mediation Analysis}
    \label{fig:MLM4bMediationAnalysis}
  \end{figure}


    \subsubsection{Pre- to Post-Tournament change}
    NA


    \subsubsection{4.b Overall Tournament}

      %\subsubsection{3.4.b Social Bonding $\sim$ Team Performance Expectations Tournament $\times$ Team Click Tournament}

       The interaction of Team Performance Expectations and Team Click was added to the model as a fixed effect to see if an increase in social bonding associated with more positive violations of team performance expectations was heightened when feelings of team-click increased. The model revealed a significant negative interaction between Team Performance Expectations and Team Click,  $\beta = -.26$ ($95\% CI =  -.32, -.20$), $SE = .03$, $t(321.2) = -8.46$, $p < .0001$, $marginal R^2 = .69$, $conditional R^2 = .74$.  Model residuals were non-normal ($W = 0.94, p < .00001$), owing to high kurtosis ($.39$) and negative skew ($-.88$) (see Appendix Figure ~\ref{fig:MLM32aAssumptions}).
       A model in which the outcome variable was log transformed following exclusion of outliers provided the best adjustment: model residuals were normally distributed ($W = 0.99, p = .14$) and individual observations exerted low influence (Cook's Distances all < .10) (see Table ~\ref{MLM33ateamPerfBondingTournamentInteractionComparison} for full model comparison). The adjusted model revealed a significant positive interaction of Team Performance Expectations and Team Click on Social Bonding, $\beta = -.06$ ($95\% CI =  .0004, .002$), $SE = .01$, $t(310.1) = -4.73$, $p < .001$, $marginal R^2 = .61$, $conditional R^2 = .65$.

       These results supported the prediction that feelings of team-click condition the relationship between perceptions of team performance (expectation violation, in this case) and feelings of social bonding.  Below, formal mediation analysis was conducted to further test this relationship.


       
% Table created by stargazer v.5.2 by Marek Hlavac, Harvard University. E-mail: hlavac at fas.harvard.edu
% Date and time: Thu, Sep 14, 2017 - 09:52:55
\begin{table}[!htbp] \centering 
  \caption{M3.3a socialBondingTournament ~ teamPerformanceExpectationsTournament*teamClickTournament} 
  \label{tab:MLM33ateamPerfBondingTournamentInteractionComparison} 
\scriptsize 
\begin{tabular}{@{\extracolsep{5pt}}lccc} 
\\[-1.8ex]\hline 
\hline \\[-1.8ex] 
 & \multicolumn{3}{c}{\textit{Dependent variable:}} \\ 
\cline{2-4} 
 & model & log-transformed & outliers+log-transformed \\ 
\\[-1.8ex] & (1) & (2) & (3)\\ 
\hline \\[-1.8ex] 
 (constant) & 0.45$^{**}$ & 1.69$^{***}$ & 1.62$^{***}$ \\ 
  & (0.16) & (0.04) & (0.06) \\ 
  & & & \\ 
 teamPerformanceExpectations & $-$0.0001 & $-$0.0003 & $-$0.001 \\ 
  & (0.002) & (0.0004) & (0.001) \\ 
  & & & \\ 
 indPerformanceExpectations & 1.02$^{***}$ & 0.22$^{***}$ & 0.11$^{**}$ \\ 
  & (0.06) & (0.01) & (0.04) \\ 
  & & & \\ 
 objectiveCompetence & 0.001 & 0.0003 & $-$0.0001 \\ 
  & (0.001) & (0.0004) & (0.001) \\ 
  & & & \\ 
 subjectiveCompetence & 0.01 & 0.01 & 0.01 \\ 
  & (0.04) & (0.01) & (0.01) \\ 
  & & & \\ 
 finalRank & 0.05 & 0.01 & 0.02 \\ 
  & (0.04) & (0.01) & (0.01) \\ 
  & & & \\ 
 minutesTotal & $-$0.03$^{*}$ & $-$0.01$^{*}$ & $-$0.01 \\ 
  & (0.02) & (0.004) & (0.01) \\ 
  & & & \\ 
 pointsTotal & $-$0.003 & $-$0.001 & $-$0.001 \\ 
  & (0.002) & (0.0005) & (0.001) \\ 
  & & & \\ 
 pointsTotal & 0.001 & 0.0001 & $-$0.001 \\ 
  & (0.003) & (0.001) & (0.001) \\ 
  & & & \\ 
 teamPerformanceExpectations:clickFactor3 & $-$0.01$^{***}$ & $-$0.002$^{***}$ & 0.002$^{*}$ \\ 
  & (0.001) & (0.0003) & (0.001) \\ 
  & & & \\ 
\hline \\[-1.8ex] 
Marginal R-squared & .68 & .28 & .23 \\ 
Conditional R-squared & .73 & .36 & .28 \\ 
Shapiro-Wilk Test (p-value) & .93($<$.00001) & .99(.02) & .99(.14) \\ 
Observations & 331 & 331 & 294 \\ 
Log Likelihood & $-$277.92 & 176.98 & 57.06 \\ 
Akaike Inf. Crit. & 589.83 & $-$319.96 & $-$80.12 \\ 
Bayesian Inf. Crit. & 654.47 & $-$255.32 & $-$17.50 \\ 
\hline 
\hline \\[-1.8ex] 
\textit{Note:}  & \multicolumn{3}{r}{$^{*}$p$<$0.05; $^{**}$p$<$0.01; $^{***}$p$<$0.001} \\ 
\end{tabular} 
\end{table} 




    \subsubsection{3.4.b Mediation Analysis: Team Performance Expectations -> Team Click -> Social Bonding}

    A mediation analysis was performed to formally test whether feelings of team click over the course of the Tournament mediated a direct relationship between team performance expectation violations and social bonding.  Results outlined above demonstrate Tournament-wide significant positive relationships between team performance expectation violations and feelings of team click, as well as team click and social bonding. Models also demonstrate a direct relationship between expectations around team performance and social bonding.  Available statistical software can only support a 2-level structure for multilevel mediation analysis. The third team-level random effect was therefore dropped from the model, while the model controlled for the random effect of individual across each of the four time points.

    Results of the mediation analysis revealed that the average indirect effect of change in Joint Action Success on change in Social Bonding attributable to change in Team Click was highly significant albeit small, $\beta = .02, 95\% CI = .018 , .026, p < .001$.  When controlling for the effect of team click on social bonding, the average direct effect of Team Performance Expectations and Social Bonding was also significant, $\beta = .01, 95\% CI = .001 , .017, p < .00001$.  The total effect of the meditation was also significant, $\beta = .03, 95\% CI = .027, .041, p < .0001$ (see Appendix Figure ~\ref{fig:MLM34aMediationAnalysis}).
    These results suggest that Team Click \textit{partially} mediates the effect of Team Performance Expectations and Social Bonding (average proportion mediated = .66 (.57, .77)).  This result should be treated with additional caution, as it did not account for team-level variation.


    \begin{figure}[htbp]
      \includegraphics[width =\columnwidth]{images/MLM34aMediationAnalysis.pdf}
      \caption{M3.4a Mediation Analysis}
      \label{fig:MLM34aMediationAnalysis}
    \end{figure}




\clearpage

























\section{Discussion of Results}

  The results presented above generally support the central hypothesis of this dissertation, namely that relationship the relationship between joint action and social bonding is mediated by the feeling of ``team click.''  Results of analyses concerning both the post-Tournament survey and change between and pre- and post -Tournament surveys display a clear relationship between perceptions of Joint Action Success and Team Click, Team Click and Social Bonding, and a direct relationship between Joint Action Success and Social Bonding.  In the post-Tournament analysis, Team Click fully mediates the relationship between Joint Action Success and Social Bonding. In the pre-post analysis, Team Click partially mediates this relationship (this statistic was not significant, but was trending in the predicted direction, $p = .06$).

  Results of models designed to test the predicted relationships between Team Performance Expectations, Team Click, and Social Bonding also provided support for some but not all predictions. A relationship was confirmed between team performance expectation violation and team click, but not between expectation violation and social bonding. In addition, the predicted interaction between Joint Action Success, Team Performance Expectations, and Team Click was not significant, nor was the interaction between these same two predictor variables and social bonding.  Team click did not mediate the predicted relationship between violations of expectations around team performance and social bonding. The model of this direct relationship was not statistically robust using the data collected in this study.

  These results suggests that while expectation violation in joint action might be an important factor in generating feelings of team click, it might not be a strong enough mechanism to drive social bonding directly.  In contrast to the single-item measure of Team Performance Expectations, Joint Action Success is a factor made up of four items that require detailed reflection on the experience of four different aspects of team coordination.  This item may have more powerfully tapped into the implicit mechanisms involved in coordinated joint action, encouraging athletes to reflect on coordination with specific co-actors, and as such the opportunity to rehearse and reinforce feelings of trust, reliability, and cooperation \citep{Reddish2013a}.  It is also possible that expectation violation in joint action might not be immediately available to athlete reflection.
  As Frith and colleagues point out \textcite{Frith2007,Frith2010,Clark2013}, the generation of interoceptive predictive models for action implicates cognitive processes that exist largely below the surface of conscious awareness, and the relationship between these unconscious informational transfer in movement coordination and the small fraction of information that does make its way to consciousness in the form of higher order symbolic and linguistic representations is still not clearly understood \citep{Semin2008}. Team click might be supported by various subtle, implicit, and pre-perceptual processes involved in ``active inference'' \citep{Schmidt2011}.

  Evidence presented here does however support the interpretations that perceptions of joint action success predict social bonding, mediated by the experience of ``team click.'' It was predicted that technical competence may condition the relationship between perceptions of joint action, team click, and social bonding, owing to the possibility that more expert athletes could be more likely experience less pronounced discrepancy between expected and actual performance \cite{Tomeo2012}.  In this specific study, there was no evidence that technical competence---objective competence (training age, years in team, age) and self-reported competence (ability versus teammates, Chinese opponents, or international professionals)---significantly influenced variables of interest.  It is possible, as noted above, that these measures failed to access the (largely pre-perceptual) mechanisms of prediction error management hypothesised to underpin cognition and affective dispositions in joint action.  It is also possible that the nature of the intensity and consequentiality of the Tournament in which athletes were participating was high enough that no one---experts athletes included---was immune to the uncertainty and stress of this experience. Indeed, the ability of competitive group exercise contests to consistently arouse high levels of psychological stress and uncertainty could be an important reason for their cultural evolutionary success over recent centuries.

  In addition to technical competence of individual athletes, preexisting dispositional tendencies in dimensions of personality (e.g., extroversion, agreeableness) and physical endowment could impact on processes of social bonding \citep{Marsh2009,VonRueden2015}.   It has been documented that personality types correspond with basal tendencies for movement and movement coordination. In a joint action task involving two people of different heights moving wooden planks of different lengths, for example, Richardson and colleagues \textcite{Richardson2007} found that individuals’ levels of agreeableness and extroversion were positively correlated with the level of persistence of cooperation (hyteresis).  In particular, in the condition in which wooden planks increased in length (therefore increasing in demand for cooperative action), the degree of cooperation was positively correlated with the taller of the pair’s extroversion; the degree of cooperation in the random condition (where planks were received in no ordinal sequence of sizing) was correlated with the agreeableness of the shorter (i.e., more constraining) of the individuals.  I used the personality data collected in the pre-Tournament survey to test the relationship between personality types and team click. Interestingly, none of the big five personality types (extraversion, agreeableness, openness, conscientiousness, and neuroticism) significantly predicted feelings of team click in the post-Tournament survey.

  The impact of inter-individual psychophysiological variation on joint action can also be seen in the case of social dysfunctions such as autism spectrum disorder \citep{Isenhower2012} and schizophrenia \citep{Varlet2012}.  Generally speaking,  while autism spectrum disorder appears to be associated with deficits in anticipatory timing adjustments in intra- and inter-individual coordination of physical movement \citep{Martineau2010}, schizophrenia appears linked to a lack of prediction error management, due to failure in the neurological mechanisms through which predictions about the consequences of an action are derived \citep{Frith2000}.  Delusions of control owing to misattribution of agency are well-documented in schizophrenic patients---either delusions of self-control over external actions and events, or, alternatively, delusions involving the control of others over the individual (in the case of split personality and associated hallucinations) \citep{Frith2007}. It is quite possible that the experience of ``team click'' is an analogue, albeit much socially functional, to the pathological over-activity of agency-attribution mechanisms observed in schizophrenic patients.  The question of individual variation in sociability and movement tendencies should be further assessed in future studies.

  The fact that the hypothesised mechanisms of joint action and social bonding have been shown to exist pre-declaratively and predominantly below the surface of conscious experience presents a methodological challenge for psychological research which relies heavily on self-report.  Further exploring the relationship between pre-perceptual regulatory mechanisms of joint action and their psychosocial effects will require the use of other methods of data collection and analysis in order to triangulate the reliability of self-report data \citep{Newell2014}.  In this specific example, video footage showing athlete performance and coordination could be assessed using already established methods of motion capture and analysis to provide pseudo-objective measures of interpersonal and team-level synchrony in joint action \citep[e.g.][]{Passos2011}.
  Importantly, video footage analysis allows for the testing between different proposed mechanisms associated with joint action and social bonding,  For example, motion capture from video footage allows for measuring synchrony between co-actors could using both traditional methods of dimension reduction and principal component analysis \citep[see for example][]{Riley2011} as well as emerging methods borrowed from dynamic systems theory including fractal-like 1/f scaling (``pink noise'') \citep[see for example][]{Holden2013}. It is quite possible that these different measurements of synchrony could access different psychophysiological dimensions of the relationship between joint action and social bonding.  Other physiological markers of social interaction such as heart rate variability \citep{Konvalinka2011,Fischer2014a} and pain threshold \citep{Cohen2009,Tarr2015}) are beginning to be developed and tested in both experimental and real-world settings.
  These novel methodological approaches help complement existing behavioural measures, such as such as economic games \citep{Xygalatas2013} and spontaneous helping tasks \citep{Kirschner2010}, designed to access psychological mechanisms related to social bonding and cooperation.
  Together, these measures could then be compared to self-report measures derived from athlete self-report to more fully understand the ways in which component mechanisms and system dynamics of joint action generate social cohesion \citep{Marsh2009}.

  In sum, the results reported in this study provide novel evidence that feelings of team click mediate a relationship between perceived joint action success and social bonding, substantiating the claim that under certain circumstances joint action and interpersonal coordination processes can generate feelings of social connection \citep{Marsh2009}. Relationships between expectation violation, team click, and social bonding
  also confirmed, but these models are less robust, and should be treated with caution. All reported results hold when statistically controlling for measures relating to individual performance, objective competence, and actual Tournament performance outcomes in the Tournament, including Tournament rank, points scored, minutes played, and so on.  Of course, the uncontrolled and \textit{in-situ} design of this study means that results are correlational, and as such the predictions of this dissertation, only partially confirmed in this study, require further attention via a controlled experimental design.  An experiment in which joint action success and expectation violation were manipulated, and explicit feedback around performance was eliminated, could allow for the assessment of the role of the cognitive processes of movement coordination in joint action in generating team click and social bonding.  In addition, a controlled experiment offers the opportunity to utilise other measurements of synchrony, such as those derived from motion capture from video recordings.


%    [ ] what inferences can we make from these results?
%    [ ] How does it relate to other literature?
%    [ ] limitations
%    [ ] future work —> experiment (takes away explicit feedback)
%    [ ]  "Conclude the general discussion with a strong paragraph stating the main point or points again, in somewhat different terms-if possible-than used before.”
