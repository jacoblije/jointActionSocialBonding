\chapter{\label{4partAIntroMethod} Introduction to Part A - Ethnography}

  \minitoc



\section{Abstract}
In this chapter I establish justifications for ethnographic research.

                                          \begin{CJK}{UTF8}{gbsn}

    \section{Introduction}

    Theory and empirical research from the emerging field of the social cognition of joint action suggest that the affordances of particular cultural environments act to enable and constrain observable behaviour in patterned ways.  The sensitivity of joint action (or any cognitive process for that matter) to informational affordances provided by various layers of ecological and cultural context.  The cognitive inputs to joint action in real world settings are rarely limited to essentialised components administered in laboratory paradigms. It is now known that cognitive processes relevant to joint action are distributed throughout brains, bodies, and the physical environment of the ecological niche in which it is situated.

    In the present study, I address these knowledge gaps in evolutionary approaches to group exercise through an ethnographic study of the social cognition of joint action among professional Chinese rugby players.  Based on emerging research from the social cognition of joint action, in which it is increasingly understood that dynamical coupling of lower-cognitive mechanisms associated with movement regulation set the foundations for higher-cognitive processes such as social bonding and declarations of group membership, I interrogate the relationship between athletes' experience and perceptions of joint action and attitudes towards group membership.  Within this space I isolate the psychological construct of ``team click,'' which refers to athletes' tacit sense of quality joint action coordination.   I then outline the specific predictions of the ethnographic component of this dissertation, before describing the method via which I collected primary data.


    In this ethnographic study I identify evidence for 1) affordances specific to the sport of rugby and the cultural milieu of competitive sport in contemporary China, and 2) cognitive mechanisms relevant to relationships, hypothesised in this dissertation, between competitive joint action and social cohesion.  In particular, I highlight athlete's perceptions of performance in joint action, their experiences of phenomena associated with team click, and processes of group membership as relevant to study predictions.

    Accounting for human behavioural phenomena requires the consideration of a number of biological, cognitive, and ecological mechanisms that interact via reciprocal feedback loops spanning multiple scales of time and space \citep{Fuentes2015}.
    %justifications for ethnography
    The proliferation of anthropological approaches to human behaviour in the last 50 years, while at times threatening the overall coherence of the discipline as a whole \citep{Beller2012}, has also produced diverse theoretical and methodological options for documenting human variation \citep{Fuentes2016a}.
    Anthropology is thus well placed to expand upon accounts of group exercise, via methods ranging from ethnographic exploration capable of uncovering novel dimensions of behaviour and generating testable hypotheses, to quantitative techniques---e.g., experimental and mathematical simulation paradigms---capable of testing hypotheses \citep{Epstein2006,Fuentes2016}.


    %Contribution: (See Cohen2007 and CohenWhitehouse2012)
    The first time ethnographic evidence has been provided for mechanisms of joint actions (Wacquant: body and soul), Body for China (Brownell1995)
    This ethnographic theorises divergent cultural modes
    Cultural specificities may act as ``factors of attraction'' \citep{Sperber2014} that constrain and direct the fixation of cultural variants within and between populations.






    \section{Method}


    \subsection{Research Setting and Participants}

    I planned to conduct ethnographic research with athletes and coaches of the Beijing Provincial Rugby Sevens Program based at the Beijing Temple of God of Agriculture Sport Institute.  The Program consists of Men's and Women's rugby sevens teams, each with approximately 20-30 athletes and 2-4 coaches per team. Athletes train full time and live in dormitory accommodation at the Institute.  One of four vice-principals of the Institute is responsible for the administration of the program.  Permission to conduct research at the Institute was sought from the vice-principal responsible for rugby at the Institute and the head coach of the rugby Program prior to arriving in Beijing in 2015.  Permission to conduct research was sought directly from athletes at the beginning of the first research period in September 2015. The University of Oxford’s Central University Research Ethics Committee approved this study (SAME/CUREC1A/15-059).

    \subsection{Materials}

      \subsubsection{Participant Observation}
      I planned to conduct a number of stretches of participant observation with the Program between September 2015 and September 2017.  During these stretches, I planned to live full-time at the Institute and attend training sessions, team meetings, meals, and participate in any other activities relevant to the rugby program.  I planned to record field notes using Evernote (Version X), an electronic note taking software that is synchronised across my mobile and personal computer devices.

      \subsubsection{Interviews}
      I planned to conduct and record a combination of ad-hoc exploratory (unstructured) interviews with a range of research participants with knowledge of rugby in China as well as scheduled and directed (semi-structured) interviews with athletes in particular.  The script for semi-structured interviews would be designed based on a combination of existing theory and initial ethnographic observations, in order to understand the culturally specific contours and operation of mechanisms relating to joint action and social bonding.

      \subsubsection{Informal Surveys}
      In addition to conducting participant observation and interviews, I also planned to issue surveys to measure athletes' motivations for, and perceptions of joint action and group membership in the rugby Program.  The surveys would be designed based on a combination of existing theory and initial ethnographic observations.


    \subsection{Procedure}

    In May 2015, 4 months prior to beginning ethnographic research, I contacted the vice-principal responsible for the rugby Program and the head coach of the rugby Program to ask for permission to conduct research at the Institute.  Following affirmative responses from both, I made plans to conduct two periods of in-depth ethnographic research: 1) 6 months between September 2015 and March 2016, 2) 6 weeks during July-August 2016.

    \subsubsection{Participant Observation}
    Soon after arriving in Beijing at the end of August in 2015 to begin research, I met in person with the vice-principal and head coach of the rugby program to confirm permission for research and to discuss logistics.  I discovered that the Beijing Rugby Program was limited to a Men's Program.  At the time, the Women's rugby Program had yet to be resurrected after the humiliating ``match strike incident'' of the 2013 National Games (see Chapter 3 ~\ref{} for a detailed explanation). The Women's program would later be resurrected at the start of 2016, in time to participate in qualification tournaments for the 2017 National Games.  Due to this limitation, I decided to focus my attention on the Men's program, which consisted of 25-30 athletes and four coaches at any one time.

    Both the vice-principal and head coach agreed to provide me with research access to the rugby program, a room in the Institute's dormitory, and access the Institute's 1st level canteen, in exchange for assisting the Program with rugby knowledge and coaching.   The Institute had two canteens in which athletes and coaches ate all of their meals.  Athletes and coaches who had represented Beijing at national level competitions were entitled to eat at the 1st Level Canteen (\textit{yixian shitang} 一线食堂), whereas all other athletes at the Institute, or athletes who were on temporary trial at the Institute, ate at the 2nd Level Canteen (\textit{erxian shitang} 二线食堂).

    During periods of ethnographic research I lived full time with the team at the Institute, attending training sessions, team meetings, meals, and other activities with the team.   All my interactions with research participants took place in Modern Standard Chinese (Mandarin or \textit{putonghua} 普通话).  I took notes using an electronic note taking application (Evernote, version 6.11), which automatically synced and stored notes created on either my mobile phone of personal laptop computer. I collated, summarised, and tagged these notes weekly or fortnightly.


      \subsubsection{Interviews}

    Unstructured interviews were conducted with athletes and coaches on an ad-hoc basis, often when an informal discussion developed into a conversation relevant to my research questions. In such instances, I would interrupt discussion with the research participant and ask permission to record the remainder of the discussion using the digital audio recording feature on the Evernote application on my mobile phone.

    Semi-structured interviews were conducted by appointment in my dormitory room at the Institute at a period 2 months in to my first stint of participant observation.  During semi-structured interviews, I asked athletes about their personal background (including their family situation), their motivations for adherence to rugby, perceived costs and benefits of adherence to rugby, perceptions of joint action, and group membership. For a detailed script of semi-structured interviews, see Appendix ~\ref{} Figure ~\ref{}.  Questions served only as a loose structure for conversation, and at times either the athlete or I departed from these questions to talk about other dimensions of experience associated with rugby at the Institute.  The order in which athletes participated in semi-structured interviews was randomised.

    I conducted all interviews in Modern Standard Chinese (Mandarin) and interviews were recorded with participant consent using digital audio recording feature on the Evernote application on my mobile phone or laptop computer.  Once all interviews were recorded, interviews were transcribed into written Chinese by a native Chinese speaking research assistant using a ``verbatim'' method \citep[i.e., including an account of all verbal and important nonverbal (coughs, pauses, etc.) utterances, see][269-70]{Poland2003}.  I checked each transcript for accuracy by comparing the script against the original audio recording during the first phase of open coding analysis (see Section Data Analysis below). I analysed interviews in Chinese and only translated into English data extracts that were included in the main analysis of this dissertation.

    %    \subsubsection{Structured}

    \subsubsection{Surveys}

     I conducted a number of informal surveys designed to measure athletes' experience of joint action and group membership in training sessions.

       \myparagraph{Post-interview surveys}
       Following semi-structured interviews, I asked each athlete to rank 10 different possible motivations for adherence to rugby from most important to least important. Possible motivations for rugby consisted of: \textit{to gain access to education}, \textit{to represent Beijing}, \textit{to do Family proud}, \textit{to gain respect from others}, \textit{for (the benefit of) teammates}, \textit{for employment opportunities}, \textit{for money}, \textit{for enjoyment}, \textit{to find a partner}. In addition, athletes were asked to report their 1) three closest friends in the team, 2) the three team members most willing to sacrifice on behalf of the team, and 3) three most competent athletes in the team (see Appendix ~\ref{} for full script). Athletes answered these questions using a pen and paper. I later collated and uploaded these responses to Evernote.


      \myparagraph{Post-training surveys}
      I conducted informal surveys following three training sessions: 1) a session in which (predominantly junior) athletes ran an aerobic fitness test involving continuous straight line ``shuttle runs''  at and above the aerobic threshold for approximately 25 minutes (known as the ``Beep Test''), and 2) two 90-minute training sessions spread one week apart involving training scenarios that emulated high-intensity match conditions.  After each of these sessions, I administered to each participating athlete via WeChat nine items selected from a Chinese version of the Flow State Scale 2 \citep{Liu2012} designed to measure the nine conceptual dimensions of the flow experience: challenge-skills balance, action-awareness merging, clear goals, unambiguous feedback, total concentration on the task at hand, sense of control, loss of self-consciousness, transformation of time, and autotelic experience \citep{Csikszentmihalyi1990}).  All survey items used a 7-point Likert scale. For full survey details, see Appendix ~\ref{} Section ~\ref{}.

      \myparagraph{General survey administered at two time points (longitudinal)}
      I asked athletes to comment on experiences of joint action and group membership at two points in time spread three months apart.  These survey items included experience of agency in the team (weak-strong), perceived role in the team (central-marginal), perceptions of individual performance (weak-strong), perceptions of team performance (weak-strong), training intensity (\textit{qiangdu})(light-heavy) and difficulty (easy-hard).  All survey items used a 7-point Likert scale.


    \subsection{Data analysis}
    Field notes from participant observation, interview scripts, and informal survey responses formed a corpus of ethnographic data that was subjected to a recursive process of ``thematic analysis'' \citep{Braun2006}.  As Braun and Clark \textcite[10]{Braun2006} explain, ``A theme captures something important about the data in relation to the research question, and represents some level of patterned response or meaning within the data set.'' Identification of recurring themes was guided by (but not limited to) the research question and theoretical predictions of this dissertation, outlined initially in Chapters 1 and Chapter 2 (and then refined in Chapter 3 in accordance with the specific research context of rugby in China).  Themes were identified on both explicit (semantic) and implicit (latent) levels of the data \citep{Boyatzis1998}. Theoretical predictions and relevant existing research concerning the social cognition of joint action helped direct analysis of the latent level of the data.

    The thematic analysis involved three stages that unfolded in a recursive (rather than linear) fashion \citep{Braun2006}. In phase one, I familiarised myself with the each data set in the corpus (field notes, interview transcripts, and informal survey responses) and tagged relevant extracts with theoretically-guided ``codes.'' For example, upon encountering Hongwei's description of his position in the team in his interview transcript (cited in the Introduction ~\ref{}), I tagged this with codes such as ``group membership,'' ``mutual support,'' ``emotional support,'' ``knowledge of team roles,'' ``signalling commitment to team'' etc.  My coding system was thus directed by (but not limited to) pre-identified theoretical variables relating to 1) athlete perceptions and expectation violations surrounding joint action, 2) perceptions and feelings associated with the phenomenon of ``team click,'' 3) understandings of and feelings relating to ``group membership'' and social bonding, as well as 4) possible moderator variables of technical competence and personality type.  For each data set, I created a data frame using Microsoft Excel (Version 14.7.1) in which research participants formed the rows, and distinct codes formed individual columns. Data extracts from interviews and field notes were imputed into the matrix, with an emphasis on including data surrounding the code's target, in order to preserve context \citep[see][]{Bryman2001}.

    In phase 2, I sorted the different codes into potential themes and collated all the relevant coded data extracts within the identified themes and judged on the dual criteria internal homogeneity of codes within themes (coherence) and heterogeneity of codes between themes (distinction) \citep{Patton1990}.  I then produced a master data-frame (participants x themes), in which data extracts from all data sets were included.  In phase 3, I generated a definition of each theme, and a refined list of data extracts capable of representing that theme in subsequent analysis \citep{Braun2006}.


    \section{Qualifications and positionality of the researcher}

    Before arriving in Beijing in 2015 to begin my doctoral research, the last time I was in China was two years earlier in 2013, when I spent eight months coaching the Chinese men's youth rugby 7s team in the lead up to the Nanjing Asian Youth Olympics.  Before that, I had spent one year studying on Exchange at Beijing University in 2008, and another year before that on an intensive Chinese language course at Liaoning University, Shenyang, in 2006---my first trip to China.  Rugby featured heavily in both instances.  In 2006, an Australian classmate and friend Ed had caught wind of the fact that there was a rugby program down the road from Liaoning University at the Shenyang Sports College (SSC).  Despite the fact that we had both been diligently attending class and courageously deploying our elementary Chinese to order food at restaurants and befriend local taxi drivers, Ed and I were, nonetheless, three months into our intensive language exchange and feeling that our Chinese skills were floundering.  We suspected that this was in large part due to the fact that we had met very few local Chinese people our age.  So one afternoon we rode our bikes over to the Shenyang Sports College in time for the rugby team's afternoon training session.  Less than six months later, we were boarding an overnight train from Shenyang to Shanghai with the SSC rugby team to compete in the annual Shanghai Rugby 7s Tournament.  We had become closely integrated into the community of rugby athletes at SSC, due in part to the common language of rugby that we all shared, and perhaps mostly due to the overwhelming hospitality of the SSC athletes and coaches.  The decision to find the rugby team may have also helped us improve our Chinese. Ed and I were the only two in our cohort to finish the year in Shenyang with a Level 6 in the Chinese Proficiency Exam, which qualified us to study alongside Chinese local students at an undergraduate level.

    Buoyed by this experience with the SSC rugby team in 2006, I followed a similar template two years later when I arrived at Beijing University on exchange from Sydney University to study sociology at Beijing University.  I had just finished working at the 2008 Beijing Olympics. At that time in Beijing, the only Chinese rugby program was based at the Chinese Agricultural University, a forty minute cycle north of Beijing University.  It was during my time training and generally ``hanging out'' at CAU that I met and developed a strong friendship with Kai, who was at the time playing for CAU and China, while also finishing a Master's degree in Labour Law.  I also met and developed relationships with many rugby players, coaches, and general fans of the Beijing rugby community.  The CAU rugby program was the strongest in the country: CAU consistently outperformed its rivals at the time (Shanghai Sports Institute, the People's Liberation Army, and SSC) and it was awarded with the responsibility of hosting the Chinese national team.  When the International Olympic Committee announced in late 2009 that rugby would be played in the 2016 Rio De Janeiro Olympics, it was subsequently decided in 2010 that rugby would be inducted into the state sponsored sports system and played in the next Chinese National Games in 2013.  Following this announcement, many of CAU's athletes and coaches dispersed to various professional provincial rugby programs, the main ones being Beijing and Shandong.

    Between 2009 and 2013 I returned to Australia to finish my undergraduate degree, during which time my own rugby career also rapidly developed.  After a successful season in the Sydney Premiership competition in 2009, I was selected to play for the Australian Rugby Sevens Team. I represented Australia from 2009 through to the end of 2012.  In 2013, during the 9 month gap between my Australian rugby contract ending and the start of my graduate studies at Oxford University, I returned to China to coach the Chinese Youth Men's 7s program in their lead-up to the 2013 Nanjing Asian Youth Olympics.  Along with a small team of Chinese coaches and management, I coached a core group of roughly 25 athletes aged between 15 and 18 years old. We trained 6 days a week for approximately 6 months, with only occasional breaks for National holidays, or for athletes to return to their home provinces to complete compulsory exams.  The program was based predominantly in Anhui province, and we travelled from Anhui to other provinces further afield to find suitable practice opportunities against provincial programs.  Soon after the completion of the Asian Youth Olympics in Nanjing in 2013, the Chinese National Games were held in Shenyang. Rugby was played---for the first time in National Games history---with dramatic consequences, explained above.


    \subsection{Social cognition of joint action among professional rugby players in China: recalibration of predictions}
    Although my unique set of qualification prepared me somewhat with the language ability, knowledge of rugby, and relationships necessary to conduct research in this particular setting, I was not fully prepared for the depth of social complexity that I would encounter at the Institute.  Perhaps too strong were my (cultural) intuitions concerning the structure and function of a rugby team that I found the process of training and observing the behaviour of athletes in the Beijing team particularly challenging, and at times psychologically jarring (as I did in 2013 when coaching the National Men's Youth Team).  It was only following my first 6 month stint of ethnographic research, when I had an opportunity to properly synthesise existing literature concerning the social cognition of joint action on the one hand, and evidence concerning culturally divergent biases of social cognition and modes group membership, as well as the historical foundations of an indigenous Chinese psychology, on the other, that I was able to fully comprehend the fullness of my initial observations and (re)formulate predictions for the purpose of future empirical observation.

    In Chapter 2, I make a series of novel theoretical predictions concerning a relationship between joint action and social bonding, broadly relevant to group exercise contexts in which joint action and group formation features.  The introduction provided in this Chapter to the specific group exercise context of professional rugby players in the PRC necessitates a recalibration of study predictions outlined in Chapter 2.  In particular, I expect that the generalisable cognitive mechanisms and systems dynamics outlined in Chapter 2 will operate with culturally specific contours that require careful exegesis and articulation.  As explained above, the sport of rugby union (in its particular form of rugby sevens), the historical-cultural context of the PRC, and the specific modern history of rugby in China are all important factors in tracing the contours of these mechanisms, and are crucial to comprehend as affordances in the shaping of processes relevant to the social cognition of joint action.

    For the ethnographic observations that follow, I predict that culturally ancient and socially dominant hierarchical relationism interact with inputs from the modern history of sport in China and the specific history of the team sport of rugby to produce distinct patterns of behaviour observable at the level of 1) institutions, namely the Institute and the rugby program, 2) group membership, namely expression and endorsement of group identity, and 3) action and perception when participating in the joint action requirements of rugby.  With this specific cultural terrain in mind, I expect to observe evidence of the predictions outlined in Chapter 2, namely: 1) a positive relationship between perceptions of joint action and team click, possibly mediated by expectation violation, 2) a positive relationship between perceptions of team click and social bonding, and 3) a direct relationship between perceptions of joint action and social bonding.



                                                          \end{CJK}
