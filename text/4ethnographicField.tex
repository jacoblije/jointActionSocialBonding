First week in China (introduction vinnete):
- ``rou'': mysterious carnal sport --> social community
- XNT fall from grace:
- Qingdao:
    - basketball (actionScrutiny)
    - Asia 7s: (Qi gaige & Li sheng) (system, incentives)
    - National 7s: Beijing:WCY out of the huddle


XNT:
- Sport as a upward mobility strategy:
    - Many from athletics (IND) to rugby (TEAM)
- Fall from grace --> Lack of click?
- System: coach ()
- Zero to one: new comers, uni students, senior players, old motherlandStrength

In this system, how do they interaction

%%%%%%%%%%%%%%%%%%%%%%%%%%%%%%%%%%%%%%%%%%%%%%%%%%%%%%%%%%%%%%%%%%%%%%%%%%%%%%%%%%%%%%
ABSTRACT:
Fieldwork in Beijing


















XNT fall from grace:
I arrived at the Beijing Temple of Agriculture Sports Institute (hereafter the Institute) first thing on Monday morning. I entered via the main entrance in the south, and made my way west to the main administration building by hugging the the perimeter of the 30,000 capacity multi-purpose stadium that dominates the Institute's campus (see map).  I had scheduled meetings with Jenny, the vice-principal responsible for the administration of the rugby program, and ZPH, the head coach of the Beijing rugby program. Jenny I knew from interactions with the Beijing team in 2012 and prior to the National Games in 2013. I had originally met ZPH in 2008 when he was an assistant coach at CAU.  I hoped that Jenny and ZPH would both grant me the permission I needed to conduct research with the Beijing team.

The Institute is located on the South 2nd Ring Road of Beijing, on the western flank of at Yongding gate. Yongding gate marks the southern end of the city's central north-south axis, which includes Tian'anmen Square, the Forbidden City, and the Drum and Bell Towers to the north (see map 2).  As one might expect, given the relatively central location of the Institute and considering Beijing's 3000 years of recorded history, the land on which the Institute sits was not always home to athletes and sport facilities.  The Institute takes its name from the temple that was built on the land in the 15th century.  The temple was used by Ming and Qing dynasty emperors to perform ceremonial sacrifices for harvest. In 1936, the land was reappropriated to build the Republic of China's first ever sport stadium, originally named the Beiping Public Stadium.  Officially established in 1952, the Institute was also the People's Republic of China's first dedicated national institute of sport.  The Institute is now one of four major sports institutes in Beijing, and is home to seven of Beijing's professional sport programs: Table Tennis, Athletics, Gymnastics, Women's Football, Tennis, Weightlifting, and Rugby Union.  Rugby is the most recent addition in 2010, in preparation for the 2013 National Games.

When announcing to friends within the Chinese rugby community that I was preparing to conduct research with the Beijing rugby team, many warned me about becoming too involved, worried that I might suffer a similar fate to the group of coaches and athletes associated with the 2013 National Games controversy.  Adrian, for example, told me that the Institute was riddled with ghosts: ``There really are ghosts there, I'm telling you Lijie, you should keep your distance.'' (真的有鬼啊,我告诉你李杰,你最好离远点吧)  Others hypothesised that the dramatic events of 2013 could be attributed in part to the blatant and irreversible disturbance of ``fengshui'' caused by the act of implanting a sports stadium on top of a sacred dynastic temple.  I naturally scoffed at these predominantly tongue-in-cheek warnings, although I was intrigued to learn more about the cosmological (as well as historical and political) factors that lay beneath the history of Beijing rugby and the Institute.

I had heard from members of the Beijing rugby community that the rugby program at the Institute had experienced a dramatic fall from grace after the drama of 2013.  All of the old guard of coaches, and almost all senior athletes from both the women's and men's squads left the Institute immediately after the Games in late 2013.  By all reports, the rugby program was all but deserted for at least 6 months after the games, before the Institute resurrected the men's program in 2014 by appointing a new pair of coaches (ZPH and SY) and enlisting the junior athletes from the previous National Games cycle to step in to the senior team.  The women's program was inactive for a full two years, and was only just starting to re-activate when I arrived in October 2015.  Having let the dust settle on the embarrassment of the women's program, the Institute had decided to continue with both the men's and women's team for the next 2017 National Games.


I sat down in Jenny's office waiting for her to finish on a phone call. Jenny was a local Beijinger and a former National Champion athlete at the Institute in her chosen event of high jump. She spoke on the phone with all of the affectations that only a true Beijing local could produce---her word endings were coloured with coarse yet elegant ``erhuayin'' suffixes, and the person to which she was speaking was addressed using the respectful version of the 2nd person pronoun ``nin.''  Jenny possessed a rare and valuable combination of credentials: Beijing residency and outstanding athletic achievement.  China's infamously rigid ``Hukou''residency system means that it is still the case that only individuals with Beijing residency can hold employment at government institutions such as the Institute.  Non-locals can become Beijing residents, but the criteria for this process have become more and more stringent, and fewer and fewer applications are successfully processed, particularly in industries like sport.  Undoubtedly, Jenny's combination of outstanding athletic achievement and Beijing residency facilitated her post-athletic career as a coach and administrator at the Institute.  Listening to Jenny negotiate charismatically on the phone also reminded me of something that was clear during our first interactions in 2012, and that was that Jenny also appeared to posses a natural ability to manage the people and politics that came with the territory of her job.

Once she had finished with the call, Jenny welcomed me and tactfully explained to me that rugby at the Institute had indeed experienced a dramatic fall from grace. From its pre-2013 status as the Institute's flagship sport (including the bold performance targets of a gold medal for the women, and a top three finish for the men), rugby had dropped to sport that the Institute intended to continue to playing, but with no additional ambition beyond that goal. The men's team were performing at around the level of 3rd or 4th in the country, which was not terrible, but it was also against provinces who had yet to field their strongest teams.  JXZ indicated that the head coach ZPH and his assistant SY really had their work cut out for them, and that my presence as observer and occasional coach would benefit the team.  She agreed to organise a room in the rugby program dormitory, as well as access to the Institute's canteen.


%Indeed, it seemed that the biggest performance goal for rugby was for nothing to publicly go amiss in the next round of the National Games.
%I didn't quite understand it at the time, but a large component of ZPH's was the lack of support he was receiving from the Institute leadership.  ``不出事就行''

%I later discovered that The original altar of the Temple of Agriculture had been preserved and restored, and was in the north-west corner of the campus.

After meeting with JXZ, I walked further north into the campus to where the rugby dormitory was located to meet with the head coach ZPH. My connection to ZPH went back to CAU in 2008, where he had been a coach at the time. From Shandong originally, ZPH was a graduate of of the Shanghai Sports University, another prominent rugby program at the time.  ZPH had been recruited from Shanghai to CAU by ZHJ to coach so that he could continue to play for the Chinese national team after he had completed his undergraduate studies.  After we had discussed my research and he had provisionally approved my plan to spend the next period with the team, I asked him about the current situation with the Beijing team.  ZPH explained that he was quite frustrated that the group of athletes he was coaching lacked experience and maturity. I asked him exactly what areas of the team's performance, and he indicated that all areas were not great, suggesting that not enough players had found that ``feeling'' for gameplay and very few were motivated to train hard.

The team were preparing for the final national Tournament of the season in Qingdao early the following week. The year's final national Tournament in Qingdao was planned to immediately follow the China leg of the Asia Sevens series also hosted in Qingdao. The After that the team would break for three weeks and then resume training for the following season after the China National Dat Holiday in early October.  We agreed that we would watch the Asian 7s and the National 7s in Qingdao this week and then reconvene in Beijing to discuss a strategy for the team's next chapter of training at the Institute.



Qingdao:
Kai's hometown was Qingdao, and so we travelled down together on the high speed train after Kai finished work on the Friday evening in time to watch the Asia Sevens Tournament on the Saturday.  The Asia Sevens is an annual series of three regional rugby sevens tournaments featuring men's and women's national sevens teams.  The men's series has been held regularly since 2009, and the women's series was established in 2013. The series usually consists of two three annual Tournaments, alternating between various locations including Hong Kong, South Korea, Sri Lanka, China, Japan, Malaysia, India, Singapore, and Thailand.  In 2015, The Asia Sevens Series Tournaments were held in Colombo, Bangkok, and Qingdao. Would be played on the Saturday and Sunday, and then the National Tournament would follow immediately after on the Monday and Tuesday.



Olympic Qualification:

When I arrived at the stadium in time for the beginning of the Asia Sevens Series, I met a number of coaches and athletes in the stands who knew from my time in Beijing in 2008 and then coaching in 2013. I sensed from my various interactions that there was an air of nervousness around the Tournament, particularly for the Chinese women's team.   The Qingdao tournament was the final tournament before the all-important Olympic Qualification Tournaments, to be held in Hong Kong and Tokyo in November 2015.  The top ranked team from those two legs would qualify for the Rio Olympics in 2016. It was clear that the Chinese men's team were not in serious contention for Olympic qualification, given the superiority of more established men's rugby programs like Japan and Hong Kong. The Chinese women's team, on the other hand, were expected to qualify for the Olympics. Since the Chinese women's team was first established in 2002, China had made great strides in women's rugby, both in Asia and globally.  Not including occasional losses to closest rivals Kazakstan, between 2002 and 2012 the Chinese women's sevens team was the dominant women's team in Asia, easily outcompeting Japan and Hong Kong, and at times was competitive against the world's best including New Zealand and Australia.  The main reason for China's dominance in the women's game during this period was that other traditional rugby nations, despite having developed professional infrastructure for the men's game, lacked almost entirely an equivalent infrastructure for the women's game. China`s state sponsored sport system, on the other hand, was relatively agnostic towards gender in sport, especially when according to the incentive structures of the Chinese sports system a gold medal is a gold medal, regardless of the gender of the recipient.
Indeed, beginning with the Chinese Women's Volleyball Team's gold medal victory at the LA Olympics in 1982, China has enjoyed a comparative advantage in women's sport due to the relative agnosticism of the state sponsored sport system towards gender.b







 fact that China's athletes trained in an environment that more closely mimicked a proffesional environment


 in environments that mimicked in full-time programs


 which had everything to do with the impending Qualification Tournaments for the Rio Olympics.


While the Chinese men's team was not hopeful to qualify






Basketball:



National Women's teams: Li Sheng
Beijing Team: Huddle at the end
















XNT system:



Cultural Modes of group membership and social cognition:
















\section{The Beijing provincial men's rugby team}
I focus my ethnography on the Beijing provincial men's rugby team ($n=26, avg. age=21.3, range = 17-27, SD = 2.96$).  The Beijing provincial men's and women's rugby programs are based at the Xiannongtan Sports Institute in Beijing (one of Beijing's four major sports institutes, and home to seven different full-time sports programs, hereafter Xiannongtan).  These athletes represent Beijing at a provincial level, playing against other provinces in annual tournaments, and every four years at the all-important National Games.  Five athletes have previously represented China in international rugby sevens tournaments.  There is also a Beijing women's rugby team at Xiannongtan, but I was not able to accompany the activities of both teams closely enough to perform adequate ethnographic research.

Members of the Beijing men's rugby sevens team are either already-contracted (n=13) or aspiring professional athletes (n=13) who live and train 6 days a week at Xiannongtan and occasionally attend university or high school classes as part of their ongoing education---what can only be termed part-time education.  A head coach and assistant coach look after the day to day organisation of team schedules and training, and these two coaches are assisted by a further two player-coaches, who are in the gradual process of moving from athlete to coach status.  One of Xiannongtan's four principals is responsible for the management and administration of the rugby program, and is occasionally present at team meetings and national competitions.

Athletes are all from relatively modest socio-economic backgrounds, many hailing from suburban and rural areas of northern China (Shandong (11), Beijing (6), Jiangsu (3), Liaoning (2), Hebei (2), Anhui (1), Henan (1), Heilongjiang (1)).  The squad consists of 10 fully contracted senior athletes (\textit{xieyi}), three provisionally-contracted athletes \textit{shixun}, and six student-athletes \textit{erjiban} who do not receive a salary but receive training, food, board, and educational support.  The remaining athletes (7) are classed as athletes in training \textit{jixun} and are effectively on trial until they either show promise or withdraw from the squad either voluntarily or upon suggestion by the head coach.  Provided that they meet the relevant academic and athletic requirements, contracted and student athletes are able to attend the Beijing Sports University—considered to be the country's most prestigious sports university and one of China's ``top brand universities'' (\textit{mingpai daxue}).

The average rugby training age (years spent playing rugby) of Beijing athletes is 3.12 years ($range = 0.16 – 10 years$).  Contracted senior athletes ($average age = 24.3 years$) have trained for an average of 5.4 years, whereas the average training age of junior non-contracted athletes (average age = 19.3 years) was 1.7 years.  Over half of the athletes have a background in other sports (15 athletes from track and field, one from football, one from basketball), usually beginning part-time or full-time physical training at the age of 11-13.  Those who transferred to rugby did so either at the beginning of senior high school (16 years) or at university age (18yrs).  The rest of the group had no particular sporting background before starting at Xiannongtan, and were scouted by school athletics coaches based on their basic athletic attributes (running speed, strength, coordination, and potential for physical growth).  Of the 26 athletes in the squad, three junior athletes who were part of the squad when I arrived in September 2015 have now left, and three new athletes have arrived. This movement of non-contracted players in and out for trials is quite common.

\subsubsection{Training schedule}
Every year between April and September there are five national tournaments held in different locations across the country.  October –-- February constitutes the off- and pre-seasons for these yearly competitions, during which time teams travel to domestic or international training locations depending on amount of program funding and training strategy.  In 2015, before an unexpected change in coaching team at the end of December, plans were to travel to Yunnan in the new year province for altitude training (January) before moving to sea-level somewhere in the south (February/March).  Following the coaching leadership change, the team did not leave Beijing until after Chinese New Year (15th February), which meant that training during this period was influenced by Beijing's cold winter weather and air pollution.

Below is a table of a typical weekly training schedule. A typical week consists of 10x2.5hr training sessions, three of which are strength and conditioning sessions, seven of which are on-field rugby sessions.  In addition, two one hour evening skills sessions are also added for junior athletes to hone their basic skills of passing, catching, and game-play.  Athletes live full-time on campus at XNT in dormitory accommodation (usually 3 athletes per ensuite room), and are permitted leave on the weekend after the conclusion of Saturday morning training.  Athletes with family in Beijing usually take this leave, while the remaining athletes spend weekends at XNT.  Athletes break at the end of season (September) for two weeks, and occasionally around Chinese New Year for 7-10 days, unless New Year interrupts pre-season training plans, in which case training continues throughout.


\begin{table}[htpb]\caption{Weekly Training Schedule}
  \begin{center}
    \begin{small}
        \begin{tabular}{| c | c | c | c | c | c | c | c |}
          \hline
          & \bf M & \bf T & \bf W & \bf T & \bf F & \bf S & \bf S \\
          \hline
          0600 & Training &  &  & & & & \\
          \hline
          0900 &  & Training & Training & Training & Training & Training &  \\
            \hline
          1500 & Training & Training & & Training & Training & Training &  \\
            \hline
          1900 &  & Training (junior athletes) & & Training (junior athletes) & & & \\
             \hline
        \end{tabular}
      \end{small}
    \end{center}
  \end{table}


\section{Research methods}
Ethnographic data included: unstructured and semi-structured interviews with athletes and coaches (yet to be analysed in-depth), general and activity-specific surveys, and field notes based on participant observation of daily activities of the team.


\subsubsection{Participant observation}
For 6 months in 2015-16 and 6 weeks in summer 2016, I lived and trained full time with the team at XNT, during which time I took daily field notes using a note taking application (Evernote, version 6.11), which was synced between my mobile phone and personal computer. I collated, summarised, and tagged these notes weekly or fortnightly.

\subsubsection{Interviews}
In addition to ad-hoc unstructured interviews with athletes, coaches, and other officials XNT officials and individuals in the Chinese rugby community (n = 15), I conducted 26 semi-structured interviews, each lasting between 15 and 70 minutes.  All interviews were conducted in Modern Standard Chinese (Mandarin) and were recorded with participant consent using a sound recording application on my smartphone or laptop computer.

During semi-structured interviews, I asked athletes about their personal background (including their family situation), their motivations for adherence to rugby, perceived costs and benefits of adherence to rugby, their perceptions of role in the team, their experience of playing rugby (particularly feelings of flow, dissonance, and team click).  The structured interviews also involved two tasks, one in which the athlete was required to rank motivations for adherence to rugby, and one in which the athlete was asked to report their three closest friends in the team, the three team members most willing to sacrifice on behalf of the team, and three most competent athletes in the team (see Appendix for full script). Athletes answered these questions using a pen and paper. I later collated and uploaded these responses to Evernote.


\subsubsection{Surveys}
 I conducted a number of informal surveys designed to understand athletes' general and specific experiences group membership.  I conducted surveys following three training sessions: a session in which athletes (predominantly junior athletes) ran an aerobic fitness test involving straight-line running shuttle-running at and above the aerobic threshold (Beep Test), and two training sessions involving internal game-like scenarios.  In addition, I asked athletes about their general experiences of team membership agency over the team (weak-strong), role in the team (central-marginal), individual performance (weak-strong), team performance (weak-strong), training intensity (\textit{qiangdu})(light-heavy) at two three-month intervals.














%%%%%%%%%%%%%%%%%%%%%%%%%%%%%%%%%%%%%%%%%%%%%%%%%%%%%%%%%%%%%%%%%%%%%%%%%%%%%%%%%%%%%%%



I arrived in Beijing late on a Friday evening at the end of August in 2015.  My close friend Kai---a former Chinese National rugby team representative, and now a lawyer working in Beijing---picked me up at the airport and drove me back to his home.  The plan was to stay with Kai until I was able to make solid arrangements with the Beijing Temple of Agriculture Sports Institute, the home of the Beijing Provincial Rugby Team.

After a day of acclimatising and running errands, on Saturday evening I was invited to a dinner hosted by Adrian, a respected elder within the Chinese rugby community of Beijing.  Adrian was the captain of the second ever class of rugby players to graduate from the Chinese Agricultural University in Beijing---the birthplace of rugby in China.
I first met Adrian through Kai in 2013,

 while coaching in China.  I was originally introduced to Adrian by Kai, a close friend of mine who I met earlier during another stint in Beijing in 2008.

 Kai was also at the dinner, as was Mr Shi, a sports television producer at Chinese Central Television (CCTV).  CCTV had just accepted the rights to the Rugby World Cup, which World Rugby---the international governing body of rugby---had made available to CCTV in an attempt to promote the development of the game in non-traditional rugby playing nations.


I arrived in Beijing late on a Friday evening at the end of August in 2015.  My close friend Kai---a former Chinese rugby player, graduate the Chinese Agricultural University (Chinese rugby's birthplace), and now a lawyer working in Beijing---met me at the airport and drove me back to his home.  When we got to his home, Kai turned on the television and we caught up while some footage from a rugby documentary played in the background.  As it so happened, the international rugby world was on the verge of another Rugby World Cup, which was being hosted by England in the coming months. World Rugby, the world governing body of rugby union, had made the television broadcast rights for the World Cup available to Chinese Central Television (CCTV), in an attempt to promote the game globally.  Having accepted the rights to the tournament, which is the 3rd largest sporting event in the world behind the Olympics and the Football World Cup, CCTV were in search of Chinese rugby experts to help produce the 48-match tournament.  As Kai quickly explained, CCTV's search had led them to the Chinese rugby community based in Beijing, who were almost all, like Kai, graduates of the Chinese Agricultural University---the birthplace of rugby in China and the base for the Chinese National Rugby team between 1996 and 2010.  In fact, CCTV's search led them first to Adrian, the captain of one of the first CAU rugby teams (1992), and currently working for a large international sport organisation in Beijing.  Adrian had then contacted his younger university brother (师弟) Kai, who like him was fluent in English and able to assist in sourcing and translating rugby materials relevant to the broadcast. The CCTV producer responsible for the broadcast, Mr Shi, had scheduled a dinner with Adrian and Kai on Saturday (tomorrow) night to thank them for their willingness to assist in the production.  I was also invited to the dinner. I wasn't scheduled to meet with the Principle and head coach of the Beijing Temple of Agriculture Sports Institute until the following Monday, so I agreed to accompany Kai.
