



\begin{savequote}[8cm]

    We hear a lot these days about genes and molecules, but how does one human brain, or one human being coordinate, or entrain, or resonate with another?  We may not realise it but we live in a world of coordination, at every level and every scale of endeavour.

  \qauthor{--- J. A. S. Kelso  \textit{Coordination and the Complimentary Nature} Presentation to the The New York Academy of Sciences - May 12, 2010}
\end{savequote}




\chapter{\label{chap:intro}Introduction}



\minitoc





                                          \begin{CJK}{UTF8}{gbsn}

\section{Group exercise as carnal mystery\label{sect:adrian}}


On my first night in Beijing, Adrian, Kai, and I sat in the upstairs area of the Korean BBQ restaurant in a quiet willow lined street just inside Beijing's East 4th Ring Road.  Adrian was the host of the dinner, and so he naturally held the floor in conversation while the three of us waited for his colleague Mr Shi to arrive.

Adrian was a veritable elder of Chinese rugby.  He had been captain of the second class of rugby players to graduate from the Chinese Agricultural University (CAU), the home of China's first official rugby union program, established in 1990.  Kai---also a CAU graduate (2007)---was a close friend of mine of many years, and had invited me to join at the dinner soon after I touched down earlier that day.  I was eager to catch up with old friends as well as begin my fieldwork, and so, despite my jet lag, I accepted the invitation.

Adrian reminisced fondly about his time playing rugby at CAU, as well as his time after graduation playing with the Beijing Devils, a rugby club in Beijing whose members were predominantly expats.  He assured us that rugby in China was, in those days, fun and free-spirited.  Not like today, now that rugby has become a professional program in the state-sponsored sport system (owing to the Olympic status of the modified form of the game, rugby sevens, see Chapter~\ref{chap:researchSetting}).  Adrian talked about going on a rugby tour to the UK with the Beijing Devils:  ``Everyone only just scraped together the money to go on tour.  We all paid our own way. Sometimes you'd get a bit of help from someone or whatever. We did it because we loved the game, not for any other reason.''   Kai and I listened intently.  All of a sudden I realised that this conversation could be relevant, so I started taking notes.

When Mr Shi finally arrived, Adrian continued his nostalgic story telling mode, but naturally shifted his target audience from Kai and me to Mr Shi---his guest who knew very little about rugby.  Adrian began to describe in rich detail the experience of camaraderie between he and his Beijing Devils team mates when they participated in overseas rugby tour.  At one point Adrian interrupted his own story to make an explanatory aside directed at Mr Shi, accommodating for the fact that Mr Shi was relatively unacquainted with the sport: ``This sport, rugby: it's actually very mysterious. If you haven't played it yourself you might not know this type of feeling,'' (英式橄榄球这个项目其实特别神秘,没玩过的话您可能不知道这种感觉) Adrian respectfully suggested to Mr Shi.  ``Because rugby, you know, you're all on the field together, there's body contact...'' (因为英式橄榄球么,大家在场上有身体接触) he paused to find the right phrasing,  ``It's a very \textit{carnal} type of feeling.'' (是一种``肉''的感觉) His attempts to enrich his communication by gesticulating had led him to have both of his hands clenched as fists in front of him like they were cradling a rugby ball or gripping the steering wheel of a car---a lit cigarette smouldering between the index and middle finger of his right hand.  Adrian concluded by looking into the distance and repeating: ``Its very mysterious.'' (特别神秘) He shook his head as if baffled and finally released his clenched fists to dab the ash from his cigarette into the ashtray in front of him.  After taking another drag from his cigarette he finally added: ``So it means this rugby circle here in China is very tight...'' (所以在橄榄球这个圈子特别亲)---a short pause for another dab of his cigarette--- ``...but it doesn't mean that this circle is not also not also complete chaos!'' (但这不是说这个圈儿也不乱!)
  \footnote{Circle (\textit{quanzi} 圈子) is a common colloquial way to refer to a social group or community of people in modern standard Chinese.}
The wisdom of Adrians's final punchline was confirmed with a knowing chuckle from all of us, including Mr Shi. Adrian concluded his performance silently, by taking a long, satisfying drag of his cigarette and looking off into the most distant corner of the restaurant.

I was captivated---but also somewhat surprised---by Adrian's monologue.  I was not expecting, so early into my fieldwork, to happen upon a declaration in which a link between the carnal (\textit{rou} 肉) or visceral sensations associated with on-field joint action, and social processes of interpersonal affiliation (\textit{qin} 亲) and group cohesion of the rugby community (\textit{quanzi} 圈子) was so explicitly and spontaneously emphasised.  It was clear that rugby's visceral dimension continued to capture Adrian emotionally; some 10 years after he had finished playing rugby his fists still clenched with energy, and his head still shook with amazement.

I was also intrigued that Adrian cited the source of his emotional capture as at once both very specific (derived from playing together with others on the field) and, at the same time, ultimately ``mysterious.''  The aim of my research is to contribute to an explanation the human behavioural phenomenon of group exercise in terms of its social, evolutionary, cognitive, and physiological causal processes and dynamics.  In essence, the aim is to somehow move from mystery to scientific mechanism.  At this first dinner in Beijing, Adrian's comments both captured the phenomenological mystery of group exercise, and pointed me in the direction of the underlying explanatory mechanisms.

Needless to say, I left that first dinner eager to investigate the sources of Adrian's experience of mysterious carnality and social connection in rugby's joint action.  My next stop, on Monday morning, was the Temple of God of Agriculture Sports Institute, where I began ethnographic research with the Beijing Provincial rugby program.


                            \begin{center}
                              * * *
                            \end{center}





\section{Group exercise as evolutionary puzzle}
Competitive team sports, whirling Sufi dervishes, late-night electronic music raves, Masi ceremonial dances, or the fitness cults of Cross-fit and Soul Cycle---endless examples can be plucked from across cultures and throughout time to exemplify the human compulsion to come together and move together.  How can we explain the prevalence of these activities in the human record?  In this dissertation I contribute to a scientific understanding of ``group exercise''---defined herein as physiologically exertive and socially coordinated movement---by way of a focussed study of the social cognition of joint action among professional Chinese rugby players.

Because physical movement is a metabolically expensive task, it is justifiable from an evolutionary standpoint only if the benefits somehow outweigh the costs.  Using this basic calculus, it is easy to imagine how group exercise would have served important survival functions in our ancestral past.  Activities involving group exercise such as hunting, travel, communication, and defence all appear to confer immediate and obvious benefits to individuals and groups \citep{Sands2010}.

In more recent domains of human history, however, the task of explaining the persistent recurrence of group exercise is more complicated.  At least since the late Pleistocene era (approx. 500ka), and particularly since the Holocene transition from hunter-gatherer to agricultural (approx. 11ka), and later industrial and post-industrial societies, group exercise can be identified in shared cultural practices as varied as religion, organised warfare, music, dance, play, and sport.  But unlike group hunting or defence, the fitness-relevant benefits of group exercise in cultural practices such as sport, music, or dance are not always as obvious.  On the contrary, many of these activities appear on the face of it to entail extreme time and energy costs for very little immediate reward.

Thus, the prevalence of group exercise in a diverse array of shared cultural practices in the more recent human record presents an evolutionary puzzle.  A solution to this puzzle requires a more nuanced calculus that incorporates an appreciation of humans' species-unique evolutionary trajectory, defined by increasingly complex cognitive and cultural capacities, including technical manipulation of extra-somatic materials and ecologies; advanced theory of mind; and information-rich, malleable, and scalable communication systems \citep{Roepstorff2010,Clark2015,Fuentes2016}.  A theory capable of satisfactorily explaining group exercise within humans' distinctive evolutionary parameters is yet to be fully formulated \citep{Cohen2017}.

As Adrian's monologue demonstrates, the experience of group exercise appears to be at once a physical, emotional, and social phenomenon.  The fact that these dimensions appear to coalesce in experiences of group exercise suggests that cultural activities in which group exercise features may have played an important human-specific processes of social cohesion \citep{Dunbar2010,Whitehouse2004,Cohen2017}.  In  this dissertation, I advance existing cognitive and evolutionary understandings of group exercise by formulating and testing a novel theoretical relationship between group exercise and social bonding---the glue of group cohesion.  Existing theories of group exercise and social bonding do not yet satisfactorily account for the variation in, and complexity of, interpersonal movement coordination common to many real-world settings of group exercise.  I concentrate in particular on the proximate cognitive mechanisms  ``joint action,'' defined herein as any form of social interaction whereby two or more individuals coordinate their actions in space and time to bring about a change in the environment \citep{Sebanz2006}.  I draw attention to the bio-psycho-social effects known to occur when active, in-the-moment, and on-line joint action functions successfully---i.e., when joint action ``clicks'' between co-actors.  I propose the phenomenon of ``team click'' as a construct that captures the phenomenology of optimal performance in dynamic joint action.  More importantly, I propose team click as a psychological construct that can help explain a hitherto under-appreciated causal pathway between joint action and social bonding.  Team click offers a vehicle through which the various dynamical interlocking dimensions of experience in group exercise can be analysed in terms of their cognitive and evolutionary significance.

%As I demonstrate in this introduction, novel theoretical synthesis---in addition to empirical research---is required in order to more fully align scientific explanations for group exercise with their mysterious, carnal, and embodied (i.e., subjective) dimensions.

In the chapters that follow, I formulate and test a novel theory of joint action and social bonding through a series of three empirical studies with professional Chinese rugby players.  These studies include 1) an ethnographic study of the Beijing men's rugby team (Chapters~\ref{chap:ethnoSetting}\nobreakdash~\ref{chap:ethnoResults}), 2) an \textit{in situ} survey study of professional Chinese rugby players during a National Rugby Tournament (Chapter~\ref{chap:tournamentSurvey}), and 3) a controlled field experiment with a sample of professional Chinese rugby players across two of China's provincial programs (Chapter~\ref{chap:trainingExperiment}).  In each of these studies I find evidence in support of the predictions of this dissertation.  Specifically, more positive perceptions of joint action predict higher levels of perceived team click; higher levels of team click predict higher levels of social bonding, and in some instances, team click fully or partially mediates a direct positive relationship between  joint action and social bonding.  Each study progresses from its predecessor in a step-wise manner: I start with broad and rich ethnographic observation and analysis, and build on these observations towards a more quantitative verification of hypothesised mechanisms.  Findings from each study offer initial substantiation of a novel theory of joint action and social bonding in group exercise.  In so doing, this dissertation sheds new light cognitive and evolutionary processes in human behaviour, psychology, and sociality.

In what remains of this chapter, I review existing theories of group exercise and social cohesion (Section~\ref{sect:GESoCo}), and point to empirical and theoretical knowledge gaps that require attention (Section~\ref{sect:empKnowGaps}).  In particular, I identify the under-theorised relationship between successful performance of complex joint action and perceptions of team click.  I conclude by previewing the research setting and the main components of this dissertation (Section~\ref{sect:components}).

%and conclude with an outline of the chapters of the dissertation (Section~\ref{sect:chapters}).


\section{Existing scientific explanations of group exercise\label{sect:GESoCo}}

In this section I outline existing cognitive and evolutionary accounts of group exercise.  In this dissertation, I define group exercise broadly as any activity that minimally entails 1) sustained physical exercise (i.e. physical activity that reaches at least low intensity physiological exertion of 45\% of max heart rate or above), and 2) coordination of behaviour between two or more individuals in time and space to bring about change in the environment \citep[a.k.a., joint action, see][]{Sebanz2006,Vesper2010}.  A broad definition of this nature allows for the identification of instances of group exercise in a wide array of contexts throughout the human record, from music making and dance, to ritual practices, to competitive sport, warfare, play, and even pair-bonding activities such as sexual intercourse.  Anecdotal and ethnographic evidence pertaining to humans' subjective experience of all of these activities suggests that group exercise commonly entails a prominent and unmistakable visceral dimension. But, as I explain below, the natural scientific origins of this dimension of experience have remained unclear.


\subsection{Evolutionary origins of group exercise}
Archeological, paeleological, and primatological evidence suggests that human’s capacity for group exercise runs deep into our evolutionary trajectory.  Humans have evolved distinct morphological and physiological adaptations that enable sustained aerobic exercise.  Adaptations include derived skeletal features that support stable bipedal running and respiratory function \citep[see ][]{Bramble2004}, as well as exercise-specific neurobiological reward \citep{Raichlen2012}.  The archeological record indicates that a capacity for sustained physical exercise may have emerged at some point in the transition from Pan (i.e., Australopithecus) to Homo, and could be associated with activities such as persistence hunting, defence of territory, and communication over vast distances such as those conceivable in woodland or grassland savanna, as opposed to dense canopy rainforests of pre-human ancestors \citep{Sands2012}.
    \footnote{While all species of the Homo genus (Sapiens, Erectus, and Habilis) all appear to show evidence of morphological adaptation for running, none of these species appear to born sprinters.  Even elite human sprinters (capable of sustaining maximum speeds of only 10.2 meters per second for less than 15 seconds) are slow compared to mammalian cursorial specialists such as horses, greyhounds and antelopes, who can maintain maximum galloping speeds of 15–20 meters per second for several minutes \citep{Garland1983}.  Moreover, running is more costly for humans than for most mammals, demanding roughly twice as much metabolic energy per distance travelled than is typical for a mammal of equal body mass \citep{Taylor1982}.  This suggests that humans have evolved capacities suited specifically to endurance running at moderate intensities, rather than short bouts of high intensity sprinting \citep{Bramble2004}.}

The co-occurence of physical exertion with joint action has been identified in activities of non-human primates, suggesting that the human propensity for group exercise scaffolds on top of strong evolutionary foundations laid well before our last common ancestor with Chimpanzees ($\sim 5-7$ mya).  Group hunting in Chimpanzees is by now well-documented and is known to be both spontaneous, highly organised \citep[for example involving divisions of set roles][]{Boesch1989}, and causally linked to social bonding \citep{Mitani2001}.  Chimpanzees behave similarly in group territorial ``warfare'' against members of neighbouring groups \citep{Boehm1992,Wilson2014a}.
In addition to group hunting and territorial conflicts, Chimpanzees also engage in exertive group activities such as group laughter \citep{Waller2005}, or coordinated display patterns of tree buttress drumming, pant-hooting, and branch-dragging \citep[for example, observed as part of a ``rain dance,'' see][]{Goodall2000,Whiten2001}.  Meanwhile, bonobos display a propensity for highly arousing socio-sexual encounters, which are hypothesised to quell intra- and inter-group tensions (relating to competition over food or reproductive resources) through the activation of neuropharmacological reward mechanisms \citep{Dunbar1992,Parr2005,Clay2015}.  Taken together, this evidence suggests that group exercise a core underpinning to various adaptive social behaviours in non-human primates, ranging from hunting, to group defence, and even social affiliation and diplomacy.

\subsection{Group exercise as the object of social scientific analysis\label{sect:GEsoSci}}
Social scientists and anthropologists have long speculated that cultural activities in which group exercise feature are somehow central to the function of human sociality.  Sociologist Emile Durkheim emphasised the emotional importance of shared ritual practice, which he thought could ``change the conditions of psychic activity'' (p. 469).  This group-level property of ``collective effervescence,'' Durkheim argues, ``strengthen[s] the bonds attaching the individual to the society'' and has important behavioural consequences relating to prosociality and well-being (Durkheim 1915/1965; pp. 257-258). ``Once the individuals are gathered together,'' Durkheim notes of an indigenous Australian tribe engaged in ritual dance, ``a sort of electricity is generated from their closeness and quickly launches them to an extraordinary height of exaltation...''  As a compliment to his more famous essay on the role of shared cultural practices in generating ties of social reciprocity---`The Gift''---Marcel Mauss (Durkheim's nephew) also drew attention to the importance of the visceral, embodied dimension to human social life in his 1935 essay ``Techniques of the Body'' \citep{Mauss1935}.
    \footnote{The intuition that the body is at the centre of social processes has been developed throughout an entire French social scientific lineage.  Mauss was Durkheim's nephew, and phenomenologist Maurice Merleau-Ponty \citep{Merleau-Ponty1956}, in turn was influenced by Mauss.  Anthropologist Pierre Bourdieu \citep{Bourdieu1990}, and contemporary sociologist Loic Wacquant \citep{Wacquant2004} are also direct descendants of this line of inquiry, which invariably fixates on the causal relevance of ``embodied practice'' to social phenomena.}
Meanwhile, anthropologist Victor Turner highlighted the drama of ritual performance (invariably involving group exercise) in creating a playful and liminal space separate from the usual structure of everyday social life in which ``spontaneous communitas'' and ``humankindness'' between ritual participants is fostered \citep{Turner1974}.  These accounts share a common focus on the relationship between embodied psychophysiological processes and social processes.

While initial anthropological accounts of group exercise in human sociality lacked an a rigorous scientific framework through which explanatory mechanisms could be tested and verified, the importance of the visceral and emotional dimensions of group exercise to human sociality was nonetheless intuitively grasped and duly emphasised.

\subsection{The sidelining of group exercise through the modern evolutionary synthesis and the cognitive revolution \label{sect:visceralSideline}}

Generally speaking, more testable scientific accounts of human behaviour have emerged only in the last $\sim$70 years,  enabled by 1) the gradual refinement of evolutionary theory, now known as the ``modern evolutionary synthesis'' (hereafter MES) and 2) the ``cognitive revolution'' (hereafter CR) of the 1950s and 60s \citep{Laland2010}.  The MES amounts to a unification of the theory of evolution by natural selection (attributed to Darwin and Wallace in the second half of the 19th century) with a theory of genetic inheritance \citep[replacing a previously popular theory of blended inheritance, see][]{Calcott2013}.  The MES thus paved the way for a ``gene-centred'' understanding of biological evolution, as changes in the frequency of heritable DNA sequences in a population due to selection pressures exerted at the level of the phenotype \citep{Grafen1984}.  Meanwhile, concepts from information theory, cybernetics, and computation provided the necessary language to describe cognition and evolution as variational, digital, and sequential processes of change \citep{Yockey2005}.  Forefront to initial applications of cognitive and evolutionary approaches to human behaviour were questions of the biological origins of human sociality, including our species-unique capacities (e.g., language and culture) for complex patterns of cooperation and coordination \citep{Wilson1975,Chomsky1965}.

Together, MES and CR enabled anthropologists to draw upon empirical findings and theoretical frameworks from neighbouring fields of cognitive science, psychology, evolutionary biology, and behavioural ecology, to develop hypotheses concerning proximate psychological and cultural causes of, and ultimate (evolutionary) explanations for human sociality \cite[e.g.][see Appendix~\ref{sect:modernSynthesis} for a more detailed explanation of MS, CR, and their applications to human behaviour)]{Dawkins1976,Wilson1978,Sperber1996,Whitehouse2004,Dunbar1996}.  However, the necessarily rudimentary parameters of initial MES models of human cognition and evolution also limited the extent to which these models could account for a full spectrum of behaviour observable in humans' evolutionary niche.

For instance, the gene-centred view of the MES deploys the assumption that an organism's developmental processes are not causally relevant to evolutionary change, and instead population-level distribution of gene frequencies is determined solely by natural selection \citep[including associated stochastic mechanisms of mutation, and genetic drift][]{Grafen1991}.  \textcite{Hamilton1964}, for example, explained how social behaviours that appear seemingly costly to the individual phenotype could be \textit{indirectly} beneficial in the case that such behaviours increased the reproductive success of other individuals carrying the same gene (expressed famously in Hamilton's rule of $rb > c$).  As such, it was shown that the evolutionary value of a behaviour or phenotypic trait could be calculated as the sum of direct and indirect fitness benefits to an organism over its lifespan \citep{Grafen2006}.  This core concept of the MES, known as \textit{inclusive fitness}, has enabled a thorough consideration of the evolutionary origins of human social behaviour since it was first applied to human populations \citep{Axelrod1981,West2011,Abbot2011}.  Within the gene-centred orthodoxy of the MES, cognition, social learning, and (cultural) communication systems (among other aspects of the human evolutionary niche) are understood as (adaptive) extensions of the phenotype, which have evolved according to their contribution to the inclusive fitness of the organism over evolutionary time.
  \footnote{Evolutionary biologist Richard Dawkins, for example, famously proposed that human culture should be understood as a digital, variational, and heritable evolutionary system in its own right that furnishes humans' ``extended phenotype'' \citep{Dawkins1982}.  Just as evolutionary biologists model genes, anthropologists can model ``memes'' as units of cultural selection that transmit and fixate within populations.}

%Here, human culture is understood as a proximate mechanism of the phenotype, but with evolutionary properties in its own right (related to but distinct from the replication dynamics of genes).

The implication of these assumptions for cognitive and evolutionary approaches to human behaviour has been to direct the attention of empirical research towards cognitive mechanisms that are fixed by processes of natural selection \citep{Lickliter2003,Kenrick2001}.  Researchers began to hypothesise the existence of evolved cognitive mechanisms that 1) function according to their contribution to inclusive fitness \citep[an approach that has since matured into the field of evolutionary psychology][]{Cosmides1992}, or 2) are conducive to the effective \citep[i.e. ``cumulative'' see][]{Tomasello1993} transmission of cultural variants \citep[an approach that has since matured into the field of cultural evolution][]{Cavalli-Sforza1981,Boyd1988}.  As such, cognitive mechanisms pertaining to aspects of social decision making \citep[in the case of game-theoretical models of cooperation, see][]{Cosmides1989,West2011} or memory and social learning \citep[in the case of gene-culture coevolutionary models][]{Henrich2003} have been most actively examined for evidence of their ultimate evolutionary significance \citep{Badcock2012}.  Focus on these cognitive mechanisms and evolutionary processes was further assisted by the furnishing of MES with a theory of change advanced by the CR, which conceived of species-unique aspects of human cognition as primarily logical, grammatical, or symbolic processes of digital (and amodal) information transfer.\footnote{Refer, for example, to the language of digital computation utilised by Ernst Mayr in his explication of the need to distinguish between ``proximate'' and ``ultimate'' levels of biological phenomena;][]{Mayr1961}).}  In brief, initial gene-centred approaches to human cognition and evolution focussed less on the visceral, embodied, and emotional dimensions to human experience, and thus afforded less space for a testable account of group exercise in human sociality.

%---traditionally understood to be ``implicit''---in favour of an analysis of

To be sure, researchers have worked steadily over the past 70 years to nuance cognitive and evolutionary approaches to human behaviour, which has enabled more inclusive explanations of observable features of human behaviour and sociality.  Evolutionary psychologists, for example, propose that evolved cognitive mechanisms (adaptations) may not predict survival or reproductive success in proximate contexts in which they are currently observable \citep[as per the assumption utilised in human behavioural ecology known as the ``phenotypic gambit;'' see][]{Grafen1984}.  Rather, evolved cognitive mechanisms influence behaviour in ways that were likely to have performed an adaptive function over evolutionary time \citep[a theoretical formulation known as the ``environment of evolutionary adaptiveness'' (EEA)][]{Cosmides1992a,Buss1988}.  Models of cultural evolution adjust population genetic models to take into account the observable differences between cultural and genetic information, such as culture's capacity to support one-to-many transmission, the blending of cultural variants, and non-randomly guided variation \citep{Cavalli-Sforza1981,Boyd1988}.  These adjustments are part of the concession that cultural variants are not as dependent on high fidelity replication as their genetic cousins, but instead are shaped by evolved cognitive biases that favour the acquisition and transmission of some cultural variants over others due to their memorability or effectiveness \citep[i.e., context sensitivity][]{Henrich2007}. E:

Meanwhile, the functional role of emotion has been foregrounded---in neuroscience \citep{Damasio1994}, cognitive science \citep{Lazarus1982}, and various strands of developmental \citep{Campos1989}, social \citep{Parrott2001}, positive \citep{Frederickson2001}, and cultural \citep{Nisbett2003} psychology.
Within evolutionary psychology, it has been recognised that emotion may have evolved as a superordinate mechanism for regulatory function \citep{Cosmides2000}, and may play an important (albeit subordinate) role in regulatory function  decision making \citep{Dalgleish2004} and communication \citep{Rime2009}.  In all of these approaches, however, the foundational assumptions of the MES (selection at the level of the phenotype according to inclusive fitness) and CR (primacy of explicit, propositional, or semantic processes of information transfer) remain largely in tact.  As such, the role of physiological, emotional, and ecological dimensions of human cultural activity---including those that feature group exercise---are not afforded causal primacy and therefore remain deemphasised in accounts of the natural origins of human life \citep{Badcock2012}.

To summarise, the first 70 years of cognitive and evolutionary approaches to human behaviour and sociality have produced explanatory models that tend to draw empirical attention to species-unique traits that can be understood to enhance inclusive fitness (in the case of evolutionary psychology) or otherwise facilitate cultural transmission (in the case of cultural evolution).  While these scientific approaches have been groundbreaking for understanding human behaviour and sociality, these attempts do not yet satisfactorily account for the mysteriously visceral dimension to group exercise reported by Adrian and identified by Durkheim.

%, which hinge on the cognitive capacities---e.g. imitation, teaching, and memory---for ``cumulative culture'' \citep{Tomasello1993}).

\subsection{Bondedness and sociality\label{sect:bondednessSociality}}
More recent waves of scholarship have addressed the sidelining of physicality in human cognition and evolution exemplified by group exercise.  As Anthropologists Robin Dunbar and Susanne Shultz have aptly pointed out, humans do not merely adhere to a``dung fly'' model of sociality, by incidentally \textit{aggregating} in time and space according to genetically hard-wired protocols \citep[see][]{Wilson1975}.  Rather, humans actively \textit{congregate} in structured, cohesive groups, which are held together by emotionally-mediated relational bonds \citep[777]{Dunbar2010}.  What makes human sociality unique, according to Dunbar, is the quality of ``bondedness''---the human capacity to forge social bonds with genetically unrelated members, which results in a more adaptive ecological niche \citep[see][]{Odling-Smee2003}.
Because being able to maintain the effective functionality of a group through time may have very significant individual fitness benefits for its members, the emergent property of sociality itself can be understood as part of the individual’s fitness strategy \citep{Dunbar2010b,Nowak2010}.

Dunbar argues that human bondedness and sociality cannot simply entail cognitive processes \citep[at least not in the way cognitive processes are narrowly rendered by game-theoretic and gene-culture coevolutionary models][]{Dunbar2010}.  Rather, bonded relationships involve two parallel and quite distinct mechanisms—--a cognitive mechanism (derived from what Dunbar calls the ``social brain''\citep{Dunbar1998}, or what has been otherwise understood as an evolved ``norm psychology'' \citep{Chudek2011}), and an emotionally based form of attachment \citep[often involving a psycho-pharmacological mechanism][]{Dunbar2010b}.  Thus, by appealing to the physiological (emotional) basis of relational ties between individuals, Dunbar scientifically reinstates the visceral dimension to human sociality that is observable in cultural activities involving group exercise.

  %This and which was temporarily absent in initial waves of cognitive and evolutionary approaches to human behaviour.

Humans' evolved capacity for social bonding is thought to have arisen in primates as an adaptive response to the pressures of group living.  Aggregating in groups serves to reduce threat from predation, but at the same time can be individually costly due to stress arising from interaction at close proximity, and conflict over resources among genetically unrelated individuals.  These costs can lead the group to disband, and are hypothesised to have led to selection for social bonding via dyadic grooming, as the coalitional alliances that arise among grooming partners allow for the maintenance of the group by buffering the stresses of group living \citep{Dunbar2012}.  Primate social grooming leads to the release of endorphins \citep[a type of endogenous opioid, see][]{Keverne1989}, presumably leading to sustained rewarding and relaxing effects \citep{Dunbar2010}.  While other neurotransmitters such as dopamine, oxytocin and/or vasopressin may also be important in facilitating social interaction, it has been suggested that endorphins allow individuals who are not related or mating to interact with each other long enough to build ``cognitive relationships of trust and obligation'' \citep[1839]{Dunbar2012}.

As the homo genus evolved more complex collaborative capacities for survival in interdependent group contexts \citep[see][]{Dunbar1998,Tomasello2012a}, grooming-like behavioural technologies (such as group laughter, music making and dance, and collective ritual) also evolved \citep[via processes of multi-level cultural group selection, see][]{Wilson2008} to sustain social bonding in larger group sizes where dyadic grooming would take too much time \citep{Dunbar2012,Tarr2014,Launay2016}. When neuropharmacologically-mediated mood-elevating effects are experienced in a group they seem to lead to participants embodying each other’s affective experiences, resulting in more positive, trusting, and cooperative relationships among participants \citep{Dunbar2012}.

Bondedness provides a vehicle for a scientific comprehension of what Durkheim described as collective effervescence.  Accounting for bondedness in human sociality requires, on ultimate, evolutionary level of analysis \citep[see][]{Mayr1961,Tinbergen1963}, the loosening of the strict singular determinacy of natural selection on population level gene frequency.  In its place, bondedness suggests a multi-level or niche-construction approach, whereby individual-level benefits can be generated through the production and evolution of cultural activities involving genetically unrelated social actors operating within a shared ecological niche \citep{Dunbar2012,Laland2010,Laland2015}.  On a behavioural or ``proximate'' level of analysis, bondedness requires a conceptual broadening of mechanisms of cognition to incorporate the causal role of physiological (as opposed to purely cognitive) mechanisms.  Social bonding provides the immediate ``social'' glue for social cohesion \citep[see][]{Lakin2003,Bastian2014a}, and this achievement depends crucially on physiological (and not just cognitive) processes.

\subsection{The social high theory of group exercise and social bonding \label{sect:socialHigh}}

Does group exercise generate bondedness in human sociality?  Research focussed on the proximate physiological, cognitive, and social mechanisms associated with group exercise confirms that group exercise is responsible for generating a psychophysiological environment conducive to social bonding.  Anthropologist Emma Cohen and colleagues have recently identified bi-directional causal links between the two essential ingredients of group exercise---1) physiological exertion and 2) interpersonal movement coordination---and their common psychophysiological effects, including increased pain tolerance, athletic performance, positive affect, wellbeing, pro-sociality, and cooperation \citep{Davis2015}.  This evidence amounts to what can be called the ``social high'' theory of group exercise and social bonding \citep[hereafter ``the social high theory,'' see][]{Cohen2017}. Here, social bonding is understood as the psychological experience of increased social closeness, which facilitates affiliation between non-kin group members \citep{Tarr2014}.  I outline the key principles of the social high theory below, before introducing some empirical and theoretical knowledge gaps around group exercise and bondedness that require further attention.

\myparagraph{Physiological exertion\label{sect:physExertion}}
The social high theory proposes a causal link betweeen physiological exertion and social bonding.  Group exercise necessarily entails rigorous physiological exertion.
The health and wellbeing benefits associated with regular physical exercise---including reduced risk of cardiovascular disease, autonomic dysfunction, and early mortality; as well as enhanced neurogenesis, cognitive ability, and mood---are becoming increasingly well-known \citep{Blair1994,Nagamatsu2014}. Evidence suggests that strenuous and prolonged physical exercise is modulated by the same neuropharmacological systems responsible for regulating pain, fatigue, and reward \citep{Boecker2008,Raichlen2013}.  Neurobiological rewards in exercise are associated with both central effects (improved affect, sense of well-being, anxiety reduction, post-exercise calm) and peripheral effects (analgesia), and appear to be dependent for their activation on exercise type, intensity, and duration \citep{Dietrich2004}.  Exercise-specific activity of neurobiological reward systems offers a plausible explanation for commonly reported sensations of positive affect, anxiety reduction, and improved subjective well-being during and following exercise---extremes of which are popularly referred to as the ``runner's high'' \citep{Dietrich2004,Boecker2008,Raichlen2012}.  This neurobiological evidence maps on to more extensive literature concerning the psychological effects of exercise, which indicates a duration and intensity ``sweet spot'' for exercise and positive affect, whereby moderate intensity exercise for durations of $\sim45$ minutes appears most optimal \citep{Reed2006}.

It is possible that the function of exercise-induced positive affect extends to the realm of social bonding, particularly when achieved in group exercise contexts \citep{Cohen2009,Machin2011}.  Endocannabinoids and opioids have been implicated in mammalian social bonding \citep{Fattore2010,Keverne1989}, and in humans specifically, there is evidence that endorphins (a particular class of endogenous opioids) mediate social bonding \citep{Dunbar2012,Shultz2010}.

\myparagraph{Interpersonal movement coordination\label{sect:synchrony}}
The social high theory also proposes a causal link between joint action and social bonding.  Experimental evidence (predominantly from social psychology) suggests that time-locked coordination of behaviour between two or more individuals is conducive to psychological processes of self-other merging, liking, trust, and psychological affiliation.  In these contexts, interpersonal coordination is primarily operationalised as behavioural synchrony---i.e., stable time- and phase-locked movement of two or more independent components (limbs, bodies, fingers, etc.) \citep{Pikovsky2007}. Researchers suggest that synchrony enables a tight attentional union between individuals who match the timing and content of their actions, leading to the enhancement of interpersonal similarity and the blurring of self-other boundaries in cognitive processing and recall \citep{Cohen2017}.  Relative to non-synchronous group activities, synchrony increases social bonding and pro-social behaviour \citep{Reddish2013,Reddish2013a,Wiltermuth2009,Tarr2014}---an evolutionarily important outcome of bonded relationships.  Recent studies have also found that, compared to solo and non-synchronous group exercise, synchronous group exercise leads to significantly greater post-workout pain threshold \citep{Cohen2009,Sullivan2014,Sullivan2013a, Sullivan2013b}.

A recent meta analysis of the behavioural synchrony literature in social psychology suggests three candidate mediators of the relationship between behavioural synchrony and social bonding: 1) lower cognitive affective mechanisms implicating neuropharmacological reward systems (e.g., opioidergic and dopaminergic systems), 2) neurocognitive action-perception networks responsible for the experience of self-other merging, and 3) processes of group-centred cognition responsible for perception and reinforcement of cooperation \citep{Mogan2017}.  The current balance of existing evidence suggests that affective physiological mechanisms may be more relevant to joint action involving larger group sizes in which generalised feelings of euphoria and pro-sociality are common \citep[e.g., mass religious rituals or music festivals, see][]{Weinstein2016}, whereas neurocognitive mechanisms linking joint action and social bonding may be more applicable to smaller group sizes in which individuals can share intentions through ostensive communicative signals and implicit movement regulation cues \citep{Lang2017}.  Studies linking synchrony with social bonding and cooperation are supported by a literature than connects nonconscious mimicry with liking and affiliation \citep{VanBaaren2009}.

\myparagraph{The social high $=$ exertion $\times$ synchrony}
In addition to recorded independent effects of exertion synchrony,  preliminary evidence suggests that exertion and coordination in group exercise interact to produce social effects \citep{Jackson2018}.  Social features of the exercise environment (for example, perceived social support, level and quality of behavioural synchrony, etc.) modulate exercise-induced mechanisms of pain and reward \citep{Cohen2009,Sullivan2014,Tarr2015,Davis2015,Weinstein2016}. This work is bolstered by existing literature on the social modulation of pain \citep{Eisenberger2012a}, and links between pain and prosociality \citep{Bastian2014a}.  The social high theory thus combines these two bodies of literature to tell a story in which positive affect---associated with neuropharmacologically-mediated pain analgesia and reward—--is extended to the social group via synchrony-activated cognitive mechanisms of self-other merging, and the perception and reinforcement of in-group cooperation.


\section{Empirical knowledge gaps in the relationship between group exercise and social cohesion\label{sect:empKnowGaps}}
While the social high theory has served to empirically flesh out---particularly at the proximate level of analysis---the relationship between group exercise and Dunbar's notion of bondedness, the theory remains limited in its ability to account for a full spectrum of profiles and subjective experiences in group exercise.
Even a cursory survey of human sociality reveals that group exercise scenarios often deviate markedly from the narrowly defined profile (the exertion $\times$ coordination sweet spot) and subjective experience of group exercise set out by the prevailing social high theory.  In this section, I review some of the empirical gaps that remain unexplained by the existing social high theory.  In particular, group exercise contexts often involve extreme (and not just moderate) levels of physiological exertion, as well as complex coordination demands (beyond exact synchrony).  In addition, the effects of participation in group exercise appear to extend well beyond those of a feel-good social high, and into the realm of rich meaning making and fine-grained sensitivity to the click of joint action.  I suggest that these empirical gaps in the social high theory point to a more fundamental \textit{theoretical} gap in cognitive and evolutionary approaches to human behaviour. In short, the vast majority of cognitive and evolutionary approaches to human behaviour are unable to sufficiently account for the dynamic interlocking of cognitive, physiological, and social processes.  The dynamic dimensions of group exercise contexts serve to bring this theoretical gap into sharp focus.

%In this dissertation I propose to develop cognitive and evolutionary understandings of group exercise through closer attention to the immediate causal processes and psycho-social effects of interpersonal movement.  I suggest that the carnal mystery that Adrian attempted to articulate on my first night in Beijing can be explained in part by the way in which extreme physiological exertion and interpersonal movement coordination combine in rugby to enable and constrain physiological, psychological, emotional, and social processes.

\myparagraph{Group exercise involves both extreme physiological cost and profound meaning\label{sect:linkCostMeaning}}
Anecdotal and observational evidence suggests that group exercise contexts often entail extreme levels of psychophysiological exertion.  High-stakes professional competitive sporting contexts (international-level sports such as rowing or ultra marathon running), extreme adventure sports (big wave surfing, free-diving), high-intensity contact (rugby union, American football, ice hockey) and combat sports (MMA, boxing, wrestling) are known to involve extreme physiological demands, often including high levels of pain.  Although it is expected that extremely physiologically costly exercise contexts will involve activation of neurobiological reward mechanisms outlined above (see Section~\ref{sect:physExertion} in this Chapter, and Appendix~\ref{sect:neuroRewardGE}), some contexts may on average exceed (or alternatively never reach) the intensity and duration sweet-spot for optimal activation of neurobiological reward \citep{Raichlen2013} or positive affect \citep{Ekkekakis2011,Reed2006}.

At the same time, physical exercise involving extreme physiological, psychological, and social costs also appear to offer participants and observers an opportunity for profound meaning.  Many people do not engage in exercise \textit{just} enjoyment or health; rather, in some contexts sport forms part of a life of purpose and self-discovery \citep[see, for example][]{Jackson1995,Jones2004,White2011}.  Modern sport has always been much more than ``just a game,'' and instead offers an arena in which virtues and vices are learned, and the ``morality plays''—--of community, nation, or globe—--thus performed \citep{Elias1986,McNamee2008}.  Psychological and physiological resilience in exercise contexts is lauded as virtuous, as is evidenced by the numerous idioms in the English language that receive currency in exercise lore: ``when the going gets tough, the tough get going,'' ``no pain, no gain,'' ``you get out what you put in'' and so on \citep{Sarkar2014}.

Whereas the social high theory predicts motivation for exercise based on ``hedonic'' enjoyment, anecdotal and ethnographic perspectives emphasise instead the ethical and moral dimensions of athletes' experiences, and contextualise these experiences within political processes relating to the construction of the self, community, and nation-state \citep{Alter1993,Brownell1995,Downey2005,Wacquant2004}.
Social anthropologists and sociologists have for some time emphasised the social function of exercise and sport in diverse cultural contexts, and various attempts have been made to analyse the phenomenological experience of exercise in terms of its sociological and psychological meaning \citep{Bourdieu1978}.

Social anthropologist Joseph \textcite{Alter1993}, for example, argues that for wrestlers in north India, the body functions as a nexus through which the symbolic and material structures of the state, family, and the individual coalesce.  In a similar vein, in their seminal ethnography of sport in China, cultural anthropologist Susan Brownell \textcite{Brownell1995}, argues that sport functions as a crucial national symbolic practice for the Chinese nation-state in a project of ``rejoining the world,'' and that the ``micro-techniques'' (c.f. Foucault, 1977) of this project entail significant cost to (and rich meaning for) the individual athlete.   Similarly, French sociologist Loic Wacquant \textcite{Wacquant2004}, in an ethnography of boxers in Chicago's south side, describes a ``social logic'' of physical activity, claiming that the costs associated with ``the daily dedication and high technique that training demands; the regimented diet; the control, mutual respect, and tacit understandings necessary for actual fist-to-fist competition serve to create for the boxer an island of order and virtue'' \textcite[17]{Wacquant2004}. In many instances, it may be that the primary psychological motivation for exercise is not immediate, hedonic wellbeing, but instead \textit{eudemonic} wellbeing, or the psychological awareness of a process through which life becomes ``well-lived'' \citep{Fave2009,Huta2013}.

\myparagraph{Group exercise demands complex coordination which can lead to team click\label{sect:linksComplexClick}}
A similar connection between physiological, cognitive, and social processes in group exercise can be witnessed in joint action of group exercise contexts.  Currently, the social high theory relies upon exact behavioural synchrony as an idealisation of successful coordination in joint action.  While some group exercise contexts do contain high levels of behavioural synchrony \citep[rowing, synchronised swimming, diving, mass calisthenics, and forms of dance such as ballet, see][]{McNeill1995}, exact in-phase synchrony is not typical of most instances of group exercise.  More generally, interpersonal coordination is more often achieved through flexible, function-specific assemblages of complimentary and contrasting behaviours \citep[for example, coordination in an interactional team sport, a dyadic conversation, or an ensemble music performance, see][]{Fusaroli2014}.  Real-world instances of joint action in group exercise usually entail various distinct action elements, organised hierarchically within a sequence \citep{Schmidt1975,Rosenbaum2009}.  Successful execution of the structure of real-world joint action requires temporal and spatial precision and flexibility of movement across multiple timescales and sensorial modalities \citep{Sebanz2006,Pacherie2012}.
In essence, successful coordination in joint action typical of group exercise contexts requires considerable cognitive resources \citep{Turvey1978}, which are not budgeted for when joint action is modelled as exact in-phase synchrony \citep{Keller2014}.

% As discussed below in Section~\ref{sect:pathBeyondSynch}, this reliance on synchrony could occlude important causal mechanisms in a relationship between joint action and social bonding.

At the same time, participants in group exercise contexts involving complex joint action often scrutinise the quality of coordination, and derive powerful psychological reward when complex joint action clicks.  Technically demanding group exercise contexts such as competitive interactional team sports or music-making and dance, depend upon fine-grained precision of coordination of behaviours between individuals: the movements and goals of one individual must align precisely in time and space with the movements and goals of another.  For highly skilled expert practitioners, often the ecstasy of group activity is contingent not just on on reaching a certain level of physiological exertion, or resting on exact synchronisation of behaviours with others, but on the extent to which performance in joint action satisfies or exceeds implicit and explicit expectations.  Consider the passage below, taken from a series of interviews that psychologist Susan Jackson performed with elite figure skaters:
%Psychologist Susan Jackson has accumulated considerable evidence of elite level athletes' subjective experience of ``flow'' in joint action:
  \begin{quotation}
    It was just one of those programs that clicked. I mean everything went right, everything felt good . . . it's just such a rush, like you feel it could go on and on and on, like you don't want it to stop because it's going so well.  It's almost as though you don't have to think, it's like everything goes automatically without thinking . . . it's like you're in automatic pilot, so you don‘t have any thoughts.  You hear the music but you're not aware that you're hearing it, because it's a part of it all. \citep[168]{Jackson1992}.
  \end{quotation}

The psychological literature of flow and optimal human performance in sport suggests that athletes engaged in team coordination often report total absorption in and complete focus on the task at hand, a transformation of the experience of time (either speeding up or slowing down), and a blurring or transcendence of individual agency, or a ``loss of self''   \citep{Csikszentmihalyi1992,Jackson1995,Jackson1999,McNeill1995}.  Research suggests that flow often occurs in scenarios in which there are clear goals inherent in the activity, as well as unambiguous feedback concerning extent to which goals are either being achieved or not.  In addition, scenarios most conducive to the experience of flow are those in which the technical requirements are challenging but achievable if practitioners are able to extend (slightly) beyond their normal capabilities\citep{Fong2015}.  The coalescence of these factors is intrinsically rewarding and autotelic \citep{Csikszentmihalyi1975}, activating both ``hedonic'' and ``eudemonic'' dimensions of subjective well-being \citep{Huta2010,Fave2009}.

In sum, the social high theory is not yet sufficiently equipped to explain instances of group exercise that deviate from a narrow profile of moderate intensity exertion, exact synchrony, and a feel-good social high.  Anecdote and ethnographic observation suggest that group exercise contexts also involve extreme levels of physiological cost, rich psychological meaning making, cognitive complexity, and feelings of team click.  Importantly, extreme physiological cost appears to be tethered to cognitive and social processes of meaning making and social identity (Section~\ref{sect:linkCostMeaning}).  Similarly, cognitive complexity in joint action appears to be linked to embodied, affective experiences of flow, eudemonic wellbeing, and team click.  Together, these empirical gaps in the social high theory suggest the importance of dynamic interlocking between proximate cognitive and physiological mechanisms for the generation of psychophysiological experience of group exercise.

\myparagraph{The empirical knowledge gaps in the social high theory expose the need for theoretical synthesis}
How do we develop a scientific explanation for the mystery to which Adrian's monologue alerts us, or, as I have described it above, the  dynamic interlocking of physical, cognitive, and social processes in group exercise?


In this dissertation I suggest that developing a naturalistic account for the knowledge gaps outlined above will require more than simply patient empirical examination of the componential mechanisms already identified by the social high theory.

The social high theory's predominant focus on physiological and affective dimensions of group exercise and social bonding serves to address the sidelining of these dimensions mechanisms in processes of human evolution (see Section~\ref{sect:visceralSideline}).  However, in both the social high theory and the theory of bondedness and sociality, cognitive and physiological mechanisms are conceptually and causally partitioned on the proximate level of analysis. Thus, while it is understood that both physiological and cognitive processes are co-determinant of social bonding and---ultimately---social cohesion, the theoretical partitioning of these two mechanisms limits the capacity of both theories to account for the observable fact that human activity (such as that observable in group exercise) can be at once a physiological, cognitive, and social phenomenon.

Empirical examination of group exercise using the current formulation of the social high theory will continue to achieve more and more detailed appreciation of exercise-induced neurobiological reward, and the link between in-phase behavioural synchrony and social bonding. But a more microscopic grasp of these two componential mechanisms alone will not serve to shed light on the knowledge gaps outlined above, or the mystery of group exercise suggested by Adrian.  Novel theoretical synthesis is required to unify physiological, affective, and social dimensions of human activity.

I argue that advancing cognitive and evolutionary understandings of group exercise can benefit from dedicated attention to the longstanding conceptual divides in human behavioural science between physiological (emotional and visceral) and cognitive (mental or semantic) processes.   The fact that group exercise contexts demonstrate a nagging and unmistakable dynamic cross-cutting of these various processes serves to expose a theoretical issue in the bondedness account of human sociality and the social high theory of group exercise and social bonding.  In the following section, I consider the strengths, weaknesses, and applicability to group exercise of existing attempts within cognitive and evolutionary anthropology to address this issue.






\subsection{An active and integrated framework for the social cognition of joint action\label{sect:AIFsoCoJA}}
%could be interwoven with emotion and affect \citep[see][]{Damasio1994},
As explained above in Section~\ref{sect:GEsoSci}, social scientists have long intuited that processes of human sociality are somehow generated by and rooted in physicality.  However, as I then explained in Section~\ref{sect:visceralSideline}, scientific explanations for the role of physicality in cognition and evolution have lagged behind other trajectories of research.  In particular, cognitive and evolutionary approaches to human behaviour have found it comparatively difficult to empirically demonstrate precisely how physicality, cognition, and social organisation could coordinate dynamically to co-determine observable behavioural traits.  As I argue in this Chapter and throughout this dissertation, existing evolutionary understandings of observable behaviour (advanced in the sub-disciplines of evolutionary psychology and cultural evolution, for example) could benefit from a more dynamical conception of cognition \citep{Kenrick2001,Badcock2012,Ramstead2017}.

\myparagraph{Evolutionary Systems Theory}
A long-standing movement within the cognitive and behavioural sciences has called for a greater recognition of the physical and dynamical properties of information transfer in biological systems  \citep{Kauffman1993,Kenrick2001,Hoelzer2006,Yufik2013}.  Some researchers within evolutionary biology have called ardently for an ``extended evolutionary synthesis'' (EES), in which the principles of the MES---specifically the causal primacy of Darwinian selection, inclusive fitness, and the distinction between proximate and ultimate levels of scientific questioning---should be extended to account for the causal role of developmental processes in evolutionary change \citep{Ingold2004,Odling-Smee2003,Pigliucci2007,Laland2011,Laland2013,Fuentes2016}. However, the EES has been (fairly) criticised for lacking a testable alternative model of evolutionary causation and for often  misapprehending principles of the MES such as inclusive fitness and the proximate-ultimate distinction in biological enquiry
\citep{Badcock2012,Abbot2011,Svensson2017,Dunbar1996,Scott-Phillips2011}.

A more ``consilient'' \citep{Wilson1999} and testable proposal entails a marriage between a theory of (general) selection (MES), with a theory of dynamical self-organisation  \citep{Kenrick2001,Hoelzer2006,Badcock2012}. In this arrangement, known as ``Evolutionary Systems Theory''  \citep[EST][]{Badcock2012,Badcock2017,Ramstead2017} traditional (substantive) evolutionary theories of general selection allow for a scientific account of species-typical evolved cognitive mechanisms, while self-organisation provides a (non-substantive) set of tools and principles to analyse phenotypic-level variation in, and coordination of, these evolved mechanisms \citep{Badcock2012}.  Despite proposing consistent principles of self-organisation occurring across cognitive and evolutionary timescales, EST honours the practical scientific need for distinction between proximate and ultimate levels of biological analysis \citep{Ramstead2017}.

%\citep[or alternatively in response to Tinbergen's four questions concerning evolutionary function, phylogeny, ontogeny, and real-time mechanism][]{Tinbergen1963}.

For example, on a proximate level, the EST approach has shown how self-organisation can function as a kind of phenotypic adaptation \citep[by facilitating phenotypic plasticity and adaptability][]{Lewis2000}, or as a source of variation (additional to mutuation and drift) on which selection can operate \citep[in addition to ][]{Ploeger2008}. On an ultimate level, meanwhile, self-organisation has enabled the consideration of a plurality of evolutionary mechanism beyond natural selection alone \citep{Kauffman1993,Carporael2001}.  \textcite{Hoelzer2006} argue for the complementarity of selection and self-organisation at an evolutionary level, demonstrating how selection typically results in more efficient energy processing at any level of biological organisation (the cell, multicellular organism, population or ecosystem)---thus the maximisation criteria of both selection and self-organisation are highly complementary.  While cognitive and evolutionary approaches to human behaviour have begun to consider the consilience of selection and self-organisation \citep{Lansing2003,Claidiere2007,Claidiere2014}, principles of self organisation have yet to be fully incorporated into proximate level questions of analysis, particularly at the level of cognitive mechanism \citep{Badcock2012}.


\myparagraph{4E Cognition}
Proponents of so-called embodied, embedded, enactive, or extended cognition \citep[now collectively referred to as ``4E cognition,'' see][]{Menary2010}, emphasise that cognition typically involves acting with a physical body on an environment in which that body is immersed.  Researchers in this movement call out traditional ``stimulus-response'' model of human cognition for being too static, abstract, and compartmentalised: static and abstract because traditional models of cognition render human behaviour as the outcome of a linear chain of perception, a-modal representation, and action-selection; compartmentalised because human cognitive processes are traditionally understood to be located discretely either within the brain or else within certain (often dualistic) cognitive subsystems \citep[e.g., emotional and cognitive, System 1 (fast) and System 2 (slow), implicit and explicit, and so on; see][]{Diennes1999,Kahneman2011}.  4E proponents contend that traditional models of cognition create a dichotomy between cognition and physicality, and reproduces a fixation on semantic representations as the foundation of cognitive and cultural processes.

The most radical proponents of the 4E perspective on cognition challenge the idea that human interaction and communication requires that humans be endowed with the capacity for explicit and content-rich cognitive representation.  Instead, humans' diverse cultural repertoires can be explained primarily by embodied processes of dynamic coupling with the physical features of the environment \citep{Gallagher2001,Gallagher2008,Fuchs2009}.  The core of the 4E argument is that human inferential processes not only activate physical movement, but are also activated by movement, in a dynamical loop of reciprocal causation.

The 4E approach has made valuable contributions to articulating what traditional stimulus-response theories of human cognition have lacked, namely, a conception of the dynamical and physical properties of information transfer \citep{}.  However, in directing attention to the dynamical and content-neutral (or content-free) processes of movement regulation, the 4E turn in cognitive science has traditionally struggled to provide a viable account of humans' species-unique capacity for complex and semantically rich mechanisms of information transfer, embodied in language, shared narratives, and physical artefacts \citep{Ramstead2016}.  In order to transcend the theoretical tension between traditional and 4E approaches to cognition, a dynamical model of cognition must be able to account for the spectrum of cognitive processes that enable human-typical behaviour.

\myparagraph{An ``active'' theoretical solution}
A combination of recent advances in neuroimaging technologies \citep{Frith2007}, neurocomputational theories of brain function \citep{Friston2010,Frith2010,Yufik2013}, and constructive attempts to extend the theoretical paradigm of human social cognition to account for inter-individual processes of interaction and coordination \citep{Sebanz2006,Dale2014}, has led to eventual coalescence of theoretical approach capable of explaining the fact that human cognition entails an interlocking ensemble of cognitive processes that span dynamical coupling, through to content-rich semantic representations \citep{Roepstorff2011,Ramstead2016}. The prevailing paradigm, which I consider in this dissertation, conceives of this broadened spectrum of cognition as a process of ``active inference'' \citep{Friston2010}.  Active inference depicts a human cognitive system in which perception, attention, and action are functionally and temporally integrated to manage uncertainty or ``free energy'' inherent in interactions with the environment \citep{Clark2013}.  As I explain in detail in Chapter ~\ref{chap:theory}, the Active inference approach allows for the unification of physicality and cognition (traditionally understood) in human sociality and its evolution.

Active inference enlists the neurocomputational paradigm of predictive coding \citep{Rao1999,Clark2013} in order to theorise the free-energy minimising mechanics of human cognition.  Predictive coding depicts a radical inversion of traditional models of cognition that rely predominantly on bottom-up sensory inputs and top-down feature detection \citep{Marr1985}.  Instead, active inference posits that top-down predictive models themselves shape perception and action, and the only information that travels forward (or from the ``bottom-up'') is the error signals that arise from discrepancies between predictions and the sensorium \citep{Clark2015}.

Rather than being restricted to placing emphasis on either a) functionally distinct cognitive modes of inference ( \citep[e.g., habitual or mental, explicit or implicit, slow or fast; ][]{Dienes1999,Kahneman2011}, or b) central or peripheral \citep[see][]{Fodor1985}, modular or interactionist \citep{Barrett2006}, content-rich or content-free \citep{OBrien2004,Ingold2001} cognitive mechanisms, active inference predicts that humans benefit from flexible deployment of multiple strategies from a unified web of neural and extra-neural affordances \citep{Pezzulo2013,Clark2015}.  The strategies that inform generative predictive models in joint action can be understood as existing along a continuum, which ranges from basal (and more encapsulated or modular) lower cognitive mechanisms of movement regulation that facilitate more direct coupling with the extra-neural resources of the task-specific environment on the one hand \citep[see][]{Riley2011}, to higher-order (and more associative) cognitive mechanisms responsible for explicit and propositional social communication, on the other \citep[see][]{Semin2012,Annilla2016}.  The flexible deployment and operation of these strategies is governed by an overarching principle of free energy minimisation \citep[the ``Free Energy Principle''][]{Friston2006}.

%Team click and social connection in JA (AIF):The active inference approach, and its relevance to joint action, team click, and social bonding will be thoroughly reviewed in the following chapter.


%The active inference approach depicts a human cognitive system in which perception, attention, and action are functionally and temporally integrated to manage uncertainty inherent in interactions with the environment.
%Patterned practices, affordances cue predictions and create sites of causally dense relationships between cognition and culture.regimes of attention which give rise to shared understandings, skilled intentionality,Just as selection is relaxed as the primary determinant of






\section{Thesis road map}


To summarise the ground covered thus far: this thesis is driven by the overarching goal of contributing to a scientific explanation of the puzzling ubiquity of group exercise in the (more recent) human record. Current cognitive and evolutionary accounts suggest a relationship between group exercise and social cohesion, due in part to the way in which group exercise contexts uniquely generate social bonding between participants.  The social high theory posits that social bonding in group exercise can be explained by the way in which physiological exertion and interpersonal movement coordination combine to generate a psychophysiological environment conducive to affiliation and trust.  However, in its current formulation, the social high theory does not account for a full spectrum of experiences of group exercise, particularly instances in which physical, cognitive, and social processes appear to dynamically interlock. In order to address this gap, I propose the prevailing active inference paradigm of human cognition \citep{Friston2010}. When applied to joint action\citep{Friston2015,Friston2015a}, active inference offers a way of reconciling the distinction between physicality and cognition that is impeding empirical progress in existing cognitive and evolutionary theories of human behaviour and sociality.








\subsection{Research aims and questions (and will funnel to Hypotheses and then Predictions)}



1.	Contribute to a scientific explanation of how coordination of physical movement in joint action can lead to social bonding in group exercise.
2.	Develop a general theory account of "team click" in group exercise
3.	Consider how dynamical principles of information transfer can improve existing theories of cognition and evolution in human sociality



\subsubsection{Research questions}

•	What is the relationship between joint action and social bonding in group exercise?

•	Existing research suggests a functional, bi-directional relationship between dynamical coordination of physical movement and social communication in joint action.
o	Dynamical coordination of physical movement in joint action entails "functional interpersonal synergies" (Riley et al. 2011), whereby components of the movement system (e.g., agents and their constituent arms, limbs, and intentions) couple to 1) reduce the overall complexity and uncertainty (i.e., the dimensionality or degrees of freedom) of the system, and 2) increase reciprocal compensation between system components.
•	Dynamical coordination of physical movement can be identified by "pink noise" and action-perception links.
o	Social communication in joint action entails a spectrum of strategies ranging from more implicit and embodied informational cues to more explicit and deliberate communicative signals that enable co-actors to establish, monitor (predict), and sustain joint action.
•	More explicit and measurable forms of social communication in joint action include: 1) attribution of agency to, and evaluation of, social categories of self, other, and collective in joint action, and 2) shared (cultural) norms that "smooth" coordination by providing a shared referent for joint action.
•	The combination of dynamical coordination of physical movement and social communication function to reduce cognitive complexity and uncertainty of joint action scenarios.
o	In general, more effective dynamical coordination of physical movement leads to more effective, fluent, and efficient social communication, and vice-versa.
•	This thesis demonstrates that, in group exercise contexts, extreme levels of uncertainty and complexity in joint action make it more likely that more positive perceptions of collective performance in joint action will generate high quality social connection, i.e., social bonding.



















\subsection{Research setting}

Research setting:
  Suitability
  Qualifications of the researcher

In this dissertation, I focus my research on a group exercise context that is well suited to application of the theoretical synthesis outlined above.  Rugby union (hereafter ``rugby'') is a dynamic field-based contact sport that requires of its participants high levels of physiological exertion and complex coordination of joint action  (see Chapter~\ref{sect:rugbyUnion} for a more detailed explanation).  As I outline in more detail below, in many of the contexts in which it is commonly played, rugby is anecdotally and colloquially associated with experiences of team click in joint action, as well as social processes of group membership \citep{Dunning2005}.

``Rugby'' and ``China'' are two words that are not usually mentioned in the same sentence.  Rugby---''a game for barbarians played by gentlemen''---first took root in the elite education institutions of Britain's colonial empire.  China has enthusiastically adopted sport and exercise at different stages throughout the country's turbulent modern history. But for most of this history, rugby accorded neither with a dominant Olympic-centred logic of the Chinese sport system, nor with dominant cultural dispositions and modes of understandings physicality \citep[which derived from hundreds of years of continuous history of Confucian and Daoist traditions of thought, see][]{Morris2004}.  Rugby was eventually officially established in China in 1990, as a university sport program at the Chinese Agricultural University (CAU).  However, since 2009, when rugby became an Olympic sport in the form of ``rugby sevens'' (the modified seven-a-side version of rugby), rugby has been ``embosomed'' (\textit{huaibao} 怀抱) by the state-sponsored sport system \citep{Xu2010}.  At the time of writing, more than ten of China's 34 provincial level prefectures now have full time men's and women's professional programs.

While rugby has an institutional footprint in China, Chinese adherents to the sport face various challenges in the process of acquiring rugby's technical and social skills.  Rugby in China contains fewer of the cultural scaffolds that are erected around the sport in traditional rugby playing nations: young children do not grow up playing rugby in the schoolyard or watching their heroes and heroines play rugby on television.  Beyond rugby, China has traditionally struggled to perform well on the world (and regional) stage in interactional team sports like association football \citep{Gallagher2018}.  A number of social, economic, and cultural factors are at play in the phenomenon of China's poor performance in team sport.  Suffice to say, however, the social construct of an abstract and arbitrary egalitarian assembly---well-known in Western cultures as a ``team''---is far from an indigenous psychological concept in China \cite[instead, the family functions as a primary metaphor for social interaction][]{Liu2009}.  As I discuss in greater detail in the ethnographic study of this dissertation (Chapters~\ref{chap:ethnoField}\nobreakdash~\ref{chap:ethnoResults}), the technical and social requirements of rugby appear to chafe against more predominant cultural affordances in modern China.

In spite of the lack of snug fit between rugby and dominant modes of social cognition in contemporary China, rugby is evidently responsible for generating a mysterious ``carnal'' feeling in its participants, as Adrian's monologue and my ethnographic observations professional rugby players in China confirm (see Chapters~\ref{chap:ethnoField} and~\ref{chap:ethnoResults}).  Rugby in China thus presents an excellent opportunity to explore the role of cultural variation in reliably shaping patterns of cognition in human activities such as group exercise \citep{Ramstead2016,Mesoudi2013}.  Indeed, the viability of the active inference framework on a proximate level hinges on its capacity to account for phenotypic (including cultural) variation \citep{Badcock2012}.  The powerful predictions of the active inference framework are yet to be thoroughly tested against real-world instances of human behaviour, however.  How does the imported cultural affordances of rugby (including the on-field parameters of rugby's joint action and the social constructs of ``team'') interact with affordances more indigenous to China's ancient and continuous cultural trajectory?  How does this interaction shape experiences of joint action, team click, and social bonding?

In addition to these questions, rugby in China presents an opportunity to subject an inherently ``WEIRD'' \citep[Western, Educated, Industrial, Rich, and Democratic; see][]{Henrich2010d} suite of cognitive and evolutionary theories to a non-WEIRD empirical setting.

Finally, as I explain in more detail in Chapter~\ref{chap:researchSetting}, my personal qualifications uniquely position me to conduct research into rugby in China: I am professionally fluent in Modern Standard Chinese (Mandarin), have a background as a professional rugby players, and I have spent four of the last 12 years building trusted relationships with members of the Chinese rugby community.


In sum, researching rugby in China offers an opportunity to address many of the outstanding theoretical and empirical questions in scientific explanations of group exercise.  The core contribution of this dissertation is to develop (and test) a theory of joint action and social bonding capable of explaining the seemingly universal and hitherto under-appreciated visceral dimension to activities in which humans come together and move together.







\subsection{Methodological approach}

Anthropology is thus well placed to expand upon accounts of group exercise, via methods ranging from ethnographic exploration capable of uncovering novel dimensions of behaviour and generating testable hypotheses, to quantitative techniques---e.g., experimental and mathematical simulation paradigms---capable of testing hypotheses.

Method: Anthropology is well placed to contribute (Whitehouse bit)
  Step-wise:
  Ethnography (situate general account and test validity of hypotheses in real-world setting) (Whitehouse, Fuentes, )
  In situ studies:
   survey study (move beyond the tribe)
   controlled field experiment (Pseudo experiment: Xgalatas2013)
Important compliment to close experimental work testing proximate mechanisms and ultimate evolutionary work


In this dissertation, I demonstrate the utility of a combination of ethnographic observation for exploration and hypothesis generation, and quantitative field-experimental methods for hypothesis testing. In honour of the capacity of cultural and ecological affordances to shape and direct observable behaviour in distinctive ways, I deliberately confine the three empirical studies of this dissertation to one specific research setting—--i.e., professional rugby players in China.
Each study builds on the previous study in a step-wise manner, and as such the cultural and ecological affordances associated with the group exercise context can be identified and held relatively constant.




\subsection{Main components of the thesis}

Table etc




%\section{Addressing empirical gaps through a focussed study of rugby in China\label{sect:researchSetting}}
%\section{Thesis overview\label{sect:contributions}}


%\myparagraph{Team click mediates a relationship between joint action and social bonding}
As discussed above, in this dissertation I focus on the theoretical pathway between successful performance of complex joint action and feelings of team click and social bonding.  I propose the phenomenon of team click as a psychological construct that can help explain a relationship between joint action and social bonding in group exercise.  I test the validity of an hypothesised relationship between joint action, team click, and social bonding, through three empirical studies.  In addition to the three empirical studies, the theory I develop in Chapters~\ref{chap:theory}\nobreakdash~\ref{chap:theoryGE} is also highlighted as a major contribution of this dissertation.

I begin with an in-depth ethnographic study of a real world group exercise setting---the Beijing men's rugby team ~\ref{chap:ethnoSetting}\nobreakdash~\ref{chap:ethnoResults}.  I then refine my theoretical predictions based on the results of ethnographic analysis, and test these in an \textit{in-situ} survey study of a more representative sample of professional Chinese rugby players (Chapter~\ref{chap:tournamentSurvey}). I administered surveys to athletes ($n = 174, male = 90$) before, during, and after a two-day National Championship Tournament in order to ascertain information about their experience of a high-intensity and high stakes joint action scenario.  These two studies provided the necessary empirical motivation for a controlled field experiment ($n = 58, male = 30$) designed to interrogate specific mechanisms hypothesised to underpin the phenomenology of team click and social bonding in joint action.  The core components of this thesis are outlined in Table  ~\ref{tab:thesisComponents}).

% Please add the following required packages to your document preamble:
% \usepackage{booktabs}
\begin{table}[]
\centering
\begin{tabular}{@{}llcl@{}}
\toprule
\textbf{} & \textbf{Components}                                                                               & \multicolumn{1}{l}{\textbf{Chapters}} & \textbf{Description}                                                                                                                                         \\ \midrule
          &                                                                                                   & \multicolumn{1}{l}{}                  &                                                                                                                                                              \\
1         & \begin{tabular}[c]{@{}l@{}}A novel theory of social\\ bonding through joint\\ action\end{tabular} & Ch. 2-3                               & \begin{tabular}[c]{@{}l@{}}Supported by an active inference framework; \\ proposes team click as a mediating construct\end{tabular}                          \\
          &                                                                                                   &                                       &                                                                                                                                                              \\
2         & Ethnography                                                                                       & Ch. 5-7                               & \begin{tabular}[c]{@{}l@{}}Beijing men's rugby team (n = 26); \\ Data: participant observation, \\ informal surveys, semi-structured interviews\end{tabular} \\
          &                                                                                                   &                                       &                                                                                                                                                              \\
3         & \begin{tabular}[c]{@{}l@{}}\textit{In situ} survey \\ study\end{tabular}                                   & Ch. 8                                 & \begin{tabular}[c]{@{}l@{}}Conducted during the The National 7s Rugby\\ Championship (n = 174, male = 93)\end{tabular}                                       \\
          &                                                                                                   &                                       &                                                                                                                                                              \\
4         & \begin{tabular}[c]{@{}l@{}}Controlled field\\  experiment\end{tabular}                            & Ch. 9                                 & \begin{tabular}[c]{@{}l@{}}Conducted with athletes from Beijing \\ and Shandong provincial rugby \\ programs, (n = 58, male = 30)\end{tabular}               \\ \bottomrule
\end{tabular}
\caption{Core components of this dissertation}
\label{tab:thesisComponents}
\end{table}




%In this section, I outline the components of this thesis, namely, the novel theory of social bonding through joint action, and the three empirical studies I performed to substantiate this theory (see Table






\end{CJK}{UTF8}{gbsn}
