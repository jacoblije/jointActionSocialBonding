

\chapter{\label{introduction}Introduction}

\minitoc
%%%%%%%%%%%%%%%%%%%%%%%%%%%%%%%%%%%%%%%%%%%%%%%%%%%%%%%%%%%%%%%%%%%%%%%%%%%%%%%%%%%%%%%%%%
%%%%%%%%%%%%%%%%%%%%%%%%%%%%%%%%%%%%%%%%%%%%%%%%%%%%%%%%%%%%%%%%%%%%%%%%%%%%%%%%%%%%%%%%%%
%%%%%%%%%%%%%%%%%%%%%%%%%%%%%%%%%%%%%%%%%%%%%%%%%%%%%%%%%%%%%%%%%%%%%%%%%%%%%%%%%%%%%%%%%%
Sun Hongwei arrived escorted by his high school athletics coach to the Beijing Agricultural Temple Institute of Sport (hereafter the Institute) soon after I began my fieldwork in August 2015.  An 18 year old with a very slight build and timid demeanour---his gaze always diverted to the ground during the first few months of his tenure at the Institute---Hongwei had never seen a rugby ball before the day he arrived.
Hongwei was from Hebei province, the province that immediately surrounds the special prefecture of Beijing, China's capital.  As was relatively common practice in professional sport programs such as this one, Hongwei's coach had organised a trial for Hongwei with the Beijing Provincial Men’s and Women's Rugby Program (hereafter the Rugby Program) by calling upon social connections to the leadership of the Institute.

Athletes come to the Rugby Program from all over the country.  Representing Beijing at a provincial level in a sport like rugby can translate into the opportunity to gain entrance---via a designated ``specialist athlete'' (\textit{tiyu techan} 体育特产) pathway---to one of
China's top universities (Beijing Sports University, in this case) and enhanced career employment opportunities thereafter.  Rugby is not a popular sport in China, and is referred to as a neglected or ``cold-gate'' branch of the Chinese sport system (\textit{lengmen xiangmu} 冷门项目, a term that refers to a profession, trade or branch of learning that receives little attention). Despite its minnow status in China in terms of its popularity, rugby's recent inclusion in the Olympics (in the form of the modified 7-a-side version of ``rugby sevens'') means that it now occupies a prominent place in the Chinese sport system, which has been defined since its inception by a strong Olympic logic \citep{Brownell2008}.  If a sport is a current member of the summer or winter Olympic Games, then the sport is included in the roster of sports played at the quadrennial China National Games (\textit{quanhunhui}). As such, rugby programs such as the one at the Institute now exist in twleve of China's 34 provincial level regions, either embedded within or somehow associated with tertiary education institutions. Thus, although rugby and China are two words that have not historically not commonly featured within the same utterance, rugby in China now affords athletes a rare and under-capitalised opportunity to pursue attractive life-course opportunities of education and employment, in an otherwise intensely competitive sport system.

Almost without exception, the athletes who arrive at the Institute to join the rugby team have not spent their childhoods playing rugby in their school yards or watching professional rugby on television.  In fact, many who come to rugby transition from other more popular sports such as athletics, basketball, or association football, and often---like Hongwei---have never seen a rugby ball before they arrive.  Most ``start from scratch,'' so to speak, in terms of their grasp of the technical and social requirements of the highly interactive and technically complex team sport. In addition to complex patterns of movement coordination, rugby also involves ``full contact'' body-on-body collisions and intense bouts of high physiological exertion, requiring speed, strength, agility, and endurance to perform all of rugby's technical requirements successfully.  Learning the game of rugby from a baseline of essentially zero, while also navigating the demanding social and political dynamics within the team and the Institute, was clearly going to be a daunting task for Hongwei.


\section{The cognitive and evolutionary anthropology of group exercise}
Whirling Sufi dervishes, late-night electronic music raves, Maasi ceremonial dances, competitive team sports, or the fitness cults of Cross-fit and Soul Cycle---endless examples can be plucked from across cultures and throughout time to exemplify the human compulsion to come together and move together.  How is it possible to explain the prevalence of physiologically exertive and socially coordinated movement in the human story?

Physical movement is a metabolically expensive endeavour for all biological organisms,  and appears justifiable, in an evolutionary sense at least, only if the benefits somehow outweigh the costs.  It is easy to imagine how physiologically exertive and coordinated group activities (hereafter group exercise) would have served important survival functions in our ancestral past, such as hunting, travel, communication, and defence \citep{Sands2010}.  The task of explaining the persistent recurrence of group exercise in more recent domains of human history (at least since the late Pleistocene era (\sim500ka), and particularly since the Holocene (\sim11ka)), however, appears to be much more complicated.  Cross-cutting shared cultural practices as varied as religion, organised warfare, music, dance, play, and sport, the fitness-relevant benefits of group exercise are not so immediate or obvious.  Instead, causal explanations for the prevalence of group exercise are thoroughly intertwined with the processes of a species-unique evolutionary trajectory, defined by increasingly complex cognitive and cultural capacities, including technical manipulation of extra-somatic materials and ecologies, advanced theory of mind \citep{Tomasello1999}), and information-rich, malleable, and scaleable communication systems \citep{Fuentes2016}.  A theory capable of satisfactorily explaining group exercise within these distinctive evolutionary parameters is yet to be fully formulated \citep{Cohen2017}.

In this dissertation, I critically consider the adequacy of existing cognitive and evolutionary theories of human behaviour for explaining the ubiquity of group exercise in human sociality.  I do so through specific reference to an empirical case study of professional Chinese rugby players, conducted periodically between August 2015 and September 2017. Central to this case study is observation and analysis of a psychological phenomenon I term ``team click''---a subjective perception  of the tacit \textit{quality} of coordination in joint action among athletes.  Similar in many respects to psychological states associated with ``flow'' and peak performance \citep{Csikszentmihalyi1992}, team click is anecdotally common in a wide range of joint action contexts, and is often associated in these contexts with psychological processes of positive affect and wellbeing, as well as personal agency, social affiliation, and group membership \citep{Marsh2009, Wheatley2012}.  Despite the fact that the phenomenologies of flow and team click exhibit clear social dimensions, very few direct attempts have been made to empirically verify the mechanisms of these experiences in the context of their cognitive and evolutionary explanations \citep[but, for (neuro)cognitive see][]{Dietrich2004,Dietrich2011}\citep[and for an evolutionary framing, see see][]{Slingerland2014}.

From an evolutionary perspective, interpersonal coordination is an adaptive behaviour from which humans derive numerous benefits \citep{Tomasello2014}. It is therefore plausible that we have evolved physiological, cognitive, and social mechanisms that reward effective joint action with others. This rationale underpins the core prediction of this dissertation, namely, that the phenomenon of team click mediates a relationship between joint action and social bonding.  I present ethnographic and field-experimental evidence that tests and confirms this prediction.  I evaluate these results in terms of their implications for understanding the proximate cognitive mechanisms, ecological system dynamics, and ultimate evolutionary processes relevant to the anthropology of group exercise.  As a result of this analysis, I suggest that the social cognition of joint action should be identified as an important research domain in which the sturdiness of assumptions associated with existing cognitive and evolutionary accounts of human behaviour can be critically examined.
% through methodologically diverse and innovative empirical research programs.  While most methodological innovations in this domain will necessarily involve technological advances in quantifying variables of joint action, I also argue for the importance of preserving the ethnographic in order to discern which variables ought to be isolated (hypothesis generation) (for a full discussion, see Chapter 8).
In the sections that follow in this introduction, in addition to continuing the story of Chinese rugby's newest recruit, Hongwei, I also introduce the theoretical framework of this dissertation, by assessing the usefulness of existing cognitive and evolutionary approaches to human behaviour to the study of group exercise.  In particular, I review and identify knowledge gaps in (1) existing research into the proximate social, cognitive, psychological, and neurobiological mechanisms hypothesised to underpin group exercise, and (2) existing cognitive and evolutionary theories that attempt to account for the transmission and proliferation of human cultural activities in which group exercise commonly features (cultural evolution). Based on this review, I formulate the specific predictions of my thesis and introduce the empirical research designed to test them.  Finally, I outline the empirical contributions of this dissertation and suggest theoretical and methodological considerations for the cognitive and evolutionary anthropology of group exercise (which I discuss at length in Chapter 8).

% SHW Story here??
\section{Sun Hongwei, rugby's newest recruit}
Hongwei was the first of the program’s new recruits that I followed closely, and so perhaps I was more attentive to his journey than others.  Compared even to other newly arrived junior athletes, he was noticeably timid and shy, particularly in his interactions with the coaches (myself included) and senior players.   At the same time, however, Hongwei clearly signalled diligence and commitment in the way that he participated in team activities.  He arrived early to each training session carrying more than his fair share of the training equipment required for the session (balls, plastic cone markers, tackling shields, and so on---a task delegated to the most junior members of the team).  Owing to his initial lack of grasp of the basic techniques of rugby, Hongwei was unable to properly participate in normal training with the rest of the team. Instead, during the first month or so Hongwei stood on the sidelines of the training session and practiced the basics with other athletes who were unable to fully participate in training due to injury: learning how to pass and catch both standing stationary and in-motion. In my eyes at least---those of an observer accustomed to instinctual grasp of these distinctive movements from a young age---Hongwei's attempts to accustom himself with the skills of rugby were distinctly jarring.  The bizarre idiosyncrasies of rugby's ovular ball often foiled him, and I would see him chasing after a recently fumbled ball on the ground as if he was chasing after a scurrying rabbit tactfully evading his pursuit.  In my time coaching and playing rugby in China I have watched many start exactly where Hongwei was starting, on the side of training session learning how to pass the ball.  But for some reason I found Hongwei’s attempts to learn particularly unusual.  Hongwei's actions appeared so mechanical that it was almost as if he was deliberately (over)imitating the required actions of passing, catching, and running as a signal of diligence and commitment.

%There was no ``flow'' in his actions, let alone any ``click'' with fellow teammates.  But more than just lack of fluidity of movement, Hongwei also appeared in my eyes lacked personality in his attempts to learn: there was no indication that he was bringing his own understanding to the actions, he remained rigid and mechanical.

A few weeks in to Hongwei’s time at the Institute I asked head coach Zhu Peihou about his newest recruit.  He immediately shook his head and scrunched up his face dismissively, adding in a disappointing whisper, ``no good'' (\textit{buxing} 不行).  Chinese rugby coaches were well acquainted with athletes starting from scratch with the technical requirements of rugby—--they were used to things looking awkward and ugly at the start.  In my experience, coaches appeared more interested in the physical raw materials that would enable athletes to develop into rugby players over time.  Often this meant that coaches had a habit of fixating on an athlete's baseline characteristics of speed, height, and body frame, (as an indication of capacity to build physical size and strength).  Also important, but less crucial than these characteristics, was an athlete’s baseline ball-handling skills, and game sense (which were often assessed in analogous interactive team sport scenarios like basketball or football).  For a rugby player, Hongwei was still relatively young and physically undeveloped, but at the same time was by no means endowed with a big physical frame, nor was he noticeably fast or agile compared to other athletes.  For all these reasons, head coach Peihou couldn’t help but let on to me that he was not particularly excited about Hongwei's future prospects in the Program.  In fact, I got the sense that his reaction to my question contained an element of annoyance or frustration with the terms under which Hongwei had arrived to the Institute via the arrangements of Hongwei’s athletics coach.  Based on his own projections about Hongwei's ability and future potential, Peihou had been forced to accept a ``dud'' into the team.

I interviewed Hongwei approximately 6 weeks after he had first arrived at the Institute.   Hongwei's demeanour during the interview mirrored the demeanour that he presented publicly at training.  Although he was timid and shy, he did also show some signs of excitement and captivation with his new sport and social environment.  When I asked him about his initial impressions of the on-field demand of rugby, however, Hongwei was quick to confess that he felt somewhat unacquainted:

\begin{quotation}
  Hongwei: I still haven’t really started to practice any of the team plays or anything, all I can do so far is pass and run a little bit. \\
  JT: How do you feel about it (so far)? \\
  Hongwei: Its quite fun! \\
  JT: And what do you think is the most difficult component of rugby? \\
  Hongwei: Umm..well, coordinating with teammates [on the field], particularly coordination in attack.  Because I can't figure it out. When I first arrived I didn’t even know what a ``switch play'' or a ``blocker play'' was.
\end{quotation}

\begin{CJK}{UTF8}{gbsn}
  \begin{quotation}
    :战术没怎么接触,就是像传球啊、跑动什么的会一点了 \\
    ?感觉怎么样?\\
    :挺好玩的!\\
    ?你认为橄榄球最难的一部分是什么? \\
    :...打配合,进攻的配合,因为搞不明白,刚来的时候也不知道什么交叉,后插什么的 \\
  \end{quotation}
\end{CJK}


When discussing the technical demands of rugby, Hongwei was tentative and somewhat embarrassed when confessing his lack of grasp of these requirements. As I will explain in more detail in the ethnographic sections of this dissertation, on-field coordination with teammates was by far the most common answer to this question ("what is the most difficult aspect of rugby?"), particularly among junior athletes (rugby training age < 3 years).

When we moved on in the interview to other topics beyond the on-field technical demands of rugby, Hongwei was more positive, framing rugby as an exciting new opportunity, and commenting that his friends and family were in awe of the fact that he is playing such an impressively ``strong'' physical sport like rugby.  Consistent with his diligence at training and timid social demeanour, when I moved on to ask what was new about rugby that he hadn't experienced elsewhere, Hongwei was very automatic in emphasising the social dimensions of his experience at the Institute:

\begin{quotation}
  ``...I think its mainly this thing of having teammates. Before when I was training for an individual sport it was just me training by myself. (In that environment) it was a case of whoever trained well was successful.  But now with this team of brothers, elder teammates will take care of younger teammates. We all train together, and if you can’t do something you can always ask your elder teammates...[Rugby] is so much better, because in an individual sport, if you can’t master something, you have to go to your coach for help. Other athletes don’t want to teach you, because if you surpass other people, then they have to work even harder to keep up... I have learned about helping each other, because unlike individual sports, where you look after your own performance and thats it, in a team sport, if you don’t do well, you can’t get too frustrated or upset, because other athletes will help you out, and I will also help others out, that type of collaboration.''
\end{quotation}

\begin{CJK}{UTF8}{gbsn}
  \begin{quotation}
    :我觉得主要是师哥师弟的这一块儿,原来练个体项目都是自己练自己的,谁练好了谁厉害,但是现在师哥师弟,有师哥照顾师弟带着,互相练,我不会我可以问师哥
    :好多了,因为个人项目你不会就必须要找教练,但是别人不愿意教你,因为你把别人超越了,那别人还还得努力。(3) :学到互相帮助,因为向个体项目自己成绩自己来拿就行,而像团体项目,即使自己做不好,也不用太泄气太沮丧,因为别人会帮你做好,我也会帮别人做好,互相协作的那种.
  \end{quotation}
\end{CJK}

Having migrated from his previous sport of pentathlon (track and field / athletics), it was clear that it was not only the technical skills of rugby were new to Hongwei.  Hongwei makes explicit reference to the mutual social support of the fraternity of the Rugby Program, and his place in it as a junior member.

 %emphasise the importance of the fraternal dimension of the Beijing rugby team, and in particular his role as junior member and the support he gains from more senior athletes.  When I asked Hongwei about what he had learned from being part of the rugby program, he replied:
%At the same time, however, it was also clear from Hongwei’s responses that Hongwei was far from familiar with the on-field requirements of rugby:
%It was possible that Hongwei's diligence and eagerness to declare his allegiance to rugby and to the team through his actions and his declarations to me in interview were overcompensations for insecurity regarding his capacity to grasp the technical components of the sport.  Just like I felt that Hongwei was overly mechanical in his imitation of rugby’s foundational techniques as a signal of diligence and team commitment, perhaps his flourishes about mutual collaboration and team membership were also designed to signal similar diligence and commitment.
\begin{center}
  * * *
\end{center}



A few months passed and Hongwei continued to train: he was eager and committed, and I did notice some improvement in his abilities. But he also remained extremely timid, keeping his head low at all times in team settings.  (I felt like giving him a pat on the back and suggesting that he try to relax and take the weight off his shoulders, thinking that his anxiety about fitting in may be getting in the way of the ultimate goal of fast-tracking skill acquisition!)  Then, one evening when I had returned to my room in the Institute dormitory from a three-week hiatus in Australia over Christmas of 2015, I heard a knock on my open door, and to my surprise Hongwei took an assertive stride into my room, carrying in two arms a bag of rugby balls in need of air before the next day's training session.  Hongwei had never ventured into my room before on his own accord, apart from when he was scheduled for the interview that I conducted with him two months earlier.  Remarkably, Hongwei looked me straight in the eyes with his head held high and energy beaming from his face and chest.  I couldn’t help but smile and ask, with genuine intrigue, ``How has training been recently?''
``Very good'' he said, assertively and excitedly.  ``At least now I know what’s going on at training, I can keep up with the plays!''  A big smile grew on his face as he continued to hold my gaze.  ``Oh good!'' I said, and I congratulated him for his hard work in training while I had been away, and encouraged him to keep working hard. ``Wow,'' ``the ``force'' was in him,'' I remember thinking to myself.  Rugby, Hongwei, and the team in which he was by now enmeshed had somehow interlocked to instil him with a sense of agency.

%Hongwei had begun to develop an innate feel for the game.

A few weeks later, the new head coach of the Program, Chongyi (who took over from Peihou who abruptly resigned while I was away in Australia) told me that he had decided to take Hongwei to pre-season training in Guangzhou, for a month starting in March 2016.  Chongyi admitted that while he was perhaps not the most promising of the junior athletes, his attitude was very good:

\begin{CJK}{UTF8}{gbsn}
  \begin{quotation}
``He [Hongwei] likes to train, and he is very diligent. I want to take his positivity with us [to Guangzhou]'' 他爱练,而且很用心,带上他的积极性过去
  \end{quotation}
\end{CJK}

\begin{center}
  * * *
\end{center}

These ethnographic observations relating to Hongwei's first four months at the Institute highlight key themes of this dissertation.  As I will outline in Chapter 2, successful coordination in joint action appears to hinge on the alignment and maintenance of expectations between co-actors.  Importantly, evidence also suggests that violation of expectations in joint action has strong affective consequences.  The variation I observed in the development of Hongwei's familiarity with the technical requirements of rugby, his demeanor in team settings, and his attitudes towards group membership was worthy of further investigation.  Indeed, Hongwei is an emblematic example of broader team-wide patterns of associations between performance in joint-action, team click, and social bonding.

Interestingly, in Hongwei's case (and in the case of many others, as I will demonstrate in the ethnographic sections of this thesis (Chapters 3 and 4)), what was obvious to me as an observer was not his development in attitudes towards team membership.  Only 6 weeks into his internship at the Institute, Hongwei was relatively articulate concerning his role in the team as junior member and the social support he derived from team membership.  Ideas about familial modes of group membership such as the ones Hongwei expressed in his interview above are culturally salient in China, and are emphasised on every level of social organisation from the immediate nuclear family unit, community neighbourhoods, corporate organisations, or even the Nation State (literally ``State Family'' \textit{guojia}). Be it six weeks or six months into his stay at the Institute, I suspect Hongwei would have produced a relatively similar account of his attitudes towards group membership.  It is also worth considering that Hongwei's fluency in declaring positive, pro-social ideas surrounding group membership may also function as a  signal of prosocial commitment, made from a position of underlying insecurity in the team as one of the newest---and therefore most socially precarious---recruits.  Indeed, as I go on to report in the ethnographic sections of this dissertation, junior athletes of the Beijing Men's Rugby program were generally more willing to make positive pro-team declarations, emphasising ethics of mutualism and feelings of social support. Senior athletes by contrast appeared under less pressure to reproduce such ideas in interview context. Instead, senior athletes were more likely to either remain agnostic to ideas about group membership, or complain to me about the lack of cohesive group membership in the program, due mainly to the lack of consciencioussness of junoir athletes, who were more concerned with computer games, girlfriends, or university degrees than they were with rugby.

Crucially in Hongwei's case, what did appear to vary markedly during this four month snapshot, however, were quantities much less accessible in an interview setting. Conceding that many factors are at play all at once in an ethnographic setting, I nonetheless pursue the possibility that the unmistakable vitality that I noticed in Hongwei after returning to the field that evening was in some part associated with cognitive processes of aligning expectations and coordinating movement in the specific joint action context of rugby.  As I explain below and in Chapter 2, various strands of existing research support this possibility.

%Importantly, here I want to draw attention to the fact that I suspect if I had measured Hongwei's prosociality, it would have been constant pre-post.  What changed was his spirit and agency, the vitality -- how do we measure that?
%- personal agency is contingent on social coordination in - something about the vitality of movement here, the agency found in synergy and flow of movement...

\section{What are the theoretical methodological options available for explaining group exercise?}
In this section, I consider existing cognitive and evolutionary theories of human behaviour and assess their usefulness in explaining group exercise. In addition to the empirical contributions of this dissertation, I also offer a theoretical contribution in the form of a novel cognitive and evolutionary framing of joint action and social bonding, centred on the phenomenon of ``team click'' (detailed in Chapter 2). It is therefore necessary in the following review that like Hongwei I start from scratch, so to speak, by carefully considering core theoretical assumptions of existing evolutionary and cognitive theories, rather than simply making reference to them in passing before proceeding to a discussion of immediate causal mechanisms.  As such, I begin with the foundations of the ``modern synthesis'' of evolutionary theory, before evaluating the ways in which this approach has been applied to the problem of human behaviour.  With these options in mind, I proceed to a review of current approaches to accounting for group exercise.  Subsequently, I preview the characteristics of team click and provide justifications for a novel cognitive and evolutionary framework for the social cognition of joint action, which I outline in detail in Chapter 2.
%as well as literature in social cognition, psychology, and coordination dynamics relevant to its scientific understanding.  I suggest that empirical research of the social cognition of joint action can offer novel insights into the proximate and ultimate evolutionary processes of behaviour, specifically the dynamism of these processes.
%Although the quantification of the complexity of systems dynamics in human behavioural processes will rely heavily on continued innovations in methodological techniques (discussed further in the conclusion), I also suggest an important role for Ethnography as an important (and low cost) qualitative methodology of coming to terms with the behavioural system in question.

\subsection{The Modern Synthesis}
Generally speaking, rigorous and testable (i.e., scientific) attempts to account for human behaviour have emerged in the last ~70 years, owing to 1) the the gradual refinement of evolutionary theory over the last 200 years now known as the ``modern synthesis,'' and 2) the ``cognitive revolution'' of the 1950s and 60s, in which mechanisms associated with information theory, cybernetics, and computation provided useful conceptual metaphors for understanding population-level transmission and fixation of biological and cultural variants.

The modern synthesis (also known as ``neo-Darwinism'', hereafter simply MS) refers specifically to the unification of the theory of evolution by natural selection (attributed to Darwin and Wallace in the second half of the 19th century) with a theory of genetic inheritance (replacing a previously popular theory of blended inheritance).  The mathematical formalisation of this merger by Fisher (1930) and Haldane (1932), and advances in genetics including the verification of the structure of the DNA molecule by Wallace and Crick in 1954, paved the way for a definition of biological evolution as changes in the frequency of heritable DNA sequences in a population due to selection pressures exerted at the level of the phenotype (Dawkins, Grafen).  The mathematically plausible mechanism of genetic inheritance serves to explain observable intra-species phenotypic variation (for which the preceding theory of blended inheritance failed to account), and confirms Darwin's original insight that organismic change occurs via gradual population-level accumulation of adaptive traits over evolutionary time. Application of the MS to biological phenomena has spawned enormously productive and wide ranging research programs, which have contributed to communication between previously disparate strands of biology (structural, functional, evolutionary) \citep{Svensson2017}.  In short, the MS and associated methodological innovations and empirical findings have collectively transformed scientific knowledge of the evolutionary origins and developmental processes of biological life, including human life, and has been widely touted as the second most successful theory in the history of science after modern quantum physics in its ability to explain and predict the world in which we live \citep{Dunbar1996}.

The power of the MS to account for observable biological phenomena (phenotypic traits or behaviours) hinges on two interrelated assumptions: (1) the evolution of a biological trait is determined by the process of natural selection, and (2) the consequences of natural selection accumulate in the germ-line (genome) as programs for the development of subsequent phenotypes.  That is to say, if natural selection is the primary agent of shaping biological information (assumption 1), then it can be predicted that the programs for subsequent life contained in the germ-line (assumption 2) have been subject to gradual historical processes of selection.  As Mayr first formally suggested in 1961, these assumptions create conceptual space for unitary explanation of biological phenomena via a division of labour between two separate but complimentary scales---one evolutionary or ``ultimate'' level, and one developmental or ``proximate'' level \citep{Mayr1961}.  Assuming that natural selection is the sole (or at least primary) agent of evolutionary change, biologists interested in an evolutionary explanation can largely ignore proximate-level processes of development and immediate causation of that trait, and instead focus on the adaptive value of a trait and its phylogenetic history \citep{Mayr1961,Tinbergen1963}.  Proximate mechanisms and developmental processes are not passed on in the germ-line and are thus relatively inconsequential to macro-scale processes of evolutionary change (\citep{Dawkins1982,Grafen1991,Svensson2017} but see \citep{Laland2012,Laland2015}.  Meanwhile, biologists interested in an explanation of a trait based on its developmental processes and immediate mechanisms can proceed under the assumption that observable phenotypic variation emanates from an interaction between the adaptive potentialities and capacities contained in the germ-line and its external environment (its phenotypic niche) (SOURCE).  Importantly, proximate and ultimate levels of causation must be considered together as distinct but complimentary explanations of the biological phenomena\citep{Mayr1961,Tinbergen1963}.
%Something on Aristotle's four questions here...? Philosophical groundings in logic here?

In the case of the commonly cited example of a human infant crying (taken from \citep{Scott-Phillips2011,Nettle2009,Zeifman2001}), a proximate explanation of this behaviour would require an account of both the external (e.g., physical separation from the caregiver, lack of food, cold) and internal (e.g., activity of the limbic system to initiate crying or the role of endogenous opioids in the cessation of crying) factors responsible for the crying behaviour \citep[38]{Scott-Phillips2011}. An ultimate causal explanation for human infants crying includes both a description of the adaptive value of crying (e.g., crying elicits support and defence from mothers and other care-givers; infants that do not cry when in need of assistance are less likely to survive), as well as an account of its phylogenetic history \citep{Mayr1961,Tinbergen1963}.  Any explanation of biological phenomena requires an account of both proximate and ultimate levels of causation---simply knowing \textit{how} it is that an infant cries (proximate), or alternatively only knowing \textit{why} it is present in the phenotypic repertoire (ultimate) is not enough to satisfactorily account for the observed phenomena \citep[38]{Scott-Phillips2011}.



- earliest call from X in 1940 for Extended Evolutionary synthesis
1950s by C. H. Waddington,
- Gould 1980s punctuated equilibrium
- 1990s Anthropologists: Ingold / Dunbar

Most recent wave of revision comes from within developmental biology, calling for a broadening of the scope of evolution to include factors other than natural selection that are responsible for shaping the course of evolutionary trajectories:

- multilevel selection (disrupts the selection of gene frequencies?),
- transgenerational epigenetic inheritance,
- niche construction,
- cultural inheritance
- and evolvability.
- developmental bias

EXPLAIN: paragraph on dynamism and ``extensions,'' including relaxing natural selection as only way of determining biological information.
The MS has been applied to a an incredibly diverse range of biological phenomena, including human phenomena, as I discuss below.  Biology is an enormous discipline spanning time- and space-scales, and as such man empirical knowledge gaps exist.


Although many amendments to the MS have been made, and various theoretical extensions proposed, the core assumptions of the theory---namely genetic inheritance and natural selection---remain in tact at its foundation \citep{West2011}.

Reciprocal causation has been understood in evolution since the inception of the MS,  the problem is probably one of methodology rather than theory necessarily (Svensson2017).

Largely an empirical problem in each case --

It is important to keep this in mind when considering the application of the MS to human behavioural phenomena such as group exercise (\textit{this is a note to self as much as anything!}).


\subsection{Application of the Modern Synthesis to human behaviour}
The proximate/ultimate distinction in the MS has attracted large amounts of scrutiny from within various sub-disciplines of biology, and periodical calls are made for extension of the MS, or even a ``revolution'' in biological theory \citep{Svensson2017,Laland2017}.


 Applications of the MS to relatively straightforward examples such as human infants crying or avian migration (as Mayr chooses to do in his seminal paper ``On Cause and Effect in Biology'' in 1961) work relatively cleanly and without too much conceptual difficulty.  As I discuss in this section, when dealing with more complex behavioural phenomena, such as the socially complex and culturally rich evolutionary trajectory of humans, however, the empirical application of the MS becomes slightly more involved.

 The phenotypic niche of many organisms involve the interaction of complex spatial, ecological, and social spheres, which are made up of social partners, other species, and structural ecologies.  Effective strategies such as ``Hamilton's rule'' (rb > c) of genetic relatedness (1964) can been used to model inclusive fitness (direct + indirect fitness benefits) of behaviours that appear on the face of things to be individually costly (altruistic) at the level of the phenotype and therefore potentially maladaptive (SOURCE).

 - Sexual selection (Fischer)
 - the ecological dynamics have been modelled: difference between selective environment and ecological environment
 -

Initial applications of such strategies human social behaviour in the 1970s  (\citep{Wilson1975,Dawkins1976}) were met with extreme skepticism from fields beyond evolutionary biology, due to a lack of ontological and epistemological frameworks capable of bridging the MS paradigm with existing paradigms in the social sciences,

and became susceptible to misinterpretation, perhaps due to the radically counter-paradigmatic and therefore counterintuitive nature of claims (SOURCE).

While most of the sparks that flew following works such as Wilson's ``Sociobiology'' and Dawkin's ``The Selfish Gene'' were based on general confusion

 there was also a also legitimate concern over the ability of MS approaches to human behaviour to wrangle with the problem conventionally labelled as ``culture.'' It is clear that the human lineage entails species-distinct capacities for complex manipulation of extra-somatic materials (producing artefacts ranging from stone tools to quantum computing), information rich, malleable and scalable communication systems (producing language), and the and physiological, psychological, social affordances of shared cultural practices (producing phenomena such as religious activities).


Dawkins was the first to publicly recognise that population-level transmission and fixation of cultural variants (cultural practices, norms, languages, etc) resembled, in many ways, the equivalent processes of biological evolution, and should therefore be modelled as such \citep{Dawkins1976}.  Just as biological evolution is a system of information transmission, so too is cultural evolution a complex informational system.
The strictest versions of this approach proposes the ``meme'' as a gene-like unit of cultural information, which, like a gene, is subject to selection pressures of replication.  Memes that successfully replicate successfully populate\citep{Dawkins1976}.

Subsequently, theorists adopted game-theoretical population genetics models borrowed from evolutionary biology, to help demonstrate the selective pressures placed upon cultural information.  First articulated by Cavalli-Sforza and Feldman \textcite{Cavalli-Sforza1981} and then by Boyd and Richerson \textcite{Boyd1988}, the working approach to human cultural evolution begins with the assumption that cultural information, i.e., information capable of affecting individuals' behaviour that they acquire from other members of their species through teaching, imitation, and other forms of social transmission, evolves via mechanisms similar to those that act on genetic information.

Other theories of cultural evolution adjust population genetic models to take into account the observable differences between cultural and genetic information, such as culture's capacity to support one-to-many transmission, the blending of cultural variants, and non-randomly guided variation.  These adjustments are part of the concession that that cultural variants are not as dependent on high fidelity replication as their genetic cousins, but instead are shaped by evolved cognitive biases that favour the acquisition and transmission of some cultural variants over others due to their memorability or effectiveness \citep{Henrich2007}.

These start-up assumptions of models of cultural evolution directed attention towards the causal role of micro-evolutionary cognitive mechanisms of imitation, teaching, and memory, in enabling high fidelity copying (with occasional mutation-like errors) of cultural variants between individuals and throughout populations with distributions stable enough for selection to operate.  Models indicate that for social learning to actually enhance population fitness, it must be cumulative throughout generations, i.e., individuals must be able to socially learn what they could not learn on their own \citep{Boyd1995}.  Thus, particular attention has been paid to the mechanisms that could be responsible for facilitating species-unique \textit{cumulative} culture \citep{Tomasello2008}.

Evidence from comparative and developmental psychology indeed suggests a precocious and species-unique tendency to accurately imitate the actions of trusted or authoritative others (even when the goal of the action is unclear) sets the cognitive foundation for the transmission of cultural representations \citep{Tomasello2014a}.  The microevolutionary processes have also been enhanced by supplementing macroevolutiony processes, also known as cultural phylogenetics \citep{Mace1994}.  The phylogenetic comparative method seeks to understand long-term cultural change at or above the level of the society by 1) reconstructing the cultural evolutionary history of a particular trait or set of traits and 2) testing functional hypotheses concerning the spread or distribution of cultural variation across societies while controlling for evolutionary history.  The combination of these micro- and macroevolutionary approaches supports the theory that  ``dual-inheritance'' or ``co-evolution'' of genetic and cultural information in humans over time has led to the development of prosocial norms and institutions that facilitate collective adherence to shared cultural practices \citep{Richerson2008,Chudek2011}.





Application of MS to human behavioural phenomena therefore began with the realisation

Memes

The systemic under-theorisation of the social dimensions of human cognition is due in part to the youth of the scientific approach to human cognition and evolution. The idealisations and assumptions originally employed to kick start this science, most of which were borrowed form evolutionary biology and behavioural ecology, are necessarily limited in their capacity to model the precise details oh human social interaction.

The central theoretical challenge for cognitive and evolutionary explanations of human behaviour involves accounting for the mechanisms through which cultural practices transmit and fixate in populations.

The modern evolutionary synthesis and cognitive revolution created the scientific conditions in which a testable, Darwinian theory of cultural transmission became possible.



Cognitive and evolutionary approaches to culture and social cohesion have made productive empirical strides in identifying the proximate cognitive mechanisms most relevant to ultimate evolutionary explanations for the distribution of shared cultural practices around which human groups cohere, particularly in relation to large scale cultural forms such as religion \citep{Henrich2015,Purzycki2016b}.  In contrast to the informal, idiosyncratic, and subjective schemas of historical linguistics, archaeology, and social and cultural anthropology, these micro and macro theories of human cultural evolution are explicit in their assumptions, repeatable and extendable by others, and easily scaled up to large datasets \citep{Mesoudi2017}. The fact that many details of human social interaction still remain open scientific questions is a necessary part of the trade-off involved in building a scientific formulation of cultural evolutionary processes.


However, the interactive and affective mechanisms of shared cultural practices (i.e., the unmistakably ``visceral'' dimension referenced above) appear, phenomenologically at least, to be of distinct relevance to processes of cultural information transfer, and thus require careful and considered incorporation into a theory of cultural transmission.




Gene culture co-evolution

he human capacity for culture is by far the most notable technological achievement of the Homo genus since its branching from a shared chimpanzee ancestor at least 6 million years ago. Many anthropologists understand our cultural capacity as an adaptive solution to an evolutionary environment in which early humans were forced, perhaps as a result of ecological change, to “collaborate or die” (e.g., Tomasello et al. 2014, 188). This environment selected for more sophisticated forms of cooperation between individuals and within groups, particularly the ability to derive, represent, and verify useful information from others about the natural environment (e.g., how to avoid predation) and the social environment (e.g., how to coordinate with others) (Chudek and Henrich 2011).
Experimental studies in comparative psychology have shown that humans display a precocious tendency to diligently imitate the actions of trusted or authoritative others, even when the goal of the action is unclear (Tomasello et al. 2014). The human capacity for high-fidelity transmission of representations enables the proliferation and fixation of cultural practices horizontally within groups and vertically over generations. It is hypothesized that social cohesion emerges from this cumulative evolution of cultural practices, or “ratchet effect.” Combined with the related ability to infer and share the intentions of others, humans are able to simulate and commit to complex multiagent joint goals, including “everything from bi-directional linguistic conventions to social institutions with their publicly created joint goals and individual roles that can be filled by anyone” (Tomasello et al. 2014, 190).
<a>Social cohesion through cumulative culture and evolved norm psychology
<pf> Does a capacity for cumulative culture and an evolved norm psychology make groups more cohesive? Equipped with a unique capacity for cumulative culture, humans are able to establish and to engage in social norms and institutions. However, this capacity alone does not necessarily imply species-wide homogeneity of cultural practices, nor does it ensure that social norms will lead to social cohesion within groups. As Chudek and Henrich (2011) explain with the assistance of game-theoretical models, culturally competent humans with an evolved norm psychology could plausibly reproduce a multitude of self-reinforcing social norms and institutions according to the different ecological niches they inhabit. Some norms may facilitate group-level benefits, whereas others may have a detrimental impact. Nevertheless, processes of cultural group selection can lead to the persistence of those groups with norms and institutions that create group-beneficial social cohesion and cooperation (Chudek and Henrich 2011). Furthermore, intergroup competition is thought to have led to increasing levels of interdependence between cultural group members. This means that, throughout human evolution, individuals would have increasingly found themselves in cultural groups that enforced cooperation (and the punishment of non-cooperators), promoted ingroup favoritism, and facilitated stable and durable relationships between group members.
Evidence from cross-cultural surveys has shown that beliefs about punishment and reward in the afterlife and in a personal god (which result from religious norms and institutions) make individuals more likely to deem antisocial behaviors as unjustifiable and less likely to commit crimes. Further, ethnographic and experimental research has investigated how the presence of these “moralizing gods,” who know about and punish moral transgressions, facilitate large-scale, cooperative, and cohesive groups through encouraging norm-based prosocial behavior toward non-kin strangers (Purzycki et al. 2016). Cultural environments that favor the prosocial and punish the antisocial are thought to have influenced genetic evolution in humans, in a process referred to as gene–culture coevolution, such that humans have evolved a suite of cognitive mechanisms that presuppose and propagate prosocial and cooperative norms.
Human social cohesion in humans can thus be explained in terms of an ongoing evolutionary interaction between genes, culture, and cognition. This line of research has produced testable, though not uncontroversial, explanations for the emergence of religion and supernatural beliefs, warfare, costly ritual practices, and even more recently the transmission and spread of ritualized human sacrifice as cultural technologies that reliably establish and sustain levels of cooperation between individuals and within groups. Debate in this area of research centers on the specific mechanisms that influence the transmission and reproduction of cultural information. Some authors highlight the importance of the social context in which information is encountered, arguing that, for example, individuals are more likely to acquire cultural information when it is transmitted by prestigious individuals or by the majority of their social group (e.g., Chudek and Henrich 2011).





CAT

\subsubsection{Cultural Attraction}
There is evidence to suggest that, beyond microevolutionary mechanisms of transmission, other factors may also have an important causal impact on the accumulation and distribution of cultural variants. For example, demographic factors such as population size, structure, and interconnectedness have been shown to determine cultural complexity (variation) in hunter gatherer populations, with adaptive implications \citep{Henrich2004}.

It has also been suggested that variation in prosociality, social bonding, and social cohesion could have an important bearing on information transfer between individuals and within groups \citep{Heyes2011,Whitehouse2014,Wheatley2016}.

As researchers in comparative and social psychology have pointed out, humans do not merely aggregate, but rather actively congregate around shared cultural practices, seemingly driven by species-unique affective and motivational mechanisms\citep{Dunbar2010,Tomasello2005a}.

In addition, there is evidence of cross-cultural variation in the microevolutionary dynamics of cultural evolution, for example, with specifically higher social learning in collectivistic East Asian societies than in individualistic Western societies \citep{Mesoudi2015,DiYanni2015}.

These empirical details have formed the basis of models of multilevel selection, which propose selection pressures at the gene, individual, and group for society-level cultural variants such as religious ritual, warfare, and agricultural practices  \citep{Turchin2013,Atkinson2011a}.

In light of this collection of evidence, researchers have sought to broaden the scope of cultural evolution by relaxing the strictly selectional logic of memetic and dual-inheritance models, instead suggesting that cultural variants will tend locally towards certain ``attractor points'' depending on the diverse cognitive, demographic, ecological factors of attraction to which they are subjected \citep{Sperber1996}.  Rather than explaining patterns of cultural diversity, stability, and change in terms of the differential selection of certain cultural variants (e.g., content biases) or differential copying of certain individuals (e.g., success bias, prestige bias),  ``cultural attraction theory''(CAT) focuses on how cultural variants are systematically \textit{re-produced} by a combination of frequency dependent (i.e. conformity) and context sensitive (i.e. prestige) transmission biases, and the biophysical, psychological, historical, and ecological dynamics by which these biases are constrained and directed \citep{Claidiere2014}.  It has been pointed out that in contrast to genetic evolution, the mechanisms responsible for transmitting cultural information in humans (imitation, learning, and memory) cannot alone explain population level stabilisation of cultural variants, because they are not faithful enough to stabilise distributions of cultural variants on which selection can operate\cite{Claidiere2014}. CAT suggests instead that population level cultural variation is produced by processes that are partly preservative (i.e., occur via mechanisms of transmission), and partly re-constructive---the combination of which will result in cultural variants that tendentially converge upon particular types, called attractor points.

Broadening the theoretical scope of cultural evolution in this way enables a more detailed consideration of the microevolutionary processes of cultural information transfer beyond the traditional candidates for preservative transmission. In particular, the role of proximate cognitive, neurological, and social mechanisms of interpersonal interaction in facilitating social cohesion around shared cultural practices can be afforded much closer attention.  Thus, the highly visceral quality of shared cultural practices that contain elements of rigorous and coordinated group exercise could be a signal of important and hitherto unarticulated information concerning the proximate componential mechanisms, dynamical constraints, and ultimate evolutionary explanations for human uniqueness \citep[3]{Claidiere2014}.  What, then, is the role of group exercise in human sociality? Specifically, how does the joint movement associated with group exercise contexts uniquely generate social cohesion, and how do particular mechanisms vary by activity and culture?

Working within this specific research domain, in the following section I attempt to account for a for a theoretical link between proximate mechanisms of highly interactive and exertive collective movement common in competitive team sports such as rugby union, and the feeling of ``team click'' that appears to arise in athletes when the synchronisation of collective movement meets (or exceeds) athletes' prior expectations.

In turn, I seek to account for the process mechanisms that could explain how high quality joint action responsible for team click could by extension generate social bonding effects.  I review existing literature concerning the relationship between joint action and social bonding. In particular I review research concerning behavioural synchrony, a special case of joint action. The bonding effects of behavioural synchrony have been studied directly for over 10 years, and empirical evidence has confirmed and developed hypothesised relationships between coordinated physical movement and prosocial behaviour, perceived social bonding, positive affect, and modulation of feelings of exertion and fatigue.




Applications to human behaviour:
    4.the puzzle of culture and individually costly social behaviours (social cohesion article)
    5.inclusive fitness Hamilton
    6.meme theory
    7.gene-culture



  How is it possible to understand the evolution of social behaviours, especially seemingly costly or altruistic behaviours that appear to convey no or negative fitness benefits to the individual organism? Within gene-centred view of the MS framework, Hamilton (1964) demonstrated that social behaviours that appear seemingly costly to the individual phenotype could be \textit{indirectly} beneficial in the case that such behaviours increased the reproductive success of other individuals carrying the same gene (expressed famously in Hamilton’s rule of rb > c).  As such, it was shown that the evolutionary value of a behaviour or trait could be calculated as the sum of direct and indirect fitness benefits to an organism over its lifespan \citep{Grafen2006}.  This core concept of the MS, known as \textit{inclusive fitness} has enabled a thorough consideration of the evolutionary origins of human social behaviour since it was first applied to human populations (rather controversially) by biologists such as Dawkins and Wilson.
























On the one hand, given that natural selection is the sole determinant of evolutionary change, biologists interested in explaining the ultimate evolutionary causation of a trait (i.e., the ``why?'')
 has been selected for, and is therefore in some ultimate, historical sense adaptive.
when attempting to sufficiently explain biological life, these two assumptions allow for a crucial division of labour.
These assumptions allow for a division of labour in the explanation

By treating the gene as a program containing source code for biological life, predictions of the expression of traits in a population can be made on the basis of mathematical probabilities of gene combinations \citep{Dawkins1972;Dawkins1982}.

the  The MS predicts that, on the one hand, natural selection is the fundamental shaping (direction-giving) cause of evolutionary change in biological life, and that variation in phenotypic traits is a historical product of (as well as drift, flow, and random mutation in the gene frequencies (SOURCE)).  On the other hand, MS assumes the development of biological life.

As Ernst Mayr (1961:1504), a prominent advocate of the MS, explains in his seminal paper, ``On Cause and Effect in Biology'':

``The completely individualistic and yet also species-specific DNA code ...is the program for the behavior computer of this individual. Natural selection does its best to favor the production of codes guaranteeing behavior that increases fitness.''

B


The MS was first proposed by Huxley \citep{Huxley1942} following mathematical formalisations performed by population geneticists (Fisher1930, Haldane1932, Wright1931) that confirmed the compatibility of the two theories.  The MS became even more widely accepted after events such as the verification of the structure of the DNA molecule by Wallace and Crick in 1954 (SOURCE), and the







%As I will discuss in more detail below, MS has also been applied to a human evolutionary trajectory defined by advanced complex cognitive capacities, cultural affordances, and social organisation.

When attempting to explain biological behaviour as complex as human group exercise,

 that the genome contains in-built instructions for biological life which are either expressed or suppressed based on the

A phenotypic trait can be explained, on the one hand, in terms of the developmental processes and immediate mechanisms that trigger the observable occurrence of the trait---essentially, ``how does it function?'' \citep{Mayr1961,Tinbergen1963}  The viability of this ``proximate'' causal explanation rests on the assumption of separation between the germ-line and the soma (genotype and phenotype). The development of a phenotypic trait begins initially with biological instructions encoded the human genome, which subsequently interact with the environment to produce the phenotypic trait.   (WHAT WOULD BE THE ALTERNATIVE??)

On the other hand, a phenotypic trait can also be explained in terms of its evolutionary history, or ``why does it function?''  The answer to this question is known in evolutionary biology as an ``ultimate'' causal explanation, and includes both a description of the adaptive value of the trait, as well as an account of its phylogenetic history \citep{Mayr1961,Tinbergen1963}. An ultimate causal explanation for human infants crying is that it elicits support and defence from mothers and other care-givers.  Infants that do not cry when in need of assistance are less likely to survive \citep[38]{Scott-Phillips2011}. An ultimate explanation would also include a consideration of the ancestral lineage of the phenotypic trait (its phylogeny), as well as the selective pressures of past environments that could have contributed to the transmission and accumulation of allele frequencies in the genome that determine a capacity for the trait.  This level of explanation is enabled by the second assumption of the MS, namely that natural selection at the level of the phenotype is the primary determinant of population-level transmission and fixation of biological variants.  In fact, in some c




paragraphs:
Part 1: existing cognitive and evolutionary theories of human behaviour
The modern synthesis
    1. The moment of synthesis
    2. neo-Darwinism: gene --> selection
    3. Proximate and ultimate distinctions, Phenotypic gambit of evo biology?




Problems?
Under-theorisation of the social dynamics of interaction

    8.cultural group selection

    9.CAT: relax selection, model as a complex system.

What are the problems here?
  The EES suggestion: consideration of ontogeny in processes of natural selection, reciprocal causation is suggested

  but then it appears that MS has always recognised complexity to some degree?
   (gene frequency selection, sexual selection, ecological impact)
   Mayr 1963 versus oft-quoted 1961


  CAT problem is also a Only ever measuring gene or meme (unit of transmission), not actually quantifying the dynamics?

Emerging ``Systems'' science of neurology, cognition, behaviour, social interaction
Free energy principle


4.

%The theoretical refinement of evolutionary theory has been aided by an array of methodological innovations and technologies, and has generated enormous and wide ranging research programs. The result has been a revolution in scientific knowledge of the evolutionary origins and developmental processes of biological life.  Although many amendments and extensions have been made to the MS, the core assumptions of the theory---namely genetic inheritance and natural selection---remain as its foundation.

%1.1 The synthesis history 1.2 neo-Darwinian gene - selection definition

The partitioning between prox and ultimate roles was a division of labour that served to unify disparate strands of biology

processes both internal and external to the organism that enable a certain behaviour to be observed in the environment.

In the case of an infant crying, this includes a description of both external and internal triggers to the behaviour (physical separation from the caregiver, lack of food, cold)

Before proceeding to a focus on group exercise, it might be easier to start with a more intuitive and commonly referenced example of human infants crying

that occur between the germline and the soma (or the genotype and its environment).

The MS has convincingly shown that the information determining the capacities and potentialities of specific life present in the germline (the genome)



The empirical verification of


over the course of the 2nd half of the 20th century facilited a more formal division of labour within biology between functional/structural and evolutionary strands.




The MS proposes that assuming the two core levers responsible for directing evolutionary change---namely natural selection and genetic inheritance---makes it possible to separate the task of explaining behaviour into two complimentary questions.


and that this information interacts with external factors in the environment to produce a


exist as statistical allele frequencies at the level of the genome


the evolutionary causes of a phenotypic trait (``how did this behaviour evolve?'') can be inferred by describing the history of that trait's adaptiveness in an environment.

The immediate mechanical influences that cause the phenotypic trait are 1) natural selection and 2) inheritance of biological source code in genes, it is not crucial for an evolutionary explanation of behaviour to account for the details of how certain genetically encoded capacities and potentialities came to be expressed in the environment

 that 1) selection pressures operate at the level of the phenotype, while 2) the consequences of selection, i.e., the transmission (inheritance) of biological instructions for capacities and potentialities, occur at the level of the genotype,

The theory of natural selection proposes that the phenotype will represent a history of evolved traits fit to the environment

Within the MS paradigm, it is possible when answering this question without bothering too much with developmental processes or mechanisms enabled the phenotypic



1.4  EES?
This gives rise to the problem of the system, and the possibility that evolution is a system niche


2. Application of MS to GE
%GE and MS intro
As I will explain in more detail below, MS accounts of group exercise have been attempted at two levels of analysis.  On the level of proximate physiological, neurolbiological, cognitive, and social mechanisms of coordinated and exertive group exercise, there is a prevailing theory that such activities serve to generate positive psychological states associated with well-being, euphoria, and X. On the level of evolutionary explanation of the persistent prevalence of group exercise in human sociality, it has been suggested that shared cultural practices in which group exercise often features prominently (particularly ritualised practices associated with religiosity), have been selected for  due to their ability to facilitate social cohesion.

Proximate mechanisms:

Ultimate is harder:
costly signaling,
cultural group selection
    Dunbar2008
    Whitehouse2004
But CAT ?
Best option is to focus in on the proximate mechanisms...



Extended evolutionary synthesis,
which suggests that evolution should be modelled as a complex system in which evolutionary causation is perceived to cycle through biological systems recursively, thus blurring the line between proximate and ultimate levels of causation and relaxing the strictness of selection in the landscape of evolutionary models to include other factors relevant to population level distribution of biological variants.

Functional synergies: the best work against chaose.

Game theory supports:
* kin selection
* mutualism
* costly signaling


Gene culture

(known as Hamilton’s rule, which states that a behaviour or trait will be favoured by evolution when rb > c, where c is the cost to the actor, b is the fitness benefit to the recipient, and r is the coefficient of relatedness—a statistical concept describing the genetic similarity between two individuals relative to the average similarity of all individuals in the population (Grafen, 1985; Hamilton, 1970)).

involved a unification of the theory of evolution by natural selection, which first fixated in the second half of the 19th century and became attributed to Charles Darwin, with the theory of genetic inheritance, which emerged gradually in the 20th century, spurred by mathematical contributions by population genetics such as Ronald Fisher, and culminating in the confirmation of the structure of the DNA molecule by Wallace and Crick in 1954.  MS provides a ``unified'' theory of evolution, explaining both how an organism has adapted to its environment, and how adaptive traits are passed on to subsequent generations




T











\section{Exertive and Coordinated Group Exercise}
Physical activity, exercise, and sport have well-known positive effects on psychological and physical health (Ekkekakis, 2003; Fiuza-Luces, Garatachea, Berger, \& Lucia, 2013).
Social scientists have also long speculated about the benefits of energetic group activities such as ritual, music, and dance for social cohesion (Durkheim, 1965).  Group exercise contexts typically require the coordination of both movement and intentions (Reddish, Fischer, \& Bulbulia, 2013), which together activate neurobiological mechanisms implicated in social reward (Dunbar, 2010; Eisenberger, 2012), as well as those involved in enduring the pain and discomfort of physiological exertion, i.e., the ``runner’s high'' (Boecker et al., 2008; Dietrich \& McDaniel, 2004; Sullivan, Rickers, \& Gammage, 2014).


It is also apparent that group exercise contexts offers to its participants and observers an opportunity for profound meaning.  Many people do not engage in exercise just for health or enjoyment; rather, in some contexts it forms part of a life of meaning, purpose, and self-discovery (see for example Jackson, 1995; Jones, 2004; White \& Murphy, 2011). Modern sport has always been much more than ``just a game'', and instead offers an arena in which virtues and vices are learned, and the ``morality plays''—--of community, national, and globe—--thus performed (Elias \& Dunning, 1986; McNamee, 2008).  Ethnographic perspectives on group exercise contexts emphasise the ethical and moral dimensions of athletes’ experiences, and contextualise these experiences within political processes relating to the construction of the self, community, and nation-state (Alter, 1993; Brownell, 1995; Downey, 2005b; Wacquant, 2004).  Athletes who engage in extremely physiologically costly exercise, particularly at the elite apex, often report the autotelic experience of ``flow''—--described as full immersion in the ``here and now,'' effortlessness, or optimal experience (Csikszentmihalyi, 1992; Dietrich, 2004).  At the level of the group, ``team click'' and ``group flow'' are highly elusive possibilities, coveted by athletes, coaches, and fans alike (Novak, 1993; Sawyer, 2006).  The complexity of motivation for exercise presented both ethnographically and anecdotally presents an opportunity for further research into the cognitive, evolutionary, and social mechanisms that produce participation and adherence to exercise, particularly in contexts in which exercise is extremely costly.


Exercise is healthy
The health benefits associated with regular exercise, including reduced risk of cardiovascular disease, autonomic dysfunction, and early mortality, are becoming increasingly well-known (Blair \& Powell, 1994; Nagamatsu et al., 2014).  It is not clear exactly how much exercise (and of what type and intensity) is optimal for health and well-being.  Physical activity guidelines released by the U.S. Department of Health and Human Services in 2008 recommended a minimum of 150 minutes per week of moderate-to-vigorous aerobic exercise for substantial health benefit (U.S. Department of Health and Human Services, 2008).  However, a recent meta-analysis suggests an exercise benefit threshold at approximately three to five times this recommended minimum, and no excess risk of physical harm or ill-health at 10 or more times the minimum (i.e., >25 hours of exercise a week) (Arem et al., 2015).
This comparatively high level of recommended exercise can perhaps be understood in light of our evolutionary past.  Despite recent selection pressures (Voight, Kudaravalli, Wen, & Pritchard, 2006), human genetic makeup is largely shaped to support the physical activity patterns of hunter-gatherer societies living in the Paleolithic era for which food and fluid procurement was obligatorily linked to exercise (Konner & Eaton, 2010).  The estimated energy expenditure of hunter-gatherers during exercise (1000–1500 kcal per day) is equivalent to three to four hours of moderate-to-vigorous exercise, or 21-28 hours per week (Fiuza-Luces et al., 2013).  Technological improvements over ~350 generations (agricultural followed by industrial and, most recently, digital revolution) have led to dramatic reductions in human physical exercise levels (Balish, Eys, & Schulte-Hostedde, 2013).  An estimated one third of adults worldwide are chronically inactive, by standards much less stringent than these figures cited above, and this endemic inactivity trend starts in early life (Dishman, 2001).   While the precise details of exercise in our evolutionary past remain unclear, it is likely that selection favored human movement at intensities and durations far greater than those observed in industrialised and postindustrialised societies, with a few notable exceptions such as individuals who exercise professionally, or whose profession involves physical activity (rigorous manual labour).  The implications of this global trend are becoming clear, with inactivity being linked to the aggressive emergence of endemic lifestyle diseases such as obesity, type II diabetes, arteriosclerosis, and high blood pressure (Penedo & Dahn, 2005).

2.1.2	Exercise induces positive affect
Exercise is now also firmly established as an important factor in the development and maintenance of healthy cognitive function. Exercise (of particular types and intensities) has been shown to increase neurogenesis (Fabel et al., 2003), improve cognitive ability (Draper, McMorris, & Parker, 2010), and reduce the risk of neurodegenerative diseases such as Alzheimer’s and dementia (Rockwood & Middleton, 2007; Rovio et al., 2005).  On the level of individual psychology, leisure exercise has also been linked to self-reported “positively activated affect” (PAA) measure in a ~30 minute window directly following exercise, and is most common in low-moderate intensity (15–39\% oxygen uptake reserve (\%VO2R) and duration (~35 minutes) (Reed & Ones, 2006).  Ekkekakis suggests that exercise-induced positive affect results from a combination of top-down “cognitive factors” (such as a perception of physical self-efficacy), and implicit interoceptive (muscular and respiratory) cues that reach the affective centres of the brain via subcortical routes (Ekkekakis, 2003).  Dietrich corroborates this argument, proposing the “reticular-activating hypofrontality” (RAH) model of acute exercise and neurocognitive function (Dietrich & Audiffren, 2011).  According to the RAH model, as the acute exercise event persists (e.g., a marathon run just below the aerobic threshold), the brain is forced, as part of an energy tradeoff, to down-regulate higher executive regions that are less essential to sustaining exercise.  According to Dietrich, this hierarchical down-regulation of the frontal areas of the brain (hypofrontality) underpins the oft-reported subjective experiences associated with exercise, including reduction in anxiety, feelings of positive affect, euphoria, and self-transcendence—popularly known as the “runner’s high” (Dietrich & McDaniel, 2004).
2.1.3	The Runner’s High: neurobiology of exercise
Common to all goal-oriented (human) behaviors that impose risks or high energy costs are neurobiological reward mechanisms, which are thought to condition these fitness-enhancing activities (Burgdorf & Panksepp, 2006).  Neurobiological rewards in exercise are associated with both central effects (improved affect, sense of well-being, anxiety reduction, post-exercise calm) and peripheral effects (analgesia), and appear to be dependent for their activation on exercise type, intensity, and duration (Dietrich & McDaniel, 2004).  Sustained aerobic exercise at a moderate intensity (~70-85\% of maximum heart rate) – but not low (~45\%) or high (~90\%) intensities – induces activity in the endocannabinoid (eCB) system (Raichlen et al., 2013), and similar results have been obtained in studies on the opioidergic system (Boecker et al., 2008).  Endocannabinoids appear to play an influential role in exercise-specific neurobiological reward, with studies showing activation eCB activation in moderately intense exercise and in cursorial mammals, such as humans and dogs (but not non-cursorial mammals, e.g., ferrets; Raichlen et al., 2012).  In addition to direct peripheral (analgesic) and central (psychological well-being and alteration) effects, eCBs are also responsible for activating “traditional” neurotransmitters (opioids, dopamine, and serotonin) also responsible for rewarding and reinforcing behaviour (Sparling, Giuffrida, Piomelli, Rosskopf, & Dietrich, 2003).  These findings lead Raichlen to propose eCBs as a key neurobiological substrate responsible for motivating habitual engagement in aerobic exercise, by generating “appetitive” and hedonic associations with exercise behaviour (Raichlen et al., 2012).  This neurobiological evidence maps on to more extensive literature concerning the psychological effects of exercise, which indicates a duration and intensity “sweet spot” for exercise and PAA, whereby moderate intensity exercise for durations of ~45 minutes appears most optimal.

2.1.4	Group exercise and the Social High
In addition to motivating adherence to regular aerobic exercise, it is possible that the function of exercise-induced PAA extends to the realm of social bonding, particularly when achieved in group exercise contexts (Cohen et al., 2009; Machin & Dunbar, 2011).  Endocannabinoids and opioids have been implicated in mammalian social bonding (Fattore, Melis, Fadda, Pistis, & Fratta, 2010; Keverne, Martensz, & Tuite, 1989), and in humans specifically, there is evidence that endorphins (a particular class of endogenous opioids) mediate social bonding (Dunbar, 2012; Shultz & Dunbar, 2010).  Meanwhile, behavioural synchrony—a common component of group exercise—similarly affects social bonding and pain modulation.  A growing number of experimental studies have shown that, relative to non-synchronous group activities, behaviorally synchronous movement increases social bonding and pro-social behavior – an evolutionarily important outcome of bonded relationships (Reddish, Bulbulia, & Fischer, 2013; Reddish, Fischer, et al., 2013; Wiltermuth & Heath, 2009).  Recent studies have also found that, compared to solo and non-synchronous group exercise, synchronous group exercise leads to significantly greater post-workout pain threshold (Cohen et al., 2009; Sullivan et al., 2014; Sullivan & Rickers, 2013a, 2013b).  Recent research on pain processing and perception has shown that cues of social support reduce perceptions of experimentally induced pain, reduce activity in neural circuitry related to pain affect, and increase activity in neural regions known to respond to cues of safety and to modulate the threat response through top-down processes (Eisenberger, 2012; Eisenberger et al., 2011).  In the context of physically challenging group exercise, behavioral synchrony can act as a cue of social support by increasing perceptions of togetherness and cohesion (Miles, Nind, & Macrae, 2009), thus activating social support based analgesic mechanisms in a similar manner, enabling increased endurance and enhanced performance (Davis et al., 2015; Noakes, 2012).










DUNBAR: BONDING HYPOTHESIS
a ``social high'' theory of exercise and social bonding, whereby exercise-induced positive affect mediates a reciprocal relationship between pro-social bonding and physiological performance (Cohen, Ejsmond-Frey, Knight, & Dunbar, 2009; Davis, Taylor, & Cohen, 2015).

Recent research on the affective dimensions of human social cohesion has devoted considerable attention to the way in which exertive, coordinated group activities activate psychophysiological mechanisms associated with forging and maintaining social bonds with fellow group members.

The capacity for social bonding is thought to have arisen in primates as an adaptive response to the pressures of group living. Aggregating in groups serves to reduce threat from predation but at the same time can be individually costly as a result of stress arising from interaction at close proximity and conflict over resources between genetically unrelated individuals. These pressures are hypothesized to have led to selection for social bonding (e.g., via dyadic grooming). The coalitional alliances that arise between close partners allow for the maintenance of the group by buffering the stresses of group living (Dunbar 2012). Primate social grooming is associated with the release of endorphins (a type of endogenous opioid), presumably leading to sustained rewarding and relaxing effects. While other neurotransmitters such as dopamine, oxytocin, and/or vasopressin may also be important in facilitating social interaction, it has been suggested that endorphins allow individuals who are not related or mating to interact with each other long enough to build “cognitive relationships of trust and obligation” (Dunbar 2012, 1839).

As the Homo genus evolved more complex collaborative capacities for survival in interdependent group contexts, grooming-like behavioral technologies sustained social bonding in larger group sizes where dyadic grooming would take too much time (Dunbar 2012). Experimental studies suggest that neurophysiological mechanisms activated by activities that involve physical exertion and coordinated movement such as group laughter, dance and music making, exercise, and group ritual can bring groups closer together, mediated by the psychological effects of endogenous opioid and endocannabinoid release. Endogenous opioids and endocannabinoids are neurotransmitters with analgesic, mood-elevating effects that have been implicated in mammalian social bonding. When these mood-elevating effects are experienced in a group they seem to lead to what Durkheim ([1915] 1965) described as collective effervescence, and, through embodying each other’s affective experiences, to result in more positive, trusting, and cooperative relationships between participants (Dunbar 2012).
These studies give an empirical basis to anthropologists’ observations of the relationships between coordinated, exertive group rituals, collective effervescence, and social cohesion. Across cultures, anthropologists have described rituals involving group dance and music making as practices that lead to the sharing of positive mental states and feelings of “oneness” with the group. Others have discussed the importance of joint movement during sport and exercise in facilitating collective group action and prosocial behavior (e.g., Cohen 2017).

Unifying theories WHITEHOUSE
<pf>Recently, anthropologists have attempted to integrate analyses of ethnographic, archaeological, and phylogenetic information in order to develop broader theories of social cohesion. Drawing initially from ethnographic observations of ritual practice in the Papua New Guinean Highlands, Harvey Whitehouse developed a general theory of human social cohesion based on two divergent modes of ritual practice and their associated psychological and sociopolitical effects (Whitehouse and Lanman 2014). High-frequency, low-arousal religious rituals (weekly attendance at church sermon, praying, etc.) are associated with identification with the prototypical features of the group (“group identification”) whereas low-frequency, high-arousal rituals (tribal initiation rituals, dysphoric experiences such as frontline combat) generate “identity fusion”—a visceral feeling of oneness with the group.
Whereas group identification can be understood as a psychological adaptation deriving from a norm coalitional cooperative mechanisms, identity fusion arises from the generalization of kin-detection mechanisms, whereby individuals recognize others with whom they have shared core self-defining experiences as “fictive kin.” Whitehouse and colleagues (2014) argue that these two distinct psychological states, and the ritual practices that reliably generate such states, represent “attractor positions” in the cultural evolution of religion and human sociality. In a return to Durkheim’s original outlay for the study of social cohesion, the modes theory incorporates the two key underpinning mechanisms of social cohesion (i.e., cumulative culture and its interaction with human capacities for social bonding) by accounting for variance both in the modes of shared cultural practices and in the emotional quality of group-level commitment.
In his study of religion, prosociality, and extreme altruistic behavior, Scott Atran similarly insists on the need to carefully consider the interaction between cultural, cognitive, and affective mechanisms, in particular the role of communal rituals in “rhythmically coordinat[ing] emotional validation of, and commitment to, moral truths” (Atran and Norenzayan 2004, 714). Atran suggests that extreme altruism can be explained only by considering the codependent relationship between the affective motivational processes (of arousal, pain, and reward) and the cultural representations (“sacred values” of the group) with which these psychophysiological mechanisms interact and become associated.
Relatedly, recent empirical studies of extreme ritual practice present evidence of a positive relationship between pain experienced during high ordeal rituals (e.g., walking on hot coals) and subsequent expressions of parochial prosociality. Interestingly, in contrast to the prevailing adherence to multilevel selection and cultural-group selection theories of social cohesion, these researchers speculate that the selective advantages available in extreme group ritual are afforded primarily to the individual (e.g., moral cleansing, improvements in social standing), with group-level benefits arising only secondarily (Fischer and Xygalatas 2014). In so doing, Fischer and Xygalatas, for example, centralize the explanatory role of proximate psychological mechanisms (of arousal, pain, and reward) in establishing and maintaining social cohesion.


Costly Signaling:
Costly behaviors in ritual contexts have been shown to fit game-theoretical models of cooperation, acting as “hard-to-fake” signals of in-group commitment, and helping to solve collective action problems in which extra-kin social groups become susceptible to free-riding and loafing (Atran & Henrich, 2010; Henrich, 2009; Sosis, 2004, 2006).

* Integrative approaches:
    * Dunbar
    * Whitehouse
    * But then Sperber (CAT)



4. What about an EES approach to GE, in order to model the vitality of the activity?
5. Possible reconciliation between MS & EES?
6. Target Phenomenon: team click
7. The Present study
    1. Beijing Men’s rugby team
        1. Rugby as joint action context
        2. Cultural variation: China & group membership
    2. Methods
        1. Ethnography
        2. Survey
        3. Controlled experiment
    3. Results
    4. Discuss relevance
8. Contributions of thesis
    1. Novel theoretical framing
    2. Mixed methods (first ethnographic application of these theories?)
    3. Cross-cultural dimension
    4. Novel evidence linking joint action and social bonding in a real world context
    5.


Modern Synthesis approaches to group exercise:



\subsection{Cultural Evolution} %1490




\subsection{Team Click}
Recent research has made a ceremony of invoking one particular passage from Durkheim (1965, pg. 217) to capture the ``collective effervescence'' of exertive and coordinated group activity found in arenas as diverse as music making, dance, military drills, and sport:  ``Once the individuals are gathered together, a sort of electricity is generated from their closeness and quickly launches them to an extraordinary height of exaltation'' \citep{McNeill1995,Konvalinka2011,Fischer2014,Mogan2017}. Indeed, this passage powerfully captures the role of collective activity in generating positive emotional states and joint arousal, and lends itself nicely to the hypothesis that this visceral ``electricity'' is attributable in part to neuropharmacologically-mediated affective mechanisms associated with pain and reward\citep{Dunbar2008,Cohen2009,Fischer2014,Launay2016}.   \textit{EXPLAIN MORE RE grooming hypoethesis here  SYNCHRONY, EXERTION = BONDING}

Missing from Durkheim's passage, however, is an aspect of group activity that is heavily scrutinised in
technically demanding joint action scenarios such as competitive interactional team sports or, music-making and dance: the \textit{quality} of movement synchronisation in joint action.  Activities such as music-making, dance, and sport depend upon highly complex coordination of behaviours between individuals, in which the movements of one individual must align in time and space with the movements of another.  Highly skilled practitioners who develop a fine-grained sensitivity concerning the perceived outcome of joint action, often the ecstasy of group activity is contingent not just on participation, but on the extent to which joint action with co-participants ``clicks.''   The psychological literature of optimal human experience (also known as ``Flow'' \citep{Csikszentmihalyi1992}) offers extensive documentation of the positive psychological and social effects of technically complex movement in individual and, to a lesser extent, joint action. To date, however, very little research has dealt directly with the relationship between perceptions of \textit{quality} joint action and processes of social bonding and group formation \citep[but see][]{Marsh2009}.
This dissertation locates the phenomenology of perceived ``click'' of joint action

is grounded in the current state of the art of social cognition, and centred aroun
an ethnographically verified phenomenology of skill acquisition and movement
presents a theory of social bonding through joint action that serves to
 looks beyond the generalised neuropharmacologically-mediated electricity of  ``collective effervescence'' and into the social cognition of movement
evidence suggesting that an explanation of social bonding through coordinated group activity must involve more than just



One of the big mysteries of competitive team sport, particularly at the elite level, is the elusiveness of peak team performance.  While each individual athlete may exhibit expert-level competence in sport-specific skills, the much sought-after aggregation of these components, i.e. a team that consistently performs ``in the zone,'' and ``firing on all cylinders,'' in reality often proves frustratingly difficult to achieve and sustain.  As King and De Rond \textcite[568]{King2011} note in their ethnography of the 2008 Cambridge University rowing crew who participated (and who were eventually victorious) in the famous annual Boat Race against Oxford University, the search for collective rhythm is a universal in human social interaction, but  the physiological and psychological complexity of finding that rhythm ``...is extremely difficult to attain; collective performance is a possibility not a certainty.''   The moment in which everything ``clicks'' into place in team sport can, for various reasons, disappear as abruptly as it arrives, if indeed it arrives at all.

But, when team click is somehow cultivated, and even sustained, it is celebrated as the ultimate, albeit often inexplicable magic of sporting feats. Consider Leicester City Football Club's unbelievable outhouse-to-penthouse title run in the 2015 English Premier League, the recent dominance of the Golden State Warriors in the American National Basketball League, or the astonishingly consistent performance of the New Zealand men's national rugby union team.  The ``All Blacks'' are arguably the most successful sporting team ever, with a winning percentage of 77\% in the last 150 years (88\% in the last 6 years).  All of these successful teams carry with them a powerful ``aura'' associated with their capacity to effectively coordinate their behaviours on the field over extended time scales: individual games, seasons, and, in the case of the All Blacks, entire generations.  Although it is tempting to be seduced by the aura of such rare instances of collective performance, this dissertation attempts the (admittedly) more banal task of moving from mystery to scientific mechanism, in order to explain these collective phenomena in terms of their social, historical, physiological, and psychological components and dynamics.

In this dissertation, I use the term ``team click'' to describe the phenomenology of peak performance within a team of athletes engaged in joint action.  For athletes, coaches, and spectators alike, team click can be a hugely powerful sensation. As theologian Michael Novak explains, ``[f]or those who have participated on a team that has known the click of communality, the experience is unforgettable, like that of having attained, for a while at least, a higher level of existence'' \citep[11]{White2011}. As has been extensively documented in the psychological literature of ``flow'' \citep{Csikszentmihalyi1992} and optimal human performance in sport \citep{Jackson1999}, athletes engaged in team coordination often report total absorption in and complete focus on the task at hand, a transformation of the experience of time (either speeding up or slowing down), and a blurring or transcendence of individual agency, or a ``loss of self''   \citep{Csikszentmihalyi1992,Jackson1995,Jackson1999,McNeill1995}.  Research suggests that flow often occurs in scenarios in which there are clear goals inherent in the activity, as well as unambiguous feedback concerning extent to which goals are either being achieved or not.
In addition, scenarios most conducive to the experience of flow are those in which the technical requirements are challenging but achievable if practitioners are able to extend slightly beyond their normal capabilities\citep{Fong2015}.
The coalescence of these factors is intrinsically rewarding and autotelic\citep{Csikszentmihalyi1975}, activating both ``hedonic'' and ``eudaimonic'' dimensions of subjective well-being \citep{Huta2010,Fave2009}.  The vast majority of flow research has focussed on the experience of the individual athlete, musician, or performer.  However, more recent attempts have been made to extended an analysis of flow and its antecedents to the level of the group and dynamics of interpersonal coordination---a phenomenon termed ``group flow'' \citep{Sawyer2006}. Indeed, as the phenomenon of team click suggests, individual experiences of flow are almost always embedded in and contingent upon cognitive processes and contextual dynamics involving co-actors and the physical environment, even if the existing literature preferences an individual-centred account of the phenomenon\citep{Kirsh2006,Marsh2009,Noy2015}.

Importantly, team click appears to have important flow-on consequences relevant to social bonding and affiliation. Tightly synchronised activity in particular, found in team sports such as rowing, can help dissolve the boundaries between individual and social agency: ``In rowing...it feels like you have at your command the power of everybody else in the boat. You are exponentially magnified. What was a strain before becomes easier. It is absolutely the ultimate team sport'' \citep{Brown2016}.
The blurring of agency between self and team may be responsible for facilitating affiliation and trust between teammates in competitive athletic environments such as professional rugby, which often involves high physiological stress and uncertainty: ``...you always wanted a guy who would go into the trenches with you and he always played consistently well...he could really play and was just one of the good lads that you enjoyed his company'' \citep{Fox-Sports2017}. In this sense, the experience of team click may act as a social diagnostic tool, a powerful signal of commitment to joint action and willingness to cooperate \citep{Reddish2013a}.

The experience of flow has by now been extensively studied by psychologists and neuroscientists, from which a series of neuropharmacological \citep{Boecker2008}, neurocognitive \citep{Dietrich2006,Dietrich2011,Labelle2013}, and psychological \citep{Csikszentmihalyi1992} theories for its emergence have been tabled.  However, throughout this process, the social dimensions of optimal human experience have been less scrutinised, despite strong anecdotal and observational evidence of phenomena such as group flow, team click, and social bonding emanating from these collective states. In the sections that follow, I draw upon related strands from cognitive, neuroscientific, and psychological---including social psychological---literatures in order to develop a novel theoretical account of the relationship between coordinated interpersonal joint action and social bonding, and the mediating role of ``team click.''


% The extent to which synchronised joint action is responsible for generating social bonding may depend crucially on the accordance of action with culturally directed expectations.


Physical activity, exercise, and sport have well-known positive effects on physical and  psychological health (Ekkekakis, 2003; Fiuza-Luces, Garatachea, Berger,  Lucia, 2013).
The health benefits associated with regular exercise, including reduced risk of cardiovascular disease, autonomic dysfunction, and early mortality, are becoming increasingly well-known (Blair  Powell, 1994; Nagamatsu et al., 2014).

While the physiological, psychological, and social processes that combine in instances of exerted, coordinated movement are rich and varied, many strands of research suggest a link between group exercise and social bonding \citep{Davis2015,Cohen2017}. It is now understood that strenuous and prolonged physical exercise is modulated by the same neuropharmacological systems (namely, the opioidergic and endocannabinoid systems) responsible for regulating pain, fatigue, and reward \citep{Boecker2008,Raichlen2013}.
Exercise-specific activity of these systems offers a plausible neurobiological explanation for commonly reported sensations of positive affect, anxiety reduction, and improved subjective well-being during and following exercise---extremes of which asre popularly referred to as the ``runner's high'' (Dietrich  McDaniel, 2004; Boecker et al. 2008; Raichlen, Foster, Gerdeman, Seillier,  Giuffrida, 2012).  This neuropharmacological account of group exercise and social bonding has its roots in studies of social grooming in non-human primates.  Dunbar and colleagues propose a neuropharmacologically mediated affective mechanism linking dyadic grooming practices with group-size maintenance \citep{Machin2011}.

The capacity for social bonding is thought to have arisen in primates as an adaptive response to the pressures of group living.  Aggregating in groups serves to reduce threat from predation.  At the same time, it can be individually costly due to stress arising from interaction at close proximity and conflict over resources among genetically unrelated individuals.  These pressures are hypothesised to have led to selection for social bonding (e.g., via dyadic grooming). Resulting coalitional alliances among close partners allow for the maintenance of the group by buffering the stresses of group living.  Primate social grooming, for example, is associated with the release of endorphins, presumably leading to sustained rewarding and relaxing effects.  While other neurotransmitters such as dopamine, oxytocin, or vasopressin may also be important in facilitating social interaction, endorphins allow individuals who are not related or mating to interact with each other long enough to build ``cognitive relationships of trust and obligation'' \citep[1839]{Dunbar2012}.  It is thought that, as the homo genus evolved more complex collaborative capacities for survival in interdependent group contexts, grooming-like behaviours sustained social bonding in larger groups where dyadic grooming would cumulatively take too much time \citep{Dunbar2012}.
Experimental studies suggest that neurophysiological mechanisms activated by activities that involve physical exertion and coordinated movement, such as group laughter, dance and music-making, exercise, and group ritual can bring groups closer together, mediated by the psychological effects of endogenous opioid and endocannabinoid release \citep{Cohen2009,Fischer2014a,Fischer2014,Sullivan2014,Tarr2016,Tarr2015}.

In addition to reports of exercise-induced euphoria and positive affect, adherents to (group) exercise and other activities---particularly highly skilled practitioners---also commonly report experiencing states of ``optimal'' or ``peak'' performance, which include feelings of heightened focus, personal transcendence, time-warp (the experience of time either speeding up or slowing down), spontaneity, creativity, and effortlessness \citep{Jackson1995a}.  ``Flow,'' as this particular cluster of states has commonly been referred to, is a powerful, autotelic and embodied experience, which combines components of both ``hedonic'' (sensation-centred, see \citep{Huta2010}) and ``eudaimonic'' (meaning-centred, see \cite{Ryff1989,Ryff2015}) dimensions of subjective well-being, and is theorised to emerge when activity strikes a balance for the individual between challenge and skill requirements \citep{Csikszentmihalyi1990,Abuhamdeh2012}.  At the level of the group, the ``team click'' and ``group flow'' are highly elusive possibilities, coveted by athletes, coaches, and fans alike \citep{Novak1993,Sawyer2006}.  While the experience of flow associated with prolonged exercise may be in part neuropharmacologically mediated by the opioidergic and endocannabinoidergic systems, phenomenological accounts suggest that there is something distinct about the experience of flow in exercise that requires a more complete cognitive and social explanation \citep{Dietrich2006,Dietrich2011}.  One speculative neurocognitive account of acute exercise, for example, suggests that the metabolic costs associated with complex or prolonged regulation of movement forces an energetic trade-off in the brain in which lower level neurocognitive processes win out, forcing a down-regulation of the pre-frontal areas of the brain \citep{Dietrich2011}. Dietrich and colleagues propose that the down-regulation of cortical processes induces a decline in executive control \citep{Labelle2013} and possibly dampens self-monitoring and personal agency. If this hypothesis is correct, it is highly plausible that flow and its neurocognitive underpinnings are relevant to the affective and prosocial effects of group exercise.

Meanwhile, research in social psychology focusing on the relationship between time-locked behavioural synchrony and processes of self-other merging, social alignment, and affiliation has shed light on the social and affective significance of interactive and coordinated movement typical of many group exercise contexts \cite{Wiltermuth2009,Kirschner2010,Reddish2013,Tuncgenc2016}. Experimental evidence suggests that time-locked coordination of behaviour between two or more individuals in the stable attractor/equilibrium states of either in-phase or anti-phase synchrony is conducive to psychological processes of self-other merging, liking, trust and affiliation.  It is believed that lower cognitive processes of joint attention mediate the link between synchrony and social bonding, with synchronised activity (common in music, dance, and some sports) providing a shared spatio-temporal (and often haptic) referent around which to coordinate attention and behaviour \cite{Launay2016,Wolf2015}.

Studies linking synchrony with social bonding and cooperation are supported by a literature than connects nonconscious mimicry with liking and affiliation\citep{VanBaaren2009}.  The experimental studies above predominantly refer to dyad synchronisation of behaviour.  The social and psychological effects of group level synchronisation have been harder to induce and measure in experimental settings. However, in addition to in- and anti-phase behavioural matching, group synchronisation may be subject to more complex and dynamical processes of coupling, which could entail specific psychological consequences. This also appears to be true in cases of joint---but not necessarily explicitly synchronised---action, whereby implicit processes of movement regulation link two or more individuals in a complex and dynamic coupling. The variation and stabilisation of such dynamic couplings could have psychological effects (see \citep{Schmidt2008,Marsh2009a}).  Most encouraging is evidence that manages
to integrate the social and neurophysiological dimensions of group exercise.  Recent experimental evidence suggests that social features of the exercise environment (for example, perceived social support, level and quality of behavioural synchrony, etc.) modulate exercise-induced mechanisms of pain, and reward \citep{Cohen2009,Sullivan2014,Tarr2015,Davis2015,Weinstein2016}. This work is bolstered by existing literature on the social modulation of pain \citep{Eisenberger2012a} and links between pain and prosociality \citep{Bastian2014a}.


\section{Theoretical Grounding}
A combination of recent advances in neuroimaging technologies \citep{Frith2007}, emerging neurocomputational theories of brain function \citep{Friston2010,Frith2010,Clark2013}, and constructive attempts to extend the theoretical paradigm of human social cognition to account for inter-individual processes of interaction and coordination \citep{Sebanz2006,Dale2014}, has created an opportunity to examine in finer-grained detail the relationship between coordinated and exertive group activities and social cohesion.  It is now more clearly understood that basal human capacities for physical movement regulation and coordination set the foundation for social cognitive systems whose resources are distributed between brains, bodies, and physical features of task-specific environments \citep{Hutchins2000,Kirsh2006,Semin2008,Semin2012,Coey2012}.
Human cognition appears to be driven by a processes of ``active inference'' \citep{Friston2010} about the world.  Agents generate top-down interoceptive predictions about the state of the world and test these representations against bottom-up sensory evidence \citep{Clark2013}.  In this account, perception, representation, emotion, and action are unified by the logic of prediction-error management, and neurocognitive components interact to align the organism with its expectations \citep{Pezzulo2014}.  Conceiving of social cognition in this way, as an embodied, embedded, and immediate process of inference, centralises the role of automatic movement regulation strategies---traditionally classed as ``lower-cognitive'' processes---in establishing and maintaining the transfer of cultural information between individuals, within groups, and throughout populations---traditionally thought to be executed by  ``higher-cognitive'' processes \citep{Claidiere2014}.

A review of the available literature suggests that successful joint action in humans is  contingent on the ability to share functionally equivalent task representations. Considering the cognitive principles of ``active inference'' referenced above, shared task representation amounts to minimising prediction error in social cognitive systems involving two or more co-actors \citep{Semin2008,Frith2010}.  Humans appear to employ an array of explicit and implicit behavioural strategies in order to achieve this.   The ways in which co-actors ``close the loop'' \citep{Frith2007} on joint action through deliberate ostensive communication has been the traditional focus of developmental, comparative \cite{Tomasello2005a}, and social psychologists \citep{Sebanz2006}.
More recently, however, analysis of dynamic coupling of co-actors in joint action scenarios reveals that synchronised movement implicates an array of implicit and pre-perceptual cognitive processes of alignment and prediction error minimisation \citep{Schmidt2011}, which, in addition to more explicit forms of communication, could be central to the generation of feelings of self-other merging, self-other distinction, and perceived reliability and trust associated with social bonding \citep{Marsh2009}. By interrogating the ways in which component mechanisms and system dynamics of joint action generate social bonding, this dissertation seeks to offer a novel contribution to the cognitive and evolutionary anthropology of social cohesion.


\section{Joint Action in Group Exercise}
There is considerable variation in the nature and dynamics of joint action, even within the sub-category of group exercise. Joint action in group exercise ranges from tightly coupled dyadic or group activities such as rowing, synchronised diving, or dance sport, to interactive competitive team sports like basketball, ice hockey, and rugby, through to more loosely coupled (but still time- and space-coordinated) mass participation activities such as marathons and triathlons.  It is sensible to assume that, as the scale and requirements of these contexts vary, so too will the psychophysiological mechanisms most responsible for enabling successful joint action, feelings of team click, and social bonding \citep{Mogan2017,Launay2016}.

Interactive and co-active team sports in particular contain dimensions of complexity that are not directly addressed by the existing experimental literature concerning synchrony or joint action.  The competitive nature of these sporting practices means that co-actors in joint action scenarios will perform roles that either facilitate or obstruct shared goal achievement, depending on team assignment \citep{Renshaw2009}. Competitive joint action scenarios facilitate two modes of communication between individual participants: more predictable behaviour between cooperators and less-predictable action behaviour between opponents \citep{Glover2017}. Thus the competitive dimension of interactive team sports introduces complexity, whereby subunits of cooperating co-actors coordinate their behaviours around a shared goal (winning the specific contest) \citep{Passos2012},  and co-actors from both teams coordinate with each other around the higher order shared goal of completing a competitive game.
In addition, interactive team sports involve the nesting of coordinated subunits of actors and sub-phases of actions \citep{Vilar2012}.  For example, a dyadic joint action such as passing a ball between two attacking players in association football is nested within a larger attacking sub-phase goal of advancing towards the opposing team's goal in order to score a goal, which is in turn nested within a larger shared goal of beating the opposing team in a 90 minute match, and so on.  These dimensions of complexity in interactive team sports increase the overall degrees of freedom of joint action tasks, thus demanding higher technical competence in order to successfully establish functional interpersonal synergies capable of reducing such uncertainty and behaving adaptively \citep{Duarte2012}.

\subsection{Rugby Union Football}
Rugby Union (hereafter rugby) is an interactional team sport played on a rectangular field (100m x 70m), by two teams, usually of 15 players, who physically contest possession of an egg-shaped ball that can be used to score points \citep{IRB2014}.\footnote{Descending from a variety of locally-specific folk-games played in pre-industrial England, all loosely grouped as ``football'', rugby developed within the elite public school system as a deliberate physical activity arbitrated by rules and regulations, before circulating through the arteries of England's colonial empire as a leisurely pastime—a ``sport'' \citep{Dunning2005}.}
``Rugby sevens'' (hereafter Sevens), the version of rugby that is the focus of this research, is a shorter 7-on-7 version of rugby. Sevens is a highly interactive and physiologically demanding sport at all levels at which the game is currently played, even more so than the 15-a-side version of the game.   Sevens requires players to participate in frequent bouts of intense (anaerobic) activity such as sprinting, physical collisions, tackles, and grappling, separated by short bouts of low intensity activity such as walking and jogging. Sevens requires high levels of interdependence between team members due to the uncertainty and complexity of interactive coordination tasks.  At the elite level in particular, the physiological costs and complexity of joint action requirements of sevens are amplified.

There is evidence to suggest that dynamic coupling occurs between dyads and sub-units of attack and defence in rugby \citep{Passos2011,Correia2014}.  Passos and colleagues \textcite{Passos2011} for example find that functional coupling tendencies emerge between attacking dyads and adapt to specificities of the task environment.  Correia and colleagues \textcite{Correia2014} show that coupling tendencies also emerge between co-actors of opposing teams in rugby, for example, in a 1-on-1 attacker/defender sub-phase.  These results are confirmed in similar joint action contexts in other equivalent sports such as basketball and association football \citep{Duarte2013}. There is evidence to suggest that the establishment and maintenance of functional interpersonal synergies in rugby joint action depend on an athlete's perception of affordances of the task-specific cognitive system made up of constraints including other athletes, the physical environment, and the rules of the game \citep{Passos2012}.

Very little direct empirical evidence specific to rugby can be used to substantiate a link between joint action and team click, and team click and social bonding.  Rugby is, however, a sport heavily associated with ``social bonding'' in the more popular discursive sense, particularly in all-male social organisation common in the elite educational institutions of England and Commonwealth countries in which rugby originally developed \citep{Dunning2005,Richards2007,Collins2008}.\footnote{Rugby union has been the site of much criticism worldwide due to the fact all-male social spaces cultivated by rugby appear to support and sustain hyper-masculine and hyper-normative behaviours, including gender-related violence \citep{Cosslett2014,Guinness2016}.
}   ``Rugby is a game for barbarians played by gentlemen,'' or so the saying goes.\footnote{The origins of this oft-cited Rugby adage is unclear.  The phrase is supposedly the adopted motto of the British Barbarians Football Club, established in 1890 \citep[34]{Dunning2005}.  The complete phrase reads ``Rugby is a game for barbarians played by gentlemen, football is a game for gentlemen played by barbarians.''  However, official club history cites its original motto as, ‘Rugby Football is a game for gentlemen in all classes, but for no bad sportsman in any class' \citep[vii]{Starmer-Smith1977}.  Some sources attribute the saying to British writer and poet Oscar Wilde (1854-1900) \citep{Fleenor2015}}. Different inflections on this adage have been reproduced by people in all parts of the world that rugby has reached (including China), presumably as a way of linking the nature of rugby's physical requirements with social virtues of fair play, cooperation, and moral integrity.  Although direct experimental evidence concerning rugby is scant, the physiological demands, joint action complexity, and social-historical trajectory of rugby suggests that it is extremely suited to an investigation of the social bonding effects of joint action in group exercise.

\section{Cultural Variation}
In addition to micro-level details and dynamics of joint action, macro-level variation in the cultural contexts of joint action also vary extensively. Importantly, macro-cultural expectations appear to frame and direct micro-level movement dynamics of joint action.  As sporting anecdote indicates, different teams from different places and times appear to play the same game in very different ways---embodying different ``styles'' of play.  While there is very little literature devoted to examining the effect of cultural variation on joint action and social bonding in particular, there is extensive evidence to suggest that cultural variation impacts on processes of cognition \citep{Nisbett2003,Hoshino-Browne2005}, social learning \citep{Mesoudi2015}, and prosocial behaviour \citep{Yuki2005,Yuki2003}.
It has been suggested that cultural environments structure joint action scenarios in ways that help ``smooth'' coordination by providing equivalent expectations between co-participants \citep{Vesper2017}.  Indeed, as anecdote and observations concerning suggest, perception of ``team click'' is not necessarily limited to the most proximal dimensions of joint action perception, but is rather contingent on the snug fit between a given joint action and a whole assemblage of hierarchically ordered expectations.\footnote{It is also important to bear in mind that, while the neurological, cognitive, and psychological theories from which the predictions of this dissertation strive for universal generalisability, these theories are nonetheless heavily grounded in Western epistemological assumptions, intuitions, and ``WEIRD'' empirical evidence \citep{Henrich2010a}.}


\section{The present Study}
\section{Overview of research}
The empirical content of this dissertation is drawn from on one contemporary instance of group exercise, in one geographic region.  Rugby union football and the People's Republic of China are subjects not commonly heard uttered in the same breath or pictured in the same sentence.  Nonetheless, the Olympic status of rugby union, and the deep Olympic logic of the state-sponsored Chinese sports system, means that today hundreds of professional Chinese rugby players are meaningfully engaged in one of the world's most physiologically strenuous interactive team sports.  During a two year period between August 2015 and September 2017, I spent three separate periods in China during which time I conducted a total of 10 months of ethnographic research with the Beijing Men's Provincial Rugby Team.  I then extended this ethnographic analysis by conducting as two field studies, for which I sampled from a broader population of professional Chinese rugby players from 9 different provinces.

Between August 2015 --- March 2016, I spent seven months in Beijing engaged in participant observation and conducting unstructured and semi-structured interviews, and informal surveys with the Beijing Men's Rugby Team. Between July --- August 2016 I returned to China for a further two months, during which time I continued ethnographic observations of the Beijing team, while also conducting two pseudo-experimental field studies spanning two other locations, Hebei province and Shandong province. Finally, I spent one month in Beijing and Tianjin between August --- September 2017 during which time I conducted follow-up structured interviews the with 10 athletes who participated in the Chinese National Games, as well as follow up informal interviews with athletes from the Beijing Men's Rugby Team.



\subsection{Early developments of physical culture and sport in China}
Throughout China's modern history, a rich indigenous physical culture merged with modern waves of cultural importation beginning in the mid 19th Century. Modern sport and exercise was first introduced to China as part of the ``New Culture Movement'' at the start of the 20th Century—--a movement in which student intellectuals problematised traditional Daoist and Confucian understandings of the body as ``passive'' and ``feudal,'' and suggested that a new, active and competitive body, should be realised \citep{Ge2005}.  Spenserian ideology celebrated the cultivation of the physical body as foundational to the cultivation of the modern Nation-State \citep{Morris2004}. In this vein, the passive and weak Chinese body of the feudal past was publicly identified by student intellectuals and nationalist political movements as the cause of the China's collective weakness as it grappled with colonialism in the early 20th century.
In its place, a strong, masculine, and active body as was established as a central public representation of China's bright future \citep{Brownell1995}, see Figure ~\ref{fig:motherlandStrength}).  As such, towards the end of the 19th century, traditional practices of self-cultivation such as the Daoist notion of ``cultivating life'' (\textit{yangsheng zhidao}), which included traditional martial practices of taichi and qigong, were denounced by reformers in favour of a variety of imported physical regimes and an associated philosophy of ``training the body'' (\textit{duanlian shenti} 锻炼身体, see \cite{Farquhar2012}).

\section{Study Findings}

\section{Dissertation Overview}

\section{Contribution}
