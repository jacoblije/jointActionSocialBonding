\chapter{\label{app1:intro}Appendix: Introduction}












\section{Group hunting in Chimpanzees\label{sect:groupHunt}}
Group hunting has been well documented in various geographically distinct populations of Chimpanzees (i.e. by \textcite{Goodall1986}  in Gombe national park, by \textcite{Uehara1997} in Mahale Mountains National Park, and by \textcite{Boesch1989, Boesch1994} in the Tai National Park in Ivory Coast West Africa.  This triangulation of observations suggests a species-wide capacity for elaborate coordination combined with bouts of intense physiological exertion.   Observational studies suggest that the size and membership of hunting parties vary greatly, from a single chimpanzee to as many as 35 \citep{Stanford1994}.  In addition, hunting frequency appears to covary with fluctuations in group size (and particularly number of sexually receptive females in the group \citep{Stanford1996}.
Observational studies suggest that the size and membership of hunting parties vary greatly, from a single chimpanzee to as many as 35 \citep{Stanford1994}: Typically, a group first gathers and then simultaneously fans out in search for a victim while reducing the chances of detection by avoiding vocalisations and using slow and careful steps to minimise noise \citep{Boesch1989,Mitani2001}.  There is also a division of labor based on roles that was characterized by Boesch and Boesch (1989) as collaboration: a ``blocker;'' a ``chaser;'' an ``ambusher;'' and others remaining on the ground to track the hunt and intercept a fleeing target if the opportunity arises.

The evolutionary function of group hunting in Chimpanzees may not be limited to a nutritional shortfall hypothesis (which suggests that chimpanzees hunt to compensate for seasonal shortages in food availability).  A second hypothesis argues that male chimpanzees hunt to obtain meat that they swap for matings. A third hypothesis proposes that males use meat as a social tool to develop and maintain alliances with other males \citep{Mitani2001}


\section{The Modern Synthesis\label{sect:modernSynthesis}}
Generally speaking, rigorous and scientifically testable accounts of human behaviour have emerged in the last ~70 years, facilitated by 1) the gradual refinement of evolutionary theory over the last 200 years now known as the ``modern synthesis,'' as well as 2) the ``cognitive revolution'' of the 1950s and 60s, in which mechanisms associated with information theory, cybernetics, and computation provided useful conceptual metaphors for understanding population-level transmission and fixation of biological and cultural variants. Below, I provide a brief overview of the main assumptions, protocols, and critiques of the modern synthesis, identifying knowledge gaps and research opportunities along the way.

The modern synthesis (also known as ``neo-Darwinism'', hereafter simply MS) refers generally to the gradual maturation of evolutionary theory in the last 200 years, and specifically to the unification of the theory of evolution by natural selection (attributed to Darwin and Wallace in the second half of the 19th century) with a theory of genetic inheritance (replacing a previously popular theory of blended inheritance).  The MS was first proposed by Huxley in 1942, following successful mathematical formalisations performed by population geneticists between 1930 and 1947 (e.g., Fisher (1930) and Haldane (1932)).  Subsequent advances in molecular biology and genetics, including the verification of the structure of the DNA molecule by Wallace and Crick in 1954, paved the way for a definition of biological evolution as changes in the frequency of heritable DNA sequences in a population due to selection pressures exerted at the level of the phenotype \citep{Dawkins1976,Grafen1984}.  Shown to be mathematically plausible, the mechanism of genetic inheritance served to explain observable intra-species phenotypic variation (for which the preceding theory of blended inheritance failed to account), and confirmed Darwin's original insight that organismic change occurs via gradual population-level accumulation of adaptive traits over evolutionary time. The MS and its associated methodological innovations and empirical findings have collectively transformed scientific knowledge of the historical origins and function of biological life, including human life. The modern synthesis has been widely touted as the second most successful theory in the history of science after modern quantum physics in its ability to explain and predict the world in which we live \citep{Dunbar1996}.


\section{Neurobiological reward in exercise\label{sect:neuroRewardGE}}
A series of laboratory studies involving human and non-human subjects exercising on stationary exercise equipment (treadmills, watt bikes) show that sustained aerobic exercise at a moderate intensity ($\sim70-85\%$ of maximum heart rate)---but not low ($\sim45\%$) or high ($\sim90\%$) intensities---induces activity in the endocannabinoid (eCB) system \citep{Raichlen2013}, and similar results have been obtained in studies on the opioidergic system \citep{Boecker2008}.  Endocannabinoids appear to play an influential role in exercise-specific neurobiological reward, with studies showing activation eCB activation in moderately intense exercise and in cursorial mammals, such as humans and dogs \citep[but not non-cursorial mammals, e.g., ferrets;][]{Raichlen2012}.  In addition to direct peripheral (analgesic) and central (psychological well-being and alteration) effects, eCBs are also responsible for activating ``traditional'' neurotransmitters (opioids, dopamine, and serotonin) also responsible for rewarding and reinforcing behaviour \citep{Sparling2003}.  These findings lead Raichlen to propose eCBs as a key neurobiological substrate responsible for motivating habitual engagement in aerobic exercise, by generating ``appetitive'' and hedonic associations with exercise behaviour \citep{Raichlen2012}.
