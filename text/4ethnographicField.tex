First week in China (introduction vinnete):
- ``rou'': mysterious carnal sport --> social community
- XNT fall from grace:
- Qingdao:
    - basketball (actionScrutiny)
    - Asia 7s: (Qi gaige & Li sheng) (system, incentives)
    - National 7s: Beijing:WCY out of the huddle


XNT:
- Sport as a upward mobility strategy:
    - Many from athletics (IND) to rugby (TEAM)
- Fall from grace --> Lack of click?
- System: coach ()
- Zero to one: new comers, uni students, senior players, old motherlandStrength

In this system, how do they interaction

%%%%%%%%%%%%%%%%%%%%%%%%%%%%%%%%%%%%%%%%%%%%%%%%%%%%%%%%%%%%%%%%%%%%%%%%%%%%%%%%%%%%%%
ABSTRACT:
Fieldwork in Beijing

















Qingdao:
Kai's hometown was Qingdao, and so we travelled down together on the high speed train after Kai finished work on the Friday evening in time to watch the Asia Sevens Tournament on the Saturday.  The Asia Sevens is an annual series of three regional rugby sevens tournaments featuring men's and women's national sevens teams.  The men's series has been held regularly since 2009, and the women's series was established in 2013. The series usually consists of two three annual Tournaments, alternating between various locations including Hong Kong, South Korea, Sri Lanka, China, Japan, Malaysia, India, Singapore, and Thailand.  In 2015, The Asia Sevens Series Tournaments were held in Colombo, Bangkok, and Qingdao. Would be played on the Saturday and Sunday, and then the National Tournament would follow immediately after on the Monday and Tuesday.



Olympic Qualification:

When I arrived at the stadium in time for the beginning of the Asia Sevens Series, I met a number of coaches and athletes in the stands who I knew from my time in Beijing in 2008 and then coaching in 2013. I sensed from my various interactions that there was an air of nervousness around the Tournament, particularly on behalf of the Chinese women's team.   The Qingdao tournament was the final tournament before the Olympic Qualification Tournaments, to be held in Hong Kong and Tokyo in November 2015.  The top ranked team from those two legs would qualify for the Rio Olympics in 2016.  The Chinese men's team were not in serious contention for Olympic qualification, given the clear superiority of the more established men's rugby programs like Japan and Hong Kong. The Chinese Sports Commission did, however, expect the Chinese women's team to qualify for the Olympics.

Since the Chinese women's team was first established in 2002, China had made great strides in women's rugby, in both Asia and globally.  Not including occasional losses to closest rivals Kazakstan, between 2002 and 2012 the Chinese women's sevens team was the dominant women's team in Asia, easily outcompeting Japan and Hong Kong, and at times was competitive against the world's best including New Zealand and Australia.  The main reason for China's dominance in the women's game during this period was that other traditional rugby nations, despite having developed professional infrastructure for the men's game, lacked almost entirely an equivalent infrastructure for the women's game. China`s state sponsored sport system, on the other hand, was relatively agnostic towards gender in sport. According to the incentive structures of the Chinese sports system a gold medal is a gold medal, regardless of the gender of the recipient. (This is not to say that there are not distinct gender inequalities in relation to sport in China).
Indeed, beginning with the Chinese Women's Volleyball Team's gold medal victory at the LA Olympics in 1982, China has enjoyed a comparative advantage in women's sport due to the fact that the Chinese sport system was comparatively more supportive of women's sport.

Alarmingly for the Chinese women's rugby team's hopes of Olympic qualification, by 2014 it had become obvious that other more traditional rugby playing nations in Asia, namely Japan and Hong Kong, had begun to make up serious ground on China and Kazakstan in the women's game.  This naturally prompted nervousness among the GAS and CRFA.  In mid 2015 it was decided by the GAS and CRFA that they would enlist the services of a foreign coach.  According to sources close to CRFA, apparently the original plan was for the appointed foreign coach, BG, to act as a consultant for LXH and his existing group of coaches.  By the time I arrived in Qingdao in early September, however, the initial arrangement had since transformed into a situation in which BG was given 100\% control over the program as head coach, and LXH was more or less sidelined as coach.

The complication was that before being dethroned, LXH preferred to use his own athletes.  Most of the starting team at the LDN7s in June 2015 were indeed Shandong athletes, many of the women who had won gold at the National Games in 2013.  When BG took over the reigns as head coach, as directed by the GAS and CRFA, he set about reorganising the starting team and also scouting for new talent outside the squad, which was predominantly made up of Shandong athletes.

When I arrived in Qingdao it appeared that tensions between the old and new guard were at their peak.  LXH and many of his favoured athletes had been relegated to the bench, and some had been completely removed from the squad altogether.  There were 6 weeks to go before the all important first Olympic qualification tournament in Hong Kong.  I sat and watched the first few games of the women's Tournament, and I was indeed surprised to see that the Chinese women's side was missing some of its usual stalwarts, and indeed appeared in my eyes to lack the flow and familiarity that I had come to expect in LXH's clan.  In the stands I came across a group of Shandong women, one of which, QGG I knew quite well from when she toured to Australia with the Shandong team when I was still with the Australian rugby sevens team in early 2013.  I asked QGG why she wasn't playing for China, and it turned out that she was injured, and so wasn't eligible for selection.  But a few of the other women around her, who were all wearing Chinese national team uniforms, were all part of LXH's clan who had been effectively stood down by BG. ``How do you think they're playing?'' She asked me, after we had exchanged pleasantries. ``Not great'' I commented, hesitating, not knowing how much I should prime her, but also feeling obliged to be honest: ``it feels like they’re not coordinating together very well at the moment.  What do you think?'' (一般吧。感觉她们的配合不太好,目前。你觉得呢?) I asked.  ``They’re out there playing as individuals, not playing as a team! They can't get it together; there's no shared goal.'' `(她们都在打个人的,不打团体提的。打不到一块儿去啊,没有共同的目标.)  ``Hmm. Yes it does look like that.'' ``Hey, Lijie...'' she asked me quietly, ``...don’t you think they’re not even playing as well as our Shandong team could play?'' (嘿,李杰,是不是她们现在打的没有我们山东队打的好,是吧?)

As was common on my journeys through the world of rugby in China, I often did not quite grasp all the details and pieces of the puzzle that contextualised the interactions I was having until after the fact.

Later that day I bumped into LS, who had been assistant coach of China with LXH and now BG since 2014.  LS was a Qingdao local, a CAU graduate, and a member of the Beijing Men's team from 2010-2013.  I asked him about the current situation with the Chinese women's side and their prospects 6 weeks out from the first Olympic qualifier.  ``Chinese athletes need to see the (individual) benefits if they are going to put their bodies on the line and put in for each other'' he insisted, and he went on to explain why for these athletes, there were no obvious benefits available sufficient to motivate them.  There were indeed very few material benefits associated with representing China in rugby at the national level.  Athletes were payed a nominal USD100 per month on top of their provincial contracts when training and touring with the national program.  If athletes were injured while playing for China, CRFA at the time did not have access to sufficient health insurance to cover the costs of treatment, and athletes had no choice but to return to their provincial programs and seek treatment at the expense of the province.  The less tangible benefits of playing for China, for example, access to high quality coaching, or the pride of representing the country in a sport, or even the promise of a trip to the Olympics, were heavily outweighed by other less tangible costs: long stints of time away from family, the constant risk of falling out of favour with provincial programs...  In effect, the lack of incentives at the national level meant that athletes were by definition more committed to their provincial systems---the programs that provided athletes with the benefits that they were most interested in obtaining, such as tertiary education, future employment, modest but compared to CRFA, a reasonable salary (most national level players were paid 3-8k RMB/month). ``Its no wonder these athletes aren't performing well,'' LS exclaimed.

EXPLANATION: weakness of the institution?

Parallel to Beijing?


Stand talk about attributes and critical of technique etc.
Basketball:



National Women's teams: Li Sheng
Beijing Team: Huddle at the end




















%%%%%%%%%%%%%%%%%%%%%%%%%%%%%%%%%%%%%%%%%%%%%%%%%%%%%%%%%%%%%%%%%%%%%%%%%%%%%%%%%%%%%%%%%%
%%%%%%%%%%%%%%%%%%%%%%%%%%%%%%%%%%%%%%%%%%%%%%%%%%%%%%%%%%%%%%%%%%%%%%%%%%%%%%%%%%%%%%%%%%
%%%%%%%%%%%%%%%%%%%%%%%%%%%%%%%%%%%%%%%%%%%%%%%%%%%%%%%%%%%%%%%%%%%%%%%%%%%%%%%%%%%%%%%%%%




XNT system:



Cultural Modes of group membership and social cognition:





\section{The Beijing provincial men's rugby team}
I focus my ethnography on the Beijing provincial men's rugby team ($n=26, avg. age=21.3, range = 17-27, SD = 2.96$).  The Beijing provincial men's and women's rugby programs are based at the Xiannongtan Sports Institute in Beijing (one of Beijing's four major sports institutes, and home to seven different full-time sports programs, hereafter Xiannongtan).  These athletes represent Beijing at a provincial level, playing against other provinces in annual tournaments, and every four years at the all-important National Games.  Five athletes have previously represented China in international rugby sevens tournaments.  There is also a Beijing women's rugby team at Xiannongtan, but I was not able to accompany the activities of both teams closely enough to perform adequate ethnographic research.

Members of the Beijing men's rugby sevens team are either already-contracted (n=13) or aspiring professional athletes (n=13) who live and train 6 days a week at Xiannongtan and occasionally attend university or high school classes as part of their ongoing education---what can only be termed part-time education.  A head coach and assistant coach look after the day to day organisation of team schedules and training, and these two coaches are assisted by a further two player-coaches, who are in the gradual process of moving from athlete to coach status.  One of Xiannongtan's four principals is responsible for the management and administration of the rugby program, and is occasionally present at team meetings and national competitions.

Athletes are all from relatively modest socio-economic backgrounds, many hailing from suburban and rural areas of northern China (Shandong (11), Beijing (6), Jiangsu (3), Liaoning (2), Hebei (2), Anhui (1), Henan (1), Heilongjiang (1)).  The squad consists of 10 fully contracted senior athletes (\textit{xieyi}), three provisionally-contracted athletes \textit{shixun}, and six student-athletes \textit{erjiban} who do not receive a salary but receive training, food, board, and educational support.  The remaining athletes (7) are classed as athletes in training \textit{jixun} and are effectively on trial until they either show promise or withdraw from the squad either voluntarily or upon suggestion by the head coach.  Provided that they meet the relevant academic and athletic requirements, contracted and student athletes are able to attend the Beijing Sports University—considered to be the country's most prestigious sports university and one of China's ``top brand universities'' (\textit{mingpai daxue}).

The average rugby training age (years spent playing rugby) of Beijing athletes is 3.12 years ($range = 0.16 – 10 years$).  Contracted senior athletes ($average age = 24.3 years$) have trained for an average of 5.4 years, whereas the average training age of junior non-contracted athletes (average age = 19.3 years) was 1.7 years.  Over half of the athletes have a background in other sports (15 athletes from track and field, one from football, one from basketball), usually beginning part-time or full-time physical training at the age of 11-13.  Those who transferred to rugby did so either at the beginning of senior high school (16 years) or at university age (18yrs).  The rest of the group had no particular sporting background before starting at Xiannongtan, and were scouted by school athletics coaches based on their basic athletic attributes (running speed, strength, coordination, and potential for physical growth).  Of the 26 athletes in the squad, three junior athletes who were part of the squad when I arrived in September 2015 have now left, and three new athletes have arrived. This movement of non-contracted players in and out for trials is quite common.

\subsubsection{Training schedule}
Every year between April and September there are five national tournaments held in different locations across the country.  October –-- February constitutes the off- and pre-seasons for these yearly competitions, during which time teams travel to domestic or international training locations depending on amount of program funding and training strategy.  In 2015, before an unexpected change in coaching team at the end of December, plans were to travel to Yunnan in the new year province for altitude training (January) before moving to sea-level somewhere in the south (February/March).  Following the coaching leadership change, the team did not leave Beijing until after Chinese New Year (15th February), which meant that training during this period was influenced by Beijing's cold winter weather and air pollution.

Below is a table of a typical weekly training schedule. A typical week consists of 10x2.5hr training sessions, three of which are strength and conditioning sessions, seven of which are on-field rugby sessions.  In addition, two one hour evening skills sessions are also added for junior athletes to hone their basic skills of passing, catching, and game-play.  Athletes live full-time on campus at XNT in dormitory accommodation (usually 3 athletes per ensuite room), and are permitted leave on the weekend after the conclusion of Saturday morning training.  Athletes with family in Beijing usually take this leave, while the remaining athletes spend weekends at XNT.  Athletes break at the end of season (September) for two weeks, and occasionally around Chinese New Year for 7-10 days, unless New Year interrupts pre-season training plans, in which case training continues throughout.


\begin{table}[htpb]\caption{Weekly Training Schedule}
  \begin{center}
    \begin{small}
        \begin{tabular}{| c | c | c | c | c | c | c | c |}
          \hline
          & \bf M & \bf T & \bf W & \bf T & \bf F & \bf S & \bf S \\
          \hline
          0600 & Training &  &  & & & & \\
          \hline
          0900 &  & Training & Training & Training & Training & Training &  \\
            \hline
          1500 & Training & Training & & Training & Training & Training &  \\
            \hline
          1900 &  & Training (junior athletes) & & Training (junior athletes) & & & \\
             \hline
        \end{tabular}
      \end{small}
    \end{center}
  \end{table}


\section{Research methods}
Ethnographic data included: unstructured and semi-structured interviews with athletes and coaches (yet to be analysed in-depth), general and activity-specific surveys, and field notes based on participant observation of daily activities of the team.


\subsubsection{Participant observation}
For 6 months in 2015-16 and 6 weeks in summer 2016, I lived and trained full time with the team at XNT, during which time I took daily field notes using a note taking application (Evernote, version 6.11), which was synced between my mobile phone and personal computer. I collated, summarised, and tagged these notes weekly or fortnightly.

\subsubsection{Interviews}
In addition to ad-hoc unstructured interviews with athletes, coaches, and other officials XNT officials and individuals in the Chinese rugby community (n = 15), I conducted 26 semi-structured interviews, each lasting between 15 and 70 minutes.  All interviews were conducted in Modern Standard Chinese (Mandarin) and were recorded with participant consent using a sound recording application on my smartphone or laptop computer.

During semi-structured interviews, I asked athletes about their personal background (including their family situation), their motivations for adherence to rugby, perceived costs and benefits of adherence to rugby, their perceptions of role in the team, their experience of playing rugby (particularly feelings of flow, dissonance, and team click).  The structured interviews also involved two tasks, one in which the athlete was required to rank motivations for adherence to rugby, and one in which the athlete was asked to report their three closest friends in the team, the three team members most willing to sacrifice on behalf of the team, and three most competent athletes in the team (see Appendix for full script). Athletes answered these questions using a pen and paper. I later collated and uploaded these responses to Evernote.


\subsubsection{Surveys}
 I conducted a number of informal surveys designed to understand athletes' general and specific experiences group membership.  I conducted surveys following three training sessions: a session in which athletes (predominantly junior athletes) ran an aerobic fitness test involving straight-line running shuttle-running at and above the aerobic threshold (Beep Test), and two training sessions involving internal game-like scenarios.  In addition, I asked athletes about their general experiences of team membership agency over the team (weak-strong), role in the team (central-marginal), individual performance (weak-strong), team performance (weak-strong), training intensity (\textit{qiangdu})(light-heavy) at two three-month intervals.














%%%%%%%%%%%%%%%%%%%%%%%%%%%%%%%%%%%%%%%%%%%%%%%%%%%%%%%%%%%%%%%%%%%%%%%%%%%%%%%%%%%%%%%



I arrived in Beijing late on a Friday evening at the end of August in 2015.  My close friend Kai---a former Chinese National rugby team representative, and now a lawyer working in Beijing---picked me up at the airport and drove me back to his home.  The plan was to stay with Kai until I was able to make solid arrangements with the Beijing Temple of Agriculture Sports Institute, the home of the Beijing Provincial Rugby Team.

After a day of acclimatising and running errands, on Saturday evening I was invited to a dinner hosted by Adrian, a respected elder within the Chinese rugby community of Beijing.  Adrian was the captain of the second ever class of rugby players to graduate from the Chinese Agricultural University in Beijing---the birthplace of rugby in China.
I first met Adrian through Kai in 2013,

 while coaching in China.  I was originally introduced to Adrian by Kai, a close friend of mine who I met earlier during another stint in Beijing in 2008.

 Kai was also at the dinner, as was Mr Shi, a sports television producer at Chinese Central Television (CCTV).  CCTV had just accepted the rights to the Rugby World Cup, which World Rugby---the international governing body of rugby---had made available to CCTV in an attempt to promote the development of the game in non-traditional rugby playing nations.


I arrived in Beijing late on a Friday evening at the end of August in 2015.  My close friend Kai---a former Chinese rugby player, graduate the Chinese Agricultural University (Chinese rugby's birthplace), and now a lawyer working in Beijing---met me at the airport and drove me back to his home.  When we got to his home, Kai turned on the television and we caught up while some footage from a rugby documentary played in the background.  As it so happened, the international rugby world was on the verge of another Rugby World Cup, which was being hosted by England in the coming months. World Rugby, the world governing body of rugby union, had made the television broadcast rights for the World Cup available to Chinese Central Television (CCTV), in an attempt to promote the game globally.  Having accepted the rights to the tournament, which is the 3rd largest sporting event in the world behind the Olympics and the Football World Cup, CCTV were in search of Chinese rugby experts to help produce the 48-match tournament.  As Kai quickly explained, CCTV's search had led them to the Chinese rugby community based in Beijing, who were almost all, like Kai, graduates of the Chinese Agricultural University---the birthplace of rugby in China and the base for the Chinese National Rugby team between 1996 and 2010.  In fact, CCTV's search led them first to Adrian, the captain of one of the first CAU rugby teams (1992), and currently working for a large international sport organisation in Beijing.  Adrian had then contacted his younger university brother (师弟) Kai, who like him was fluent in English and able to assist in sourcing and translating rugby materials relevant to the broadcast. The CCTV producer responsible for the broadcast, Mr Shi, had scheduled a dinner with Adrian and Kai on Saturday (tomorrow) night to thank them for their willingness to assist in the production.  I was also invited to the dinner. I wasn't scheduled to meet with the Principle and head coach of the Beijing Temple of Agriculture Sports Institute until the following Monday, so I agreed to accompany Kai.
