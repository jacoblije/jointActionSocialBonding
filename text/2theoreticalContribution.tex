
\begin{savequote}[8cm]

    We hear a lot these days about genes and molecules, but how does one human brain, or one human being coordinate, or entrain, or resonate with another?  We may not realise it but we live in a world of coordination, at every level and every scale of endeavour.

  \qauthor{--- J. A. S. Kelso  \textit{Coordination and the Complimentary Nature} Presentation to the The New York Academy of Sciences - May 12, 2010}
\end{savequote}


%I do not see any way to avoid the problem of coordination and still understand the physical basis of life.
%\qauthor{--- H. H. Pattee  \textit{The role of instabilities in the evolution of control hierarchies} 1976}


\chapter{\label{theory}A theory of social bonding through joint action}


\minitoc

\section{Abstract}

In this chapter I outline a novel theory of social bonding through joint action, which will be tested in subsequent empirical studies of Chinese rugby players.  In particular, I identify team click as a special case phenomenon of joint action and a potential mediator of the relationship between joint action and social bonding in group exercise contexts.  Emerging research from the social cognition of joint action suggests that a continuum of cognitive mechanisms are responsible for establishing and maintaining complex and intricate behavioural coordination between two or more individuals.  I consider the role of inter-individual and cultural variation in knowledge, expertise, experience, and personality type in modulating these mechanisms. I conclude the chapter by summarising a theory of social bonding through joint action, and outlining predictions that arise from this theory.







                                            \begin{CJK}{UTF8}{gbsn}

\section{Vignette: Sun Hongwei, Beijing's newest recruit}}
Sun Hongwei arrived, escorted by his high school athletics coach, to the Beijing Temple of God of Agriculture Institute of Sport (hereafter the Institute) soon after I began my fieldwork in August 2015.  An 18-year-old with a slight build and timid demeanour---his gaze remained diverted to the ground during his first few months at the Institute---Hongwei had never seen a rugby ball before that day he arrived.

Hongwei was from Hebei province, immediately surrounding the special prefecture of Beijing, China's capital.  As was relatively common practice at the time in the Chinese competitive sport system, Hongwei's coach had organised a trial for Hongwei with the Beijing Provincial Men’s and Women's Rugby Program (hereafter the Rugby Program) by calling upon social connections to the leadership of the Institute.

Athletes come to the Rugby Program from all over the country.  Representing Beijing at a provincial level in a sport like rugby can translate into the opportunity to gain entrance---via a designated ``specialist athlete'' (\textit{tiyu techan} 体育特产) pathway---to one of China's top universities (Beijing Sports University, in this case) and enhanced career employment opportunities thereafter.  Rugby is not a popular sport in China. It is referred to as a neglected branch of the Chinese sport system, or a ``cold-gate'' (\textit{lengmen xiangmu} 冷门项目), referring to a profession, trade or branch of learning that receives little attention). Despite its minnow status in China in terms of its popularity, rugby's recent inclusion in the Olympics (in the form of the modified seven-a-side version of ``rugby sevens'') means that it now occupies a prominent place in the Chinese sport system, which has been defined since its inception by a strong Olympic logic \citep{Brownell2008}.  If a sport is a current member of the summer or winter Olympic Games, then the sport is included in the roster of sports played at the quadrennial China National Games (\textit{quanhunhui}). As such, rugby programs such as the one at the Institute now exist in twelve of China's 34 provincial level regions, either embedded within, or somehow associated with, tertiary education institutions.  Thus, although rugby and China are two words that have not historically featured within the same utterance, rugby in China now affords athletes a rare and under-capitalised opportunity to pursue attractive life-course opportunities of education and employment in an intensely competitive education system.

Without exception, the athletes who arrive at the Institute to join the rugby team have not spent their childhoods playing rugby in their schoolyards or watching professional rugby on television. Many who come to rugby transition from other more popular sports such as athletics, basketball, or association football, and often---like Hongwei---have never seen a rugby ball before they arrive.  Most ``start from scratch,'' so to speak, in terms of their grasp of the requirements of the highly interactive and technically complex team sport. In addition to complex patterns of movement coordination, rugby also involves unrestrained body-on-body collisions and intense bouts of high physiological exertion, requiring speed, strength, agility, and endurance to perform all of rugby's technical requirements successfully.  Learning the game of rugby from a baseline of essentially zero, while also navigating the inevitably demanding  social and political dynamics within the team and the Institute, was clearly going to be a daunting task for Hongwei.

                        \begin{center}
                          * * *
                        \end{center}

Hongwei was the first of the Program’s new recruits that I followed closely.  Even compared to other newly arrived junior athletes, he was noticeably timid and shy, particularly in his interactions with the coaches (myself included) and senior players. Nevertheless, Hongwei clearly signalled diligence and commitment through his participation in team activities, arriving early to each training session, and carrying more than his fair share of the training equipment---a task shared by the most junior members of the team.  Each time I passed Hongwei in the corridors of the Institute he would greet me with a polite bow and greeting, ``Hello Coach'' (\textit{'jiaolian hao'} `教练好').  In these instances, Hongwei would coordinate his greeting with a moment's eye contact, only to return his gaze to the floor and continue walking.

Due to his initial lack of familiarity with the basic techniques of rugby, Hongwei was unable to properly participate in normal training with the rest of the team.  Instead, during the first month or so, Hongwei stood on the sidelines and practiced the basics with other athletes who were unable to fully participate in training due to injury: learning how to pass and catch the rugby ball, both stationary and in-motion. In my eyes at least---those of an observer accustomed to instinctual grasp of these movements from a young age---Hongwei's attempts to accustom himself with the skills of rugby were jarring.  The idiosyncrasies of rugby's bizarrely shaped ovular ball often foiled Hongwei. I would regularly see him chasing after a ball on the ground he'd just fumbled, as if he was chasing in vein after a scurrying rabbit tactfully evading his pursuit.

Whilst coaching and playing rugby in China, I watched many start exactly as did Hongwei: on the sidelines of training, learning how to pass the ball.  But for some reason I found Hongwei’s attempts to learn particularly unusual.  His actions appeared so mechanical that it was almost as if he was deliberately (over)imitating the required actions of passing, catching, and running as an overt signal of diligence and commitment.
%---> I will work on re-wording this

A few weeks into Hongwei’s time at the Institute, I asked head coach Zhu Peihou about his newest recruit.  He immediately shook his head and scrunched up his face dismissively, adding in a disappointing whisper, ``no good'' (\textit{buxing} 不行).  Chinese rugby coaches are well acquainted with athletes starting from scratch with the technical requirements of rugby---they were used to things looking awkward and ugly at the start.  Coaches are usually more interested in the physical raw materials that enable athletes to develop into rugby players over time.  Often this means that coaches have a habit of fixating on an athlete's baseline characteristics (e.g. height and body frame) as an indication of his or her capacity to develop the physical speed, size and strength deemed crucial for elite performance.  Also important, but less crucial than physical attributes, are an athlete’s baseline ball-handling skills and ``game sense'' (\textit{qiugan} 球感)(which coaches often assess by observing new recruits participating in analogous interactive team sports, like basketball or association football).

Hongwei was still relatively young and physically undeveloped when he arrived, but it was already clear that he was not endowed with a big physical frame; nor was he noticeably fast or agile compared to other athletes.  For these reasons, I assumed, head coach Peihou couldn’t help but let on to me that he was not particularly excited about Hongwei's future prospects in the Program.  In fact, I got the sense that Peihou's reaction to my question contained an element of annoyance or frustration with the terms under which Hongwei had arrived to the Institute, i.e., via the arrangements of Hongwei’s athletics coach.  According to Zhu's own assessment of Hongwei's ability and future potential, the head coach was perhaps conceding to me that he had been forced to accept an athletic dud into the team.

                        \begin{center}
                          * * *
                        \end{center}

I first interviewed Hongwei approximately six weeks after he arrived at the Institute.   Hongwei's demeanour during the interview was consistent with the timid and shy one that he presented publicly at training.  As explain below, he did show some signs of captivation with his new sport and social environment.  When I asked about his initial impressions of the on-field demands of rugby, however, Hongwei was quick to confess that he felt utterly unacquainted:

\begin{quotation}
  SHW: I still haven’t really started to practice any of the team plays or anything; all I can do so far is pass and run a little bit...(but) it's quite fun! \\
  JT: What do you think is the most difficult component of rugby? \\
  SHW: Um\textellipsis well, coordinating with teammates [on the field], particularly coordination in attack.  Because I can't figure it out.  When I first arrived, I didn’t even know what a ``switch play'' or a ``blocker play'' was.
\end{quotation}

  \begin{quotation}
    SHW: 战术没怎么接触,就是像传球啊、跑动什么的会一点了 \\
    JT: 感觉怎么样?\\
    SHW: 挺好玩的!\\
    JT: 你认为橄榄球最难的一部分是什么? \\
    SHW: ...打配合,进攻的配合,因为搞不明白,刚来的时候也不知道什么交叉,后插什么的 \\
  \end{quotation}

Hongwei was self-critical in his confession regarding his minimal grasp of the technical requirements of joint action in rugby. His confession was indicative of a broader trend with the other athletes I interviewed, as I will explain in more detail in the ethnographic sections of this dissertation (see Chapters ~\ref{4partAIntroMethod}\nobreakdash~\ref{6ethnographicResults}). When athletes were asked in interviews about the most difficult aspect of rugby, on-field coordination with teammates was by far the most common answer, particularly among the new recruits and most junior athletes.

As I directed the interview towards topics beyond the on-field technical demands of rugby, Hongwei was more positive, framing rugby as an exciting new opportunity, and commenting that his friends and family were in awe of the fact that he is playing such an impressively ``strong'' physical sport like rugby.  When I asked him about something new that he had learnt through playing rugby, Hongwei automatically responded by emphasising the social dimensions of his experience at the Institute:

\begin{quotation}
  ``...I think it's mainly this thing of having teammates. Before, when I was training for an individual sport, it was just me training by myself. [In that environment] it was a case of whoever trained well was successful.  But now with this team of brothers, elder teammates will take care of younger teammates. We all train together, and if you can’t do something, you can always ask your elder teammates...[Rugby] is so much better, because in an individual sport, if you can't master something, you have to go to your coach for help. Other athletes don't want to teach you, because if you surpass other people, then they have to work even harder to keep up... I have had to learn about helping each other, because rugby is not like an individual sport, where you look after your own performance and that's it.  In a team sport, if you don't do well, there's no need to get too frustrated or upset, because other athletes will help you out, and I will also help others out, that type of collaboration with each other.''
\end{quotation}

\begin{CJK}{UTF8}{gbsn}
  \begin{quotation}
    我觉得主要是师哥师弟的这一块儿,原来练个体项目都是自己练自己的,谁练好了谁厉害,但是现在师哥师弟,有师哥照顾师弟带着,互相练,我不会我可以问师哥
    ...因为个人项目你不会就必须要找教练,但是别人不愿意教你,因为你把别人超越了那别人还还得努力。 学到互相帮助,因为向个体项目自己成绩自己来拿就行,而像团体项目,即使自己做不好,也不用太泄气太沮丧,因为别人会帮你做好,我也会帮别人做好,互相协作的那种.
  \end{quotation}
\end{CJK}

Through his explicit reference to the fraternity of the team, and his position as junior member, Hongwei highlights that the technical skills of rugby were not the only important novelties.  Hongwei's background was as a pentathlete (track and field / athletics---an individual sport), and the team environment was completely new to him, as it was to many other athletes in the team.  As I listened to his experiences associating rugby and group membership, I could not help but associate the quality of these declarations of prosociality with his overly mechanical imitation of rugby's foundational techniques.  It was as if both were equivalently diligent (albeit somewhat awkward and un-habituated) signals of team commitment.

                              \begin{center}
                                * * *
                              \end{center}

A few months passed, and Hongwei continued to train.  He was as eager and committed as when he began, and I did notice some gradual improvement in his grasp of rugby's basic skills.  But he also remained extremely reserved, keeping his head low at all times in team settings, unless addressed by senior players or coaches.  Then, one evening when I had returned to my room in the Institute dormitory from a three-week hiatus in Australia over Christmas of 2015, I heard a knock on my open door, and to my surprise Hongwei took an assertive stride into my room, carrying in two arms a draw-string bag containing rugby balls (which were in need of more air before the next day's training session).  Hongwei had never ventured into my room before, apart from our first interview two months earlier, but never on his own accord.  Remarkably, Hongwei looked me straight in the eyes with his head held high and energy beaming from his face and chest.  I couldn't help but smile and ask, with genuine intrigue, ``How has training been recently?''
``Very good'' he said, assertively and excitedly.  ``Much better than before.  At least now I know what’s going on at training, I can keep up with the plays!''  A big smile grew on his face as he continued to hold my gaze.  ``Oh good!'' I said. I congratulated him for his hard work in training while I had been away, and encouraged him to keep at it.  Wow, I remember thinking to myself, Hongwei was possessed by a powerful force.  Somehow, Hongwei, rugby, and the team in which he was by now more comfortably enmeshed had clicked into place to instil him with a visceral sense of agency.
%(I felt like giving him a pat on the back and suggesting that he try to relax and take the weight off his shoulders, thinking that his anxiety about fitting in may be getting in the way of the ultimate goal of fast-tracking skill acquisition!)
%Hongwei had begun to develop an innate feel for the game.

                          \begin{center}
                            * * *
                          \end{center}


\subsection{Vignette 4}
A few weeks later, head coach Wang (who took over from Peihou, who abruptly resigned while I was away in Australia), told me that he had decided to take Hongwei to pre-season training in Guangzhou, for a month starting in March 2016.  Chongyi admitted that while Hongwei was perhaps not the most promising of the junior athletes, his attitude was very good:

  \begin{quotation}
He [Hongwei] likes to train, and he is very diligent. I want to take his positivity with us [to Guangzhou] \\
他爱练,而且很用心,带上他的积极性过去
  \end{quotation}


                          \begin{center}
                            * * *
                          \end{center}


These ethnographic observations relating to Hongwei's first four months at the Institute highlight key themes of this dissertation.  As I discuss in more detail in this chapter, existing research suggests that successful coordination in joint action requires alignment and maintenance of equivalent expectations between co-actors \citep{Sebanz2006,Vesper2017,Pesquita2017}.  Importantly, evidence also suggests that violation of expectations in joint action has strong affective consequences \citep{Chetverikov2016}.  The fact that Hongwei's familiarity with the technical requirements of rugby appeared over time to co-vary with aspects of his personal demeanour suggests a relationship worthy of further investigation.  As I explore in Chapters LINK and LINK, my ethnographic observations reveal broader patterns of within-group variation between perceptions of joint action performance, team click, and social bonding.  These observations, coupled with predictions from existing literature within the social cognition of joint action, set the foundations for subsequent field-experimental studies in which I test specific relationships with a broader subset of Chinese professional rugby players beyond the Beijing Men's Team.

How is it possible to scientifically account for the unmistakably ``visceral'' quality of Hongwei's transformation from timid newcomer to budding Beijing rugby player?  How can these ethnographic observations be explained in terms of generalisable cognitive mechanisms and systems dynamics of joint action?  Finally, and ultimately, how do answers to these questions improve our ability to comprehend the evolutionary significance of group exercise in the human record?


HW's change is a change in attitude (Bateson)
agency










\section{Introduction}
The human capacity to coordinate behaviour within cohesive social groups is a fundamental explanation for our species' evolutionary success \citep{Tomasello2009}.  It is surprising that sport---an ubiquitous organiser of modern social life and physical movement across cultures---has not been more intensively studied for evidence concerning the cognitive and social foundations of human evolution \citep{Blanchard1995,Downey2005a}.  An integrated scientific study of the physiological and cognitive mechanisms and ecological dynamics associated with coordination of movement stands to offer novel insights into the phenomenon of group exercise.  In this dissertation I attend specifically to the relationship between joint action and social bonding in the group exercise context of professional rugby in China.  In this chapter I formulate a novel theory of social bonding through joint action in order to address the knowledge gaps in the social high theory of group exercise and social bonding.

Physical movement is central to the adaptive success of biological life, particularly life for which movement can intentionally directed in order to bring about change in the environment.  The ability of humans to cooperate with one another vastly increases the range of their potential actions \citep{Clark1996}.  Humans display a rich capacity for complex forms of autonomic and preconceived physical movement, which is employed both when coordinating behaviours with the environment and socially with conspecifics.  The ultra-social nature of the human species dictates that all human movement has a latent action potential---everything from ostensive communicative signals to implicit cues derived from inadvertent movement can carry a social charge \citep{Danchin2004}, and it appears that humans have evolved reward mechanisms for adaptive coordination of movement with conspecifics and the physical environment \citep{Wheatley2012,Parkinson2015,Wheatley2016}. Within cognitive science and psychology, the term \textit{action} is used to distinguish intentional, agent-directed movement from other forms of movement \citep{Davidson1980}.  Specifically, the term ``joint action'' can be used to describe any form of social interaction whereby two or more individuals coordinate their actions in space and time to bring about a change in the environment \citep{Sebanz2006a}.

\subsection{Joint action beyond synchrony}
As discussed in the previous chapter, experimental evidence from the behavioural synchrony and mimicry literatures suggests that exact synchronisation of movement between co-actors in joint action may be a powerful generator of trust and affiliation between co-actors in joint action.  Synchrony appears to be responsible for positive affect, blurring of self-other agency, spontaneous pro-social behaviour, and reinforcement of cooperation \citep{Mogan2017}, and these effects could be underwritten by the activation of neurobiological reward systems \citep{}.  This evidence leads to the suggestion that the effects of behavioural synchrony, in addition to those of physiological exertion, combine in group exercise contexts to generate a psychophysiological environment conducive to forging social bonds \citep{Cohen2017}.

Synchronisation is defined as a process in which two independent components continuously influence each other toward greater entrainment (within a certain lag tolerance) such that synchronising parties reduce overall variance of their joint activity, making them more similar and more regular \citep{Pikovsky2007}.  For the social high theory of group exercise (outlined in Chapter ~\ref{introduction}), behavioural synchrony an obvious place to begin.  Synchrony is a highly ordered form of phase- and frequency-locked coordination, which is relatively easy to operationalise in experimental settings with naive participants, and which provides a clear demonstration of motor and cognitive entrainment between individuals.  Care should be taken in not becoming overly fixated on behavioural synchrony as an ideal expression of coordination in group exercise contexts, for at least two reasons.

\subsubsection{Detailed components of Joint action}
First, while Synchrony is a fast and efficient way in which to achieve motor and cognitive entrainment between two or more individuals, using synchrony as an experimental stand-in for for interpersonal coordination in joint action simplifies joint action and occludes its various component mechanisms.  In real world situations, coordination in joint action involves temporal and spatial precision and flexibility of interpersonal movement regulation across multiple timescales and sensorial modalities in order to bring about change in the environment \citep{Sebanz2006}. Joint action demands a suite of realtime coordination skills.  In order to successfully entrain with others in joint action, participants must constantly anticipate, attend, and adapt to the actions of others and the circumstances of the task-specific environment \citep{Keller2014}. Participants rehearse task-specific skills individually and together over extended periods of time to produce a hierarchy of aligned representations that form the basis of an implicit common ground for successful joint action \citep{Noy2017}.  These skills are constrained by shared knowledge of the task environment \citep{Vesper2017} and modulated by individual differences in personality type and social orientation \citep{Marsh2009,Keller2014}.

When performing joint action, people move in a more predictable way than when moving alone \citep{Vesper2011}, and demonstrate an ability to be more or less cooperative by altering the level of compensatory temporal and phase correction in response to partner's actions \citep{Repp2008}.  These tendencies give rise to a range of entrainment patterns in joint action, such as a leader-follow dynamic \citep[usually in situations where there is a skill disparity between participants,][]{Fairhurst2014}, or a mutually synchronised ``co-confident'' entrainment, which lacks the typical jitter resulting from the reactive control of a leader-follower dynamic \citep{Noy2011,Noy2015}.  Participants often deviate from planned movement (either intentionally or unintentionally), according to familiarity, technical competence, or past experience with the specific joint task \citep{Goebl2009}.  In sum, when coordination is operationalised as strict synchrony, many crucial component mechanisms that precede successful entrainment and social bonding are occluded from view and are potentially overlooked as candidates for explaining a relationship between joint action and social bonding.


\subsubsection{Synchrony as a special case in functional synergies}
Second, a fixation on synchrony alone can lead to the misleading prediction that optimal coordination in joint action should resemble progress toward similarity and equivalence of action between co-actors \citep{Fusaroli2014}.  Strict in-phase behavioural synchrony, although immediately eye-catching and identifiable in well-known group activities such as group dancing, military drills, or sports like rowing, is in fact atypical of most instances of interpersonal coordination.  The corpus of human interpersonal coordination reveals instead that coordination in joint action is more often achieved through function-specific assemblages of complimentary and contrasting behaviours (for example, a dyadic conversation or an ensemble music performance).  Dynamical systems approaches to behaviour and cognition demonstrate that movement coordination involves processes that are in constant flux and subject to self-organising dynamics typical of multi-element systems \citep{Turvey1977}. As such, coordination processes should be modelled as ``functional synergies'' (SOURCE).  Functional synergy offers a unifying model for different levels of analysis from brain \citep{Yufik2013} and motor coordination \citep{Latash2007}, to perception-action coupling \citep{Kelso2009} and to interpersonal coordination in joint action \citep{Riley2011,Schmidt1990}, and even cultural evolution \citep{Claidiere2007a}.  From a ``coordination-as-functional-synergy'' perspective, synchronisation in joint action is only one special case, rather than coordination's inevitable or ideal end-state \citep[128][, Richardson, personal communication]{Kelso2013}.  The relationship between states of coordination in joint action beyond exact in-phase synchrony and social bonding is yet to be thoroughly explored \citep[but for a preliminary approach, see][]{Marsh2009}.




%Synchrony is a predictable and cheap coordinator of behaviour, and is easily scalable to the coordination of behaviour in large groups, which may explain its cross-cultural ubiquity in cultural practices \citep{Dunbar2010,Tarr2016}.



\subsubsection{Overview of a novel theory of social bonding through joint action}



%Anecdotal and observational evidence from anthropology and psychology---particularly the psychology of ``flow'' \citep{Csikszentmihalyi1992,Jackson1999}---suggests that perceptions of joint action success may set the psychological foundation for processes of affiliation.  Various neurological, cognitive, and sociological strands of evidence support this proposal.  Perceptions of successful coordination of behaviour in joint action appears to have positive implications for individual psychophysiological function, health, and subjective well being \citep{Wheatley2012}.  Likewise, there is well-documented evidence of a link between psycho-social isolation and ill-health and developmental and neurocognitive deficits in behaviours key to dynamic interpersonal interaction \citep[e.g.][]{Blakemore2005,Baron-Cohen1991}. An account of the full significance of the role of proximate mechanisms of movement regulation and coordination in social bonding is yet to be fully articulated. In this dissertation I address this gap through theoretical synthesis and empirical research.


In this chapter, I draw upon literature from the social cognition of joint action to formulate a theory of social bonding through joint action.







%\subsection{Team Click}
%This dissertation locates the phenomenology of perceived ``click'' of joint action

% The extent to which synchronised joint action is responsible for generating social bonding may depend crucially on the accordance of action with culturally directed expectations.

%The phenomenology of team click may be momentary subjective awareness of participation in  functional interpersonal synergy---an extra-neural mechanism for minimisation of cognitive free energy.

\section{Theoretical building blocks for a novel theory}



\subsection{Thermodynamic cognition}
There is mounting evidence in favour of a paradigm shift in cognitive science, in which cognition is conceived as a program driven by, and designed for the thermodynamic constraints of living systems \citep{Yufik2017}.  ``Thermodynamic cognition'' of biological life is predicted to adhere to the dual criteria of 1) minimising surprise \citep{Friston2010,Sengupta2013,Sengupta2016,SenguptaFriston2017} and 2) maximising thermodynamic efficiency \citep{Yufik2002,Yufik2013}.  The minimisation of surprise enables an organism to reduce the likelihood of encountering conditions impervious to regulation (for example, the inability to block inflows of destructive substances), while the maximisation of efficiency implies maintaining net energy intakes above some survival thresholds.  Efficient regulation requires mechanisms that necessarily incorporate models of the system and its relation to environment \citep{Conant1970}.

While primitive animals possess small repertoires of genetically fixed, rigid models, more advanced animals possess larger and more flexible repertoires that are amenable to experience-driven modifications. Both evolutionary and experience-driven modifications are forms of statistical learning: models are sculpted by external feedback conveying statistical properties of the environment.  In humans, implicit models become amenable to self-directed composition and modification based on interoceptive, as opposed (or in addition) to exteroceptive, feedback \citep{Yufik1998}.
Constructing and manipulating mental models are means for reaching thermodynamic efficiency in the human brain \citep{Yufik2013}.  Constructing and correcting mental models consumes energy, and thus successful models that persist without or with minimal correction  yield the double benefit of reducing internal energy consumption while also increasing or stabilising the external inflows.  Within this paradigm, thermodynamic ``free energy'' is used as a measure of the energy available to a system to do useful ``work'' \citep{Stoner2000}.  In the case of living systems, free energy enables work that contributes to the organismic regulation with the environment. The better the model fit, the lower the free energy, or in other words, more of the system's resources are being put to effective work in representing the world.
%Evolution obtains progressively more efficient mechanisms for detecting and exploiting free energy deposits, culminating in consciousness that emerges in systems pertaining to the ability to ``integrate various neural networks for coherent consumption of free energy...'' \citep{Annila2017}.

\subsubsection{Predictive Coding}
The ``predictive coding'' hypothesis depicts nervous system that is constantly in the business of predicting probable worldly sources of sensory signals, and adjusting action schemas in order to minimise the error of these predictions (or maximise the precision of these models) \citep{Friston2010,Clark2013}. Neurocomputational evidence suggests that this inferential strategy is supported in the brain by multilevel and hierarchical arrangement of cortical structures, which enable bi-directional cascades of information between levels.  Higher levels of the cortical hierarchy formulate models based on prior experience, which are employed to ``explain away'' sensory signals at lower levels. Lower level signals unaccounted for by higher level predictions are incorporated into higher level structures and strengthen the robustness of the model.  Predictive coding has been likened to a process of ``empirical Bayes,'' whereby prediction errors function to strengthen prior probability distributions of models for future inference \citep{Robbins1964}.  In a typically Bayesian manner, predictive processes appear capable of factoring representations of uncertainty around sensory signals into the predictive model itself \citep{Clark2013}.  The brain's capacity to quantify the uncertainty of any given sensory state facilitates optimal selection between competing predictions pertaining to the same bottom-up sensory signals, judged probabilistically.  At any one moment, an individual has access to multiple hypotheses derived from priors, which compete for the best fit of the sensation, until that process leads to fixation on the best hypothesis for the sensory state.
Binocular rivalry, (for example looking at a necker cube) is an instance in which there is insufficient sensory information available in order to reach fixation on one model over another (see Image ~\ref{fig:neckerCube} \citep{Frith2007}.

\subsubsection{Active Inference}
Importantly, in the case of motor systems, the agents are able to move their sensors in ways that amount to actively seeking or generating the sensory consequences that they (or rather, their brains) expect.  In this way, ``error signals self-suppress, not through neuronally mediated effects, but by eliciting movements that change bottom-up proprioceptive and sensory input''\citep[186]{Clark2013}. Perception, representation, and action functionally and temporally integrate to fulfil an ever evolving set of sub-personal expectations about the state of the world.  Perception, cognition, and action work closely together to minimise sensory prediction errors by selectively sampling, and actively sculpting, the stimulus array.

Thermodynamic cognition encourages a conceptual shift away from individual-centred computational models of information processing (originally inspired by the mechanics of the electronic computer), which tend to render cognition as the final product of a linear sequence of sensory perception, amodal mental representation, and action selection \citep{Lewis2005}.  By contrast, thermodynamic cognition offers a model in which perception, representation, emotion, and action are functionally and temporally integrated in the service of informational processes of free energy minimisation.  Perception, cognition, and action work closely together to minimise sensory prediction errors by selectively sampling, and actively sculpting, the stimulus array.  As I explain in the sections below, thermodynamic cognition offers the most appropriate theoretical framework for analysing the phenomenon of coordination in joint action.


\subsubsection{The degrees of freedom problem and its solutions in human movement}

Bernstein \textcite{Bernstein1967} was the first to point out the astounding computational feat of coordinated physical movement in multi-component living systems.  In the case of human movement for example, an impressive balance is somehow struct between flexibility, precision, and control, whereby hundreds of muscles and joints coordinate to perform many different tasks of everyday life. When we grasp a cup or catch a ball, many individual muscles and joints---each with their degrees of freedom---work together in fine concert.  Bernstein found it unlikely, from a computational perspective, that the central nervous system would be able to finely control all the possible movements (the degrees of freedom) of each single muscle individually to create coherently directed movements---it would be computationally impossible. Rather, he suggested that muscles form flexible function-specific self-organising assemblies by locally coupling and constraining each other’s degrees of freedom, greatly reducing the amount of control needed.

Since this initial observation, the existence of ``functional synergies'' have been identified on multiple levels of behaviour, from brain function \citep{Yufik1998,Sengupta2013}, to interpersonal interactions \citep{Kelso2009,Fusaroli2014} to large-scale human societal dynamics \citep{Nowak2017}.  It is currently understood that functional synergies in human movement systems are regulated by a continuum of processes that span interoceptive action-orientated predictive models on one end, to  basal (lower-cognitive) movement regulation mechanisms that enable direct (i.e., largely extra-neural) coupling with the task-specific environment \citep{Semin2012}.  Across this continuum, the priorities of thermodynamic cognition remain consistent: free energy is minimised by highly precise and flexible (but cognitively expensive and slow) predictive models on one end of the continuum, and by automatic and situated processes that outsource cognitive demands to informational affordances of the physical body and the external environment, on the other end.




%Interoceptive feedback underlies the feeling of grasp, or understanding that accompanies the organization of disparate “representations” into cohesive structures amenable to further operations (mental modelling).



\subsection{The cognitive challenge of joint action}

Both intra-personal and extra-personal human movement both rely on the nervous system’s capacity to anticipate, attend, and adapt to the conditions of the environment \citep{Keller2014}.  In the case of intra-personal human movement, successful coordination of is made possible by direct and habituated coupling of movement control system to the organism's various degrees of freedom.  Predictive models, for example, appear to benefit from direct and privileged access to models  of impending action plans before they are executed (``efferent copies'' in control theory, see \citep{Vesper2011,Pesquita2017}).  By contrast, interpersonal movement coordination poses a much more significant challenge.  Unlike intra-personal coordination, in interpersonal coordination, information about other’s actions can only be acquired from indirect sources \citep{Wilson2005,Wolpert2003}.  The combination of 1) limited reliability of sensory modalities as a source of information about the action of others \citep{Frith2007}, and 2) the informational complexity associated with a cognitive system comprising multiple autonomous agents \citep{Bernstein1967} means that successful coordination in joint action is an inherently difficult and highly improbable cognitive challenge.


\subsection{Cognitive solutions to the problem of joint action}

In the case of human interpersonal movement coordination, it is unlikely that solutions to joint action could be driven solely by central, executive control processes in the nervous systems.  It is similarly unlikely that processes of unconstrained automatic alignment with no higher-level coordination are enable joint action. Rather, joint action is facilitated by a continuum of mechanisms and structured by the specific function of the activity.
When establishing and sustaining interpersonal coordination, participants selectively recruit multiple behaviours and processes, which become more interdependent and constrained by the function of the ongoing activity.  Current research suggests that, faced with the challenges of joint action, humans have devised a number of solutions to interpersonal coordination.  Generally speaking, faculties for joint action appear to involve re-purposing of mechanisms core to succesful intra-personal coordination, such as motor simulation \citep{Vesper2012}, direct action-perception coupling \citep{Novembre2014}, and lower cognitive movement regulation mechanisms \citep{Semin2008}.

Generally speaking, these solutions appear to exist on a continuum, with more (neuro)computationally intensive mechanisms of anticipation, attention, and adaption occupying one end, and extra-neural mechanisms of direct coupling with the components of the movement system, on the other.  Successful joint action relies on the recruitment of a suite of mechanisms from this continuum in order to establish and sustain interpersonal coordination.  Evidence suggests that more thermodynamically efficient solutions to joint action recruit more extra-neural resources, whereas more rudimentary solutions rely on computationally more intensive procedures to reduce uncertainty.




\subsubsection{Interoceptive predictive models of joint action}

One effective way of minimising cognitive free energy in joint action scenarios is to generate interoceptive predictive models specifically tailored for joint action \citep{Graziano2013,Manera2013,Sparenberg2012}.

Vesper textcite{Vesper2010} outlines a minimal architecture for joint action, which contains three components:

\begin{enumerate}
  \item Represent a shared goal, as well as representing one’s own individual contribution to the shared goal.
  \item Apply monitoring and prediction processes to each partner’s actions. This includes monitoring the extent to which shared goals or tasks are being fulfilled while at the same time predicting a partner’s actions.
  \item Facilitate continuous coordination via coordination smoothing, defined as the process of continuously improving one’s prediction of the partner’s action.
\end{enumerate}

\myparagraph{PJAM}

Pesquita and colleagues \textcite{Pesquita2017} argue that best way to account for these minimal requirements is through a hierarchical predictive approach, which they term a ``predictive joint action model'' (PJAM, see figure ~\ref{fig:PJAM})

\begin{figure}[htbp]
  \begin{center}
    \includegraphics{images/PJAM.png}
      \caption{The Predictive Joint Action Model \citep{Pesquita2017}}
        \label{fig:PJAM}
   \end{center}
\end{figure}

PJAM assumes that each participant in a joint-action maintains internal models of themselves, their partners, and the joint task.  is composed of three hierarchical levels of inference: goal representation, action-planning, and sensory routing.

The goal-representation level of PJAM suggests that participants in joint action generate and monitor representations of the shared task. This proposal is formulated from evidence that musicians maintain shared representation of desired unified sound of an ensemble \citep{Keller2008}.  In one study using a neurophysiological measurement that codes unexpected events (ERP), Loehr and colleagues \textcite{Loehr2013} showed that musicians tracked deviations from a desired joint state;  in another study, Loehr and Vesper \textcite{Loehr2016} demonstrated that playing music together generates shared expectations regarding the desired state of joint action, and this process guides future action. Taken together, this evidence suggests that participants in joint action utilise prediction errors arising from lower levels to update representations of shared tasks.
%(Wolpert MOSAIC model doesn't account for updating rep. of shared task)

At the action-planning level of PJAM, pairs of models represent the expected contributions to a joint-action of self and others.  This level of the model is supported by evidence that participants in joint action generate internal models of self and other, known as co-representation, either spontaneously and involuntarily, as in the commonly used social simon task experimental paradigm \citep{Sebanz2003,Atmaca2008}, or more deliberately, as in a coordinated dyadic horizontal jumping task \citep{Vesper2012}.  Studies show that co-representation is modulated by mood \citep[positive or negative affect, see][]{Kuhbandner2010}, self-concept and social orientation \citep{Colzato2012,Colzato2012a}, and group membership \citep{deBruijn2008,Iani2013}.

Action-planning entails, on the highest level, action roles (); on a mid-level, movement trajectories (); and on the lowest level, movement of muscle groups of self and others.  As with goal representation, action-planning for self and others is a predictive process \citep{Flanagan2003}, in which individuals encode motor predictions and resulting errors of others' actions in addition to their own \citep{Vanschie2004,Radke2011}.  Predictive models of self and other action plans appear to be grounded in an individual's own motor simulation processes, such that each participant maintains covert motor activations relating to expected contributions of their partners \citep{Hollander2012}.  There is evidence to suggest that motor simulation of self and other action plans is mediated by the existence of a shared goal between co-actors \citep{Kourtis2010}.  Loehr and Vesper \textcite{Loehr2016} demonstrate that when learning a joint piano piece, musicians are able to better perform the piece together rather than solo, representations about each individual in a social interaction are encoded within the joint context of the interaction. In other words, when we learn a joint task we learn our own role and the impact of other's roles on our own role in the joint action.  In sum, PJAM proposes that models of the self and models of partners can be paired to represent the possible combinations of individual contributions to the joint action. These models transform the desired joint-state signal, descending from the goal representation level, into expectations of the unique motor states of the self and the partner.  Error signals arrive from levels of the model below, with sensory routing.

The sensory routing level receives the inflow of sensory input and compares it to internal model predictions pertaining to each participant's action outcomes.  This comparison serves as a gate for parsing sensory information into their corresponding predictive streams (self or others).
This allows the predictive system to attribute external consequences to each individual’s actions.   For example, two people carrying a table will receive haptic input from the table. This input will be confounded with respect to its source, in that the haptic input itself does not differentiate between the forces that each brother applies to the table. However, the comparison between the haptic input and the separate predictions about one’s own and the other’s action will help feed each predictive cascade of self and other into their respective streams.  As always, deviations between sensory input and sensory predictions (i.e., sensory predictive errors) are fed-back to sensory predictive models in order to continuously improve sensory parsing.

Evidence indicates that accurate predictions of others actions is associated with the dampening of the sensorial experience of joint action outcomes.  Predictions about motor outcomes are used to filter out the sensory feedback produced by the same action \citep{Blakemore1999}.  This proposal is confirmed by evidence that greater sensorial experience occurs when an outcome is unexpected, as oposed to predicted from prior self or other action \citep{Sato2008}.  In addition, attributing sensory consequences to joint action partners is linked to cooperative success \citep{Chaminade2012}, suggesting that finely tuned sensory routing based on predictions of self and other actions could be key to successful coordination.

In real world joint action scenarios such as sport, co-actors interact and rehearse their behaviours to produce a hierarchy of aligned representations, an implicit common ground on which joint action can unfold.  PJAM provides a framework for the function of interoceptive predictive modelling in this process.  Each level of PJAM generates predictions of the information that it expects to be found on the level below.  Continuous comparison between adjacent levels results in error signals that are sent up to optimise subsequent predictions in the level above.  This hierarchical predictive architecture for joint action allows humans to generate accurate yet flexible models for joint action.  The combination of accuracy and flexibility comes at a cost, however, and it is clear that humans also utilise, in addition to interoceptive modelling techniques, a range of more direct and extra-cortical mechanisms in order to establish and maintain joint action with others.


\subsubsection{Driect action-perception links with co-actors}

Mirror system and motor resonance with co-actors:
It can be regarded as a basic link between sender and receiver (Rizzolatti1998) that provides procedural, perceptual, and emotional common ground between individuals (Sebanz2006).


\subsubsection{Extra-neural direct coupling}

There is also evidence of extra-neural solutions to the problem of high cognitive uncertainty joint action problems. Establishing and sustaining direct coupling with resources of the joint action environment, namely other actors and the physical resources of the task-specific environment, can help solve Bernstein's degrees of freedom problem.   In interactive, cooperative tasks, individuals have been found to couple and reciprocally constrain their movements reducing the overall control needed to maintain effective cooperation \citep{Ramenzoni2011,Ramenzoni2012,Riley2011,Schmidt1990}.  Individuals’ behaviours become increasingly interdependent, so that a higher-level structure of the interaction emerges. This kind of emerging organisation has previously been referred to as soft-assembly \citep{Kello2009}: individuals preserve a degree of autonomy, but their behaviour is constrained by the interaction. They can flexibly engage and disengage from it, as well as become part of other soft-assemblies \citep{DeJaegher2010,DiPaolo2012}.

Research into the coordination dynamics of natural joint actions  has shown evidence of dynamic coupling (synchronisation) in joint-action tasks, such as dancing, martial arts, and moving objects like furniture.  In these studies, specific component degrees of freedom are modelled as coupled oscillators (using the HKB model \citep{Haken1985,Kelso1986}, which describes the change in the relative phase between two oscillatory components).  Models are analysed for non-random fluctuations in relative phase over multiple time scales.  This type of synchronisation is said to be of a fractal or semi-fractal organisation, also known as 1/f scaling or ``pink noise'' \citep{Caron2017}. According to Anderson and colleagues \citep{Anderson2012}, 1⁄f scaling is ubiquitous in smooth cognitive activity, and indicates a self-similar structure in the fluctuations that occur over time (within a time series of measurements).
1⁄f scaling indicates that the connections among the cognitive system's components are highly nonlinear \citep{Ding2002,Holden2013,Kello2010,Riley2011,VanOrden2003,VanOrden2005}. Pink noise has been measured beyond dyadic synchronisation, in the analysis of sub-phases of team sports \citep{Passos2014,Duarte2012} and group dancing \citep{Chauvigne2017}.\footnote{1⁄f scaling is temporal long-range dependencies in the fluctuations of a repeatedly measured behaviour or activity. Analogous to spatial fractals, 1⁄f scaling denotes a fractal or self-similar structure in the fluctuations that occur over time. That is, higher frequency, lower amplitude fluctuations are nested within lower frequency, higher amplitude fluctuations as one moves from finer to courser grains of analysis \cites(for a more detailed description see, for example)(){Holden2005}{Kello2009}}



In addition to the pink noise of dynamic coupling in joint action, research has also attempted to capture movement signatures of joint action that can capture the phenomenon of ``group flow'' \citep{Sawyer2006} or being ``in the zone.''  Working within the common dyadic ``mirror game'' paradigm, for example, Noy and colleagues \textcite{Noy2011,Noy2015,Hart2014} have developed an experimental proxy for an optimal state of togetherness in joint action.  ``Co-confident motion'' (CC motion) is canonical movement pattern of synchronised motion characterised by smooth and jitter-less motion, without the typical jitter resulting from reactive control in more commonly encountered leader-follower patterns.  In CC motion, different players appear to shift their basic motion signatures to a movement shape that is altogether different from their individually preferred shapes \citep{Hart2014}. Importantly, the pattern of CC motion shares the same sine wave shape as the optimal solution of the minimum jerk model, a well-known motor control model for rhythmic motion \citep{Hogan2007}. Noy and colleagues suggest that it is possible that during CC motion periods of joint action, two players converge to a canonical pattern stemming from an optimal state of each participant’s motor control system.  The resulting motion may be easier to predict and to agree on. Furthermore, participants appear to use smooth elementary strokes of CC motion as the building blocks for more complex motion \citep{Noy2017}.  This observation raises the possibility that CC motion, a state of alignment in which individual components converge in a transcendent, functional synergy, could set the cognitive foundation for more efficient and effective higher level processes of communication and information transfer \citep[15]{Lerique2016}.

  % Experimental evidence has shown that functional interpersonal synergies facilitate performance of social cognitive or linguistic tasks, such as gaze coordination and turn taking in conversation \citep{Miles2010,Richardson2005,Shockley2009}.  Conversely, being psychologically distanced from another individual can inhibit the emergence of interpersonal synergies \citep{Miles2010}.  The ways in which functional interpersonal synergies facilitate adaptive information transfer between individuals and within groups suggests that psychological mechanisms and cultural practices responsible for generating these synergies could have been subject to cultural evolutionary forces of selection and attraction \citep{Claidiere2014,Mesoudi2016a}.


\subsubsection{More efficient joint action relies on more on direct coupling?}


In studies involving skilled versus non-skilled practitioners in dyadic interactions, it has been shown that more skilled practitioners create stronger dynamical coupling through flexibly modulating their actions with others \citep{Schmidt2011, Caron2017}. These findings are corroborated by other studies that find that professional footballers (versus novice controls) are able to more accurately predict the direction of a kick from another player's body kinematics (\cite{Tomeo2012}, see also \cite{Aglioti2008,Mulligan2016} for similar results with basketball and dart players).

Interestingly, when analysing co-regulation between members of basketball teams, it was shown by Bourbousson \textcite{Bourbousson2015} that more expert teams made fewer mutual adjustments (at the level of the activity that was meaningful for co-actors), suggesting an enhanced capability of expert social systems to achieve and maintain an optimal level of awareness during the unfolding activity, potentially implicating down-regulation of prediction error management processes, and greater reliance on extra-neural couplings with co-actors and the physical environment.


R'Kiouak 2016: ROWING TASK


the team members combined two ways of regulating their joint action throughout the race, namely meaningless regulation and a salient, meaningful regulation of the joint action.

Second, the team members also combined two distinct modes of regulation, inter- and extra- personal.

(a) the extent to which rowers simultaneously experienced salient, meaningful sensations of effectiveness (i.e., effective or detrimental) in their joint action correlated with the extent to which supporting a mechanical signature captured expert-like pattern of team coordination.

(b) the participants spent a large amount of their activity not having a salient, meaningful experience of their joint action.

extra-personal regulation processes might also have underlain the joint action dynamics

Extra-personal regulation has been used to explain the emergence of team coordination patterns while rowers were only regulating their individual coupling to the environment separately. The environment is thus used by individuals to mediate/organize the arrangement of individual activities at each moment of the collective activity. This process differs from inter-personal regulation processes that are grounded on a direct co-regulation of the joint action dynamics itself.
When a rower is involved in an extra-personal regulation and acts on his/her oar, he/she can adjust his/her movements in response to the reaction of the water and the boat information. Both rowers can thus respond similarly, thanks to this mediation. Interestingly, as observed within social insects that act together through environmental mediation (e.g., termites, ants), such a process does not need individual agents to be aware of the collective motion to which they are contributing, which might thus explain the very few instances in the present study where the rowers made salient, meaningful experiences of their joint action.


Joint action was perceived simultaneously as a salient, meaningful experience for only 24,5% of the race under study.

at the pre-reflective level of their activity, the rowers did not pay attention to the effectiveness of their joint action for the remaining 75.5% of the studied period, indicating that the rowers did not make an extensive salient experience of their joint action at the scale of the overall race

In other words, and as labeled in the thematic analysis of the phenomenological data, the rowers were able to coordinate their strokes through experiencing their joint action as “meaningless” during a large part of their crew activity.

along the coordination process under study, some events occurred at the inter-personal level of organisation (i.e., synchronization breakdowns) to which the rowers were sensitive, causing them to exhibit an inter-personal mode of regulation at the level of the activity that was salient and meaningful to them (see


This corroborates that team coordination patterns of movement may occur without a perfectly shared experience about the ongoing joint action (Bourbousson et al., 2011, 2012)


the present study provided further evidence that the full coordination of sense-making activities is not needed to allow for a viable patterned joint action in a natural task, as long as actors are simultaneously involved in co-regulating their collective behavior (Froese and Di Paolo, 2011; Froese et al., 2014a,b).


stigmergic theory of collective behavior (Susi and Ziemke, 2001; Avvenuti et al., 2013) in which holistic phenomena of coordination might be considered as emerging from the behavior–environment coupling.

















\subsubsection{Individual differences}

Personality, social preferences (Keller & Novembre 2015),
Technical competence
Experience



\subsubsection{Culture}

FRAMING:

Joint actions that involve complex sequences and divisions of labour between participants appear to rely heavily on capacities to explicitly signal intention for the assigning of roles, forward planning, and repair of failed coordination \citep{Frith2010}. These ``coordination smoothers'' \citep{Vesper2017} often function to reduce spatial and temporal variation in action by providing a shared spatiotemporal referent for co-alignment of predictions.  Cultural conventions are examples of effective framing devices for joint action.  Depending on the context of the joint action, it could be subject to a pre-existing, mutually recognised power relation typical in the established culture (e.g., favouring hierarchical or egalitarian communication, \citep[see]{Cheon2011}) and the particular situational context (e.g., formal or informal).
Establishing roles, such as leader or follower, also has a similar smoothing effect, and often the affordances in the task environment shape the smoothing strategies available to co-actors \citep{Marsh2009}.


A key insight overlooked by the existing social high account of group exercise and social cohesion, but revealed by the paradigm shift surrounding the social cognition of joint action, is the sensitivity of joint action (or any cognitive process for that matter) to informational affordances provided by various layers of ecological and cultural context.  The cognitive inputs to joint action in real world settings are rarely limited to essentialise components administered in laboratory paradigms. It is known that cognitive processes relevant to joint action are distributed throughout brains, bodies, and the physical environment of the ecological niche in which it is situated.  There is evidence to suggest that joint action co-participants rely on a series ``frames of reference'' for joint action execution \citep{Ray2018}, and it appears that social interaction functions best in situations where there is a snug fit between individuals' implicit cultural expectations and explicit rules for engagement \citep{Vollan2017}.  Indeed, widespread attention to the ``reproducibility crisis'' in social science \citep{Earp2015,Rathmacher2017}, suggests that even the most controlled spaces and procedures of experimental studies involving human subjects are laden with unquantified and often culturally specific cognitive affordances that constrain and enable quantified outcomes.
%The experimental approach to science requires re-evaluation in light of an emerging paradigm emphasising distribution of cognitive processes throughout an array of coordinated informational affordances.

Discussed as part of the theoretical framework outlined in Chapter 2, shared cultural knowledge can act as a ``coordination smoother'' \citep{Vesper2017} for joint action, enhancing the effectiveness and efficiency of joint action between co-participants who share a similar informational framework.  In the predictive coding paradigm, cultural habits and frames of reference act as ``hyper-priors'' that set the macro-contextual coordinates for joint action\citep{Clark2013}.  Contextual affordances for joint action appear to be dictated by processes operating at multiple conceptual levels---from the micro-level predictive processes associated with movement action and perception, to the macro-level predictive frames offered by specific cultural and contextual niches---interact in complex processes of reciprocal causation to shape joint action (SOURCE).  Conceptualisation of the causal complexity of cognitive processes relevant to joint action in this way echoes a broader reconceptualisation of the causal complexity associated with change on an evolutionary timescale, which recognises that human behavioural phenomena is the result of a number of biological, cognitive, and ecological mechanisms that interact via reciprocal feedback loops spanning varying scales of time and space \citep{Fuentes2015}.











\subsection{Social connection through joint action}

The link between interpersonal coordination and social bonding has been addressed in the behavioural mimicry and synchrony literatures \citep[e.g.,][]{Wheatley2012,Launay2016,Mogan2017}, but there is less substantive evidence in relation to dynamic interpersonal coordination in natural joint action settings such as those found in group exercise contexts \citep{Marsh2009,Miles2009,Lumsden2012}.  More recently, however, analysis of dynamic coupling of co-actors in joint action scenarios reveals that synchronised movement implicates an array of implicit and pre-perceptual cognitive processes of alignment and prediction error minimisation \citep{Schmidt2011}, which, in addition to more explicit forms of communication, could be central to the generation of feelings of self-other merging, self-other distinction, and perceived reliability and trust associated with social bonding \citep{Marsh2009}.





\subsubsection{Emotion in thermodynamic cognition}

cortical processes of prediction error management appear to be mediated by the activity of the dopaminergic system \citep{Schultz2016}, while subcortical neuromodulatory systems, such as those responsible for producing norepinephrine, acetylcholine, and endogenous opioids, appear to be involved in attuning cortical processing to signals from the body and environment that are important for survival \citep{Lewis2005}.  There is now evidence to suggest that complex cognitive processes (traditionally understood to be confined to cortical regions) and subcortical neuromodulatory systems (traditionally understood to be responsible only for affective response and exogenous to the brain's inferential processes) work in a loop of reciprocal interaction in order to enhance processes of error management \citep{Damasio1994,Lewis2005,Miller2017,Barrett2017}.
Emotions can in this sense be understood more as superordinate programs for regulating disparate subordinate cognitive modules for the purposes of global coordination with the environment \citep{Cosmides2000}.  By collapsing the common neurocognitive distinction between cortical and subcortical processes, PP, in addition to unifying perception, representation, and movement in a common theoretical framework, also helps integrate the role of affective processes into ``active inference.''

The distinction between cognition and emotion is supported by the idea that segregated brain areas implement cognitive and emotional functions and that there are two independent processing routes, one cognitive/controlled and one emotional/automatic, which usually compete (but also occasionally cooperate) to control behaviour \citep{Kahneman2003}.  However useful this ``dual-systems'' view has been thus far in cognitive science, prevailing evidence concerning the complexity of functional integration and segregation of brain processes challenges the cognitive-emotional distinction \citep{Pessoa2013}.  The emerging view is not only that cognition interacts with emotion at many levels, but that in many respects they are functionally integrated and continuously impact each other's processing.




Surprisal in this inferential cognitive process is simply due to a prediction error that is subsequently fed-back to the cognitive system for the purposes of tuning and enhancing models of future action and inference\citep{Clark2013}.



It has been hypothesised that affective responses from to expectation violation


surprise is a basal emotion that can carry both positive or negative valences \citep{Ortony1990}.







\subsubsection{Prediction affect: positive violation of expectations}




We have suggested that affect serves as feedback on our predictions, reflecting their accuracy and regulating them so that confirmed predictions are more likely to be used again (Chetverikov, 2014; Chetverikov & Kristjansson, 2015). Furthermore, if predictions are confirmed (lowprediction error), feed- back is weightedwith inverse prior probabilities of predictions, so that more probable predictions receive less positive feedback. In other words, confirmation of more probable predictions yields less positive feedback than confirmed less-probable predictions. Notably, within this framework there is no need to invoke additional concepts, such as values or rewards, to explain the relationship between affect and pre- dictions. Affect represents a distinct dimension in experience: in addi- tion to our “best hypothesis” about the world, people experience a feeling of howgood this hypothesis actually is. The literature describing affect fromthis perspective has largely been limited to the perception of art (Salimpoor, Zald, Zatorre,Dagher, &Mcintosh, 2014; van de Cruys & Wagemans, 2011).Wefill this gapbyproviding amore generalperspec- tivewithin a predictive coding framework.









A recent study used a finger tapping paradigm (in which participants coordinated taps with a virtual partner prone to drift in tap tempo) to address the relationship between locus of control and temporal adaptation (error correction) \citep{Fairhurst2014}.  Individuals with an internal locus of control (who attribute the cause of events to their own actions) engaged in less phase correction than individuals with an external locus of control (who attribute events to external factors). This result indicates that individual variation may lead to certain default patterns of interpersonal movement coordination (such the default leader-follower dynamic).





\subsubsection{Team Click}









\subsection{A novel theory of social bonding through joint action}


\subsubsection{Predictions}
