\documentclass[12pt]{report}
\usepackage[utf8]{inputenc}
\usepackage{graphicx}
\graphicspath{{images/}}

\usepackage[english]{babel}
\usepackage[backend=bibtex,style=apa,natbib=true]{biblatex}

\addbibresource{referencesBibtex.bib}



\begin{document}

\maketitle{}


2. Specific Investigation: Joint-Action and Social Bonding between professional Chinese Athletes

  2.1 Rugby in China

    2.1.0 Overview

  In this dissertation I focus my investigation on one contemporary type of group exercise—rugby union—in one geographical location-The People's Republic of China. I conducted 10 months of ethnographic research with the Beijing Men's Provincial Rugby Team, and two pseudo-experimental field studies for which I sampled from a broader population of professional rugby players from 9 different provinces.  My ethnographic observations were spread out over three separate trips to China, beginning in August 2015 and culminating at the Chinese National Games held in Tianjin, August 18-20 2017.

	Between August 2015 - March 2016, I spent seven months in Beijing conducting participant observation, unstructured and semi-structured interviews, and informal surveys with the Beijing Men's Rugby Team. In July - August 2016 I returned to China for a further two months, during which time I continued ethnographic observations of the Beijing team, while also conducting two pseudo-experimental field studies in Hebei province and Shandong provinces. Finally, I spent one month in Beijing and Tianjin between August - September 2017 during which time I conducted follow-up ethnographic analysis with the Beijing Men's Rugby Team in the form of structured interviews.

    2.1.1 Rugby in China

        2.1.1.1 Rugby
  Rugby and China are two words not usually heard mentioned in the same breath.

  Rugby union is an Anglo-European interactive team sport that originated in the Public School system in the United Kingdom during the mid 1800s, and subsequently became popular in British Commonwealth countries owing to the processes of colonialism.

  Rugby 7s is an extremely physiologically exertive and interactive team sport, that requires high levels of of interdependence between team members due to the uncertainty and complexity.


        2.1.1.2 China (Sport)

        Team sports have never quite been a ``snug fit'' for China

        Thanks to the persisting Olympic logic of the Chinese sports system, rugby (along with golf) is the most recent sport to be inducted into the bosom of the state-funded sports system.

        Not expecting to find traditional discourses of social cohesion and sport, ones that are more recognisable on the rugby pitches and in the boathouses of Oxford or any high school American Football movie made in the 1990s.  Instead, I was looking for evidence of the predication that the cognition of collective movement relates in significant ways to social processes of affiliation, social proximity, and liking.  In other words, I am looking for signs of a relationship between collective movement and social resonance (no matter what specific cultural  formations that social social energy (for want of a less mystical word) is resonating through.

        Relational and categorial processes of self construal and group formation










\section{Introdcution}




\subsection{Ethnographic Setting}

Physical culture and Chinese modern history
The history of sport and exercise in China's modern transformation is, in many fascinating ways, emblematic of that history itself.  The introduction in the late 19th Century of a new ethics of group membership centred around the activities of the nation-state, required importation en masse of novel linguistic, cultural, and social categories and practices.  Throughout the 20th Century, physical culture (tiyu 体育) became a (if not the) primary pedagogical vehicle for fostering an explicit link between the strength of the physical body and the strength of the Chinese nation (Morris 2004: 32; Brownell 1995: 49).  From the initial embrace of the Olympic Games by an urban Chinese elite at the turn of the century, all the way through to Beijing’s eventual hosting of the Olympics in 2008, physical culture has provided the means through which new and normative ways of thinking and behaving have been publicly displayed and transmitted.  Inherent in this process has been the tension and interaction between imported categorical modes of group membership (citizen of nation-state) with indigenous understandings of social identity which centred on intragroup relational processes rooted in Confucian, rural, and dynastic cultural traditions.  Physical culture in China choreographs—perhaps more explicitly than any other facet of contemporary Chinese life—the interaction between imported and traditional modes of group membership at the psychological heart of Chinese processes of group cohesion.

China’s rich indigenous physical culture has merged with modern waves of cultural import beginning in the 18th Century. Modern sport and exercise was first introduced to China as part of the “New Culture Movement” at the start of the 20th Century—a movement in which student intellectuals problematised traditional Daoist and Confucian understandings of the body as “passive” and “feudal” and suggested that a new, “active” and competitive body, should be realised (Ge and G. 2005).  In this sense, the “weak” Chinese body of the feudal past was identified as the cause of the nation’s weakness as it grappled with colonialism in the early 20th century, thus calling for the production of a strong, masculine and active body as essential for China’s future (Morris, 2004).  As such, towards the end of the 19th century, traditional practices of self-cultivation such as the Daoist notion of “cultivating life,” which included traditional martial practices of taichi and qigong, were denounced by reformers in favour of a variety of imported physical regimes and an associated philosophy of “training the body” (锻炼身体 duanlian shenti).  The first recorded organised practice of non-Chinese techniques of the body by Chinese bodies, was when the Chinese military adopted calisthenics and military drills (体操 ticao) in 1895 in an attempt to modernise practices in line with the German and Japanese armies that occupied Chinese territory (Knuttgen, Ma and Wu 1990: viii).  Such techniques were soon popularised within elite intellectual communities as pedagogical tools designed to foster an explicit link between the strength of the physical body and the strength of the Chinese nation (Morris 2004: 32; Brownell 1995: 49).

The urban elite began to embrace a range of Anglo-American competitive sports that were promoted by Western missionaries, in particular the YMCA (Young Men’s Christian Association).  The YMCA’s mission of prescribing “Christian manhood” for an emerging Chinese youth, including the propagation values of fair play, sportsmanship, masculinity and internationalism, gelled with the values of a nationally (and internationally) motivated urban elite (Morris 2004: 240).  A significant aspect of competitive sports was the fact that they provided a public spectacle, in the form of “games meets” in which the performance of emerging national and international political identities could take place.  As early as 1908, the Chinese sport community enshrined the Modern Olympic Games as the pinnacle of participation in an international community of nations, and as such, the quadrennial global ritual has since preoccupied a collective sporting consciousness, and a Chinese national consciousness more broadly (Burnett 2009; Barme 2009; Brownell 2008b: 19; Morris 2004: 3; Xu 2008).

Rugby in China
Rugby Union (hereafter rugby) is an interactional team sport played on a rectangular field (100m x 70m), by two teams, usually of 15 players, who physically contest possession of an egg-shaped ball that can be used to score points (IRB, 2014).  Rugby is a highly interactional and physiologically demanding at all levels at which the game is currently played, requiring players to participate in frequent bouts of intense activity such as sprinting, physical collisions, and tackles separated by short bouts of low intensity activity such as walking and jogging.  At an elite level in particular, these physiological costs are amplified.  Rugby sevens is a modified version of the conventional 15-a-side game involving only 7-a-side that has grown in popularity more recently, particularly since its introduction to the Olympics for the 2016 Rio de Janeiro Olympics.

Rugby has been a professional sport in China for six years (in the form of the Olympic event rugby sevens), before which it had existed as a non-professional university sport for 20 years, first established in 1990 at the Chinese Agricultural University, Beijing).   Rugby is part of a large collection of “cold-gate” sports (lengmen xiangmu 冷门项目) in China, with a relatively small participation base compared to other interactive team sports like basketball or football.  However, due to the persistent Olympic focus of the Soviet-modelled Chinese competitive professional sports system (juguo tizhi 举国体制), rugby’s recently acquired Olympic status in the form of rugby sevens means that it has now been inducted into the state-sponsored competitive sports system and is one of 33 sports featured in the all important quadrennial National Games.  Ten of China’s collection of 32 provinces and municipalities that participate in the National Games have full time men’s and women’s rugby programs.

While football and basketball have matured as standalone market-based professional industries, most other sports in China (i.e. all other Olympic events, including rugby) exist primarily due to the support of the enormous state-sponsored national provincial sport system.  Whereas the commercial basketball and football industries might offer a small percentage of prospective athletes incentives of fame and fortune, the benefits of a state-sponsored sports programs like rugby are more modest but also more stable.  Chinese children gravitate towards sporting careers primarily due to potential life-course opportunities such as access to tertiary education and post-athletic career employment (in the sports industry).  The extent to which an athlete is able to maximize these potential benefits depends on the strength of an athlete’s results (chengji 成绩).  In Olympic sports, the most important measure of a province’s success in state-sponsored Chinese sporting terms is the National Games, a quadrennial multi-sport event hosted on rotation by provincial capital cities (Hong & Hua, 2002).  The amount of funding a province and its provincial sporting institutes and programs receive is decided to a large extent by results at the national games.

Beijing men’s rugby team
I focus my ethnography on the Beijing provincial men’s rugby team (n=26, avg. age =21.3, range 17-27, SD = 2.96).  The Beijing provincial men’s and women’s rugby programs are based at the Xiannongtan Sports Institute in Beijing (one of Beijing’s four major sports institutes, and home to seven different full-time sports programs, hereafter Xiannongtan).  These athletes represent Beijing at a provincial level, playing against other provinces in annual tournaments, and every four years at the all-important National Games.  Five athletes have previously represented China in international rugby sevens tournaments.  There is also a Beijing women’s rugby team at Xiannongtan, but I was not able to follow both teams closely enough to perform adequate ethnographic research.

Members of the Beijing men’s rugby sevens team are either already-contracted (n=13) or aspiring professional athletes (n=13) who live and train 6 days a week at Xiannongtan and occasionally attend university or high school classes as part of their ongoing—what could only be called “part-time”—education.  A head coach and assistant coach look after the day to day organisation of team schedules and training, and these two coaches are assisted by a further two player-coaches, who are in the gradual process of moving from athlete to coach status.  One of Xiannongtan’s four principals is responsible for the management and administration of the rugby program, and is occasionally present at team meetings and national competitions.

Athletes are all from relatively modest socio-economic backgrounds, many hailing from suburban and rural areas of northern China (Shandong (11), Beijing (6), Jiangsu (3), Liaoning (2), Hebei (2), Anhui (1), Henan (1), Heilongjiang (1)).  The squad consists of 10 fully contracted senior athletes (xieyi 协议), three provisionally-contracted athletes (shixun 试训), and six student-athletes (erjiban 二级班)—who do not receive a salary but receive training, food, board, and educational support.  The remaining athletes (7) are classed as “athletes in training” (jixun 集训) and are effectively on trial until they either show promise or withdraw from the squad either voluntarily or upon suggestion by the head coach.  Provided that they meet the relevant academic and athletic requirements, contracted and student athletes are able to attend the Beijing Sports University—considered to be the country’s most prestigious sports university and one of China’s “top brand universities” (mingpai daxue 名牌大学).

The average rugby training age (years spent playing rugby) of Beijing athletes is 3.12 years (range = 0.16 – 10 years).  Contracted senior athletes have trained for an average of 5.4 years (average age = 24.3 years), whereas the average training age of junior non-contracted athletes is only 1.7 years (average age = 19.3 years).  Over half of the athletes have a background in other sports (15 athletes from track and field, one from football, one from basketball), usually beginning part-time or full-time physical training at the age of 11-13.  Those transferred to rugby did so either at the beginning of senior high school (16 years) or at university age (18yrs).  The rest of the group had no particular sporting background before starting at Xiannongtan, and were scouted by school athletics coaches based on their basic athletic attributes (running speed, strength, coordination, and potential for physical growth).  Of the 26 athletes in the squad, three junior athletes who were part of the squad when I arrived in September 2015 have now left, and three new athletes have arrived.

Training and playing schedule
Every year between April and September there are five national tournaments held in different locations across the country.  October – March constitutes the off- and pre-seasons for these yearly competitions, during which time teams travel to domestic or international training locations depending on amount of program funding and training strategy.  This year, before the change in coaching team at the end of December, plans were to travel to Yunnan in the new year province for altitude training (January) before moving to sea-level somewhere in the south (February/March).  Following the leadership change, the team will not leave Beijing until after Chinese New Year (15th February), which means that training during this period has been influenced by Beijing’s cold winter weather and air pollution.

Below is a table of a typical weekly training schedule for this off-season period. A typical week consists of 10x2.5hr training sessions, three of which are strength and conditioning sessions, seven of which are on-field rugby sessions.  In addition, two one hour evening skills sessions are also added for junior athletes to hone their basic skills of passing, catching, and game-play.

	Monday	Tuesday	Wednesday	Thursday	Friday	Saturday	Sunday
06:00	Training
09:00		Training	Training	Training	Training	Training
15:00	Training	Training		Training	Training
19:00		Training		Training

Athletes live full-time on campus at XNT in dormitory accommodation (usually 3 athletes per ensuite room), and are permitted leave on the weekend after the conclusion of Saturday morning training.  Athletes with family in Beijing usually take this leave, while the remaining athletes spend weekends at XNT.  Athletes break at the end of season (September) for two weeks, and occasionally around Chinese New Year for 7-10 days, as long as this does not interrupt pre-season training plans.



2.2 Ethnographic Observations

		Indeed found culturally specific variation in processes of group formation and self-construal
			- Relational rather than categorical self construal - team was an awkward notion
			- Complaints about Chinese athletes' ability to work as a team, and commit to rugby, described as a ``lack of belief"
					- Chinese people are too selfish, cunning, and Chinese society is too ``complex''
					- Performance-related anxiety, particularly in younger cohort
					- Agency over team from older set
					- Exhilaration surrounding processes of technical competence skill acquisition (Interviews)
					- Moqi, tuandui peihe, qichang, etc

			E: The performance anxiety and the exhilaration concerning team click was a sign of a smoking gun regarding the significance of coordination dynamics to social cognition of affiliation and bonding.  These phenomenological experiences emerge from beneath the surface of overt cultural discourses around group membership




5.	Preliminary Ethnographic Findings
My ethnographic data include unstructured and semi-structured interviews with athletes and coaches (yet to be analysed in-depth), general and activity-specific surveys, and field notes based on participant observation of daily activities of the team.

Evidence of a predominant relational mode of group membership
I predict that professional Chinese rugby players understand group membership less as a function of categorical equivalence between abstract concepts of the self and the in-group, and more in terms of a relational mode of group membership, which entails varying levels of commitment to, and harmonisation of specific relationships that lead to self-enhancement.

A categorical mode of group membership is prominent in official team discourse: institute principals, coaches, and senior players often refer to the importance of selfless individual devotion to the “team.”  The activation of this categorical “team” consciousness, however, appears to be hopeful or aspirational, and forms part of an attempt to learn the foreign physical and social skills required for success in rugby.  In addition, attempts to promote this “team consciousness” (yishi意识) often appear to require the deployment of indigenous relational metaphors and logic: coaches and officials often refer to “the team is a big family,” (我们团队是一大家庭), junior players refer to older players as elder brothers and so on,  almost to disguise the egalitarian pretense that these abstract categories demand.  Indeed, when I probe team officials, coaches, and senior players in private conversations, about these discourses of categorical group membership, all, without fail, acknowledge the team’s and China’s general psychological deficiency in this, explaining that they are not behaviours produced by the Chinese “system” (tizhi 体制).  Instead, it is claimed that team spirit, team-directed individual initiative, and intrinsic motivation are qualities cultivated by the educational and sporting systems of “the West.”

In brief, while athletes appear able to switch frames to a categorical consciousness in order to process the technical and cultural demands of the team sport of rugby (J. Liu et al., 2009), a relational social identity still dictates the behaviours and dispositions of this process.  While athletes acknowledge and aspire to categorical modes of group membership (identification between categories of “self” and “team”) emblematic of modern Western sports such as rugby, adherence to the costly practices of rugby appear to be motivated by a more predominant commitment to strengthening and harmonising a network of hierarchically structured relationships for the ultimate purpose of individual-enhancement.  During interviews, athletes ranked individual motivations for commitment to rugby (family (1st), education opportunities (2nd), earning respect of others (3rd)), before motivations based on the team (teammates (6th)), or enjoyment of the sport (8th).  In addition, there is distinct focus by senior members (coaches and players) on individual- (and not team-) centred commitment and discipline.  Individuals are subject to monetary fines for relatively minor discipline-related transgressions (for example, catching a common cold). In addition, I observe a pattern of technical instruction whereby coaches and senior players publicly single-out and scrutinise the technique of individuals for the benefit of the group, rather than for example making group-level generalisations about technical deficiencies.   While an awareness of the importance of "team cohesion,”(tuanjiue 团结) "team spirit” (tuandui jingshen 团队精神), and team-motivated individual initiative (zijue 自觉) is referred to aspirationally by coaches, senior players, and team officials at official meetings and team events, the same individuals admit to me in private conversations that the Chinese system of group relations does not produce innate motivation for self-less team contribution, and that Chinese athletes lack the essential characteristics of initiative and spirit that Western athletes appear to demonstrate.


Support for central hypothesis: exercise-induced arousal generates social bonds moderated by athletic performance.

In addition to interviews and participant observation, I conducted a number of surveys designed to understand athletes’ general and specific experiences group membership.  I conducted surveys following three training sessions: a session in which athletes (predominantly junior athletes) ran an aerobic fitness test involving straight-line running shuttle-running at and above the aerobic threshold (“Beep Test”), and two training sessions involving internal game-like scenarios.   In addition, I asked athletes about their general experiences of team membership agency over the team (weak-strong), role in the team (central-marginal), individual performance (weak-strong), team performance (weak-strong), training intensity (qiangdu强度)(light-heavy) at two three-month intervals (T1 & T2).  Selected results are collated in the table below (All responses are based ong 7-point Likert).  I divide athletes into junior and senior categories based on training age (junior athletes 0-2 years; senior athletes 2-10yrs).








Athletes (26)	Avg. performance-related anxiety	Agency over the team 	Role in the team 	Self Performance	Team performance	Training intensity
	Beep Test	Internal Games 1	Internal Games 2	T1	T2	T1	T2	T1	T2	T1	T2	T1	T2
Junior (13)	3.4	5.1	4.2	2.3	1.7	2.8	2.9	3.6	3.6	4	4.5	5.1	6.2
Senior (13)	n/a	3.2	3.25	4.5	4.6	4	3.9	3.6	3.6	3.6	3.9	4	4.8
(Please note that these figures are raw and are yet to be analysed statistically).

First, junior athletes report higher performance-related anxiety for the two game-like training scenarios (5.1 & 4.2) compared to the Beep Test (3.4), indicating that performance-related anxiety increases for junior athletes when training demands real-time performance of rugby-related skills.  This result is confirmed in interview responses, with junior athletets emphasizing anxieties related to performance in game-like scenarios compared to fitness-only training sessions.  These observations support my hypothesis that the relationship between exercise-induced arousal and social bonding is moderated by athletic performance.

On average, junior athletes (5.1 & 4.2) appear to exhibit higher performance-related anxiety than senior athletes (3.2 & 3.25), potentially supporting the prediction that less-competent junior players experience more “ethical dissonance” (however performance-related anxiety does not measure dissonance directly).  In line with the prediction that junior athletes would exhibit more pro-sociality than senior athletes, junior athletes (4 & 4.5) reported stronger overall team performance than senior athletes (3.6 & 3.9), whereas senior athletes (4.5 & 4.6) reported much higher feelings of agency over the team than junior athletes (2.3 & 1.7).  Similarly, senior athletes (4 & 3.9) reported a more central role in the team than junior athletes (2.8 & 2.9).  These findings are supported by interview responses, in which junior athletes pay more attention to the details team structures and relationships, whereas senior athletes devote more time to talking about themselves in relation to the team (figures to come).
Interestingly, junior athletes reported higher training intensity (5.1 & 6.2) than senior athletes (4 & 4.8), indicating less habituation to the psychophysiological stress associated with full-time professional rugby training.


\subsection{Discussion of Preliminary Results}


Competence,



Return to FLow





\printbibliography


\end{document}
