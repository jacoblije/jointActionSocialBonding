'
\begin{savequote}[8cm]

  It takes two to know one.

  \qauthor{--- Gregory Bateson}

\end{savequote}









\chapter{\label{chap:intro}Introduction}



\minitoc





                                          \begin{CJK}{UTF8}{gbsn}

\section{Group exercise as carnal mystery\label{sect:adrian}}


On my first night in Beijing, Adrian, Kai, and I sat in the upstairs area of the Korean BBQ restaurant in a quiet willow lined street just inside Beijing's East 4th Ring Road.  Adrian was the host of the dinner, and so he naturally held the floor in conversation while the three of us waited for his colleague Mr Shi to arrive.

Adrian was a veritable elder of Chinese rugby.  He had been captain of the second class of rugby players to graduate from the Chinese Agricultural University (CAU), the home of China's first official rugby union program, established in 1990.  Kai---also a CAU graduate (2007)---was a close friend of mine of many years, and had invited me to join at the dinner soon after I touched down earlier that day.  I was eager to catch up with old friends as well as begin my fieldwork, and so, despite my jet lag, I accepted the invitation.

Adrian reminisced fondly about his time playing rugby at CAU, as well as his time after graduation playing with the Beijing Devils, a rugby club in Beijing whose members were predominantly expats.  He assured us that rugby in China was, in those days, fun and free-spirited.  Not like today, now that rugby has become a professional program in the state-sponsored sport system (owing to the Olympic status of the modified form of the game, rugby sevens, see Chapter~\ref{chap:researchSetting}).  Adrian talked about going on a rugby tour to the UK with the Beijing Devils:  ``Everyone only just scraped together the money to go on tour.  We all paid our own way. Sometimes you'd get a bit of help from someone or whatever. We did it because we loved the game, not for any other reason.''   Kai and I listened intently.  All of a sudden I realised that this conversation could be relevant, so I started taking notes.

When Mr Shi finally arrived, Adrian continued his nostalgic story telling mode, but naturally shifted his target audience from Kai and me to Mr Shi---his guest who knew very little about rugby.  Adrian began to describe in rich detail the experience of camaraderie between he and his Beijing Devils team mates when they participated in overseas rugby tour.  At one point Adrian interrupted his own story to make an explanatory aside directed at Mr Shi, accommodating for the fact that Mr Shi was relatively unacquainted with the sport: ``This sport, rugby: it's actually very mysterious. If you haven't played it yourself you might not know this type of feeling,'' (英式橄榄球这个项目其实特别神秘,没玩过的话您可能不知道这种感觉) Adrian respectfully suggested to Mr Shi.  ``Because rugby, you know, you're all on the field together, there's body contact...'' (因为英式橄榄球么,大家在场上有身体接触) he paused to find the right phrasing,  ``It's a very \textit{carnal} type of feeling.'' (是一种``肉''的感觉) His attempts to enrich his communication by gesticulating had led him to have both of his hands clenched as fists in front of him like they were cradling a rugby ball or gripping the steering wheel of a car---a lit cigarette smouldering between the index and middle finger of his right hand.  Adrian concluded by looking into the distance and repeating: ``Its very mysterious.'' (特别神秘) He shook his head as if baffled and finally released his clenched fists to dab the ash from his cigarette into the ashtray in front of him.  After taking another drag from his cigarette he finally added: ``So it means this rugby circle here in China is very tight...'' (所以在橄榄球这个圈子特别亲)---a short pause for another dab of his cigarette--- ``...but it doesn't mean that this circle is not also not also complete chaos!'' (但这不是说这个圈儿也不乱!)
  \footnote{Circle (\textit{quanzi} 圈子) is a common colloquial way to refer to a social group or community of people in modern standard Chinese.}
The wisdom of Adrians's final punchline was confirmed with a knowing chuckle from all of us, including Mr Shi. Adrian concluded his performance silently, by taking a long, satisfying drag of his cigarette and looking off into the most distant corner of the restaurant.

I was captivated---but also somewhat surprised---by Adrian's monologue.  I was not expecting, so early into my fieldwork, to happen upon a declaration in which a link between the carnal (\textit{rou} 肉) or visceral sensations associated with on-field joint action, and social processes of interpersonal affiliation (\textit{qin} 亲) and group cohesion of the rugby community (\textit{quanzi} 圈子) was so explicitly and spontaneously emphasised.  It was clear that rugby's visceral dimension continued to capture Adrian emotionally; some 10 years after he had finished playing rugby his fists still clenched with energy, and his head still shook with amazement.

I was also intrigued that Adrian cited the source of his emotional capture as at once both very specific (derived from playing together with others on the field) and, at the same time, ultimately ``mysterious.''  The aim of my research is to contribute to an explanation the human behavioural phenomenon of group exercise in terms of its social, evolutionary, cognitive, and physiological causal processes and dynamics.  In essence, the aim is to somehow move from mystery to scientific mechanism.  At this first dinner in Beijing, Adrian's comments both captured the phenomenological mystery of group exercise, and pointed me in the direction of the underlying explanatory mechanisms.

Needless to say, I left that first dinner eager to investigate the sources of Adrian's experience of mysterious carnality and social connection in rugby's joint action.  My next stop, on Monday morning, was the Temple of God of Agriculture Sports Institute, where I began ethnographic research with the Beijing Provincial rugby program.


                            \begin{center}
                              * * *
                            \end{center}





\section{Group exercise as evolutionary puzzle}
Competitive team sports, whirling Sufi dervishes, late-night electronic music raves, Masi ceremonial dances, or the fitness cults of Cross-fit and Soul Cycle---endless examples can be plucked from across cultures and throughout time to exemplify the human compulsion to come together and move together.  How can we explain the prevalence of these activities in the human record?  In this dissertation I contribute to a scientific understanding of ``group exercise''---defined herein as physiologically exertive and socially coordinated movement---by way of a focussed study of the social cognition of joint action among professional Chinese rugby players.

Because physical movement is a metabolically expensive task, it is justifiable from an evolutionary standpoint only if the benefits somehow outweigh the costs.  Using this basic calculus, it is easy to imagine how group exercise would have served important survival functions in our ancestral past.  Activities involving group exercise such as hunting, travel, communication, and defence all appear to confer immediate and obvious benefits to individuals and groups \citep{Sands2010}.

In more recent domains of human history, however, the task of explaining the persistent recurrence of group exercise is more complicated.  At least since the late Pleistocene era (approx. 500ka), and particularly since the Holocene transition from hunter-gatherer to agricultural (approx. 11ka), and later industrial and post-industrial societies, group exercise can be identified in shared cultural practices as varied as religion, organised warfare, music, dance, play, and sport.  But unlike group hunting or defence, the fitness-relevant benefits of group exercise in cultural practices such as sport, music, or dance are not always as obvious.  On the contrary, many of these activities appear on the face of it to entail extreme time and energy costs for very little immediate reward.

Thus, the prevalence of group exercise in a diverse array of shared cultural practices in the more recent human record presents an evolutionary puzzle.  A solution to this puzzle requires a more nuanced calculus that incorporates an appreciation of humans' species-unique evolutionary trajectory, defined by increasingly complex cognitive and cultural capacities, including technical manipulation of extra-somatic materials and ecologies; advanced theory of mind; and information-rich, malleable, and scalable communication systems \citep{Roepstorff2010,Clark2015,Fuentes2016}.  A theory capable of satisfactorily explaining group exercise within humans' distinctive evolutionary parameters is yet to be fully formulated \citep{Cohen2017}.

As Adrian's monologue demonstrates, the experience of group exercise appears to be at once a physical, emotional, and social phenomenon.  This coalescence of factors in experiences of group exercise may shed light on the mechanisms through which humans have achieved social cohesion.  In  this dissertation, I advance existing cognitive and evolutionary understandings of group exercise by formulating and testing a novel theoretical relationship between group exercise and social bonding---the glue of group cohesion.  Existing theories of group exercise and social bonding do not yet satisfactorily account for the variation in, and complexity of, interpersonal movement coordination common to many real-world settings of group exercise.  I concentrate in particular on the proximate cognitive mechanisms  ``joint action,'' defined herein as any form of social interaction whereby two or more individuals coordinate their actions in space and time to bring about a change in the environment \citep{Sebanz2006}.  I draw attention to the social and psychological effects known to occur when active, in-the-moment, and on-line joint action functions successfully---i.e., when joint action ``clicks'' between co-actors.  I propose the phenomenon of ``team click'' as a construct that captures the phenomenology of optimal performance in dynamic joint action.  More importantly, I propose team click as a psychological construct that can help explain a hitherto under-appreciated causal pathway between joint action and social bonding.  Team click offers a vehicle through which the various interlocking dimensions of experience in group exercise can be analysed in terms of their cognitive and evolutionary significance.

%As I demonstrate in this introduction, novel theoretical synthesis---in addition to empirical research---is required in order to more fully align scientific explanations for group exercise with their mysterious, carnal, and embodied (i.e., subjective) dimensions.

In the chapters that follow, I formulate and test a novel theory of joint action and social bonding through a series of three empirical studies with professional Chinese rugby players.  These studies include 1) an ethnographic study of the Beijing men's rugby team (Chapters~\ref{chap:ethnoSetting}\nobreakdash~\ref{chap:ethnoResults}), 2) an \textit{in situ} survey study of professional Chinese rugby players during a National Rugby Tournament (Chapter~\ref{chap:tournamentSurvey}), and 3) a controlled field experiment with a sample of professional Chinese rugby players across two of China's provincial programs (Chapter~\ref{chap:trainingExperiment}).  In each of these studies I find evidence in support of the predictions of this dissertation.  Invariably, more positive perceptions of joint action predict higher levels of perceived team click; higher levels of team click predict higher levels of social bonding, and in some instances, team click fully or partially mediates a direct positive relationship between  joint action and social bonding.  Each study progresses from its predecessor in a step-wise manner: I start with broad and rich ethnographic observation and analysis, and build on these observations towards a more quantitative verification of hypothesised mechanisms.  Findings from each study offer initial substantiation of a novel theory of joint action and social bonding in group exercise.  In so doing, this dissertation sheds new light light on questions of human sociality and evolution.

In this chapter, I review existing theories of group exercise and social cohesion (Section~\ref{sect:existingGESoCo}), and point to empirical and theoretical knowledge gaps that require attention (Section~\ref{sect:empKnowGaps}).  In particular, I identify the under-theorised relationship between successful performance of complex joint action and perceptions of team click.  I then preview the main components of this dissertation (Section~\ref{sect:components}), and conclude with an outline of the chapters of the dissertation (Section ~\ref{sect:chapters}).


\section{Existing scientific explanations of group exercise\label{sect:GESoCo}}

In this section I outline existing cognitive and evolutionary accounts of group exercise.  In this dissertation, I define group exercise broadly as any activity that minimally entails 1) sustained physical exercise (i.e. physical activity that reaches at least low intensity physiological exertion of 45\% of max heart rate or above), and 2) coordination of behaviour between two or more individuals in time and space to bring about change in the environment \citep[a.k.a., joint action, see][]{Sebanz2006,Vesper2010}.  A broad definition of this nature allows for the identification of instances of group exercise in a wide array of contexts throughout the human record, from music making and dance, to ritual practices, to competitive sport, warfare, play, and even pair-bonding activities such as sexual intercourse.  Anecdotal and ethnographic evidence pertaining to humans' subjective experience of all of these activities suggests that group exercise commonly entails a prominent and unmistakable visceral dimension. But, as I explain below, the natural scientific origins of this dimension of experience have remained unclear.


\subsection{Evolutionary origins of group exercise}
Archeological, paeleogical, and primatological evidence suggests that human’s capacity for group exercise runs deep into our evolutionary trajectory.  Humans have evolved distinct morphological and physiological adaptations that enable sustained aerobic exercise.  Adaptations include derived skeletal features that support stable bipedal running and respiratory function \citep[see ][]{Bramble2004}, as well as exercise-specific neurobiological reward \citep{Raichlen2012}.  The archeological record indicates that a capacity for sustained physical exercise may have emerged at some point in the transition from Pan (i.e., Australopithecus) to Homo, and could be associated with activities such as persistence hunting, defence of territory, and communication over vast distances such as those conceivable in woodland or grassland savanna, as opposed to dense canopy rainforests of pre-human ancestors \citep{Sands2012}.
    \footnote{While all species of the Homo genus (Sapiens, Erectus, and Habilis) all appear to show evidence of morphological adaptation for running, none of these species appear to born sprinters.  Even elite human sprinters (capable of sustaining maximum speeds of only 10.2 meters per second for less than 15 seconds) are slow compared to mammalian cursorial specialists such as horses, greyhounds and antelopes, who can maintain maximum galloping speeds of 15–20 meters per second for several minutes \citep{Garland1983}.  Moreover, running is more costly for humans than for most mammals, demanding roughly twice as much metabolic energy per distance travelled than is typical for a mammal of equal body mass \citep{Taylor1982}.  This suggests that humans have evolved capacities suited specifically to endurance running at moderate intensities, rather than short bouts of high intensity sprinting \citep{Bramble2004}.}

The co-occurence of physical exertion with joint action has been identified in activities of non-human primates, suggesting that the human propensity for group exercise scaffolds on top of strong evolutionary foundations laid well before our last common ancestor with Chimpanzees ($\sim 5-7$ mya).  Group hunting in Chimpanzees is by now well-documented and is known to be both spontaneous, highly organised \citep[for example involving divisions of set roles][]{Boesch1989}, and causally linked to social bonding \citep{Mitani2001}.  Chimpanzees behave similarly in group territorial ``warfare'' against members of neighbouring groups \citep{Boehm1992,Wilson2014a}.
In addition to group hunting and territorial conflicts, Chimpanzees also engage in exertive group activities such as group laughter \citep{Waller2005}, or coordinated display patterns of tree buttress drumming, pant-hooting, and branch-dragging \citep[for example, observed as part of a ``rain dance,'' see][]{Goodall2000,Whiten2001}.  Meanwhile, bonobos display a propensity for highly arousing socio-sexual encounters, which are hypothesised to quell intra- and inter-group tensions (relating to competition over food or reproductive resources) through the activation of neuropharmacological reward mechanisms \citep{Dunbar1992,Parr2005,Clay2015}.  Taken together, this evidence suggests that group exercise a core underpinning to various adaptive social behaviours in non-human primates, ranging from hunting, to group defence, and even social affiliation and diplomacy.

\subsection{Group exercise in our more recent past}
Social scientists and anthropologists have long speculated that cultural activities in which group exercise feature are somehow central to the function of human sociality.  Sociologist Emile Durkheim emphasised the emotional importance of shared ritual practice, which he thought could ``change the conditions of psychic activity'' (p. 469).  This group-level property of ``collective effervescence,'' Durkheim argues, ``strengthen[s] the bonds attaching the individual to the society'' and has important behavioural consequences relating to prosociality and well-being (Durkheim 1915/1965; pp. 257-258). ``Once the individuals are gathered together,'' Durkheim notes of an indigenous Australian tribe engaged in ritual dance, ``a sort of electricity is generated from their closeness and quickly launches them to an extraordinary height of exaltation...''  As a compliment to his more famous essay on the role of shared cultural practices in generating ties of social reciprocity---`The Gift''---Marcel Mauss (Durkheim's nephew) also drew attention to the importance of the visceral, embodied dimension to human social life in his 1935 essay ``Techniques of the Body'' \citep{Mauss1935}.
    \footnote{The intuition that the body is at the centre of social processes has been developed throughout an entire French social scientific lineage.  Mauss was Durkheim's nephew, and phenomenologist Maurice Merleau-Ponty \citep{Merleau-Ponty1956}, in turn was influenced by Mauss.  Anthropologist Pierre Bourdieu \citep{Bourdieu1990}, and contemporary sociologist Loic Wacquant \citep{Wacquant2004} are also direct descendants of this line of inquiry, which invariably fixates on the causal relevance of ``embodied practice'' to social phenomena.}
Meanwhile, Victor Turner drew attention to the way in which the drama of ritual performance (invariably involving group exercise) creates a playful and liminal space separate from the usual structure of everyday social life in which ``spontaneous communitas'' and ``humankindness'' between ritual participants is fostered \citep{Turner1974}.  These accounts share a common focus on the relationship between embodied psychophysiological processes and social processes.

While initial anthropological accounts of group exercise in human sociality lacked an a rigorous scientific framework and verifiable explanatory mechanisms, the importance of the visceral and emotional dimensions of group exercise to human sociality was nonetheless intuitively grasped and duly emphasised.

\subsection{The sidelining of group exercise through the modern evolutionary synthesis and the cognitive revolution}

Generally speaking, more testable scientific accounts of human behaviour have emerged only in the last $\sim$70 years,  enabled by 1) the gradual refinement of evolutionary theory, now known as the ``modern evolutionary synthesis'' (hereafter MES) and 2) the ``cognitive revolution'' (hereafter CR) of the 1950s and 60s \citep{Laland2010}.  The MES amounts to a unification of the theory of evolution by natural selection (attributed to Darwin and Wallace in the second half of the 19th century) with a theory of genetic inheritance (replacing a previously popular theory of blended inheritance) \citep{Calcott2013}.  The MES thus paved the way for a definition of biological evolution as changes in the frequency of heritable DNA sequences in a population due to selection pressures exerted at the level of the phenotype \citep{Grafen1984}.  Meanwhile, concepts deriving from information theory, cybernetics, and computation provided the necessary language to describe the causal structures behind observable phenomena, be they on an organism's ontogenetic (development) or phylogenetic (evolutionary) timescales.  Forefront to initial applications of cognitive and evolutionary approaches to human behaviour were questions of the biological origins of human sociality, including our species-unique capacities (e.g. formal, syntactical language) for complex patterns of cooperation and coordination \citep{Wilson1975,Chomsky1965}.

Together, MES and CR enabled anthropologists to draw upon empirical findings and theoretical frameworks from neighbouring fields of cognitive science, psychology, evolutionary biology, and behavioural ecology, to develop hypotheses concerning proximate psychological and cultural causes and ultimate (i.e., evolutionary) explanations for human sociality \cite[e.g.][see Appendix ~\ref{app1:intro} Section ~\ref{sect:modernSynthesis} for a more detailed explanation of MS, CR, and their applications to human behaviour)]{Dawkins1976,Wilson1978,Sperber1996,Whitehouse2004,Dunbar1996}.

However, the necessarily rudimentary nature of formal mathematical gambits and directionality of causation in initial models of human cognition and evolution have limited the capacity of these hypotheses to account for the full extent of humans' species-unique behavioural niche.  For instance, strictest adherents to the MES ignore the causal significance of developmental processes, relying solely on the determinants of natural selection, mutation (copying error), and genetic drift to determine population level variation in gene frequencies (e.g., via the ``phenotypic gambit'' \citep{Grafen1984,Grafen1991}).  In this sense, cognition, culture, and other aspects of human ontogenetic creation are understood as extensions of the phenotype, upon which natural selection operates (with stochastic error and drift) to determine gene frequencies at the level of the genotype \citep{Dawkins1982}.  Biologist Richard Dawkins famously proposed that human culture could be modelled as evolutionary biologists model genes; as units of selection---''memes''---that transmit and fixate within populations to furnish humans' extended phenotype \citep{Dawkins1976}.  Thus, in initial applications of cognitive and evolutionary theory to human behaviour, the complexity of human culture is understood as a proximate mechanism of the phenotype, but with evolutionary properties in its own right (related to but distinct from the replication dynamics of genes).

The implication of these assumptions for scientific approaches to human behaviour was to direct the attention of empirical research  towards a small segment of cognitive mechanisms responsible for facilitating behaviour that was deemed either 1) \textit{prima facie} adaptive \citep[an approach that has since matured into the field of  evolutionary psychology, see][]{Cosmides1992}, or else conducive to facilitating transmission of cultural variants \citep[an approach that has since matured into the field of cultural evolution][]{Cavalli-Sforza1981,Boyd1988}.  As such, cognitive mechanisms pertaining to aspects of social decision making (in the case of game-theoretical models of cooperation) or memory and social learning (in the case of gene-culture coevolutionary models) have been most actively examined for evidence of their ultimate evolutionary significance \citep{Fuentes2016}.  This focus was further assisted by the furnishing of MES with a theory of change advanced by the CR, which conceived of species-unique aspects of human cognition as (primarily) logical, grammatical, and semantic processes.\footnote{Refer, for example, to the language of digital computation utilised by Ernst Mayr in his explication of the need to distinguish between ``proximate'' and ``ultimate'' levels of biological phenomena;][]{Mayr1961}).} Amidst all of this, the visceral, embodied, and emotional dimensions to human experience have been sidelined.

To be sure, human scientists have worked steadily to nuance cognitive and evolutionary approaches.  Models of cultural evolution, for example, adjust population genetic models to take into account the observable differences between cultural and genetic information, such as culture's capacity to support one-to-many transmission, the blending of cultural variants, and non-randomly guided variation \citep{Cavalli-Sforza1981,Boyd1988}.  These adjustments are part of the concession that cultural variants are not as dependent on high fidelity replication as their genetic cousins, but instead are shaped by evolved cognitive biases that favour the acquisition and transmission of some cultural variants over others due to their memorability or effectiveness \citep{Henrich2007}.  Meanwhile, the functional role of emotion has been foregrounded---in neuroscience \citep{Damasio1994}, and various strands of developmental, social, positive, and cultural psychology \citep{}. Within evolutionary psychology, it has been recognised that emotion may have evolved as a superordinate mechanism for regulatory function \citep{Cosmides2000}, and may play an important (albeit subordinate) role in regulatory function  decision making \citep{Dalgleish2004} and communication \citep{Rime2009}.  In all of these approaches, however, the foundational assumptions of the MES (sole determinism of Darwinian selection) and CR (primacy of explicit, propositional, or semantic processes of information transfer) remain largely in tact.  Thus, the role of the visceral, affective dimensions of human cultural activities (including those that feature group exercise),remain deemphasised in accounts of the natural origins of human life.

To summarise, the first 70 years of cognitive and evolutionary approaches to human behaviour and sociality have thus produced explanatory models that have tended to emphasise the rule-based (i.e., semantic) and rational dimensions of human experience.  While useful and groundbreaking, these attempts do not yet satisfactorily account for the dynamic interaction of cognition and physicality, and sociality typically observable in contexts involving group exercise.

%, which hinge on the cognitive capacities---e.g. imitation, teaching, and memory---for ``cumulative culture'' \citep{Tomasello1993}).

\subsection{Bondedness and sociality}
More recent waves of scholarship have attempted to address the visceral dimension to human sociality exemplified by group exercise contexts.  As Anthropologists Robin Dunbar and Susanne Shultz have aptly pointed out, humans do not adhere to a``dung fly'' model of sociality, by incidentally \textit{aggregating} in time and space according to genetically hard-wired protocols \citep[cf.][]{Wilson1975}.  Rather, humans actively \textit{congregate} in structured, cohesive groups, which are held together by emotionally-mediated relational bonds \citep[777]{Dunbar2010}.  What makes human sociality unique, according to Dunbar, is the quality of ``bondedness''---the human capacity to forge social bonds with genetically unrelated members, which results in a cohesive
and supportive environmental niche \citep[cf.][]{Odling-Smee2003}.
Because being able to maintain the effective functionality of a group through time may have very significant individual fitness benefits for its members, the emergent property of sociality itself can be understood as part of the individual’s fitness strategy \citep{Dunbar2010b,Nowak2010}.

Dunbar argues that human bondedness and sociality cannot simply entail cognitive processes \citep[at least not in the way cognitive processes are narrowly rendered by game-theoretic and gene-culture coevolutionary models][]{Dunbar2010}.  Rather, bonded relationships involve two parallel and quite distinct mechanisms—--a cognitive mechanism (derived from what Dunbar calls the ``social brain''\citep{Dunbar1998}, or what has been otherwise understood as an evolved ``norm psychology'' \citep{Chudek2011}), and an emotionally based form of attachment \citep[often involving a psycho-pharmacological mechanism][]{Dunbar2010b}.  Thus, by appealing to the physiological (emotional) basis of relational ties between individuals, Dunbar scientifically reinstates the visceral dimension to human sociality that is observable in cultural activities involving group exercise.

  %This and which was temporarily absent in initial waves of cognitive and evolutionary approaches to human behaviour.

Humans' evolved capacity for social bonding is thought to have arisen in primates as an adaptive response to the pressures of group living.  Aggregating in groups serves to reduce threat from predation, but at the same time can be individually costly due to stress arising from interaction at close proximity, and conflict over resources among genetically unrelated individuals.  These costs can lead the group to disband, and are hypothesised to have led to selection for social bonding via dyadic grooming, as the coalitional alliances that arise among grooming partners allow for the maintenance of the group by buffering the stresses of group living \citep{Dunbar2012}.  Primate social grooming leads to the release of endorphins \citep[a type of endogenous opioid, see][]{Keverne1989}, presumably leading to sustained rewarding and relaxing effects \citep{Dunbar2010}.  While other neurotransmitters such as dopamine, oxytocin and/or vasopressin may also be important in facilitating social interaction, it has been suggested that endorphins allow individuals who are not related or mating to interact with each other long enough to build ``cognitive relationships of trust and obligation'' \citep[1839]{Dunbar2012}.

As the homo genus evolved more complex collaborative capacities for survival in interdependent group contexts \citep[see][]{Dunbar1998,Tomasello2012a}, grooming-like behavioural technologies (such as group laughter, music making and dance, and collective ritual) also evolved \citep[via processes of multi-level cultural group selection, cf.][]{Wilson2008} to sustain social bonding in larger group sizes where dyadic grooming would take too much time \citep{Dunbar2012,Tarr2014,Launay2016}. When neuropharmacologically-mediated mood-elevating effects are experienced in a group they seem to lead to participants embodying each other’s affective experiences, resulting in more positive, trusting, and cooperative relationships among participants \citep{Dunbar2012}.

Bondedness provides a vehicle for a scientific comprehension of what Durkheim described as collective effervescence.  Measuring bondedness requires, on an evolutionary or ultimate level of analysis \citep[cf.][]{Mayr1961,Tinbergen1963}, the loosening of the strict singular determinacy of natural selection on population level gene frequency.  In its place, bondedness suggests a multi-level or niche-construction approach, whereby individual-level benefits can be generated through the production and evolution of cultural activities within a phenotypic niche \citep{Dunbar2012,Laland2010,Laland2015}.  On a behavioural or ``proximate'' level of analysis, bondedness requires a conceptual broadening of mechanisms of cognition to incorporate the causal role of physiological (as opposed to purely cognitive) mechanisms.  Social bonding provides the immediate ``social'' glue for social cohesion \citep[cf.][]{Lakin2003,Bastian2014a}, and this achievement depends crucially on physiological (and not just cognitive) processes.

\subsection{The social high theory of group exercise and social bonding \label{sect:socialHigh}}

Does group exercise generate bondedness in human sociality?  Research focussed on the proximate physiological, cognitive, and social mechanisms associated with group exercise confirms that group exercise is responsible for generating a psychophysiological environment conducive to social bonding.  Anthropologist Emma Cohen and colleagues have recently identified bi-directional causal links between the two essential ingredients of group exercise---1) physiological exertion and 2) interpersonal movement coordination---and their common psychophysiological effects, including increased pain tolerance, athletic performance, positive affect, wellbeing, pro-sociality, and cooperation \citep{Davis2015}.  This evidence amounts to what can be called the ``social high'' theory of group exercise and social bonding \citep[hereafter ``the social high theory,'' see][]{Cohen2017}. Here, social bonding is understood as the psychological experience of increased social closeness, which facilitates affiliation between non-kin group members \citep{Tarr2014}.  I outline the key principles of the social high theory below, before introducing some empirical and theoretical knowledge gaps around group exercise and bondedness that require further attention.

\myparagraph{Physiological exertion\label{sect:physExertion}}
The social high theory proposes a causal link betweeen physiological exertion and social bonding.  Group exercise necessarily entails rigorous physiological exertion.
The health and wellbeing benefits associated with regular physical exercise---including reduced risk of cardiovascular disease, autonomic dysfunction, and early mortality; as well as enhanced neurogenesis, cognitive ability, and mood---are becoming increasingly well-known \citep{Blair1994,Nagamatsu2014}. Evidence suggests that strenuous and prolonged physical exercise is modulated by the same neuropharmacological systems responsible for regulating pain, fatigue, and reward \citep{Boecker2008,Raichlen2013}.  Neurobiological rewards in exercise are associated with both central effects (improved affect, sense of well-being, anxiety reduction, post-exercise calm) and peripheral effects (analgesia), and appear to be dependent for their activation on exercise type, intensity, and duration \citep{Dietrich2004}.  Exercise-specific activity of neurobiological reward systems offers a plausible explanation for commonly reported sensations of positive affect, anxiety reduction, and improved subjective well-being during and following exercise---extremes of which are popularly referred to as the ``runner's high'' \citep{Dietrich2004,Boecker2008,Raichlen2012}.  This neurobiological evidence maps on to more extensive literature concerning the psychological effects of exercise, which indicates a duration and intensity ``sweet spot'' for exercise and positive affect, whereby moderate intensity exercise for durations of $\sim45$ minutes appears most optimal \citep{Reed2006}.

It is possible that the function of exercise-induced positive affect extends to the realm of social bonding, particularly when achieved in group exercise contexts \citep{Cohen2009,Machin2011}.  Endocannabinoids and opioids have been implicated in mammalian social bonding \citep{Fattore2010,Keverne1989}, and in humans specifically, there is evidence that endorphins (a particular class of endogenous opioids) mediate social bonding \citep{Dunbar2012,Shultz2010}.

\myparagraph{Interpersonal movement coordination\label{sect:synchrony}}
The social high theory also proposes a causal link between joint action and social bonding.  Experimental evidence (predominantly from social psychology) suggests that time-locked coordination of behaviour between two or more individuals is conducive to psychological processes of self-other merging, liking, trust, and psychological affiliation.  In these contexts, interpersonal coordination is primarily operationalised as behavioural synchrony---i.e., stable time- and phase-locked movement of two or more independent components (limbs, bodies, fingers, etc.) \citep{Pikovsky2007}. Researchers suggest that synchrony enables a tight attentional union between individuals who match the timing and content of their actions, leading to the enhancement of interpersonal similarity and the blurring of self-other boundaries in cognitive processing and recall \citep{Cohen2017}.  Relative to non-synchronous group activities, synchrony increases social bonding and pro-social behaviour \citep{Reddish2013,Reddish2013a,Wiltermuth2009,Tarr2014}---an evolutionarily important outcome of bonded relationships.  Recent studies have also found that, compared to solo and non-synchronous group exercise, synchronous group exercise leads to significantly greater post-workout pain threshold \citep{Cohen2009,Sullivan2014,Sullivan2013a, Sullivan2013b}.

A recent meta analysis of the behavioural synchrony literature in social psychology suggests three candidate mediators of the relationship between behavioural synchrony and social bonding: 1) lower cognitive affective mechanisms implicating neuropharmacological reward systems (e.g., opioidergic and dopaminergic systems), 2) neurocognitive action-perception networks responsible for the experience of self-other merging, and 3) processes of group-centred cognition responsible for perception and reinforcement of cooperation \citep{Mogan2017}.  The current balance of existing evidence suggests that affective physiological mechanisms may be more relevant to joint action involving larger group sizes in which generalised feelings of euphoria and pro-sociality are common \citep[e.g., mass religious rituals or music festivals, see][]{Weinstein2016}, whereas neurocognitive mechanisms linking joint action and social bonding may be more applicable to smaller group sizes in which individuals can share intentions through ostensive communicative signals and implicit movement regulation cues \citep{Lang2017}.  Studies linking synchrony with social bonding and cooperation are supported by a literature than connects nonconscious mimicry with liking and affiliation \citep{VanBaaren2009}.

\myparagraph{The social high $=$ exertion $\times$ synchrony}
In addition to recorded independent effects of exertion synchrony,  preliminary evidence suggests that exertion and coordination in group exercise interact to produce social effects \citep{Jackson2018}.  Social features of the exercise environment (for example, perceived social support, level and quality of behavioural synchrony, etc.) modulate exercise-induced mechanisms of pain and reward \citep{Cohen2009,Sullivan2014,Tarr2015,Davis2015,Weinstein2016}. This work is bolstered by existing literature on the social modulation of pain \citep{Eisenberger2012a}, and links between pain and prosociality \citep{Bastian2014a}.  The social high theory thus combines these two bodies of literature to tell a story in which positive affect---associated with neuropharmacologically-mediated pain analgesia and reward—--is extended to the social group via synchrony-activated cognitive mechanisms of self-other merging, and the perception and reinforcement of in-group cooperation.


\section{Empirical knowledge gaps in the relationship between group exercise and social cohesion\label{sect:empKnowGaps}}
While the social high theory has served to empirically flesh out---particularly at the proximate level of analysis---the relationship between group exercise and Dunbar's notion of bondedness, the theory remains limited in its ability to account for a full spectrum of profiles and subjective experiences in group exercise.
Even a cursory survey of human sociality reveals that group exercise scenarios often deviate markedly from the narrowly defined profile (the exertion $\times$ coordination sweet spot) and subjective experience of group exercise set out by the prevailing social high theory.  In this section, I review some of the empirical gaps that remain unexplained by the existing social high theory.  In particular, group exercise contexts often involve extreme (and not just moderate) levels of physiological exertion, as well as complex coordination demands (beyond exact synchrony).  In addition, the effects of participation in group exercise appear to extend well beyond those of feel-good social high, and into the realm of rich meaning making as well as fine-grained sensitivity to the click of joint action.  I suggest that these empirical gaps point to a more fundamental \textit{theoretical} gap in cognitive and evolutionary approaches to human behaviour. In short, the vast majority of cognitive and evolutionary approaches to human behaviour are unable to sufficiently account for the dynamic interlocking of cognitive, physiological, and social processes.  The dynamic dimensions of group exercise contexts serve to bring this theoretical gap into sharp focus.

%In this dissertation I propose to develop cognitive and evolutionary understandings of group exercise through closer attention to the immediate causal processes and psycho-social effects of interpersonal movement.  I suggest that the carnal mystery that Adrian attempted to articulate on my first night in Beijing can be explained in part by the way in which extreme physiological exertion and interpersonal movement coordination combine in rugby to enable and constrain physiological, psychological, emotional, and social processes.

\myparagraph{Group exercise involves both extreme physiological cost and profound meaning\label{sect:linkCostMeaning}}
Anecdotal and observational evidence suggests that group exercise contexts often entail extreme levels of psychophysiological exertion.  High-stakes professional competitive sporting contexts (international-level sports such as rowing or ultra marathon running), extreme adventure sports (big wave surfing, free-diving), high-intensity contact (rugby union, American football, ice hockey) and combat sports (MMA, boxing, wrestling) are known to involve extreme physiological demands, often including high levels of pain.  Although it is expected that extremely physiologically costly exercise contexts will involve activation of neurobiological reward mechanisms outlined above (see Section~\ref{sect:sect:physExertion} in this Chapter, and Appendix~\ref{app1:intro} Section~\ref{sect:neuroRewardGE}), some contexts may on average exceed (or alternatively never reach) the intensity and duration sweet-spot for optimal activation of neurobiological reward \citep{Raichlen2013} or positive affect \citep{Ekkekakis2011,Reed2006}.

At the same time, physical exercise involving extreme physiological, psychological, and social costs also appear to offer participants and observers an opportunity for profound meaning.  Many people do not engage in exercise \textit{just} enjoyment or health; rather, in some contexts sport forms part of a life of purpose and self-discovery \citep[see, for example][]{Jackson1995,Jones2004,White2011}.  Modern sport has always been much more than ``just a game,'' and instead offers an arena in which virtues and vices are learned, and the ``morality plays''—--of community, nation, or globe—--thus performed \citep{Elias1986,McNamee2008}.  Psychological and physiological resilience in exercise contexts is lauded as virtuous, as is evidenced by the numerous idioms in the English language that receive currency in exercise lore: ``when the going gets tough, the tough get going,'' ``no pain, no gain,'' ``you get out what you put in'' and so on \citep{Sarkar2014}.

Whereas the social high theory predicts motivation for exercise based on ``hedonic'' enjoyment, anecdotal and ethnographic perspectives emphasise instead the ethical and moral dimensions of athletes' experiences, and contextualise these experiences within political processes relating to the construction of the self, community, and nation-state \citep{Alter1993,Brownell1995,Downey2005,Wacquant2004}.
Social anthropologists and sociologists have for some time emphasised the social function of exercise and sport in diverse cultural contexts, and various attempts have been made to analyse the phenomenological experience of exercise in terms of its sociological and psychological meaning \citep{Bourdieu1978}.

Social anthropologist Joseph Alter \textcite{Alter1993}, for example, argues that for wrestlers in north India, the body functions as a nexus through which the symbolic and material structures of the state, family, and the individual coalesce.  In a similar vein, in their seminal ethnography of sport in China, cultural anthropologist Susan Brownell \textcite{Brownell1995}, argues that sport functions as a crucial national symbolic practice for the Chinese nation-state in a project of ``rejoining the world,'' and that the ``micro-techniques'' (c.f. Foucault, 1977) of this project entail significant cost to (and rich meaning for) the individual athlete.   Similarly, French sociologist Loic Wacquant \textcite{Wacquant2004}, in an ethnography of boxers in Chicago's south side, describes a ``social logic'' of physical activity, claiming that the costs associated with ``the daily dedication and high technique that training demands; the regimented diet; the control, mutual respect, and tacit understandings necessary for actual fist-to-fist competition serve to create for the boxer an island of order and virtue'' \textcite[17]{Wacquant2004}. In many instances, it may be that the primary psychological motivation for exercise is not immediate, reward-induced hedonic wellbeing, but instead \textit{eudemonic} wellbeing, or the psychological awareness of a process through which life becomes ``well-lived'' \citep{Fave2009,Huta2013}.

\myparagraph{Group exercise demands complex coordination which can lead to team click\label{sect:linksComplexClick}}
A similar connection between physiological, cognitive, and social processes in group exercise can be witnessed in joint action of group exercise contexts.  Currently, the social high theory relies upon exact behavioural synchrony as an idealisation of successful coordination in joint action.  While some group exercise contexts do contain high levels of behavioural synchrony \citep[rowing, synchronised swimming, diving, mass calisthenics, and forms of dance such as ballet, see][]{McNeill1995}, exact in-phase synchrony is not typical of most instances of group exercise.  More generally, interpersonal coordination is more often achieved through flexible, function-specific assemblages of complimentary and contrasting behaviours \citep[for example, coordination in an interactional team sport, a dyadic conversation, or an ensemble music performance, see][]{Fusaroli2014}.  Real-world instances of joint action in group exercise usually entail various distinct action elements, organised hierarchically within a sequence \citep{Schmidt1975,Rosenbaum2009}.  Successful execution of the structure of real-world joint action requires temporal and spatial precision and flexibility of movement across multiple timescales and sensorial modalities \citep{Sebanz2006,Pacherie2012}.
In essence, successful coordination in joint action typical of group exercise contexts requires considerable cognitive resources \citep{Turvey1978}, which are not budgeted for when joint action is modelled as exact in-phase synchrony \citep{Keller2014}.

% As discussed below in Section ~\ref{sect:pathBeyondSynch}, this reliance on synchrony could occlude important causal mechanisms in a relationship between joint action and social bonding.

At the same time, participants in group exercise contexts involving complex joint action often scrutinise the quality of coordination, and derive powerful psychological reward when complex joint action clicks.  Technically demanding group exercise contexts such as competitive interactional team sports or music-making and dance, depend upon fine-grained precision of coordination of behaviours between individuals: the movements and goals of one individual must align precisely in time and space with the movements and goals of another.  For highly skilled expert practitioners, often the ecstasy of group activity is contingent not just on participation, or on resting on exact synchronisation of behaviours with others, but on the extent to which performance in joint action satisfies or exceeds implicit and explicit expectations.  Consider the passage below, taken from a series of interviews that psychologist Susan Jackson performed with elite figure skaters:
%Psychologist Susan Jackson has accumulated considerable evidence of elite level athletes' subjective experience of ``flow'' in joint action:
  \begin{quotation}
    It was just one of those programs that clicked. I mean everything went right, everything felt good . . . it's just such a rush, like you feel it could go on and on and on, like you don't want it to stop because it's going so well.  It's almost as though you don't have to think, it's like everything goes automatically without thinking . . . it's like you're in automatic pilot, so you don‘t have any thoughts.  You hear the music but you're not aware that you're hearing it, because it's a part of it all. \citep[168]{Jackson1992}.
  \end{quotation}

The psychological literature of flow and optimal human performance in sport has documented that athletes engaged in team coordination often report total absorption in and complete focus on the task at hand, a transformation of the experience of time (either speeding up or slowing down), and a blurring or transcendence of individual agency, or a ``loss of self''   \citep{Csikszentmihalyi1992,Jackson1995,Jackson1999,McNeill1995}.  Research suggests that flow often occurs in scenarios in which there are clear goals inherent in the activity, as well as unambiguous feedback concerning extent to which goals are either being achieved or not.  In addition, scenarios most conducive to the experience of flow are those in which the technical requirements are challenging but achievable if practitioners are able to extend (slightly) beyond their normal capabilities\citep{Fong2015}.  The coalescence of these factors is intrinsically rewarding and autotelic\citep{Csikszentmihalyi1975}, activating both ``hedonic'' and ``eudemonic'' dimensions of subjective well-being \citep{Huta2010,Fave2009}.


In sum, the social high theory is not yet sufficiently equipped to deal with instances of group exercise that deviate from a narrow profile of moderate intensity exertion, exact synchrony, and a feel-good social high.  Anecdote and ethnographic observation suggest that group exercise contexts also involve extreme levels of physiological cost, rich psychological meaning making, cognitive complexity, and feelings of team click.  Importantly, extreme physiological cost appears to be tethered to cognitive and social processes of meaning making and social identity (Section ~\ref{sect:linkCostMeaning}).  Similarly, cognitive complexity in joint action appears to be linked to embodied, affective experiences of flow, eudemonic wellbeing, and team click.  These empirical gaps in the social high theory suggest the importance of dynamic interlocking between proximate cognitive and physiological mechanisms for the generation of psychophysiological experience of group exercise.

The social high theory's predominant focus on the proximate physiological and affective dimensions of bondedness in group exercise means that it is less able to articulate an observable relationship between lower-order physiological and higher-order cognitive processes and social effects. In this dissertation I suggest that this missing link in the social high theory is not merely an empirical one that will be rectified over time with closer and closer examination of the componential proximate mechanisms of group exercise. Rather, novel theoretical synthesis is required to unify physiological, affective, and social dimensions of human activity.

The fact that group exercise contexts demonstrate a nagging, unmistakable cross-cutting of these various processes serves to expose a theoretical issue in the bondedness account of human sociality. While cognitive and physiological mechanisms are identified as co-determinant of bondedness in human sociality, these two mechanisms remain conceptually and causally partitioned on the proximate level of analysis.  I this dissertation, I argue that advancing cognitive and evolutionary understandings of bondedness in human sociality can benefit from dedicated attention to the addressing longstanding conceptual divides in human behavioural science between cognitive, emotional, and visceral processes.


%How can we address these gaps in scientific understandings of group exercise?  Is it possible that these gaps are causally linked via mechanisms beyond those currently specified by the social high theory?
%Based on the evidence reviewed above, it seems likely that, in addition to moderate intensity exertion, in-phase synchrony, and a feel good social high, an investigation into extreme levels of pain, the ``click'' of complex joint action, and profound psychological meaning making, could also offer important insights into a relationship between group exercise and social cohesion.  If so, what cognitive and evolutionary mechanisms are required to link the full diversity of profiles of group exercise with the full diversity of their psycho-social effects?

%In sum, clear variation in the types, intensities, and durations of group exercise, and the complex structure and subjective experience of, and motivation for group exercise presents an opportunity for further research into explanatory cognitive, evolutionary, and social mechanisms underlying these observable phenomena.

%In particular, I identify two relationships hitherto unaccounted for by existing accounts: the relationship between extreme cost and profound meaning, and the relationship between successful performance of complex joint action and perceptions of flow or ``team click.''


\subsection{Attempts within the anthropology of ritual to incorporate physiological, cognitive, and social dimensions of cultural activity\label{sect:cogEvAnth}}

Recently, cognitive and evolutionary accounts of ritual practice have attempted broaden understandings of proximate mechanisms that contribute social cohesion.  For instance, Anthropologist Harvey Whitehouse has developed a theory in which divergent modes of ritual practice account for two different types of social cohesion \citep[commonly known as the ``modes theory''][]{Whitehouse1996,Whitehouse2004,Whitehouse2014}.  In one mode, high frequency, low-arousal rituals (such as regularly attending church, practicing daily superstitions, etc.) help instantiate in semantic memory cues that enable personal identification with the prototypical features of the group \cite[i.e., ``group identification,'' cf.][]{Turner1987}.  Doctrinal rituals appear to evolve according to their memorability, which can be understood as an optimal curve of cognitive complexity \citep[][]{Whitehouse2005,Kapitany2015}.  Alternatively, low frequency, high arousal rituals (such as self-immolation, fire walking, rites of passage, hazing ceremonies, etc.), generate ``identity fusion''—--a psychological construct that captures an individual’s agentic and visceral sense of oneness with the group \citep{Swann2009,Swann2015}.  It is suggested that costly ritual practices reliably activate neurocognitive mechanisms of pain and reward \citep{Fischer2014a}, as well as autonomic mechanisms of physiological arousal \citep{Swann2010,Jackson2018}, which serve to activate a psychological process of alignment between self and group identity \citep{Xygalatas2013}, such that memories of the group become self-defining or ``autobiographical'' \citep[see][]{Whitehouse2014}.

Whitehouse and colleagues propose that these two groupings of ritual practices (``doctrinal'' and ``imagistic'') can be understood as ``attractor points'' in a landscape of possible cultural practices \citep{Atkinson2011,Whitehouse2014}.  This proposal appeals to a model of cultural evolution known as Cultural Attraction Theory \citep[hereafter CAT; an extension of Sperber's ``epidemiology of representations,'' see][]{Sperber1996,Claidiere2007,Claidiere2014}.  In counter distinction to gene-culture co-evolutionary models \citep[, which largely adhere to the ``selectionist'' paradigm, see][]{Acerbi2015}, CAT utilises a dynamical systems model of evolutionary change \citep[known generally as ``evolutionary systems theory,'' cf.][]{Ramstead2018}, in which the status of selection (for individual-level adaptiveness of cultural or biological variants) as the sole determinant of population-level distribution of cultural variants is relaxed.  Instead, selection is understood as only one ``factor of attraction'' among other factors such as ecological dynamics, cross-cultural variation, and social cohesion \citep{Claidiere2014,Heyes2011}. In turn, the relaxation of selection reduces a fixation on cognitive mechanisms of social learning (imitation, learning, instruction), and makes space for a consideration of how a fuller spectrum of cognitive capacities facilitate cultural transmission \citep{Acerbi2015}.  Supported by CAT, the modes theory is thus able to connect extremely physiologically costly (and often dysphoric) experiences in ritual with cognitive processes of meaning making and social processes of prosociality \citep[including extreme levels of pro-group sacrifice ][]{Whitehouse2014,Whitehouse2017}.  In this way, the modes theory---more so than the social high theory, for example---offers an account in which physiological, cognitive, and social processes are more satisfactorily intertwined, at least at the level of evolutionary change.

A closer examination of the modes theory reveals that, on a proximate level, the dynamic and interlocking relationship between physicality, cognition, and sociality is less well articulated.  The modes theory technically supports the possibility that any one observable ritual could contain both doctrinal (predominantly cognitive) and imagistic (predominantly physiological) dimensions.  According to CAT (which backgrounds the modes theory) the two ritual modes are understood as statistical points in an n-dimensional space to which ritual practices are only tendentially attracted \citep{Atkinson2011,Whitehouse2014a,Scott-Phillips2017}.  Thus, it is theoretically conceivable that an activity such as competitive interactional team sport, which combines both highly arousing \textit{and} cognitively complex dimensions to behaviour activity,  could contemporaneously engage both physiological and cognitive mechanisms and generate psycho-social effects which, similarly, blend both cognitive and affective dimensions.

Thus far, however, there is little empirical evidence that integrates proximate physiological, cognitive, and social dimensions of human activity on a non-binary distribution \citep{Atran2010}.  In the case of the modes theory, for example, experimental paradigms test either ritual complexity memorability (semantic) on the one hand \citep{Whitehouse2005}, or physiological arousal and identity fusion on the other \citep{Whitehouse2014,Whitehouse2017,Swann2010a,Richert2005}, and offer little opportunity for measuring a gradient in between \citep[but see][]{Russell2014}.  In this dissertation I suggest that this lack of evidence is fundamentally a theoretical issue \citep{Clark2015}, rather than a mere empirical issue that will be worked out in time \citep[cf.][]{Whitehouse2014a}.

%The modes theory \citep[cf.][]{Whitehouse2004} does propose a causal link between physiological cost and profound meaning, by theorising a relationship between ritual practices involving extreme dysphoric arousal and identity fusion---a high quality and identity-laden form of social bonding.  This link sits at one end of a theoretical distribution of ritual practices, the other end of which being a link between low arousal/high frequency rituals defined by their memorability (cognitive complexity) and responsible for a semantically-mediated social identification.  Although the modes theory offers the theoretical flexibility to empirically address an entire gradient of proximate causal mechanisms within these two ``attractors'' of human ritual practice, to date very few attempts have been made to verify the interlocking physiological, cognitive, and social mechanisms of bondedness in ritual practice.

A related issue pertains to CAT more generally.  On an empirical level, CAT theorists are active in a long-standing debate within cultural evolution, which centres around the proximate cognitive mechanisms that facilitate cultural transmission \citep{Acerbi2015,Scott-Phillips2018}.  Traditional approaches to cultural evolution endorse the reasoning that human culture resembles an evolutionary system in its own right---distinct from (albeit contingent upon) biological evolution---and should therefore be modelled as such.  In this approach, culture is modelled as discrete semantic units, which are understood to be preserved between minds and within populations via humans' species unique capacity for precocious and high fidelity imitation \citep[i.e., like genes, culture is preserved through processes of exact replication with natural copying error and drift][]{Henrich2003,Tomasello2011}. This approach has naturally focussed attention on human-unique aspects of culture, particularly those aspects that are rich in semantic content such as explicit theory of mind, language, or other more or less explicit cultural practices \citep{Tomasello2005}.  CAT instead predicts that cultural transmission encompasses a fuller spectrum of cognitive and physiological (end ecological) mechanisms in a dynamical process of ``reproduction,'' rather than mere imitation \citep{Claidiere2007,Mesoudi2017}.

Indeed, CAT itself, as a deliberate rebranding of Sperber's ``epidemiology of representations'' \citep{Sperber1996}, signals a conscious attempt to advance a theory of cultural evolution that is content-neutral (e.g., ``cultural representation'' is replaced by ``cultural variant'' \citep{Scott-Phillips2018}) and causally diverse/plural \citep{Claidiere2014}.  Accumulating evidence has confirmed the prediction that transmission is best understood as a process of dynamical reproduction (rather than strict preservative copying) \citep[e.g.,][]{Morin2016,Scott-Phillips2017}.  However, this evidence  pertains primarily to explicit cultural variants that are rich in semantic content (e.g., language, literature, and concrete cultural artefacts), and therefore less contingent on dynamical variables (for example, the presence or absence of others, or emotional valence) for their reproduction \citep[15]{Ramstead2016}.
As such, the debate around cultural transmission remains fundamentally entrenched around explicit and semantic cultural variants. Physiological foundations theorised to be central to cultural transmission, meanwhile, receive comparatively little empirical attention \citep{Ramstead2016,Lerique2016}.\footnote{Apparently (according to the conference grapevine), when Sperber is asked to explain what he understands as culture, he uses the example of cross-cultural variation in interpersonal distance regulation in conversation.  This example is very much an implicit example of culture depending on dynamical cognitive processes of movement regulation.  However, most of CAT's existing empirical contributions pertain to ``transmission chains'' of explicit linguistic productions rich in semantic meaning.}

Thus, while bondedness, the modes theory, and CAT all advance a more causally dynamic model of cultural evolution, these approaches still suffer in their own ways from a lack of a sufficiently dynamical (and testable) proximate theory of change (i.e., cognition) that is capable of supporting evolutionary-level predictions.  The dynamically interlocking physical, cognitive, and social dimensions identifiable in real world instances of group exercise---identified by Adrian, Durkheim, and Dunbar (among many others)---serve to expose this theoretical issue, and offer an opportunity to conduct important theoretical and empirical research within anthropology and the human sciences more generally.

In the following section, I consider a prevailing dynamical theory of cognition, which can be used to better understand the dynamical interlocking of physiological, cognitive, and social processes in group exercise, and thus shed new light on the relationship between group exercise and social cohesion.

 %theoretical issue at hand resides in the fact that the modes theory in particular and CAT more generally rely on models of behaviour in which cognition and emotion are conceptually and functionally separate.

 %While both theories draw up a theoretical space for a fuller scientific conception of human sociality on an evolutionary level, both face impediments to empirical progress due to their inability to substantiate ultimate level claims with proximate-level behavioural evidence.

 %The visceral dimension of group exercise remains to be fully appreciated in existing cognitive and evolutionary accounts of human behaviour.

 %Dunbar provides an adequate concept (bondedness) and Sperber and colleagues provide an adequate evolutionary framework (CAT), which has received some empirical attention (modes theory), the outdated model of cognition that services these approaches ultimately impedes empirical progress.

 %As already suggested, real-world instances of group exercise are complex and multidimensional behavioural phenomena involving dynamic interaction of multiple brains and bodies, situated in varied and richly resourced cultural ecologies.  Development of cognitive and evolutionary theories that can account for this complexity, and in so doing reconcile some of the identifiable gaps in existing accounts, remains a work in progress \citep{Fuentes2016}.  Particular attention must be devoted to accounting for the relationship between cognitive, physiological, and social dimensions of humans' evolutionary niche.


\subsection{Active inference framework and the social cognition of joint action}
%could be interwoven with emotion and affect \citep[cf.][]{Damasio1994},
It has been long intuitive that cognitive processes are contained within and could be contingent upon physicality. However, it has been comparatively difficult to empirically demonstrate within cognitive science how physicality, emotion, cognition, and culture could co-determine processes of behaviour.

A long-standing movement within the cognitive and behavioural sciences has called for a greater recognition of the physical and dynamical properties of information transfer in biological systems.  Proponents of so-called embodied, embedded, enactive, or extended cognition \citep[now collectively referred to as ``4E cognition,'' see][]{Menary2010}, emphasise that cognition typically involves acting with a physical body on an environment in which that body is immersed.  Researchers in this movement call out traditional ``stimulus-response'' model of human cognition for being too static, abstract, and compartmentalised: static and abstract because human behaviour is traditionally rendered as the outcome of a linear chain of perception, a-modal representation, and action-selection; compartmentalised because human cognitive processes are usually understood to be located discretely either within the brain or else within certain (often dualistic) cognitive subsystems \citep[e.g., emotional and cognitive, System 1 (fast) and System 2 (slow), implicit and explicit, and so on; cf.][]{Diennes1999,Kahneman2011}.  So contend 4E proponents, traditional models of cognition create a dichotomy between cognition and physicality, and reproduces a fixation on semantic representations as the foundation of cognitive and cultural processes.

The most radical proponents of the 4E perspective on cognition challenge the idea that human interaction and communication requires that humans be endowed with the capacity for explicit and content-rich cognitive representation.  Instead, humans' diverse cultural repertoires can be explained primarily by embodied processes and dynamic coupling \citep{Gallagher2001,Gallagher2008,Fuchs2009}.  The core of the 4E argument is that human inferential processes not only activate physical movement, but are also activated by movement, in a dynamical loop of reciprocal causation.

The 4E approach has made valuable contributions to articulating what traditional stimulus-response theories of human cognition have lacked, namely, a conception of the dynamical and physical properties of information transfer.  However, in directing attention to the dynamical and content-neutral/free processes of movement regulation, the 4E turn in cognitive science has traditionally struggled to provide a viable account of humans' species-unique capacity for complex and semantically rich mechanisms of information transfer, embodied in language, shared narratives, and physical artefacts \citep{Ramstead2016}.  In order to transcend the theoretical tension between traditional and 4E approaches to cognition, a dynamical model of cognition must be able to account for the spectrum of cognitive processes that enable human-typical behaviour.

\myparagraph{An ``active'' theoretical solution}
A combination of recent advances in neuroimaging technologies \citep{Frith2007}, neurocomputational theories of brain function \citep{Friston2010,Frith2010,Yufik2013,Clark2013}, and constructive attempts to extend the theoretical paradigm of human social cognition to account for inter-individual processes of interaction and coordination \citep{Sebanz2006,Dale2014}, has led to the emergence of a theoretical approach capable of explaining the fact that human cognition entails an interlocking ensemble of cognitive processes that span dynamical coupling, through to content-rich semantic representations \citep{Roepstorff2011,Ramstead2016}.  The prevailing paradigm, which I consider in this dissertation, conceives of human cognition as a process of ``active inference'' \citep{Friston2010}.  Active inference \citep[and the predictive coding paradigm which it extends, see][]{Clark2013} proposes a radical inversion of traditional models of cognition that rely predominantly on bottom-up sensory inputs and top-down feature detection \citep{Marr1985}. Instead, active inference posits that top-down predictive models themselves shape perception and action, and the only information that travels forward (or from the ``bottom-up'') is the error signals that arise from discrepancies between predictions and the sensorium \citep{Clark2015}.  The active inference approach depicts a human cognitive system in which perception, attention, and action are functionally and temporally integrated to manage uncertainty inherent in interactions with the environment \citep{Clark2013}.

Rather than being restricted to a dualistic either/or choice between functionally distinct cognitive modes of inference (e.g. habitual or mental, explicit or implicit, fast and slow \citep[cf.][]{Dienes1999,Kahneman2011}, active inference predicts that humans benefit from flexible deployment of of multiple strategies from a unified web of neural and extra-neural affordances \citep{Pezzulo2013,Clark2015}.  These strategies can be seen to like on a continuum, which ranges from more computationally intensive generative models, on one end, to lower cognitive mechanisms of movement regulation that facilitate more direct coupling with the extra-neural resources of the task-specific environment, on the other \citep{Riley2011}.

The active inference approach, and its relevance to joint action, team click, and social bonding will be thoroughly reviewed in the following chapter.

%Patterned practices, affordances cue predictions and create sites of causally dense relationships between cognition and culture.regimes of attention which give rise to shared understandings, skilled intentionality,Just as selection is relaxed as the primary determinant of

\section{Addressing empirical gaps through a focussed study of rugby in China}

In this dissertation, I focus my research on a group exercise context well suited to address the empirical knowledge gaps outlined above.  Rugby union (hereafter ``rugby'') is a dynamic field-based contact sport that requires of its participants high levels of physiological exertion and complex coordination of joint action (see Chapter~\ref{chap:researchSetting} Section ~\ref{sect:rugbyUnion} for a more detailed explanation). As I outline in more detail below, rugby is also anecdotally and colloquially associated with experiences of team click and social processes of group membership in many of the contexts in which it is commonly played \citep{Dunning2005}.

``Rugby'' and ``China'' are two words that are not usually mentioned in the same sentence.  Rugby---''a game for barbarians played by gentlemen''---first took root in the elite education institutions of Britain's colonial empire.  China has enthusiastically adopted sport and exercise at different stages throughout the country's turbulent modern history. But for most of this history, rugby accorded neither with a dominant Olympic-centred logic of the Chinese sport system, nor with dominant cultural dispositions and modes of understandings physicality \citep[derived from hundresds of years of continuous history of Confucian and Daoist traditions of thought, see][]{Morris2004}.  Rugby eventually made it to China in 1990, as a university sport program set up through an exchange with a Japanese university.  However, since 2009, when rugby became an Olympic sport in the form of ``rugby sevens'' (the modified seven-a-side version of rugby), rugby has been ``embosomed'' (\textit{huaibao} 怀抱) by the state-sponsored sport system \citep{Xu2010}.  At the time of writing, more than ten of China's 32 provinces have full time men's and women's professional programs.

While rugby has an institutional footprint in China, Chinese adherents to the sport face various challenges in the process of acquiring rugby's technical and social skills.  Rugby in China contains fewer of the cultural scaffolds that are erected around the sport in traditional rugby playing nations: young children do not grow up playing rugby in the schoolyard or watching their heroes and heroines play rugby on television.  Beyond rugby, China has traditionally struggled to perform well on the world (and regional) stage in interactional team sports like association football.  A number of social, economic, and cultural factors are at play in the phenomenon of China's poor performance in team sport.  Suffice to say, however, the construct of an abstract and arbitrary egalitarian social assembly---well-known in Western cultures as a ``team''---is far from an indigenous psychological concept in China \cite[instead, the family functions as a primary metaphor for social interaction][]{Liu2009}.  As I discuss in greater detail in the ethnographic study of this dissertation (Chapters ~\ref{chap:ethnoField}\nobreakdash~\ref{chap:ethnoResults}), the technical and social requirements of rugby appear to chafe against more predominant cultural affordances in modern China.

In spite of the lack of snug fit between rugby and dominant modes of social cognition in contemporary China, rugby is evidently responsible for generating a mysterious ``carnal'' feeling in its participants, as Adrian's monologue and my ethnographic observations professional rugby players in China confirm (see Chapters~\ref{chap:ethnoField} and~\ref{chap:ethnoResults}).  Rugby in China thus not only presents an excellent opportunity to explore the role of cultural variation in shaping patterns of behaviour in group exercise.  In addition, rugby in China presents an opportunity to subject an inherently ``WEIRD'' \citep[Western, Educated, Industrial, Rich, and Democratic; cf.][]{Henrich2010d} suite of cognitive and evolutionary theories to a non-WEIRD empirical setting.  Indeed, the viability of the active inference framework hinges on its capacity to incorporate cognitive and cultural variation into one inferential framework.  These predictions are yet to be thoroughly tested against real-world instances of human behaviour.  Finally, as I explain in more detail in Chapter~\ref{chap:researchSetting}, my personal qualifications uniquely position me to conduct research into rugby in China.

In sum, researching rugby in China offers an opportunity to address many of the outstanding theoretical and empirical questions in scientific explanations of group exercise.  The core contribution of this dissertation is to develop and test a theory of joint action and social bonding in group exercise that incorporates an active inference approach to cognition.



\section{Thesis overview}
To summarise the ground covered thus far: this thesis is driven by the overarching goal of contributing to a scientific explanation of the puzzling ubiquity of group exercise in the (more recent) human record. Current cognitive and evolutionary accounts suggest a relationship between group exercise and social cohesion, due in part to the way in which group exercise contexts uniquely generate social bonding between participants.  The social high theory posits that social bonding in group exercise is due to the way in which physiological exertion and interpersonal movement coordination combine to generate a psychophysiological environment conducive to affiliation and trust (social bonding).  The social high theory, in its current formulation, does not fully account for various empirical knowledge gaps in the relationship between group exercise and social bonding.  In this dissertation I focus on the theoretical pathway between successful performance of complex joint action and feelings of team click.  I propose the phenomenon of ``team click'' as a candidate construct that can help explain a hitherto under-examined link between joint action and social bonding.  I also adopt the active inference framework \citep{Friston2010} and apply it to joint action ~\ref{Friston2015,Friston2015a} as a way of reconciling an impeding distinction between physicality and cognition in joint action.

%\subparagraph{Focussed research}
In this dissertation, I demonstrate the utility of ethnographic observation for exploration, and quantitative field-experimental methods for the purpose of hypothesis testing.  In honour of the capacity of cultural and ecological trajectories to shape and direct observable behaviour in distinctive ways, the three empirical components of this dissertation are confined to one specific research setting---i.e., professional rugby players in China.

To explore the validity of the theoretical predictions formulated in Chapter~\ref{chap:theory}, I begin with an in-depth ethnographic study of a real world group exercise setting---the Beijing men's rugby team.  I then refine my theoretical predictions based on the results of ethnographic analysis, and test these in an \textit{in-situ} survey study of a more representative sample of professional Chinese athletes. I administered surveys to athletes ($n = 174$) before, during, and after a National Championship Tournament in order to ascertain information about their experience of a high-intensity and high stakes joint action scenario.  These two studies provided the necessary empirical motivation for a controlled field experiment designed to interrogate specific mechanisms hypothesised to underpin the phenomenology of team click and social bonding in joint action.  Each study builds on the previous study in a step-wise manner, and as such the cultural and ecological affordances associated with the group exercise context can be identified and held relatively constant.


In this section, I outline the components of this thesis, namely, the novel theory of social bonding through joint action, and the three empirical studies I performed to substantiate this theory (see Table ~\ref{tab:contributions}).

% Please add the following required packages to your document preamble:
% \usepackage{booktabs}
\begin{table}[]
\centering
\begin{tabular}{@{}llcl@{}}
\toprule
\textbf{} & \textbf{Components}                                                                               & \multicolumn{1}{l}{\textbf{Chapters}} & \textbf{Description}                                                                                                                                         \\ \midrule
          &                                                                                                   & \multicolumn{1}{l}{}                  &                                                                                                                                                              \\
1         & \begin{tabular}[c]{@{}l@{}}A novel theory of social\\ bonding through joint\\ action\end{tabular} & Ch. 2-3                               & \begin{tabular}[c]{@{}l@{}}Supported by an active inference framework; \\ proposes team click as a mediating construct\end{tabular}                          \\
          &                                                                                                   &                                       &                                                                                                                                                              \\
2         & Ethnography                                                                                       & Ch. 5-7                               & \begin{tabular}[c]{@{}l@{}}Beijing men's rugby team (n = 26); \\ Data: participant observation, \\ informal surveys, semi-structured interviews\end{tabular} \\
          &                                                                                                   &                                       &                                                                                                                                                              \\
3         & \begin{tabular}[c]{@{}l@{}}\textit{In situ} survey \\ study\end{tabular}                                   & Ch. 8                                 & \begin{tabular}[c]{@{}l@{}}Conducted during the The National 7s Rugby\\ Championship (n = 174, male = 93)\end{tabular}                                       \\
          &                                                                                                   &                                       &                                                                                                                                                              \\
4         & \begin{tabular}[c]{@{}l@{}}Controlled field\\  experiment\end{tabular}                            & Ch. 9                                 & \begin{tabular}[c]{@{}l@{}}Conducted with athletes from Beijing \\ and Shandong provincial rugby \\ programs, (n = 58, male = 30)\end{tabular}               \\ \bottomrule
\end{tabular}
\caption{Core components of this dissertation}
\label{tab:thesisComponents}
\end{table}



\end{CJK}{UTF8}{gbsn}
