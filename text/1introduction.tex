\chapter{\label{introduction}Thesis Outline}


First Week in Beijing:

The night after I arrived in Beijing in late August 2015, I was invited to a dinner hosted by Adrian, an elder of the Chinese rugby community in Beijing.  Adrian was the captain of the second graduating class of rugby players the Chinese Agricultural University (CAU), the home of China's first official rugby union program established in 1990.  I first met Adrian two years earlier in 2013, through a good friend Kai. Kai was  a more recent CAU graduate, a former Chinese National rugby team representative, and now a lawyer in Beijing.  Kai was also invited to the dinner, as was Mr Shi, a sports television producer from Chinese Central Television (CCTV).  The background to the dinner was that World Rugby, the international governing body of rugby, had recently given CCTV the broadcasting rights to the 2015 Rugby World Cup, to be held imminently in England during September and October 2015.  Mr Shi had been charged with the production of the 48-match tournament, which would be the first time international rugby was televised on Chinese national television.  Mr Shi was completely new to rugby, and so needed help making rugby accessible and understandable for Chinese audiences.  Shi reached out through his network in Beijing and soon tracked down Adrian, who was working in the digital media department for NBA China. Adrian in turn tracked down Kai. Both Adrian and Kai were well connected to the Chinese rugby community and fluent in English, and so were well placed to assist Mr Shi in the tasks of translating relevant rugby materials and organising expert commentators for the broadcast.  My arrival in Beijing was timely for this project, and Kai was quick to recruit me to join them at dinner.  I was eager to begin my fieldwork and so, despite my mild to moderate jet lag, I accepted the invitation and set off to the restaurant on the Saturday evening with my notepad and audio recorder (i.e., my mobile phone) in hand.

The Dinner:
Adrian, Kai, and I waited for Mr Shi to arrive in the upstairs area of the Korean BBQ restaurant in a quiet Peking willow lined street just inside Beijing's East 4th Ring Road.  Adrian, as host and elder, held the floor as we waited: he reminisced fondly about his time playing rugby at CAU as well as his time in Beijing after graduating, when he played with the Beijing Devils, a predominantly expat rugby club in Beijing:

  ``Rugby was so much fun in those days, not like today (in the professional era of Chinese rugby).  Everyone was just scraping together the money to go on tour, we all payed our own way, sometimes you'd get a bit of help from someone or whatever. It was for the love of the game, not for any other reason.''

When Mr Shi finally arrived, Adrian continued the nostalgic story telling mode but naturally shifted his target audience from Kai and me to Mr Shi.  When describing in rich detail the experience of participating in an overseas rugby tour with the Beijing Devils, he interrupted his own story with an explanatory aside directed at Mr Shi, accommodating for the fact that Mr Shi was relatively unacquainted with the sport: ``This sport, rugby union, it's actually very mysterious. If you haven't played it yourself you might not know this type of feeling,'' Adrian respectfully suggested to Mr Shi. ``Because rugby, you know, you're on the field playing together, there's body contact...'' he paused to find the right phrasing,  ``...its a very ``carnal'' type of feeling. Everyone is very close.'' His attempts to enrich his communication by gesticulating had led him to have both of his hands clenched as fists in front of him like they were cradling a rugby ball, a lit cigarette smouldering between the index and middle finger of his right hand.  Adrian concluded by reiterating: ``Its very mysterious,'' shaking his head as if baffled and releasing his clenched fists to dab the ash from his cigarette into the ashtray in front of him. He took another drag and  finally added,``So it means this rugby ``circle'' here in China is very tight...'' (Circle (quanzi) is a common way to refer to a social group or community of people) ``...but it doesn't mean that its not also a mess!'' The wisdom in this final note was confirmed with a knowing chuckle from all of us, including Mr Shi.

%英式橄榄球这个项目其实特别神秘,没玩过的话您可能不知道这种感觉,因为英式橄榄球么,大家在场上有身体接触,是一种``肉''的感觉,大家互相都特别亲,特别神秘。
%所以在橄榄球这个圈子特别亲, 但这不是说这个圈儿也不乱!

I was particularly struck by this snippet of Adrian's monologue, perhaps because it was so soon in to my fieldwork that I had happened upon such a rich description, in which the visceral sensation of playing rugby (rou) was linked conceptually to social closeness (qin) and the broader social cohesion of the rugby community (quanzi).  I did not fully realise it at the time, but Adrian's closing caveat would also be particularly relevant to the coming months of fieldwork with the Beijing provincial rugby team. I would come to experience first hand the complexity beneath Adrian's sarcastic assertion that the social closeness derived from the visceral experience of taking the field with teammates did not necessarily buffer against the (political) messiness of off-field social interactions between individuals within the imagined community of Chinese rugby players.  These interactions were, after all, structured by broader institutionalised incentives and cultural dispositions that existed well beyond and well before the existence of the rugby field in China.

Interestingly, however, I also came to understand the complexity of this statement from a different standpoint to Adrian.  When going looking for evidence of bonding and social closeness in rugby in the Beijing team, I was bombarded with testimonies and rationales for why social cohesion and closeness did not and could not exist.  (Chinese society is too complicated, athletes are not innately motivated to play rugby, etc).  Nonetheless, when I interrogated deeper in to the details of athlete experience of joint action in rugby, I was able to find evidence that the visceral dimensions of rugby were related to processes of social affiliation.

I left that first dinner motivated to investigate these relationships further.  My next stop was Xiannontan Sports Institute (XNT) on Monday morning, where I was scheduled to meet separately with the vice-principal of the school in charge of the rugby program, and the head coach of the Beijing rugby program, former CAU coach, ZPH.











%%%%%%%%%%%%%%%%%%%%%%%%%%%%%%%%%%%%%%%%%%%%%%%%%%%%%%%%%%%%%%%%%%%%%%%%%%%%%%%%%%%%%%%%%%
%%%%%%%%%%%%%%%%%%%%%%%%%%%%%%%%%%%%%%%%%%%%%%%%%%%%%%%%%%%%%%%%%%%%%%%%%%%%%%%%%%%%%%%%%%
%%%%%%%%%%%%%%%%%%%%%%%%%%%%%%%%%%%%%%%%%%%%%%%%%%%%%%%%%%%%%%%%%%%%%%%%%%%%%%%%%%%%%%%%%%









% 252 word intro:
The human capacity to coordinate behaviour within cohesive social groups is a fundamental explanation for our species' evolutionary success.  It is therefore surprising that sport---an ubiquitous organiser of modern social life and physical movement across cultures---has not been more intensively studied for evidence concerning the cognitive and social foundations of human evolution \citep{Blanchard1995,Downey2005a}.  An integrated scientific study of the social, cognitive, and physiological mechanisms associated with participation in coordinated and physiologically exertive group activity, and the ecological dynamics by which these mechanisms are constrained, stands to offer novel insights into the science of human evolution.

Whirling Sufi dervishes, late-night electronic music raves, Maasi ceremonial dances, competitive team sports, or the modern cult of Cross-fit---endless examples can be plucked from across cultures and throughout time to exemplify the human compulsion to come together and move together.  It is easy to imagine how exertive and coordinated group activity would have served important survival functions in the past, such as hunting, travel, communication, and defence \citep{Sands2010}. More puzzling, but also more relevant to the study of human social cohesion, is the persistent recurrence of group exercise in social dimensions of life, cross-cutting shared cultural practices as varied as religion, sport, music and play, even when its fitness-enhancing benefits are less obvious or immediate.  Social scientists have long speculated about the benefits of energetic group activities for social cohesion (Durkheim, 1965). In so far as tightly bonded and well coordinated groups face better survival odds than those which are less so, bonding activities which foster social cohesion and trust can be considered collectively advantageous and adaptive \citep{Dunbar2010}.  It is not yet clear, however, how or whether group exercise uniquely generates social cohesion, or in what ways particular mechanisms vary by activity and culture.

In this dissertation I attend specifically to the relationship between joint action and social bonding in the group exercise context of professional rugby in China.  Experimental evidence from the behavioural synchrony and mimicry literatures suggests that high quality coordination of movement between co-actors in joint action may be a powerful source of positive affect, blurring of self-other agency, pro-sociality, and cooperation \citep{Mogan2017}.  Beyond these literatures, the relationship between less tightly coupled joint action and social bonding is yet to be thoroughly tested. Anecdotal and observational evidence from anthropology and psychology---particularly the psychology of ``flow'' \citep{Csikszentmihalyi1992,Jackson1999}---suggests that perceptions of joint action success may set the psychological foundation for processes of affiliation and cohesion.  Various neurological, cognitive, and sociological strands of evidence support this proposal.  Perceptions of successful synchronisation of behaviour in joint action appears to have positive implications for individual psychophysiological function, health, and subjective well being \citep{Wheatley2012}.  Likewise, there is well-documented evidence of a link between psycho-social isolation and ill-health and developmental and neurocognitive deficits in behaviours key to dynamic interpersonal interaction \citep[e.g.][]{Blakemore2005,Baron-Cohen1991}. An account of the full significance of the role of proximate mechanisms of movement regulation and coordination in social bonding is yet to be fully articulated. In this dissertation I address this gap through theoretical synthesis and empirical research.

\section{Group exercise and social cohesion}
% 527 words
There is something unmistakably ``embodied'' about coordinated and exertive group activity. Indeed, perhaps this visceral quality has helped ground the traditional intuition within anthropology that coordinated group activity (broadly construed) is somehow causally relevant to broader social processes, despite the lack of precise theoretical frameworks within which to test such speculations (see for example, \citep{Durkheim1965,Mauss1935,Radcliffe-Brown1952,Turner1974,Merleau-Ponty1956,Bourdieu1990}).  The modern evolutionary synthesis and cognitive revolution have since helped spawn rigorous scientific inquiry into the relationship between shared cultural practices (including those in which group exercise commonly features) and human social cohesion, and great strides have been made in identifying proximate mechanisms and ultimate evolutionary explanations for the transmission of, and widespread adherence to, cultural practices throughout populations \citep{Dawkins1976,Boyd1988,Sperber1996,Barrett2002,Whitehouse2004,Whitehouse2014,Henrich2007}.  However, still in their relative infancy, these theories are necessarily limited in their range and scope by start-up assumptions and idealisations, particularly in regards to the details concerning proximate cognitive mechanisms and psychophysiological, social, and contextual constraints on the transmission of cultural information transfer \citep{Sperber1996,Dunbar2012,Claidiere2014}. A theory capable of satisfactorily explaining the visceral and social dimensions of group exercise so strongly substantiated by observation, anecdote, and intuition is yet to be fully formulated \citep{Cohen2017}.

A combination of advances in neuroimaging technologies \citep{Frith2007}, emerging neurocomputational theories of brain function \citep{Friston2010,Frith2010,Clark2013}, and constructive attempts to extend the theoretical paradigm of human social cognition to account for inter-individual processes of interaction and coordination \citep{Sebanz2006,Dale2014}, has created an opportunity to empirically examine the relationship between coordinated and exertive group activities and social cohesion.  It is now more clearly understood that basal human capacities for physical movement regulation and coordination set the foundation for social cognitive systems whose resources are distributed between brains, bodies, and physical features of task-specific environments \citep{Hutchins2000,Kirsh2006,Semin2008,Semin2012,Coey2012}.  Furthermore, it has been shown that the quality of coordinated movement within these cognitive systems has implications for psychophysiological health and subjective well-being \citep{Wheatley2012}, and is relevant to the effective function of more complex goal oriented social activities, including the large-scale reproduction and transmission of shared cultural practices \citep{Dunbar2012,Roepstorff2010,Claidiere2014,Launay2016}. Thus, the somewhat nagging visceral intuition associated with the observable human compulsion to come together and move together could in fact prove useful as a source of insight for progressing the science of human evolution.  By interrogating the ways in which component mechanisms and system dynamics of joint action generate social bonding, this dissertation seeks to offer a novel contribution to cognitive and evolutionary anthropology.

\subsection{Cultural Evolution} %1490
The systemic under-theorisation of the social dimensions of human cognition is due in part to the youth of the scientific approach to human cognition and evolution. The idealisations and assumptions originally employed to kick start this science, most of which were borrowed form evolutionary biology and behavioural ecology, are necessarily limited in their capacity to model the precise details oh human social interaction.

The central theoretical challenge for cognitive and evolutionary explanations of human behaviour involves accounting for the mechanisms through which cultural practices transmit and fixate in populations. The modern evolutionary synthesis and cognitive revolution created the scientific conditions in which a testable, Darwinian theory of cultural transmission became possible.  First articulated by Cavalli-Sforza and Feldman \textcite{Cavalli-Sforza1981} and then by Boyd and Richerson \textcite{Boyd1988}, the working approach to human cultural evolution begins with the assumption that cultural information, i.e., information capable of affecting individuals' behaviour that they acquire from other members of their species through teaching, imitation, and other forms of social transmission, evolves via mechanisms similar to those that act on genetic information.  Borrowing from evolutionary biology, theorists adopt game-theoretical population genetics models to help demonstrate the selective pressures placed upon cultural information.

The strictest versions of this approach propose the ``meme'' as a gene-like unit of cultural information, which, like a gene, is subject to selection pressures of replication.  Memes that successfully replicate successfully populate\citep{Dawkins1976}. Other theories of cultural evolution adjust population genetic models to take into account the observable differences between cultural and genetic information, such as culture's capacity to support one-to-many transmission, the blending of cultural variants, and non-randomly guided variation.  These adjustments are part of the concession that that cultural variants are not as dependent on high fidelity replication as their genetic cousins, but instead are shaped by evolved cognitive biases that favour the acquisition and transmission of some cultural variants over others due to their memorability or effectiveness \citep{Henrich2007}.

These start-up assumptions of models of cultural evolution directed attention towards the causal role of micro-evolutionary cognitive mechanisms of imitation, teaching, and memory, in enabling high fidelity copying (with occasional mutation-like errors) of cultural variants between individuals and throughout populations with distributions stable enough for selection to operate.  Models indicate that for social learning to actually enhance population fitness, it must be cumulative throughout generations, i.e., individuals must be able to socially learn what they could not learn on their own \citep{Boyd1995}.  Thus, particular attention has been paid to the mechanisms that could be responsible for facilitating species-unique \textit{cumulative} culture \citep{Tomasello2008}.

Evidence from comparative and developmental psychology indeed suggests a precocious and species-unique tendency to accurately imitate the actions of trusted or authoritative others (even when the goal of the action is unclear) sets the cognitive foundation for the transmission of cultural representations \citep{Tomasello2014a}.  The microevolutionary processes have also been enhanced by supplementing macroevolutiony processes, also known as cultural phylogenetics \citep{Mace1994}.  The phylogenetic comparative method seeks to understand long-term cultural change at or above the level of the society by 1) reconstructing the cultural evolutionary history of a particular trait or set of traits and 2) testing functional hypotheses concerning the spread or distribution of cultural variation across societies while controlling for evolutionary history.  The combination of these micro- and macroevolutionary approaches supports the theory that  ``dual-inheritance'' or ``co-evolution'' of genetic and cultural information in humans over time has led to the development of prosocial norms and institutions that facilitate collective adherence to shared cultural practices \citep{Richerson2008,Chudek2011}.

Cognitive and evolutionary approaches to culture and social cohesion have made productive empirical strides in identifying the proximate cognitive mechanisms most relevant to ultimate evolutionary explanations for the distribution of shared cultural practices around which human groups cohere, particularly in relation to large scale cultural forms such as religion \citep{Henrich2015,Purzycki2016b}.  In contrast to the informal, idiosyncratic, and subjective schemas of historical linguistics, archaeology, and social and cultural anthropology, these micro and macro theories of human cultural evolution are explicit in their assumptions, repeatable and extendable by others, and easily scaled up to large datasets \citep{Mesoudi2017}. The fact that many details of human social interaction still remain open scientific questions is a necessary part of the trade-off involved in building a scientific formulation of cultural evolution and social cohesion. However, the interactive and affective mechanisms of shared cultural practices (i.e., the unmistakably ``visceral'' dimension referenced above) appear, phenomenologically at least, to be of distinct relevance to processes of cultural information transfer, and thus require careful and considered incorporation into a theory of cultural transmission.

\subsubsection{Cultural Attraction}
There is evidence to suggest that, beyond microevolutionary mechanisms of transmission, other factors may also have an important causal impact on the accumulation and distribution of cultural variants. For example, demographic factors such as population size, structure, and interconnectedness have been shown to determine cultural complexity (variation) in hunter gatherer populations, with adaptive implications \citep{Henrich2004}.  It has also been suggested that variation in prosociality, social bonding, and social cohesion could have an important bearing on information transfer between individuals and within groups \citep{Heyes2011,Whitehouse2014,Wheatley2016}.  As researchers in comparative and social psychology have pointed out, humans do not merely aggregate, but rather actively congregate around shared cultural practices, seemingly driven by species-unique affective and motivational mechanisms\citep{Dunbar2010,Tomasello2005a}.
In addition, there is evidence of cross-cultural variation in the microevolutionary dynamics of cultural evolution, for example, with specifically higher social learning in collectivistic East Asian societies than in individualistic Western societies \citep{Mesoudi2015,DiYanni2015}.  These empirical details have formed the basis of models of multilevel selection, which propose selection pressures at the gene, individual, and group for society-level cultural variants such as religious ritual, warfare, and agricultural practices  \citep{Turchin2013,Atkinson2011a}.

In light of this collection of evidence, researchers have sought to broaden the scope of cultural evolution by relaxing the strictly selectional logic of memetic and dual-inheritance models, instead suggesting that cultural variants will tend locally towards certain ``attractor points'' depending on the diverse cognitive, demographic, ecological factors of attraction to which they are subjected \citep{Sperber1996}.  Rather than explaining patterns of cultural diversity, stability, and change in terms of the differential selection of certain cultural variants (e.g., content biases) or differential copying of certain individuals (e.g., success bias, prestige bias),  ``cultural attraction theory''(CAT) focuses on how cultural variants are systematically \textit{re-produced} by a combination of frequency dependent (i.e. conformity) and context sensitive (i.e. prestige) transmission biases, and the biophysical, psychological, historical, and ecological dynamics by which these biases are constrained and directed \citep{Claidiere2014}.  It has been pointed out that in contrast to genetic evolution, the mechanisms responsible for transmitting cultural information in humans (imitation, learning, and memory) cannot alone explain population level stabilisation of cultural variants, because they are not faithful enough to stabilise distributions of cultural variants on which selection can operate\cite{Claidiere2014}. CAT suggests instead that population level cultural variation is produced by processes that are partly preservative (i.e., occur via mechanisms of transmission), and partly re-constructive---the combination of which will result in cultural variants that tendentially converge upon particular types, called attractor points.

Broadening the theoretical scope of cultural evolution in this way enables a more detailed consideration of the microevolutionary processes of cultural information transfer beyond the traditional candidates for preservative transmission. In particular, the role of proximate cognitive, neurological, and social mechanisms of interpersonal interaction in facilitating social cohesion around shared cultural practices can be afforded much closer attention.  Thus, the highly visceral quality of shared cultural practices that contain elements of rigorous and coordinated group exercise could be a signal of important and hitherto unarticulated information concerning the proximate componential mechanisms, dynamical constraints, and ultimate evolutionary explanations for human uniqueness \citep[3]{Claidiere2014}.  What, then, is the role of group exercise in human sociality? Specifically, how does the joint movement associated with group exercise contexts uniquely generate social cohesion, and how do particular mechanisms vary by activity and culture?

Working within this specific research domain, in the following section I attempt to account for a for a theoretical link between proximate mechanisms of highly interactive and exertive collective movement common in competitive team sports such as rugby union, and the feeling of ``team click'' that appears to arise in athletes when the synchronisation of collective movement meets (or exceeds) athletes' prior expectations.

In turn, I seek to account for the process mechanisms that could explain how high quality joint action responsible for team click could by extension generate social bonding effects.  I review existing literature concerning the relationship between joint action and social bonding. In particular I review research concerning behavioural synchrony, a special case of joint action. The bonding effects of behavioural synchrony have been studied directly for over 10 years, and empirical evidence has confirmed and developed hypothesised relationships between coordinated physical movement and prosocial behaviour, perceived social bonding, positive affect, and modulation of feelings of exertion and fatigue.



\subsection{Team Click}
Recent research has made a ceremony of invoking one particular passage from Durkheim (1965, pg. 217) to capture the ``collective effervescence'' of exertive and coordinated group activity found in arenas as diverse as music making, dance, military drills, and sport:  ``Once the individuals are gathered together, a sort of electricity is generated from their closeness and quickly launches them to an extraordinary height of exaltation'' \citep{McNeill1995,Konvalinka2011,Fischer2014,Mogan2017}. Indeed, this passage powerfully captures the role of collective activity in generating positive emotional states and joint arousal, and lends itself nicely to the hypothesis that this visceral ``electricity'' is attributable in part to neuropharmacologically-mediated affective mechanisms associated with pain and reward\citep{Dunbar2008,Cohen2009,Fischer2014,Launay2016}.   \textit{EXPLAIN MORE RE grooming hypoethesis here  SYNCHRONY, EXERTION = BONDING}

Missing from Durkheim's passage, however, is an aspect of group activity that is heavily scrutinised in
technically demanding joint action scenarios such as competitive interactional team sports or, music-making and dance: the \textit{quality} of movement synchronisation in joint action.  Activities such as music-making, dance, and sport depend upon highly complex coordination of behaviours between individuals, in which the movements of one individual must align in time and space with the movements of another.  Highly skilled practitioners who develop a fine-grained sensitivity concerning the perceived outcome of joint action, often the ecstasy of group activity is contingent not just on participation, but on the extent to which joint action with co-participants ``clicks.''   The psychological literature of optimal human experience (also known as ``Flow'' \citep{Csikszentmihalyi1992}) offers extensive documentation of the positive psychological and social effects of technically complex movement in individual and, to a lesser extent, joint action. To date, however, very little research has dealt directly with the relationship between perceptions of \textit{quality} joint action and processes of social bonding and group formation \citep[but see][]{Marsh2009}.
This dissertation locates the phenomenology of perceived ``click'' of joint action

is grounded in the current state of the art of social cognition, and centred aroun
an ethnographically verified phenomenology of skill acquisition and movement
presents a theory of social bonding through joint action that serves to
 looks beyond the generalised neuropharmacologically-mediated electricity of  ``collective effervescence'' and into the social cognition of movement
evidence suggesting that an explanation of social bonding through coordinated group activity must involve more than just



One of the big mysteries of competitive team sport, particularly at the elite level, is the elusiveness of peak team performance.  While each individual athlete may exhibit expert-level competence in sport-specific skills, the much sought-after aggregation of these components, i.e. a team that consistently performs ``in the zone,'' and ``firing on all cylinders,'' in reality often proves frustratingly difficult to achieve and sustain.  As King and De Rond \textcite[568]{King2011} note in their ethnography of the 2008 Cambridge University rowing crew who participated (and who were eventually victorious) in the famous annual Boat Race against Oxford University, the search for collective rhythm is a universal in human social interaction, but  the physiological and psychological complexity of finding that rhythm ``...is extremely difficult to attain; collective performance is a possibility not a certainty.''   The moment in which everything ``clicks'' into place in team sport can, for various reasons, disappear as abruptly as it arrives, if indeed it arrives at all.

But, when team click is somehow cultivated, and even sustained, it is celebrated as the ultimate, albeit often inexplicable magic of sporting feats. Consider Leicester City Football Club's unbelievable outhouse-to-penthouse title run in the 2015 English Premier League, the recent dominance of the Golden State Warriors in the American National Basketball League, or the astonishingly consistent performance of the New Zealand men's national rugby union team.  The ``All Blacks'' are arguably the most successful sporting team ever, with a winning percentage of 77\% in the last 150 years (88\% in the last 6 years).  All of these successful teams carry with them a powerful ``aura'' associated with their capacity to effectively coordinate their behaviours on the field over extended time scales: individual games, seasons, and, in the case of the All Blacks, entire generations.  Although it is tempting to be seduced by the aura of such rare instances of collective performance, this dissertation attempts the (admittedly) more banal task of moving from mystery to scientific mechanism, in order to explain these collective phenomena in terms of their social, historical, physiological, and psychological components and dynamics.

In this dissertation, I use the term ``team click'' to describe the phenomenology of peak performance within a team of athletes engaged in joint action.  For athletes, coaches, and spectators alike, team click can be a hugely powerful sensation. As theologian Michael Novak explains, ``[f]or those who have participated on a team that has known the click of communality, the experience is unforgettable, like that of having attained, for a while at least, a higher level of existence'' \citep[11]{White2011}. As has been extensively documented in the psychological literature of ``flow'' \citep{Csikszentmihalyi1992} and optimal human performance in sport \citep{Jackson1999}, athletes engaged in team coordination often report total absorption in and complete focus on the task at hand, a transformation of the experience of time (either speeding up or slowing down), and a blurring or transcendence of individual agency, or a ``loss of self''   \citep{Csikszentmihalyi1992,Jackson1995,Jackson1999,McNeill1995}.  Research suggests that flow often occurs in scenarios in which there are clear goals inherent in the activity, as well as unambiguous feedback concerning extent to which goals are either being achieved or not.
In addition, scenarios most conducive to the experience of flow are those in which the technical requirements are challenging but achievable if practitioners are able to extend slightly beyond their normal capabilities\citep{Fong2015}.
The coalescence of these factors is intrinsically rewarding and autotelic\citep{Csikszentmihalyi1975}, activating both ``hedonic'' and ``eudaimonic'' dimensions of subjective well-being \citep{Huta2010,Fave2009}.  The vast majority of flow research has focussed on the experience of the individual athlete, musician, or performer.  However, more recent attempts have been made to extended an analysis of flow and its antecedents to the level of the group and dynamics of interpersonal coordination---a phenomenon termed ``group flow'' \citep{Sawyer2006}. Indeed, as the phenomenon of team click suggests, individual experiences of flow are almost always embedded in and contingent upon cognitive processes and contextual dynamics involving co-actors and the physical environment, even if the existing literature preferences an individual-centred account of the phenomenon\citep{Kirsh2006,Marsh2009,Noy2015}.

Importantly, team click appears to have important flow-on consequences relevant to social bonding and affiliation. Tightly synchronised activity in particular, found in team sports such as rowing, can help dissolve the boundaries between individual and social agency: ``In rowing...it feels like you have at your command the power of everybody else in the boat. You are exponentially magnified. What was a strain before becomes easier. It is absolutely the ultimate team sport'' \citep{Brown2016}.
The blurring of agency between self and team may be responsible for facilitating affiliation and trust between teammates in competitive athletic environments such as professional rugby, which often involves high physiological stress and uncertainty: ``...you always wanted a guy who would go into the trenches with you and he always played consistently well...he could really play and was just one of the good lads that you enjoyed his company'' \citep{Fox-Sports2017}. In this sense, the experience of team click may act as a social diagnostic tool, a powerful signal of commitment to joint action and willingness to cooperate \citep{Reddish2013a}.

The experience of flow has by now been extensively studied by psychologists and neuroscientists, from which a series of neuropharmacological \citep{Boecker2008}, neurocognitive \citep{Dietrich2006,Dietrich2011,Labelle2013}, and psychological \citep{Csikszentmihalyi1992} theories for its emergence have been tabled.  However, throughout this process, the social dimensions of optimal human experience have been less scrutinised, despite strong anecdotal and observational evidence of phenomena such as group flow, team click, and social bonding emanating from these collective states. In the sections that follow, I draw upon related strands from cognitive, neuroscientific, and psychological---including social psychological---literatures in order to develop a novel theoretical account of the relationship between coordinated interpersonal joint action and social bonding, and the mediating role of ``team click.''


% The extent to which synchronised joint action is responsible for generating social bonding may depend crucially on the accordance of action with culturally directed expectations.


Physical activity, exercise, and sport have well-known positive effects on physical and  psychological health (Ekkekakis, 2003; Fiuza-Luces, Garatachea, Berger,  Lucia, 2013).
The health benefits associated with regular exercise, including reduced risk of cardiovascular disease, autonomic dysfunction, and early mortality, are becoming increasingly well-known (Blair  Powell, 1994; Nagamatsu et al., 2014).

While the physiological, psychological, and social processes that combine in instances of exerted, coordinated movement are rich and varied, many strands of research suggest a link between group exercise and social bonding \citep{Davis2015,Cohen2017}. It is now understood that strenuous and prolonged physical exercise is modulated by the same neuropharmacological systems (namely, the opioidergic and endocannabinoid systems) responsible for regulating pain, fatigue, and reward \citep{Boecker2008,Raichlen2013}.
Exercise-specific activity of these systems offers a plausible neurobiological explanation for commonly reported sensations of positive affect, anxiety reduction, and improved subjective well-being during and following exercise---extremes of which asre popularly referred to as the ``runner's high'' (Dietrich  McDaniel, 2004; Boecker et al. 2008; Raichlen, Foster, Gerdeman, Seillier,  Giuffrida, 2012).  This neuropharmacological account of group exercise and social bonding has its roots in studies of social grooming in non-human primates.  Dunbar and colleagues propose a neuropharmacologically mediated affective mechanism linking dyadic grooming practices with group-size maintenance \citep{Machin2011}.

The capacity for social bonding is thought to have arisen in primates as an adaptive response to the pressures of group living.  Aggregating in groups serves to reduce threat from predation.  At the same time, it can be individually costly due to stress arising from interaction at close proximity and conflict over resources among genetically unrelated individuals.  These pressures are hypothesised to have led to selection for social bonding (e.g., via dyadic grooming). Resulting coalitional alliances among close partners allow for the maintenance of the group by buffering the stresses of group living.  Primate social grooming, for example, is associated with the release of endorphins, presumably leading to sustained rewarding and relaxing effects.  While other neurotransmitters such as dopamine, oxytocin, or vasopressin may also be important in facilitating social interaction, endorphins allow individuals who are not related or mating to interact with each other long enough to build ``cognitive relationships of trust and obligation'' \citep[1839]{Dunbar2012}.  It is thought that, as the homo genus evolved more complex collaborative capacities for survival in interdependent group contexts, grooming-like behaviours sustained social bonding in larger groups where dyadic grooming would cumulatively take too much time \citep{Dunbar2012}.
Experimental studies suggest that neurophysiological mechanisms activated by activities that involve physical exertion and coordinated movement, such as group laughter, dance and music-making, exercise, and group ritual can bring groups closer together, mediated by the psychological effects of endogenous opioid and endocannabinoid release \citep{Cohen2009,Fischer2014a,Fischer2014,Sullivan2014,Tarr2016,Tarr2015}.

In addition to reports of exercise-induced euphoria and positive affect, adherents to (group) exercise and other activities---particularly highly skilled practitioners---also commonly report experiencing states of ``optimal'' or ``peak'' performance, which include feelings of heightened focus, personal transcendence, time-warp (the experience of time either speeding up or slowing down), spontaneity, creativity, and effortlessness \citep{Jackson1995a}.  ``Flow,'' as this particular cluster of states has commonly been referred to, is a powerful, autotelic and embodied experience, which combines components of both ``hedonic'' (sensation-centred, see \citep{Huta2010}) and ``eudaimonic'' (meaning-centred, see \cite{Ryff1989,Ryff2015}) dimensions of subjective well-being, and is theorised to emerge when activity strikes a balance for the individual between challenge and skill requirements \citep{Csikszentmihalyi1990,Abuhamdeh2012}.  At the level of the group, the ``team click'' and ``group flow'' are highly elusive possibilities, coveted by athletes, coaches, and fans alike \citep{Novak1993,Sawyer2006}.  While the experience of flow associated with prolonged exercise may be in part neuropharmacologically mediated by the opioidergic and endocannabinoidergic systems, phenomenological accounts suggest that there is something distinct about the experience of flow in exercise that requires a more complete cognitive and social explanation \citep{Dietrich2006,Dietrich2011}.  One speculative neurocognitive account of acute exercise, for example, suggests that the metabolic costs associated with complex or prolonged regulation of movement forces an energetic trade-off in the brain in which lower level neurocognitive processes win out, forcing a down-regulation of the pre-frontal areas of the brain \citep{Dietrich2011}. Dietrich and colleagues propose that the down-regulation of cortical processes induces a decline in executive control \citep{Labelle2013} and possibly dampens self-monitoring and personal agency. If this hypothesis is correct, it is highly plausible that flow and its neurocognitive underpinnings are relevant to the affective and prosocial effects of group exercise.

Meanwhile, research in social psychology focusing on the relationship between time-locked behavioural synchrony and processes of self-other merging, social alignment, and affiliation has shed light on the social and affective significance of interactive and coordinated movement typical of many group exercise contexts \cite{Wiltermuth2009,Kirschner2010,Reddish2013,Tuncgenc2016}. Experimental evidence suggests that time-locked coordination of behaviour between two or more individuals in the stable attractor/equilibrium states of either in-phase or anti-phase synchrony is conducive to psychological processes of self-other merging, liking, trust and affiliation.  It is believed that lower cognitive processes of joint attention mediate the link between synchrony and social bonding, with synchronised activity (common in music, dance, and some sports) providing a shared spatio-temporal (and often haptic) referent around which to coordinate attention and behaviour \cite{Launay2016,Wolf2015}.

Studies linking synchrony with social bonding and cooperation are supported by a literature than connects nonconscious mimicry with liking and affiliation\citep{VanBaaren2009}.  The experimental studies above predominantly refer to dyad synchronisation of behaviour.  The social and psychological effects of group level synchronisation have been harder to induce and measure in experimental settings. However, in addition to in- and anti-phase behavioural matching, group synchronisation may be subject to more complex and dynamical processes of coupling, which could entail specific psychological consequences. This also appears to be true in cases of joint---but not necessarily explicitly synchronised---action, whereby implicit processes of movement regulation link two or more individuals in a complex and dynamic coupling. The variation and stabilisation of such dynamic couplings could have psychological effects (see \citep{Schmidt2008,Marsh2009a}).  Most encouraging is evidence that manages
to integrate the social and neurophysiological dimensions of group exercise.  Recent experimental evidence suggests that social features of the exercise environment (for example, perceived social support, level and quality of behavioural synchrony, etc.) modulate exercise-induced mechanisms of pain, and reward \citep{Cohen2009,Sullivan2014,Tarr2015,Davis2015,Weinstein2016}. This work is bolstered by existing literature on the social modulation of pain \citep{Eisenberger2012a} and links between pain and prosociality \citep{Bastian2014a}.


\section{Theoretical Grounding}
A combination of recent advances in neuroimaging technologies \citep{Frith2007}, emerging neurocomputational theories of brain function \citep{Friston2010,Frith2010,Clark2013}, and constructive attempts to extend the theoretical paradigm of human social cognition to account for inter-individual processes of interaction and coordination \citep{Sebanz2006,Dale2014}, has created an opportunity to examine in finer-grained detail the relationship between coordinated and exertive group activities and social cohesion.  It is now more clearly understood that basal human capacities for physical movement regulation and coordination set the foundation for social cognitive systems whose resources are distributed between brains, bodies, and physical features of task-specific environments \citep{Hutchins2000,Kirsh2006,Semin2008,Semin2012,Coey2012}.
Human cognition appears to be driven by a processes of ``active inference'' \citep{Friston2010} about the world.  Agents generate top-down interoceptive predictions about the state of the world and test these representations against bottom-up sensory evidence \citep{Clark2013}.  In this account, perception, representation, emotion, and action are unified by the logic of prediction-error management, and neurocognitive components interact to align the organism with its expectations \citep{Pezzulo2014}.  Conceiving of social cognition in this way, as an embodied, embedded, and immediate process of inference, centralises the role of automatic movement regulation strategies---traditionally classed as ``lower-cognitive'' processes---in establishing and maintaining the transfer of cultural information between individuals, within groups, and throughout populations---traditionally thought to be executed by  ``higher-cognitive'' processes \citep{Claidiere2014}.

A review of the available literature suggests that successful joint action in humans is  contingent on the ability to share functionally equivalent task representations. Considering the cognitive principles of ``active inference'' referenced above, shared task representation amounts to minimising prediction error in social cognitive systems involving two or more co-actors \citep{Semin2008,Frith2010}.  Humans appear to employ an array of explicit and implicit behavioural strategies in order to achieve this.   The ways in which co-actors ``close the loop'' \citep{Frith2007} on joint action through deliberate ostensive communication has been the traditional focus of developmental, comparative \cite{Tomasello2005a}, and social psychologists \citep{Sebanz2006}.
More recently, however, analysis of dynamic coupling of co-actors in joint action scenarios reveals that synchronised movement implicates an array of implicit and pre-perceptual cognitive processes of alignment and prediction error minimisation \citep{Schmidt2011}, which, in addition to more explicit forms of communication, could be central to the generation of feelings of self-other merging, self-other distinction, and perceived reliability and trust associated with social bonding \citep{Marsh2009}. By interrogating the ways in which component mechanisms and system dynamics of joint action generate social bonding, this dissertation seeks to offer a novel contribution to the cognitive and evolutionary anthropology of social cohesion.


\section{Joint Action in Group Exercise}
There is considerable variation in the nature and dynamics of joint action, even within the sub-category of group exercise. Joint action in group exercise ranges from tightly coupled dyadic or group activities such as rowing, synchronised diving, or dance sport, to interactive competitive team sports like basketball, ice hockey, and rugby, through to more loosely coupled (but still time- and space-coordinated) mass participation activities such as marathons and triathlons.  It is sensible to assume that, as the scale and requirements of these contexts vary, so too will the psychophysiological mechanisms most responsible for enabling successful joint action, feelings of team click, and social bonding \citep{Mogan2017,Launay2016}.

Interactive and co-active team sports in particular contain dimensions of complexity that are not directly addressed by the existing experimental literature concerning synchrony or joint action.  The competitive nature of these sporting practices means that co-actors in joint action scenarios will perform roles that either facilitate or obstruct shared goal achievement, depending on team assignment \citep{Renshaw2009}. Competitive joint action scenarios facilitate two modes of communication between individual participants: more predictable behaviour between cooperators and less-predictable action behaviour between opponents \citep{Glover2017}. Thus the competitive dimension of interactive team sports introduces complexity, whereby subunits of cooperating co-actors coordinate their behaviours around a shared goal (winning the specific contest) \citep{Passos2012},  and co-actors from both teams coordinate with each other around the higher order shared goal of completing a competitive game.
In addition, interactive team sports involve the nesting of coordinated subunits of actors and sub-phases of actions \citep{Vilar2012}.  For example, a dyadic joint action such as passing a ball between two attacking players in association football is nested within a larger attacking sub-phase goal of advancing towards the opposing team's goal in order to score a goal, which is in turn nested within a larger shared goal of beating the opposing team in a 90 minute match, and so on.  These dimensions of complexity in interactive team sports increase the overall degrees of freedom of joint action tasks, thus demanding higher technical competence in order to successfully establish functional interpersonal synergies capable of reducing such uncertainty and behaving adaptively \citep{Duarte2012}.

\subsection{Rugby Union Football}
Rugby Union (hereafter rugby) is an interactional team sport played on a rectangular field (100m x 70m), by two teams, usually of 15 players, who physically contest possession of an egg-shaped ball that can be used to score points \citep{IRB2014}.\footnote{Descending from a variety of locally-specific folk-games played in pre-industrial England, all loosely grouped as ``football'', rugby developed within the elite public school system as a deliberate physical activity arbitrated by rules and regulations, before circulating through the arteries of England's colonial empire as a leisurely pastime—a ``sport'' \citep{Dunning2005}.}
``Rugby sevens'' (hereafter Sevens), the version of rugby that is the focus of this research, is a shorter 7-on-7 version of rugby. Sevens is a highly interactive and physiologically demanding sport at all levels at which the game is currently played, even more so than the 15-a-side version of the game.   Sevens requires players to participate in frequent bouts of intense (anaerobic) activity such as sprinting, physical collisions, tackles, and grappling, separated by short bouts of low intensity activity such as walking and jogging. Sevens requires high levels of interdependence between team members due to the uncertainty and complexity of interactive coordination tasks.  At the elite level in particular, the physiological costs and complexity of joint action requirements of sevens are amplified.

There is evidence to suggest that dynamic coupling occurs between dyads and sub-units of attack and defence in rugby \citep{Passos2011,Correia2014}.  Passos and colleagues \textcite{Passos2011} for example find that functional coupling tendencies emerge between attacking dyads and adapt to specificities of the task environment.  Correia and colleagues \textcite{Correia2014} show that coupling tendencies also emerge between co-actors of opposing teams in rugby, for example, in a 1-on-1 attacker/defender sub-phase.  These results are confirmed in similar joint action contexts in other equivalent sports such as basketball and association football \citep{Duarte2013}. There is evidence to suggest that the establishment and maintenance of functional interpersonal synergies in rugby joint action depend on an athlete's perception of affordances of the task-specific cognitive system made up of constraints including other athletes, the physical environment, and the rules of the game \citep{Passos2012}.

Very little direct empirical evidence specific to rugby can be used to substantiate a link between joint action and team click, and team click and social bonding.  Rugby is, however, a sport heavily associated with ``social bonding'' in the more popular discursive sense, particularly in all-male social organisation common in the elite educational institutions of England and Commonwealth countries in which rugby originally developed \citep{Dunning2005,Richards2007,Collins2008}.\footnote{Rugby union has been the site of much criticism worldwide due to the fact all-male social spaces cultivated by rugby appear to support and sustain hyper-masculine and hyper-normative behaviours, including gender-related violence \citep{Cosslett2014,Guinness2016}.
}   ``Rugby is a game for barbarians played by gentlemen,'' or so the saying goes.\footnote{The origins of this oft-cited Rugby adage is unclear.  The phrase is supposedly the adopted motto of the British Barbarians Football Club, established in 1890 \citep[34]{Dunning2005}.  The complete phrase reads ``Rugby is a game for barbarians played by gentlemen, football is a game for gentlemen played by barbarians.''  However, official club history cites its original motto as, ‘Rugby Football is a game for gentlemen in all classes, but for no bad sportsman in any class' \citep[vii]{Starmer-Smith1977}.  Some sources attribute the saying to British writer and poet Oscar Wilde (1854-1900) \citep{Fleenor2015}}. Different inflections on this adage have been reproduced by people in all parts of the world that rugby has reached (including China), presumably as a way of linking the nature of rugby's physical requirements with social virtues of fair play, cooperation, and moral integrity.  Although direct experimental evidence concerning rugby is scant, the physiological demands, joint action complexity, and social-historical trajectory of rugby suggests that it is extremely suited to an investigation of the social bonding effects of joint action in group exercise.

\section{Cultural Variation}
In addition to micro-level details and dynamics of joint action, macro-level variation in the cultural contexts of joint action also vary extensively. Importantly, macro-cultural expectations appear to frame and direct micro-level movement dynamics of joint action.  As sporting anecdote indicates, different teams from different places and times appear to play the same game in very different ways---embodying different ``styles'' of play.  While there is very little literature devoted to examining the effect of cultural variation on joint action and social bonding in particular, there is extensive evidence to suggest that cultural variation impacts on processes of cognition \citep{Nisbett2003,Hoshino-Browne2005}, social learning \citep{Mesoudi2015}, and prosocial behaviour \citep{Yuki2005,Yuki2003}.
It has been suggested that cultural environments structure joint action scenarios in ways that help ``smooth'' coordination by providing equivalent expectations between co-participants \citep{Vesper2017}.  Indeed, as anecdote and observations concerning suggest, perception of ``team click'' is not necessarily limited to the most proximal dimensions of joint action perception, but is rather contingent on the snug fit between a given joint action and a whole assemblage of hierarchically ordered expectations.\footnote{It is also important to bear in mind that, while the neurological, cognitive, and psychological theories from which the predictions of this dissertation strive for universal generalisability, these theories are nonetheless heavily grounded in Western epistemological assumptions, intuitions, and ``WEIRD'' empirical evidence \citep{Henrich2010a}.}


\section{The present Study}

\section{Study Findings}

\section{Dissertation Overview}

\section{Contribution}
