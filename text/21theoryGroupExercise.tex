
\begin{savequote}[8cm]

  \qauthor{}
\end{savequote}



\chapter{A novel theory of social bonding through joint action
\label{chap:theoryGE}}

\minitoc



  \begin{CJK}{UTF8}{gbsn}


\section{A novel theory of social bonding through joint action\label{sect:novelTheory}}

As explained in the previous chapter, The AIF begins with the assumption that, on a basic level, joint action consists of two or more Bayesian inferential systems attempting to minimise free energy by modelling the hidden causes of sensory stimuli.  From sheer computational perspective, joint action is an inherently challenging task, but it appears that ``general synchronisation'' derived from the coupling of two or more agents facilitates a shared narrative that sets the foundation for functionally equivalent generative models for joint action.  This perceptual common ground resembles a situation in which ``adjustments for ``Am I?'' and ``Are You?'' and ``Where Are We?'' and stuff like that...'' (see Section~\ref{sect:teamClickIntro}) dissolve, and a transcendent ``we-mode'' of collective agency is glimpsed \citep[][]{Friston2015}.  Anecdote and observation of real-world joint action scenarios suggests that achieving the ``click'' of joint action---in spite of its uncertainty, and however momentarily it lasts---can be a deeply rewarding experience.  In the case of joint action in group exercise contexts, it appears that various features of group exercise serve to amplify both the challenges of joint action, as well as its psycho-social rewards.

The explanatory power of the AIF for team click and social connection in joint action lies in its capacity to incorporate various cognitive strategies within one overarching mandate of free energy minimisation \citep{Clark2015}.  The distinction between cortical and extra-cortical, mental and emotional, model-based and model-free cognitive processes dissolves, and in place of these dualisms, a continuum of cognitive strategies are flexibly deployed to exploit brain, body, and bio-external resources \citep{Pezzulo2013}.  Bayesian optimisation facilitates flexibility: the volume on sensory prediction error signals is either ``dialled-up'' or ``dialled-down,'' depending on their reliability judged by prior experience (inverse of variance of the prior distribution).  Thus, just as the binary distinction between mental and emotional collapses, so too does the distinction between external and internal, exteroceptive and interoceptive \citep[and proprioceptive][]{Seth2013}.

This conception of joint action allows for a full consideration of the interlocking physical, cognitive, and social processes outlined in in Chapter~\ref{chap:intro}.  Anecdotal and observational evidence suggests that team click is at once mental and embodied, cognitive as well as emotional, metaphysical (i.e., mysterious) as well as deeply grounded in visceral sensations (see Chapter~\ref{chap:intro} Section ~\ref{sect:adrian}).  The active and embodied approach of the AIF allows for an account of the surprising, visceral, and agentic dimensions of team click; while the hypothesised coupling between generative models and cultural affordances explains how visceral experience of team click can set the foundation for higher-order processes of emotional affiliation, perceptions of common goal, and social identity formation with co-participants.

In the next section, I outline the core relationships of a novel theory of social bonding through joint action.  The theory predicts three causal paths: 1) joint action to team click, 2) team click to social bonding, and 3) a direct path from joint action to social bonding, 4) mediated by team click.

I then outline the application of AIF to group exercise in particular, explaining how group exercise environments appear almost deliberately rigged to produce extreme levels of cognitive uncertainty, while at the same containing social and cultural affordances capable of attenuating such uncertainty (Section ~\ref{}).  Finally, I present a novel theory of social bonding through joint action, and outline general predictions that emerge from this theory.


%The first relationship is between joint action and team click. Team click in group exercise resembles a scenario in which individuals perceive success in joint action.  The tacit but powerful experience of surprise, agency, and certainty (about the reliability of co-actors) deriving from team click sets the foundation for higher-order cognitive processes by virtue of their connection with these processes via an intricate  helix of dynamically coupled generative models and cultural affordances.  Thus, the predicted relationship between team click and social bonding, if proven true, will have implications for ultimate processes, such as social cohesion and cultural transmission.


    \section{Predictions of the theory}


        The overarching prediction of this thesis is that the psychological phenomenon of team click mediates a relationship between joint action and social bonding.

        Within this main hypothesis, I also formulate the following sub-hypotheses:
        \begin{enumerate}
          \item Athletes who perceive greater success in joint action will experience higher levels of felt ``team click.'' I predict that relevant perceptions of joint action success will relate to athlete perceptions of:
            \begin{enumerate}
              \item a combination of specific technical components; or
              \item an overall perception of team performance relative to prior expectations; or
              \item an interaction between these two dimensions of team performance.
            \end{enumerate}
          \item Athletes who experience higher levels of team click will report higher levels of social bonding.
          \item More positive perceptions of joint action success will predict higher levels of social bonding, driven by more positive:
          \begin{enumerate}
            \item perceptions of components of team performance; or
            \item violation of team performance expectations; or
            \item an interaction between these two predictors.
          \end{enumerate}
        \end{enumerate}

    In addition to these core predictions, I also make the following predictions for the role variation to the theory based on cultural and individual variation:

    \begin{enumerate}
      \item Individual variation in predispositions towards different movement coordination strategies will influence the relationship betweens joint action, team click, and social bonding.  In particular:
          \begin{enumerate}
            \item Athletes with more prosocial disposition (measured by personality type, e.g. extroversion) will experience higher levels of team click and social bonding in joint action.
            \item Athletes with higher levels of technical competence or experience will experience lower levels of team click and social bonding to the team, do to a lack of surprise associated with the experience.
          \end{enumerate}

      \item Informational affordances that are more dominant in an ecology will have a higher impact on shaping patterns of behaviour in joint action.

    \end{enumerate}



\subsection{More positive perceptions of success in joint action will predict higher levels of team click\label{sect:JASuccessTeamClick}}

    The components of team click outlined above indicate that this commonly observed phenomenon contains elements of positive expectation violation or surprise, deriving from an experience of tacit or implicit coordination in joint action.  The phenomenology of team click can also involve the blurring of boundaries between self and others in the team, the perception of atmosphere or aura surrounding the group, as well as a perception of reliability of teammates and the self to successfully perform joint action.




\subsection{Higher levels of team click predict higher levels of social bonding \label{sect:teamClickSocialBonding}}

    The combination of positively valenced surprise deriving from dialled-up interoceptive (``emotional'') and dialled-down exteroceptive inputs, a blurring of agency between self and other due to the strain on proprioceptive inputs during ``in the moment'' movement execution appear to provide powerful ingredients for flow on psycho-social effects.  Team click appears to be associated with both an atmosphere around the team and reliability of teammates, both of which can be understood as social cognitions.  Is team click responsible for generating feelings of emotional closeness, a sense of shared goal, and shared identity with teammates?  In other words, is it possible to identify a link between team click and social bonding?

    The link between interpersonal coordination and social bonding has been addressed in the behavioural mimicry and synchrony literatures \citep[e.g.,][]{Wheatley2012,Launay2016,Mogan2017}, but there is less substantive evidence in relation to dynamic interpersonal coordination in natural joint action settings such as those found in group exercise contexts \citep{Marsh2009,Miles2009,Lumsden2012}.  More recently, however, analysis of dynamic coupling of co-actors in joint action scenarios reveals that synchronised movement implicates an array of implicit and pre-perceptual cognitive processes of alignment and prediction error minimisation \citep{Schmidt2011}, which, in addition to more explicit forms of communication, could be central to the generation of feelings of self-other merging, self-other distinction, and perceived reliability and trust associated with social bonding \citep{Marsh2009}.


%\subsection{Team click will mediate a direct relationship between joint action and social bonding in group exercise\label{sect:JASuccessSocialBonding}}

\subsection{Cultural affordances}

Evidence suggests that informational affordances provided by the specific cultural milieu can also serve to shape patterns of behaviour relevant to joint action.  The term ``culture'' can be understood as shared elements that provide standards for perceiving, believing, evaluating, communicating, and acting among those who share a language, a historical period, and a geographical location \citep{Triandis1996}.  As cultural psychologists Kitayama and Markus (2020, p. 422) explain, culture is a:

    \begin{quote}
      ...stand-in for a similarly untidy and expansive set of material and symbolic concepts...that give form and direction to behaviour [and that] culture is located in the world, in patterns of ideas, practices, institutions, products, and artefacts.
    \end{quote}

    A key insight overlooked by the existing social high account of group exercise and social cohesion, but revealed by the paradigm shift surrounding active inference, is the sensitivity of joint action (or any cognitive process for that matter) to informational affordances provided by various layers of ecological and cultural context.  Affordances for joint action appear to be dictated by processes operating at multiple conceptual levels---from the micro-level predictive processes associated with movement action and perception, to the macro-level predictive frames offered by specific cultural and contextual niches---interact in complex processes of reciprocal causation to shape joint action.  Conceptualisation of the causal complexity of cognitive processes relevant to joint action in this way echoes a broader reconceptualisation of the causal complexity associated with change on an evolutionary timescale, which recognises that human behavioural phenomena is the result of a number of biological, cognitive, and ecological mechanisms that interact via reciprocal feedback loops spanning varying scales of time and space \citep{Fuentes2015}.

    In the context of joint action, conventional affordances can be understood as shared frames of reference---``hyper-priors'' that set the macro-contextual coordinates for joint action \citep{Clark2013}. Joint actions that involve complex sequences and divisions of labour between participants appear to rely heavily on capacities to explicitly signal intention for the assigning of roles, forward planning, and repair of failed coordination \citep{Frith2010}.  These ``coordination smoothers'' \citep{Vesper2017} often function to reduce spatial and temporal variation in action by providing a shared spatiotemporal referent for co-alignment of predictions.  Cultural conventions are thus examples of effective framing devices for joint action.  Depending on the context of the joint action, it could be subject to a pre-existing, mutually recognised power relations typical in the established culture (e.g., favouring hierarchical or egalitarian communication, \citep[see]{Cheon2011}) and the particular situational context (e.g., formal or informal).  Establishing roles, such as leader or follower, also has a similar smoothing effect, and often the affordances in the task environment shape the smoothing strategies available to co-actors \citep{Marsh2009}.
    or the ``common narrative'' that stabilises alignment of predictions \citep{Friston2015}.


    \subsection{Individual differences\label{sect:individualDifferences}}

    There is evidence to suggest that individual differences may play a role in joint action processes.  Research suggests that preexisting dispositional tendencies in sociality dimensions of personality (e.g. extroversion, agreeableness) and social orientation (locus of control, communication styles), as well as the nature of pre-existing interpersonal relationships, and technical competence in joint action \citep{Novembre2014}, will impact on the structure and quality of interpersonal movement coordination.  Individual differences may also influence the social effects of joint action \citep{Marsh2009}.

    The personality dimension of empathy—--understanding others’ thoughts and feelings—--has been linked to anticipatory mechanisms related to action simulation \citep{Sevdalis2014,Keller2014}.  In the piano studies mentioned above \citep{Novembre2012}, scores on the ``perspective-taking'' sub-scale of an empathy questionnaire correlated positively with neurophysiological measures of representing the other’s part in their own motor system, as well as how much this ``other-representation'' was relied upon for coordination \citep{Novembre2014a}.  In a synchronised finger-tapping task, Pecenka and Keller (2011) found that scores on a perspective-taking questionnaire correlated with the degree that individuals predicted event micro-timing in a tempo-changing pacing sequence.  \textcite{Richardson2007} found in a dyadic plank moving experiment, individuals’ levels of agreeableness and extroversion were positively correlated with persistence of cooperation in the task.

    In addition to personality type, social orientation and motivation have also been shown to effect interpersonal coordination.  A study of unintentional coordination revealed that prosocial-oriented individuals spontaneously synchronised arm movements with others more than pro-self-oriented individuals, whether their social/self-orientation reflected their pre-existing disposition or resulted from an experimental manipulation \citep{Lumsden2012}.  Studies have found that interacting with a late-arriving partner reduced stepping synchronisation, compared with interacting with a partner who arrived on time \citep{Miles2010}, and bodily synchrony decreased during arguments compared with affiliative conversations \citep{Paxton2013}.  A recent study addressed the relationship between locus of control (i.e. the degree to which life events are perceived to result from one’s own actions) and temporal adaptation (error correction) \citep{Fairhurst2014}.   Results indicated that individuals with an internal locus of control (who attribute the cause of events to their own actions) engaged in less phase correction than individuals with an external locus of control (who attribute events to external factors), which may reflect individual variation in predispositions towards different movement coordination strategies.

    This assertion is corroborated by a study conducted by \textcite{Schmidt1994}, which used interpersonal wrist-pendulum coordination to investigate the effects of self-reported social competence \citep[c.f.][]{Riggio1996} upon social coordination stability.  Subjects were selected to create homogeneous social competence dyads (High–High or Low–Low pairs) and heterogeneous dyads (High–Low pairs). The heterogeneous (High–Low) pairs demonstrated significantly greater stability and fewer breakdowns in coordination than the homogeneous (High–High and Low–Low) social competence pairings, suggesting that reciprocity (leader-follower) rather than symmetry (leader-leader or follower-follower) of social competence facilitates social coordination.  In this sense,  ``internal'' individuals may stabilise the tempo of their own performance (at the expense of synchrony) and take a leader role, whereas ``external'' individuals may synchronise with their partner (at the expense of maintaining a steady tempo) and take a follower role.  In sum,  dispositional tendencies in movement coordination and in sociality dimensions might set the initial conditions that make pull to the cooperation attractor stronger (or weaker) than a pull to the autonomy attractor (independent action).

    In addition to the research sketched above, there is considerable research detailing links between neuropsychiatric disorders such as autism and schizophrenia and deficits in interpersonal movement coordination \citep{Frith2013,Wheatley2016}.  Autism spectrum disorder appears to be associated with a deficit in the capacity to process the communicative function of emotional expression. In particular, while children on the autism spectrum may learn to recognise a frown in a photograph, they are relatively impaired in decoding dynamic social cues in real time \citep{Hobson1986}.  The ability to recognise emotional voice and music (in healthy participants) has been localised to the mid-posterior extent of the right superior temporal cortex---this region of the cortex is most often associated with the visual perception of biological motion \citep{Pelphrey2005}.  Biological motion, a dynamic stimulus, is known to engage a different neural pathway than static, structural features \citep{Haxby2000}.  As such, the ability to process emotion from static images may not translate to equivalent ability to process emotion from visual motion cues.

    It is interesting to note that a failure of sensory attenuation – in particular the relative strength of sensory and prior (extrasensory) precision – has been proposed as the basis of autism – whose cardinal features include an impoverished theory of mind (Happe  Frith, 2006; Lawson, Rees, Friston, 2014; Pellicano Burr, 2012; Van de Cruys et al., 2014).

    %Autism: Low cog biases Hyper reality Sensory acuity

    %schizophrenia:High cognitive biases
    %Apophenia: spontaneous perception of connections and meaningfulness of unrelated phenomen


    %Grau-Moya, J., Ortega, P. A., & Braun, D. A. (2012). Risk-sensitivity in Bayesian sensorimotor integration. PLoSComputationalBiology, 8, e1002698. doi:10.1371/journal.pcbi.1002698





\section{Active inference in group exercise \label{sect:activeInfGE}}

Joint action in group exercise creates a unique environment for social cognition, defined by extreme levels of cognitive uncertainty.  Group exercise involves high cognitive load associated with complex joint action requirements involving many actors, many hierarchical layers of joint goals over various sensory modalities and spatiotemporal scales. The ``in the moment'' execution demands also constrain cognitive processing, and encourage a reliance on cognitive resources located in extra-neural and bio-external domains \citep{Bourbousson2016}.  In addition, strenuous physical exercise could also entail neurocognitive tradeoffs that further strain individual ability to reduce free energy in joint action \citep{Dietrich2004b}.

Evidence discussed above suggests that optimal solutions to joint action typical of group exercise may tend to favour the recruitment of more extra-neural resources as a way of minimising free energy, whereas less efficient solutions to joint action may rely on more computationally intensive procedures in order to reduce free energy (see Section ~\ref{sect:extraNeural}).  The social cognition of these processes in joint action have not yet been closely considered \citep[but see ][]{Marsh2009,Lumsden2012}.




    \subsection{Group exercise spikes uncertainty of joint action}


    \myparagraph{Group exercise involves multi-agent (and not just dyadic) joint action}
    Above and beyond normal day-to-day instances of communication and exchange, joint action in group exercise contexts such as sport place extreme cognitive load on participants. The active inference approach to joint action outlined above is based on preliminary models of dyadic joint action involving turn taking \citep[i.e., in bird song exchanges][]{Friston2015}.  Group exercise contexts, particularly modern sport contexts, often involve large numbers of co-participants, in either ``inter-active'' or ``co-active'' modes of coordination.
        \footnote{
        In co-active sports (e.g., bowling, archery), team members perform separately and the team outcome is a product of combined individual performances. In interactive sports (e.g., volleyball, soccer, rugby), goal accomplishment requires the establishment of complex patterns of interaction and coordination among team members \citep{Filho2014}.
        }
    Thus, achieving cognitive synchronisation in joint action of group exercise contexts may be much more difficult.  Evolutionary Anthropologist Robin Dunbar \textcite{Dunbar1992} proposes that the ratio of human neocortex size to total brain volume imposes an upper cognitive limit on real-time coordination of behaviour of approximately four to five individuals.  The sheer computational burden of modelling multiple agents in group exercise may place an unmanageable cognitive load on our normal healthy processes of active inference.  Indeed, at the very least, multi-agent joint action poses a challenge for the existing theoretical model for joint action (PJAM), which is formulated primarily based on dyadic interactions \citep{Pesquita2017}.

    \myparagraph{Joint action in group exercise is ``on-line'' and ``in-the-moment''}
    Particularly in the case of interactive team sports, interpersonal movement coordination is often executed ``on-line'' and ``in the moment,'' as opposed to step-by-step turn taking.  This fact poses a challenge to Friston and Frith's proposal that active inference in joint action comprises two modes (either actively attending to sensory stimuli, or else moving while in a state of sensory attenuation, see Section ~\ref{sect:activeInfJA} above).  Exactly what occurs when actors need to concurrently move and sense others moving at the same is poorly understood.  What happens to the precision weightings---the volume gauges---on proprioceptive, interoceptive, and exteroceptive prediction errors In instances of dynamic interactive joint action involving co-occurence of movement between agents in joint action?   What is the impact of on-line and in the moment join action on the experiences of agency in group exercise?  Empirical research is yet to provide answers to these questions.

    \myparagraph{Joint action in group exercise often involves competition}
    As if the cognitive load of multiple agents and on-line coordination of complex schemas for joint action was not enough for the humble human brain, interactional team sports also usually involves \textit{competition}.  While competition in sport is usually adorned with elaborate social meanings surrounding the ethics of winning and losing \citep{McNamee2008}, on a cognitive level, competition in joint action entails one individual or team of individuals actively attempting to foil or disrupt the predictive models of another individual or team of individuals \citep{Reimer2006}.  The competitive dimension of interactional team sports thus serves to further spike cognitive uncertainty between co-actors in joint action.  The uncertainty involved in competitive joint action scenarios could have further implications for the ability of second-order Bayesian inference concerning the reliability of sensory inputs \citep{Pezzulo2014}.

    \myparagraph{Group exercise involves metabolic tradeoffs in the brain}
    In addition to these heightened cognitive challenges associated with complex and dynamic joint action in group exercise, high levels of physiological exertion characteristic of group exercise could also serve to spike uncertainty.  Neuroscientist Arne Dietrich draws attention to the fact that physical exercise is at its core a stressor that places extreme energy demands the organism \citep{Dietrich2011}.
    Such a situation will necessitate an energy tradeoff in the brain, whereby energetically costly brain regions inessential to movement execution are temporarily downregulated \citep{Dietrich2004b}.  Dietrich suggests that experiences of flow and the ``runner's high'' in exercise could be the result of temporary downregulation of energetically costly brain regions inessential to movement execution  \citep{Dietrich2004b}.  Dietrich and colleagues propose candidate areas of the dorsolateral prefrontal cortex responsible for self-monitoring and proprioceptive sensory attenuation \citep[commonly known as the ``inner critic'' regions of the brain, see][]{Limb2008}.  It is currently not well understood precisely whether or how neurometabolic tradeoffs in the brain could impact upon joint action and the experience of team click.  However, it is conceivable that metabolic tradeoffs in the brain owing to prolonged physiological stress may have implications for the second-order inferential processes of precision weighting sensory inputs.


    \subsection{Free energy minimisation in group exercise demands greater reliance on extra-neural affordances \label{sect:extraNeural}}
    To summarise, the combination of cognitive demands associated with tracking and modelling multiple agents ``in-the-moment,'' the neurometabolic tradeoffs associated with (often extreme) levels of physiological exertion, and even the competitive dimension of some group exercise contexts (for example interactive team sports), could create an environment in which humans' usual cognitive capacities are strained and compromised.  The fact that experiences of team click are particularly prevalent in group exercise contexts (compared to more mundane or quotidian instances of joint action) suggests that amplification of uncertainty and stress in group exercise could be a critical factor in facilitating powerful psychosocial effects.

    The active inference approach would predict that individuals, when faced with the extreme cognitive uncertainty of joint action in group exercise, will tend to preference mechanisms that maximise uncertainty reduction (minimise uncertainty).  In the case of joint action in group exercise, this may entail reliance on predictive models that outsource the computational cost to affordances beyond the brain, or at least the metabolically expensive cortical areas of the brain \citep{Dietrich2004,Clark2015}.  In the case of highly skilled practitioners, whose predictive models for action have been finely tuned to the affordances of the task environment (inlcuding co-actors), it is plausible that extra-neural and even extra-personal resources (e.g., physical features of the task environment) could provide a more cognitively efficient and effective route to the performance of successful joint action and thus the minimisation of free energy.

    Various strands of evidence support these predictions.  Studies of highly skilled practitioners in joint action demonstrate that more technically competent practitioners generate more accurate predictive models for joint action than less technically competent practitioners \citep{Tomeo2012,Aglioti2008,Mulligan2016}.   In studies involving skilled versus non-skilled practitioners in dyadic interactions, it has been shown that more skilled practitioners create stronger dynamical coupling through flexibly modulating their actions with others \citep{Schmidt2011,Caron2017}. These findings are corroborated by other studies that find that professional footballers (versus novice controls) are able to more accurately predict the direction of a kick from another player's body kinematics (\cite{Tomeo2012}, see also \cite{Aglioti2008,Mulligan2016} for similar results with basketball and dart players).  When analysing co-regulation between members of basketball teams, \textcite{Bourbousson2015} showed that more expert teams made fewer mutual adjustments (at the level of the activity that was meaningful for co-actors), suggesting an enhanced capability of expert social systems to achieve and maintain an optimal level of awareness during the unfolding activity.

    A recent field study with expert rowers revealed that athletes predominantly utilised
    extra-personal (rather than inter-personal) regulation processes in order to facilitate and sustain joint action, and attention to interpersonal regulation occurred only during instances of expectation violation concerning performance in joint action \citep[; for a full explanation of this study, see Appendix ~\ref{app2:theory} Section ~\ref{sect:rowerStudy}]{RKiouak2016}. These results suggest that athletes used the affordances of the environment to mediate the arrangement of individual and joint activities \citep{Bourbousson2011,Bourbousson2012}.  Taken together, this evidence supports the prediction for a tendency for co-actors in joint action to utilise neurocomputationally conservative models and coupling with extra-neural affordances under circumstances of high levels of free energy (such as those common to group exercise).

Parallel strands of research in psychology \citep{Prinz1990,Prinz1997,Prinz2013}, neurophysiology \citep{Rizzolatti2004,Rizzolatti2010}, and neurocognition \citep{Wolpert1998,Wolpert2000} suggest that interpersonal behavioural coordination in joint action is facilitated by the intrinsic links between action perception and action execution in the human brain.  In essence, action-perception links refers to the ostensive co-occurrence of a stimulus for action and its motor representation.  For example, for individuals who have mastered a certain sensorimotor task, the representation of a perceptual effect (say the sound of a middle-C on a piano) can trigger the movement necessary to produce the effect itself (motor instructions for playing the middle-C key on a piano) \citep{Novembre2014}. Evidence suggests that skilled individuals not only develop generative models for self action, but also for the actions of others in joint action \citep{Novembre2012}. Action-perception links can be used for monitoring and integrating (e.g., timing or combined pitches) the actions of other ensemble members with self-generated actions \citep{Loehr2013}, and these effects appear to be stronger in individuals with high perspective taking skills \citep{Novembre2012,Loehr2013}.  The overlap between mechanisms for action production and action observation suggests that individuals may represent their own and others’ actions in a commensurable format.  Training-induced motoric representation of self and others' actions may facilitate various capacities important for joint action, such as prediction, adaptation, and entrainment (for a more detailed treatment of action-perception links and their relevance to joint action, see Appendix~\ref{app2:theory}).



    \subsection{Team click and the dark room dilemma}
    The evidence reviewed above demonstrates the human ability to flexibly deploy a range of cognitive strategies, seemingly towards the ultimate goal of free energy minimisation.  Inherent in the application of the free energy principle to group exercise is the paradox that group exercise is an activity that appears to deliberately amplify uncertainty.
    The details of these proximate mechanisms could shed light on the phenomenon of team click and its associated downstream psychosocial effects, including social bonding and social cohesion.  Joint action in group exercise, and the phenomenon of team click in particular thus offers a fascinating space in which to investigate the social cognition of active inference in joint action.  Active inference provides a theoretical frame through which the social cognition of joint action can be more clearly rendered and comprehended.

    %How, then, do we explain the widespread prevalence of group exercise contexts, which appear to be riddled with uncertainty owing to the overwhelming complexity of joint action requirements and extreme physiological demands?  The phenomenon of team click could lie at the heart of an explanation to this paradox.



    \subsection{Group exercise offers ready-made semantic inputs}



    Team Membership

    Virtue plays of nation etc

    movemen










































































  \end{CJK}
