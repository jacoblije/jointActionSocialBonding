\begin{savequote}[8cm]

  It takes two to know one.

  \qauthor{--- Gregory Bateson}

\end{savequote}


%I do not see any way to avoid the problem of coordination and still understand the physical basis of life.
%\qauthor{--- H. H. Pattee  \textit{The role of instabilities in the evolution of control hierarchies} 1976}


\chapter{\label{chap:theory}Team click and social connection in joint action}

\minitoc


                                  \begin{CJK}{UTF8}{gbsn}

\section{The development of visceral agency in rugby's newest recruit\label{sect:SHW}}
Sun Hongwei arrived, escorted by his high school athletics coach, to the Beijing Temple of God of Agriculture Institute of Sport (hereafter the Institute) soon after I began my fieldwork in August 2015.  An 18-year-old with a slight build and timid demeanour---his gaze remained diverted to the ground during his first few months at the Institute---Hongwei later told me that he had never seen a rugby ball before that day he arrived.

Hongwei was from Hebei province, immediately surrounding the special prefecture of Beijing, China's capital.  Hongwei's coach had organised a trial for Hongwei with the Beijing Provincial men’s and women's Rugby Program (hereafter the Program) by calling upon social connections to the leadership of the Institute.  Athletes come to the Rugby Program from all over the country.  As I explain in more detail in Chapter~\ref{chap:researchSetting}, to represent Beijing at a provincial level in a sport like rugby can translate into the opportunity to gain entrance to one of China's top universities and enhanced career employment opportunities thereafter.

Rugby is not a popular sport in China, but its recent inclusion in the Olympic games (in the form of the modified seven-a-side version of ``rugby sevens'') means that it now occupies a prominent place in the Chinese sport system.  Rugby programs such as the one at the Institute now exist in 12 of China's 34 provincial level regions, either embedded within, or somehow associated with, tertiary education institutions.  Thus, although rugby and China are not commonly associated terms, rugby now affords Chinese athletes a rare and under-capitalised opportunity to pursue attractive life-course opportunities of education and employment in an intensely competitive education system.

Without exception, the athletes who arrive at the Institute to join the rugby team were not like me; they had not spent their childhoods playing rugby in their schoolyards or watching professional rugby on television. Many who come to rugby transition from other more popular sports such as athletics, basketball, or association football, and often---like Hongwei---have never seen a rugby ball before they arrive.  Most athletes ``start from scratch,'' so to speak, in terms of their grasp of the requirements of the highly interactive and technically complex team sport.  In addition to complex patterns of movement coordination, rugby also involves unrestrained body-on-body collisions and intense bouts of high physiological exertion.  To perform all of rugby’s technical requirements successfully requires a combination of speed, strength, agility, and endurance (see Chapter~\ref{sect:rugbyUnion}).  Learning the game of rugby from a baseline of essentially zero, while also navigating the inevitably demanding social and political dynamics within the team and the Institute, was clearly going to be a daunting task for Hongwei.


                        \begin{center}
                          * * *
                        \end{center}

 Even compared to other newly arrived junior athletes, I noticed that Hongwei was particularly timid and shy, especially in his interactions with the senior players and coaches (myself included).  Nevertheless, Hongwei clearly signalled diligence and commitment through his participation in team activities.  He arrived early to each training session, and carried more than his fair share of the training equipment---a task shared by the most junior members of the team.  Each time I passed Hongwei in the corridors of the Institute he would greet me with a polite bow and greeting, ``Hello Coach'' (\textit{'jiaolian hao'} `教练好').  In these instances, Hongwei would coordinate his greeting with a moment's eye contact, only to return his gaze to the floor and continue walking.

Due to his lack of familiarity with the basic techniques of rugby, Hongwei was initially unable to fully participate in normal training with the rest of the team.  Instead, during the first month or so, Hongwei remained on the sidelines of the field and practiced the basics with other athletes who were unable to fully participate in training due to injury. He began with the absolute basics: learning how to pass and catch the rugby ball, while stationary and running in-motion. In my eyes at least---eyes of an observer accustomed to instinctual grasp of these movements from a young age---I found Hongwei’s attempts to accustom himself with the skills of rugby jarring. The idiosyncrasies of rugby’s ovular ball often foiled him.  I would regularly see him on the sidelines of training chasing after a ball that he’d just fumbled; the idiosyncratic shape of the ovular rugby ball meant that it would change direction like it were a scurrying rabbit tactfully evading its pursuer.

                        \begin{center}
                          * * *
                        \end{center}

I interviewed Hongwei approximately six weeks after he arrived at the Institute.  Hongwei's demeanour during the interview was consistent with the timid and shy one that he presented publicly at training.  As I explain below, he did show some signs of captivation with his new sport and social environment.  When I asked about his initial impressions of the on-field demands of rugby, however, Hongwei was quick to confess that he felt utterly unacquainted:

  \begin{quote}
    SHW: I still haven’t really started to practice any of the team plays or anything; all I can do so far is pass and run a little bit...(but) it's quite fun! \\
    JT: What do you think is the most difficult component of rugby? \\
    SHW: Um\textellipsis well, coordinating with teammates [on the field], particularly coordination in attack.  Because I can't figure it out.  When I first arrived, I didn’t even know what a ``switch play'' or a ``blocker play'' was.
  \end{quote}

  \begin{quote}
    SHW: 战术没怎么接触,就是像传球啊、跑动什么的会一点了 \\
    JT: 感觉怎么样?\\
    SHW: 挺好玩的!\\
    JT: 你认为橄榄球最难的一部分是什么? \\
    SHW: ...打配合,进攻的配合,因为搞不明白,刚来的时候也不知道什么交叉,后插什么的 \\
  \end{quote}

Hongwei was self-critical in this confession regarding his minimal grasp of the technical requirements of joint action in rugby.  The obvious stress and anxiety he expressed was indicative of a broader trend with the other athletes I interviewed, as I will explain in more detail in the ethnographic sections of this dissertation (see Chapters ~\ref{chap:ethnoField}\nobreakdash~\ref{chap:ethnoResults}). When athletes were asked in interviews about the most difficult aspect of rugby, on-field coordination with teammates was by far the most common answer, particularly among the new recruits and most junior athletes.

As I directed the interview towards topics beyond the on-field technical demands of rugby, Hongwei was more positive, framing rugby as an exciting new opportunity, and commenting that his friends and family were in awe of the fact that he is playing such an impressively ``strong'' (\textit{qiangzhuang} 强壮) physical sport like rugby.  When I asked him about something new that he had learnt through playing rugby, Hongwei automatically responded by emphasising the social dimensions to his experience at the Institute:

  \begin{quote}
    ...I think it's mainly this thing of having teammates. Before, when I was training for an individual sport, it was just me training by myself. [In that environment] it was a case of whoever trained well was successful.  But now with this team of brothers, elder teammates will take care of younger teammates. We all train together, and if you can’t do something, you can always ask your elder teammates...[Rugby] is so much better, because in an individual sport, if you can't master something, you have to go to your coach for help. Other athletes don't want to teach you, because if you surpass them, then they have to work even harder to keep up... I have had to learn about helping others, because rugby is not like an individual sport, where you look after your own performance and that's it.  In a team sport, if you don't do well, there's no need to get too frustrated or upset, because other athletes will help you out, and I will also help others out, that type of collaboration with each other.
  \end{quote}

\begin{CJK}{UTF8}{gbsn}
  \begin{quote}
    我觉得主要是师哥师弟的这一块儿,原来练个体项目都是自己练自己的,谁练好了谁厉害,但是现在师哥师弟,有师哥照顾师弟带着,互相练,我不会我可以问师哥
    ...因为个人项目你不会就必须要找教练,但是别人不愿意教你,因为你把别人超越了那别人还还得努力。 学到互相帮助,因为向个体项目自己成绩自己来拿就行,而像团体项目,即使自己做不好,也不用太泄气太沮丧,因为别人会帮你做好,我也会帮别人做好,互相协作的那种.
  \end{quote}
\end{CJK}

Hongwei's explicit reference to the collegiality of the team, and his position as junior member, highlights that the technical skills of rugby were not the only novel components of his experience.  Hongwei's background was in track and field (his event was pentathlon, very much an individual sport), and the team environment was completely new to him.  This was also the case for many of the other athletes in the team.  As I listened to his experiences associating rugby and group membership, I could not help but associate the quality of these explicit declarations of group membership with his overly mechanical imitation of rugby's foundational techniques.  In both I saw his willingness and desire to signal commitment; but both lacked---at the outset at least---a certain level of grasp or conviction.  In the case of his explicit celebration of rugby's social resources in interview, for example, I got the impression that Hongwei was telling me what he thought he ought to say, without actually deeply believing these things to be unequivocally true.


                            \begin{center}
                              * * *
                            \end{center}

A few months passed, and Hongwei continued to train.  He was as eager and committed as when he began, and I did notice some gradual improvement in his grasp of rugby's basic skills.  But he also remained extremely reserved, keeping his head low at all times in team settings, unless addressed by senior players or coaches.

Then, one evening when I had returned to my room in the Institute dormitory from a three week hiatus in Australia for Christmas, I heard a knock on my open door, and to my surprise Hongwei took an assertive stride into my room, carrying in two arms a draw-string bag containing rugby balls (which were in need of more air before the next day's training session).  Hongwei had never ventured into my room before, apart from when I invited him in for our first interview two months earlier---but certainly never before had he entered on his own accord.  Hongwei looked me straight in the eyes with his head held high and energy radiating from his face and upright chest.  I could not help but smile and ask, with genuine intrigue, ``How have you found training recently?'' (最近练得怎样?)
``Very good'' (很好)he said, assertively and excitedly.  ``Much better than before.  At least now I know what’s going on at training, I can keep up with the plays!'' (比以前好多了, 至少现在训练的时候知道该干什么,战术什么的能跟得上!) A big smile spontaneously grew on his face as he continued to hold my gaze confidently.  ``Oh good!'' I said, a little bit shocked.  I congratulated him for his hard work in training while I had been away, and encouraged him to keep at it.  At this, Hongwei took leave by bowing politely and saying ``Thanks coach'' (谢谢教练).  ``Wow,'' I remember thinking to myself.  What quantity had all of a sudden possessed Hongwei? Was it possible that his diligent adherence to rugby over those first four months at the Institute had instilled him with a visceral sense of personal and social agency?

                          \begin{center}
                            * * *
                          \end{center}


  %  the fact that aspects of Hongwei's personal and social demeanour at the Institute appeared to covary with his familiarity with the technical requirements of rugby over time suggests a relationship between joint action and group membership that is worthy of further investigation.
    %In the same way that Adrian's opening monologue (Chapter~\ref{chap:intro} section~\ref{sect:adrian}) suggested a relationship between the on-field dimensions of rugby and social processes between teammates, Hongwei's gradual development














\section{The need for a theory of joint action and social bonding in group exercise}

%HW's change is a change in attitude (Bateson) agency
How can we scientifically account for the unmistakably visceral quality of Hongwei's transformation from timid newcomer to budding Beijing rugby player?  As the story of Hongwei anecdotally suggests, the visceral and social dimensions of experience in group exercise may take time to develop.  Presumably in Adrian (see Chapter~\ref{sect:adrian}) I witnessed the finished product: through many years of on- and off-field engagement with the sport of rugby, Adrian came to personally embody rugby as a social identity, and was thus able to profess its carnal mystery from a place of deep, intuitive knowing.  When I met him during my first stint of research at the Institute, Hongwei was only just beginning this journey.  Over an extended period of ethnographic research it became clear to me that---for Hongwei and others like him---an increase in familiarity with both the on-field technical requirements of rugby and the off-field social requirements of being an athlete at the Institute came an increase in personal confidence, and a sense of group membership.  As explained in the previous chapter, a scientific theory capable of accounting for these dynamic and interlocking physical, cognitive, and social dimensions of group exercise is yet to be fully formulated.

In this chapter, I draw on existing empirical evidence and prevailing models of social cognition to formulate a novel theory of social bonding through joint action.  I introduce the phenomenon of team click as a psychological construct capable of explaining the relationship between joint action and social bonding in group exercise settings.  The theory proposed herein builds upon the hypothesis that group exercise is causally relevant to processes of social cohesion \citep{Dunbar2010,Whitehouse2014,Cohen2017}. In addition to ``cognitive'' processes (as they are narrowly defined by traditional stimulus-reponse paradigm of human cognition \citep[e.g.][]{Marr1985}), recent empirical research programs within cognitive and evolutionary anthropology have begun to draw attention the role of physiological mechanisms (e.g., neurobiological reward and autonomic arousal) in shaping human behaviour, sociality, and cultural evolution.  As \textcite{Dunbar2010} suggest, it is impossible to accept that humans merely adhere to a ``dung fly'' model of social aggregation; clear scientific evidence now suggests, rather, that humans have evolved a species-unique tendency to actively congregate and cohere around shared cultural practices \citep[see][]{Tomasello2005}.

But, as explained in Chapter~\ref{chap:intro}, existing anthropological approaches to the question of human bondedness and social cohesion could benefit from a more dynamical model of cognition, in which physiological and social processes are more integrated in an account of information transfer  \citep{Roepstorff2010,Badcock2012,Ramstead2017}.  Such a model would allow for scientific analysis of a fuller spectrum of component mechanisms and coordination dynamics in joint action that have been traditionally sidelined by cognitive and evolutionary approaches explanations of social cohesion (see Chapter~\ref{sect:cogEvAnth}).  In essence, to fully account for the diversity of profiles, subjective experiences, and social effects of group exercise, cognitive and evolutionary approaches require an appreciation not only of the singular operation of discrete proximate mechanisms (e.g., exercise-induced neuropharmacological reward or synchrony-induced self-other merging), but also a conception of how these discrete mechanisms dynamically \textit{coordinate} with each other in specific individuals and socio-cultural ecologies to produce subjective experience.  In this chapter I consider how physical and social processes interlock in joint action; in order to explain how social connection can emerge in instances in which joint action clicks.

%movement coordination and social communication
From a pure computational perspective, achieving and sustaining joint action is a daunting cognitive task, as it requires the coupling and constraining of various degrees of freedom belonging to autonomous co-actors \citep{Bernstein1967,Turvey1978}.  However, it is now more clearly understood that human cognition—--and indeed, cognition of biological life more generally—--does not rely on passive, ``brute force'' computation---like that of a digital computer---to manage to the challenges of interacting with the environment \citep{Yufik2013}.  Rather, prevailing theories of cognition suggest that humans flexibly and actively deploy a range of strategies in order to anticipate and regulate uncertainties inherent in joint action settings \citep{Friston2010,Clark2015}.
This emerging area of research sheds light on the ways in which dynamical coordination of physical movement is tethered to processes of social communication via an integrative neurocognitive architecture \citep{Semin2008,Wheatley2016,Ramstead2016}.
In particular, greater levels of (actual or perceived) physical coordination appear to generate greater levels of social alignment \citep[e.g., emotional support, perceptions of common goal, and shared social identity][]{Semin2008,Wheatley2012,Launay2016,Mogan2017}.

% GE amplifier
More so than most cultural variants that have traditionally formed the focus of scientific analysis, activities in which group exercise feature demand a dynamic conception of cognition.  Group exercise invariably involves numerous brains and bodies engaged in ``in-the-moment'' and ``on-line'' joint action that unfolds over various timescales and engages multiple sensory modalities. In addition, group exercise often involves sustained and rigorous physiological exertion, which appears to impose metabolic constraints on neurocognitive function.  In the case of interactional team sports, the goal of successful joint action within one team is often compromised by the competitive efforts of another team to foil their opponent's shared goal.  Taken together, all of these dimensions of group exercise serve to spike the cognitive uncertainty inherent in joint action.  Regulation and management of cognitive uncertainty in group exercise requires canny recruitment of and coordination with the resources distributed throughout brains, bodies, and physical features of the specific joint action environment \citep{Clark2015}.  In essence, just as team click can be understood as optimal or maximal coordination of physical movement, social bonding in joint action can be understood as optimal or maximal recruitment of social affordances for the functional purpose of reducing uncertainty in joint action and enabling ongoing adaptive behavioural coordination \citep{Ramstead2016}.

The phenomenology of team click captures a scenario in which coordination in joint action is perceived to be either optimal or maximal.  Accordingly, the surprising, visceral, and agentic dimensions of team click appear to set the foundation for high quality social bonding among co-actors.  While the social relevance of team click and the related concept of flow have evaded rigorous analysis within more traditional paradigms of social cognition \citep[for explanations as to why, see][]{Dietrich2004,Slingerland2014}, within the novel theory of social bonding through joint action proposed herein, team click can be explained in naturalistic terms as a special case of joint action in which the bonding effects of movement coordination are maximised.

In this chapter I formulate a novel theory of social bonding through joint action, mediated by team click.  I begin by introducing the phenomenon of team click---specifically its surprising, visceral, agentic, and social dimensions (Section~\ref{sect:teamClickIntro}).  I then outline existing research suggesting a link between dynamical coordination of physical movement in joint action and social communication (Section~\ref{sect:physicalMovementSocialCommunication}). In particular, this research suggests that greater alignment of expectations, perceptions, and movements in joint action leads to greater perceptions of social connection.  With this theoretical grounding in place, I explain the parameters of joint action typical in group exercise contexts (Section~\ref{sect:JAinGE}), and propose team click as a construct capable of explaining a link between joint action high quality social connection---social bonding---in group exercise (Section~\ref{}).  Specifically, I theorise causal relationships between 1) more positive perceptions of performance in joint action and higher levels of team click, and 2) higher levels of team click and higher levels of social bonding.  I conclude this chapter by outlining the predictions of a novel theory of social bonding through joint action (Section~\ref{}), which I test through empirical research in the chapters that follow.


\subsection{Introducing the phenomenon of team click \label{sect:teamClickIntro}}
In this section I outline the phenomenon of team click, and point to evidence for its potential as an explanatory construct linking joint action and social bonding in group exercise.

One of the big mysteries of competitive team sport, particularly at the elite level, is the elusiveness of peak team performance.  While each individual athlete may exhibit expert level competence in sport specific skills, the much sought after aggregation of these components, i.e., a team that consistently performs ``in the zone,'' and ``firing on all cylinders,'' in reality often proves frustratingly difficult to achieve and sustain.  As \textcite[568]{King2011} note in their ethnography of the 2008 Cambridge University rowing crew who participated (and who were eventually victorious) in the famous annual Boat Race against Oxford University, the search for collective rhythm is a universal in human social interaction, but  the physiological and psychological complexity of finding that rhythm ``...is extremely difficult to attain; collective performance is a possibility not a certainty.''   The moment in which everything ``clicks'' into place in team sport can, for various reasons, disappear as abruptly as it arrives, if indeed it arrives at all.

But, when team click is somehow cultivated, and even sustained, it is celebrated as the ultimate, albeit often inexplicable magic of sporting feats. Consider Leicester City Football Club's unbelievable outhouse-to-penthouse title run in the 2015 English Premier League, the recent dominance of the Golden State Warriors in the American National Basketball Association (NBA), or the astonishingly consistent performance of the New Zealand men's national rugby union team (the ``All Blacks'').
    \footnote{The All Blacks are arguably the most successful sporting team ever, with a winning percentage of 77\% in the last 150 years (and 88\% in the last 6 years) \citep{Kerr2013}.}
All of these successful teams carry with them a powerful ``aura'' associated with their capacity to effectively coordinate their behaviours on the field over extended time scales: individual games, seasons, and, in the case of the All Blacks, entire generations.  The aura associated with such rare instances of collective performance can seduce collective fascination and elaborate exegesis.

    %In this thesis I commit to the more banal task of naturalising the phenomenon of team click, by locating it within a novel theory of social bonding through joint action.

For athletes, coaches, and spectators alike, team click can be a hugely powerful sensation.  As theologian Michael Novak explains, ``[f]or those who have participated on a team that has known the click of communality, the experience is unforgettable, like that of having attained, for a while at least, a higher level of existence'' \citep[11]{White2011}. For the purposes of this dissertation, I use the term team click to describe the phenomenology of peak performance within a team of athletes engaged in joint action, but the term has potentially broader applications beyond group exercise (for example, in music making, spoken conversation, or other similar activities).

As explained in Chapter~\ref{sect:linksComplexClick}, the experience of optimal performance in sport has been extensively documented in the psychological literature of flow \citep{Csikszentmihalyi1992,Jackson1995,Jackson1999,McNeill1995}.  Psychologists and neuroscientists have extensively studied the experience of flow, and have tabled a series of neuropharmacological \citep{Boecker2008}, neurocognitive \citep{Dietrich2006,Dietrich2011,Labelle2013}, and psychological \citep{Csikszentmihalyi1992} theories for its emergence.  The vast majority of flow research has focussed on the experience of the individual---the athlete, musician, or performer.  Some attempts have been made to extended an analysis of flow and its antecedents to the level of the group and dynamics of interpersonal coordination---a phenomenon termed ``group flow'' \citep{Sawyer2006}---but these attempts lack coherence and development.  Anecdote and observational evidence suggests that in addition to the components of flow already identified at the level of the individual, successful performance of technically complex
\textit{joint} action (beyond exact in-phase synchrony) could have distinct psychological and social consequences.

    %The experience of flow has by now been extensively studied by psychologists and neuroscientists, from which a series of neuropharmacological \citep{Boecker2008}, neurocognitive \citep{Dietrich2006,Dietrich2011,Labelle2013}, and psychological \citep{Csikszentmihalyi1992} theories for its emergence have been tabled.  The vast majority of flow research has focussed on the experience of the individual---the athlete, musician, or performer.  Some attempts have been made to extended an analysis of flow and its antecedents to the level of the group and dynamics of interpersonal coordination---a phenomenon termed ``group flow'' \citep{Sawyer2006}---but these attempts lack coherence and development.

Team click shares many similarities with the psychological states associated with flow, but is distinct in that it specifically delineates perceptions of joint action from individual action, and therefore implicates physiological, cognitive, and social mechanisms unique to joint action \citep{Vesper2010}, including the complexity systems dynamics associated with participating in a socially-coordinated, multi-agent system of physical movement \citep{Kelso2009}.  Team click is anecdotally present in a wide range of joint action contexts, and is often associated in these contexts with psychological processes of positive affect and wellbeing, as well as personal agency, social affiliation, and group membership \citep{Jackson1995,Marsh2009,Wheatley2012,Slingerland2014}.


The experience of team click, like flow, also often involves positive violations of individual expectations about action.  While elite athletes presumably always aim for optimal joint action, the experience of actually achieving that desired state, and even exceeding it---often with the help fo co-actors---appears to be a powerful, and somewhat ``surprising'' experience.  In the 1990s, psychologist Susan Jackson conducted a number of qualitative studies documenting athletes' (in this case elite figure skaters) experience of flow in solo and joint action

    \begin{quote}
      Oh, it's awesome! Everything's at peace...the crowd was involved, we were involved as a pair team, and we were just aware of everything, it just kept snowballing, like we weren’t going to miss, and it was great, it was an awesome experience \citep[168]{Jackson1992}.
    \end{quote}

Here, the athlete appears to be surprised and in awe of his or her own performance.  Part of the awesomeness may derive from the rare feeling of absolute alignment between self and co-actor(s):

    \begin{quote}
      What was really so special about this performance was that there was nothing between (partner's name) and I but flow.  You know, through the whole three minutes.  That meant that her mind and my mind were clear and in the same...in a partnership. It's always adjustments for ``Am I?'' and ``Are You?'' and ``Where Are We?'' and stuff like that.  That day was really a marriage of (partner) and (self) and the ice.  So I think it's a different kind of flow experience than a single skater who has much more control over themselves. \citep[173-4]{Jackson1992}.
    \end{quote}

This passage highlights the cognitive complexity of joint over individual action, suggesting the cost and uncertainty associated with attending to and attempting to control the movement degrees of freedom of others in addition to one's own.  The achievement of ``click'' in joint action may instead represent a jitter-less ``co-confidence'' \citep{Noy2015,Noy2017}, in which fixation on the contributions of ``you'' and ``I'' are momentarily abandoned or down regulated, and the agency of ``we'' is realised \citep{Gallotti2013,Friston2015}.

%This passage demonstrates the key elements of distinction between the experience of click in solo and joint action. As I will explain in more detail below, joint action requires constant finessing (adjustments) of behaviours and intentions between co-participants in which individuals are consulting internal models of self, other, and joint action \citep{Pesquita2017}.

Importantly, team click appears to have important flow-on consequences relevant to social bonding and affiliation. Tightly synchronised activity in particular, found in team sports such as rowing, can help dissolve the boundaries between individual and social agency: ``In rowing...it feels like you have at your command the power of everybody else in the boat. You are exponentially magnified. What was a strain before becomes easier. It is absolutely the ultimate team sport'' \citep[x]{Brown2016}.
The blurring of agency between self and team may be responsible for facilitating affiliation and trust between teammates in competitive athletic environments such as professional rugby, which often involves high physiological stress and uncertainty.  As Jeremy Paul, a famous Australian Rugby player commented of his late teammate, Dan Vickerman: ``...you always wanted a guy who would go into the trenches with you and he always played consistently well...he could really play and was just one of the good lads that you enjoyed his company'' \citep{Fox-Sports2017}.  As this excerpt suggests, the experience of team click may act as a social diagnostic tool, a powerful signal of commitment to joint action and willingness to cooperate \citep{Reddish2013a}. \\

Team click refers to the perception of peak team performance.  Based on anecdote and evidence from psychology and anthropology, an individual's perception of team click should contain some or all of the components outlined in Figure ~\ref{tab:teamClickComponents}.

% Please add the following required packages to your document preamble:
% \usepackage{booktabs}
\begin{table}[]
  \centering

\begin{tabular}{@{}ll@{}}
\toprule
\textbf{Component} & \textbf{Description} \\ \midrule
 &  \\
\textbf{Viscerality} &  \\
 &  \\
Tacit understanding & \begin{tabular}[c]{@{}l@{}}Perception of unspoken or implicit alignment\\ of expectations with co-actors\end{tabular} \\
 &  \\
Group flow & Perception of flow or coherence in joint action \\
 &  \\
Atmosphere or aura & \begin{tabular}[c]{@{}l@{}}Perceived positive, energetic quality within \\ or surrounding the team\end{tabular} \\
 &  \\
 &  \\
\textbf{Social Agency} &  \\
 &  \\
\begin{tabular}[c]{@{}l@{}}Blurring of self \\ and other agency\end{tabular} & \begin{tabular}[c]{@{}l@{}}Blurring of distinction between self and other\\  agency in joint action\end{tabular} \\
 &  \\
\begin{tabular}[c]{@{}l@{}}Ability extended \\ by others\end{tabular} & \begin{tabular}[c]{@{}l@{}}Feeling that individual abilities are extended\\ by the agency of others\end{tabular} \\
 &  \\
\begin{tabular}[c]{@{}l@{}}Reliability of self\\  and others\end{tabular} & \begin{tabular}[c]{@{}l@{}}Perceived certainty concerning the ability of \\ self and others to contribute effectively to joint action\end{tabular} \\
 &  \\ \bottomrule
\end{tabular}

\caption{Phenomenological components of team click}
\label{tab:teamClickComponents}
\end{table}



As explained above, team click is a phenomenon distinct to joint action.  Team click involves an experience of pleasurable flow or coherence of joint action, tacit (rather than explicit) understanding between co-participants, a degree of aura or atmosphere around joint action, a blurring of agency between self and teammates or group (or ``we''), and perception of reliability of others and self in performing their roles. As I develop in more detail below, these components of team click can be reduced to three main dimensions: surprise, viscerality, and agency.

Researchers have pointed out that flow---a concept that shares many similarities with team click---has traditionally escaped rigorous scientific analysis \citep{Dietrich2010a,Slingerland2014}.  Dietrich, for example, explains that flow presents a paradox that is difficult to explain according to traditional theories of attention and mental effort, which tend to assume better performance, on any task, is associated with increased conscious effort allocated to that task \citep{Dietrich2004b}.  As \textcite{Slingerland2014} surmises, flow reflects a paradox in which an individual in flow appears to be ``trying not to try.''  Flow appears to entail spectrum of both effortful (intentional, mental) and seemingly effortless (implicit, automatic) cognitions---some of which appear to depend on the resources beyond the brain.  Team click, like flow, is a multidimensional process involving mental, embodied, emotional, and social dimensions.  In this dissertation, I suggest that the most effective way to reconcile the paradox associated with flow, team click, and the mystery of group exercise more generally, is via a dynamical and integrative model of cognition.

Team click offers a powerful empirical anchor for exploring a broader relationship between joint action and social bonding in group exercise contexts.  In the sections that follow, I outline the building blocks necessary to formulate a novel theory of social bonding through joint action.  In particular, I review existing theory and research that suggests a direct link between dynamical coordination of physical movement and social communication, and suggest how this link can produce social connection when movement in joint action aligns between co-actors.  In addition, I demonstrate how the unique parameters of complex joint action coordination and physiological exertion in group exercise contexts create conditions conducive to high quality social bonding when the team clicks.  Within this theory, team click is proposed as a mediating construct, through which the bonding effects of joint action are maximised in group exercise.


















\section{The link between dynamic coordination of physical movement and social communication in joint action \label{sect:physicalMovementSocialCommunication}}
%\subsection{The degrees of freedom problem and its solutions in human movement}

In this section I outline evidence from neurophysiology, cognitive neuroscience, and psychology for link between dynamical coordination of physical movement and social communication.  This link serves as a foundation for a theory of social bonding through joint action in group exercise.  The physical--social link in joint action is immediate, automatic, and multimodal \citep{Semin2008}, but the fact that individuals engaged in joint action may occupy a ``shared manifold'' \citep[see][]{Gallese2003,Friston2015} does not necessarily guarantee that this common cognitive foundation will generate feelings of social connection.  Rather, social connection appears to depend on whether or not individuals perceive the click of joint action.

Humans are hyper-social \citep{Tomasello2012a}, to the extent that our capacities for social interaction---particularly those that are more basal and implicit---like physical movement coordination---are often taken for granted \citep{Wheatley2016}.  It is important to realise, however, that successful coordination of both physical and social processes in joint action represents a cognitively improbable and astounding achievement.

Neurophysiologist Nikolai \textcite{Bernstein1967} was the first to point out the astounding computational challenge associated with coordinated physical movement in multi-component living systems.  In the case of human intra-personal movement, for example, an impressive balance is somehow struct between flexibility, precision, and control, whereby hundreds of muscles and joints coordinate to perform many different tasks of everyday life.  When we grasp a cup or catch a ball, many individual muscles and joints---each with their individual degrees of freedom---work together in fine concert.  Bernstein found it unlikely, from a computational perspective, that the central nervous system would be able to finely control all the possible movements (the degrees of freedom) of each single muscle individually to create coherently directed movements---it would be computationally impossible.  Rather, he suggested that muscles and joints form  ``functional synergies''---flexible function-specific self-organising assemblies---by locally coupling and constraining each other’s degrees of freedom to greatly reduce the amount of control needed.

As I explain in more depth below, intra-personal and interpersonal human movement both rely on the nervous system’s capacity to anticipate, attend, and adapt to the conditions of the environment \citep{Keller2014}.  In the case of intra-personal human movement, successful coordination is made possible by the nervous system's direct access to information concerning the spatiotemporal position of the body and its relationship to the environment.  Privileged access to this information enables habituated coupling of movement control system to the organism's various degrees of freedom, and therefore the emergence of self-organisng functional synergies \citep{Riley2011}.  By contrast, interpersonal movement coordination poses a much more significant challenge.  In the case of interpersonal coordination, the combination of 1) limited reliability of sensory modalities as a source of information about the action of others \citep{Wilson2005,Wolpert2003,Frith2007} and 2) the informational complexity associated with a cognitive system comprising multiple autonomous agents \citep[each with their own movement degrees of freedom; see][]{Turvey1978} means that successful coordination in joint action is inherently difficult.  How exactly do humans face up to this cognitive challenge, and, quite often, successfully surmount it?

An answer to this question requires close attention to the various sophisticated cognitive, behavioural, and cultural capacities that define human's species-unique evolutionary niche \citep{Roepstorff2010,Clark2015,Fuentes2016}.  Our ability for technical manipulation of extra-somatic materials and ecologies; advanced theory of mind; and information-rich, malleable, and scalable communication systems affords complex behavioural repertoires involving interaction between groups as small as two and conceivably as large as two million \citep{Pacherie2012,Nowak2017}.  Indeed, while joint action is daunting and improbable from a purely computational perspective, it is also clear that humans have evolved cognitive mechanisms through which joint action is effectively established and maintained.

What social cognitive processes are required to establish and maintain dynamic coordination of physical movement between two or more individuals? How do physical processes interact with social processes and vice-versa?  Cognitive scientist Natalie Sebanz and colleagues have proposed a minimal architecture for joint action \citep{Sebanz2006,Vesper2010}, which requires that individuals adhere to the following conditions:

\begin{enumerate}
  \item Represent a shared goal, as well as representing one’s own individual contribution to the shared goal.
  \item Apply monitoring and prediction processes to each partner’s actions. This includes monitoring the extent to which shared goals or tasks are being fulfilled while at the same time predicting a partner’s actions.
  \item Facilitate continuous coordination via coordination smoothing, defined as the process of continuously improving one’s prediction of the partner’s action.
\end{enumerate}

These three minimal conditions for joint action hinge on capacities for dynamic coordination of physical movement, as well sharing information pertaining to physical movement with others.  Based on current understandings of cognition and motor control, it is unlikely that adherence to these minimal requirements of representing, monitoring, and sustaining joint action could be driven solely by central, executive control processes in the central nervous system---joint action would be too slow and costly.  It is similarly unlikely that processes of unconstrained automatic alignment (with no higher-level coordination) could possibly enable joint action of the type observed in humans \citep{Fusaroli2014}.  Rather, it seems more plausible that joint action is facilitated by a continuum of mechanisms spanning executive control through to direct coupling with the environment.

For some time cognitive researchers have argued that joint action entails a special mode of social cognition in which processes of information transfer are fundamentally shared between brains, bodies, and physical features \citep{Hutchins1995,Kirsh2006,Susi2001}.  Theoretical conceptualisation and empirical substantiation of this understanding of social cognition in joint action has, however, proven more difficult to achieve, due in part to the tendency of the MES and CR to preference the functional role of symbolic and amodal cognitive processes \citep{Semin2008,Yufik2013}.
But, more recent advances in neuroimaging technologies \citep{Frith2007}, neurocomputational theories of brain function \citep{Friston2010,Frith2010,Yufik2013,Clark2013}, and constructive attempts to extend the theoretical paradigm of human social cognition to account for inter-individual processes of interaction and coordination \citep[e.g.][]{Sebanz2006,Semin2008,Dale2014} have coalesced as a theoretical paradigm with testable predictions.  Through this emerging paradigm, dynamic coordination of physical movement and social communication in joint action can be understood as a functional and integrated processes of social cognition which draws upon informational resources distributed throughout brains, bodies, and bio-external features of a task-specific environment \citep{Clark2015}.

In the sections below, I outline theory and evidence for cognitive mechanisms that perform dual roles as facilitators of dynamical coordination of physical movement and social communication.  The immediate link between coordination of physical movement in joint action and social communication suggests a functional, bi-directional coupling between physiological processes and social communication that sets the foundation for a theoretical relationship between joint action and social bonding in group exercise.

%In the following section I review a minimal architecture for joint action and cognitive theories capable of accounting dynamical coordination of physical movement.
% In this dissertation, I focus on joint action involving dynamic ``on-line'' interpersonal coordination of physical movement.
%To establish and sustain interpersonal coordination, participants selectively recruit multiple behaviours and processes, which in turn become more interdependent and constrained by the function of ongoing activity.
%Importantly, it appears that successful coordination of behaviour in joint action can be rewarded in the form of powerful psychophysiological states of arousal, euphoria, and eudemonic wellbeing.
% Below I outline emerging theory and existing empirical evidence that supports a full-spectrum and multimodal cognitive relationship between physical movement and social communication

%\subsection{Coordination solutions to the challenge of joint action}

\subsection{Evidence for the physical--social link in joint action}
In this section I outline empirical evidence for a link between dynamical coordination of physical movement and social communication in joint action.

\myparagraph{Biological motion and social motion}
Humans display fine-tuned sensitivity to the quality of joint action and readily infer social significance from the outcomes of joint action ~\citep{Wheatley2016}.  As early as 1944, psychologists \textcite{Heider1944} demonstrated that complex social intensions and emotions could be deciphered from the motion trajectories of simple shapes.  Individuals readily extract social information (such as gender, emotion, and personality) from dynamic point-light displays \citep{Atkinson2004,Clark2005a,Johansson1973}.  Neuroscientific evidence suggests that auditory, visual, and biological motion
are processed by shared neural cortices \citep[e.g., the right superior temporal cortex; see][]{Zatorre2007,Beaucousin2007,Beauchamp2007}, and patients with aprosodia (inability to convey or interpret dynamical properties of speech) and amusia (the inability produce or interpret music) display correlated deficits in recognising gestures and facial expressions.  Taken together, these results suggest a shared neural mechanism for multisensory perception of dynamical motion and social communication \citep{Wheatley2012,Wheatley2016}.

\myparagraph{Action-perception links}
Parallel strands of research in psychology \citep{Prinz1990,Prinz1997,Prinz2013}, neurophysiology \citep{Rizzolatti2004,Rizzolatti2010}, and neurocognition \citep{Wolpert1998,Wolpert2000} suggest that interpersonal behavioural coordination in joint action is facilitated by the intrinsic links between action perception and action execution in the human brain.  In essence, action-perception links refers to the ostensive co-occurrence of a stimulus for action and its motor representation.  For example, for individuals who have mastered a certain sensorimotor task, the representation of a perceptual effect (say the sound of a middle-C on a piano) can trigger the movement necessary to produce the effect itself (motor instructions for playing the middle-C key on a piano) \citep{Novembre2014}.

Evidence suggests that skilled individuals develop action-perception links for the actions of others in joint action \citep{Novembre2012}.  Action-perception links can be used for monitoring and integrating (e.g., timing or combined pitches) the actions of others with self-generated actions \citep{Loehr2013}, and these effects appear to be stronger in individuals with high perspective taking skills \citep{Novembre2012,Loehr2013}.  The overlap between mechanisms for action production and action observation suggests that individuals may represent their own and others’ actions in a commensurable format.   Importantly, action-perception links between co-actors suggests that ``neural parity'' \citep{Liberman2000} in joint action can be achieved in an automatic and pre-reflexive manner.  Training-induced motoric representation of self and others' actions may facilitate various capacities important for joint action, such as prediction, adaptation, and entrainment (for a more detailed treatment of action-perception links and their relevance to joint action, see Appendix~\ref{app2:actionPerceptionLinks}).

%* The type of ``mirroring'' noted by Buccino et al. (2001) does not require voluntary control.
\myparagraph{Agency}
Agency refers to the feeling of control (attributed to self, other, and/or group) over actions and their consequences \citep{Moore2016}.  The perception agency in joint action can thus be understood as a social inferential process with roots in sensorimotor (physical) processes.  Perception of agency in action has been formally defined as requiring: 1) \textit{priority} (whether the intention of action precedes the action), 2) \textit{consistency} between the action and the original intention, and 3) \textit{exclusivity} of explanations for the cause of the action \citep{Wegner1999}.  Existing empirical evidence suggests that consistency may be the most important condition for the attribution of agency \citep{VanderWel2012}.  It has been shown that agency in action is strongly correlated with an individual's ability to anticipate individual contribution to action.  For example, \textcite{Sato2008} and colleagues demonstrated that discrepancy between prediction and sensory input can alter the experience of agency: unpredicted sensory input can lead to ascribing agency for that input to an external source, for example, other participants in joint action or the external environment \citep{Sato2005,Frith2007}.  It is generally understood that this result can be sensory attenuation that reliably occurs when individuals anticipate action:  researchers have shown a strong correlation between attenuation of proprioception (predictions relating to individual contribution to action) and experiences of self agency \citep{Wolpert2003,Sato2008}, as well as an inverse correlation between sensory attenuation and ascribing agency to sources external to the self \citep{Brown2013}.  As has been well documented in the case of schizophrenia, attribution of agency in social interaction may be modulated by individual variation in ``locus of control'' (the degree to which events are perceived to result from one’s own actions or not), and this may be related to improper function of the parietal cortex \citep{Frith2000}.  Taken together, this evidence suggests that the attributions of agency to self and other play an important function in successfully coordinating physical movement.

Empirical evidence addressing the question of how people experience agency in actions they intentionally produce in coordination with others (joint agency) is less abundant \citep[but see][]{VanderWel2012,VanderWel2013}. van der Wel and colleagues show that perceptions of self agency do not decrease when an individual transfers from performing a solo action to performing the same action with a partner (thus challenging the condition of exclusivity).  Individuals do however experience a boost in agency when they transfer from performing a joint action to the same action solo, suggesting that this transfer may induce enhanced consistency owing to the greater reliability of predictions pertaining to action that is individually controlled \citep{VanderWel2012}.  Recent research suggests that joint agency may be underwritten by  movement signatures consistent with being ``in the zone'' with a co-actor.  Working within the common dyadic ``mirror game'' paradigm, \textcite{Noy2011,Noy2015,Hart2014} have identified ``co-confident motion'' (CC motion) as a canonical movement pattern of synchronised motion characterised by smooth and jitter-less motion, without the typical jitter resulting from reactive control in more commonly encountered leader-follower patterns.  In CC motion, different players appear to shift their basic motion signatures to a movement shape that is altogether different from their individually preferred shapes \citep{Hart2014}.  Importantly, the pattern of CC motion shares the same sine wave shape as the optimal solution of the minimum jerk model, a well-known motor control model for rhythmic motion \citep{Hogan2007}.  This evidence accords with the proposal that perceptions of shared agency in joint action derives from the transcendence of individual action tendencies of self and other, to produce a ``we-mode'' of social cognition \citep{Gallotti2013}.

\myparagraph{Type-based representations of self, other, and collective}
Beyond immediate sensorimotor linkages and perceptions of agency,  individuals generate higher-order social inferences about the self, others, and groups in joint action known as ``type-based social representations'' \citep{Moutoussis2014}.  It is well established in psychology that individuals form beliefs about states and traits of self and others \citep{Bem1967,Fowler2006}.  Research suggests that people automatically prescribe traits to self, other, and the target social group based on 1) perceived to social outcomes of joint action (e.g., ``success'' or ``failure''), 2) perceived capabilities in joint action (e.g.,``weak'' or ``talented'') and 3) perceived preferences for acting \citep[e.g., ``good,'' ``fair,'' or ``trustworthy'';][]{Moutoussis2014}.  Clinical interpretations of this line of research has suggested a functional role for self-representation (i.e., self-esteem) as an indicator of one's likely social evaluation by the social milieu \citep{Leary1995}. Thus, it appears that type-based representations of self, other, and group in joint action (and the attribution of social values to these representations) may function as useful heuristics that serve to drastically reduce the computational burden associated with movement coordination in joint action \citep{Moutoussis2011}.

\myparagraph{Coordination smoothers}
At an even higher level of perceptual abstraction, real world joint action scenarios appear to rely heavily on assumptions or norms shared between co-actors.  In the context of joint action, cultural conventions can be understood as shared frames of reference that set the macro-contextual coordinates for joint action \citep{Clark2013}. Joint actions that involve complex sequences and divisions of labour between participants appear to rely heavily on capacities to explicitly signal intention for the assigning of roles, forward planning, and repair of failed coordination \citep{Frith2010}.  These ``coordination smoothers'' \citep{Vesper2017} often function to reduce spatial and temporal variation in action by providing a shared spatiotemporal referent for co-alignment of predictions.  Depending on the context of the joint action, it could be subject to a pre-existing, mutually recognised power relations typical in the established culture (e.g., favouring hierarchical or egalitarian communication, \citep[see]{Cheon2011}) and the particular situational context (e.g., formal or informal).  Establishing roles, such as leader or follower, also has a similar smoothing effect, and often the affordances in the task environment shape the smoothing strategies available to co-actors \citep{Marsh2009}.

\myparagraph{Summary of evidence of a physical--social link in joint action}
Taken together, this evidence suggests that social meaning automatically and immediately resonates from dynamic coordination of physical movement in joint action, owing to shared neurocognitive mechanisms for processing dynamical motion and social inferences.  On a basal level, perceptions of dynamical biological motion can induce social inference, and sensorimotor predictions can induce perceptions of self, other, and collective agency.  On a higher level of perceptual abstraction, evaluations of performance in joint action can lead to the formation of type-based representations pertaining to self, other, and group, and these models can function as frugal heuristics for future movement coordination.  On the most abstract and conventional level of perception and action, shared conventions and frames of reference are readily enlisted as ``coordination smoothers'' for reducing spatial and temporal variation in action.

\myparagraph{Prevailing theoretical explanations for the physical--social link in joint action}

Prevailing integrative models of cognition propose that the observable link between dynamical coordination of physical movement and social communication cannot be sufficiently explained purely through reference to ultimate selectionist priorities of inclusive inclusive fitness \cite{Badcock2012}.  In addition, social cognition must be understood through reference to an overarching mandate deriving from non-equilibrium thermodynamics \citep{Schrodinger1944}.  In brief, the most effective way for a (non-dissipative) biological system to maintain internal equilibrium (order, life) against thermodynamic equilibrium (chaos, death) is to develop models capable of reflecting the sensory regularities to which the system will be subject over the course of its lifespan \citep{Conant1970,Yufik2016}. On a level of mathematical formalism, this process equates to a need for biological organisms to minimise information-theoretic ``free energy'' in their interactions with the environment \citep{Friston2006,Ramstead2017,Yufik2017}.  In other words, the best models are those that can anticipate (and therefore minimise) ``surprise'' events that may push the organism over a critical (and irreversible) thermodynamic threshold towards chaos or death \citep[here surprise can be understood as a rough equivalent of free energy, see][]{Friston2013}.
  \footnote{The drive of the organism to minimise free energy is known as the ``Free Energy Principle'' \citep[hereafter FEP; see][]{Friston2010}, and biological life's adherence to the FEP can be identified at both ontogenetic and phylogenetic timescales: an organism's genome represents the accumulation of informational variants most calibrated to the regularities of the evolutionary niche in which it has evolved (and therefore most likely to reduce free energy); an organism's phenotype represents an organism's best attempt to reduce free energy over the course of its lifespan \citep{Ramstead2017}.  The universality of the FEP across proximate and ultimate levels of biology offers an opportunity to better understand the dynamical and self-organising principles (in additional to principles of Darwinian and general selection) that cross-cut and connect these levels of analysis \citep{Caporael2001,Badcock2012,Laland2015,Ramstead2017}.}

This ``thermodynamic'' conception of cognition \citep[see][]{Yufik2013}, also referred to as ``active inference'' \citep[see][]{Friston2010} allows for the integration of traditionally disparate processes (e.g. physical and social) in one integrative inferential framework.  Thermodynamic cognition enlists the neurocomputational paradigm of predictive coding \citep[hereafter PC, see][]{Rao1999} in order to operationalise the FEP in the brain as a process of prediction error minimisation \citep{Clark2013}.  PC posits a dynamical and hierarchical process of ``generative modelling'' in which predictive ``forward models'' for action and perception, rather than being auxiliary to, are \textit{integrated} into action and perception \citep[as the instructions for action and perception themselves; for a more detailed review of Auxiliary Forward Models and Integrative Forward Models, see Appendix~\ref{app2:motorControl};]{Pickering2014}. In the PC framework, forward models are updated only when violated by prediction error signals arriving from lower levels of the informational hierarchy.

%Importantly, this ``modelling view'' of computation \citep{Grush 2001;Chirimuuta2014} refers to a specific kind of minimal, structural or analogical model based on statistical correlations rather than necessarily involving semantic or propositional content \citep[also known as ``generative models''][]{Friston2001,OBrien2004,Huto2015}.  In the modelling view of cognition, a model is said to ``represent'' a target domain only in the sense that the relations between its computational processes preserve the higher-order statistical, structural-relational properties of the target domain, which can be leveraged to guide adaptive action \citep[see][8]{Ramstead2016}.  In this sense, generative models do not create auxiliary ``representations'' of the world, rather, they are a direct correlation of the statistical regularities of the world to which the organism has been exposed over the course of its lifespan \citep{Ramstead2017}.


In a radical inverse of traditional models of cognition and motor control, in the PC framework, hypotheses concerning likely causes of sensory stimuli are adjudicated through an inferential process likened to ``empirical Bayes'' \citep{Clark2013}.  Hierarchical predictions are informed by three calculations: 1) prior experience, 2) available sensory evidence, and 3) a second-order calculation of the relative precision or reliability of the available sensory evidence.  Second-order precision weighting of sensory evidence enables flexible direction of attention to prediction errors occurring at multiple levels of generative model (e.g. lower-order sensorimotor levels or higher-order perceptual or conceptual levels), and deriving from exteroceptive (relating to stimuli that are external to an organism, i.e. visual, auditory, haptic perception), interoceptive \citep[relating to stimuli produced within an organism, particularly by the body's organs (viscera) e.g., ``gut feelings,'' or elevated heart rate; see][]{Seth2013,FeldmanBarrett2015}, or ``proprioceptive'' information \citep[relating to stimuli that are produced and perceived within an organism, especially those connected with the position and movement of the body][]{Friston2011a}.  Iterative learning adjusts the precision weighting mechanism to strengthen prior probability distributions of models for future inference \citep{Robbins1964}.
    \footnote{In effect, second order Bayesian inference allows the brain to decide which prediction errors to pay attention to (e.g., exteroceptive or proprioceptive) at which level of the coding hierarchy \citep[for example, high and conceptual or deep and sensory][]{Friston2015}.  Each error signal is conditioned with a precision weighting, which functions to either ``dial-up'' or ``dial-down'' the volume on that signal's influence on the overall predictive model \citep{Clark2015}.  Precision weightings are proportional to the inverse variance of the model prior for each sensory input \citep{Ernst2004,FitzGerald2014}.  In other words, the volume of more reliable or more commonly encountered sensory inputs will be dialled-up; while the volume on less reliable or less commonly encountered sensory inputs will be dialled-down. This process of Bayesian precision weighting can be understood as the thermodynamic cognition's version of attention \citep{Ramstead2016}.}

Finally, the concept of affordances \citep{Gibson1979} has been enlisted by researches to explain how predominantly interoceptive generative models combine with cognitive resources located beyond the brain.  In brief, affordances can be understood as extra-neural resources that couple with generative models to produce loops of action and perception \citep{Ramstead2016,Clark2015}.  While affordances have traditionally been studied narrowly as localised resources for basic perception \citep[e.g.][]{Fajen2011}, researchers working with the thermodynamic framework have proposed an extension of the the concept of affordances to include a spectrum of objects, ranging from basal ``natural affordances'' (physical features of the environment) through to ``conventional affordances'' mediated by highly contingent human cultural conventions and institutions  \citep[see][]{Roepstorff2010,Ramstead2016}.  Repeated coupling between generative models and their extra-neural correlations in the environment give rise to dense causal relations between particular affordances and particular predictions.
      \footnote{
      As \textcite[906]{Pezzulo2014} points out, PC hierarchies extend well beyond hypotheses concerning the source and reliability of immediate sensory inputs. At higher levels of the PC hierarchy, more profound regularities can be represented, such as long-term beliefs that are increasingly more removed from sensorimotor events. Indeed, the human capacity for social learning is such that higher order beliefs may be acquired mainly through cultural learning (rather than purely ``from the ground up'' via natural affordances). But, these conventional beliefs may still remain ``grounded'' through the linkage with lower-level sensory events).
      }


\subsubsection{Predictions for social connection in joint action}
When specifically applied to joint action, thermodynamic cognition posits two (or more) Bayesian predictive brains committed to preemptively modelling the world (including each other) and coupling with appropriate affordances in the world in order to minimise free energy \citep{Moutoussis2014,Friston2015,Friston2015a}.  Thermodynamic cognition therefore predicts that joint action will involve strategies that enable functional equivalence of predictive models between co-actors, and in turn, the minimisation of free energy.
        \footnote{
        To achieve free energy minimisation, the sensory stimuli produced by co-actors in joint action must be pre-emptively modelled, just like other feature of the sensorium (as explained above, see Section~\ref{sect:thermoCog}).  This necessitates a scenario in which brain A has a model of brain B, which includes the fact brain B is modelling brain A, and so on---\textit{ad infinitum}.  The recurrent predictions of both brains about one another threatens an infinite runaway regress that could preclude accurate modelling of either brain due to the computational intractability of the task \citep{Moutoussis2014,Friston2015}.  As Friston and Frith demonstrate formally (mathematically), this recursion (and its computational complexity) dissolves if the models of the two brains are formally similar \citep{Friston2015,Friston2015a}.
        When grounded in computational similarity, each brain is able to generate predictions of the sensory outcomes caused by itself and the other in the same way, owing to the fact that the state that brain A is in imposes constraints on the states brain B can occupy \citep{Richardson2015}.
        }

\myparagraph{Dynamic coupling}
Friston and Frith predict that two or more systems that are trying to predict each other  will form a dynamic coupling, known as ``general synchronisation'' \citep{Friston2015}.
    \footnote{Generalised synchrony refers to the synchronisation of chaotic dynamics \citep{Barreto2003}.  The basic principle is as follows.  If the universe comprised two biological (free energy minimising) systems---you and me---then your states and my states have to be restricted to an attracting set of states that is small relative to all possible states we could be in \citep{Friston2015}.  This attracting set of states enforces a generalised synchrony in the sense that the state you are in imposes constraints on states I occupy \citep{Richardson2015}.  It is in this sense that generalised synchronisation is a fundamental aspect of coupled dynamical systems that are free energy minimising.  If we are both trying to minimise the free energy of our attracting set (by reducing surprise or entropy), then synchronisation will be more manifest \citep{Friston2015a}. In the case of the cognition of joint action, generalised synchronisation can be understood as the level of interdependence between levels of hierarchical models generated by two more (Bayesian) brains.  Friston and Frith call this the ``shared narrative'' that enables joint action.  Within general synchronisation, ``identical synchronisation'' refers a moment in which two or more models show identical equivalence between each hierarchical layer of prediction \citep{Friston2015}.   In contrast to behavioural synchrony, general synchronisation refers to alignment on a cognitive (information-theoretic) level, as opposed to merely a behavioural level.}
Dynamic coupling can be identified by two core properties:  1) dimensional compression (potentially independent DF are coupled so that the synergy possesses a lower dimensionality than the set of components from which it arises) and 2) reciprocal compensation \citep[the ability of one component of a synergy to react to changes in others][]{Turvey1978,Schmidt1990,Riley2011}.

An accumulation of evidence on multiple levels of human behaviour, from brain function \citep{Yufik1998,Sengupta2013}, to interpersonal interactions \citep{Kelso2009,Riley2011,Fusaroli2014}, to large-scale human societal dynamics \citep{Nowak2017} supports the existence of dynamic coupling in human social activity.  In the case of dynamic multi-agent joint action in particular, individuals have been found to couple and reciprocally constrain their movements reducing the overall control needed to maintain effective cooperation \citep{Ramenzoni2011,Ramenzoni2012,Riley2011,Schmidt1990}.  Studies of real-world joint action scenarios such as dancing, martial arts, and moving objects like furniture have revealed evidence for dynamic coupling between co-actors.  In these studies, specific component degrees of freedom are modelled as coupled oscillators \citep[using the HKB model, which describes the change in the relative phase between two oscillatory components. See][]{Haken1985,Kelso1986}.  Dynamic coupling of movement has been measured beyond dyadic synchronisation, in the analysis of sub-phases of team sports  \citep{Passos2014,Duarte2012} and group dancing \citep{Chauvigne2017}.  For a more thorough explanation of dynamic coupling in human movement systems, including methods for measuring dynamical quantities, see Appendix~\ref{app2:dynamicCoupling}.






























\section{Joint action in group exercise \label{sect:JAinGE}}

Joint action in group exercise creates a unique environment for social cognition, defined by extreme levels of cognitive uncertainty.  Group exercise involves high cognitive load associated with complex joint action requirements involving many actors, many hierarchical layers of joint goals over various sensory modalities and spatiotemporal scales. The ``in the moment'' execution demands also constrain cognitive processing, and encourage a reliance on cognitive resources located in extra-neural and bio-external domains \citep{Bourbousson2016}.  In addition, strenuous physical exercise could also entail neurocognitive tradeoffs that further strain individual ability to reduce free energy in joint action \citep{Dietrich2004b}.

Evidence discussed above suggests that optimal solutions to joint action typical of group exercise may tend to favour the recruitment of more extra-neural resources as a way of minimising free energy, whereas less efficient solutions to joint action may rely on more computationally intensive procedures in order to reduce free energy (see Section ~\ref{sect:extraNeural}).  The social cognition of these processes in joint action have not yet been closely considered \citep[but see ][]{Marsh2009,Lumsden2012}.




    \subsection{Group exercise spikes uncertainty of joint action}


    \myparagraph{Group exercise involves multi-agent (and not just dyadic) joint action}
    Above and beyond normal day-to-day instances of communication and exchange, joint action in group exercise contexts such as sport place extreme cognitive load on participants. The active inference approach to joint action outlined above is based on preliminary models of dyadic joint action involving turn taking \citep[i.e., in bird song exchanges][]{Friston2015}.  Group exercise contexts, particularly modern sport contexts, often involve large numbers of co-participants, in either ``inter-active'' or ``co-active'' modes of coordination.
        \footnote{
        In co-active sports (e.g., bowling, archery), team members perform separately and the team outcome is a product of combined individual performances. In interactive sports (e.g., volleyball, soccer, rugby), goal accomplishment requires the establishment of complex patterns of interaction and coordination among team members \citep{Filho2014}.
        }
    Thus, achieving cognitive synchronisation in joint action of group exercise contexts may be much more difficult.  Evolutionary Anthropologist Robin Dunbar \textcite{Dunbar1992} proposes that the ratio of human neocortex size to total brain volume imposes an upper cognitive limit on real-time coordination of behaviour of approximately four to five individuals.  The sheer computational burden of modelling multiple agents in group exercise may place an unmanageable cognitive load on our normal healthy processes of active inference.  Indeed, at the very least, multi-agent joint action poses a challenge for the existing theoretical model for joint action (PJAM), which is formulated primarily based on dyadic interactions \citep{Pesquita2017}.

    \myparagraph{Joint action in group exercise is ``on-line'' and ``in-the-moment''}
    Particularly in the case of interactive team sports, interpersonal movement coordination is often executed ``on-line'' and ``in the moment,'' as opposed to step-by-step turn taking.  This fact poses a challenge to Friston and Frith's proposal that active inference in joint action comprises two modes (either actively attending to sensory stimuli, or else moving while in a state of sensory attenuation, see Section ~\ref{sect:activeInfJA} above).  Exactly what occurs when actors need to concurrently move and sense others moving at the same is poorly understood.  What happens to the precision weightings---the volume gauges---on proprioceptive, interoceptive, and exteroceptive prediction errors In instances of dynamic interactive joint action involving co-occurence of movement between agents in joint action?   What is the impact of on-line and in the moment join action on the experiences of agency in group exercise?  Empirical research is yet to provide answers to these questions.

    \myparagraph{Joint action in group exercise often involves competition}
    As if the cognitive load of multiple agents and on-line coordination of complex schemas for joint action was not enough for the humble human brain, interactional team sports also usually involves \textit{competition}.  While competition in sport is usually adorned with elaborate social meanings surrounding the ethics of winning and losing \citep{McNamee2008}, on a cognitive level, competition in joint action entails one individual or team of individuals actively attempting to foil or disrupt the predictive models of another individual or team of individuals \citep{Reimer2006}.  The competitive dimension of interactional team sports thus serves to further spike cognitive uncertainty between co-actors in joint action.  The uncertainty involved in competitive joint action scenarios could have further implications for the ability of second-order Bayesian inference concerning the reliability of sensory inputs \citep{Pezzulo2014}.

    \myparagraph{Group exercise involves metabolic tradeoffs in the brain}
    In addition to these heightened cognitive challenges associated with complex and dynamic joint action in group exercise, high levels of physiological exertion characteristic of group exercise could also serve to spike uncertainty.  Neuroscientist Arne Dietrich draws attention to the fact that physical exercise is at its core a stressor that places extreme energy demands the organism \citep{Dietrich2011}.
    Such a situation will necessitate an energy tradeoff in the brain, whereby energetically costly brain regions inessential to movement execution are temporarily downregulated \citep{Dietrich2004b}.  Dietrich suggests that experiences of flow and the ``runner's high'' in exercise could be the result of temporary downregulation of energetically costly brain regions inessential to movement execution  \citep{Dietrich2004b}.  Dietrich and colleagues propose candidate areas of the dorsolateral prefrontal cortex responsible for self-monitoring and proprioceptive sensory attenuation \citep[commonly known as the ``inner critic'' regions of the brain, see][]{Limb2008}.  It is currently not well understood precisely whether or how neurometabolic tradeoffs in the brain could impact upon joint action and the experience of team click.  However, it is conceivable that metabolic tradeoffs in the brain owing to prolonged physiological stress may have implications for the second-order inferential processes of precision weighting sensory inputs.


    \subsection{Free energy minimisation in group exercise demands greater reliance on extra-neural affordances \label{sect:extraNeural}}
    To summarise, the combination of cognitive demands associated with tracking and modelling multiple agents ``in-the-moment,'' the neurometabolic tradeoffs associated with (often extreme) levels of physiological exertion, and even the competitive dimension of some group exercise contexts (for example interactive team sports), could create an environment in which humans' usual cognitive capacities are strained and compromised.  The fact that experiences of team click are particularly prevalent in group exercise contexts (compared to more mundane or quotidian instances of joint action) suggests that amplification of uncertainty and stress in group exercise could be a critical factor in facilitating powerful psychosocial effects.

    The active inference approach would predict that individuals, when faced with the extreme cognitive uncertainty of joint action in group exercise, will tend to preference mechanisms that maximise uncertainty reduction (minimise uncertainty).  In the case of joint action in group exercise, this may entail reliance on predictive models that outsource the computational cost to affordances beyond the brain, or at least the metabolically expensive cortical areas of the brain \citep{Dietrich2004,Clark2015}.  In the case of highly skilled practitioners, whose predictive models for action have been finely tuned to the affordances of the task environment (including co-actors), it is plausible that extra-neural and even extra-personal resources (e.g., physical features of the task environment) could provide a more cognitively efficient and effective route to the performance of successful joint action and thus the minimisation of free energy.

    Various strands of evidence support these predictions.  Studies of highly skilled practitioners in joint action demonstrate that more technically competent practitioners generate more accurate predictive models for joint action than less technically competent practitioners \citep{Tomeo2012,Aglioti2008,Mulligan2016}.   In studies involving skilled versus non-skilled practitioners in dyadic interactions, it has been shown that more skilled practitioners create stronger dynamical coupling through flexibly modulating their actions with others \citep{Schmidt2011,Caron2017}. These findings are corroborated by other studies that find that professional footballers (versus novice controls) are able to more accurately predict the direction of a kick from another player's body kinematics (\cite{Tomeo2012}, see also \cite{Aglioti2008,Mulligan2016} for similar results with basketball and dart players).  When analysing co-regulation between members of basketball teams, \textcite{Bourbousson2015} showed that more expert teams made fewer mutual adjustments (at the level of the activity that was meaningful for co-actors), suggesting an enhanced capability of expert social systems to achieve and maintain an optimal level of awareness during the unfolding activity.

    A recent field study with expert rowers revealed that athletes predominantly utilised
    extra-personal (rather than inter-personal) regulation processes in order to facilitate and sustain joint action, and attention to interpersonal regulation occurred only during instances of expectation violation concerning performance in joint action \citep[; for a full explanation of this study, see Appendix ~\ref{app2:theory} Section ~\ref{sect:rowerStudy}]{RKiouak2016}. These results suggest that athletes used the affordances of the environment to mediate the arrangement of individual and joint activities \citep{Bourbousson2011,Bourbousson2012}.  Taken together, this evidence supports the prediction for a tendency for co-actors in joint action to utilise neurocomputationally conservative models and coupling with extra-neural affordances under circumstances of high levels of free energy (such as those common to group exercise).













\section{Team click mediates a relationship between joint action and social bonding}


    \subsection{Perceptions of performance in joint action predict team click}

      \myparagraph{Surprise}
      \myparagraph{Viscerality}
      \myparagraph{Agency}

    \subsection{Team click predicts social bonding}

    \myparagraph{Emotional Support}
    \myparagraph{Common goal}
    \myparagraph{Social Identity}










\section{Predictions of the theory}


    The overarching prediction of this thesis is that the psychological phenomenon of team click mediates a relationship between joint action and social bonding.

    Within this main hypothesis, I also formulate the following sub-hypotheses:
    \begin{enumerate}
      \item Athletes who perceive greater success in joint action will experience higher levels of felt ``team click.'' I predict that relevant perceptions of joint action success will relate to athlete perceptions of:
        \begin{enumerate}
          \item a combination of specific technical components; or
          \item an overall perception of team performance relative to prior expectations; or
          \item an interaction between these two dimensions of team performance.
        \end{enumerate}
      \item Athletes who experience higher levels of team click will report higher levels of social bonding.
      \item More positive perceptions of joint action success will predict higher levels of social bonding, driven by more positive:
      \begin{enumerate}
        \item perceptions of components of team performance; or
        \item violation of team performance expectations; or
        \item an interaction between these two predictors.
      \end{enumerate}
    \end{enumerate}

In addition to these core predictions, I also make the following predictions for the role variation to the theory based on cultural and individual variation:

\begin{enumerate}
  \item Individual variation in predispositions towards different movement coordination strategies will influence the relationship betweens joint action, team click, and social bonding.  In particular:
      \begin{enumerate}
        \item Athletes with more prosocial disposition (measured by personality type, e.g. extroversion) will experience higher levels of team click and social bonding in joint action.
        \item Athletes with higher levels of technical competence or experience will experience lower levels of team click and social bonding to the team, do to a lack of surprise associated with the experience.
      \end{enumerate}

  \item Informational affordances that are more dominant in an ecology will have a higher impact on shaping patterns of behaviour in joint action.

\end{enumerate}














                                              \end{CJK}{UTF8}{gbsn}
