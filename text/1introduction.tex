\chapter{\label{introduction}Thesis Outline}

The human capacity to cooperate within cohesive social groups is a fundamental explanation for our species' evolutionary success.  It is therefore surprising that sport---an ubiquitous organiser of modern social life across cultures---has not been more intensively studied for evidence concerning the cognitive and evolutionary foundations of human sociality \citep{Blanchard1995,Downey2005a}.  An integrated scientific study of the social, cognitive, and physiological mechanisms associated with participation in collective group exercise (broadly construed), and the ecological dynamics by which these mechanisms are constrained, stands to offer novel insights into the science of human cognition, cooperation, and evolution.

Whirling Sufi dervishes, late-night electronic music raves, Maasi ceremonial dances, competitive team sports, or the modern cult of Cross-fit---endless examples can be plucked from across cultures and throughout time to exemplify the human compulsion to come together and move together.  It is easy to imagine how exertive and coordinated group activity would have served important survival functions in the past, such as hunting, travel, communication, and defence \citep{Sands2010}.  More puzzling, but also more relevant to the study of human behaviour and evolution, is the persistent recurrence of group exercise in social life, cross-cutting shared cultural practices as varied as religion, sport, music and play, even when its substantive rewards are less obvious or immediate.  Social scientists have long speculated about the benefits of energetic group activities for social cohesion \citep{Durkheim1965}. It is not yet clear, however, how or whether group exercise uniquely generates social cohesion, or in what ways particular mechanisms vary by activity and culture.

In this dissertation I attend specifically to the relationship between joint action and social bonding in the group exercise context of professional rugby in China.  Experimental evidence from the behavioural synchrony and mimicry literatures suggests that high quality coordination of movement between co-actors in joint action may be a powerful source of positive affect, blurring of self-other agency, pro-sociality, and cooperation \citep{Mogan2017}.  Beyond these literatures, the relationship between less tightly coupled joint action and social bonding is yet to be thoroughly tested. Anecdotal and observational evidence from anthropology and psychology---particularly the psychology of ``flow'' \citep{Csikszentmihalyi1992,Jackson1999}---suggests that perceptions of joint action success may set the psychological foundation for processes of affiliation and cohesion.
Various neurological, cognitive, and sociological strands of evidence support this proposal.  Perceptions of successful synchronisation of behaviour in joint action appears to have positive implications for individual psychophysiological function, health, and subjective well being \citep{Wheatley2012}.  Likewise, there is well-documented evidence of a link between psycho-social isolation and ill-health and developmental and neurocognitive deficits in behaviours key to dynamic interpersonal interaction \citep[e.g.][]{Blakemore2005,Baron-Cohen1991}. An account of the full significance of the role of proximate mechanisms of movement regulation and coordination in social bonding is yet to be fully articulated. This dissertation attempts to address this gap through theoretical synthesis and empirical research.

% The extent to which synchronised joint action is responsible for generating social bonding may depend crucially on the accordance of action with culturally directed expectations.


\section{Theoretical Grounding}
A combination of recent advances in neuroimaging technologies \citep{Frith2007}, emerging neurocomputational theories of brain function \citep{Friston2010,Frith2010,Clark2013}, and constructive attempts to extend the theoretical paradigm of human social cognition to account for inter-individual processes of interaction and coordination \citep{Sebanz2006,Dale2014}, has created an opportunity to examine in finer-grained detail the relationship between coordinated and exertive group activities and social cohesion.  It is now more clearly understood that basal human capacities for physical movement regulation and coordination set the foundation for social cognitive systems whose resources are distributed between brains, bodies, and physical features of task-specific environments \citep{Hutchins2000,Kirsh2006,Semin2008,Semin2012,Coey2012}.
Human cognition appears to be driven by a processes of ``active inference'' \citep{Friston2010} about the world.  Agents generate top-down interoceptive predictions about the state of the world and test these representations against bottom-up sensory evidence \citep{Clark2013}.  In this account, perception, representation, emotion, and action are unified by the logic of prediction-error management, and neurocognitive components interact to align the organism with its expectations \citep{Pezzulo2014}.  Conceiving of social cognition in this way, as an embodied, embedded, and immediate process of inference, centralises the role of automatic movement regulation strategies---traditionally classed as ``lower-cognitive'' processes---in establishing and maintaining the transfer of cultural information between individuals, within groups, and throughout populations---traditionally thought to be executed by  ``higher-cognitive'' processes \citep{Claidiere2014}.

A review of the available literature suggests that successful joint action in humans is  contingent on the ability to share functionally equivalent task representations. Considering the cognitive principles of ``active inference'' referenced above, shared task representation amounts to minimising prediction error in social cognitive systems involving two or more co-actors \citep{Semin2008,Frith2010}.  Humans appear to employ an array of explicit and implicit behavioural strategies in order to achieve this.   The ways in which co-actors ``close the loop'' \citep{Frith2007} on joint action through deliberate ostensive communication has been the traditional focus of developmental, comparative \cite{Tomasello2005a}, and social psychologists \citep{Sebanz2006}.
More recently, however, analysis of dynamic coupling of co-actors in joint action scenarios reveals that synchronised movement implicates an array of implicit and pre-perceptual cognitive processes of alignment and prediction error minimisation \citep{Schmidt2011}, which, in addition to more explicit forms of communication, could be central to the generation of feelings of self-other merging, self-other distinction, and perceived reliability and trust associated with social bonding \citep{Marsh2009}. By interrogating the ways in which component mechanisms and system dynamics of joint action generate social bonding, this dissertation seeks to offer a novel contribution to the cognitive and evolutionary anthropology of social cohesion.

\section{Team Click}
Recent research in cognitive and evolutionary anthropology has made a ceremony of invoking one particular passage from Durkheim (1965, pg. 217) to capture the ``collective effervescence'' of exertive and coordinated group activity found in arenas as diverse as music making, dance, military drills, and sport:  ``Once the individuals are gathered together, a sort of electricity is generated from their closeness and quickly launches them to an extraordinary height of exaltation'' \citep{McNeill1995,Konvalinka2011,Fischer2014,Mogan2017}. Indeed, this passage powerfully captures the role of collective activity in generating positive emotional states and joint arousal, and lends itself nicely to the hypothesis that the visceral ``electricity'' of group activity is attributable in part to neuropharmacologically mediated affective mechanisms associated with pain and reward \citep{Dunbar2008,Cohen2009,Fischer2014,Launay2016}.
What is not so effectively evoked in this passage, however, is an aspect of group activity heavily scrutinised in the domain of competitive interactional team sportsabd  other technically demanding joint action scenarios such as music making and dance, namely, the \textit{quality} of movement synchronisation in joint action.  For highly skilled practitioners, often the ecstasy of joint action is contingent on the extent to which the actions of one individual \textit{click} into place with the actions of another---when expectations for joint action are completely aligned.  Fine-tuned sensitivity to the intricacies of joint action, and the psycho-social effects of this sensitivity, suggest that a full explanation of the link between group exercise and social bonding must involve more than just a generalised, neurobiologically mediated, affective mechanism.  Indeed, a full array of cognitive mechanisms involved in movement regulation and coordination must be considered.

One of the big mysteries of competitive team sport, particularly at the elite level, is the elusiveness of peak team performance.  While each individual athlete may exhibit expert-level competence in sport-specific skills, the much sought-after aggregation of these components, i.e. a team that consistently performs ``in the zone,'' and ``firing on all cylinders,'' in reality often proves frustratingly difficult to achieve and sustain.  \footnote{As King and De Rond \textcite[568]{King2011} note in their ethnography of the 2008 Cambridge University rowing crew who participated (and who were eventually victorious) in the famous annual Boat Race against Oxford University, while the search for collective rhythm is a universal in human social interaction, the physiological and psychological complexity of finding that rhythm ``...is extremely difficult to attain; collective performance is a possibility not a certainty.''}   The moment in which everything ``clicks'' into place in team sport can, for various reasons, disappear as abruptly as it arrives, if indeed it arrives at all.
But when ``team click'' is cultivated, and sustained, it is celebrated as the ultimate, albeit often inexplicable magic of sporting feats.  For athletes, coaches, and spectators alike,``team click'' can be a hugely powerful sensation. As theologian Michael Novak explains, ``[f]or those who have participated on a team that has known the click of communality, the experience is unforgettable, like that of having attained, for a while at least, a higher level of existence'' \citep[11]{White2011}.

As has been extensively documented in the psychological literature of ``flow'' \citep{Csikszentmihalyi1992} and optimal human performance in sport \citep{Jackson1999}, athletes engaged in team coordination often report total absorption in and complete focus on the task at hand, a transformation of the experience of time (either speeding up or slowing down), and a blurring or transcendence of individual agency, or a ``loss of self''   \citep{Csikszentmihalyi1992,Jackson1995,Jackson1999,McNeill1995}.  Research suggests that flow often occurs in scenarios in which there are clear goals inherent in the activity, as well as unambiguous feedback concerning extent to which goals are either being achieved or not.
In addition, scenarios most conducive to the experience of flow are those in which the technical requirements are challenging but achievable if practitioners are able to extend slightly beyond their normal perceived capabilities \citep{Fong2015}.
The coalescence of these factors is intrinsically rewarding and autotelic \citep{Csikszentmihalyi1975}, activating both ``hedonic'' and ``eudaimonic'' dimensions of subjective well-being \citep{Huta2010,Fave2009}.  The vast majority of flow research has focussed on the experience of the individual athlete, musician, or performer.  However, recent attempts have been made to extended an analysis of flow and its antecedents to the level of the group and dynamics of interpersonal coordination---a phenomenon termed ``group flow'' \citep{Sawyer2006}.

The experience of flow has by now been extensively studied by psychologists and neuroscientists, from which a series of neuropharmacological \citep{Boecker2008}, neurocognitive \citep{Dietrich2006,Dietrich2011,Labelle2013}, and psychological \citep{Csikszentmihalyi1992} theories for its emergence have been tabled. However, throughout this process, the social dimensions of optimal human experience have been less scrutinised, despite strong anecdotal and observational evidence of phenomena such as ``team click,'' and ``group flow.''   Despite the fact that the phenomenology of flow is experienced and described at the individual level, considering the prevailing theoretical paradigm outlined above, it should be expected that the cognitive mechanisms responsible for the phenomenology are embedded in and distributed throughout brains, bodies, and other other resources of the task-specific environment \citep{Kirsh2006,Marsh2009,Noy2015}.

Importantly, team click appears to have psychological consequences relevant to social bonding and affiliation. Tightly synchronised activity in particular, found in team sports such as rowing, can help dissolve the boundaries between individual and social agency: ``In rowing...it feels like you have at your command the power of everybody else in the boat. You are exponentially magnified. What was a strain before becomes easier. It is absolutely the ultimate team sport'' \citep{Brown2016}.
Blurring of agency between self and team may be responsible for facilitating affiliation and trust between teammates in competitive environments often involving high stress and uncertainty: ``...you always wanted a guy who would go into the trenches with you and he always played consistently well...he could really play and was just one of the good lads that you enjoyed his company'' \citep{Fox-Sports2017}. The experience of team click may in this sense act as a social diagnostic tool, a powerful signal of commitment to joint action and willingness to cooperate \citep{Reddish2013a}.  In this dissertation, I draw upon these related strands of evidence from cognitive, neuroscientific, and psychological, and anthropological literatures, in order to develop a novel theoretical account of the relationship between synchronised interpersonal joint action and social bonding, and the mediating role of ``team click.''

\section{Joint Action in Group Exercise}
There is considerable variation in the nature and dynamics of joint action, even within the sub-category of group exercise. Joint action in group exercise ranges from tightly coupled dyadic or group activities such as rowing, synchronised diving, or dance sport, to interactive competitive team sports like basketball, ice hockey, and rugby, through to more loosely coupled (but still time- and space-coordinated) mass participation activities such as marathons and triathlons.  It is sensible to assume that, as the scale and requirements of these contexts vary, so too will the psychophysiological mechanisms most responsible for enabling successful joint action, feelings of team click, and social bonding \citep{Mogan2017,Launay2016}.

Interactive and co-active team sports in particular contain dimensions of complexity that are not directly addressed by the existing experimental literature concerning synchrony or joint action.  The competitive nature of these sporting practices means that co-actors in joint action scenarios will perform roles that either facilitate or obstruct shared goal achievement, depending on team assignment \citep{Renshaw2009}. Competitive joint action scenarios facilitate two modes of communication between individual participants: more predictable behaviour between cooperators and less-predictable action behaviour between opponents \citep{Glover2017}. Thus the competitive dimension of interactive team sports introduces complexity, whereby subunits of cooperating co-actors coordinate their behaviours around a shared goal (winning the specific contest) \citep{Passos2012},  and co-actors from both teams coordinate with each other around the higher order shared goal of completing a competitive game.
In addition, interactive team sports involve the nesting of coordinated subunits of actors and sub-phases of actions \citep{Vilar2012}.  For example, a dyadic joint action such as passing a ball between two attacking players in association football is nested within a larger attacking sub-phase goal of advancing towards the opposing team's goal in order to score a goal, which is in turn nested within a larger shared goal of beating the opposing team in a 90 minute match, and so on.  These dimensions of complexity in interactive team sports increase the overall degrees of freedom of joint action tasks, thus demanding higher technical competence in order to successfully establish functional interpersonal synergies capable of reducing such uncertainty and behaving adaptively \citep{Duarte2012}.

\subsection{Rugby Union Football}
Rugby Union (hereafter rugby) is an interactional team sport played on a rectangular field (100m x 70m), by two teams, usually of 15 players, who physically contest possession of an egg-shaped ball that can be used to score points \citep{IRB2014}.\footnote{Descending from a variety of locally-specific folk-games played in pre-industrial England, all loosely grouped as ``football'', rugby developed within the elite public school system as a deliberate physical activity arbitrated by rules and regulations, before circulating through the arteries of England's colonial empire as a leisurely pastime—a ``sport'' \citep{Dunning2005}.}
``Rugby sevens'' (hereafter Sevens), the version of rugby that is the focus of this research, is a shorter 7-on-7 version of rugby. Sevens is a highly interactive and physiologically demanding sport at all levels at which the game is currently played, even more so than the 15-a-side version of the game.   Sevens requires players to participate in frequent bouts of intense (anaerobic) activity such as sprinting, physical collisions, tackles, and grappling, separated by short bouts of low intensity activity such as walking and jogging. Sevens requires high levels of interdependence between team members due to the uncertainty and complexity of interactive coordination tasks.  At the elite level in particular, the physiological costs and complexity of joint action requirements of sevens are amplified.

There is evidence to suggest that dynamic coupling occurs between dyads and sub-units of attack and defence in rugby \citep{Passos2011,Correia2014}.  Passos and colleagues \textcite{Passos2011} for example find that functional coupling tendencies emerge between attacking dyads and adapt to specificities of the task environment.  Correia and colleagues \textcite{Correia2014} show that coupling tendencies also emerge between co-actors of opposing teams in rugby, for example, in a 1-on-1 attacker/defender sub-phase.  These results are confirmed in similar joint action contexts in other equivalent sports such as basketball and association football \citep{Duarte2013}. There is evidence to suggest that the establishment and maintenance of functional interpersonal synergies in rugby joint action depend on an athlete's perception of affordances of the task-specific cognitive system made up of constraints including other athletes, the physical environment, and the rules of the game \citep{Passos2012}.

Very little direct empirical evidence specific to rugby can be used to substantiate a link between joint action and team click, and team click and social bonding.  Rugby is, however, a sport heavily associated with ``social bonding'' in the more popular discursive sense, particularly in all-male social organisation common in the elite educational institutions of England and Commonwealth countries in which rugby originally developed \citep{Dunning2005,Richards2007,Collins2008}.\footnote{Rugby union has been the site of much criticism worldwide due to the fact all-male social spaces cultivated by rugby appear to support and sustain hyper-masculine and hyper-normative behaviours, including gender-related violence \citep{Cosslett2014,Guinness2016}.
}   ``Rugby is a game for barbarians played by gentlemen,'' or so the saying goes.\footnote{The origins of this oft-cited Rugby adage is unclear.  The phrase is supposedly the adopted motto of the British Barbarians Football Club, established in 1890 \citep[34]{Dunning2005}.  The complete phrase reads ``Rugby is a game for barbarians played by gentlemen, football is a game for gentlemen played by barbarians.''  However, official club history cites its original motto as, ‘Rugby Football is a game for gentlemen in all classes, but for no bad sportsman in any class' \citep[vii]{Starmer-Smith1977}.  Some sources attribute the saying to British writer and poet Oscar Wilde (1854-1900) \citep{Fleenor2015}}. Different inflections on this adage have been reproduced by people in all parts of the world that rugby has reached (including China), presumably as a way of linking the nature of rugby's physical requirements with social virtues of fair play, cooperation, and moral integrity.  Although direct experimental evidence concerning rugby is scant, the physiological demands, joint action complexity, and social-historical trajectory of rugby suggests that it is extremely suited to an investigation of the social bonding effects of joint action in group exercise.

\section{Cultural Variation}
In addition to micro-level details and dynamics of joint action, macro-level variation in the cultural contexts of joint action also vary extensively. Importantly, macro-cultural expectations appear to frame and direct micro-level movement dynamics of joint action.  As sporting anecdote indicates, different teams from different places and times appear to play the same game in very different ways---embodying different ``styles'' of play.  While there is very little literature devoted to examining the effect of cultural variation on joint action and social bonding in particular, there is extensive evidence to suggest that cultural variation impacts on processes of cognition \citep{Nisbett2003,Hoshino-Browne2005}, social learning \citep{Mesoudi2015}, and prosocial behaviour \citep{Yuki2005,Yuki2003}.
It has been suggested that cultural environments structure joint action scenarios in ways that help ``smooth'' coordination by providing equivalent expectations between co-participants \citep{Vesper2017}.  Indeed, as anecdote and observations concerning suggest, perception of ``team click'' is not necessarily limited to the most proximal dimensions of joint action perception, but is rather contingent on the snug fit between a given joint action and a whole assemblage of hierarchically ordered expectations.\footnote{It is also important to bear in mind that, while the neurological, cognitive, and psychological theories from which the predictions of this dissertation strive for universal generalisability, these theories are nonetheless heavily grounded in Western epistemological assumptions, intuitions, and ``WEIRD'' empirical evidence \citep{Henrich2010a}.}

\subsection{Rugby in China}
Rugby union football and the People's Republic of China are subjects not commonly heard uttered in the same breath.  Nonetheless, the Olympic status of rugby union, and the deep Olympic logic of the state-sponsored Chinese sports system, means that today hundreds of professional Chinese rugby players are meaningfully engaged in one of the world's most physiologically strenuous interactive team sports.  Rugby has been a professional sport in China for seven years (in the form of the Olympic event rugby sevens), before which it had existed as a non-professional university sport for 20 years, first established in 1990 at the Chinese Agricultural University, Beijing. Rugby is part of a large collection of ``cold-gate''\footnote{\textit{lengmen xiangmu}, a term that refers to a profession, trade, or branch of learning that receives little attention} sports in China, with a relatively small participation base compared to other interactive team sports like basketball or football.
However, due to the persistent Olympic focus of the Soviet-modelled Chinese competitive professional sports system (\textit{juguo tizhi}), rugby's recently acquired Olympic status (announced in 2009) in the form of rugby sevens means that it has now been inducted into the state-sponsored competitive sports system and is one of 33 sports featured in the all important quadrennial National Games.  Ten of China's collection of 32 provinces and municipalities that participate in the National Games have full time men's and women's rugby programs.

While football and basketball have matured as standalone market-based professional industries, most other sports in China (i.e., all other Olympic events, including rugby) exist primarily due to the support of the enormous state-sponsored national provincial sport system.  Whereas the commercial basketball and football industries might offer a small percentage of prospective athletes incentives of fame and fortune, the benefits of a state-sponsored sports programs like rugby are more modest.  Chinese youth either gravitate or are ushered by their parents towards  sporting careers primarily due to potential life-course opportunities such as access to tertiary education and post-athletic career employment (in the government sports system).  The extent to which an athlete is able to maximise these potential benefits depends on the strength of an athlete's performance record (\textit{chengji}).
The most important measure of a province's success in state-sponsored sporting terms is the National Games, a quadrennial multi-sport event hosted on rotation by provincial capital cities \citep{Hong2002}.  The amount of funding a province and its provincial sporting institutes and programs receive is decided to a large extent by results at the National Games.

The history of sport and exercise in China's modern transformation is, in many fascinating ways, emblematic of that history itself.  The introduction in the late 19th Century of a new ethics of group membership centred around the activities of the nation-state, required importation \textit{en masse} of novel linguistic, cultural, and social categories and practices \citep{Liu1995}.
Throughout the 20th Century, physical culture (\textit{tiyu}) became a primary pedagogical vehicle for fostering an explicit link between the strength of the physical body and the strength of the Chinese nation \cites[32]{Morris2004}[49]{Brownell1995}.  From the initial embrace of the Olympic Games by an urban Chinese elite at the turn of the century, through to Beijing's eventual hosting of the Olympics in 2008, and now the use of sport as a site of domestic commercial consumption, physical culture has provided the means through which new and normative ways of thinking and behaving have been publicly displayed and transmitted.
Inherent in this process has been the tension and interaction between imported categorical modes of group membership fostered by the state (i.e., civil society, citizenship), with more local and indigenous understandings of social identity centred on intragroup relational processes rooted in Confucian, rural, and dynastic cultural traditions \citep{Fei1992}.  Physical culture in China choreographs---perhaps more explicitly than any other facet of contemporary Chinese life—--the interaction between imported and traditional modes of group membership at the psychological heart of China's culturally-specific patterns of sociality.

The professional Chinese rugby players that form the focus of this dissertation are young men and women predominantly from rural areas of China's northeast, and are therefore likely subject to relational modes of group membership made predominant and durable by persistent cultural and linguistic processes associated with group membership in contemporary China \citep{Liu2009}.  In addition, athletes are members of a relatively small and stable team environment, for which access to benefits should require attention to the maintenance of productive intragroup relationships, more so than processes of intergroup mobility or comparison \citep{Schug2010}, even though intergroup comparison is an inherent component of competitive interactional team sport.
I expect the metaphors of family to be prominent in the scaffolding of team processes, and I also expect to find a tension between loyalties to the team, and loyalties to an athlete's pre-existing relational networks of family and friends \citep{Yang1994}.  Thus, the specific cultural setting is such that I do not necessarily expect to encounter the type of public representation or testimony of group membership common in Western sporting parlance, more indigenous to the rugby pitches and boathouses of Oxford or any high school American Football movie produced in the 1990s.  Instead, in my ethnography and in the subsequent field studies, I expect to find evidence of a link between joint action, team click, and social bonding expressed in cultural representations that may vary distinctly in structure and content from Western intuitions.

\section{Overview of research}
During a two year period between August 2015 and September 2017, I spent three separate periods in China during which time I conducted a total of 10 months of ethnographic research with the Beijing Men's Provincial Rugby Team, as well as two field studies, for which I sampled from a broader population of professional Chinese rugby players from 9 different provinces.  Between August 2015 --- March 2016, I spent seven months in Beijing engaged in participant observation and conducting unstructured and semi-structured interviews (n = 30), and informal surveys (n = 4) with the Beijing Men's Rugby Team.  In July --- August 2016 I returned to China for a further two months, during which time I continued ethnographic observations of the Beijing team, while also conducting two pseudo-experimental field studies spanning two other locations in addition to Beijing: Hebei province and Shandong province.  Finally, I spent one month in Beijing and Tianjin between August --- September 2017 during which time I conducted follow-up structured interviews the with 10 athletes who participated in the Chinese National Games, as well as follow up interviews with athletes from the Beijing Men's Rugby Team.

\clearpage


\section{Thesis Overview}
This dissertation consists of the following contributions:

\begin{enumerate}
  \item \textbf{Introduction}
  \item \textbf{A novel theory of social bonding through joint action.}  This section features an introduction to the phenomenology of ``team click'' as a potential mediator of the relationship between perceptions of joint action and social bonding. The theoretical synthesis outlined in this section forms the basis of study predictions \textit{Selections from this section, including study predicitons, are included in my CoS assessment materials.}
  \item \textbf{Introduction to the ethnographic setting.} In this section I outline in detail the joint action requirements of rugby, as well as the cultural specificities of the niche in which the joint action occurs (China).
  \item \textbf{Analysis of ethnographic data}.  In this section I present the results of analysis of interviews, participant observation, and informal ethnographic surveys collected with the Beijing men's rugby team. I find evidence of performance-related anxiety, strongest among younger less experienced athletes. I also find evidence for an overlap of individual and team agency, with more experienced athletes emphasising individual agency and younger athletes emphasising team agency. In addition, I find strong evidence for the phenomenology of ``team click'' among athletes.  The terms and discourses encountered through ethnographic observation, particularly regarding perceptions of joint action, notions of ``team click,'' and understandings of social bonding, inform the design of survey items for subsequent field studies.
  \item \textbf{National Tournament survey study}.  In this section I report the methods and results of the national survey study, in which I test the prediction that the relationship between perceptions of joint action and social bonding is mediated by ``team click.'' \textit{Selections from this section are included in my CoS assessment.}
  \item \textbf{Training Experiment.} In this section I present the results of the controlled field experiment.  The training experiment was designed to further test the relationship between joint action, team click, and social bonding in an environment in which performance was less influenced by external sources of explicit performance feedback (i.e., winning or losing the game).  58 Professional Chinese rugby players (men = 30) participated in a between-subjects design involving two manipulations of a common rugby training drill known as ``invasion drill'' \citep{Passos2011}.  In the ``low uncertainty'' condition, athletes were primed with information to suggest that the training drill would be very easy to complete (2/10 difficulty rating).  In the ``high uncertainty'' condition, athletes were primed to expect the training drill to be relatively difficult (8/10 difficulty).
  Pre and post survey measures were recorded, along with video footage of each experimental trial.  It was predicted that those athletes in the high-uncertainty condition would experience higher levels of team click and social bonding owing to higher than expected positive violation of expectations around group performance.
  Athletes in the ``low uncertainty'' condition would on average experience less positive violations of expectations, and thus would feel less strongly the phenomenon of ``team click'' and flow-on feelings of social bonding.  Video footage was analysed for evidence of dynamic coupling between co-actors as well as defenders \citep{Schmidt2011,Richardson2012,Passos2012}, and these data were compared to psychological measures.  Results are yet to be fully analysed.
  \item \textbf{General Discussion.}  In this section I make inferences about the findings of each empirical chapter, identify weaknesses, and point towards future research directions.
  \item Conclusion
\end{enumerate}

\clearpage
\section{Timeline for Completion}
I propose the following 11 month timeline for completion:\\


\begin{description}
  \item [2017]
    \item [September:] Prepare for CoS Viva, Analyse training experiment
    \item [October:] Analyse training experiment and prepare report
    \item [November:] Finalise training experiment chapter, analyse ethnographic data
    \item [December:] Analyse ethnographic data\\
  \end{description}

  \begin{description}
    \item [2018]
    \item [January:] Prepare ethnographic chapters
    \item [February:] Prepare general discussion
    \item [March:] Compose introduction and conclusion
    \item [April:] Submit 1st Draft
    \item [May:] Compose 2nd Draft
    \item [June:] Submit 2nd Draft to EC
    \item [July:] Final submission
  \end{description}



%\subsection{Qualifications of the researcher}
%The research of this dissertation was facilitated by a confluence of factors, many of which have to do with my personal background and experience with both rugby union, China, and rugby union in China specifically.  I have professional-level proficiency in Modern Standard Chinese (HSK Level 6), having studied the language for many years, including one year of intensive language study at Liaoning University in 2006, and 8 months studying sociology and social anthropology alongside the local undergraduate cohort at Beijing University in 2008.  On the rugby side, I have extensive personal experience with in the sport. I have recently completed a career as a professional rugby player, which culminated in representing the Australian Rugby Sevens Team (2009-2012). My personal rugby background meant that while based in China I developed a relationship with athletes and coaches of the national Chinese rugby team.
%Following my own professional rugby career (2009-2012) and before beginning post-graduate study, in 2013 I spent 8 months coaching the Chinese National Youth Rugby Sevens Team in preparation for the 2013 Nanjing Asian Youth Olympics.  These various factors facilitated direct access to research participants and meant that I was able to conduct research in Modern Standard Chinese, and did not require extended period of ethnographic immersion for the purposes of language acquisition or relationship development.



%\section{Cognitive \& Evolutionary Anthropology of Group Exercise and Social Cohesion}


%Physical activity, exercise, and sport have well-known positive effects on physical and  psychological health (Ekkekakis, 2003; Fiuza-Luces, Garatachea, Berger,  Lucia, 2013).
%The health benefits associated with regular exercise, including reduced risk of cardiovascular disease, autonomic dysfunction, and early mortality, are becoming increasingly well-known (Blair  Powell, 1994; Nagamatsu et al., 2014).

%While the physiological, psychological, and social processes that combine in instances of exerted, coordinated movement are rich and varied, many strands of research suggest a link between group exercise and social bonding \citep{Davis2015,Cohen2017}. It is now understood that strenuous and prolonged physical exercise is modulated by the same neuropharmacological systems (namely, the opioidergic and endocannabinoid systems) responsible for regulating pain, fatigue, and reward \citep{Boecker2008,Raichlen2013}.
%Exercise-specific activity of these systems offers a plausible neurobiological explanation for commonly reported sensations of positive affect, anxiety reduction, and improved subjective well-being during and following exercise---extremems of which asre popularly referred to as the ``runner's high'' (Dietrich  McDaniel, 2004; Boecker et al. 2008; Raichlen, Foster, Gerdeman, Seillier,  Giuffrida, 2012).  This neuropharmacological account of group exercise and social bonding has its roots in studies of social grooming in non-human primates.  Dunbar and colleagues propose a neuropharmacologically mediated affective mechanism linking dyadic grooming practices with group-size maintenance \citep{Machin2011}.\footnote{The capacity for social bonding is thought to have arisen in primates as an adaptive response to the pressures of group living.  Aggregating in groups serves to reduce threat from predation.  At the same time, it can be individually costly due to stress arising from interaction
%at close proximity and conflict over resources among genetically unrelated individuals.  These pressures are hypothesised to have led to selection for social bonding (e.g., via dyadic grooming). Resulting coalitional alliances among close partners allow for the maintenance of the group by buffering the stresses of group living.  Primate social grooming, for example, is associated with the release of endorphins, presumably leading to sustained rewarding and relaxing effects.  While other neurotransmitters such as dopamine, oxytocin, or vasopressin may also be important in facilitating social interaction, endorphins allow individuals who are not related or mating to interact with each other long enough to build ``cognitive relationships of trust and obligation'' \citep{Dunbar2012}(1839)}  It is thought that, as the homo genus evolved more complex collaborative capacities for survival in interdependent group contexts, grooming-like behaviours sustained social bonding in larger groups wher
%e dyadic grooming would cumulatively take too much time \citep{Dunbar2012}.
%Experimental studies suggest that neurophysiological mechanisms activated by activities that involve physical exertion and coordinated movement, such as group laughter, dance and music-making, exercise, and group ritual can bring groups closer together, mediated by the psychological effects of endogenous opioid and endocannabinoid release \citep{Cohen2009,Fischer2014a,Fischer2014,Sullivan2014,Tarr2016,Tarr2015}.

%In addition to reports of exercise-induced euphoria and positive affect, adherents to (group) exercise and other activities---particularly highly skilled practitioners---also commonly report experiencing states of ``optimal'' or ``peak'' performance, which include feelings of heightened focus, personal transcendence, time-warp (the experience of time either speeding up or slowing down), spontaneity, creativity, and effortlessness \citep{Jackson1995a}.  ``Flow,'' as this particular cluster of states has commonly been referred to, is a powerful, autotelic and embodied experience, which combines components of both ``hedonic'' (sensation-centred, see \citep{Huta2010}) and ``eudaimonic'' (meaning-centred, see \cite{Ryff1989,Ryff2015}) dimensions of subjective well-being, and is theorised to emerge when activity strikes a balance for the individual between challenge and skill requirements \citep{Csikszentmihalyi1990,Abuhamdeh2012}.  At the level of the group, the ``team click'' and ``group f
%low'' are highly elusive possibilities, coveted by athletes, coaches, and fans alike \citep{Novak1993,Sawyer2006}.  While the experience of flow associated with prolonged exercise may be in part neuropharmacologically mediated by the opioidergic and endocannabinoidergic systems, phenomenological accounts suggest that there is something distinct about the experience of flow in exercise that requires a more complete cognitive and social explanation \citep{Dietrich2006,Dietrich2011}.  One speculative neurocognitive account of acute exercise, for example, suggests that the metabolic costs associated with complex or prolonged regulation of movement forces an energetic trade-off in the brain in which lower level neurocognitive processes win out, forcing a down-regulation of the pre-frontal areas of the brain \citep{Dietrich2011}. Dietrich and colleagues propose that the down-regulation of cortical processes induces a decline in executive control \citep{Labelle2013} and possibly dampens
%self-monitoring and personal agency. If this hypothesis is correct, it is highly plausible that flow and its neurocognitive underpinnings are relevant to the affective and prosocial effects of group exercise.

%Meanwhile, research in social psychology focusing on the relationship between time-locked behavioural synchrony and processes of self-other merging, social alignment, and affiliation has shed light on the social and affective significance of interactive and coordinated movement typical of many group exercise contexts \cite{Wiltermuth2009,Kirschner2010,Reddish2013,Tuncgenc2016}. Experimental evidence suggests that time-locked coordination of behaviour between two or more individuals in the stable attractor/equilibrium states of either in-phase or anti-phase synchrony is conducive to psychological processes of self-other merging, liking, trust and affiliation.  It is believed that lower cognitive processes of joint attention mediate the link between synchrony and social bonding, with synchronised activity (common in music, dance, and some sports) providing a shared spatio-temporal (and often haptic) referent around which to coordinate attention and behaviour \cite{Launay2016,Wolf2015}.

%Studies linking synchrony with social bonding and cooperation are supported by a literature than connects nonconscious mimicry with liking and affiliation\citep{VanBaaren2009}.  The experimental studies above predominantly refer to dyad synchronisation of behaviour.  The social and psychological effects of group level synchronisation have been harder to induce and measure in experimental settings. However, in addition to in- and anti-phase behavioural matching, group synchronisation may be subject to more complex and dynamical processes of coupling, which could entail specific psychological consequences. This also appears to be true in cases of joint---but not necessarily explicitly synchronised---action, whereby implicit processes of movement regulation link two or more individuals in a complex and dynamic coupling. The variation and stabilisation of such dynamic couplings could have psychological effects (see \citep{Schmidt2008,Marsh2009a}).  Most encouraging is evidence that manages
%to integrate the social and neurophysiological dimensions of group exercise.  Recent experimental evidence suggests that social features of the exercise environment (for example, perceived social support, level and quality of behavioural synchrony, etc.) modulate exercise-induced mechanisms of pain, and reward \citep{Cohen2009,Sullivan2014,Tarr2015,Davis2015,Weinstein2016}. This work is bolstered by existing literature on the social modulation of pain \citep{Eisenberger2012a} and links between pain and prosociality \citep{Bastian2014a}.



 %I conducted research with Chinese professional rugby players.  The empirical studies of this dissertation consist of an ethnographic analysis of the Beijing Provincial men's rugby team (n = 26), an \textit{in-situ} survey study of Chinese professional provincial rugby players during a National tournament (n = 174), and a controlled field experiment (n = 58, drawn from Beijing and Shandong provincial teams). For the Confirmation of Status, I present samples from a theoretical chapter and results from the survey

 %It has been suggested that mechanisms relevant to synchronised movement in joint action and social bonding range in scope from lower cognitive affective mechanisms implicating neuropharmacological reward systems, through to higher cognitive processes of self-other merging, group centred cognition, and perceived cooperation \citep{Mogan2017}. In particular, affective physiological mechanisms may be more relevant to joint action involving larger group sizes in which generalised feelings of euphoria and pro-sociality are common, whereas neurocognitive mechanisms linking joint action and social bonding may be more applicable to smaller group sizes in which individuals can share intentions through ostensive signals and implicit (emotional) cues \citep{Semin2008,Frith2010}.
 %While it has been well documented that neuropharmacologically mediated affective mechanisms are relevant to joint action scenarios in exertive group exercise contexts\citep{Cohen2009,Sullivan2013,Tarr2015}, the distinct emphasis on the fine-grained \textit{quality} of synchronisation common to subjective reports of technically demanding joint action, suggests a key role of neurocognitive mechanisms in team click.

 %Furthermore, recent evidence indicates that neuromodulatory systems, traditionally considered exogenous to cortical function, are in fact functionally relevant to neurocognitive processes of ``active inference,'' including perception, prediction error minimisation, and movement coordination\citep{Pessoa2013,Krahe2013,Buchel2014,Miller2017}.  Neuromodulatory systems such as the endogenous opioid system may be particularly responsible for the ``affective tuning'' of neural representations \citep{Panksepp1998,Pessoa2013}. These findings suggest that the traditional distinction between lower and higher cognitive processes requires reassessment and development in light of emerging theories of neuro-computation and social cognition \citep{Pessoa2013,Clark2013}.



 %\textit{The pathway from psychological experience of team click to social bonding, via sensations of team click, implicates a cascade of hierarchically-organised interoceptive expectations generated in specific cultural niches.  Positive violations around expectations around team performance appears to be a powerful (but not exclusive) mechanism through which individual and group agency are dynamically interwoven.  The euphoria and transcendentalism experienced when team performance clicks is a phenomenology that requires exegesis and integration into an individual's theory of the world.  The most convincing candidate for this mode of self-defining explanation in group exercise is often the social environment in which joint action is embedded.
 %Thus, in addition to the opportunities to establish and maintain interpersonal synergies with conspefics, and in so doing rehearse cooperation, the (mis)attribution of arousal \citep{Drachman1976} to group-level cognitions could also be a key mechanism responsible for bridging the gap between proximal sensory experience of joint action and prosocial group commitment}


 %In order to test these predictions, I conduct a series of three interrelated studies.  The first study consists of extended ethnographic research with one Chinese professional rugby team, the Beijing Provincial men's rugby team.  The specific ethnographic context (sport in China) demands a careful consideration of the predictions formulated above.  Culturally specific processes of self construal and social group formation challenge assumptions built in to existing theories.  However, I argue that the predictions outlined above are robust to these cultural specificities, due to the fact that they are grounded in an agent neutral distributed social cognition framework. Indeed, despite distinct cultural variation processes of team membership, ethnographic analysis reveals that the experience of team click is strongly identifiable.

%Following this ethnographic research, I broaden my scope of analysis to include all available Chinese professional provincial rugby players.  Study 2 of this dissertation was conducted around a Chinese national tournament in which 174 athletes (men = 93) in 15 different teams from 9 different provinces competed over two days for the 2016 Championship.  This tournament allowed me to investigate my hypotheses concerning the relationship between joint action, team click, and social bonding in a real-world instance of high intensity, high stakes joint action.  Finally, I conducted a controlled field experiment in order to more definitively assess the causal implications of the predictions of this dissertation.  58 athletes (men = 30) were recruited from two provinces (Beijing and Shandong) to participate in a between-subjects design. I present and and analyse the results of these three studies in the chapters that follow.

%Dunbar \textcite{Dunbar1992} proposes that the ratio of human neocortex size to total brain volume imposes an upper cognitive limit on realtime coordination of behaviour of \sim4-5 individuals.  The group size of joint action sub-phases and sub-units in rugby sevens (\sim2-4) fall within this upper limit, or just above the upper limit if attacking and defending subunits are grouped together (\sim4-8).  Each team of 7 is complemented by a further 5 reserves to make up a total squad of 12 who compete in a tournament setting.  In addition, the size of squads that train together outside of tournaments can range anywhere from 16 to 28.
%These group sizes also exist within the cognitive limits for maintaining face-to-face intimate relationships, thought to be in the realm of \sim15-25 \citep{Dunbar1992,Dunbar2010}. These specific requirements of joint action in rugby sevens suggest that neurocognitive mechanisms responsible for tracking and monitoring coordination between co-actors and establishing interpersonal synergies between co actors will be particularly relevant \citep{Mogan2017}.

%\clearpage
%\section{A theory of social bonding through joint action}
%I introduce the phenomenon of team click as an instance of human cultural interaction that has hitherto been under-appreciated---both theoretically  empirically---in the science of human cognition and evolution.  Drawing upon existing research concerning behavioural synchrony and social bonding, it is possible to identify proximate mechanisms that could be responsible for generating the experience of team click and social bonding in joint action scenarios involving less tightly coupled synchronisation.

%Thus, the nagging visceral intuition associated with the observable human compulsion to come together and move together could in fact prove useful as a source of insight for progressing the science of human evolution.


%The history of sport and exercise in China's modern transformation is, in many fascinating ways, emblematic of that history itself.  The introduction in the late 19th Century of a new ethics of group membership centred around the activities of the nation-state, required importation \textit{en masse} of novel linguistic, cultural, and social categories and practices \citep{Liu1995}.
%Throughout the 20th Century, physical culture (\textit{tiyu}) became a primary pedagogical vehicle for fostering an explicit link between the strength of the physical body and the strength of the Chinese nation \cites[32]{Morris2004}[49]{Brownell1995}.  From the initial embrace of the Olympic Games by an urban Chinese elite at the turn of the century, all the way through to Beijing's eventual hosting of the Olympics in 2008, physical culture has provided the means through which new and normative ways of thinking and behaving have been publicly displayed and transmitted.
%Inherent in this process has been the tension and interaction between imported categorical modes of group membership fostered by the state (i.e., civil society, citizenship), with more local and indigenous understandings of social identity centred on intragroup relational processes rooted in Confucian, rural, and dynastic cultural traditions \citep{Fei1992}.  Physical culture in China choreographs---perhaps more explicitly than any other facet of contemporary Chinese life—--the interaction between imported and traditional modes of group membership at the psychological heart of China's culturally-specific patterns of sociality.
