
\begin{savequote}[8cm]
By long years of military experience he knew, and with the wisdom of age understood, that it is impossible for one man to direct hundreds of thousands of others struggling with death, and he knew that the result of a battle is decided not by the orders of a commander-in-chief, nor the place where the troops are stationed, nor by the number of canons or of slaughtered men, but by that intangible force called the spirit of the army, and he watched this force and guided it in as far as that was in his power\dots
  \qauthor{--- Leo \textcite{Tolstoy1956}}
\end{savequote}

\chapter{\label{chap:generalDiscussion}General Discussion}




\section{Overview}
Group exercise in human sociality is an intriguing evolutionary puzzle, and thorough scientific explanations for its prevalence in the human record are currently limited. This thesis developed and tested a novel theoretical account of team click in group exercise.

Sporting anecdote and ethnography suggests that team click is a mysterious yet unmistakable experience of peak performance in joint action.  In this thesis, I developed a general account of team click grounded in a dynamical theory of cognition (the AIF, see Chapter~\ref{sect:AIF}).  A dynamical approach enabled the generation of hypotheses capable of accounting for the coordination of physical and social processes in instances of team click, and in real-world group exercise settings more broadly.  Following formulation of a general account of team click (Chapter~\ref{chap:theory}), I situated (Chapters ~\ref{chap:researchSetting} and ~\ref{chap:ethnoField}), assessed (Chapter~\ref{chap:ethnoResults}), and tested (Chapters ~\ref{chap:tournamentSurvey} and ~\ref{chap:trainingExperiment}) this theory in the real-world group exercise setting of rugby in China.  By providing evidence that team click mediates a relationship between between perceptions of performance and social bonding in joint action, this thesis offers preliminary substantiation of a general account of team click in group exercise.  Findings of this thesis also suggest that positive violation of expectations concerning team performance may precede team click, and function to draw attention to its visceral and socially agentic dimensions.  This chapter will provide a discussion of the findings, point to their limitations, and propose areas for future research.  More fundamentally, in this discussion I propose that future development of the theory of team click in group exercise could serve as a valuable complement to existing theories of group exercise and social bonding.

%I also point to a broader complementarities between dynamical principles of the AIF and existing evolutionary theories of social cohesion and cultural evolution.

\section{Synopsis of findings}

\subsection{Ethnographic evidence validated research hypotheses}
Ethnographic evidence from rugby in China situated and validated research hypotheses of a general account of team click.  Following an introduction to the research setting, I situated a general account of team click through an ethnography of the Beijing men's rugby team.   Results demonstrated that athlete experience of rugby's joint action was associated with salient levels of uncertainty and coordination.  In addition, I found clear evidence to suggest that athletes experienced visceral and socially agentic dimensions hypothesised to underpin team click.

\subsubsection{Rugby in China entailed uncertainty and coordination}
First, ethnographic evidence validated the claim that team click emerges from a dynamical relationship between uncertainty and coordination in joint action (see Chapter~\ref{sect:coordinationPerform}).  Evidence
that athletes experienced rugby's joint action as challenging, with team performance perceived as more challenging than individual performance, supported the suggestion that the parameters of joint action in rugby reliably generated uncertainty. Athlete testimonies relating to the importance of on-field awareness as a mitigating strategy for the challenges of joint action, and the fact that awareness appeared to develop with familiarity with the sport, was treated as evidence to suggest that with perceptions of uncertainty, rugby's joint action also entailed perceptions of coordination.  From these results it was possible to suggest that higher levels of perceived difficulty associated with components of team performance were associated with lower expectations for team performance success, and that this relationship was underwritten by levels of uncertainty in joint action.  Thus, ethnographic evidence validated H2a: that higher levels of uncertainty in joint action were associated with lower expectations for team performance success (denoted by perceived difficulty or challenge).These results suggested that despite the diversity of cultural affordances salient in the Beijing men's rugby team, the set parameters of rugby's joint action reliably fostered experiences of physical and social coordination.

This ethnographic analysis demonstrated the importance of detailed contextualisation of theoretical claims when performing \textit{in situ} research.  Evidence reported in this chapter demonstrated that diverse cultural processes can act as affordances that shape athlete perceptions of uncertainty and coordination in joint action \citep{Ramstead2016}.  Athletes appeared to navigate the challenges of on- and off-field coordination through recourse to a dominant relational mode of social cognition.  As a researcher not indigenous to this relational mode, I often experienced group processes as jarring and counter-intuitive.  But, over time, I learnt that processes of coordination were imbued with a deeply social logic, of which athletes' strategic pursuit of life-course opportunities of education, concerns for relational \textit{guanxi} networks, and group ethics of personal cultivation were all constituent components.  From these results I inferred that athlete experience of uncertainty and coordination in joint action were both present and salient, albeit in terms that expressed a relational mode of social cognition that was distinct from my own deep-seated intuitions as a researcher familiar with rugby contexts elsewhere. Here, the concept of cultural affordances \citep{Ramstead2016} allowed for a conception of the relationship between uncertainty and coordination, upon which the general account of team click is grounded.  In effect, I inferred from ethnographic data that affordances pertaining to rugby's joint action parameters, contemporary China's relational mode of social cognition, and specific strategic motivations for adherence to rugby, interacted to enable and constrain experience of joint action. In sum, this ethnographic situation of dynamical principles of joint action of rugby in China validated further application and testing of a general theory of team click in group exercise.


\subsubsection{Team click existed in context}
Having identified the key variables of relevance to a general account of team click, I presented evidence that overwhelmingly confirmed athlete recognition and perceptions of the phenomenon of team click.  Perceptions of team click appeared to be derived from athlete perceptions of team and individual performance, including the experience of positive violation of expectations concerning success of joint action.  Evidence also suggested that athletes experienced stress, anxiety, and guilt related to pressure of performance, which can be taken as indirect evidence for the way in which the uncertainty of joint action bears upon processes of physical and social coordination.  Together, these results validated the hypothesis that more positive violation of team performance expectations predicted higher levels of team click (H2b).  Further, evidence of an immediate and automatic tendency to utilise social explanations (rather than purely technical explanations) for team click (or its lack) suggested that, as hypothesised, higher levels of team click will lead to higher levels of social bonding (H1b). Taken together, results from the ethnographic components of this thesis served to validate the core theoretical claims of a general theory of team click in group exercise, and encouraged the formulation of testable predictions and field experimental research methods.

According to my knowledge of published literature, this analysis represented the first thorough ethnographic portrayal of contemporary Chinese athletes since anthropologist Susan Brownell's seminal work.  Also based on my knowledge of available literature, this analysis represents the first attempt to situate dynamical principles of uncertainty (i.e., variational free energy \citep{Friston2010}, coordination \citep{Kelso2009}, and cultural affordances \citep{Ramstead2016} in joint action to a real-world ethnographic setting.  Results should therefore be understood in this context—--as novel contributions that raise as many questions as they answer.

\subsection{Field experiments confirm core hypothesis but leave key questions unanswered}
Overall, findings from two field experiments served to confirm the core hypothesis of this thesis (H1),  but also left many questions concerning mechanistic underpinnings of team click unresolved (H2).

\subsubsection{Team click mediates a relationship between joint action and social bonding in a high-stakes National Tournament}
Findings showed that athlete perceptions of team click mediated a relationship between of joint action and social bonding (H1) in a high-stakes \textit{in situ} professional rugby tournament in which uncertainty in joint action was assumed to be high.  Specifically, results revealed significant relationships between positive perceptions of team performance and team click (Prediction 1.a), as well as team click and social bonding (Prediction 1b).  The direct relationship between perceptions of team performance and social bonding was also significant (Prediction 1c), and evidence of this significant direct path motivated mediation analyses in which the direct effect of perceptions of team performance on social bonding was fully explained by perceptions of team click (Prediction 1d).  In addition, results showed that more positive perceptions of team performance relative to prior expectations predicted higher levels of team click (Prediction 2a).  The direct relationship between more positive perceptions of team performance and social bonding was not significant (Prediction 2b).  Taken together, these results offered the first empirical substantiation of a theory of team click in group exercise. On average, athletes who felt the click of joint action also reported higher levels of social bonding to their team.

In addition to this result, additional results suggested that positive violation of expectations concerning team performance preceded team click (Prediction 2a), potentially by serving to direct athlete attention to visceral and socially agentic experiences associated with team click.  However, results did not support the predicted direct relationship between positive expectation violation and social bonding (Prediction 1b), which suggested that positive expectation violation to drive bonding effects in joint action.


\subsubsection{Training experiment failed to manipulate uncertainty (or control for performance)}
The results of a follow-up training experiment did not directly confirm the hypothesis that better-than-expected execution of joint action will mediate a relationship between prior levels of uncertainty and team click (H2).  Motivated by ethnographic and theoretical evidence to suggest that uncertainty should be a factor in athlete experience (see Chapter~\ref{sect:uncertaintyAthlete}), a field experiment was designed in which uncertainty was manipulated via pre-drill prime of high or low difficulty.  Manipulation checks suggested that the experimental manipulation failed to successfully alter athletes' prior expectations for team performance (i.e., confidence in team performance; (Prediction 1a), or perceptions of team performance relative to prior expectations (Prediction 1b).  The fact that the experimental design did not satisfactorily control for levels of performance may have also explained the fact that no significant condition-wise effects were found.

Following these null results, a supplementary analysis was performed, in which participants were collapsed into one data set.  Results of this analysis did confirm the prediction that more positive perceptions of team performance relative to prior expectations were associated with higher levels of team click (Prediction 2), and higher levels of team click were also significantly associated with higher levels of social bonding (Prediction 3), to the training group and to the Provincial team (in the Post-Tournament data set only).  More positive violations of expectations concerning team performance did not directly predict social bonding (Prediction 4) and, as such, the relationship between positive expectation violation and social bonding mediated by team click could not be appropriately assessed (Prediction 5).  While these findings offered some confirmation of results reported in the National Tournament study, the failure of the experimental manipulation raised concerns around 1) the validity of hypothesis concerning the role of uncertainty in joint action, and 2) methodological approaches to field experiment design, including the challenges associated with operationalising and quantifying dynamical processes such as uncertainty, coordination, team click, and their social effects \textit{in situ}.

Considered together, these field studies offered meaningful substantiation of a general account of team click in a real-world group exercise setting (H1).  In addition, there was also encouraging evidence to suggest that positive expectation violation may be responsible for generating feelings of team click (but perhaps not social bonding; H2).  In sum, these results suggest that team click may explain a relationship between group exercise and social bonding.  In essence, successful coordination in joint action typical of group exercise can be a visceral and agentic source of social connection \citep{Marsh2009}.








  \section{Contributions to cognitive and evolutionary understandings of group exercise}
In this section, I consider the relevance of a general account of team click to existing theories of group exercise and social bonding, as well as the implications of team click---and its proposed dynamical underpinnings---for cognitive and evolutionary approaches to human sociality more broadly.  First, I suggest that team click can serve as a complement to existing accounts of group exercise and social bonding, by helping explain the fleeting and dynamical dimensions to the bonding effects of group exercise.  Second, I suggest that dynamical approaches to the social cognition of joint action could enhance evolutionary understandings of processes such as social cohesion and cultural evolution (Research Question 4, see Chapter~\ref{sect:researchQuestions}).


  \subsection{Group exercise and social bonding}
The account of team click in group exercise developed in this thesis offers a dynamical compliment to existing accounts of group exercise and social bonding.  Generally speaking, existing accounts hypothesise the operation of discrete proximate mechanisms that generate social bonding.  It is possible that,  if properly developed and substantiated, a theory of team click in group exercise could offer a second-order framework for understanding how already-identified component mechanisms of bonding in group exercise are subject to dynamics of uncertainty and coordination in real-world settings.

Anthropologist Harvey Whitehouse has termed this general cognitive and evolutionary approach to understanding human cultural practices (group exercise, in this case), as one of ``fractionating'' elements of the category of interest (group exercise, in this case) in order to see how they independently affect social bonding \citep[][2]{Whitehouse2014}.  Existing theories of social bonding in group exercise identify links between group exercise and social bonding by hypothesising the activity of discrete cognitive and physiological mechanisms, such as neuropharmacological reward \citep[as a result of either behavioural synchrony; see][]{Dunbar2010,Tarr2014} or physiological exertion \citep{Cohen2009,Davis2015}.  Researchers have also proposed that mechanisms associated with pain \citep{Bastian2012,Xygalatas2013}, autonomic arousal \citep{Swann2010a}, and down-regulation of cortical processing \citep{Dietrich2004,Slingerland2014} could be associated with bonding effects in group exercise contexts.  While a fractionating approach has (and will continue to) contribute to the development of an increasingly granular picture of the component mechanisms of social bonding in group exercise, pure fractionation alone will be less able to account for the ``mystery'' of phenomena such as the fleeting and delicate achievement of peak team performance (see Chapter~\ref{}).

As proposed in this thesis, team click is a phenomenon that cannot be explained by appeal to one fractionated element alone.  Instead, team click appears to depend crucially on the coordination of various neurological, physiological, and social mechanisms as a response to uncertainty.  Here, I suggest that a general account of team click can complement existing fractionated elements of social bonding in group exercise, by offering a theory of how—i.e., under what circumstances---these elements coordinate to produce the observable quantities of behaviour and experience in joint action.

In other words, not one discrete mechanism alone (e.g., ``endorphins''), or even an aggregation of several discrete mechanisms (e.g., ``endorphins plus mirror neurons'') can satisfactorily account for the empirical phenomenon of team click.  What is needed, in addition to fractionation of discrete elements of the relationship, is an explanation for how these mechanisms \textit{coordinate} to produce assemblages of action, perception, and experience \citep{Kelso2013}.  From one day to the next, athletes, music makers, soldiers, or participants in costly ritual practices perform quantitatively similar activities in which they physically (and socially) exert and synchronise.  But, existing anecdotal and ethnographic evidence suggests that despite best collective efforts and intensions, for some distinct and hitherto unexplained reason, it is not guaranteed that joint action in group exercise will engender positive psychological processes characteristic of social bonding \citep[e.g.,][]{King2011}.  Here, I propose that an explanation of the mystery of fleeting collective performance in joint action requires a second-order explanatory framework that can explain how constituent physiological, neurological, psychological, and sociological mechanisms coordinate to generate quality (and not just quantity) of performance in joint action. A more developed and empirically substantiated account of team click can offer such a framework for social bonding in group exercise.

The behavioural synchrony and social bonding literature offers an example of the complementarity between team click and existing theories of group exercise and social bonding.  It has been reliably shown in experimental settings that behavioural synchrony—defined as time- and phase-locked matching of physical movement—reliably facilitates social bonding \citep{Launay2016}, prosociality \citep{Kirschner2010}, and cooperation \citep{Reddish2013}.  But, there is also accumulating evidence to suggest that the dynamical properties of physical coordination in joint action may bear upon social outcomes in important ways unaccounted for in the synchrony literature \citep{Fusaroli2014}.

Analysis of naturalistic joint action scenarios reveals that overt time- and phase-locked matching of movement is often complemented by coordination dynamics that may crucially modulate synchrony's social effects.  November and Keller (2014:8) explain that synchrony often serves to occlude the dynamical complexity of joint action:

\begin{quote}
    ..most actions performed by humans consist of a series of different elements that can be combined in a potentially infinite number of ways (similarly to morphemes composing a word or words forming a sentence), and...it is the combination of these elements, and the context in which they are produced, that determine their meaning.
\end{quote}

Highly skilled and multilayered joint action of the kind examined in this dissertation is characterised by the need not only for correct timing (as in strict behavioural synchrony), but also for the correct movement sequencing \citep{Palmer2003}.  Predicting the structure and sequence of actions from an infinite number of possibilities calls for additional cognitive resources, such as stronger use of cortical motor areas for action simulation and representation \citep{Bekkering2009}, or lower cognitive mechanisms of movement regulation that support the emergence of functional interpersonal synergies \citep{Riley2011}.

As outlined in Chapter ~\ref{chap:theory}, joint action success may be contingent on layers of coordination that are more dynamical.  Emerging research from psychology and the cognitive science of joint action has reliably shown that indicators of optimal synchronisation in real-world joint action scenarios (e.g., a music ensemble performance) rarely take the form of exact matching of behaviours.  Rather, peak performance in complex joint action appears to be predicted by the existence of dynamical coupling (e.g. of postural sways of musicians) rather than more discrete measures of exact coordination between co-actors (e.g., the synchronisation of musicians with the movement of an ensemble conductor's wand \citep{Miyata2017}. In addition, there is evidence to suggest that the act of symmetry breaking appears to be just as important to interpersonal coordination as symmetry formation represented by exact behavioural synchrony \citep{Richardson2015}.  These dynamical cognitive processes may also involve related social effects \citep{Marsh2009}.  Taken together, it appears that dynamical processes of movement regulation serve to modulate coordination in joint action at levels less appreciated by the synchrony—bonding literature \citep{Launay2016}.

As such, appreciation of the real-world (i.e., dynamical) social bonding effects of synchrony may require a dual consideration.   When asking how bonding is generated in joint action, it appears necessary to ask two questions:
one concerning the componential cognitive and neuropharmacological mechanisms that are known to generate social bonding effects, and one concerning the (system) dynamics within which these componential mechanisms are enabled and constrained \citep{Coey2012}.

Considering the second question from the perspective of the AIF,  it can be predicted that the social bonding effects of behavioral synchrony for any given individual will be contingent upon whether or not that individual attends to synchrony-induced experiences as salient.  In other words, the bonding effects of synchrony will be contingent not just upon whether or not synchrony is achieved or not, in an explicit binary sense.  Instead, perceptions of synchrony in joint action can be understood in terms of a function of uncertainty, such that: higher levels of uncertainty will lead to higher salience or attention directed towards the sensory event under circumstances in which the unlikely event occurs (e.g., shock, and awe).  In this sense, social connection in joint action will be underwritten not just by the achievement of exact behavioural synchrony (or exercise-induced activation of neuropharmacolgocical reward, but by two or more co-actors occupying a shared dynamical manifold  that is sustained by the ability of this coupling to reduce uncertainty inherent in the movement system of which each individual is a component part (Marsh2009; Riley2011;Fusaroli2014).  One way to test the proposal that team click is relevant and complementary to the synchrony—bonding literature would be to design an experiment in which exact synchrony is held constant as perceived uncertainty in joint action varies.



\section{Limitations and directions for future research}
The general account of team click presented in this thesis is preliminary, and only provisionally substantiated. In particular, its explanatory power is limited by theoretical and methodological concerns, which I outline below.  In particular, I suggest that future development of an account of team click in group exercise will need to pay much closer attention to the operationalisation of key variables of interest, namely, uncertainty, team click, social bonding, and expectation violation.

\subsection{Uncertainty and coordination}
Uncertainty as defined in this thesis lacks rigorous development for the purposes of in situ behavioural research.  The AIF itself is an emerging theory well-supported by mathematical formalisms and preliminary neurocomputational and clinical applications, but with very few applications to instances of human behaviour in the field \citep{Clark2013}.  Thus, the failure of the Training Experiment to demonstrate a clear empirical link between uncertainty and joint action is not overly surprising at this early stage of application.  However, the reasons for a null result in relation to the uncertainty manipulation are more fundamental than a question of logistics and location, and future studies must carefully consider how uncertainty can be operationalised in context.

In experimental research, uncertainty has been aroused either via the introduction of time-pressure and novel learning requirements on cognitive decision-making \citep[see:][]{Daw2005,Kording2006}, or the threat of impending pain \citep[e.g.,][]{Voigt1990,Moutoussis2014}.  Particularly in the case of pain stimuli, these manipulations are immediate and consequential to the individual's goals and experiences.  As discussed in the discussion of the Training Experiment (see ~\ref{sect:discussionTrain}), it is highly likely that the uncertainty manipulation failed due to a lack of believability and consequentiality of the experimental prime.  Many well-known psychological effects, such as the placebo effect, or the minimal group paradigm \citep[586]{Liu2009}, and even cognitive dissonance paradigm \citep{Kenworthy2011}, have proven famously difficult to evoke experimentally, particularly in initial stages of paradigm development. \footnote{See, for example, a fascinating interview in which Elliot Aronson reflects on his time working under Leon Festinger at Stanford University during the initial development of experimental paradigms designed to measure cognitive dissonance.  Successfully arousing dissonance required both authentic performance and considerable consequentiality (i.e., monetary incentives; see https://www.psychologicalscience.org/video/elliot-aronson-itps.html)}.

Ethnographic research presented in this thesis suggested that performance-related anxiety and social guilt were experienced primarily through reference to authoritative figures such as the coach or senior players (see Chapter~\ref{sect:relationalPerformance}).  Perhaps framing uncertainty in terms of a relational mode of social cognition may have been necessary to effectively arouse an effect in the case of Chinese rugby players.  This proposal is consistent with cognitive dissonance literature, which suggests that cognitive dissonance is aroused in East Asian contexts only when interpersonal relationships are made salient\citep{Hoshino-Browne2005}.   As as discussed in Chapter~\ref{sect:discussEthnoField}, these considerations raise the issue of social processes confounding physical processes.  For example, if a coach were present at each experimental session of the Training Experiment, how could it be inferred that perceived uncertainty was driven by the the effects of physical movement coordination, and not just by the demand characteristics associated with the social pressure of the coach.  Is it possible to disentangle the dynamical relationship between physical and social processes in response to uncertainty?  Future research must grapple with this question in order to make empirical progress.

%Team click is founded on the principles of self-organisation and non-linearity, and these processes are inherently hard to ``pin down'' via conventional methodological paradigms and linear inferential statistics \citep{Badcock2012}.


\subsection{Team Click}
A general account of team click will of course demand greater attention to its core construct.  In particular,  I identified team click as entailing two dimensions of experience, one pertaining to the visceral recognition of coordination in joint action, and another pertaining to the feelings of social agency through joint action. As formulated in this thesis, team click represents a ``rough and ready'' first attempt at operationalising extremely novel theoretical principles (AIF) and relatively limited anecdotal and ethnographic observations. Closer scrutiny of the dimensionality of team click and the proposed mechanisms that underpin each dimension will enhance the explanatory power of construct.

\subsubsection{Viscerality}
I identified and measured three components of viscerality (tacit understanding (\textit{moqi}), team aura, and flow or coherence in joint action), and proposed that these components could be understood in terms of the way in which uncertainty in joint action bears upon processes of ``Bayesian multisensory integration'' \citep{Ernst2004}.
The role of visceral signals in shaping human cognition and behaviour is being increasingly recognised, particularly in the fields of autonomic psychophysiology, feeding and metabolism, osmoregulation, and psychoneuroimmunology \citep[see:][]{Crtichely2015}.  Recent findings suggesting an extensive bi-directional relationship between the central nervous system and the body's viscera known as ``neurovisceral integration'' \citep[the gut-brain axis; see][]{Carabotti2015} has numerous implications for psycho-physiological health and wellbeing \citep{Porges2009}, and has recently been applied to the domain of social interaction \citep[see:][]{Porges2001,VanKleef2008,Akinola2014}.  Deeper insights into the social function of neurovisceral processes will help inform and develop the contours of viscerality that underpin perceptions of peak team performance.  In particular, studies that specifically investigate the relationship between multisensory integration and neurovisceral processes could help shed light on this dimension of team click.

\subsubsection{Social agency}
I identified social agency in team click as involving perceptions of extension of individual agency by the contribution of others to joint action, and the perception of reliability of self and others in joint action.  I identified  processes of sensory attenuation and second-order precision-weighting as relevant mechanisms to the sense of agency in joint action.  As discussed in Chapter~\ref{sect:visceralAgency}, research detailing the sense of joint agency is scant compared to research pertaining to ascriptions of individual agency.  Emergin research investigating the distinction between I- and we-mode cognition \citep{Gallotti2013,VanderWel2015,Noy2017} will provide a theory of team click with valuable empirical learnings.  To date, however, most research produced on we-mode cognition and dynamical joint action more generally are utilise models that are dyadic in structure, and usually involve a turn-taking sequence \citep{Friston2015,Pesquita2017}.  Group exercise contexts pose a challenge to these models invariably involve multiple agents acting contemporaneously. To fully capture the empirical phenomenon of social agency in team click, testable models of joint action (and agency) involving two or more co-actors engaged in dynamical, co-occuring action are required.

\subsubsection{Precedents and antecedents to team click}
Future research will need to more clearly delineate the construct of team click from its precedents and antecedents.  In this thesis I identified and tested the role of one possible precedent—positive expectation violations concerning team performance—and one potential antecedent----social bonding—--to team click.

\myparagraph{Positive expectation violation}
The preceding role of positive expectation violation more detailed development.  Failure of both field studies to show evidence for a relationship between positive expectation violation and social bonding supports the interpretation that positive expectation violation may be more relevant to team click directly, as a mechanism that directs attention to the visceral and agentic experiences of team click \citep{Chetverikov2016}.  This result was contrasted by the finding in the Tournament Study, that perceptions of team performance did directly explain variance in social bonding, in both Post-Tournament and Pre- to Post-Tournament data sets.  These results suggests that while expectation violation in joint action might be an important factor in generating feelings of team click, it might not be a strong enough mechanism to drive social bonding directly.

However, the reasons for this result may be in part due to study design and issues of measurement.  Team Performance Components was a factor made up of four items, each of which require detailed reflection on the experience of four different aspects of team coordination.  Team Performance Components may have more powerfully tapped into the implicit mechanisms involved in coordinated joint action, encouraging athletes to reflect on coordination with specific co-actors, and as such the opportunity to rehearse and reinforce feelings of trust, reliability, and cooperation \citep{Reddish2013a}.  By contrast, Team Performance Vs Expectations was only a single-item measure, the central tendency for which was well above the mid-point of the scale (i.e., on average athletes reported team performance that was better, rather than worse than expected).

It is also possible that expectation violation in joint action might not be immediately available to athlete perception or reflection.  As \textcite[1277]{VanderWel2012} point out in a discussion of perceptions of shared agency, there is evidence to suggest that sensorimotor prediction errors may be functionally de-coupled from higher-order processes of awareness in action and perception.  In essence, dynamic joint action implicate cognitive processes that exist largely below the surface of conscious awareness, and the relationship between largely unconscious informational transfer rooted in physical movement coordination, and higher order conscious recognition of these sensorimotor referents, is still not clearly understood \citep{Semin2008,Frith2007,Frith2010,Clark2013}.  Thus, the use of a single-item self report of perceptions of performance relative to prior expectations may not have been rich enough to capture the full dimensionality of the experience of positive expectation violation concerning team performance.  The four item factor of Team Performance Components may have more effectively preserved more variance relevant to social bonding.

\myparagraph{Social bonding}
 Team click correlated highly with social bonding in the Post-Tournament, Overall Tournament, and Post-Drill (Training Experiment) data sets (see Table~\ref{tab:clickBondingCorrTable}).  While these correlations were considered high relative to the suggested threshold of .70 for multicolinearity \citep{Field2012}, high correlation between team click and social bonding was expected based on analysis of interview data.  When asked specifically about on-field team click, many athletes immediately ventured into more social territory, and very rarely produced detailed technical explanations for   on-field team performance  athletes automatically drew upon off-field social explanations for on-field team performance.  The high correlations reported in this study between team click and social bonding do, however, raise the issue of how team click can be better isolated as a psychological construct from its precedent and antecedent mechanisms for the purposes of experimental analysis.

\input{images/clickBondingCorrTable}

The hypothesised relationship between team click and social bonding (H1b) also requires more development.  It was hypothesised that feelings of team click would set a sensorimotor foundation that would facilitate—or, more accurately, demand---more elaborate representations of group identification characteristic of social bonding \citep{Pezzulo2014}.  This hypothesis motivated the Prediction 3.b of the Training Experiment that higher levels of perceive team click to the immediate training group would generate higher levels of social bonding to the broader category of the Provincial team.  Results of the Post-Drill analysis supported this prediction. However, analysis of the Pre- to Post-Drill data set did not reveal any signs that increased perceptions of team click predicted increase perceptions of social bonding.  As referenced in Chapter ~\ref{sect:clickBondingAIF}, one previous study has proposed a link between autonomic arousal and identity fusion, whereby short bouts of physical exercise performed as a group lead to higher levels of identification with the national identity of Spain, measured by contributions to a public goods game \citep[824][]{Swann2010}.

In sum, enriching understandings of team click's internal dimensions, and disentangling team click from its precedents and antecedents will be an important project for future experimental research on team click or related concepts.


\section{Implications of team click for applied fields}
Understanding the dynamics of coordination that underwrite social bonding in group exercise can have implications for a wide range of applied contexts, including competitive sport, psychosocial health, education settings, and artificial intelligence.  This thesis sought specifically to develop a novel theoretical account of team click. As such, theoretical formulations and results are both preliminary, and not suitably translated to applied fields.
%this section aims to briefly propose possible implications that the current studies may have beyond strictly theoretical debates.It is important to note, however, that future studies investigating the durability and range of the positive social effects of movement synchrony are essential before firm conclusions about applications can be made.



    %  Enhancing strategies for coordination and performance in the face of uncertainty
    %      Measure dynamical coupling of movement and social communicatio as a dual measure of ``click'' in systems: analysis of click requires both social and physical
    %      Measure affordances
    %      Sport:
    %          Training schedule

    %  Achieving social connection through joint action:
    %      Mental health
    %          Schizophrenia (Friston2015)
    %          Depression
    %          ASD
    %%              The affordances necessary


    \section{Final Remarks}

  This thesis began with the mystery of joint action, as articulated by my friend Adrian.  Since, I have suggested a mechanism through which this mystery can be empirically tested. This thesis progressed in a step-wise manner to formulate, situate, assess, and test claims of a general account of team click in group exercise.  Findings from ethnographic research and field experiments substantiated the core hypothesis that team click mediates a relationship between positive perceptions of team performance and social bonding.  Positive expectation violation also significantly predicted team click, suggesting that the visceral and socially agentic dimensions of team click.  The relationship between uncertainty and coordination in joint action remains an open question, however.

  Theoretical synthesis and empirical results presented in this thesis require ongoing consideration.  In particular, the future research must address both the distinctiveness of team click as a construct, as well as methods through which this construct can be reliably tested.  In this thesis, I demonstrated that testing team click required first considering the construct \textit{in situ}.  Future research will need to deal more directly with the dynamical nature of team click and its underpinnings, in particular how to quantify team click in real-world settings.

  I propose that team click can serve as a complement to existing theories of group exercise and social bonding, by offering a second-order framework within which more fractionated component mechanisms of group exercise can be understood to coordinate.  Beyond team click, dynamical approaches to cognition and evolution are gaining traction owing to their ability to shed light on the complexity of biological life \citep{Ramstead2017,Badcock2012}.  Team click offers an explanation for why when we move together, we do not  always bond together.  This thesis substantiates the claim that we bond together when the team clicks.
