
\usepackage[english]{babel}
\usepackage[backend=bibtex,style=apa,natbib=true]{biblatex}

\addbibresource{referencesBibtex.bib}


\begin{document}





  3. Theoretical justification for the links between joint-action, team click, and social bonding.


How is it possible to account for a link between the highly exertive and interactive collective movement that defines rugby sevens, and the feeling of ``team click'' that appears to arise in athletes following perceptions of successful coordination? In turn, how do these processes relate to social bonding and group formation?  In this section I review existing literature concerning the cognitive science of movement and prevailing theories of action, joint-action, and (situated) social cognition.  I then consider existing hypotheses that connect mechanisms of joint-action and social bonding.  Together, these components form the theoretical basis of a link between collective movement dynamics and proximate mechanisms of social cohesion.


    3.1 Joint-Action

Physical movement is central to the adaptive success of biological life, particularly life for which movement can intentionally directed in order to bring about change in the environment.  Within cognitive science and psychology, the term action is used to distinguish intentional, agent-directed movement from other forms of movement \citep{Davidson1980}.  Humans display a rich capacity for complex forms of autonomic and ostensive physical movement, which is employed in the coordination of behaviours with the environment and conspecifics.  Indeed, the ultra-social nature of the human species dictates that almost all action is, in some sense, ``inter-action'', in the sense that any human movement in a social niche can potentially function either as an ostensive signal or inadvertent cue in interpersonal communication \citep{Danchin2004}. Specifically, ``Joint-action'' can be regarded as any form of social interaction whereby two or more individuals coordinate their actions in space and time to bring about a change in the environment \citep{Sebanz2006a}.  In an ethnographic environment where athletes train, eat, and sleep together, and where on-field performance is directly contingent on effective coordination of behaviours with teammates, joint-action provides the most appropriate theoretical starting point for analysis of the collective movement of professional rugby sevens athletes in China.

The ability to coordinate behaviour with one or more other individuals in order to bring about a change in the environment (joint-action) depends on the the extent to which action-oriented representations are shared between between actors and calibrated to the task-specific environment.  Thus, Sebanz and colleagues propose that successful joint-action depends on the abilities (1) to share representations, (2) to predict actions, and (3) to flexibly integrate predicted effects of own and others’ actions \citep{Sebanz2006}.

Extensive research within developmental and comparative social psychology has established key component mechanisms that enable joint-action. The basal human capacity to maintain joint attention to a common object (which starts to emerge by 9-12 months in infants, \cite{CarpenterMalinda;Call2013}) underwrites higher order and elaborate process of sharing attention around mental representations and articulating explicit common goals (O'Madagain & Tomasello 2017).  Evidence suggests that humans are particularly adept, even from a very young age, at attending to biological motion (Scholl & Tremoulet 2000), especially when it carries socially-relevant information (Kozlowski & Cutting 1977; Dittrich et al. 1996).  Whether it is a physical piece of furniture that is being moved, or an abstract mental object that is being collectively ``moved'', so to speak, through a spoken conversation — joint-action demands the ability to coordinate attention around a common object.  Joint attention appears to set the foundation for higher level social cognitive processes of mentalising and reading others' intentions (Frith & Frith 2010).  Experimental evidence suggests that joint attention, more so than a shared goal, generates higher levels of liking and social bonding in a basic dyadic problem solving task (Wolf et al. 2016).

In addition to maintaining shared attention, the ability to predict and emulate the behaviours of others is crucial for the performance of successful joint-action. While the precise mechanisms remain unclear, it appears that humans have evolved some capacity to activate predictions about the behaviour of others, possibly recruiting a neural assemblage known as the ``mirror neuron system'' \citep{Rizzolati2004}. Watching someone engage in an action is hypothesised to activate the actor’s own motor representation of that action, suggesting a common neural basis for motor control and action understanding. The implication of this theory is that coordination observed between agents results because agents execute similar but independent motor programs anchored in shared representations co-activated by the mirror neuron system.

Supression Studies


Effectiveness of joint-action appears to rely on each individual possessing a strong interoceptive action-oriented representation (memory) of the joint-action (Moreau et al. 2016).

Previous research cited also found an effect of training in the motor domain on performance in the cognitive domain, on a visual perspective-taking task [7].




3. Joint-actions, particularly those joint-actions that involve complex sequences and divisions of labour between participants, rely heavily on capacities to explicitly signal intention for the assigning of roles, forward planning, and repair of failed coordination (Frith & Frith 2010: ).  These complex behaviours are interlinked with the capacity for a formal theory of mind and language (Tomasello )



Self-other blurring (synchrony, Launauye etc)

Self-other distinction? ()
increasing other-related activations, rather than the control of self and other representations as such.

The literature on agency, the experience of controlling one’s actions, indeed suggests that there may be unique mechanisms that serve to distinguish the (motor) actions of self and other [9]

co-representation - a more cognitive interoceptive process rather than just direct perception-action links only.
(However, this may not be the whole story, as co-representation occurs when a partner is absent but believed to be acting in another room [14] and when the part- ner’s actions are hidden from view ([12], experiments 1 and 2),)



Robust complex system requires both individual and group (celebration of individual/self in China)











This description of component mechanisms of joint-action is useful but ultimately unsatisfactory given the mounting evidence that componential mechanisms of behaviour are constrained by complex systems dynamics.

Traditional individual-centric computational models of cognition, which tend to render movement as the product of a linear sequence of sensory perception, internal mental representation, and action selection (see for example, SOURCE), have been superseded by models of cognition in which perception, representation, emotion, and action are functionally and temporally integrated in the service of informational processes situated in—but also distributed throughout—human brains, conscious minds, active bodies, and feature-rich physical environments (SOURCE). As such, individual centred developments in cognitive neuroscience and psychology have been supplemented with a social framework that allows to encompass the processes involved from joint perception to joint action and its temporally distributed regularities.


    The theoretical approach to joint-action and social cohesion that I develop in this dissertation relies on upon five interrelated hypotheses concerning human social cognition:

    1. Human cognition is generated by a multitude of cognitive mechanisms responsible for regulating movement, evidence for which can be identified in a spectrum of social behaviours ranging from more explicit and deliberate communicative signals, through to more implicit (often pre-perceptual) processes of synchronisation and alignment of the organism with the environment (Frith & Frith 2010, Semin & Cacciopo 2008). Prevailing models of social cognition reveal that higher order explicit communication strategies (consciousness, symbolic language, rational thinking), traditionally privileged in computational accounts of cognition, draw on only a small fraction of the total amount of information processed by the nervous system at any one moment (reportedly in the realm of less than 2\%!, see SOURCES).


    2. Human cognition is constantly engaged in ``active inference'' (SOURCE) about the world, by generating interoceptive predictions about the state of the world, and testing these neurophysiological representations against actual bottom-up sensory evidence (SOURCES).  In this conception, known as the predictive processing paradigm (Clark 2013), the fundamental drive of human cognition is to minimise ``prediction error,'' or the discrepancy between predictions of the world and the bottom-up sensorial experience of the world.  Representation, emotion, and action are unified by a deep logic of prediction-error management, where action and perception work together to reduce align the organism with its expectations.  It has been shown that, and in light of hypothesis 1, much of the work of predictive processing occurs beneath the level of conscious awareness (Frith 2007, Clark 2013). It is also plausible that predictive processing is driven by a more fundamental mandate of reducing an information-theoretic isomorph of thermodynamic ``free energy'' in the nervous system's interaction with the environment (Friston 2010).


    3. Human cognition relies on resources that are distributed across neural, bodily, and environmental features (e.g., Hutchins, 1995; Kirsch, 1995; Brooks, 1999; Agre, 1997) with the social and physical environment supporting social action and interaction (Smith & Semin, 2004).  The embedded and distributed nature of cognition is observable in instances of joint-action, where the time-locked coordination of behaviour with two or more individuals in a physical environment gives rise functional interpersonal synergies.  These coordinative structures reduce the componential degrees of freedom of the system components and allow for cognitive resources to be efficiently distributed across all nodes of the emergent dynamical system (Semin & Cacciopo 2008).  The self-organising properties of dynamic coupling in joint-action have been modelled, revealing evidence of multi-scale synchronisation and complexity matching (Schmidt, Richardson).


    4. The quality of synchronisation in joint-action scenarios has implications for individual psychophysiological function, health, and subjective well-being (Wheatley et al. 2012).  In particular, successful coordination in joint-action has implications for psychological processes relating to social connectedness (Marsh et al. 2009).  Evidence from the self-report of highly skilled joint-action practitioners (musicians, performers, athletes) suggests a strong link between the perception of high-quality joint-action, and psychological wellbeing (Csikszentmihalyi 1990, 1995).  Conversely, the psycho-social isolation and ill-health experienced by individuals who suffer from developmental and neurocognitive deficits in behaviours key to dynamic interpersonal interaction is also well-documented (ASD: Frith, Baron-Cohen).

    5. The coupling dynamics of co-present joint-action have implications for higher order and more complex goal oriented settings, including the (re)production and transmission of shared cultural practices (Roepstroff, Sperber (CAT)).  (EXPLAIN)


EXPLAIN hypotheses:





Being psychologically distanced from another individual can inhibit the emergence of interpersonal synergies (Miles et al., 2010). Interpersonal syner- gies facilitate memory for those with whom we interact (Miles et al., 2010) and can more generally facilitate performance of social cognitive or linguistic tasks (Richardson and Dale, 2005; Shockley et al., 2009). Synergies might also be a mechanism for interpersonal coordination in team sports (e.g., Passos et al., 2009).









3.1.1 Human cognition is generated by a multitude of cognitive mechanisms responsible for regulating movement.

Considered from the predictive processing perspective, a key proximal goal of joint-action is the reduction of \textit{mutual} prediction error (Clark 2013).


- SC Model:
social cognition is best understood as grounded in bodily experience and intertwined with a wealth of interpersonal interaction and specialized for a distinctive class of stimuli.

- The Social Brain (Frith & Frith 2010):
Automatic synchronisation
In what psychologist Chris Frith calls ``closing the loop''









3.1.2 Human cognition is constantly engaged in ``active inference'' (SOURCE) about the world

The emergence of the predictive processing paradigm (also known as the predictive coding paradigm)
    The predictive processing paradigm is an attempt to model the way that the brain processes information to produce perception and action. The predictive processing hypothesis depicts a brain that is constantly in the business of predicting probable worldly source of sensory signals, and adjusting action schemas in order to minimise the error of these predictions (or maximise the precision of these models). Predictive coding first emerged in cognitive science as an explanation for ``black box'' problem of perception (SOURCE), but has since been extended to account for the role of action in cognition (SOURCES). The human brain confronts a black box problem when attempting to infer the causes in the external world with only partial and indirect access to that world via sensory input signals (SOURCE). This blindness to the precise nature of the world means that the most effective strategy available to the brain is to begin by generating an internal model of the world (no matter how rudimentary it may be initially) and then iteratively update that model in response to ``errors'' that arise from discrepancy between the model and the information from sensory inputs (SOURCE).

    There is mounting neurocomputational evidence to suggest that this inferential strategy is supported in the brain by multilevel and hierarchical arrangement of cortical structures, which enable bi-directional cascades of information between levels.  Higher levels of the cortical hierarchy formulate models based on prior experience, which are employed to ``explain away'' sensory signals at lower levels. Lower level signals unaccounted for by higher level predictions are incorporated into higher level structures and strengthen the robustness of the model.  Predictive processing has been likened to a process of ``empirical Bayes'' (Robbins 1956), whereby prediction errors function to strengthen prior probability distributions of models for future inference.  In a typically Bayesian manner, predictive processes appear capable of factoring representations of uncertainty around sensory signals into the predictive model itself (SOURCE).  The brain's capacity to quantify the uncertainty of any given sensory state facilitates optimal selection between competing predictions pertaining to the same bottom-up sensory signals, judged probabilistically.  At any one moment, an individual has access to multiple hypotheses derived from priors (and hyper-priors), which compete for the best fit of the sensation, until that process leads to fixation on the best hypothesis for the sensory state. Binocular rivalry, (for example looking at a necker cube) is an instance in which there is insufficient sensory information available in order to reach fixation on one model over another (see FIGURE) (SOURCE - Clark 2013, Frith 2007).

    Importantly, in the case of motor systems, the agent of the system is able to move its sensors in ways that amount to actively seeking or generating the sensory consequences that they (or rather, their brains) expect.  In this way, ``error signals self-suppress, not through neuronally mediated effects, but by eliciting movements that change bottom-up proprioceptive and sensory input'' (Clark 2013: 186). In this way, perception, representation, and action functionally and temporally integrate to fulfil an ever evolving set of sub-personal expectations about the state of the world.

    The predictive processing paradigm is also underwritten by a plausible neurobiological account. Specifically, the complex cortical processes of ``error management'' appear to be mediated by the activity of the dopaminergic system (SOURCE), while subcortical neuromodulatory systems, such as those responsible for producing norepinephrine, acetylcholine, and endogenous opioids, appear to be responsible for attuning cortical processing to signals from the body and environment that are important for survival (Lewis and Todd 2005).  There is now evidence to suggest that complex cognitive processes traditionally understood to be confined to cortical regions, and subcortical neuromodulatory systems traditionally associated with emotion and thus exogenous to inferential processes, work in a loop of reciprocal interaction in order to enhance processes of error management (Damasio 1994, Miller & Clark 2017).  By collapsing the common neurocognitive distinction between cortical and subcortical processes, emotions can be understood as superordinate programs for regulating disparate subordinate cognitive modules for the purposes of global coordination with the environment \citep{Cosmides2000}.  Thus, in addition to unifying perception, representation, and movement in a common theoretical framework, the predictive processing paradigm also helps integrate the role of emotion into the brain's processes of ``active inference.''


          3.1.2.1 What is the larger mandate?

          In so far as I have rendered it, this formulation of the brain’s inferential strategy of prediction error minimisation seems to pursue a rather narrowly neurocentric focus, albeit one that reveals intimate links between perception, representations, emotion, and action.  For the purposes of analysing complex multi-modal and multi-agent joint-actions, however, it is important to understand how human cognition is situated within larger ecological systems.  It has been suggested, for example, that the brain's overarching cognitive principle of error minimisation should be understood as a consequence of a more fundamental mandate to minimise an information-theoretic isomorph of thermo-dynamic free energy in a system’s exchanges with the environment (Friston 2010).  Thermodynamic free energy is a measure of the energy available to a system to do useful work (SOURCE), i.e. work that contributes to the maintenance of system structure and organisation.  Transposed to the cognitive domain, free energy emerges as the difference between the way the world is \textit{represented} as being, and the way it actually \textit{is}. Thus, the better the model fit, the lower the information-theoretic free energy, or in other words, more of the system’s resources are being put to effective work in representing the world.  In this formulation, ``surprisal'' (deliberately distinguished from surprise by Tribus in 1961) is a measure of sub-personally computed implausibility of some sensory state given a model of the world (Clark 2013).  Entropy, in turn, can be understood as the long-term average of surprisal, or the informational cost for representing the sensory input by its internal model (Little and Sommer 2013 (reply)).  Thus, good models of the world reduce surprisal and therefore achieve lower entropy.



      3.1.3 Human cognition relies on resources that are distributed across neural, bodily, and environmental features

          Explaining ``active inference'' within a larger framework of free-energy minimisation in a system’s exchanges with the environment is in line with a broader recognition of the ways in which complex biological and information-theoretic systems both support and constrain processes of human social cognition (Dale et al. 2014).  The human brain, body, and the various functional behavioural synergies that humans form in social interaction, are all examples of dynamical systems of varying scales, whose physiological and cognitive parameters enforce constraints on the expression of their componential mechanisms (Coey et al. 2012).  Thus, when analysing cognitive mechanisms of joint-action, it is important also to understand the ways in which these actions establish, and are constrained by, dynamics that emerge from the time-locked and multimodal interaction between components of a system.

          Understanding multimodal coordination of behaviour between two or more individuals requires an understanding of the ways in which cognition unfolds in a complex system of which brains, bodies, and physical features are all active nodes.



%In addition, understanding the characteristics of complex systems, for example the balance between flexibility and rigidity, ``robustness'' - helps improve understanding of what constitutes an optimal state of joint-action... and the extent to which individuals are capable of perceiving successful joint-action.

%Caveat: its a messy, not a neat picture.  Acknowledge the alternative interpretations / hypotehes

%[Footnote:] Solving the dark-room dilemma: all the key information-theoretic quantities are defined and computed relative to a type of agent – a specific kind of creature whose morphology, nervous system, and neural economy already render it (but only in the specific sense stressed by Friston; more on this shortly) a complex model of its own adaptive niche.
%the creature, simply because it is the creature that it is, already embodies a complex set of “expectations” concerning moving, eating, playing, exploring, and so forth. It is because surprisal at the very largest scale is minimized against the backdrop of this complex set of creature-defining “expectations” that we need fear neither darkened nor musical (nor meta-musical, nor meta-meta-musical) rooms.

%- The Dark Room Dilemma solved by a prediction/complexity ``sweet spot''? Robustness of complex system: structure and flexibility - Competing / alternative hypotheses for universal cognitive architecture - neat vs motley solutions




        3.1.3.1 Extra-neural mechanisms for joint-action:


          Beyond these component cognitive mechanisms of joint-action, mutual prediction error in joint-action can be minimised via various extra-neural mechanisms available to brains situated in multimodal and dynamically unfolding interactive systems.  When two or more humans engage in time-locked interaction in an environment, a functional ``synergy'' is established.  A synergy is a temporarily assembled, task-specific, functional coupling between a system’s componential degrees-of-freedom (see Kelso 2009).  The information-theoretic benefit of synergetic structures is that they greatly simplify the problems of motor control. When coordinated to behave as a functional unit, the individual degrees-of-freedom do not need to be controlled independently of one another, and perturbations applied to a component are automatically compensated by the coupled components (Riley et al. 2011). These coordinative structures allow for cognitive resources to be efficiently distributed across neural, bodily and environmental features of a dynamical system (Semin & Cacciopo 2008).  Establishing interpersonal synergies with one or more individuals serves the adaptive purpose of maximising extraction of fitness-relevant information from an informationally complex or \textit{uncertain} (noisy) social environment.


%Central to this (dynamical) account of human behavior is the concept of synergies or coordinative structures (e.g., Bernstein, 1967; Turvey et al., 1978; Kelso, 1984)

          EXAMPLE:
          For example, imagine your office colleague were to suddenly throw a soft rubber ``stress ball'' in your direction without announcement. An automatic response would be to move your hand or hands into a position to catch the ball.  This response should be most accurately understood not as a discrete, volitional action in response to a stimulus external to your individual cognition. Rather, attempting to catch the ball is the most efficient way of reducing uncertainty (or competing hypotheses) for the bottom-up sensory signals available.  In essence, catching the ball communicates an agreement to fixate on the hypothesis that has been provided by your office colleague who initiated the joint-action.  This example captures the essence of being ``pulled in to the orbit of JA'' (Marsh et al. 2009)

          in order to catch or otherwise interact with the in-flight ball can be more accurately understood as participation in a distributed cognitive system containing the ball, the office, your office colleague, as well as the higher-order mental representations that acknowledge the stress ball as an object to be caught (an not avoided, in the case that scissors were thrown instead!). The function of individual cognition, in this conception, is to facilitate physical movement in such a way as to reduce the uncertainty of the overall system-distributed cognition.  Attempting to catch the ball is, in this specific instance, an adaptive response to the affordances and constraints of the cognitive system in which you find yourself.



 3.1.4 Implications of successful joint-action for psychophysiological function (health, subjective wellbeing)


3.1.5 Scaling up to culture: hyper-priors etc Patterned Practices (Roepstroff 2010)





















3.3 Team Click? - the perception of high quality joint-action?


- There is an extensive literature on the phenomenology of optimal human performance, usually involving complex patterns of movement. ``Flow"  Music, sport, creative arts, but also reported in reading, cooking, knitting, etc.
- This literature also includes the notion of ``group flow" - or alternatively described by team sport athletes as the moment when things ``click'' (Novak)
- Flow is reported (always self-reported, after the fact) as an autopoetic, pleasurable and visceral sensation of successful intra- or inter-personal coordination of movement.  Often, athletes or other practitioners will report feelings of loss of personal agency, and a slowing down or speeding up of time.

- Mechanisms that account for flow experience? Dietrich etc



Considering the dynamical systems properties of joint-action, optimal state of joint-action may resemble a statistical sweet spot between error minimisation and complexity of social interaction aligned within a functional, action-oriented synergy.




Psychological consequences of joint-action:
Flow:
expectation violation: positive violation?


neural error reduction (individual) (Neural social resonance: synchrony etc)

extra-neural error reduction: achieving flow with others, robust stable system

Function of emotion?

Role of exercise in stimulating neuromodulatory systems?






    3.4 Team Click -> Social Bonding?



\end{document}
