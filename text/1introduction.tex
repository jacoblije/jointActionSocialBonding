'
\begin{savequote}[8cm]

  It takes two to know one.

  \qauthor{--- Gregory Bateson}

\end{savequote}









\chapter{\label{chap:intro}Introduction}



\minitoc





                                          \begin{CJK}{UTF8}{gbsn}

\section{The carnal mystery of group exercise \label{sect:adrian}}

Adrian, Kai, and I waited for Mr Shi to arrive in the upstairs area of the Korean BBQ restaurant in a quiet willow lined street just inside Beijing's East 4th Ring Road.  Adrian, the host of the dinner, is a veritable elder of Chinese rugby.  He had been captain of the second class of rugby players to graduate from the Chinese Agricultural University (CAU), the home of China's first official rugby union program, established in 1990.

I first met Adrian two years earlier through Kai. Kai---a close friend of mine of many years---was a more recent graduate of CAU (2007), a former Chinese National rugby team representative, and since graduating from CAU, a lawyer in Beijing.  Mr Shi, the guest of honour for whom the three of us were waiting, was a technical producer for Chinese Central Television's Sport Channel, ``CCTV5.''

The backstory was that CCTV5 needed help producing the commentary for the Rugby World Cup, which they were planning to broadcast for the first time in October 2015.  Mr Shi reached out through his network of relationships in Beijing and soon tracked down Adrian; Adrian tracked down Kai; Kai, in turn, tracked down me.  I was eager to catch up with old friends as well as begin my fieldwork, and so, despite my jet lag, I accepted the invitation. Early that first Saturday evening I set off to the Korean BBQ restaurant with my notebook and audio recorder in hand.

Adrian naturally held the floor in conversation while the three of us waited for Mr Shi to arrive.



He reminisced fondly about his time playing rugby at CAU, as well as his time after graduation playing with the Beijing Devils, a rugby club in Beijing whose members were predominantly expats.  He assured us that rugby in China was, in those days, fun and free-spirited.  Not like today, now that Chinese rugby has become a professional program in the state sponsored sport system, (owing to its Olympic status in the modified form of the game, rugby sevens, see Chapter ~\ref{chap:researchSetting}).  Adrian talked about going on a rugby tour to the UK with the Beijing Devils :  ``Everyone only just scraped together the money to go on tour, we all payed our own way, sometimes you'd get a bit of help from someone or whatever. We did it because we loved the game, not for any other reason,'' he insisted.  Kai and I listened intently.  All of a sudden I realised that this conversation could be relevant, so I started taking notes.

When Mr Shi finally arrived, Adrian continued the nostalgic story telling mode, but naturally shifted his target audience from Kai and me to Mr Shi.  Adrian began to describe in rich detail the experience of camaraderie between he and his Beijing Devils team mates when they participated in overseas rugby tour. But he then interrupted his own story to make an explanatory aside directed at Mr Shi, accommodating for the fact that Mr Shi was relatively unacquainted with the sport: ``This sport, rugby: it's actually very mysterious. If you haven't played it yourself you might not know this type of feeling,'' Adrian respectfully suggested to Mr Shi.  ``Because rugby, you know, you're all on the field together, there's body contact...'' he paused to find the right phrasing,  ``...its a very \textit{carnal} type of feeling.''  His attempts to enrich his communication by gesticulating had led him to have both of his hands clenched as fists in front of him like they were cradling a rugby ball or gripping the steering wheel of a car---a lit cigarette smouldering between the index and middle finger of his right hand.  Adrian concluded by repeating: ``Its very mysterious.'' He shook his head as if baffled and finally released his clenched fists to dab the ash from his cigarette into the ashtray in front of him.  After taking another drag from his cigarette he finally added: ``So it means this rugby circle here in China is very tight...''---a short pause for another dab of his cigarette--- ``...but it doesn't mean that this circle is not also not also complete chaos!''
  \footnote{Circle (\textit{quanzi} 圈子) is a common colloquial way to refer to a social group or community of people in modern standard Chinese.}
The wisdom of Adrians's final punchline was confirmed with a knowing chuckle from all of us, including Mr Shi. Adrian concluded his performance silently, by taking a long, satisfying drag of his cigarette.

%英式橄榄球这个项目其实特别神秘,没玩过的话您可能不知道这种感觉,因为英式橄榄球么,大家在场上有身体接触,是一种``肉''的感觉,大家互相都特别亲,特别神秘。
%所以在橄榄球这个圈子特别亲, 但这不是说这个圈儿也不乱!

I was captivated---but also somewhat surprised---by Adrian's monologue.  I was not expecting, so early into my fieldwork, to happen upon a declaration in which a link between the carnal (\textit{rou} 肉) sensations associated with on-field joint action, and social processes of interpersonal emotional affiliation (\textit{qin} 亲) and group cohesion of the rugby community (\textit{quanzi} 圈子) was so explicitly and spontaneously emphasised.  It was clear that rugby's visceral dimension continued to capture Adrian emotionally; some fifteen years after he had finished playing his fists still clenched with energy, and his head still shook with amazement.

I was also intrigued that Adrian cited the source of his emotional capture as at once both very specific (derived from playing together with others on the field) and, at the same time, ultimately ``mysterious.''  The aim of my research, which by that time I had broadly formulated, was to contribute to an explanation the human behavioural phenomenon of group exercise in terms of its social, evolutionary, cognitive, and physiological causal processes and dynamics.  In essence, the aim was to somehow move from mystery to scientific mechanism.  At this first dinner in Beijing, Adrian's comments both captured the phenomenological mystery of group exercise, and pointed me in the direction of the underlying scientific mechanisms.

Needless to say, I left that first dinner eager to investigate the sources of Adrian's experience of mysterious carnality and social connection in rugby's joint action.  My next stop, on Monday morning, was the Temple of God of Agriculture Sports Institute, where I had organised to conduct ethnographic research with the Beijing Provincial rugby program.


                            \begin{center}
                              * * *
                            \end{center}





\section{Scientific explanations of group exercise}
Competitive team sports, whirling Sufi dervishes, late-night electronic music raves, Maasi ceremonial dances, or the fitness cults of Cross-fit and Soul Cycle---endless examples can be plucked from across cultures and throughout time to exemplify the human compulsion to come together and move together.  How can we explain the prevalence of these activities in the human record?  In this dissertation I contribute to a scientific understanding of ``group exercise''---defined herein as physiologically exertive and socially coordinated movement---by way of a focussed study of the social cognition of joint action among professional Chinese rugby players.

Because physical movement is a metabolically expensive task for all biological organisms, it is justifiable from an evolutionary standpoint only if the benefits somehow outweigh the costs.  Using this basic calculus, it is easy to imagine how group exercise would have served important survival functions in our ancestral past.  Activities involving group exercise such as hunting, travel, communication, and defence all appear to confer immediate and obvious benefits to individuals and groups \citep{Sands2010}.

In more recent domains of human history, however, the task of explaining the persistent recurrence of group exercise is more complicated.  At least since the late Pleistocene era (approx. 500ka), and particularly since the Holocene transition from hunter-gatherer to agricultural (approx. 11ka), and later industrial and post-industrial societies, group exercise can be identified in shared cultural practices as varied as religion, organised warfare, music, dance, play, and sport.  Unlike group hunting or defence, however, the fitness-relevant benefits of group exercise in cultural practices such as sport, music, or dance are not always as obvious.  On the contrary, many of these activities appear on the face of things to entail extreme time and energy costs for very little immediate reward.

The prevalence of group exercise in a diverse array of shared cultural practices in the more recent human record thus presents an evolutionary puzzle.  A solution to this puzzle requires a more nuanced calculus that incorporates an appreciation of humans' species-unique evolutionary trajectory, defined by increasingly complex cognitive and cultural capacities, including technical manipulation of extra-somatic materials and ecologies; advanced theory of mind; and information-rich, malleable, and scalable communication systems \citep{Roepstorff2010,Clark2015,Fuentes2016}.  A theory capable of satisfactorily explaining group exercise within humans' distinctive evolutionary parameters is yet to be fully formulated \citep{Cohen2017}.

In this dissertation, I advance existing cognitive and evolutionary understandings of group exercise by formulating and testing a novel theoretical relationship between joint action and social bonding.  Existing theories of group exercise and social bonding fail to satisfactorily account for the variation in, and complexity of, interpersonal movement coordination---also known as ``joint action''---common to many real world settings of human behaviour.  I concentrate on the cognitive mechanisms and system dynamics that enable and constrain joint action in group exercise, and draw attention to the social and psychological effects known to occur when joint action functions successfully---i.e., when joint action ``clicks'' between co-actors.  I propose the phenomenon of ``team click'' as a candidate construct that can help explain a hitherto under-examined link between joint action and social bonding.

I test this novel theory of joint action and social bonding through a series of three empirical studies with professional Chinese rugby players. These studies include 1) an ethnographic study of the Beijing men's rugby team (n = 26), 3) an \textit{in situ} survey study of athletes during a National Rugby Tournament (n = 174), and 4) a controlled field experiment with a sample of athletes across two provincial programs (Beijing and Shandong, n = 58). In each of these studies I find evidence in support of the main predictions of this dissertation.  Invariably, more positive perceptions of joint action predict higher levels of perceived team click; higher levels of team click predict higher levels of social bonding, and in some instances, the construct of team click mediates a direct positive relationship between perceptions of joint action and social bonding.  These findings offer initial substantiation of a novel theory of joint action and social bonding in group exercise.

In this chapter, I review existing theories of group exercise and social cohesion (Section ~\ref{sect:existingGESoCo}), and point to empirical knowledge gaps that require attention (Section ~\ref{sect:empKnowGaps}). In particular, I draw attention to the under-theorised relationship between successful performance of complex joint action and perceptions of ``team click.''  I then preview the main components of this dissertation (Section ~\ref{sect:components}), and conclude with an outline of the chapters of the dissertation (Section ~\ref{sect:chapters}).

\subsection{Existing theories of group exercise and social cohesion\label{sect:existingGESoCo}}
Social scientists and anthropologists \citep[see, for example][]{Durkheim1965} that cultural activities featuring group exercise foster social cohesion \citep{Dunbar2010,Whitehouse2004}.  Social cohesion implies proximity, coordination, and stability of relationships between members of a group, which serve some benefit to the group as a whole \citep{Taylor2018}.  In so far as more cohesive groups face better survival odds than those which are less so, activities that foster social cohesion can be considered collectively advantageous and adaptive \citep{Dunbar2010}.  It is not yet clear, however, precisely how or whether group exercise uniquely generates social cohesion, or in what ways particular mechanisms vary by activity and cultural context.

\subsubsection{The social high theory of group exercise and social bonding \label{sect:socialHigh}}
Recently, researchers interested in the cognitive and evolutionary significance of group exercise and related phenomena such as costly ritual behaviour have (perhaps unwittingly) made a scholarly ceremony out of invoking one particular passage from Durkheim (1965, pg. 217).  ``Once the individuals are gathered together,'' reads the beginning of the passage, ``a sort of electricity is generated from their closeness and quickly launches them to an extraordinary height of exaltation...'' \citep[see ][]{McNeill1995,Konvalinka2011,Fischer2014,Mogan2017}. This passage lends itself neatly to empirical studies of the immediate causal mechanisms of the ``collective effervescence'' of group exercise.

Indeed, existing research, which has focussed predominantly on detailing the proximate physiological, cognitive, and social mechanisms associated with group exercise, suggests that group exercise is responsible for generating a psychophysiological environment conducive to social bonding.  Anthropologist Emma Cohen and colleagues have recently identified links between the two essential ingredients of group exercise---1) physiological exertion and 2) interpersonal movement coordination---and their common psychophysiological effects, including increased pain tolerance, athletic performance, positive affect, wellbeing, pro-sociality, and cooperation \citep{Davis2015}. This evidence amounts to what can be called the ``social high'' theory of group exercise and social bonding \citep[hereafter ``the social high theory,'' see][]{Cohen2017}. Here, social bonding is understood as the psychological experience of increased social closeness \citep{Tarr2014}.  I outline the key principles of the social high theory below.

\myparagraph{Physiological exertion}
Group exercise necessarily entails rigorous physiological exertion.
The health and wellbeing benefits associated with regular physical exercise---including reduced risk of cardiovascular disease, autonomic dysfunction, and early mortality; as well as enhanced neurogenesis, cognitive ability, and mood---are becoming increasingly well-known \citep{Blair1994,Nagamatsu2014}. Evidence suggests that strenuous and prolonged physical exercise is modulated by the same neuropharmacological systems responsible for regulating pain, fatigue, and reward \citep{Boecker2008,Raichlen2013}.  Neurobiological rewards in exercise are associated with both central effects (improved affect, sense of well-being, anxiety reduction, post-exercise calm) and peripheral effects (analgesia), and appear to be dependent for their activation on exercise type, intensity, and duration \citep{Dietrich2004}.  Exercise-specific activity of neurobiological reward systems offers a plausible explanation for commonly reported sensations of positive affect, anxiety reduction, and improved subjective well-being during and following exercise---extremes of which are popularly referred to as the ``runner's high'' \citep{Dietrich2004,Boecker2008,Raichlen2012}.  This neurobiological evidence maps on to more extensive literature concerning the psychological effects of exercise, which indicates a duration and intensity ``sweet spot'' for exercise and positive affect, whereby moderate intensity exercise for durations of $\sim45$ minutes appears most optimal \citep{Reed2006}.

It is possible that the function of exercise-induced positive affect extends to the realm of social bonding, particularly when achieved in group exercise contexts \citep{Cohen2009,Machin2011}.  Endocannabinoids and opioids have been implicated in mammalian social bonding \citep{Fattore2010,Keverne1989}, and in humans specifically, there is evidence that endorphins (a particular class of endogenous opioids) mediate social bonding \citep{Dunbar2012,Shultz2010}.

\myparagraph{Interpersonal movement coordination}
Meanwhile, experimental evidence (predominantly from social psychology) suggests that time-locked coordination of behaviour between two or more individuals is conducive to psychological processes of self-other merging, liking, trust, and psychological affiliation.  In these contexts, interpersonal coordination is primarily operationalised as behavioural synchrony---i.e., stable time- and phase-locked movement of two or more independent components (limbs, bodies, fingers, etc.) \citep{Pikovsky2007}. Researchers suggest that synchrony enables a tight attentional union between individuals who match the timing and content of their actions, leading to the enhancement of interpersonal similarity and the blurring of self-other boundaries in cognitive processing and recall \citep{Cohen2017}.  Relative to non-synchronous group activities, synchrony increases social bonding and pro-social behaviour---an evolutionarily important outcome of bonded relationships \citep{Reddish2013,Reddish2013a,Wiltermuth2009}.  Recent studies have also found that, compared to solo and non-synchronous group exercise, synchronous group exercise leads to significantly greater post-workout pain threshold \citep{Cohen2009,Sullivan2014,Sullivan2013a, Sullivan2013b}.

A recent meta analysis of the behavioural synchrony literature in social psychology suggests three candidate mediators of the relationship between behavioural synchrony and social bonding: 1) lower cognitive affective mechanisms implicating neuropharmacological reward systems (e.g., opioidergic and dopaminergic systems), 2) neurocognitive action-perception networks responsible for the experience of self-other merging, and 3) processes of group-centred cognition responsible for perception and reinforcement of cooperation \citep{Mogan2017}.  The current balance of existing evidence suggests that affective physiological mechanisms may be more relevant to joint action involving larger group sizes in which generalised feelings of euphoria and pro-sociality are common \citep[e.g., mass religious rituals or music festivals, see][]{Weinstein2016}, whereas neurocognitive mechanisms linking joint action and social bonding may be more applicable to smaller group sizes in which individuals can share intentions through ostensive communicative signals and implicit movement regulation cues \citep{Semin2008,Frith2010}.  Studies linking synchrony with social bonding and cooperation are supported by a literature than connects nonconscious mimicry with liking and affiliation \citep{VanBaaren2009}.

\myparagraph{The social high $=$ synchrony $\times$ exertion}
In addition to recorded independent effects of exertion synchrony, there is also some preliminary evidence to suggest that exertion and coordination in group exercise interact to produce additive social effects.  Social features of the exercise environment (for example, perceived social support, level and quality of behavioural synchrony, etc.) modulate exercise-induced mechanisms of pain and reward \citep{Cohen2009,Sullivan2014,Tarr2015,Davis2015,Weinstein2016}. This work is bolstered by existing literature on the social modulation of pain \citep{Eisenberger2012a} and links between pain and prosociality \citep{Bastian2014a}.  The social high theory thus combines these two bodies of literature to tell a story in which positive affect---associated with neuropharmacologically-mediated pain analgesia and reward—--is extended to the social group via synchrony-activated cognitive mechanisms of self-other merging, and the perception and reinforcement of in-group cooperation.

\section{Empirical knowledge gaps in the relationship between group exercise and social cohesion\label{sect:empKnowGaps}}
Only a cursory survey of the many group exercise contexts identifiable in human sociality is needed to reveal that many group exercise scenarios deviate markedly from both the narrow profile and the subjective experiences of group exercise set out by the prevailing social high theory (and for which Durkheim's oft-cited passage is often enlisted to support). In this section, I draw attention to variation in types and experiences of group exercise for which the existing social high theory does not satisfactorily explain.

in this dissertation I propose to develop cognitive and evolutionary understandings of group exercise through closer attention to the immediate causal processes and psycho-social effects of interpersonal movement.  I suggest that the carnal mystery that Adrian attempted to articulate that first night in Beijing could be explained in part by the way in which the joint action of rugby enabled and constrained certain physiological, psychological, emotional, and social processes to which Adrian still has access.


\subsection{Extreme cost and profound meaning\label{sect:linkCostMeaning}}

Anecdotal and observational evidence suggests that group exercise contexts often entail extreme (and not just moderate) levels of psychophysiological exertion.  High-stakes professional competitive sporting contexts (international-level sports, rowing or ultra marathon running, for example), extreme adventure sports (big wave surfing, free-diving), or high-intensity contact (rugby union, American football, ice hockey) or combat sports (MMA, boxing, wrestling) are known to involve extreme physiological demands, often including high levels of pain.  Although it is expected that extremely physiologically costly exercise contexts will involve activation of neurobiological reward mechanisms outlined above, some contexts may on average exceed (or alternatively never reach) the intensity and duration sweet-spot for optimal activation of neurobiological reward \citep{Raichlen2013} or positive affect \citep{Ekkekakis2011,Reed2006}.

At the same time, physical exercise involving extreme physiological, psychological, and social costs also appears to offer its participants and observers an opportunity for profound meaning.  Many people do not engage in exercise \textit{just} enjoyment or health; rather, in some contexts sport forms part of a life of purpose and self-discovery \citep[see, for example][]{Jackson1995,Jones2004,White2011}.  Modern sport has always been much more than ``just a game,'' and instead offers an arena in which virtues and vices are learned, and the ``morality plays''—--of community, nation, or globe—--thus performed \citep{Elias1986,McNamee2008}.  Psychological and physiological resilience in exercise contexts is lauded as virtuous, as is evidenced by the numerous idioms in the English language that receive currency in exercise lore: ``when the going gets tough, the tough get going,'' ``no pain, no gain,'' ``you get out what you put in'' and so on \citep{Sarkar2014}.

Whereas the social high theory predicts motivation for exercise based on ``hedonic'' enjoyment, anecdotal and ethnographic perspectives emphasise instead the ethical and moral dimensions of athletes' experiences, and contextualise these experiences within political processes relating to the construction of the self, community, and nation-state \citep{Alter1993,Brownell1995,Downey2005,Wacquant2004}.
Social anthropologists and sociologists have for some time emphasised the social function of exercise and sport in diverse cultural contexts, and various attempts have been made to analyse the phenomenological experience of exercise in terms of its sociological and psychological meaning \citep{Bourdieu1978}.

Social anthropologist Joseph Alter \textcite{Alter1993}, for example, argues that, for wrestlers in north India, the body functions as a nexus through which the symbolic and material structures of the state, family, and the individual coalesce.  In a similar vein, cultural anthropologist Susan Brownell \textcite{Brownell1995}, in a seminal ethnography of sport in China, argues that sport functions as a crucial national symbolic practice for the Chinese nation-state in a project of ``rejoining the world,'' and that the ``micro-techniques'' (c.f. Foucault, 1977) of this project entail significant cost to (and rich meaning for) to the individual athlete.   Similarly, French sociologist Loic Wacquant \textcite{Wacquant2004}, in an ethnography of boxers in Chicago's south side, describes a ``social logic'' of physical activity, claiming that the costs associated with ``the daily dedication and high technique that training demands; the regimented diet; the control, mutual respect, and tacit understandings necessary for actual fist-to-fist competition serve to create for the boxer an island of order and virtue'' \textcite[17]{Wacquant2004}. In many instances, it may be that the primary psychological motivation for exercise is not immediate, reward-induced hedonic wellbeing, but instead \textit{eudemonic} wellbeing, or the psychological awareness of a process through which life becomes ``well-lived'' \citep{Fave2009,Huta2013}.

\subsection{Coordination complexity and team click\label{sect:linksComplexClick}}
In its current form, the social high theory also cannot account for the identifiable relationship between complexity of coordination in group exercise and perceptions of peak performance, flow or ``team click.''  While some group exercise contexts do contain high levels of behavioural synchrony (rowing, synchronised swimming, diving, mass calisthenics, and dance such as ballet), exact in-phase synchrony is in fact atypical of most instances of interpersonal coordination \citep{Fusaroli2014}.  Interpersonal coordination in group exercise is more often achieved through flexible, function-specific assemblages of complimentary and contrasting behaviours (for example, coordination in an interactional team sport, a dyadic conversation, or an ensemble music performance).  Real world instances of joint action in group exercise usually entail various distinct elements, often organised hierarchically within a sequence \citep{Schmidt1975,Rosenbaum2009}.  Successful execution of the complex structure of real world joint action requires temporal and spatial precision and flexibility of movement across multiple timescales and sensorial modalities \citep{Sebanz2006}.

At the same time, participants in group exercise contexts involving complex joint action often scrutinise the quality of coordination.  For their success, technically demanding group exercise contexts such as competitive interactional team sports or music-making and dance, depend upon fine-grained precision of coordination of behaviours between individuals:  the movements and goals of one individual must align precisely in time and space with the movements and goals of another.  For highly skilled expert practitioners, who develop a fine-grained sensitivity concerning the perceived outcome of joint action, often the ecstasy of group activity is contingent not just on participation, or on strict synchrony or equivalence or similarity of behaviours, but on the extent to which joint action with co-participants ``clicks.''

%Psychologist Susan Jackson has accumulated considerable evidence of elite level athletes' subjective experience of ``flow'' in joint action:

  \begin{quotation}
    It was just one of those programs that clicked. I mean everything went right, everything felt good . . . it's just such a rush, like you feel it could go on and on and on, like you don't want it to stop because it's going so well. It's almost as though you don't have to think, it's like everything goes automatically without thinking . . . it's like you're in automatic pilot, so you don‘t have any thoughts. You hear the music but you're not aware that you're hearing it, because it's a part of it all \citep[]{Jackson1992}.
  \end{quotation}

The psychological literature of flow and optimal human performance in sport has documented that athletes engaged in team coordination often report total absorption in and complete focus on the task at hand, a transformation of the experience of time (either speeding up or slowing down), and a blurring or transcendence of individual agency, or a ``loss of self''   \citep{Csikszentmihalyi1992,Jackson1995,Jackson1999,McNeill1995}.  Research suggests that flow often occurs in scenarios in which there are clear goals inherent in the activity, as well as unambiguous feedback concerning extent to which goals are either being achieved or not.  In addition, scenarios most conducive to the experience of flow are those in which the technical requirements are challenging but achievable if practitioners are able to extend slightly beyond their normal capabilities\citep{Fong2015}.
The coalescence of these factors is intrinsically rewarding and autotelic\citep{Csikszentmihalyi1975}, activating both ``hedonic'' and ``eudemonic'' dimensions of subjective well-being \citep{Huta2010,Fave2009}.

Psychologists and neuroscientists have extensively studied the experience of flow, and have tabled a series of neuropharmacological \citep{Boecker2008}, neurocognitive \citep{Dietrich2006,Dietrich2011,Labelle2013}, and psychological \citep{Csikszentmihalyi1992} theories for its emergence.  The vast majority of flow research has focussed on the experience of the individual---the athlete, musician, or performer.  Some attempts have been made to extended an analysis of flow and its antecedents to the level of the group and dynamics of interpersonal coordination---a phenomenon termed ``group flow'' \citep{Sawyer2006}---but these attempts lack coherence and development.
It is clear, however, that in addition to the components of flow already identified at the level of the individual, successful performance of technically complex joint action (beyond exact in-phase synchrony) could have distinct psychological and social consequences.

In sum, clear variation in the types, intensities, and durations of group exercise, and the complex structure and subjective experience of, and motivation for group exercise presents an opportunity for further research into explanatory cognitive, evolutionary, and social mechanisms underlying these observable phenomena.

%In particular, I identify two relationships hitherto unaccounted for by existing accounts: the relationship between extreme cost and profound meaning, and the relationship between successful performance of complex joint action and perceptions of flow or ``team click.''


\section{Addressing empirical gaps through a focussed study of rugby in China}
How is it possible to address these missing links?
Based on the evidence reviewed above, it seems likely that, in addition to moderate intensity exertion, in-phase synchrony, and a feel good social high, an investigation into extreme levels of pain, the ``click'' of complex joint action, and profound psychological meaning making, could also offer important insights into a relationship between group exercise and social cohesion.  If so, what cognitive and evolutionary mechanisms are required to link the full diversity of profiles of group exercise with the full diversity of their psycho-social effects?

In this dissertation, I focus my research on a group exercise context that is perfectly suited to address the empirical knowledge gaps outlined above.  Rugby union is a dynamic field-based contact sport that requires of its participants high levels of physiological exertion and complex coordination of joint action (see Chapter ~\ref{chap:researchSetting} Section ~\ref{sect:rugbyUnion} for a more detailed explanation).  Rugby union is also anecdotally and colloquially associated with social bonding in many of the contexts in which it is commonly played \citep{Dunning2005}.

China:
No scaffolding for bonding in the traditional sense: non-WEIRD setting
Pick up the joint action from scratch


 an instance of group exercise that contains high levels of physiological cost, complex joint action, and social bonding.



\subsection{A pathway between joint action and social bonding, via ``team click''}
In this dissertation I focus on the second missing link outlined above. Cognitive and evolutionary accounts of extreme ritual practices offer a theoretical framework for explaining a link between extreme physiological cost and social bonding, pro-sociality, and cooperation \citep{Whitehouse2014,Whitehouse2017}. This framework can be used to develop the social high theory and the relationship between group exercise and social cohesion more generally (for a full treatment of the relevance of the anthropology of extreme ritual to the present study, see Appendix ~\ref{app1:intro} Section ~\ref{sect:extremeRitual}).

The second missing link suggests a possible relationship between joint action and social bonding, mediated by the phenomenon of team click.  While various sources of evidence support this relationship, a relationship between joint action and social bonding has yet to be thoroughly theorised.

In the current formulation of the social high theory, exact in-phase synchrony acts as a stand-in for interpersonal coordination in group exercise.  Behavioural synchrony is an obvious place to begin an investigation into the social effects and evolutionary significance of group exercise.  Synchrony is a highly ordered regime of phase- and frequency-locked coordination that is predictable and efficient. Synchrony is thus easily scalable to the coordination of behaviour in large groups, and this this may explain its cross-cultural ubiquity in cultural practices \citep{Dunbar2010,Tarr2016}.  For human behavioural scientists, synchrony is relatively easy to operationalise in experimental settings with naive participants, and it provides a clear demonstration of motor and cognitive entrainment between individuals.

As I explain in more detail in Chpater ~\ref{chap:theory}, it is important not to become overly fixated on behavioural synchrony as an ideal expression of coordination in group exercise contexts \citep{Keller2014,Fusaroli2014}.  Beyond the mechanical exactness of synchronisation, there is evidence to suggest
that interpersonal coordination relies on ``functional synergies'' \citep{Riley2011,Turvey1990}.  Functional synergies are higher-order control systems formed by coupling movement system degrees of freedom to reduce the dimensionality and increase reciprocal compensation between components \citep{Turvey1978}. Importantly, functional synergies are flexibly and temporarily assembled, and exhibit a ``meta-stable'' state in which tendencies towards synchronisation and ``symmetry breaking'' are co-present and co-active \citep{Richardson2015}.  In this dissertation, I suggest that the notion of functional synergy serves as a more appropriate and inclusive model for human movement coordination \citep{Fusaroli2014}.

A fixation on strict synchrony alone is susceptible to a conceptual slippage, in which ideal coordination in human systems is understood as mechanical reproduction of exact equivalence of behaviour \citep[imagine an interpersonal version of the ``Dark Room Dilemma'' in neurocognition, cf.][]{Clark2013}).  Observation and anecdote of real world joint action scenarios and prevailing evidence from dynamical systems approaches to human movement suggest that synchronisation (synchrony) is but one expression of interpersonal coordination \citep{Kelso2013}.  It is important that a theoretical model of group exercise and social bonding is developed to account for this diversity.  This dissertation aims to incorporate the comprehensive structure of real world interpersonal movement coordination into a testable theory of social bonding through joint action.

Before proceeding, I first identify two challenges to the development of a theory of social bonding through joint action in group exercise.  The first challenge is the relative infancy of the science of sport and exercise from which a theory of joint action and social bonding relies for empirical evidence and theoretical insights.  The second and perhaps more fundamental challenge concerns the limitations of cognitive and evolutionary theories to account for the dynamic interactive phenomenon of team click in joint action.

\subsubsection{The infancy of science of sport and exercise}
The anthropological science of sport and exercise---a field that should offer a rich source of theory and data---is currently marred by a bifurcation between sports psychology and social and cultural anthropology, with very little substantive theoretical predictions and empirical evidence existing in the space between.

Although social scientists of the past few centuries have occasionally included sport and group exercise as pseudo ritual-like activities responsible for producing social cohesion \citep{Durkheim1965,Mauss1935,Turner1977}, nonetheless, core biological, cognitive, and evolutionary questions surrounding sport and exercise remain largely unexplored by the behavioural sciences, including anthropology \citep{Blanchard1995,Downey2005a}.    While efforts to identify proximate cognitive and psychological mechanisms underlying the social function of religion and ritual began soon after the modern evolutionary synthesis \citep{Huxley1942} and cognitive revolution \citep[e.g.,][]{Turner1986}—--the continuation of which has matured as a program of research in evolutionary anthropology and the cognitive science of religion \citep{Barrett2002,Lawson1993,Sperber1996,Whitehouse2004}---an equivalent program of research is yet to emerge around sport and exercise.  The samples of social and cultural anthropology of sport mentioned above (Section ~\ref{sect:linkCostMeaning}), while compelling and rich in ethnographic detail, do not explicitly engage in the project that is the central focus of this dissertation—--i.e., an \textit{explanatory} account of group exercise in human sociality.

Sport and exercise has been meanwhile been investigated within modes of scientific analysis concerned instrumentally with either the health benefits of exercise \citep{Fiuza-Luces2013,Morris1994}, or athletic performance \citep{Beedie2015}, all the while neglecting to consider the cognitive or evolutionary role of exercise in human psychology or sociality \citep{Balish2013,Coulter2015}. A strict instrumental focus on athletic adherence and performance in sports psychology has restricted the psychology of exercise to an analysis of an ``athletic personality'' and the diverse motivations of the ``whole person,'' including the moral, ethical components of exercise, have been neglected \citep{Coulter2015,Laborde2014}.  As such, the cognitive and evolutionary dimensions of group exercise have escaped rigorous empirical analysis within the psychology of sport and exercise.

In sum, existing scientific explanations of sport and exercise are split along two branches---one in which focus on causal physiological, cognitive, and social mechanisms of exercise is driven by priorities of athletic performance and public health outcomes, and another in which the study of group exercise forms part of a broader discipline in which social, ethical, and moral hermeneutics are more dominant than cognitive or evolutionary theory.  This bifurcation has created a scenario in which the evidence available for the cognitive and evolutionary anthropology of group exercise has been been derived either very close to the (figurative) treadmill---from laboratory paradigms in which components and effects of group exercise (e.g., exact behavioural synchrony and direct or assay measures of neuropharmacological activity) are hyper-essentialised experimental operationalisation.  Alternatively, evidence has been derived very far from the treadmill---by ethnographic researchers whose priorities do not include identifying or testing causal cognitive and evolutionary mechanisms.  Cognitive and evolutionary approaches to the phenomenon of group exercise suffer from this bifurcation in anthropology, and would benefit from attempts to address the space between the treadmill and the field.

\subsubsection{The static nature of cognitive and evolutionary theory}
On a second and more fundamental level, constraints of traditional theories of human cognition may involve a challenge to the project of constructing a novel theory joint action and social bonding (via the phenomenon of team click).  Real-word instances of group exercise are complex and multidimensional behavioural phenomena involving dynamic interaction of multiple brains and bodies, situated in varied and richly resourced cultural ecologies.  Development of cognitive and evolutionary theories that can account for this complexity, and in so doing reconcile some of the observable missing links in existing accounts, remains a work in progress \citep{Fuentes2016}.

Researchers have pointed out that flow and team click have traditionally escaped rigorous treatment as processes with naturalistic mechanisms and dynamics \citep{Dietrich2010a,Slingerland2014}. Dietrich, for example, explains that flow presents a paradox that is difficult to explain according to traditional theories of attention and mental effort, which tend to assume better performance, on any task, is associated with increased conscious effort allocated to that task \citep{Dietrich2004b}.  Flow appears to entail spectrum of both effortful and seemingly effortless cognitions implicating embodied action and the resources of the bio-external environment.  Team click, too, is a multidimensional process involving mental, embodied, emotional, and social dimensions.  Armed with only rudimentary explanatory tools from a youthful science of the human mind (cognitive science), it is understandable that scientific progress on the topic has been limited until very recently.

A long-standing movement within the cognitive and behavioural sciences has called for a greater recognition of the physical and dynamical properties of information transfer in biological systems.  Proponents of so-called embodied, embedded, enactive or extended cognition \citep[now collectively referred to as ``4E cognition,'' see][]{Menary2010}, emphasise that cognition typically involves acting with a physical body on an environment in which that body is immersed. Researchers in this movement call out traditional models of human cognition for being too static, abstract, and compartmentalised; human behaviour is traditionally rendered as the outcome of a linear chain of perception, amodal representation, and action-selection located discretely either in the brain or else within certain (often dualistic) subsystems (e.g., emotional and cognitive, System 1 (fast) and System 2 (slow), implicit and explicit, and so on \citep[cf.][]{Diennes1999,Kahneman2011}).  The core of the 4E argument can be boiled down to the idea that just as humans physically move, so too does the information that supports human movement (cognition); human inferential processes not only activate movement, but are also activated by movement, in a dynamical loop of reciprocal causation.  While proponents of the 4E approach have made valuable contributions to a definition of what traditional theories of human cognition are \textit{not}, empirical evidence capable of adjudicating between testable hypotheses has lagged behind high-volume and high-posture  debate.  Interestingly, the debate over the nature of cognition shares many properties with a similar debate over the nature of evolutionary change \citep{Nowak2010,Scott-Phillips2011,Laland2014,Fuentes2016}.

\subsection{An ``active'' theoretical solution}
Fortunately for all involved, a paradigm shift in cognitive and behavioural sciences is shedding new light on human solutions to the challenge of joint action.  Recent advances in neuroimaging technologies \citep{Frith2007}, neurocomputational theories of brain function \citep{Friston2010,Frith2010,Yufik2013,Clark2013}, and constructive attempts to extend the theoretical paradigm of human social cognition to account for inter-individual processes of interaction and coordination \citep{Sebanz2006,Dale2014}, have resulted in a theoretical paradigm, in which it is now more clearly understood that cognitive processes relevant to joint action are distributed throughout brains, bodies, and the physical environment of the ecological niche in which they are situated.  As I explain below and in detail in Chapter ~\ref{chap:theory}, this shift in understanding of human cognition allows for a more powerful and fine-grained explanatory model of how joint action---and, in turn, social connection---could achieved between two or more participants.

The prevailing paradigm, which I consider in this dissertation, conceives of human cognition as a process of ``active inference'' \citep{Friston2010}.  Active inference \citep[and the predictive coding paradigm which it extends, see][]{Clark2013} proposes a radical inverse of traditional models of cognition that rely predominantly on bottom-up sensory inputs and top-down feature detection \citep{Marr1982}. Instead, active inference posits that top-down predictive models themselves shape perception and action, and the only information that travels forward (or from the ``bottom-up'') is the error signals that arise from discrepancies between predictions and the sensorium \citep{Clark2015}.  The active inference approach  depicts a human cognitive system in which perception, mental simulation, emotion, and action are functionally and temporally integrated to manage uncertainty inherent in interactions with the environment \citep{Clark2013}.

Rather than being restricted to a dualistic either/or choice between functionally distinct cognitive modes of inference (e.g. habitual or mental, explicit or implicit, fast and slow \citep[cf.][]{Dienes1999,Kahneman2011}, active inference predicts that humans benefit from flexible deployment of of multiple strategies from a unified web of neural and extra-neural affordances \citep{Pezzulo2013,Clark2015}.  These strategies can be seen to like on a continuum, which ranges from more computationally intensive interoceptive (internal or mental) predictive models, on one end, to lower cognitive mechanisms of movement regulation that facilitate more direct coupling with the extra-neural resources of the task-specific environment, on the other \citep{Riley2011}.

As I explain in more detail in Chpater ~\ref{chap:theory}, of the available alternatives \citep[see, for example,][]{Keller2016}, active inference offers the most suitable theoretical model to explain social cognition of complex real world joint action contexts \citep{Friston2015,Pesquita2017}.  Joint action settings invariably contain multiple hierarchically nested levels of joint action across multiple time scales and engaging multiple sensory modalities.  Joint action typical of sport, and other performative activities such as music making and dance often require on-line and ``in the moment'' execution of complex sequences of movement incorporating the action plans of self and other, as well as goals shared by co-participants \citep{Pesquita2017}.



\section{The present study\label{sect:presentStudy}}

To summarise the ground covered thus far: this thesis is driven by the research question of explaining the puzzling ubiquity of group exercise in the (more recent) human record. Current cognitive and evolutionary accounts suggest a relationship between group exercise and social cohesion, due in part to the way in which group exercise contexts uniquely generate social bonding between participants.  The social high theory posits that social bonding in group exercise is due to the way in which physiological exertion and interpersonal movement coordination combine to generate a psychophysiological environment conducive to affiliation and trust.  The social high theory fails to account for various links between group exercise and social bonding.  In this dissertation I focus on one of these links: the relationship between successful performance of complex joint action and feelings of team click.   I propose the phenomenon of ``team click'' as a candidate construct that can help explain a hitherto under-examined link between joint action and social bonding.


\subsection{Contributions of the present study\label{sect:contributions}}

% Please add the following required packages to your document preamble:
% \usepackage{booktabs}
\begin{table}[]
\centering
\begin{tabular}{@{}llcl@{}}
\toprule
\textbf{} & \textbf{Components}                                                                               & \multicolumn{1}{l}{\textbf{Chapters}} & \textbf{Description}                                                                                                                                         \\ \midrule
          &                                                                                                   & \multicolumn{1}{l}{}                  &                                                                                                                                                              \\
1         & \begin{tabular}[c]{@{}l@{}}A novel theory of social\\ bonding through joint\\ action\end{tabular} & Ch. 2-3                               & \begin{tabular}[c]{@{}l@{}}Supported by an active inference framework; \\ proposes team click as a mediating construct\end{tabular}                          \\
          &                                                                                                   &                                       &                                                                                                                                                              \\
2         & Ethnography                                                                                       & Ch. 5-7                               & \begin{tabular}[c]{@{}l@{}}Beijing men's rugby team (n = 26); \\ Data: participant observation, \\ informal surveys, semi-structured interviews\end{tabular} \\
          &                                                                                                   &                                       &                                                                                                                                                              \\
3         & \begin{tabular}[c]{@{}l@{}}\textit{In situ} survey \\ study\end{tabular}                                   & Ch. 8                                 & \begin{tabular}[c]{@{}l@{}}Conducted during the The National 7s Rugby\\ Championship (n = 174, male = 93)\end{tabular}                                       \\
          &                                                                                                   &                                       &                                                                                                                                                              \\
4         & \begin{tabular}[c]{@{}l@{}}Controlled field\\  experiment\end{tabular}                            & Ch. 9                                 & \begin{tabular}[c]{@{}l@{}}Conducted with athletes from Beijing \\ and Shandong provincial rugby \\ programs, (n = 58, male = 30)\end{tabular}               \\ \bottomrule
\end{tabular}
\caption{Core components of this dissertation}
\label{tab:thesisComponents}
\end{table}


In this section, I outline the contributions of this thesis, namely, the novel theory of social bonding through joint action, and the three empirical studies I performed to substantiate this theory (see Table ~\ref{tab:contributions}).


\subsubsection{A novel theory of joint action through social bonding}




    The overarching prediction of this thesis is that the psychological phenomenon of team click mediates a relationship between joint action and social bonding.

    Within this main hypothesis, I also formulate the following sub-hypotheses:
    \begin{enumerate}
      \item Athletes who perceive greater success in joint action will experience higher levels of felt ``team click.'' I predict that relevant perceptions of joint action success will relate to athlete perceptions of:
        \begin{enumerate}
          \item a combination of specific technical components; or
          \item an overall perception of team performance relative to prior expectations; or
          \item an interaction between these two dimensions of team performance.
        \end{enumerate}
      \item Athletes who experience higher levels of team click will report higher levels of social bonding.
      \item More positive perceptions of joint action success will predict higher levels of social bonding, driven by more positive:
      \begin{enumerate}
        \item perceptions of components of team performance;; or
        \item violation of team performance expectations; or
        \item an interaction between these two predictors.
      \end{enumerate}
    \end{enumerate}









\subsubsection{Ethnography of the Beijing men's rugby team}
  The first study consists of extended ethnographic research with one Chinese professional rugby team, the Beijing Provincial men's rugby team (see Chapters ~\ref{4partAIntroMethod}\nobreakdash~\ref{6ethnographicResults}).  Data were collected during three trips to Beijing over an 18 month period.

\subsubsection{\textit{In situ} survey study of a Chinese National
 Rugby Tournament}
  Following the initial ethnographic starting point, I then broadened the scope of analysis to include all available Chinese professional provincial rugby players. Participants for the second study ($n = 174, men = 93$) were athletes in a Chinese national tournament, in which fifteen different teams from nine different provinces competed over two days for the 2016 Championship (Chapter ~\ref{5ethnographicField}).  The tournament offered the opportunity to investigate hypotheses concerning the relationship between joint action, team click, and social bonding \textit{in situ}, in a real-world instance of high intensity, high stakes joint action.

\subsubsection{Controlled field experiment}
  Subsequently, in order to more definitively assesses the causal mechanisms identified in the predictions of this dissertation, I conducted a controlled field experiment ($n = 58, male = 29$). In a between-subjects design, I manipulated the level of predicted difficulty prior to athletes participating in ostensibly different (but in fact identical) training drills.

%\subsection{Contributions}







\section{Overview of chapters\label{sect:chapters}}
In this introductory chapter, I outline the overarching research question of this dissertation, which is a scientific explanation for the ubiquity of group exercise in human sociality.  I explain that, while the social high theory of group exercise and social bonding has shed light on important causal mechanisms involved in many group exercise contexts, it contains obvious knowledge gaps, owing in part to a bifurcation in scientific approaches to sport and exercise.  I hone in on team click as an observable phenomenon in group exercise worthy of further theorisation and empirical investigation, due to the way in which it appears to involve a relationship between interpersonal movement coordination and social bonding.  I identify the social cognition of joint action as a field of research in which novel theoretical predictions regarding the relationship between group exercise and social bonding can be formulated.  I preview this theoretical formulation and outline this dissertation's three main empirical studies and their knowledge contribution to cognitive and evolutionary anthropology of group exercise.  In the following chapter, I outline in detail a novel theory of social bonding through joint action, in particular the mediating role of the phenomenon of team click.




                                              \end{CJK}{UTF8}{gbsn}
