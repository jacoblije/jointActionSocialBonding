
\begin{savequote}[8cm]

  It takes two to know one.

  \qauthor{--- Gregory Bateson}

\end{savequote}









\chapter{\label{chap:intro}Introduction}



\minitoc





                                          \begin{CJK}{UTF8}{gbsn}

\section{My first night in Beijing \label{sect:adrian}}

Adrian, Kai, and I waited for Mr Shi to arrive in the upstairs area of the Korean BBQ restaurant in a quiet willow lined street just inside Beijing's East 4th Ring Road.  Adrian, the host of the dinner, was an elder of Chinese rugby.  He was captain of the second class of rugby players to graduate from the Chinese Agricultural University (CAU), the home of China's first official rugby union program, established in 1990.  I first met Adrian two years earlier through Kai. Kai---aclose friend of mine---was a more recent graduate of CAU (2007), a former Chinese National rugby team representative, and since graduating from CAU, a lawyer in Beijing.  Mr Shi, the guest of honour for whom the three of us were waiting, was a technical producer for Chinese Central Television's Sport Channel, CCTV5.

The backstory was that CCTV5 needed help producing the commentary for the Rugby World Cup, which they were planning to broadcast for the first time in October 2015.  Mr Shi reached out through his network of relationships in Beijing and soon tracked down Adrian; Adrian tracked down Kai; Kai, in turn, tracked me down.  I was eager to catch up with old friends as well as begin my fieldwork. And so, despite my jet lag, I accepted the invitation and set off to the Korean BBQ restaurant on that first Saturday evening with my notebook and audio recorder (i.e., my mobile phone) in hand.

Adrian naturally held the floor in conversation while the three of us waited for Mr Shi to arrive.  He reminisced fondly about his time playing rugby at CAU, as well as his time after graduation playing with the Beijing Devils, a rugby club in Beijing whose members were predominantly expats.  He assured us that rugby in China was, in those days, fun and free-spirited.  Not like today, now that Chinese rugby has become a professional program in the state sponsored sport system, (owing to its Olympic status in the modified form of the game, rugby sevens, see Chapter ~\ref{researchSetting}).  Adrian talked about going on tour with the Beijing Devils to the UK:  ``Everyone only just scraped together the money to go on tour, we all payed our own way, sometimes you'd get a bit of help from someone or whatever. We did it because we loved the game, not for any other reason,'' he insisted.  Kai and I listened intently, and all of a sudden I realised that this conversation could be relevant so I started taking notes.

When Mr Shi finally arrived, Adrian continued the nostalgic story telling mode, but naturally shifted his target audience from Kai and me to Mr Shi.  When Adrian began to describe in rich detail the experience of camaraderie between he and his Beijing Devils team mates when they participated in overseas rugby tour, he interrupted his own story with an explanatory aside directed at Mr Shi, accommodating for the fact that Mr Shi was relatively unacquainted with the sport: ``This sport, rugby: it's actually very mysterious. If you haven't played it yourself you might not know this type of feeling,'' Adrian respectfully suggested to Mr Shi.  ``Because rugby, you know, you're all on the field together, there's body contact...'' he paused to find the right phrasing,  ``...its a very \textit{carnal} type of feeling.''  His attempts to enrich his communication by gesticulating had led him to have both of his hands clenched as fists in front of him like they were cradling a rugby ball or gripping the steering wheel of a car---a lit cigarette smouldering between the index and middle finger of his right hand.  Adrian concluded by repeating: ``Its very mysterious.'' He shook his head as if baffled and finally released his clenched fists to dab the ash from his cigarette into the ashtray in front of him.  After taking another drag from his cigarette he finally added: ``So it means this rugby circle here in China is very tight...'' (a short pause for another dab of his cigarette) ``...but it doesn't mean that this circle is not also not also complete chaos!''\footnote{Circle (\textit{quanzi} 圈子) is a common way to refer to a social group or community of people}.  The wisdom of Adrians's final punchline was confirmed with a knowing chuckle from all of us, including Mr Shi. Adrian concluded his performance silently, by taking a long, satisfying drag of his cigarette.

%英式橄榄球这个项目其实特别神秘,没玩过的话您可能不知道这种感觉,因为英式橄榄球么,大家在场上有身体接触,是一种``肉''的感觉,大家互相都特别亲,特别神秘。
%所以在橄榄球这个圈子特别亲, 但这不是说这个圈儿也不乱!

I was captivated---but also somewhat surprised---by Adrian's monologue.  I was not expecting, so early into my fieldwork, to happen upon a declaration in which the link between the carnal (\textit{rou} 肉) sensations associated with on-field joint action, and more abstract social processes of interpersonal emotional affiliation (\textit{qin} 亲) and group cohesion of the rugby community (\textit{quanzi} 圈子) was so explicitly and spontaneously emphasised.  It was clear that rugby's visceral dimension continued to capture Adrian emotionally; some fifteen years after he had finished playing his fists still clenched with energy, and his head still shook with amazement.

I was also intrigued that Adrian cited the source of his emotional capture as at once both very specific (derived from playing together with others on the field) and, at the same time, ultimately ``mysterious.''  The aim of my research, which by that time I had broadly formulated, was to explain the human behavioural phenomenon of group exercise in terms of its social, evolutionary, cognitive, and physiological components and associated dynamics.  In essence, the aim was to somehow move from mystery to scientific mechanism.  At this first dinner in Beijing, Adrian's comments both captured the phenomenological mystery of group exercise, and pointed me in the direction of the underlying mechanisms.

Needless to say, I left that first dinner eager to investigate the sources of Adrian's experience of mysterious carnality and social connection in rugby's joint action.  My next stop, on Monday morning, was the Temple of God of Agriculture Sports Institute, where I had organised to conduct ethnographic research with the Beijing Provincial rugby program.


                            \begin{center}
                              * * *
                            \end{center}





\section{Scientific explanations of group exercise}
Competitive team sports, whirling Sufi dervishes, late-night electronic music raves, Maasi ceremonial dances, or the fitness cults of Cross-fit and Soul Cycle---endless examples can be plucked from across cultures and throughout time to exemplify the human compulsion to come together and move together.  How is it possible to explain the prevalence of these activities in the human record?  In this dissertation I contribute to a scientific understanding of physiologically exertive and socially coordinated movement (hereafter simply ``group exercise'') by way of a focussed study of the social cognition of joint action among professional Chinese rugby players.

Because physical movement is a metabolically expensive task for all biological organisms, it is justifiable from an evolutionary standpoint only if the benefits somehow outweigh the costs.  Using this basic calculus, it is easy to imagine how group exercise would have served important survival functions in our ancestral past.  Activities involving group exercise such as hunting, travel, communication, and defence all appear to confer immediate and obvious benefits to individuals and groups \citep{Sands2010}.

In more recent domains of human history, however, the task of explaining the persistent recurrence of group exercise is more complicated.  At least since the late Pleistocene era (approx. 500ka), and particularly since the Holocene transition (approx. 11ka) from hunter-gatherer to agricultural, and later industrial and post-industrial societies, group exercise can be identified in shared cultural practices as varied as religion, organised warfare, music, dance, play, and sport.  Unlike group hunting or defence, however, the fitness-relevant benefits of group exercise in cultural practices such as sport, music or dance are not always as obvious.  One the contrary, many of these activities appear on the face of things to entail extreme time and energy costs for very little immediate reward.

The prevalence of group exercise in a diverse array of shared cultural practices in the more recent human record thus presents an evolutionary puzzle.  A solution to which requires a more nuanced calculus that incorporates an appreciation of humans' species-unique evolutionary trajectory, defined by increasingly complex cognitive and cultural capacities, including technical manipulation of extra-somatic materials and ecologies; advanced theory of mind; and information-rich, malleable, and scaleable communication systems \citep{Roepstorff2010,Clark2015,Fuentes2016}.  A theory capable of satisfactorily explaining group exercise within humans' distinctive evolutionary parameters is yet to be fully formulated \citep{Cohen2017}.

In this dissertation: summary?

\subsection{Group exercise and social cohesion: a social high?}
It has been long speculated by social scientists \citep[see, for example][]{Mauss1935,Durkheim1965}, that cultural activities in which group exercise feature foster social cohesion \citep{Dunbar2010,Whitehouse2004}.  Social cohesion implies proximity, coordination, and stability of relationships between members of a group, which serve some benefit to the group as a whole.  In so far as more cohesive groups face better survival odds than those which are less so, activities that foster social cohesion can be considered collectively advantageous and adaptive \citep{Dunbar2010}.  It is not yet clear, however, precisely how or whether group exercise uniquely generates social cohesion, or in what ways particular mechanisms vary by activity and cultural context.

Existing research detailing the proximate physiological, cognitive, and social mechanisms associated with group exercise suggests that group exercise is responsible for generating a psychophysiological environment conducive to social affiliation and trust.  Anthropologist Emma Cohen has thus proposed a ``social high'' theory of group exercise and social bonding \citep[][hereafter ``the social high theory'', see]{Cohen2017}.  Cohen and colleagues cite evidence linking the two essential ingredients of group exercise---1) physiological exertion and 2) interpersonal movement coordination---with positive affect, wellbeing, pro-sociality, and cooperation \citep{Davis2015}.

\myparagraph{Physiological exertion}
Group exercise necessarily entails rigorous physiological exertion.
The health and wellbeing benefits associated with regular physical exercise, including reduced risk of cardiovascular disease, autonomic dysfunction, early mortality, neurogenesis, enhanced cognitive ability, and improved mood, are becoming increasingly well-known \citep{Blair1994,Nagamatsu2014}. It is now also understood that strenuous and prolonged physical exercise is modulated by the same neuropharmacological systems responsible for regulating pain, fatigue, and reward \citep{Boecker2008,Raichlen2013}.  Neurobiological rewards in exercise are associated with both central effects (improved affect, sense of well-being, anxiety reduction, post-exercise calm) and peripheral effects (analgesia), and appear to be dependent for their activation on exercise type, intensity, and duration \citep{Dietrich2004}.  Exercise-specific activity of neurobiological reward systems offers a plausible explanation for commonly reported sensations of positive affect, anxiety reduction, and improved subjective well-being during and following exercise---extremes of which are popularly referred to as the ``runner's high'' \citep{(Dietrich2004,Boecker2008,Raichlen2012}.  This neurobiological evidence maps on to more extensive literature concerning the psychological effects of exercise, which indicates a duration and intensity ``sweet spot'' for exercise and positive affect, whereby moderate intensity exercise for durations of $\sim45$ minutes appears most optimal \citep{Reed2006}.

It is possible that the function of exercise-induced positive affect extends to the realm of social bonding, particularly when achieved in group exercise contexts \citep{Cohen2009,Machin2011}.  Endocannabinoids and opioids have been implicated in mammalian social bonding \citep{Fattore2010,Keverne1989}, and in humans specifically, there is evidence that endorphins (a particular class of endogenous opioids) mediate social bonding \citep{Dunbar2012,Shultz2010}.

\myparagraph{Interpersonal movement coordination}
Meanwhile, experimental evidence (predominantly from social psychology) suggests that time-locked coordination of behaviour between two or more individuals is conducive to psychological processes of self-other merging, liking, trust, and psychological affiliation.  In these contexts, interpersonal coordination is primarily operationalised as behavioural synchrony---i.e., stable time- and phase-locked movement of two or more independent components (limbs, bodies, fingers, etc.) \citep{Pikovsky2007}.  Relative to non-synchronous group activities, synchrony increases social bonding and pro-social behaviour---an evolutionarily important outcome of bonded relationships \citep{Reddish2013,Reddish2013a,Wiltermuth2009}.  Recent studies have also found that, compared to solo and non-synchronous group exercise, synchronous group exercise leads to significantly greater post-workout pain threshold \citep{Cohen2009,Sullivan2014,Sullivan2013a, Sullivan2013b}.

A recent meta analysis of the behavioural synchrony literature in social psychology suggests three candidate mediators of the relationship between behavioural synchrony and social bonding: 1) lower cognitive affective mechanisms implicating neuropharmacological reward systems (e.g., opioidergic and dopaminergic systems), 2) neurocognitive action perception networks responsible for the experience of self-other merging, and 3) processes of group-centred cognition responsible for perception and reinforcement of cooperation\citep{Mogan2017}.  On the balance of existing evidence,
affective physiological mechanisms may be more relevant to joint action involving larger group sizes in which generalised feelings of euphoria and pro-sociality are common \citep[][e.g., mass religious rituals or music festivals]{Weinstein2016}, whereas neurocognitive mechanisms linking joint action and social bonding may be more applicable to smaller group sizes in which individuals can share intentions through ostensive communicative signals and implicit movement regulation cues \citep{Semin2008,Frith2010}.  Studies linking synchrony with social bonding and cooperation are supported by a literature than connects nonconscious mimicry with liking and affiliation\citep{VanBaaren2009}.

\myparagraph{The social high $=$ synchrony $\times$ exertion}
There is also some preliminary evidence to suggest that exertion and coordination in group exercise interact to produce social effects.  Social features of the exercise environment (for example, perceived social support, level and quality of behavioural synchrony, etc.) modulate exercise-induced mechanisms of pain, and reward \citep{Cohen2009,Sullivan2014,Tarr2015,Davis2015,Weinstein2016}, and this work is bolstered by existing literature on the social modulation of pain \citep{Eisenberger2012a} and links between pain and prosociality \citep{Bastian2014a}.  The social high theory thus combines these two bodies of literature to tell a story in which positive affect---associated with neuropharmacologically-mediated pain analgesia and reward—--is extended to the social group via synchrony-activated cognitive mechanisms of self-other merging, and the perception and reinforcement of in-group cooperation.


\subsection{Knowledge gaps in the relationship between group exercise and social cohesion}

The social high theory serves to account for one dimension to the experience of group exercise, which may be particularly common in collective rituals involving music and dance.  Recently, for example,  researchers have made a ceremony out of invoking one particular passage from Durkheim (1965, pg. 217).  ``Once the individuals are gathered together,'' reads the passage, ``a sort of electricity is generated from their closeness and quickly launches them to an extraordinary height of exaltation...'' \citep{McNeill1995,Konvalinka2011,Fischer2014,Mogan2017}. Indeed,
Durkheim's oft-cited passage lends itself neatly to the prevailing social high theory of group exercise and social bonding, in which rigorous physiological exertion and interpersonal coordination of behaviour interact to generate a general feeling of ``collective effervescence.''

However, only a cursory survey of the spectrum of group exercise contexts identifiable in human sociality is needed to reveal that many group exercise scenarios deviate markedly from the narrow profile of moderate intensity physiological exertion, exact time- and space-locked synchrony, and a feel good social high.  Nowhere in Durkheim's passage does he detail the subjectivity of the performers he observes.  In particular, as Adrian attempted to articulate to Mr Shi that first night in Beijing, there is an unmistakably ``visceral'' dimension to the experience of group exercise that is yet to be fully comprehended by current scientific explanations of group exercise.

\subsubsection{Extreme cost and profound meaning}
While some group exercise contexts do appear to sit within the intensity duration sweet spot described above, it is also clear that many group exercise contexts require extreme (as opposed to moderate) levels of physiological exertion.  Furthermore, many group exercise contexts appear, on the surface at least, to be more fundamentally defined by pain, coercion, discipline, and even violence, instead of hedonic enjoyment.
      \footnote{Examples that come to mind here include high-stakes professional competitive sporting contexts that involve extreme physiological demands (international-level sports, rowing or ultra-marathon running, for example), extreme adventure sports (big wave surfing, free-diving), or high-intensity contact (Rugby Union, NFL, Ice Hockey) or combat  sports (MMA, boxing, wrestling).}
Although it is expected that extremely physiologically costly exercise contexts will involve activation of neurobiological reward mechanisms outlined above, some may on average exceed (or alternatively never reach) the intensity and duration sweet-spot for optimal activation of neurobiological reward \citep{Raichlen2013} or positive affect \citep{Ekkekakis2011,Reed2006}.

It is also apparent that exercise offers to its participants and observers an opportunity for profound meaning.  Many people do not engage in exercise \textit{just} for health or enjoyment; rather, in some contexts sport forms part of a life of purpose and self-discovery \citep[see, for example][]{Jackson1995,Jones2004,White2011}.  Modern sport has always been much more than ``just a game,'' and instead offers an arena in which virtues and vices are learned, and the ``morality plays''—--of community, nation, or globe—--thus performed \citep{Elias1986,McNamee2008}.  Athletes at the elite apex of their sports commonly report the autotelic experience of ``flow''--—described as full immersion in the ``here and now,'' effortlessness, or optimal experience \citep{Csikszentmihalyi1992,Dietrich2004}.  Whereas the social high theory predicts motivation for exercise based on ``hedonic'' enjoyment, anecdotal and ethnographic perspectives emphasise instead the ethical and moral dimensions of athletes’ experiences, and contextualise these experiences within political processes relating to the construction of the self, community, and nation-state \citep{Alter1993,Brownell1995,Downey2005b,Wacquant2004}.
In many instances, it may be that the primary psychological motivation for exercise is not immediate, reward-induced hedonic wellbeing, but instead \textit{eudaimonic} wellbeing, or the psychological awareness of a process through which life becomes ``well-lived'' \citep{Fave2009,Huta2013}.

\subsubsection{Beyond synchrony: the cognitive challenge of joint action}
From a cognitive perspective, establishing and sustaining successful joint action is a daunting computational task.  As Bernstein first pointed out \textcite{Bernstein1967}, just as any intra-personal gross motor movement requires a flexible yet precise assembly and coordination of thousands of muscles and hundreds of joints, so too does interpersonal movement require the coordination of the ``degrees of freedom'' of autonomous co-actors and the features of the physical environment \citep{Riley2011}.  No matter how fluid and effortless interpersonal coordination can be made to look during expert performances of music, dance, and sport, or even in simple face-to-face conversations between friends, without privileged access to high fidelity information about the intentions of others, joint action is destined to be marred by extreme levels of informational uncertainty \citep{Sebanz2009,Fusaroli2014}.  Despite the cognitive improbability of joint action, humans have managed---stoically, and at times somewhat elegantly---to devise a number of effective cognitive and behavioural solutions to joint action throughout their evolutionary trajectory.

To be sure, behavioural synchrony in human cultural practices can be understood as one such \textit{behavioural} solution to the challenge of achieving and sustaining interpersonal coordination \citep{Kirschner2010}.  But despite being easy and efficient for its performers and immediately eye catching and symmetrical for its observers, strict in-phase behavioural synchrony in fact atypical of most instances of interpersonal coordination \citep{Fusaroli2014}.  Examination of a broader corpus of human interpersonal coordination reveals instead that coordination in joint action is more often achieved through function-specific assemblages of complimentary and contrasting behaviours (for example, coordination in an interactional team sport, a dyadic conversation or an ensemble music performance).  In real world group exercise scenarios, interpersonal movement coordination requires temporal and spatial precision and flexibility multiple timescales and sensorial modalities in order to bring about change in the environment \citep{Sebanz2006}.  Time- and phase-locked synchrony is but one regime in an array of coordination regimes that enable movement coordination in joint action \citep{Kelso2013}.

The quality of coordination in joint action is heavily scrutinised in technically demanding group exercise contexts such as competitive interactional team sports or, music-making and dance.  For their success, these activities depend upon highly complex coordination of behaviours between individuals, in which the movements and goals of one individual must align precisely in time and space with the movements and goals of another.  For highly skilled practitioners who develop a fine-grained sensitivity concerning the perceived outcome of joint action, often the ecstasy of group activity is contingent not just on participation, or on strict synchrony or equivalence or similarity of behaviours but on the extent to which joint action with co-participants ``clicks.''  The psychological literature of optimal human experience (also known as ``flow'' \citep{Csikszentmihalyi1992}) offers extensive documentation of the positive psychological and social effects of successful performance of technically complex movement in individual and, to a lesser extent, joint action.  To date, however, very little research has dealt directly with the relationship between perceptions of performance of joint action and processes of social bonding and group formation \citep[but see][]{Marsh2009}.  In sum, the elaborate structure and complexity of coordination entailed in real world joint action necessitates a broader consideration of the cognitive mechanisms that underpin social bonding in joint action.

In this section I briefly sketch key knowledge gaps in the social high theory of group exercise and social bonding. It is evident that the relationship between group exercise and social cohesion requires an explanation that extends beyond moderate intensity physiological exertion, exact behavioural synchrony, and a feel-good social high.  Group exercise contexts also often entail extreme psychophysiological costs, generate profound meaning, and require fine-grained sensitivity to interpersonal coordination, often in scenarios defined by high informational uncertainty.  Variation in the types, intensities, and durations of group exercise, and the complexity of subjective experience of and motivation for group exercise presents an opportunity for further research into explanatory cognitive, evolutionary, and social mechanisms underlying these observable phenomena.   In this dissertation I aim to extend the social high theory to include a greater appreciation of the cognitive mechanisms and psychological and social effects of joint action in group exercise.

\section{The social cognition of joint action}
Fortunately, a paradigm shift in cognitive and behavioural sciences is shedding new light human solutions to the challenge of joint action.  Recent advances in neuroimaging technologies \citep{Frith2007}, neurocomputational theories of brain function \citep{Friston2010,Frith2010,Yufik2013,Clark2013}, and constructive attempts to extend the theoretical paradigm of human social cognition to account for inter-individual processes of interaction and coordination \citep{Sebanz2006,Dale2014}, have amounted to a paradigm shift in cognitive science, in which it is now more possible to gain access to the component mechanisms and system dynamics of joint action.  Inspired originally by the mechanics of the electronic computer \citep{Liu2010a}, traditional individual-bound computational models of information processing tend to render movement as the final product of a linear sequence of sensory perception, amodal mental representation, and action selection \citep{Lewis2005}.  By contrast, the prevailing paradigm, which I adopt in this dissertation, conceives of human cognition as a process of ``active inference'' \citep{Friston2010}.  Active inference depicts a human cognitive system in which perception, mental simulation, emotion, and action are functionally and temporally integrated to manage uncertainty inherent in interactions with the environment \citep{Clark2013}.

Active inference conceptualises a human brain driven by the overarching mandate of reducing ``free energy'' (informational uncertainty or entropy) in its interactions with the environment \citep{Friston2010}.  The most effective way in which to adhere to this thermodynamic mandate is to pro-actively anticipate (predict) the external causes of sensory inputs.  Active inference \citep[and the predictive coding paradigm on which it extends, see][]{Clark2013} proposes a radical inverse of traditional models of cognition that rely predominantly on bottom-up sensory inputs and top-down feature detection. Instead, the theory of active inference suggests that top-down predictive models contain the instructions for perception and action, and the only information that travels from the bottom-up is the errors that result from a discrepancy between predictions and the sensorium (prediction errors).

In this conception, humans flexibly deploy a range of cognitive strategies in order to successfully coordinate behaviour with the envuronment.  These strategies can be seen to like on a continuum, which ranges from more computationally intensive hierarchical interoceptive (mental) predictive models, on one end, to lower cognitive mechanisms of movement regulation that facilitate more direct (i.e., extra-neural) coupling with the resources of the task-specific environment, on the other \citep{Riley2011}.  Importantly, rather than being restricted to a dualistic either/or choice between functionally distinct cognitive modes of inference (e.g. habitual or mental, explicit or implicit, fast and slow \citep[cf.][]{Dienes1999,Kahneman2011}, the active inference paradigm predicts that humans benefit from flexible deployment of of multiple strategies from a unified web of neural and extra-neural resources \citep{Pezzulo2013}.  Active inference offers the most suitable theoretical model available to explain complex real world joint action contexts \citep{Friston2015,Pesquita2017}, which invariably contain multiple hierarchically nested levels of joint action across multiple time scales and engaging multiple sensory modalities.


\section{When joint action clicks}

Global success in real world joint action scenarios such as rugby requires flexible deployment

Evidence discussed below suggests that optimal solutions to joint action may tend to recruit more extra-neural resources to minimise free energy, whereas less efficient solutions to joint action may rely on more computationally intensive procedures in order to reduce free energy.


























\section{Social Connection through joint action}

Coordination of physical movement with others is fundamental to social interaction and cooperation, and it is plausible that humans have evolved social and emotional mechanisms that reward effective joint action with others.

As discussed above, successful joint action in humans requires a continuum of strategies ranging from interoceptive predictive modelling (of the shared task as well as the action plans of self and others required for the shared task), to direct coupling with the task-specific environment via the recruitment of lower-cognitive mechanisms of movement regulation (e.g. proprioceptive mechanisms of balance and orientation).  Precisely which strategies, and in which scenarios these strategies could be responsible for social connection in joint action remains poorly understood.

The link between interpersonal coordination and social bonding has been addressed in the behavioural mimicry and synchrony literatures \citep[e.g.,][]{Wheatley2012,Launay2016,Mogan2017}, but there is less substantive evidence in relation to dynamic interpersonal coordination in natural joint action settings such as those found in group exercise contexts \citep{Marsh2009,Miles2009,Lumsden2012}.

There is strong evidence from the synchrony literature to suggest that a combination of 1) neuropharmacological reward arising from lower-cognitive affective mechanisms, 2) self-other merging resulting from neurocognitive alignment, and 3) reinforcement of cooperative relationships owing to experience of interpersonal alignment in joint action generates a psychophysiological environment conducive to generating social bonds.


In particular, it is becoming better understood that basal human capacities for physical movement regulation set the foundation for social cognitive systems whose resources are distributed between brains, bodies, and physical features of task-specific environments \citep{Hutchins2000,Kirsh2006,Semin2008,Semin2012,Coey2012}.  This implicit ``common ground'' appears to be crucial to effective function of more complex goal oriented social activities, including the large-scale reproduction and transmission of shared cultural practices \citep{Dunbar2012,Roepstorff2010,Claidiere2014,Launay2016}.  Furthermore, it has been shown that the quality of coordinated movement within these cognitive systems has implications for psychophysiological health and subjective well-being \citep{Wheatley2012},
















\subsubsection{Affordance of culture}

Discussed as part of the theoretical framework outlined in Chapter 2, shared cultural knowledge can act as a ``coordination smoother'' \citep{Vesper2017} for joint action, enhancing the effectiveness and efficiency of joint action between co-participants who share a similar informational framework.  In the predictive coding paradigm, cultural habits and frames of reference act as ``hyper-priors'' that set the macro-contextual coordinates for joint action\citep{Clark2013}.  Contextual affordances for joint action appear to be dictated by processes operating at multiple conceptual levels---from the micro-level predictive processes associated with movement action and perception, to the macro-level predictive frames offered by specific cultural and contextual niches---interact in complex processes of reciprocal causation to shape joint action (SOURCE).  Conceptualisation of the causal complexity of cognitive processes relevant to joint action in this way echoes a broader reconceptualisation of the causal complexity associated with change on an evolutionary timescale, which recognises that human behavioural phenomena is the result of a number of biological, cognitive, and ecological mechanisms that interact via reciprocal feedback loops spanning varying scales of time and space \citep{Fuentes2015}.






Culture as stand in for affordances for joint action:
Firth, Vesper,

    shared elements that provide standards for perceiving, believing, evaluating, communicating, and acting among those who share a language, a historical period, and a geographical location.” Shavitt et al. (2008, p. 1103)

    stand-in for a similarly untidy and expansive set of material and symbolic concepts … that give form and direction to behaviour [and that] culture is located in the world, in patterns of ideas, practices, institutions, products, and artefacts  Kitayama (2010, p. 422)

    These widely accepted definitions incorporate factors that are both external to people,
    * such as societal values or similar cultural dimensions (Hofstede, 1991, 2001; Schwartz, 1992; Soares, Farhangmehr, & Shoham, 2006), social practices (Nisbett & Masuda, 2003), and artifacts (Craig & Douglas, 2006),
     and internal,
    * such as an independent/ interdependent self-construal or other traits (Markus & Kitayama, 1991), including the overlooked aspect of language (Ambady & Bharucha, 2009; Ross, Xun, & Wilson, 2002; Sen, Burmeister, & Ghosh, 2004).


    Rugby:
    I many real world joint action scenarios the complexity, uncertainty, and intensity of movement coordination is amplified, often imposing cognitive constraints and requiring greater technical competence and interdependence among participants in order for joint action to be successful.

%The interactive team sport of rugby union, participants engage in a series of hierarchically organised joint actions. At the top of the hierarchy, participants from two separate team jointly agree to pick up a ball and play a game on a 70m by 110m rectangular playing field; participants within teams then jointly agree to form a team and coordinate sub-phases of joint action within that team, as well as to foil the attempts of the opposing team to coordinate in joint action.  Below the team level, a series of sub phases of joint action take place, all the way down to the level of dyadic joint actions (for example, one attacker taking on one defender, or one attacker passing the ball to another attacker). Each level of organisation adds an extra layer of complexity to the structure and sequencing of interpersonal coordination.  In addition to its complex structure, joint action in rugby also entails extreme levels of physiological exertion and unmitigated body-on-body contact, and these components further amplify the base-level improbability of joint action success.  As current research suggests, this contrived uncertainty of joint action could be an important explanation for rugby's social bonding effects.  For these reasons, rugby is a highly suitable case study for studying team click in joint action.


    China:

    The cultural setting in which I conduct my research adds an extra dimension to an investigation into the joint action-social bonding link. Humans' various cultural and ecological niches show clear variation in solutions to joint action.  The way in which cultural niches enable and constrain certain solutions and psychological effects of joint action is a question yet to be thoroughly explored by cognitive science and social psychology.  This knowledge gap is due in part to the fact that cognitive science and social psychology are knowledge regimes that have themselves emerged from a specific cultural niche, and in part to the fact that it is only recently that ``culture'' has been incorporated as an informational affordance into theoretical models of social interaction. In this dissertation, I demonstrate how the culturally specific terrain of rugby in China affords more generalisable processes of social connection through joint action.


  Working within the group exercise context of rugby in China, I collect and present ethnographic and field-experimental evidence that offers some confirmation of this core prediction.  I evaluate these results in terms of their implications for understanding the proximate cognitive mechanisms, ecological system dynamics, and ultimate evolutionary processes relevant to the anthropology of group exercise.

















\section{Introducing the phenomenon of Team Click\label{sect:teamClickIntro}}


surprising, emotionally rewarding and readily attributed to the agency distributed beyond the self.

Having identified knowledge gaps in the social high theory and the anthropology of group exercise more generally, in this section I focus on the phenomenon of peak team performance in group exercise contexts, what I call ``team click.''

One of the big mysteries of competitive team sport, particularly at the elite level, is the elusiveness of peak team performance.  While each individual athlete may exhibit expert level competence in sport specific skills, the much sought after aggregation of these components, i.e. a team that consistently performs ``in the zone,'' and ``firing on all cylinders,'' in reality often proves frustratingly difficult to achieve and sustain.  As King and De Rond \textcite[568]{King2011} note in their ethnography of the 2008 Cambridge University rowing crew who participated (and who were eventually victorious) in the famous annual Boat Race against Oxford University, the search for collective rhythm is a universal in human social interaction, but  the physiological and psychological complexity of finding that rhythm ``...is extremely difficult to attain; collective performance is a possibility not a certainty.''   The moment in which everything ``clicks'' into place in team sport can, for various reasons, disappear as abruptly as it arrives, if indeed it arrives at all.

But, when team click is somehow cultivated, and even sustained, it is celebrated as the ultimate, albeit often inexplicable magic of sporting feats. Consider Leicester City Football Club's unbelievable outhouse-to-penthouse title run in the 2015 English Premier League, the recent dominance of the Golden State Warriors in the American National Basketball League, or the astonishingly consistent performance of the New Zealand men's national rugby union team (the ``All Blacks'').\footnote{The All Blacks are arguably the most successful sporting team ever, with a winning percentage of 77\% in the last 150 years (and 88\% in the last 6 years) \citep{SOURCE}}.  All of these successful teams carry with them a powerful ``aura'' associated with their capacity to effectively coordinate their behaviours on the field over extended time scales: individual games, seasons, and, in the case of the All Blacks, entire generations.  The aura associated with such rare instances of collective performance can seduce fascination and elaborate exegesis.  In this thesis I commit to the more banal task of naturalising the phenomenon of team click, by locating it within a novel theory of social bonding through joint action.

I use the term team click to describe the phenomenology of peak performance within a team of athletes engaged in joint action.  For athletes, coaches, and spectators alike, team click can be a hugely powerful sensation. As theologian Michael Novak explains, ``[f]or those who have participated on a team that has known the click of communality, the experience is unforgettable, like that of having attained, for a while at least, a higher level of existence'' \citep[11]{White2011}. It has been extensively documented in the psychological literature of flow and optimal human performance in sport that athletes engaged in team coordination often report total absorption in and complete focus on the task at hand, a transformation of the experience of time (either speeding up or slowing down), and a blurring or transcendence of individual agency, or a ``loss of self''   \citep{Csikszentmihalyi1992,Jackson1995,Jackson1999,McNeill1995}.  Research suggests that flow often occurs in scenarios in which there are clear goals inherent in the activity, as well as unambiguous feedback concerning extent to which goals are either being achieved or not.  In addition, scenarios most conducive to the experience of flow are those in which the technical requirements are challenging but achievable if practitioners are able to extend slightly beyond their normal capabilities\citep{Fong2015}.
The coalescence of these factors is intrinsically rewarding and autotelic\citep{Csikszentmihalyi1975}, activating both ``hedonic'' and ``eudaimonic'' dimensions of subjective well-being \citep{Huta2010,Fave2009}.

The experience of flow has by now been extensively studied by psychologists and neuroscientists, from which a series of neuropharmacological \citep{Boecker2008}, neurocognitive \citep{Dietrich2006,Dietrich2011,Labelle2013}, and psychological \citep{Csikszentmihalyi1992} theories for its emergence have been tabled.  The vast majority of flow research has focussed on the experience of the individual---the athlete, musician, or performer.  Some attempts have been made to extended an analysis of flow and its antecedents to the level of the group and dynamics of interpersonal coordination---a phenomenon termed ``group flow'' \citep{Sawyer2006}---but these attempts lack coherence and development.

Team click shares many similarities with the psychological states associated with flow, but is distinct in that it specifically delineates perceptions of joint action from individual action, and therefore implicates physiological, cognitive, and social mechanisms unique to joint action \citep{Vesper2010}, as well as nonlinear systems dynamics associated with participating in a socially-coordinated, multi-agent system of physical movement \citep{Kelso2009}.  Team click is anecdotally present in a wide range of joint action contexts, and is often associated in these contexts with psychological processes of positive affect and wellbeing, as well as personal agency, social affiliation, and group membership \citep{Jackson1995,Marsh2009,Wheatley2012,Slingerland2014}.

Importantly, team click appears to have important flow-on consequences relevant to social bonding and affiliation. Tightly synchronised activity in particular, found in team sports such as rowing, can help dissolve the boundaries between individual and social agency: ``In rowing...it feels like you have at your command the power of everybody else in the boat. You are exponentially magnified. What was a strain before becomes easier. It is absolutely the ultimate team sport'' \citep{Brown2016}.
The blurring of agency between self and team may be responsible for facilitating affiliation and trust between teammates in competitive athletic environments such as professional rugby, which often involves high physiological stress and uncertainty: ``...you always wanted a guy who would go into the trenches with you and he always played consistently well...he could really play and was just one of the good lads that you enjoyed his company'' \citep{Fox-Sports2017}. In this sense, the experience of team click may act as a social diagnostic tool, a powerful signal of commitment to joint action and willingness to cooperate \citep{Reddish2013a}. \\
\\
\\

\noindent\fbox{%
    \parbox{\textwidth}{%
Team click refers to the perception of peak team performance.  Based on anecdote and evidence from psychology and anthropology, an individual's perception of team click should contain some or all of the following components:
    \begin{enumerate}
      \item Flow or coherence of joint action
      \item Tacit understanding between co-actors
      \item Atmosphere or aura around team performance
      \item Teammates are responsible for extending individual ability
      \item Teammates are reliable co-actors
      \item The individual is a reliable co-actor for teammates
    \end{enumerate}
    }%
}

\\
\\
\\


As I explain in more detail in the next chapter (Chapter ~\ref{theory}), team click is a candidate concept for explaining the link between joint action and social bonding in group exercise contexts.












\section{A theory of social bonding through joint action}

In this section I introduce a novel theory of social bonding through joint action, which is described in detail in Chapter ~\ref{theoryContribution}.

  \subsection{Expectations violation: a potential affective mechanism}


\subsection{Study Predictions}


    The overarching prediction of this thesis is that the psychological phenomenon of team click mediates a relationship between joint action and social bonding.

    Within this main hypothesis, I also formulate the following sub-hypotheses:
    \begin{enumerate}
      \item Athletes who perceive greater success in joint action will experience higher levels of felt ``team click.'' I predict that relevant perceptions of joint action success will relate to athlete perceptions of:
        \begin{enumerate}
          \item a combination of specific technical components; or
          \item an overall perception of team performance relative to prior expectations; or
          \item an interaction between these two dimensions of team performance.
        \end{enumerate}
      \item Athletes who experience higher levels of team click will report higher levels of social bonding.
      \item More positive perceptions of joint action success will predict higher levels of social bonding, driven by more positive:
      \begin{enumerate}
        \item perceptions of components of team performance;; or
        \item violation of team performance expectations; or
        \item an interaction between these two predictors.
      \end{enumerate}
    \end{enumerate}


\section{Thesis Overview}

These predictions set the foundation for a particular study focussed on the relationship between joint action and social bonding in the case of professional Chinese rugby players. The specific ethnographic context (sport in China) demands a careful consideration of the predictions formulated above.  Culturally specific processes of self-construal and social group formation challenge some of the assumptions built in to the literatures mentioned above.  However, I argue that the predictions outlined above are robust to these cultural specificities, due to the fact that they are grounded in an agent neutral distributed social cognition framework. Indeed, despite distinct cultural variation processes of team membership, ethnographic analysis reveals that the experience of team click is strongly identifiable.

In order to test these predictions, I conduct a series of three interrelated studies.  The first study consists of extended ethnographic research with one Chinese professional rugby team, the Beijing Provincial men's rugby team (see Chapters ~\ref{4partAIntroMethod}\nobreakdash~\ref{6ethnographicResults}). From this ethnographic starting point, I then broadened the scope of analysis to include all available Chinese professional provincial rugby players. Participants for the second study ($n = 174, men = 93$) were athletes in a Chinese national tournament, in which fifteen different teams from nine different provinces competed over two days for the 2016 Championship (Chapter ~\ref{5ethnographicField}).  The tournament offered the opportunity to investigate hypotheses concerning the relationship between joint action, team click, and social bonding \textit{in situ}, in a real-world instance of high intensity, high stakes joint action.  Subsequently, in order to more definitively assesses the causal mechanisms identified in the predictions of this dissertation, I conducted a controlled field experiment ($n = 58, male = 29$). In a between-subjects design, I manipulated the level of predicted difficulty prior to athletes participating in ostensibly different (but in fact identical) training drills.

\subsection{Contributions}




\subsection{Chapter Summary}
In this introductory chapter, I outline the overarching research question of this dissertation, which is a scientific explanation for the ubiquity of group exercise in human sociality.  I explain that, while the social high theory of group exercise and social bonding has shed light on important causal mechanisms involved in many group exercise contexts, it contains obvious knowledge gaps, owing in part to a bifurcation in scientific approaches to sport and exercise.  I hone in on team click as an observable phenomenon in group exercise worthy of further theorisation and empirical investigation, due to the way in which it appears to involve a relationship between interpersonal movement coordination and social bonding.  I identify the social cognition of joint action as a field of research in which novel theoretical predictions regarding the relationship between group exercise and social bonding can be formulated.  I preview this theoretical formulation and outline this dissertation's three main empirical studies and their knowledge contribution to cognitive and evolutionary anthropology of group exercise.  In the following chapter, I outline in detail a novel theory of social bonding through joint action, in particular the mediating role of the phenomenon of team click.




                                              \end{CJK}{UTF8}{gbsn}
