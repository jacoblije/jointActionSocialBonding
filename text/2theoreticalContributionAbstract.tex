\chapter*{Chapter 2: Abstract}
%\addcontentsline{toc}{chapter}{\nameref{ch:study1intro}}


%\markboth{{Introduction to Part I}}{Introduction to Part I\label{ch:part1intro}

In this chapter I review evidence from neurophysiology, cognitive neuroscience, and psychology that suggests a causal link in joint action between dynamical coordination of physical movement and social communication.  This physical--social link sets the foundation for a novel theory of social bonding through joint action in group exercise. In particular, I identify team click as a potential mediator of the relationship between joint action and social bonding in group exercise contexts.  I utilise prevailing dynamical and embodied theories of cognition to account for the surprising, visceral, and agentic dimensions of the experience of team click in joint action, as well as the relationship between team click and processes of social bonding, including perceptions of emotional support, common goal, and shared social identity with co-actors.  The hypothesised relationships between 1) more positive perceptions of performance in joint action and higher levels of team click, and 2) higher levels of team click and higher levels of social bonding form the basis of a novel theory of social bonding through joint action, which is explained by the mediator variable of team click.


%As a response to the theoretical and empirical gaps in cognitive and evolutionary accounts of group exercise outlined in Chapter~\ref{chap:intro}, in this chapter I review strands of research capable of accounting for team click and social connection through joint action.

%While it is now well-established that successful interpersonal entrainment of movement can be responsible for generate positive psychological and social effects, the mechanisms through which entrainment is achieved remain poorly understood.  As a response to the theoretical and empirical gaps in cognitive and evolutionary accounts of group exercise outlined in Chapter~\ref{chap:intro}, in this chapter I review strands of research capable of accounting for team click and social connection through joint action.  This review sets the foundation for a novel theory of social bonding through joint action, presented in the next chapter (Chapter~\ref{chap:theoryGE}).

%In particular, I identify team click as a special case phenomenon of joint action and a potential mediator of the relationship between joint action and social bonding in group exercise contexts.  Emerging research from the social cognition of joint action suggests that a continuum of cognitive mechanisms are responsible for establishing and maintaining behavioural coordination between two or more individuals.

%More efficient solutions to the cognitive uncertainty of joint action appear to involve reducing the cognitive demands associated with interoceptive predictive modelling and instead increasing reliance on more direct extra-neural coupling with the environment.  These mechanisms also appear to be modulated by inter-individual and cultural variation in knowledge, expertise, experience, and personality type.  Evidence suggests coordination in joint action can set the foundation for social connection, and the phenomenon of team click can be understood as an optimal state of interpersonal coordination in joint action that maximally activates a causal pathway between joint action and social bonding.  I conclude the chapter by summarising a theory of social bonding through joint action, and outlining predictions that arise from this theory.

%In this dissertation I attend specifically to the relationship between joint action and social bonding in the group exercise context of professional rugby in China.  In this chapter I formulate a novel theory of social bonding through joint action in order to address the knowledge gaps in the social high theory of group exercise and social bonding.
