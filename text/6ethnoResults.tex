\chapter{\label{5ethnographicResults}Ethnographic Results 2}

\minitoc



\section{Abstract}
In this chapter I analyse the predictions of this dissertation in light of ethnographic evidence collected with the Beijing Men's rugby team.  I observe evidence in which the contours of generalisable cognitive mechanisms relevant to a relationship between joint action and social cohesion are visible.  I identify three core areas of analysis.  First, athletes pay close attention to their performance of joint action requirements specific to rugby union.  In particular, I notice that violation of athletes' expectations around joint action, which can be either positively or negatively valenced, is a source of strong emotional response with implications for processes of group formation.  Second, ethnographic observations reveal that the experience of ``team click'' is a prominent experience in joint action associated with rugby, and appears to be connected in athletes' reasoning to concepts of team cohesion.  Third, I identify evidence of social bonding in processes of group formation.  In addition to these three core areas of ethnographic evidence, I also flag the possibility that these three processes are moderated by within-group variation in technical competence, personality type, injury, and athlete experience of fatigue and exertion in exercise.  I conclude this chapter by considering how these ethnographic data could be combined with existing theory and evidence and operationalised in further empirical studies beyond this specific ethnographic setting.


                                      \begin{CJK}{UTF8}{gbsn}


\section{Vignette}

Story about Joint Action, Social Bonding, Team Click?



\section{Introduction}








\section{Results}

Summarise

\subsection{Perceptions of performance in joint action}

Unsurprisingly, ethnographic observations revealed that perceptions of performance of on-field requirements of rugby were a central dimension to athletes' experience of the rugby program at the Institute.  As explained in the previous section, most of the athletes in the program at the Institute transitioned from other sports, or otherwise had no specific background in sport before coming to the Institute.  Rugby's minnow status in China means that its technical requirements are not widely known or understood.  Thus, from a very low base of prior knowledge, athletes are forced to develop technical competence in rugby and a fluency in the social environment of the team.  The ability to effectively coordinate behaviour with others, both on and off the field, is therefore crucial to an athlete's ability to navigate the rugby program at the Institute and access the incentives available.

%In addition, the concept of living and training full time with a group of non-kin is, as a general rule, a relatively novel social requirement for most athletes.  Millions of Chinese university aged students leave home and cohabitate in university dormitories, prior to which point most live at home with their immediate families.

%Starting at time with almost zero prior experience of rugby, athletes must learn the technical skills of passing, tackling, running, kicking---and all of these these individual technical components must be harmonised in joint action with other athletes on the same team, as well as against opposition athletes.  Off the field, as discussed in the previous chapter, the rugby program at the Institute requires athletes to navigate a complex social environment, structured by hierarchical relations of power.

\subsubsection{Team Awareness}
During interviews, within the general theme of the costs and benefits of adherence to rugby, I asked athletes what they found to be ``the most difficult part of rugby.''  20 of the 26 athletes interviewed responded directly to the question, and all interpreted it as a question about rugby's technical technical demands.  As unruly undergrad Fang Chao summarised well:

\begin{quotation}
  The hardest thing about rugby is all of the small details. Rugby is such a comprehensive sport, and playing in a real game is very different from training. I remember when was playing a tournament in 2014, at the time I had only just started training (for rugby).  As soon as I took the field my whole brain was blank.  Now I am a bit better, my vision is broader.
\end{quotation}

\begin{quotation}
  橄榄球最难的一部分是很多小细节吧。橄榄球是很全面的项目...打比赛跟训练还是有区别的. 14年的时候打比赛,那时候刚练,一上场整个脑子就空白了,现在会好一点,视野也开阔了
\end{quotation}

Responses indicate that athletes perceive the on-line, ``in the moment'' experience of spotting opportunities and making decisions in joint action as the most difficult. 11 athletes cited an aspect of rugby that related to on-field coordination of behaviour in joint action, such as decision making in attack, ``awareness'' or vision in game play, coordinating running lines, executing planned plays, and so on.  Seven athletes cited aspects of individual performance, such as tackling, speed in defence, footwork in attack. The remaining two athletes cited both an aspect that was related to coordination of behaviours with others, as well as an aspect of individual performance.
As Wang Zhenfeng, a tall and extremely talented (but inexperienced) senior athlete, explained:

\begin{quotation}
  For me, it's probably the decision making with ball in hand: after you get the ball, how you create opportunities for others---I think that's especially difficult.  For me I think breaking the line or attack or whatever, breaking the line or stepping and so on...I'm ok at all of these things, but if you ask me to not only break the line, but also then create an opportunity for someone else---that's when I find it difficult.
\end{quotation}

\begin{quotation}
  对我来讲应该是球的分化,拿到球之后分化给别人创造机会我觉得特别难。我觉得突破或进攻还是干嘛的,去突变向什么的对我来说都还好,但是让我即能突然后给别人创造机会我觉得很难。
\end{quotation}

Wang Zhankun, a young and promising kid from Chaoyang school expressed a similar sentiment:

\begin{quotation}
  I’d say the ability to adjust to being on the field in a game.  When I’m on the field I can only see those few people who are close to me, when I’m tired I feel like I can’t see anything, like there could be a gap, or an opportunity, but often I can see it.
\end{quotation}

\begin{quotation}
  场上的适应能力吧。我在场上只能看到离自己近的几个人,累的时候感觉看不到所有,可能空档啊、机会啊经常看不到
\end{quotation}

Among responses to this question, the theme of ``awareness'' (\textit{yishi} 意识) was prominent. Of the 13 athletes who cited coordination of behaviour with others as (or one of) the most difficult aspect of rugby,  Six athletes mentioned awareness directly: either on-field awareness, continual awareness; while three talked about on-field coordination with others (\textit{peihe} 配合), two talked about on field decision-making, and the remaining talked about planned running lines (\textit{paowei} 跑位) and team moves (\textit{zhanshu} 战术). As Bao Yuhan, an unruly undergrad and Beijing local, expressed:

\begin{quotation}
  I think its about maintaining awareness on the field. Often I'm never thinking that many steps ahead.  Sometimes when I break the line, when I break through I can't go again. Suppose like that one time in Shanghai in 2014, I gave the ball to Wang Zhenfeng and he broke through, and then no one followed him.  I stood there looking at him run from where I passed the ball, so tired I couldn't do anything... as soon as I realised I was like ``Oh! Quick get after him! In the end were a step away, only one step away from losing the ball.
\end{quotation}


\begin{quotation}
  在场上连续意识. (4) 我总是想的不是那么多翻。有时候突破的时候,突破了不会再打。假如在上海14年我给王贞丰突破了,然后没人追上了,我在原地累得不行了看着,我一看,“哦!”快追上!最后差一步,就差一点会丢球。
\end{quotation}






Cui Shuocheng:
:刚开始进队的时候不会打,经常打TOUCH看他们玩,觉得特别帅。当时我不会,当时就觉得特别困难,什么时候插上,什么时候穿球,意识方面老断老断,老停,跟你们老外没法比,你们从小就有这个意识,而我们半路出家,由个人项目到团体项目,觉得不了解 (2) 就愁那种刚开始练,打TOUCH的意识脑子老中断,打个重叠不知道什么时候传球,刚开始的时候就翻这个… (对于年轻的队员)对对,拿球的时候不知道什么时候要传还是不传





LJX:
“The tactics/moves, when playing a game and you’re doing a move (in attack), coordinating/passing, and when in the situation when you don’t know what to do; sometimes someone will break the line and then not know what do.” (2) PERFORMANCE related ANXIETY, DIFFICULTY PROCESSING/THINKING: “Yes, I feel pressure.  When I first started and even now, there was once, right before a Tournament, we were training at high intensity and ZPH , sometimes he says something, and as soon as I'm tired I can't process it, that feeling that "I know what I am supposed to do, but I can't actually do it.

战术、打比赛的时候打一些战术,打配合,不知道咋办的情况, 有时候一个人出了突发状况就不知道怎么办了。 (2) ?那你训练的时候会感觉有很大的压力吗?
:有
?这些压力会影响你发挥吗,让你更做不出来吗?
:会。刚练的时候,现在也是,有一段时间,快打比赛前强度特别大然后朱导,有时候(教练)说什么一累就都听不进去,知道该干啥但是做不出来的那种感觉。

Feng Yang:
能在不同的时间、不同情况下,正确处理这个球。如何掌握好。有判断的要求 



 




\myparagraph{``I have become a social animal''}


Of the group of 26 athletes who I included in the final analysis, 17 had transferred from other sports: 15 from athletics, one from football and one from basketball.  The remaining nine athletes had no specific background in sport.  When I asked athletes what rugby in particular had given them, an overwhelming response was ``team awareness'' (团队意识).  For the majority of athletes who came from the individual sport of athletics (and the athletes with no prior team sport background), the rugby team was a distinctly novel experience, involving a suite of new social norms and requirements associated with group membership.


Lian Jianxiang:
\begin{quotation}
 I first I wasn’t used to it, and then later as began to interact with my teammates (brothers-shixiong) I discovered that it was more or less the same as my previous (athletics) team, but just more cohesive (tuanjie).  But there was still a difference, for example in an individual sport you need to manage yourself and that's all, whereas team sports you need to consider more, you need to consider a lot of things that relate to everyone...Now I am used to it. I like this feeling (of team sport membership) more, its so much better than individual sport, in an individual sport its just yourself, its to independent, this (rugby) is a collective  (a family) , isn’t it
\end{quotation}

  \begin{quotation}
    转到团体项目有什么感受,学到什么新的东西?
   :之前不适应,后来和师兄一接触发现和我之前的队也差不多,要团结。但还是有差别的比如说之前个人项目管好自己就行了,团体项目考虑的比较多,要考虑好多大家一起的东西
   ?练了有一年的时间,现在习惯了吗?习惯这个感觉吗?
   :习惯了。更喜欢这个感觉,比个人好太多了,个人只不过自己,个人项目太独立了,这个是个大家么
  \end{quotation}


Unruly undergrad Fang Chao explains how he was transformed by rugby and the team:

  \begin{quotation}
    I think before when I was doing athletics, compared to now, I think I am a completely different person.  When I first came into contact with rugby, before I was doing an individual sport, now this is a team sport. I've changed a lot in terms of my personality, before when doing athletics I thought I was very independent and self-reliant, my own person.  Now I am one person who needs to communicate a lot with other people, cooperate; I have become a social animal.
  \end{quotation}


    \begin{quotation}
      我觉得之前练田径,跟现在橄榄球比,觉得我完全不是同一个人了。橄榄球刚开始接触的时候,之前田径是个人项目,现在是团队项目。性格方便改变很多,之前练田径是觉得我是个我行我素,自己一个人,现在我一个人还需要和别人多沟通,合作,变成群恤动物
    \end{quotation}


  SHW:
  \begin{quotation}
    ...I think it's mainly this thing of having teammates. Before, when I was training for an individual sport, it was just me training by myself. [In that environment] it was a case of whoever trained well was successful.  But now with this team of brothers, elder teammates will take care of younger teammates. We all train together, and if you can’t do something, you can always ask your elder teammates...
  \end{quotation}

    \begin{quotation}
      :我觉得主要是师哥师弟的这一块儿,原来练个体项目都是自己练自己的,谁练好了谁厉害,但是现在师哥师弟,有师哥照顾师弟带着,互相练,我不会我可以问师哥
    \end{quotation}








\myparagraph{Performance related anxiety (predominantly Junior)}


Many newcomers are transitioning from individual sports over to team sports...as such there is an unfamiliarity with not only the specific joint action demands of rugby, but also the norms of group membership... an ``awareness'' of the norms and expectations of group membership at the rugby program at XNT

Su Hailiang:

:我觉得最困难的,都挺困难的,当时什么都不会,身体素质也没那么好,没有那么强壮,也没有那么好,身体素质。基本功、球技、记战术什么的 


P2: %Anxiety included reference to specific components of performance
          %TEAM: attack, defence, support play, communication
          Some examples from interviews?


\subparagraph{Individual Social Shame (negative violations)}
    %Anxiety around individual letting the team down due to individual mistakes

    %Anxiety included reference to specific components of performance
      %IND: contact, tackle, passing, decision making, and support play in attack






\myparagraph{Strategy in individual performance and deflection of responsibility (more predominantly senior athletes)}

    \subparagraph{Team Awareness: Agency over, deflection of responsibility}
 %Deflection of responsibility towards junior athletes: criticism of junior athletes for not committing, playing computer games, being complacent.  (Irony that LP was one of the most vocal when discussing computer games at the dinner table).
 %Others more generous and circumspect: WW and CSC realise that it's a progression, and that individuals
 %EXPLAIN: reduction of dissonance via deflection

  \subparagraph{Individual performance: strategy}
 %MHT, CSC, HXL, WWX, MC use of experience to strategically avoid over-exertion and injury risk.




\myparagraph{Survey Results}

  SURVEY RESULTS: relatively equal levels of flow overall, but higher performance related anxiety in junior athletes, particularly in the scratch matches, which required high levels of technical competence and joint action coordination.

  %EXPLAIN: perception of performance in relation to social expectations of the team: joint action participation, as well as individual responsibilities (team member, self-determination)

  %Senior athletes talk with more composure regarding performance, any deflect responsibility for being the agent of team performance, and show strategy regarding how to regulate energy expenditure.







\myparagraph{Potential Mechanism: Positive Violation of expectations Performance related exhilaration and generalised emotionality (predominantly Junior)}

                %SWH story: the buzz and glow from acquisition of skill

          \subparagraph{Ind components}
                %WZF: explanation of the first side-step
                %YC: Fending - the feeling of picking up
          \subparagraph{Team performance}
                %WZF: likens team performance to side-step
                %HXL: rugby wasn't interesting (wasn't motivated until he started to pick up the team dimensions)

          \subparagraph{Individual Social Shame (negative violations)}


















        \subsubsection{Team Click}


  Junior Athletes:

  Either:
          \myparagraph{Generalised emotionality, equating click with bonding (Junior)}

  unspoken understanding linked with cohesion (tuanjie)

  Or:
          \myparagraph{inability to conceive of click, not qualified to talk about it}


  Either:
          \myparagraph{Familiarity and granularity/precision (Senior)}
          %HXL: Aura/atmosphere of the teams
          %WW: flow description
          %WZF: consideration of all the individuals
          %CSC:
          \myparagraph{Problematised}
            %Juniors for not being committed (Undergrad loafers)
            %Coach/Leadership for not supporting (LP)
            %China:
              %the system - No initiative, Chinese society too complex
              %the culture - Confucian education

              %Nostalgia for old regime pre 2013: HXL, LP,
            %E: a cognitive strategy in reorganising information in a way that maintains coherence (Nowak 2017)



    \subsubsection{Social Bonding}


      \myparagraph{Emotional Support}
      %MC, LZS, SHW, GJP

      Yang Can:
      \begin{quotation}
        Because rugby is a team sport, you know. We’re all brothers, it's the same as fighting, we’re born brothers and we’ll die as brothers you know, like a family, all together, thinking about how to share the responsibility. Its really good...I had never experienced this feeling before.  At that time at primary school there weren’t any rules, you’d just muck around or at the end of school just leave straight away, there wasn’t this type of feeling there.
      \end{quotation}


      \begin{quotation}
        因为橄榄球就是团体项目嘛,都是兄弟,像打架一样,出生入死得兄弟嘛,就跟一家人一样,互相,多替别人考虑分担,挺好的...没有过这种感觉,那时候小学没有什么规矩的,直接玩或者放学就大直接走了走了,没这种感觉
      \end{quotation}



PQY:
``I think that through rugby you can meet friends, you can train your awareness of social unity, and this sort of awareness of hard work, toughness, and struggle. I think all these things are very…for example every time you run fitness, at the time it is incredibly tough, but once you’ve finished your mood is extremely cheerful, even though you are very exhausted, its still really happy.''

我觉得通过橄榄球能交朋友,能锻炼自己的团结的这种意识,刻苦,艰苦,奋斗的这种意识,我觉得这些都很。。。比如说每次跑体能啊,当时特别辛苦,跑完之后心情非常愉快,虽然跑得很累,但是跑下去很愉快的感觉。 






\myparagraph{Shared Goal}
      %WW:



\myparagraph{Identity Fusion}
      %Fun/interesting/compelling (juniors)
      %Attachment: MXK, LZS,
      %Fusion: HXL, LP,


\subsubsection{Moderator Variables}

        \myparagraph{Technical Competence}


        \myparagraph{Personality}


        \myparagraph{Injury}


        \myparagraph{Fatigue}


                                                          \end{CJK}
