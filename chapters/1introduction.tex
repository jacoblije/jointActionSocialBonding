populationsfacilities% Confirmation of Status Documents
%----------------------------------------1.Introduction -------------------------------------------%
%-------1.0.Overview --------------------------------%
%-------1.1 Social Cohesion--------------------------%
%           1.1.1 Shared Cultural Practices
%           1.1.2 Social Bonding
%           1.1.3 Unifying theories of exericse and social bonding
%-------1.2 Phylogenetic and Ontogenetic Evidence----%
%           1.2.1 SC Model of cognition
%           1.2.2 Neuro, Developmental, Behavioural:  Mimicry, Synchrony
%           1.2.3 Contextual dynamics: social resources & attention
%                   Relational / categorical self-construal
%                   Attentional biases towards higher status individuals (order parameter: status)
%-------1.3 Entrainment & Prediction ------------------------------%
%            1.3.1 Prediction for effective movement
%            1.3.2 Emotions as movement residuals
%            1.3.3
%-------1.4 Dynamic Systems--------------------------%

%-------1.5 Sport------------------------------------%
%-------1.6 China------------------------------------%
%-------1.7 Hypotheses / Predictions ----------------%
%---------------------------------------2. Ethnographic Setting -----------------------------------%
%---------------------------------------3. Survey Study -------------------------------------------%
%---------------------------------------4. Thesis Outline -----------------------------------------%
%---------------------------------------5. Discussion/Conclusion ----------------------------------%




%----------------------------------------1.Introduction -------------------------------------------%
%-------1.0.Overview --------------------------------%

\documentclass[12pt]{report}
\usepackage[utf8]{inputenc}
\usepackage{graphicx}
\graphicspath{{images/}}

\usepackage[british]{babel}
\usepackage[backend=biber,style=apa,natbib=true]{biblatex}
\addbibresource{references.bib}

\title{
{Joint-Action and Social Bonding in Chinese Professional Rugby Players}\\
}

\begin{document}

\maketitle{}

The key theoretical assertion of this dissertation is that lower-cognitive processes relating to coordinating physical movement in highly interactive social systems must be more fully incorporated into a cognitive and evolutionary anthropology of social cohesion. To explore and substantiate this assertion, I conduct research with professional Chinese athletes who participate in the competetive interactional team sport of rugby union.  In addition to a detailed ethnographic description of observable processes of skill acquisition and group membership, I examine the specific relationship between joint-action and social bonding.

The human capacity to cooperate within stable and cohesive social groups is a fundamental feature of our species' evolutionary success. It is surprising, in which case, that sport - a cross-culturally ubiquitous organiser of modern social life - has not been more intensively studied for evidence concerning the cognitive and evolutionary foundations of human sociality.  An integrated scientific study of the social, psychological, and physiological mechanisms associated with participation in sport (broadly construed), and the ecological dynamics within which these mechanisms are embedded, can offer novel insights into the science of cognition, cooperation, and evolution.

What makes the study of sport (and other exertive and coordinated group activities like music, dance, and some forms of ritual) so compelling, is the way in which its highly visceral nature reminds us that the cultural practices that populate and bind human societies together are at their root embodied phenomena.  Culture is always enacted in a specific time and place, by specific minds, bodies, and physical environments (SOURCE).


Existing cognitive and evolutionary accounts of human sociality emphasise the role of shared cultural representations and evolved cognitive mechanisms that enable collective adherence to and widespread transmission of representations throughout populations.

sport from the point of view of cognitive science, is that
the highly \texit{visceral} nature of exertive and coordinated group activities such as sport (but also activities like music and dance) serves to expose the fact that shared cultural practices are not purely symbolic representational artefacts. Instead, shared cultural practices are at their root embodied and enacted processes, participation in which entails regulation of vast transfers of implicit or ``pre-representational'' information that cannot be fully captured by symbolic representation.

Evidence from social neuropsychology suggests that behavioural outputs are the product of a complex interaction between pre-declarative ``bottom-up'' cognitive mechanisms responsible for regulating the synchrony of the organism with the environment, and an ensemble of top-down cognitive mechanisms responsible for goal-directed action.  Despite the salience of top-down goal directed mental processes to our experience, i.e., human consciousness, the information represented through such processes appears to be highly selective - a mere ``highlights reel'' showing only a small fraction (~2\%?) of the entire volume of informational transactions in which the nervous system is constantly engaged.  Time-locked coordination of behaviour with other individuals and/or the physical environment requires continual regulation of behaviour in accordance with task- and environment-specific affordances and constraints, and the lower-cognitive mechanisms responsible for these regulatory processes appear to function often largely unmediated by explicit representational schemas.


% existing (representational) accounts of social cohesion: DIT & CAT
Existing cognitive and evolutionary accounts of human sociality emphasise the role of shared cultural representations and evolved cognitive mechanisms that enable collective adherence to and widespread transmission of representations throughout populations.  While controversy surrounds the precise mechanics of how cultural practices spread and fixate in populations, there is general agreement within cognitive science that a capacity for high-fidelity imitation of mental representations enables faithful preservation of cultural practices within groups, and vertically over generations \citep{Henrich2007}. Under this assumption, humans cohere around shared cultural practices, which evolve via a step-wise, cumulative process known as the ``ratchet effect'' \citep{Tomasello1993},
% whereby cultural activities fixate in populations before natural variation innovation leads to further proliferation and fixation in a stepwise manner.
On the most basic level of behavioural coordination, mutually shared action-oriented mental representations enable humans to simulate and commit to complex multi-agent joint-goals, including “everything from bi-directional linguistic conventions to social institutions with their publicly created joint goals and individual roles that can be filled by anyone”  (Tomasello et al., 2014; p. 190).  Dual Inheritance Theorists suggest that over time, cultural environments that favoured the prosocial and punish the anti-social have influenced genetic evolution in humans, in a process referred to as gene-culture coevolution. In this account, humans have evolved a suite of cognitive mechanisms—a ``norm-psychology''—that enable the persistence of those groups with norms and institutions that create group-beneficial cooperation among large scale, non-kin social actors \citep{Chudek2011}.

These various strands of research form a prevailing ``representational'' account of social cohesion, which relies on the assumption that shared mental representations enable the formation of proximate and stable social groups.  This account of social cohesion has been extremely productive in explaining macro-scale patterns of human sociality and social cohesion, including the emergence of large-scale religious and agricultural practices.  The contention with reperesentational accounts of social cohesion, however, is the fact that they remain largely agnostic to the processes through which explicit mental representations are formed and transmitted. Alternative accounts of human sociality assert that humans do not merely aggregate according to shared mental representations, as appears to be the case with many other social organisms such as insects and fish. Instead, humans display a distinct drive to actively \textit{congregate} with conspecifics (Dunbar & Shultz 2010).  On the ground-level of behavioural coordination, for instance, the cognitive capacity for sharing and adhereing to complex multi-agent shared goals could depend crucially on the quality of lower-cognitive affective mechanisms of psychological alignment between individuals and within groups (SOURCES).  In addition, there is increasing evidence to suggest that ecological system dynamics constrain and enable social cognition in non-random ways, and as such must be incorporated into a causal account of social interaction (Marsh et al. 2009).  Taken together, these streams of research serve to challenge models of social cohesion that commit to the assumption that shared cultural practices hinge on the transmission of explicit mental representations. Further research is required in order to incorporate an appreciation of lower-cognitive affective mechanisms and the system dynamics in which these mechanisms are implicated into a cognitive and evolutionary account of human social cohesion.

On the surface level, prevailing representational accounts of social cohesion are very relevant to understanding the cognitve and evolutionary significance of sporting practices, which rely heavily on explicitly shared ``recipies for action''(Buskell 2017, i.e., rules of the game or prescribed technical competencies) for their widespread existence and proliferation. At the same time, however, the highly \texit{visceral} nature of exertive and coordinated group activities such as sport (but also activities like music and dance) serves to expose the fact that shared cultural practices are not purely symbolic representational artefacts. Instead, shared cultural practices are at their root embodied and enacted processes, participation in which entails regulation of vast transfers of implicit or ``pre-representational'' information that cannot be fully captured by symbolic representation.  Evidence from social neuropsychology suggests that behavioural outputs are the product of a complex interaction between pre-declarative ``bottom-up'' cognitive mechanisms responsible for regulating the synchrony of the organism with the environment, and an ensemble of top-down cognitive mechanisms responsible for goal-directed action.  Despite the salience of top-down goal directed mental processes to our experience, i.e., human consciousness, the information represented through such processes appears to be highly selective - a mere ``highlights reel'' showing only a small fraction (~2\%?) of the entire volume of informational transactions in which the nervous system is constantly engaged.  Time-locked coordination of behaviour with other individuals and/or the physical environment requires continual regulation of behaviour in accordance with task- and environment-specific affordances and constraints, and the lower-cognitive mechanisms responsible for these regulatory processes appear to function often largely unmediated by explicit representational schemas.

%Entrainment, dual systems,
Grasping the relevance of basal lower-cognitive mechanisms to macro-scale patterns of social cohesion requires a conceptualisation of human cognition as a process that evolved first and foremost for regulation of physical movement. Social cognition, in turn, can be understood as the regulation of movement in systems made up of interacting individuals.  Establishing interpersonal synergies with one or more individuals serves the adaptive purpose of maximising extraction of fitness-relevant information from an informationally complex or \textit{uncertain} social environment. Interpersonal synergies constrain the computational complexity of social interaction by reducing the degrees of freedom of the component parts of the system, and allow for dynamic reciprocal informational transfer between individuals (Riley et al. 2011). Interactional synergies between individuals and within groups allow for cognitive resources to be efficiently distributed across neural, bodily and environmental features of a dynamical system (Semin & Cacciopo 2008).

Evidence from the fields of neurocognitive, developmental, and comparative psychology suggest that humans have evolved a suite of cognitive mechanisms, whose function is not simply imitation and transmission of relevant cultural representions, but rather establishing and dynamically regulating interpersonal synergies.  In particular, there is a converging consensus that human social interaction is reliant on an interaction between two prevailing modes of congition for the production of behavioural outputs.  On the one hand, individuals utilise an ensemble of perceptual and regulatory mechanisms that faciliate adaptive synchronisation with affordances in the environment.  On the other hand, a higher-order, goal-directed selective response system serves to produce behaviours that are differentiaed from, and complimentary to, bottom-up synchronisation processes. When the interaction of monitoring and decision making processes of two or more individuals give rise to time-locked and complementary independent motor responses in a task-specific environment, behavoural entrainment is achieved and synergies become functional and adaptive units of social organisation.

%prediction
The initiation and maintenance of adaptive behavioural synergies necessitates, on the level of individual psychology, an ability to successfully anticipate and predict the behaviour of others and related affordances in the environment.  General recognition of the fact that cognition in biological systems is embodied (not just located in the brain, but grounded in and constrained by intra-organismic dynamics) and embedded (distributed throughout an ecological niche containing inter- and extra-personal behavioural affordances), has encouraged a re-evaluation of the function of individual human cognition.  Individual level cognition is becoming increasingly understood less as a passive process of receiving and reacting to sensoral stimuli, but as a more active process of making predictive inferences in order to participate adaptively in distributed cognitive systems \citep{Marsh2009a}.  This view of cognition entails, on a very fundamental level, processing information and, in turn, moving in such a way as to reduce uncertainty or complexity in a task-specific cognitive system.

For example, imagine your office colleague were to suddenly throw a soft rubber ``stress ball'' in your direction without announcement. Your automatic response, to move your hand or hands into a position to catch the ball should be most accurately understood not as a discrete, volitional action in response to a stimulus external to your individual cognition. Rather, moving in order to catch or otehrwise interact with the in-flight ball can be more accurately understood as participation in a distributed cognitive system containing the ball, the office, your office colleague, as well as the higher-order mental representations that acknowledge the stress ball as an object to be caught (an not avoided, in the case that scissors were thrown instead!). The function of individual cognition, in this conception, is to faciliate physical movement in such a way as to reduce the uncertainty of the overall system-distriubted cognition.  Attempting to catch the ball is, in this specific instance, an adaptive response to the affordances and constraints of the cognitive system in which you find yourself.

A conception of individual cognition as a process of reducing informational uncertainty in task-specific environments, entails an interaction (outline above) between ``bottom-up'' processes of sensoral synchronisation with environmental affordances, and ``top-down'' representational schemas which frame possibilities for action.  While on-line synchronisation facilitates adaptation to specific affordances, it can also be costly and slow.  Top-down schemas for action, by contrast, can function as fast and effective heuristics for coordinating behaviour.

Evidence suggests that top-down schemas for action are utilised particularly in environments involving high uncertainty or complexity.
Importantly, heuristics for movement appear to be grounded in intra-organismic system dynamics \citep{Vesper2013}, and shaped by and phenotypic history—personality dispositions, cultural norms, learned competencies, life-history strategies, etc.  There is evidence to suggest variance in movement tendencies for movement according to socio-economic, cultural, environmental conditions (Cohen/Nettle).  At the same time, however, priors based on passed experience can be innapropriately matched to dynamic interactional enviroments subject to continual change, leading to maladaptive behavioural responses.  It is plausible to assume therefore that human cognition has undergone selection for a capacity to flexibly deploy representational priors as well as to react adaptively to proprioceptive informational inputs which vary from prior expectation.

Interestingly, the stress-ball example also raises the causal role of system-dynamics and constraints on distributed cognitions. The terms of the cognition were initiated by the ball-thrower; the receiver was forced to complete the cognition based on the limited affordances available to them. This system resembles a ``leader-follower'' relationship between two individuals, a common solution to coordination problems (SOURCE).  If both actors involved in the cognition were able to negotiatie and arrive at a shared action plan prior to intiation, the cognition and resulting behaviour could have taken on a different, more egalitarian, form (a coordinated ``alioop" into the rubbish bin, for example!).

%emotion
Recasting cognition as an embodied and system-distributed process has lead not only to a reevaluation of the function of individual action, memory, and decision making, but also to the reevaluation of the function of lower-cognitive affective mechanisms, also known as emotions.  Rather than being exogenous to inferential processes, as conceived by rational choice theory, emotions can be understood as superordinate programs for regulating dispirate subordinate cognitive modules for the purposes of global coordination with the environment \citep{Cosmides2000}.  In other words, emotions play a role in the regulation of movement, by signaling information concerning the quality of movement coordination in interpersonal synergies.   Positive emotions in a social setting function as positive feedback for successful coordination that accords with individual expectations.  Likewise, negative emotions arising from social interaction indicates a violation of expectations around coordination of movement.  In this sense, emotions generated in social coordination with others can be understood as residuals that perform a regulatory function by providing information that can be used to inform future prediction and coordination by reinforcing or challenging existing representational schemas for action or encouraging/discouraging synchronisation processes of attention and iner-personal affiliation.  Social bonding, in this sense, is an assemblage of emotions that derive from coordination of movement between individuals or within a group, in such a way that accords with the regulatory mechanisms, schema-derived expectations, and ecological constraints of the specific cognitive system.

As such, I predict that social bonding should be related to the extent to which athletes, in this case, feel that they have engaged in successful joint-action.  More specifically, social bonding should be related to the extent to which athletes feel the ``click'' of joint-action, controlling for external factors that could contribute to more explicit recognition of team membership, for example objective measures of performance including game outcome and points scored, team values and identity, and so on.



This leads to a discussion of emotional residuals and the idea of violation of expectations.


Competetive interactional team sports scenarios involve both high levels of technical entrainment and high levels of  contrived uncertainty, and could encourage the utilisation of explicit representational schemas to direct action at the expense of synchronisation-directed behavioural regulation.  The interaction of group size effects (5, 15, 25, 50, 150) and coordination heuristics (pure flow - co-confidence motion, egalitarianism, leader-follower, etc) with measures of system entropy could be an interesting avenue of investigation. In brief, evidence suggests that interactional team sports are environments in which the complexity, uncertainty, and intensity associated with forming interpersonal synergies with others demands greater interdependence among athletes, the downstream effects of which may relate to affective processes of social alignment and affiliation.

The perception of another's action induces an equivalent motor simulation in the observer, which constitutes a neural coupling or shared bodily state. Synchronisation and goal-mediated processes jointly exercise differential inhibitory and excitatory influences on the formation of action-oriented mental representations and motor responses.  The activiation of higher order goals in response to significant stimuli weakens synchronisation processes and thereby reduces the influence of synchronisation on motor behaviour outputs.  Likewise, in the abscence of a competing or complementary higher-order goal, there will be less inhibitory influence on synchronisation processes.
Explain.



%CAT response to DIT
Theoretical controversy in this domain surrounds the way in which cultural information transfer is modelled.  Dual-Inheritance Theory (DIT) suggests that the analogy of biological evolution is sufficient to model evolution of shared cultural practices. Cultural representations, like biological genes, are subject to copying, with variation in cultural practices due to copying error, mutation, and drift. This model is contested by proponents of Cultural Attractor Theory (CAT), who suggest that transmission of cultural representations involves transformation rather than preservation (copying), and that variation in shared cultural representations is due to the fact that human social interaction is subject to unstable ecological dynamics.  In this account,  the evolution of cognitive mechanisms implicated in cultural transmission is not strict imitation per se, but rather extraction of fitness-beneficial information from the social environment. CAT employs language borrowed from chaos theory in order to bolster a theoretical departure from DIT. The transmission of cultural representations, for example, should tend towards ``cultural attractors'' - a theoretical point in 2D space of possible cultural representations - and is ordered by ``factors of attraction'', which include evolved cognitive biases, etc.  Thus, whereas DIT proponents remain agnostic to the micro-dynamics of cultural transmission, evoking the language of dynamic systems theory (the language of attractors, equilibrium, etc) allows CAT proponents to suggest that cultural information is susceptible to non-random variation produced by the constraints of evolved cognitive mechanisms and specific social and environmental interactions.  While CAT still requires more detailed emprical substantiation, the claim is that a consideration of the micro-dynamics of cultural attraction have implications for macro (population) level distrubutions of cultural practices.

A clear theoretical inconsistency of CAT is the fact that the core unit of cultural transfer, the explicit cultural representation, remains unchanged from DIT.  Both DIT and CAT rely on traditional models of information processing that are symbolic, whereby cultural practices are abstracted in an amodal fashion and dissociated from sensory and motor bases.  In this sense, while the evocation of dynamic systems theory signals a step in the direction of a more nuanced understanding of cultural information transfer and associated processes of social cohesion at the population level, the persistence of a representationl thoerisation of cultural infomration may utlimately restrict CAT's progress.





In other words, consciousness represents only a small fraction of the entire volume of information available to, and processed by, the nervous system at any one time.




% social bonding & movement dynamics
The highly choreographed nature of behavioural interactions observable in sport, particularly in interactional team sports, demonstrate that well below the surface of explicit symbolic representational communication, basal pre-declarative cacpacities for motor regulation play a foundational role in processes of social alignment and affiliation (Marsh et al. 2009). There is convincing evidence to suggest that humans have evolved neurobiological reward mechanisms to faciliate establishing and maintaining interpersonal synergies.  Indeed, experimental research involving interpersonal linguistic exchange, behavioural mimicry, behavioural synchrony, and joint-action suggests a reciprocal relationship between successful coordination of behaviour and psychological rapport, liking, and affiliation (SOURCE).  Relatedly, participation in rigorous coordinated activities such as music, dance, laughter, and other types of joint-action activates lower-cognitive affective mechanisms which in turn faciliates psychophysiological states amenable to social bonding.

Thus, researchers have suggested that human social cohesion is not purely a case of individuals aggregating via explicit rule-based cognitive programs, as game-theoretical models would suggest. In addition to explicit representational schemas, humans exhibit tendencies to actively \textit{congregate} by utilising a suite of basal lower-cognitive mechanisms associated with capacities for movement regulation.

%Social bonding literature here?







%the mystery of team click
One of the big mysteries of competetive team sport, particularly at the elite level, is the elusiveness of peak team performance.  While the individual components of the system, the athletes, may all exhibit expert-level competence in sport-specific skills, the achievement of group flow or ``click" can often disappear as abruptly as it arrives, if indeed it arrives at all.  Achieving team click also appears to often be decoupled from achieving consensus in explicit common knowledge, for example publicly transmitted representations around team identity, values and so on.  However, it does also appear that high consensus around explicit representations could function to stabilise cognition of the system, which could be of service to key performance variables (UCM approach - need to expand).  The elusiveness of team click  suggests that we must expand our understanding of the components relevant to performance in team sport to include not only individual cognitive mechanisms, but also the system dynamics in which individual cognitions are embedded and by which they are constrained.


% where: China

It should be expected that the cognitive processes relevant to social interaction outlined above will be subject to cultural, as well as intra- and inter-individual variation. Depending on the specific context, or specific onto- or phylo-genetic specifications of individuals or popualtions, individuals and populations will vary in the extent to which either lower cognitive processes of synergy formation and maintainence, or higher order decision making processes feature in the interaction for the production of action-oreinted cognitive representations.  This variance could then be identifiable in  the extent to which explicit representations that fixate at the individual or population level include contents relating to either synchronisation ("relational") cognitions, or higher order goal-directed (("categorical") cognitions.  One example of this variance in social cognition, on an individual level, is the autism-psychotic spectrum (Marsh et al. 2009); on the cultural level, the example of most relevance to this research is the psychology of self-construal and processes of group membership. There could also be sex differences in these cogntitive processes for which testable hypotheses concerning motor coordination tendencies and social cognition could be generated (Cohen 2007).

Cultural variation in social cognition is particularly relevant to the current research, the empirical data for which were conducted in China, with Chinese athletes.  China, much like Japan (more thoroughly studied by social and cultural psychology), is a site in which ``relational self-construal'' and relational modes of group membership are prominent and explicit cultural representations. The social psychology of relationalism in East Asia can be traced to ancient Chinese philosophical roots, with Daoist notions energetic harmonisation, and (much later) Confucian and Moist notions of self-cultivation for kinship and social harmonisation.  From these deep roots, cultural representations promoting a relational mode of group membership permeates manifold cultural practices in China: child-rearing practices of teaching and language acquisition, to nationalistic metaphors all emphasise relationalism, particularly the central metaphor of the individual being embdedded in familial relations to nuclear family, workplace, and nation-state (大家庭, 国家,等).

Nisbett (SOURCES) is well known for demonstrating cross-cultural differences in attention, perception, and cognition.  Nisbett shows clear attentional and perceptual differences between East and West. In a series of experiments, Nisbett and colleagues show evidence which suggests that East Asian subjects appear to attend more to the relational details of (social) stimuli, whereas Western subjects attend more to the explicit cotegorical dimensions of (social) stimuli.  In an attempt to contextualise these findings in his book, Nisbett proposes a socioecological account of cognition in which the metaphors of Ancient Greece (the market place) and Ancient China (the feudal court) serve as the idiosyncratic cultural sources of divergent modes of cognition now rendered as an ``East-West'' dichotomy in modern social psychology.

It has been shown in social and cultural psychology that these two predominant modes of group membership—categorical and relational—not only vary across cultures (i.e. East Asian versus Western European or North American, see Markus & Kitayama, 1991; Nisbett et al., 2001; Yuki, 2003), but also within groups (according to sex and personality differences Yuki & Takemura, 2014), and within individuals (depending on contextual and situational primes Lee & Jeyaraj, 2014; Wong & Hong, 2005).  More recent avenues of inquiry have focussed on the ways in which processes of group membership and self-construal are modulated depending on the availability and location of social resources. Perception of ``relational mobility'' has been shown to statistically explain (mediate) cultural differences in self-enahncement, self-disclosure, general trust, and shame proneness, and in-group/out-group attentional biases.  Environments with high relational mobility provide people with an abundance of opportunities to meet strangers and create new relationships and groups.  Societies in low relational mobility, by contrast, usually involve people maintaining comitted and long standing relationships with specific others in order to help reduce social uncertainty, such as risks of being cheated (SOURCES).





%SPORT:
Sport magnifies the importance of competence for social bonding
While not (typically) entailing self-immolation or firewalking, the specific exercise context that forms the focus of this research, “rugby sevens,” demands of its athletes willing participation in continual bouts of intense physical activity at and above the anaerobic threshold, which includes sprinting, collisions, and tackles separated by short bouts of low intensity activity such as walking and jogging.  At an elite level, in particular—the level at which my research is focussed—these physiological costs are amplified to the extreme (Takahashi et al., 2007). In addition, in order to achieve desirable levels of performance, athletes face the challenge of coordinating joint action, a task that requires interdependent commitment to the team (Carron et al., 2002).  Ritual-induced pain and acute exercise are similarly arousing physiological stressors that reliably activate neurocognitive mechanisms of attention, energetic resource allocation (Dietrich & Audiffren, 2011) and pain and reward (Boecker et al., 2008; Raichlen et al., 2012).

In particular, the extreme specificity and technicality of behaviours that an athlete must learn and execute in complex and high-pressure interactive environments, magnifies the importance of behavioural competency (and not just pro-sociality) for gaining access to the benefits of group membership (Macfarlan & Lyle, 2015).  While successful performance of extreme ritual requires learning certain ritual-specific behavioural competencies (Atran & Henrich, 2010), the complexity and specificity of these behaviours pale in comparison to those required in (elite level) group exercise contexts). The time- and rule-bound exercise environments of competitive interactive sports like rugby heighten pressure on individuals to reliably produce group-relevant technical behaviours in order to maximize individual and team performance.  Thus, while acute exercise may reliably activate and modulate neurocognitive mechanisms of pain, reward, and attention, the extent to which neurocognitive processes translate to positive psychological states and/or pro-social behaviours necessary for social bonding, may crucially depend on feedback information on the quality of technical performance of exercise-specific behaviours.

According to the “reticular-activating hypofrontality” (RAH) model of acute exercise (Dietrich & Audiffren, 2011), acute bouts of exercise activate evolved mechanisms of neurocognitive resource allocation designed to optimise behavioural responses to environmental conditions.  The biocomputational complexity of physical movement, coupled with the limits of the brain’s metabolic resources, forces a flexibility/efficiency energy trade-off between explicit and implicit modes of cognition.  According to the model, following an initial phase of exercise-induced alertness and arousal (facilitated by monoamine (adrenaline and dopamine uptake in key perceptual and decision-making regions of the prefrontal cortex and amygdala), sustained exercise forces preferential allocation of resources to fast and frugal implicit cognitive processes responsible for selecting and sustaining this movement, at the expense of more flexible explicit cognitive processes.  Under the heavy metabolic burden of acute exercise, an energetic trade-off forces the down regulation of higher-order structures of executive function such as the prefrontal cortex, which reduces the role of explicit cognition in subsequent phenomenology and decision-making processes (Dietrich, 2004).   This model of exercise-modulated cognition suggests that down regulation of explicit cognition during exercise could differentially influence individual performance of group-relevant behaviours, depending on extent to which these behaviours rely on implicit or explicit cognitive systems for their successful execution.

From this model it can be predicted that exercise-induced mechanisms of pain, reward, and attention modulation will generate psychological states necessary for social bonding in athletes for whom group-relevant behaviours are predominantly processed by implicit cognitive processes, such that their performance is not diminished by exercise-induced down regulation of explicit cognitive processes.  More competent athletes, who have already mastered (i.e., made implicit) the technical and social behaviours relevant to a specific exercise context, will be more able to perform these competencies immediately and automatically in complex interactive environments, thus gaining access to group-level benefits.  Alternatively, the performance of less competent athletes, who have not yet made implicit and automatic the social and technical skills required by a given exercise context, may instead suffer from down regulation of executive function (Dietrich & Stoll, 2010; Kane & Engle, 2002).

Thus, just as Fischer and Xygalatas identify pain as a key social technology that provides selective advantages to individuals vying for resources afforded by successful group membership, social and technical competencies may operate in a similar fashion via similar neurocognitive pathways. My ethnographic context is ideal for analysis of this mechanism in group exercise, as the athletes that I study vary significantly in their ability to effectively perform the specific skills of rugby, from those who have only just begun learning rugby (training age 0-2 years, n= 13) to those more seasoned athletes (training age 2-10 years, n=13).

Socio-ecological factors
The high-intensity, high-impact, and team-based nature of rugby makes it at first sight somewhat incongruent with China. It is rugby’s Olympic status (rugby sevens), and the Olympic-logic of the state-sponsored Chinese sports system, which dictates that hundreds of Chinese men and women play rugby professionally and semi-professionally in over ten different provinces.  The obvious individual costs associated with this dangerous and “cold-door” (lengmen 冷门) sport are mitigated by potential benefits of education and employment offered to athletes by a state-run sports system.  Throughout the 20th Century, sport, exercise, and physical culture (tiyu 体育) became a (if not the) primary pedagogical vehicle for encouraging participation in activities of the modern nation-state.  As such, physical culture in China choreographs—perhaps more explicitly than any other facet of contemporary Chinese life—the interaction between imported and traditional modes of group membership at the psychological heart of Chinese modern history.  My ethnographic setting implies a complexity in terms of processes of group membership, for which existing measures of social bonding used in cognitive and evolutionary anthropology fail to fully account.

Psychological theories and measures commonly used to show evidence of social bonding, namely “group identification” and “identity fusion,” presuppose a mode of group membership based on an abstract association between categories of the self and the in-group.  The “modes theory” of religious ritual and social cohesion (Whitehouse, 2004), for example, delineates two divergent modes of social bonding, “group identification" and "identity fusion” produced by varying forms of religious ritual.  Whereas high-frequency, low intensity rituals (such as daily prayer or church attendance) generate a form of minimal affiliation effective for flexible and strategic social exchange (group identification), low-frequency high intensity rituals (initiation rituals, for example) generate durable, familial, and agentic bonds between individuals and within groups (identity fusion, hereafter “fusion”) (Whitehouse & Lanman, 2014).  Fusion is differentiated from group identification according to its felt, emotional and personal agentic qualities (Swann, Jetten, Gómez, Whitehouse, & Bastian, 2012). However, both constructs ultimately emerge from social psychology’s self-categorisation paradigm (J. C. Turner, Hogg, Oakes, Reicher, & Wetherell, 1987), which entails psychological processes of identification between abstract categories of the self and the in-group.  This categorical mode of group membership occurs when the perceived differences between the self and other in-group members are smaller than the differences between in-group and out-group members (Yuki & Takemura, 2014).

However, as Anthropologists have emphasized for some time (Xiaotong, 1992), and as cultural psychologists are now beginning to demonstrate in experimental paradigms, group membership is not universally expressed in categorical terms (Markus & Kitayama, 1991; Nisbett, Peng, Choi, & Norenzayan, 2001).  In distinction to a categorical mode of group membership, relational group membership involves attention to the maintenance and harmonization of intragroup relationships more so than intergroup categorical comparisons (Yuki, 2003). Social identity is conceived of less as a function of distance between abstract categories of self and in-group, and more as a degree of commitment to cultivating a self-centred and self-enhancing network of hierarchically structured relationships (J. Liu, Li, & Yue, 2009; Nisbett, 2003).  It appears that two predominant modes of group membership—categorical and relational—vary across cultures (i.e. East Asian versus Western European or North American, see Markus & Kitayama, 1991; Nisbett et al., 2001; Yuki, 2003), within groups (according to sex and personality differences Yuki & Takemura, 2014), and indeed, within individuals (depending on contextual and situational primes Lee & Jeyaraj, 2014; Wong & Hong, 2005).  While the durable persistence of cultural and linguistic institutions appear to be responsible for the prominence of one mode of membership over another (East versus West, for example), recently researchers have suggested that divergent modes of group membership—categorical and relational—may be mediated by context-specific socio-ecological factors such as the level of relational mobility in any given environment (Oishi & Graham, 2010; Takagishi et al., 2014; Yuki, Maddux, Brewer, & Takemura, 2005).  Both modes of group membership have been shown to shape attention, cognition, and behaviour (Nisbett, 2003), and as such have important implications for the theorisation of proximate mechanisms and ultimate explanations for social bonding (in exercise).

Experimental evidence suggests that categorical group processes facilitate fast and effective identification with arbitrary minimal groups (Diehl, 1990; Van Bavel, LM, & Xiao, 2014), the arousal of intrapersonal cognitive dissonance between the self and experimentally constructed in-group (Festinger, 1957; Stone & Cooper, 2001), higher levels of cooperation with categorically similar strangers in economic games (Yuki et al., 2005; Yuki, 2003), and greater attention to and memory recall (Buchan, Johnson, & Croson, 2006; Ng, Steele, & Sasaki, 2016).  In cultural environments where relational processes of group membership are more prominent or salient, on the other hand, the inverse is usually observed. First, minimal group paradigms have had very little (if any) success in East Asian (particularly Japanese) contexts (J. Liu et al., 2009).  Relational group processes, on the other hand, allow for the arousal of cognitive dissonance only when it is constructed interpersonally (as opposed to intrapersonally) between an individual and specific individuals to which that individual is connected by a meaningful social relationships (Hoshino-Browne et al., 2005).  Likewise, individuals with a predominantly relational group consciousness are more willing to cooperate with and attend to strangers with whom they share relational rather than categorical ties (Ng et al., 2016; Yuki et al., 2005).

Relational group membership has a long history in China, rooted in Confucian (Hwang, 1999), folk-cultural axioms (Wang & Dai, 2009), agricultural modes of production (Talhelm et al., 2014; Xiaotong, 1992), dynastic rule, and modern reinventions of these cultural forms by processes of the nation-state (J. H. Liu, 2014).  The last 150 years of Chinese modern history has also entailed the introduction of categorical group processes associated with the activities of the nation-state, leading to a contemporary Chinese indigenous psychology in which a relational mode is predominant, and a categorical mode is contextually-activated, especially when categories such as “China” are threatened or challenged internationally (J. Liu et al., 2009).

First, the athletes I study are young Chinese men, predominantly from rural areas of China’s northeast, and are therefore subject to relational modes of group membership made predominant and durable by persistent cultural and linguistic processes associated with group membership in contemporary China (J. Liu et al., 2009).  Second, athletes are members of a relatively small and stable team (n=26), for which access to benefits should require attention to the maintenance of productive intragroup relationships, more so than processes of intergroup mobility or comparison (Schug, Yuki, & Maddux, 2010). Thus, I predict that members of the Beijing men’s rugby team will exhibit psychological tendencies relating to a predominantly relational mode of group membership.  However, given the fact that rugby is an imported team-based interactive sport, for which categorical modes of group membership are required and celebrated, I also expect athletes to exhibit a categorical mode of group membership when this mode is made salient (J. Liu et al., 2009).  Importantly, I expect pro-sociality and agency over the group (two key outcome variables of social bonding in exercise, identified above in the anthropology of extreme ritual) to be expressed in relational terms.  Pro-sociality will be expressed as attention to the maintenance and harmonisation of intragroup relationships, whereas agency of the group (or fusion) will be expressed in terms of self-focused self-enhancement



















At best, by-products of implicit information transfer occaisonally percolate up to the surface of consciousness, to be perceived as causally opaque emotional sensations or intuitions, and to be rationalised and appraised after the fact.

-- For example, the extent to which the ideal representation of rugby is performed (i.e. its assemblage of technical competencies, etc), the extent to which performance clicks as it is expected to, in line with explicit representations of ideal performance, depends entirely on the extent to which any given  dynamical system of athletes on any given day is able to effectively coordinate behaviour and reduce uncertainty


Interactive team sport clearly articulates a tension between lower cognitive tendencies for regulation and synchronisation, and the importance of idividuation, which becomes manifest in technical entrainment.  Athletes synchronise via highly representational templates for action.

%-------1.1 Social Cohesion--------------------------%
%           1.1.1 Shared Cultural Practices
The human capacity for culture is by far the most notable technological achievement of the homo genus since its branching from a shared chimpanzee ancestor at least 6 million years ago. Many anthropologists understand our cultural capacity as an adaptive solution to an evolutionary environment in which early humans were forced, perhaps due to ecological change, to “collaborate or die” (e.g., Tomasello et al., 2014; p. 188).  This environment selected for more sophisticated forms of cooperation between individuals and within groups, particularly the ability to derive, represent, and verify useful information from others about the natural environment (e.g., how to avoid predation) and the social environment (e.g., how to coordinate with others) (Chudek & Henrich 2011).

Experimental studies in comparative psychology have shown that humans display a precocious tendency to diligently imitate actions of trusted or authoritative others, even when the goal of the action is unclear (Tomasello et al., 2014).  The human capacity for high-fidelity transmission of representations enables the proliferation and fixation of cultural practices horizontally within groups, and vertically over generations. It is hypothesised that social cohesion emerges from this cumulative evolution of cultural practices, or “ratchet effect.” Combined with the related ability to infer and share intentions of others, humans are able to simulate and commit to complex multi-agent joint-goals, including “everything from bi-directional linguistic conventions to social institutions with their publicly created joint goals and individual roles that can be filled by anyone”  (Tomasello et al., 2014; p. 190)

%           1.1.2 Social Bonding
 Movement alignment and Social Bonding: Grooming Hypothesis

 Recent research on the affective dimensions of human social cohesion has devoted considerable attention to the way in which exertive, coordinated group activities activate psychophysiological mechanisms associated with forging and maintaining social bonds with fellow group members.

The capacity for social bonding is thought to have arisen in primates as an adaptive response to the pressures of group living.  Aggregating in groups serves to reduce threat from predation, but at the same time can be individually costly due to stress arising from interaction at close proximity, and conflict over resources among genetically unrelated individuals.  These pressures are hypothesised to have led to selection for social bonding (e.g., via dyadic grooming). The coalitional alliances that arise among close partners allow for the maintenance of the group by buffering the stresses of group living (Dunbar, 2012).  Primate social grooming is associated with the release of endorphins (a type of endogenous opioid), presumably leading to sustained rewarding and relaxing effects.  While other neurotransmitters such as dopamine, oxytocin and/or vasopressin may also be important in facilitating social interaction, it has been suggested that endorphins allow individuals who are not related or mating to interact with each other long enough to build “cognitive relationships of trust and obligation” (Dunbar, 2012, p. 1839).

It is thought that, as the homo genus evolved more complex collaborative capacities for survival in interdependent group contexts, grooming-like behaviours sustained social bonding in larger group sizes where dyadic grooming would take too much time (Dunbar, 2012).  Experimental studies suggest that neurophysiological mechanisms activated by activities that involve physical exertion and coordinated movement such as group laughter, dance and music making, exercise, and group ritual can bring groups closer together, mediated by the psychological effects of endogenous opioid and endocannabinoid release.  Endogenous opioids and endocannabinoids are neurotransmitters with analgesic, mood-elevating effects that have been implicated in mammalian social bonding. When these mood-elevating effects are experienced in a group they seem to lead to what Durkheim (1915/1965) described as collective effervescence, and, through embodying the each other’s affective experiences, result in more positive, trusting, and cooperative relationships among participants (Dunbar, 2012).

These studies give an empirical basis to anthropologists’ observations of the relationships between coordinated, exertive group rituals, collective effervescence, and social cohesion. Across cultures, anthropologists have described rituals involving group dance and music making as practices that lead to the sharing of positive mental states and feelings of ‘oneness’ with the group.  Others have discussed the importance of joint movement during sport and exercise in facilitating collective group action and prosocial behaviour (e.g., Cohen, 2017).

%           1.1.3 Unifying theories of exericse and social bonding:
Whitehouse / Atran,



%-------1.2 Phylogenetic and Ontogenetic Evidence----%
%           1.2.1 SC Model of cognition
%           1.2.2 Neuro, Developmental, Behavioural:  Mimicry, Synchrony
%-------1.3 Entrainment------------------------------%

%-------1.4 Dynamic Systems--------------------------%

%-------1.5 Sport------------------------------------%
%-------1.6 China------------------------------------%
%-------1.7 Hypotheses / Predictions ----------------%

%-----------------------------------------2. Ethnographic Setting ---------------------------------%

%-----------------------------------------3. Survey Study -----------------------------------------%

%-----------------------------------------4. Thesis Outline ---------------------------------------%

%-----------------------------------------5. Discussion/Conclusion --------------------------------%








%----------------1.1 Social Cohesion------------------%
%1.1.1 Shared Cultural Practices:



%----------------1.2 Phylogenetic and Ontogenetic Evidence------------------%

2.1 SC Model of cognition
  2.1.1 Neuro, Developmental, Behavioural:  Mimicry, Synchrony
2.2 Entrainment

%----------------1.3 Dynamic Systems------------------%



%---------------------1.4. Sport------------------------------%
Entrainment for joint-action in highly complex & uncertain environments

Rugby

%--------------------------1.5 China ---------------------------------%

Cultural variation,


%--------------------------1.6 Hypotheses / Predictions ---------------------------------%


%--------------------------2. Ethnographic Setting ---------------------------------%

%--------------------------3. Survey Study ---------------------------------%

%--------------------------4. Thesis Outline ---------------------------------%

%--------------------------5. Discussion/Conclusion ---------------------------------%




My basic claim is that human social cohesion is sustained by pre-representational affective (lower-cognitive) mechanisms activated through the coordination of movement in an environment.

So, while existing accounts of human social cohesion are useful models, the models need to be updated by representing more information latent in the system that can be brought to life by looking below the surface of representational accounts of human social behaviour.

A study of collective group activities such as music/dance, ritual, and activities like sport provide the oppportunity to investigate the proximate psychological mechanisms through which social cohesion—particularly social bonding—is generated.

On some fundamental level, human social cohesion may be driven by mechanisms of social alignment that involve basal cognitive tendencies to coordinate movement with the ecology (harmonise or achieve homeostasis with the environment), including other individuals that inhabit the ecology.  On the level of individual psychology, this may involve cognitive processes involved in prediction of action, in order to reduce informational uncertainty and maximise synchronisation with the environment.   In a human ecology, this involves some form of entrainment to the cultural (behavioural) regularities of movmement.  But only a protion of information regarding these regularities may be available to conscious awareness; only a portion of the information available in the environment is represented through explicit linguistic forms...

My particular focus is on the ways in which perceptions of successful coordination of movement appears to faciliate social bonding in professional Chinese athletes. Perception of joint-action, declared via surveys, is itself a representation of experience, albeit one that requires less stringent conditions for declaration than other higher-cognitive decision making processes regarding coalition formation or strategic cooperation.

``Neo-Durkheimian bullshit?''


Reduction of uncertainty by moving or responding to movement, this will generate attractor points of behaviour. It should be possible to model this in regards to movement.  fffr


























%-------------------------------------------------------------------------------------%
%-------------------------------------------------------------------------------------%
%-------------------------------------------------------------------------------------%
%-------------------------------------------------------------------------------------%
%-------------------------------------------------------------------------------------%
























1. Cognitive \& Evolutionary Anthropology of Group Exercise and Social Cohesion

The human capacity to cooperate within cohesive social groups is a fundamental reason for our species' evolutionary success.  It is surprising, in which case, that sport - a cross-culturally ubiquitous organiser of modern social life - has not been more intensively studied for evidence concerning the cognitive and evolutionary foundations of human sociality \citep{Blanchard1995,Downey2005a}.  An integrated scientific study of the social, cognitive, and physiological mechanisms associated with participation in collective group exercise (broadly construed), and the ecological dynamics by which these mechanisms are constrained, can offer novel insights into the science of human cognition, cooperation, and evolution.

Whirling Sufi dervishes, late-night electronic music raves, Maasi ceremonial dances, competitive team sports, or the cult of Cross-fit  -  endless examples can be plucked from across cultures and throughout time to exemplify the human compulsion to come together and move together.  It is easy to imagine how exertive and coordinated group activity would have served important survival functions in the past, such as hunting, travel and communication, and defence (Sands & Sands, 2010).  More puzzling, but also more relevant to the study of human social cohesion, is the recurrence of group exercise despite the fact that its substantive rewards are now less obvious or immediate.  Cross-cutting shared cultural practices as varied as religion, sport, music and play, group exercise is a ubiquitous feature of human social life.   Social scientists have long speculated about the benefits of energetic group activities for social cohesion (Durkheim, 1965). It is not yet clear, however, how or whether group exercise uniquely generates social cohesion, or in what ways particular mechanisms vary by activity and culture.



- What we know about Exercise so far?
Physical activity, exercise, and sport have well-known positive effects on physical and  psychological health (Ekkekakis, 2003; Fiuza-Luces, Garatachea, Berger, & Lucia, 2013).
The health benefits associated with regular exercise, including reduced risk of cardiovascular disease, autonomic dysfunction, and early mortality, are becoming increasingly well-known (Blair & Powell, 1994; Nagamatsu et al., 2014).

While the physiological, psychological, and social mechanisms that combine in instances of exerted, coordinated movement are rich and varied, many strands of research suggest a link between group exercise and social bonding \citep{Davis2015,Cohen2017}.  It is now understood that strenuous and prolonged physical exercise is modulated by the same neuropharmacological systems (namely, the opioidergic and endocannabinoid systems) responsible for regulating pain, fatigue, and reward \citep{Boecker2008,Raichlen2013}.
Exercise-specific activity of these systems offers a plausible neurobiological explanation for commonly reported sensations of positive affect, anxiety reduction, and improved subjective well-being during and following exercise - popularly referred to as the ``runner's high'' (Dietrich & McDaniel, 2004; Boecker et al. 2008; Raichlen, Foster, Gerdeman, Seillier, & Giuffrida, 2012).  This neuropharmacological account of group exercise and social bonding has its roots in studies of social grooming in non-human primates.  Dunbar and colleagues propose a neuropharmacologically mediated affective mechanism linking dyadic grooming practices with group-size maintenance \citep{Machin2011}.\footnote{The capacity for social bonding is thought to have arisen in primates as an adaptive response to the pressures of group living.  Aggregating in groups serves to reduce threat from predation, but at the same time can be individually costly due to stress arising from interaction at close proximity, and conflict over resources among genetically unrelated individuals.  These pressures are hypothesised to have led to selection for social bonding (e.g., via dyadic grooming). The coalitional alliances that arise among close partners allow for the maintenance of the group by buffering the stresses of group living.  Primate social grooming is associated with the release of endorphins, presumably leading to sustained rewarding and relaxing effects.  While other neurotransmitters such as dopamine, oxytocin and/or vasopressin may also be important in facilitating social interaction, it has been suggested that endorphins allow individuals who are not related or mating to interact with each other long enough to build “cognitive relationships of trust and obligation” \citep{Dunbar2012}(1839)}  It is thought that, as the homo genus evolved more complex collaborative capacities for survival in interdependent group contexts, grooming-like behaviours sustained social bonding in larger group sizes where dyadic grooming would take too much time \citep{Dunbar2012}.  Experimental studies suggest that neurophysiological mechanisms activated by activities that involve physical exertion and coordinated movement, such as group laughter, dance and music making, exercise, and group ritual can bring groups closer together, mediated by the psychological effects of endogenous opioid and endocannabinoid release \citep{Cohen2009,Fischer2014a,Fischer2014,Sullivan2014,Tarr2016,Tarr2015}.


- ``Flow''
In addition to reports of exercise-induced euphoria and positive affect, adherents to (group) exercise and other activities—particularly highly skilled practitioners—also commonly report experiencing states of ``optimal'' or ``peak'' performance, which include feelings of heightened focus, personal transcendence, time-warp (the experience of time either speeding up or slowing down), spontaneity, creativity, and effortlessness \citep{Jackson1995a}.  ``Flow'', as this particular cluster of states has commonly been referred to, is a powerful, autotelic  and embodied experience, which combines components of both ``hedonic'' (sensation-centred, see \citep{Huta2010}) and ``eudaimonic'' (meaning-centred, see \cite{Ryff1989,Ryff2015}) dimensions of subjective well-being \citep{}, and is theorised to emerge when activity strikes a balance for the individual between challenge and skill requirements \citep{Csikszentmihalyi1990,Abuhamdeh2012}.  At the level of the group, the ``team click'' and ``group flow''are highly elusive possibilities, coveted by athletes, coaches, and fans alike \citep{Novak1993,Sawyer2006}.  While the experience of flow associated with prolonged exercise may be in part neuropharmacologically mediated by the opioidergic and endocannabinoidergic systems, phenomenologically suggests that there is something distinct about the experience of flow in exercise that requires a more nuanced neurocognitive explanation \cite{Dietrich2006,Dietrich2011}.  The prevailing  neurocognitive account of acute exercise suggests that the metabolic costs associated with complex or prolonged regulation of movement forces an energetic trade-off in the brain in which lower level neurocognitive processes win out, forcing a down-regulation of the pre-frontal areas of the brain \citep{Dietrich2011}. Dietrich and colleagues propose that the down-regulation of cortical processes could induce a dampening of feelings of self-monitoring and personal agency, and an altering of attention and sense of time. It is possible that the flow and its neurocognitive underpinnings are relevant to prosocial effects of group exercise.

Error management reduction


- Click /Sync:
Meanwhile, research in social psychology focussing on the relationship between time-locked behavioural synchrony and processes of self-other merging, social alignment, and affiliation has shed light on the social and affective significance of interactive and coordinated movement typical of many group exercise contexts \cite{Wiltermuth2009,Kirschner2010,Reddish2013,Tuncgenc2016}. Experimental evidence suggests that time-locked coordination of behaviour between two or more individuals in the stable attractor/equilibrium states of either in-phase or anti-phase synchrony is conducive to psychological processes of self-other merging, liking, trust and affiliation.  It is believed that lower cognitive processes of joint attention mediate the link between synchrony and social bonding, with synchronised activity (common in music, dance, and some sports) providing a shared spatio-temporal (and often haptic) referent around which to coordinate attention and behaviour \cite{Launay2016,Wolf2015}.  Studies linking synchrony with social bonding and cooperation are supported by a literature than connects nonconscious mimicry with liking and affiliation\citep{VanBaaren2009}.  While the social and psychological effects of group level synchronisation have been harder to induce and measure in experimental settings, it appears that, in addition to in- and anti-phase behavioural matching, group synchronisation may be subject to more complex and dynamical processes of coupling, which could also entail specific psychological consequences.  Indeed, this also appears to be true in cases of joint—but not necessarily explicitly synchronised-action, whereby implicit processes of movement regulation link two or more individuals in a complex and dynamic coupling, the variation and stabilisation of such dynamic coupling could have psychological effects (see \cite{Schmidt2008,Marsh2009a}).


- Social Placebo:
Most encouraging is evidence that manages to integrate the social and neurophysiological dimensions of group exercise.  Recent experimental evidence suggests that social features of the exercise environment (for example, perceived social support, level and quality of behavioural synchrony, etc) modulate exercise-induced mechanisms of pain, and reward \citep{Cohen2009,Sullivan2014,Tarr2015,Davis2015,Weinstein2016}. This work is bolstered by existing literature on the social modulation of pain \citep{Eisenberger2012a} and links between pain and prosociality \citep{Bastian2014a}.



Considered together, current evidence suggests a deeply social dimension to the neurophysiology and cognition of group exercise.






- The Significance of Group Exercise for the Cognitive and Evolutionary Anthropology of Social Cohesion

There is something unmistakably ``embodied'' about coordinated and exertive group activity. Indeed, perhaps this visceral quality of group exercise has helped ground the traditional intuition within Anthroplogy that group exercise is somehow causally relevant to broader social processes, (despite the lack of precise theoretical frameworks within which to test such speculations (see for example, Mauss 1935, Radcliffe-Brown, Turner,  Merleau-Ponty 1955, Bourdieu 1990). Indeed, it is not until relatively recently, that, following the modern evolutionary synthesis and the cognitive revolution, a rigorous scientific inquiry into relationship between shared cultural practices (including those in which group exercise commonly features) and human social cohesion has gathered momentum (Wilson, Sperber, Henrich).  Prevailing cognitive and evolutionary theories of human social cohesion have built upon game-theoretical population genetics models adapted from evolutionary biology.  Generally speaking, these accounts emphasise the role of evolved cognitive mechanisms that enable collective adherence to and widespread transmission of cultural representations throughout populations \citep{Sperber1996,Henrich2007}.  Dual Inheritance Theory (also known as gene-culture coevolution \cite{Richerson2008}) asserts that humans have evolved, through a constant interaction of genetic and cultural information, a precocious and species-unique tendency to accurately imitate the actions of trusted or authoritative others (even when the goal of the action is unclear, \cite{Tomasello2014a}). The result of which is an evolved ``norm psychology'' which facilitates prosocial norms and institutions.  DIT employs the assumption that the transmission of cultural information is analogous to the transmission of genetic information: adaptive cultural units of information are preserved through high-fidelity copying (with error) and proliferate throughout specific cultural niches, which produces more or less cohesion social organisation.

DIT and related theoretical approaches have made productive empirical strides in identifying the proximate cognitive mechanisms most relevant to ultimate evolutionary explanations for the distribution of shared cultural practices around which human groups cohere, particularly in relation to large scale cultural forms such as religion \citep{Henrich2015,Purzycki2016b}. However, as critics of DIT have pointed out, part of the trade-off involved in the DIT formulation, is that many details concerning the interactive and affective mechanisms of shared cultural practices (i.e., the unmistakably ``visceral'' dimension referenced above) are not incorporated into its theory of cultural transmission. Researchers in comparative and social psychology have pointed out that humans do not merely aggregate around shared cultural practices, but rather actively congregate, seemingly driven by species-unique affective and motivational mechanisms \citep{Dunbar2010,Tomasello2005a}.  Other research challenges DIT's over-reliance on the cognitive mechanism of imitation, suggesting instead that the transmission of cultural information is reliant on multiple domain-specific cognitive modules involved in imitation, communication, and memory, and these mechanisms are constrained by the affordance of task-specific ecological niches \citep{Sperber2004,Claidiere2007a}.\footnote{Thus, according to the ``Theory of Cultural Attraction (CAT)'' cultural transmission is not a process of copying with error, but one of ``re-production.'' The reproduction and transmission of cultural practices thus produces statistical attractor points to which cultural representations are tendentially directed.} Taken together, these critiques suggest that the affective and interactive dimensions of shared cultural practices are causally relevant to the distribution of cultural variants throughout populations (SOURCE - Nettle).  In other words, the highly ``visceral'' quality of shared cultural practices, such as those that involve rigorous and coordinated group exercise, could be a signal of important and hitherto unarticulated information concerning the proximate componential mechanisms, dynamical constraints, and ultimate evolutionary explanations for human sociality \citep{Claidiere2014} (pg3).

Promisingly, recent advances in neuroimaging technologies \citep{Frith2007}, neurocomputational theories of brain function \citep{Friston2010,Frith2010,Clark2013}, and attempts to extend the theoretical paradigm of human cognition to encompass and satisfactorily account for inter-individual processes of interaction and coordination \citep{Sebanz2006,Marsh2009a,Dale2014}. These innovations have redirected attention from traditional individual-centred computational models of cognition (the basic unit of production being amodal, symbolic representations), towards the ways in which basal human capacities for physical movement regulation and coordination set the foundation for cognitive systems whose resources are distributed throughout brains, bodies, and physical features of task-specific environments \citep{Hutchins2000,Kirsh2006,Semin2008,Semin2012,Coey2012}. Conceiving of social cognition as embodied and embedded centralises the role of automatic movement regulation strategies\nobreakdashtraditionally classed as ``lower-cognitive'' processes\nobreakdashin establishing and maintaining in the transfer of information between individuals, within groups, and throughout populations\nobreakdashtraditionally thought to be the primary domain of ``higher-cognitive'' capacities.  In addition, traditional individual-centric computational models of cognition, which tend to render movement as the product of a linear sequence of sensory perception, internal mental representation, and action selection (see for example, SOURCE), have been superseded by models of cognition in which perception, representation, emotion, and action are functionally and temporally integrated in the service of informational processes situated in—but also distributed throughout—human brains, conscious minds, active bodies, and feature-rich physical environments.

The theoretical approach that I develop in this dissertation relies on upon five interrelated hypotheses concerning human social cognition:

1. Human cognition is generated by a multitude of cognitive mechanisms responsible for regulating movement, evidence for which can be identified in a spectrum of social behaviours ranging from more explicit and deliberate communicative signals, through to more implicit (often pre-perceptual) processes of synchronisation and alignment of the organism with the environment (Frith & Frith 2010, Semin & Cacciopo 2008). Prevailing models of social cognition reveal that higher order explicit communication strategies (consciousness, symbolic language, rational thinking), traditionally privileged in computational accounts of cognition, draw on only a small fraction of the total amount of information processed by the nervous system at any one moment.

2. Human cognition is constantly engaged in ``active inference'' (SOURCE) about the world, by generating interoceptive predictions about the state of the world, and testing these neurophysiological representations against actual bottom-up sensory evidence (SOURCES).  In this conception, known as the predictive processing paradigm (Clark 2013), the fundamental drive of human cognition is to minimise ``prediction error,'' or the discrepancy between predictions of the world and the bottom-up sensorial experience of the world. In the predictive processing paradigm, perception, representation, emotion, and action are unified by a deep logic of prediction-error management, where action and perception work together to reduce align the organism with its expectations.  It has been shown that, and in light of hypothesis 1, much of the work of predictive processing occurs beneath the level of conscious awareness (Frith 2007, Clark 2013). It is also plausible that predictive processing is driven by a more fundamental mandate of reducing an information-theoretic isomorph of thermodynamic ``free energy'' in the nervous system's interaction with the environment (Friston 2010).

3. Human cognition relies on resources that are distributed across neural, bodily, and environmental features (e.g., Hutchins, 1995; Kirsch, 1995; Brooks, 1999; Agre, 1997) with the social and physical environment supporting action and interaction (Smith & Semin, 2004).  The embedded and distributed nature of cognition is observable in instances of joint-action, where the time-locked coordination of behaviour with two or more individuals in a physical environment gives rise functional interpersonal synergies.  These coordinative structures reduce the componential degrees of freedom of the system components and allow for cognitive resources to be efficiently distributed across all nodes of the emergent dynamical system (Semin & Cacciopo 2008).  The self-organising properties of dynamic coupling in joint-action have been modelled, revealing evidence of multi-scale synchronisation and complexity matching (Schmidt, Richardson).

4. Higher quality synchronisation of behaviours in joint-action scenarios has positive implications for individual psychophysiological function, health, and subjective well-being (Wheatley et al. 2012).  In particular, successful coordination in joint-action has implications for psychological processes relating to social bonding \citep{marsh2009a,Launay2016}.  Evidence from the self-report of highly skilled joint-action practitioners (musicians, performers, athletes) suggests a strong link between the perception of high-quality joint-action, and psychological wellbeing \citep{Jackson1995,Jackson1992}.  Conversely, the psycho-social isolation and ill-health experienced by individuals who suffer from developmental and neurocognitive deficits in behaviours key to dynamic interpersonal interaction is also well-documented (ASD: Frith, Baron-Cohen).

5. The coupling dynamics of co-present joint-action have implications for higher order and more complex goal oriented settings, including the (re)production and transmission of shared cultural practices (Roepstroff, Claudiere).



-  Joint-Action and Social Bonding as the appropriate Domain of Interest

- Research Questions and General Predictions









 ``Joint-action'' can be regarded as any form of social interaction whereby two or more individuals coordinate their actions in space and time to bring about a change in the environment \citep{Sebanz2006a}.  In an ethnographic environment where athletes train, eat, and sleep together, and where on-field performance is directly contingent on effective coordination of behaviours with teammates, joint-action provides the most appropriate theoretical starting point for analysis of the collective movement of professional rugby sevens athletes in China.













\printbibliography
\end{document}
