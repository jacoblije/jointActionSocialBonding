section~\begin{savequote}[8cm]

  It takes two to know one.

  \qauthor{--- Gregory Bateson}

\end{savequote}


%I do not see any way to avoid the problem of coordination and still understand the physical basis of life.
%\qauthor{--- H. H. Pattee  \textit{The role of instabilities in the evolution of control hierarchies} 1976}


\chapter{\label{chap:theory}Team click and social connection in joint action}

\minitoc


                                  \begin{CJK}{UTF8}{gbsn}

\section{The development of visceral agency in rugby's newest recruit\label{sect:SHW}}
Sun Hongwei arrived, escorted by his high school athletics coach, to the Beijing Temple of God of Agriculture Institute of Sport (hereafter the Institute) soon after I began my fieldwork in August 2015.  An 18-year-old with a slight build and timid demeanour---his gaze remained diverted to the ground during his first few months at the Institute---Hongwei later told me that he had never seen a rugby ball before that day he arrived.

Hongwei was from Hebei province, immediately surrounding the special prefecture of Beijing, China's capital.  Hongwei's coach had organised a trial for Hongwei with the Beijing Provincial men’s and women's Rugby Program (hereafter the Program) by calling upon social connections to the leadership of the Institute.  Athletes come to the Rugby Program from all over the country.  As I explain in more detail in Chapter~\ref{chap:researchSetting}, to represent Beijing at a provincial level in a sport like rugby can translate into the opportunity to gain entrance to one of China's top universities and enhanced career employment opportunities thereafter.

Rugby is not a popular sport in China, but its recent inclusion in the Olympic games (in the form of the modified seven-a-side version of ``rugby sevens'') means that it now occupies a prominent place in the Chinese sport system.  Rugby programs such as the one at the Institute now exist in 12 of China's 34 provincial level regions, either embedded within, or somehow associated with, tertiary education institutions.  Thus, although rugby and China are not commonly associated terms, rugby now affords Chinese athletes a rare and under-capitalised opportunity to pursue attractive life-course opportunities of education and employment in an intensely competitive education system.

Without exception, the athletes who arrive at the Institute to join the rugby team were not like me; they had not spent their childhoods playing rugby in their schoolyards or watching professional rugby on television. Many who come to rugby transition from other more popular sports such as athletics, basketball, or association football, and often---like Hongwei---have never seen a rugby ball before they arrive.  Most athletes ``start from scratch,'' so to speak, in terms of their grasp of the requirements of the highly interactive and technically complex team sport.  In addition to complex patterns of movement coordination, rugby also involves unrestrained body-on-body collisions and intense bouts of high physiological exertion.  To perform all of rugby’s technical requirements successfully requires a combination of speed, strength, agility, and endurance (see Chapter~\ref{sect:rugbyUnion}).  Learning the game of rugby from a baseline of essentially zero, while also navigating the inevitably demanding social and political dynamics within the team and the Institute, was clearly going to be a daunting task for Hongwei.


                        \begin{center}
                          * * *
                        \end{center}

 Even compared to other newly arrived junior athletes, I noticed that Hongwei was particularly timid and shy, especially in his interactions with the senior players and coaches (myself included).  Nevertheless, Hongwei clearly signalled diligence and commitment through his participation in team activities.  He arrived early to each training session, and carried more than his fair share of the training equipment---a task shared by the most junior members of the team.  Each time I passed Hongwei in the corridors of the Institute he would greet me with a polite bow and greeting, ``Hello Coach'' (\textit{'jiaolian hao'} `教练好').  In these instances, Hongwei would coordinate his greeting with a moment's eye contact, only to return his gaze to the floor and continue walking.

Due to his lack of familiarity with the basic techniques of rugby, Hongwei was initially unable to fully participate in normal training with the rest of the team.  Instead, during the first month or so, Hongwei remained on the sidelines of the field and practiced the basics with other athletes who were unable to fully participate in training due to injury. He began with the absolute basics: learning how to pass and catch the rugby ball, while stationary and running in-motion. In my eyes at least---eyes of an observer accustomed to instinctual grasp of these movements from a young age---I found Hongwei’s attempts to accustom himself with the skills of rugby jarring. The idiosyncrasies of rugby’s ovular ball often foiled him.  I would regularly see him on the sidelines of training chasing after a ball that he’d just fumbled; the idiosyncratic shape of the ovular rugby ball meant that it would change direction like it were a scurrying rabbit tactfully evading its pursuer.

                        \begin{center}
                          * * *
                        \end{center}

I interviewed Hongwei approximately six weeks after he arrived at the Institute.  Hongwei's demeanour during the interview was consistent with the timid and shy one that he presented publicly at training.  As I explain below, he did show some signs of captivation with his new sport and social environment.  When I asked about his initial impressions of the on-field demands of rugby, however, Hongwei was quick to confess that he felt utterly unacquainted:

  \begin{quote}
    SHW: I still haven’t really started to practice any of the team plays or anything; all I can do so far is pass and run a little bit...(but) it's quite fun! \\
    JT: What do you think is the most difficult component of rugby? \\
    SHW: Um\textellipsis well, coordinating with teammates [on the field], particularly coordination in attack.  Because I can't figure it out.  When I first arrived, I didn’t even know what a ``switch play'' or a ``blocker play'' was.
  \end{quote}

  \begin{quote}
    SHW: 战术没怎么接触,就是像传球啊、跑动什么的会一点了 \\
    JT: 感觉怎么样?\\
    SHW: 挺好玩的!\\
    JT: 你认为橄榄球最难的一部分是什么? \\
    SHW: ...打配合,进攻的配合,因为搞不明白,刚来的时候也不知道什么交叉,后插什么的 \\
  \end{quote}

Hongwei was self-critical in this confession regarding his minimal grasp of the technical requirements of joint action in rugby.  The obvious stress and anxiety he expressed was indicative of a broader trend with the other athletes I interviewed, as I will explain in more detail in the ethnographic sections of this dissertation (see Chapters ~\ref{chap:ethnoField}\nobreakdash~\ref{chap:ethnoResults}). When athletes were asked in interviews about the most difficult aspect of rugby, on-field coordination with teammates was by far the most common answer, particularly among the new recruits and most junior athletes.

As I directed the interview towards topics beyond the on-field technical demands of rugby, Hongwei was more positive, framing rugby as an exciting new opportunity, and commenting that his friends and family were in awe of the fact that he is playing such an impressively ``strong'' (\textit{qiangzhuang} 强壮) physical sport like rugby.  When I asked him about something new that he had learnt through playing rugby, Hongwei automatically responded by emphasising the social dimensions to his experience at the Institute:

  \begin{quote}
    ...I think it's mainly this thing of having teammates. Before, when I was training for an individual sport, it was just me training by myself. [In that environment] it was a case of whoever trained well was successful.  But now with this team of brothers, elder teammates will take care of younger teammates. We all train together, and if you can’t do something, you can always ask your elder teammates...[Rugby] is so much better, because in an individual sport, if you can't master something, you have to go to your coach for help. Other athletes don't want to teach you, because if you surpass them, then they have to work even harder to keep up... I have had to learn about helping others, because rugby is not like an individual sport, where you look after your own performance and that's it.  In a team sport, if you don't do well, there's no need to get too frustrated or upset, because other athletes will help you out, and I will also help others out, that type of collaboration with each other.
  \end{quote}

\begin{CJK}{UTF8}{gbsn}
  \begin{quote}
    我觉得主要是师哥师弟的这一块儿,原来练个体项目都是自己练自己的,谁练好了谁厉害,但是现在师哥师弟,有师哥照顾师弟带着,互相练,我不会我可以问师哥
    ...因为个人项目你不会就必须要找教练,但是别人不愿意教你,因为你把别人超越了那别人还还得努力。 学到互相帮助,因为向个体项目自己成绩自己来拿就行,而像团体项目,即使自己做不好,也不用太泄气太沮丧,因为别人会帮你做好,我也会帮别人做好,互相协作的那种.
  \end{quote}
\end{CJK}

Hongwei's explicit reference to the collegiality of the team, and his position as junior member, highlights that the technical skills of rugby were not the only novel components of his experience.  Hongwei's background was in track and field (his event was pentathlon, very much an individual sport), and the team environment was completely new to him.  This was also the case for many of the other athletes in the team.  As I listened to his experiences associating rugby and group membership, I could not help but associate the quality of these explicit declarations of group membership with his overly mechanical imitation of rugby's foundational techniques.  In both I saw his willingness and desire to signal commitment; but both lacked---at the outset at least---a certain level of grasp or conviction.  In the case of his explicit celebration of rugby's social resources in interview, for example, I got the impression that Hongwei was telling me what he thought he ought to say, without actually deeply believing these things to be unequivocally true.


                            \begin{center}
                              * * *
                            \end{center}

A few months passed, and Hongwei continued to train.  He was as eager and committed as when he began, and I did notice some gradual improvement in his grasp of rugby's basic skills.  But he also remained extremely reserved, keeping his head low at all times in team settings, unless addressed by senior players or coaches.

Then, one evening when I had returned to my room in the Institute dormitory from a three week hiatus in Australia for Christmas, I heard a knock on my open door, and to my surprise Hongwei took an assertive stride into my room, carrying in two arms a draw-string bag containing rugby balls (which were in need of more air before the next day's training session).  Hongwei had never ventured into my room before, apart from when I invited him in for our first interview two months earlier---but certainly never before had he entered on his own accord.  Hongwei looked me straight in the eyes with his head held high and energy radiating from his face and upright chest.  I could not help but smile and ask, with genuine intrigue, ``How have you found training recently?'' (最近练得怎样?)
``Very good'' (很好)he said, assertively and excitedly.  ``Much better than before.  At least now I know what’s going on at training, I can keep up with the plays!'' (比以前好多了, 至少现在训练的时候知道该干什么,战术什么的能跟得上!) A big smile spontaneously grew on his face as he continued to hold my gaze confidently.  ``Oh good!'' I said, a little bit shocked.  I congratulated him for his hard work in training while I had been away, and encouraged him to keep at it.  At this, Hongwei took leave by bowing politely and saying ``Thanks coach'' (谢谢教练).  ``Wow,'' I remember thinking to myself.  What quantity had all of a sudden possessed Hongwei? Was it possible that his diligent adherence to rugby over those first four months at the Institute had instilled him with a visceral sense of personal and social agency?

                          \begin{center}
                            * * *
                          \end{center}


  %  the fact that aspects of Hongwei's personal and social demeanour at the Institute appeared to covary with his familiarity with the technical requirements of rugby over time suggests a relationship between joint action and group membership that is worthy of further investigation.
    %In the same way that Adrian's opening monologue (Chapter~\ref{chap:intro} section~\ref{sect:adrian}) suggested a relationship between the on-field dimensions of rugby and social processes between teammates, Hongwei's gradual development



\section{The need for a theory of joint action and social bonding in group exercise}

%HW's change is a change in attitude (Bateson) agency
How can we scientifically account for the unmistakably visceral quality of Hongwei's transformation from timid newcomer to budding Beijing rugby player?  As the story of Hongwei anecdotally suggests, the visceral dimension of group exercise may take time to develop.  Presumably in Adrian (see Chapter~\ref{sect:adrian}) I witnessed the finished product: through many years of on- and off-field engagement with the sport of rugby, Adrian came to embody rugby and thus profess its carnal mystery from a place of deep, intuitive knowing.  When I met him during my first stint of research at the Institute, Hongwei was only just beginning this journey.  Over an extended period of ethnographic research it became clear to me that, for Hongwei (and others), an increase in familiarity with both the on-field technical requirements of rugby and the off-field social requirements of the Institute came an increase in personal confidence, and sense of group membership.  As explained in the previous chapter, a scientific theory capable of accounting for the dynamic and interlocking physical, cognitive and social dimensions of the carnal mystery of group exercise is yet to be fully formulated.

In this chapter, I draw attention to strands of research capable of accounting for team click and social connection in joint action.  The theory proposed herein builds upon the hypothesis that group exercise is causally relevant to processes of social cohesion \citep{Dunbar2010,Whitehouse2014,Cohen2017}. In addition to ``cognitive'' processes (as they are narrowly defined by traditional stimulus-reponse paradigm of human cognition \citep[e.g.][]{Marr1985}), recent empirical research programs within cognitive and evolutionary anthropology have begun to draw attention the role of physiological and affective mechanisms of neurobiological reward and autonomic arousal in shaping human behaviour, sociality, and cultural evolution.  As Dunbar and Shultz suggest, it is impossible to accept that humans merely adhere to a ``dung fly'' model of social aggregation; clear scientific evidence now suggests, rather, that humans have evolved a species-unique tendency to actively congregate and cohere around shared cultural practices \citep[see][]{Tomasello2005}.

But, as explained in Chapter~\ref{chap:intro}, existing approaches to the empirical question of human bondedness and social cohesion could benefit from a model of cognition in which physicality is more dynamically integrated into processes of information transfer \citep{Yufik2017}.  A model of cognition that dynamically integrates cognitive processes with physiological processes could enable novel and testable predictions for a fuller spectrum of collective practices in human sociality.  In this dissertation, I propose that that an integrative model of ``active inference'' \citep{Friston2010} can assist empirical progress in the development of cognitive and evolutionary explanations of group exercise in human sociality, and indeed scientific questions relating to cognition and evolution of biological life more broadly \citep{Ramstead2017}.


%Doing so could help address empirical as well as theoretical gaps in our understanding of the role of group exercise in human sociality.  For example,  the social high theory of group exercise and social bonding \citep[see][]{Cohen2017} lacks a nuanced account of how extreme physiological cost generates profound meaning, or how cognitive complexity in joint action (beyond strict behavioural synchrony) can produce autotelic experiences of flow and team click (and flow on social effects, see Chapter~\ref{chap:intro} Section~\ref{sect:empKnowGaps} for a more thorough explanation).  Likewise, in the case of the modes theory of ritual practice and social cohesion, an integrative model of cognition could facilitate empirical verification of a more continuous gradient of causal links between ritual practices involving group exercise and social cohesion (see Chapter~\ref{sect:cogEvAnth} for a full elaboration).  In addition, despite best efforts to theorise the causal role of a fuller spectrum of ``factors of attraction,'' current models of cultural evolution (e.g., CAT, see Chapter~\ref{sect:cogEvAnth}) that underpin investigations of social cohesion remain entrenched in a theoretical debate that predominantly revolves around propositional or semantic cultural variants (see Chapter~\ref{sect:cogEvAnth}).  As such, cognitive mechanisms responsible for more explicit social learning (e.g. imitation and instruction) are promoted in causal accounts of cultural evolution, while more implicit, embodied, and dynamical mechanisms of cultural transmission remain a subsidiary program of empirical research  \citep{Lerique2016}.  In all three of these examples, proximate physical, cognitive, and social processes---while no doubt understood by researchers to be collectively relevant to human behavioural phenomena---are nonetheless treated theoretically as parallel and functionally distinct components (see Chapter~\ref{sect:cogEvAnth}).

More so than most cultural variants that have traditionally formed the focus of scientific analysis, activities in which group exercise feature demand a dynamic conception of cognition.  Group exercise invariably involves brains and bodies engaged in ``in-the-moment'' and ``on-line'' joint action.  From a pure computational perspective, achieving and sustaining joint action is a daunting cognitive task, as it requires the coupling and constraining of various degrees of freedom belonging to autonomous co-actors \citep{Bernstein1967,Turvey1978}.  However, it is now more clearly understood that human cognition—--and indeed, cognition of biological life more generally—--does not rely on passive, ``brute force'' computation---like that of a digital computer---to manage to the challenges of interacting with the environment \citep{Yufik2013}.  Rather, prevailing theories of cognition suggest that humans flexibly deploy a range of ``active'' strategies in order to anticipate and regulate the uncertainty inherent in joint action and the universe of sensory causes more broadly \citep{Clark2015}.

In fact, one of the core cognitive strategies that has been overlooked by traditional ``stimulus-response'' models of cognition (which were inspired originally by digital computers) is physical movement itself.  The integration of movement into human inference through the active inference framework'' \citep[hereafter AIF see][]{Friston2010} promises a clearer theoretical conception within which to conduct empirical research of the embodied, enactive, extended, and embedded cognitive phenomena for which we have hitherto tended to collectively shroud in mystery, myth, or metaphysics  \citep{Atran2010}, or else relegate to superordinate, supervening, or sub-systemic causal status   \citep{Clark2015,Linson2018}.  Specifically, and for the purposes of this dissertation, the AIF offers predictions for how humans click, connect, and cohere through joint action typical of group exercise contexts.

In this chapter, I use the AIF to develop a theoretical foundation for a relationship between joint action and social bonding in group exercise (outlined in full in Chapter~\ref{chap:theoryGE}).  In particular, I theorise team click as a psychological construct that mediates a relationship between joint action and social bonding.  While the social relevance of team click and the related concept of flow have evaded rigorous analysis within more traditional paradigms of social cognition \citep[for explanations as to why, see][]{Dietrich2004,Slingerland2014}, within the AIF, team click can be explained in naturalistic terms as a special case in which the bonding effects of joint action are maximised.

%Preview theory here: As I outline in full detail in the sections below, a theory of social bonding through joint action depicts two or more brains








In the sections that follow in this chapter, I first introduce the cognitive challenge of joint action, and the suitability of team click as a real-world empirical anchor and theoretical mediator of a relationship between joint action and social bonding (Section~\ref{sect:dfProblem}).  I then introduce the AIF, including its three core tenets of thermodynamic cognition, predictive coding, and affordances (Sections~\ref{sect:thermoCog}\nobreakdash~\ref{sect:affordances}).  With this grounding, I review attempts to apply AIF to joint action, and suggest that mechanisms responsible for generating surprising (expectation violating), visceral, and agentic experiences in joint action could explain experiences of team click (Section~\ref{sect:AIFteamClickJA}) and provide a grounding for more elaborate processes of social identity and meaning making (Section~\ref{sect:AIFsocialBondingJA}).  This review sets the foundation for a novel theory of social bonding through joint action, which I outline with specific reference to group exercise contexts in Chapter~\ref{chap:theoryGE}.


\subsection{Introducing the phenomenon of team click as a potential mediatory of the physical-social link in joint action \label{sect:teamClickIntro}}
In this section I outline the phenomenon of team click, and point to evidence of its connection to mechanisms of joint action and social bonding.

One of the big mysteries of competitive team sport, particularly at the elite level, is the elusiveness of peak team performance.  While each individual athlete may exhibit expert level competence in sport specific skills, the much sought after aggregation of these components, i.e., a team that consistently performs ``in the zone,'' and ``firing on all cylinders,'' in reality often proves frustratingly difficult to achieve and sustain.  As \textcite[568]{King2011} note in their ethnography of the 2008 Cambridge University rowing crew who participated (and who were eventually victorious) in the famous annual Boat Race against Oxford University, the search for collective rhythm is a universal in human social interaction, but  the physiological and psychological complexity of finding that rhythm ``...is extremely difficult to attain; collective performance is a possibility not a certainty.''   The moment in which everything ``clicks'' into place in team sport can, for various reasons, disappear as abruptly as it arrives, if indeed it arrives at all.

But, when team click is somehow cultivated, and even sustained, it is celebrated as the ultimate, albeit often inexplicable magic of sporting feats. Consider Leicester City Football Club's unbelievable outhouse-to-penthouse title run in the 2015 English Premier League, the recent dominance of the Golden State Warriors in the American National Basketball Association (NBA), or the astonishingly consistent performance of the New Zealand men's national rugby union team (the ``All Blacks'').
    \footnote{The All Blacks are arguably the most successful sporting team ever, with a winning percentage of 77\% in the last 150 years (and 88\% in the last 6 years) \citep{Kerr2013}.}
All of these successful teams carry with them a powerful ``aura'' associated with their capacity to effectively coordinate their behaviours on the field over extended time scales: individual games, seasons, and, in the case of the All Blacks, entire generations.  The aura associated with such rare instances of collective performance can seduce collective fascination and elaborate exegesis.

    %In this thesis I commit to the more banal task of naturalising the phenomenon of team click, by locating it within a novel theory of social bonding through joint action.

For athletes, coaches, and spectators alike, team click can be a hugely powerful sensation.  As theologian Michael Novak explains, ``[f]or those who have participated on a team that has known the click of communality, the experience is unforgettable, like that of having attained, for a while at least, a higher level of existence'' \citep[11]{White2011}. For the purposes of this dissertation, I use the term team click to describe the phenomenology of peak performance within a team of athletes engaged in joint action, but the term has potentially broader applications beyond group exercise (for example, in music making, spoken conversation, or other similar activities).

As explained in Chapter~\ref{sect:linksComplexClick}, the experience of optimal performance in sport has been extensively documented in the psychological literature of flow \citep{Csikszentmihalyi1992,Jackson1995,Jackson1999,McNeill1995}.  Psychologists and neuroscientists have extensively studied the experience of flow, and have tabled a series of neuropharmacological \citep{Boecker2008}, neurocognitive \citep{Dietrich2006,Dietrich2011,Labelle2013}, and psychological \citep{Csikszentmihalyi1992} theories for its emergence.  The vast majority of flow research has focussed on the experience of the individual---the athlete, musician, or performer.  Some attempts have been made to extended an analysis of flow and its antecedents to the level of the group and dynamics of interpersonal coordination---a phenomenon termed ``group flow'' \citep{Sawyer2006}---but these attempts lack coherence and development.  Anecdote and observational evidence suggests that in addition to the components of flow already identified at the level of the individual, successful performance of technically complex
\textit{joint} action (beyond exact in-phase synchrony) could have distinct psychological and social consequences.

    %The experience of flow has by now been extensively studied by psychologists and neuroscientists, from which a series of neuropharmacological \citep{Boecker2008}, neurocognitive \citep{Dietrich2006,Dietrich2011,Labelle2013}, and psychological \citep{Csikszentmihalyi1992} theories for its emergence have been tabled.  The vast majority of flow research has focussed on the experience of the individual---the athlete, musician, or performer.  Some attempts have been made to extended an analysis of flow and its antecedents to the level of the group and dynamics of interpersonal coordination---a phenomenon termed ``group flow'' \citep{Sawyer2006}---but these attempts lack coherence and development.

Team click shares many similarities with the psychological states associated with flow, but is distinct in that it specifically delineates perceptions of joint action from individual action, and therefore implicates physiological, cognitive, and social mechanisms unique to joint action \citep{Vesper2010}, including the complexity systems dynamics associated with participating in a socially-coordinated, multi-agent system of physical movement \citep{Kelso2009}.  Team click is anecdotally present in a wide range of joint action contexts, and is often associated in these contexts with psychological processes of positive affect and wellbeing, as well as personal agency, social affiliation, and group membership \citep{Jackson1995,Marsh2009,Wheatley2012,Slingerland2014}.


The experience of team click, like flow, also often involves positive violations of individual expectations about action.  While elite athletes presumably always aim for optimal joint action, the experience of actually achieving that desired state, and even exceeding it---often with the help fo co-actors---appears to be a powerful, and somewhat ``surprising'' experience.  In the 1990s, psychologist Susan Jackson conducted a number of qualitative studies documenting athletes' (in this case elite figure skaters) experience of flow in solo and joint action

    \begin{quote}
      Oh, it's awesome! Everything's at peace...the crowd was involved, we were involved as a pair team, and we were just aware of everything, it just kept snowballing, like we weren’t going to miss, and it was great, it was an awesome experience \citep[168]{Jackson1992}.
    \end{quote}

Here, the athlete appears to be surprised and in awe of his or her own performance.  Part of the awesomeness may derive from the rare feeling of absolute alignment between self and co-actor(s):

    \begin{quote}
      What was really so special about this performance was that there was nothing between (partner's name) and I but flow.  You know, through the whole three minutes.  That meant that her mind and my mind were clear and in the same...in a partnership. It's always adjustments for ``Am I?'' and ``Are You?'' and ``Where Are We?'' and stuff like that.  That day was really a marriage of (partner) and (self) and the ice.  So I think it's a different kind of flow experience than a single skater who has much more control over themselves. \citep[173-4]{Jackson1992}.
    \end{quote}

This passage highlights the cognitive complexity of joint over individual action, suggesting the cost and uncertainty associated with attending to and attempting to control the movement degrees of freedom of others in addition to one's own.  The achievement of ``click'' in joint action may instead represent a jitter-less ``co-confidence'' \citep{Noy2015,Noy2017}, in which fixation on the contributions of ``you'' and ``I'' are momentarily abandoned or down regulated, and the agency of ``we'' is realised \citep{Gallotti2013,Friston2015}.

%This passage demonstrates the key elements of distinction between the experience of click in solo and joint action. As I will explain in more detail below, joint action requires constant finessing (adjustments) of behaviours and intentions between co-participants in which individuals are consulting internal models of self, other, and joint action \citep{Pesquita2017}.

Importantly, team click appears to have important flow-on consequences relevant to social bonding and affiliation. Tightly synchronised activity in particular, found in team sports such as rowing, can help dissolve the boundaries between individual and social agency: ``In rowing...it feels like you have at your command the power of everybody else in the boat. You are exponentially magnified. What was a strain before becomes easier. It is absolutely the ultimate team sport'' \citep[x]{Brown2016}.
The blurring of agency between self and team may be responsible for facilitating affiliation and trust between teammates in competitive athletic environments such as professional rugby, which often involves high physiological stress and uncertainty.  As Jeremy Paul, a famous Australian Rugby player commented of his late teammate, Dan Vickerman: ``...you always wanted a guy who would go into the trenches with you and he always played consistently well...he could really play and was just one of the good lads that you enjoyed his company'' \citep{Fox-Sports2017}.  As this excerpt suggests, the experience of team click may act as a social diagnostic tool, a powerful signal of commitment to joint action and willingness to cooperate \citep{Reddish2013a}. \\

Team click refers to the perception of peak team performance.  Based on anecdote and evidence from psychology and anthropology, an individual's perception of team click should contain some or all of the components outlined in Figure ~\ref{tab:teamClickComponents}.

% Please add the following required packages to your document preamble:
% \usepackage{booktabs}
\begin{table}[]
  \centering

\begin{tabular}{@{}ll@{}}
\toprule
\textbf{Component} & \textbf{Description} \\ \midrule
 &  \\
\textbf{Viscerality} &  \\
 &  \\
Tacit understanding & \begin{tabular}[c]{@{}l@{}}Perception of unspoken or implicit alignment\\ of expectations with co-actors\end{tabular} \\
 &  \\
Group flow & Perception of flow or coherence in joint action \\
 &  \\
Atmosphere or aura & \begin{tabular}[c]{@{}l@{}}Perceived positive, energetic quality within \\ or surrounding the team\end{tabular} \\
 &  \\
 &  \\
\textbf{Social Agency} &  \\
 &  \\
\begin{tabular}[c]{@{}l@{}}Blurring of self \\ and other agency\end{tabular} & \begin{tabular}[c]{@{}l@{}}Blurring of distinction between self and other\\  agency in joint action\end{tabular} \\
 &  \\
\begin{tabular}[c]{@{}l@{}}Ability extended \\ by others\end{tabular} & \begin{tabular}[c]{@{}l@{}}Feeling that individual abilities are extended\\ by the agency of others\end{tabular} \\
 &  \\
\begin{tabular}[c]{@{}l@{}}Reliability of self\\  and others\end{tabular} & \begin{tabular}[c]{@{}l@{}}Perceived certainty concerning the ability of \\ self and others to contribute effectively to joint action\end{tabular} \\
 &  \\ \bottomrule
\end{tabular}

\caption{Phenomenological components of team click}
\label{tab:teamClickComponents}
\end{table}



As explained above, team click is a phenomenon distinct to joint action.  Team click involves an experience of pleasurable flow or coherence of joint action, tacit (rather than explicit) understanding between co-participants, a degree of aura or atmosphere around joint action, a blurring of agency between self and teammates or group (or ``we''), and perception of reliability of others and self in performing their roles.  Team click offers a powerful empirical anchor for exploring a broader relationship between joint action and social bonding in group exercise contexts.  In the sections that follow, I outline the building blocks necessary to formulate a novel theory of social bonding through joint action.  Within this theory, team click is proposed as a mediating construct, through which the bonding effects of joint action are maximised.

Researchers have pointed out that flow---a concept that shares many similarities with team click---has traditionally escaped rigorous scientific analysis \citep{Dietrich2010a,Slingerland2014}.  Dietrich, for example, explains that flow presents a paradox that is difficult to explain according to traditional theories of attention and mental effort, which tend to assume better performance, on any task, is associated with increased conscious effort allocated to that task \citep{Dietrich2004b}.  As \textcite{Slingerland2014} surmises, flow reflects a paradox in which an individual in flow appears to be ``trying not to try.''  Flow appears to entail spectrum of both effortful (intentional, mental) and seemingly effortless (implicit, automatic) cognitions---some of which appear to depend on the resources beyond the brain.  Team click, like flow, is a multidimensional process involving mental, embodied, emotional, and social dimensions.

In this dissertation, I suggest that the most effective way reconcile the paradox associated with flow, team click, and the mystery of group exercise more generally, is via a dynamical and integrative model of cognition, which has coalesced recently as the AIF.  In the following section, I outline the tenets of the AIF as a foundation for a theory of social bonding through joint action.




















\section{Core components of joint action: physical movement coordination, social resonance, and team click \label{sect:dfProblem}}
%\subsection{The degrees of freedom problem and its solutions in human movement}



\subsection{Dynamical coordination of physical movement}
Humans are hyper-social \citep{Tomasello2012a}, to the extent that our capacities for social interaction---particularly those that are more basal and implicit---are often taken for granted \citep{Wheatley2016}.  It is important to realise, however, that successful coordination of behaviour in particular, and the ongoing development and evolution of biological life more broadly, is a theoretically improbable and astounding achievement \citep{Schrodinger1944}.  In this section I outline the cognitive parameters of dynamical coordination of physical movement in joint action.

Neurophysiologist Nikolai \textcite{Bernstein1967} was the first to point out the astounding computational challenge associated with coordinated physical movement in multi-component living systems.  In the case of human intra-personal movement, for example, an impressive balance is somehow struct between flexibility, precision, and control, whereby hundreds of muscles and joints coordinate to perform many different tasks of everyday life.  When we grasp a cup or catch a ball, many individual muscles and joints---each with their individual degrees of freedom---work together in fine concert.  Bernstein found it unlikely, from a computational perspective, that the central nervous system would be able to finely control all the possible movements (the degrees of freedom) of each single muscle individually to create coherently directed movements---it would be computationally impossible.  Rather, he suggested that muscles and joints form  ``functional synergies''---flexible function-specific self-organising assemblies---by locally coupling and constraining each other’s degrees of freedom to greatly reduce the amount of control needed.

Intra-personal and interpersonal human movement both rely on the nervous system’s capacity to anticipate, attend, and adapt to the conditions of the environment \citep{Keller2014}.  In the case of intra-personal human movement, successful coordination is made possible by the nervous system's direct access to information concerning the spatiotemporal position of the body and its relationship to the environment.  Privileged access to this information enables habituated coupling of movement control system to the organism's various degrees of freedom, and therefore the emergence of self-organisng functional synergies \citep{Riley2011}.

By contrast, interpersonal movement coordination poses a much more significant challenge.  In the case of interpersonal coordination, the combination of 1) limited reliability of sensory modalities as a source of information about the action of others \citep{Wilson2005,Wolpert2003,Frith2007} and 2) the informational complexity associated with a cognitive system comprising multiple autonomous agents \citep{Turvey1978} means that successful coordination in joint action is an inherently difficult and highly improbable cognitive challenge.  How exactly do humans face up to this cognitive challenge, and, quite often, successfully surmount it?

An answer to this question requires close attention to the various sophisticated cognitive, behavioural, and cultural capacities that define human's species-unique evolutionary niche \citep{Roepstorff2010,Clark2015,Fuentes2016}. Our ability for technical manipulation of extra-somatic materials and ecologies; advanced theory of mind; and information-rich, malleable, and scalable communication systems affords complex behavioural repertoires involving interaction between groups as small as two and conceivably as large as two million \citep{Pacherie2012,Nowak2017}.  In this dissertation, I focus on joint action involving dynamic ``on-line'' interpersonal coordination of physical movement.

While joint action is daunting and improbable from a cognitive perspective, it is also clear that humans have devised mechanisms through which joint action is effectively established and maintained.  In the following section I review a minimal architecture for joint action and cognitive theories capable of accounting dynamical coordination of physical movement.

%To establish and sustain interpersonal coordination, participants selectively recruit multiple behaviours and processes, which in turn become more interdependent and constrained by the function of ongoing activity.
%Importantly, it appears that successful coordination of behaviour in joint action can be rewarded in the form of powerful psychophysiological states of arousal, euphoria, and eudemonic wellbeing.





\subsection{Solutions to the challenge of joint action}

Cognitive scientist Natalie Sebanz and colleagues have proposed a minimal architecture for joint action \citep{Sebanz2006,Vesper2010}, which requires that individuals:

\begin{enumerate}
  \item Represent a shared goal, as well as representing one’s own individual contribution to the shared goal.
  \item Apply monitoring and prediction processes to each partner’s actions. This includes monitoring the extent to which shared goals or tasks are being fulfilled while at the same time predicting a partner’s actions.
  \item Facilitate continuous coordination via coordination smoothing, defined as the process of continuously improving one’s prediction of the partner’s action.
\end{enumerate}

Based on current understandings of human cognition motor control, it is unlikely that adherence to these minimal requirements of representing, monitoring, and sustaining dynamic joint action could be driven solely by central, executive control processes in the central nervous system---joint action would be too slow and costly.  It is similarly unlikely that processes of unconstrained automatic alignment (with no higher-level coordination) could possibly enable joint action of the type observed in humans \citep{Fusaroli2014}.  Rather, it seems more plausible that joint action is facilitated by a continuum of mechanisms spanning executive control through to direct coupling with the environment.  Below I outline 1) existing theory and 2) empirical evidence that supports this minimal cognitive architecture for joint action.

\subsubsection{Cognitive theories of representing, monitoring, and sustaining joint action}





\myparagraph{Representing joint, self, and other}

Stimulus-response conception of a-model representation
Modelling view of cognition, based on non-equilibrium thermodynamics

    %\subsubsection{Thermodynamic cognition\label{sect:thermoCog}}
    AIF has its roots in conception of human cognition that adheres to the thermodynamic constraints of living systems \citep{Yufik2017}.  In particular, the Second Law of Thermodynamics suggests that all such systems are subject to entropy---the tendency of a closed system to dissipate toward an equilibrium state of disorder or chaos \citep{Wolfram2002}.  Over evolutionary time, biological systems have devised canny ways in which to minimise or negate entropy by maintaining an equilibrium of internal order distanced from the external equilibrium of disorder \citep{Schrodinger1944}.  The current claim is that the human brain---a ``measure-preserving dynamical system'' \citep{Friston2013}---is no exception to the Second Law, and as such, the processes of information transfer  performed by the brain (cognition) must ultimately adhere to the minimisation of information-theoretic ``free energy'' in an organism's exchanges with the environment \citep[entropy can be understood as the average quantity of free energy to which an organism is subject][]{Yufik2002,Yufik2013,Friston2010,Sengupta2013,Sengupta2016,Sengupta2017}.

    The drive of the organism to minimise free energy is known as the ``Free Energy Principle'' (FEP), and can be understood as being relevant to cognitive and evolutionary timescales \citep[see][]{Friston2010}.  Biological life's adherence to the FEP can be identified at both ontogenetic and phylogenetic timescales: an organism's genome represents the accumulation of informational variants most calibrated to the regularities of the evolutionary niche in which it has evolved (and therefore most likely to reduce free energy); an organism's phenotype represents an organism's best attempt to reduce free energy over the course of its lifespan \citep{Ramstead2017}.  The universality of the FEP across proximate and ultimate levels of biology offers an opportunity to better understand the dynamical and self-organising principles (in additional to principles of Darwinian and general selection) that cross-cut and connect these levels of analysis \citep{Caporael2001,Badcock2012,Laland2015,Ramstead2017}.

    \myparagraph{The modelling view of thermodynamic cognition}
    It appears that the most successful strategy for free energy minimisation is for a living system to incorporate models of the system and its relation to the environment \citep{Conant1970}.  While primitive animals possess small repertoires of genetically fixed, rigid models, more advanced animals possess larger and more flexible repertoires that are amenable to experience-driven modifications (learning).  In humans, implicit models become responsive to self-directed composition and modification based on ``interoceptive'' (stimuli produced within the organism), in addition to purely ``exteroceptive'' (stimuli that are external to an organism) feedback \citep{Yufik1998,FeldmanBarrett2015}.  Importantly, this ``modelling view'' of computation \citep{Grush 2001;Chirimuuta2014} refers to a specific kind of minimal, structural or analogical model based on statistical correlations rather than necessarily involving semantic or propositional content \citep[also known as ``generative models''][]{Friston2001,OBrien2004,Huto2015}.  In the modelling view of cognition, a model is said to ``represent'' a target domain only in the sense that the relations between its computational processes preserve the higher-order statistical, structural-relational properties of the target domain, which can be leveraged to guide adaptive action \citep[see][8]{Ramstead2016}.  In this sense, generative models do not create auxiliary ``representations'' of the world, rather, they are a direct correlation of the statistical regularities of the world to which the organism has been exposed over the course of its lifespan \citep{Ramstead2017}.

    Thermodynamic cognition and the associated modelling view allows for a reconceptualisation of cognition to encompass an entire spectrum of strategies designed to adhere to the overarching mandate of free energy minimisation. Crucially, this approach removes the conceptual partition between
    perception and action, or cognition and emotion.  Rather than being bound to either/or choices between fast or slow, habitual or mental, emotional or cognitive responses \citep[see][]{Kahneman2011}, AIF proposes a brain that negates entropy by flexibly deploying an hierarchical continuum of generative models that range in their statistical complexity. At the base of the hierarchy correlational models enable sensorimotor coupling with the features of the organism's environment.  At the peak of the hierarchy, elaborate models preserve higher-order statistical, structural-relational properties of the world, which can be deployed in systems in systems of communication such as human language.  What has been traditionally understood as cognition and emotion merely represent different types of informational inputs into these generative models.  Contextualising cognitive function within a higher-order mandate of the FEP sets the theoretical stage for the functional integration of physical, cognitive, and social processes in causal explanations of instances of human behaviour, such as those encountered in group exercise.







    \subsubsection{Monitoring and predicting}

    Researchers in cognitive science have for some time realised that, while adequate for experimental paradigms containing static stimuli, traditional stimulus-response models of cognition involving bottom-up sensory inputs and top-down feature detection \citep[see][]{Marr1985} are not feasible for explaining dynamical joint action involving multiple agents interacting across various sensory modalities and timescales \citep{Wolpert1997}.  More recently, it has been proposed that human cognition and motor control is driven by anticipatory mechanisms that allow for preemptive modelling of the probable sensory causes of the environment as or before they actually occur \citep{Wolpert2003}.  As cognitive scientists Martin Pickering and Andy Clark explain, ``predictions can power learning, help finesse time delays, and enable a suite of potent capacities for motor imagination and simulation-based reasoning'' \citep[6]{Pickering2014}.  Currently, all prevailing neuromputational models of cognition and action agree that prediction---through the generation of ``forward models''---is crucial for adaptive behaviour \citep{Wolpert2003,Clark2013}. Motor control models are generally understood to update via feedback mechanisms in which they discrepancy between predicted (the forward model) and actual sensory stimuli is calculated as ``prediction error,'' and used to inform the future production of forward models \citep{Wolpert1995}.

    Although it is widely agreed that prediction is important for managing the challenges of joint action, there is disagreement the neurocomputational architecture and overarching theoretical mandate

    In this dissertation, I focus in particular on a prevailing account of cognition and motor control known as active inference \citep{Friston2011}.  Whereas alternative neurcomputational theories suggest that forward models are generated via an \textit{auxiliary} mechanism separate to core mechanisms of perception of action \citep{Wolpert1997}, active inference proposes that processes of prediction are \textit{integrated} into perception and action itself \citep[for a more detailed review of Auxiliary Forward Models and Integrative Forward Models, see Appendix~\ref{app2:motorControl};][]{Pickering2014}.

    While existing theories of cognition have struggled to account for mysterious visceral experiences such as team click in group exercise \citep{Dietrich2004,Slingerland2014}, active inference offers a theory in which elements of these identifiable phenomena can be tested.  AIF conceptualises joint action as a dynamical process in which prediction, perception, action, and emotion, are functionally integrated in the ultimate pursuit of minimising free energy in the task-specific environment \citep{Clark2015}.  For team click in particular, the active inference offers a framework in the physical and social dimensions of dynamical joint action are causally linked.



    In essence, the best models not only represent the statistical properties of the environment, but \textit{preempt} them before they arise.




    %In this section I introduce the core tenets of the AIF in order to explain why the AIF is an important theoretical model for team click and social connection in joint action.  First, the AIF is rooted in a thermodynamic understanding of cognition \citep{Yufik2017}, which takes seriously the principle that living organisms adhere to non-equilibrium thermodynamics on an on ontogenetic as well as phylogenetic timescale \citep{Friston2010,Yufik2002,Sengupta2016,Linson2018}.  Second, the predictive coding paradigm \citep[see][]{Rao1999,Clark2013} offers the most plausible neurocomputational model for thermodynamic cognition, owing to its computational conservatism and unrivalled inferential flexibility \citep{Friston2006}.  Beyond the brain, the AIF proposes a reliance on cognitive resources distributed throughout the features of the organism’s environment.  These various resources, contained in the physical features of the environment and the activities and artefacts of human sociality, can be understood as ``affordances'' \citep{Gibson1979,Ramstead2016,Bruineberg2014}. These three concepts---thermodynamic cognition, predictive coding, and affordances---are the core tenets of the AIF, and lay the foundation for application of AIF to joint action.

    %are organised---on ontogenetic as well as phylogenetic timescales---to resist dissipation towards thermodynamic equilibrium \citep[i.e., entropy, uncertainty, or chaos; see][]

\subsubsection{Predictive coding\label{sect:predictiveCoding}}
The thermodynamic conception of cognition at the heart of the AIF is supported by the neurocomputational architecture of predictive coding  \citep[hereafter PC, see][]{Rao1999,Clark2013}.   As cognitive scientists Martin Pickering and Andy Clark explain, ``predictions can power learning, help finesse time delays, and enable a suite of potent capacities for motor imagination and simulation-based reasoning'' \citep[6]{Pickering2014}.  In essence, the best models not only represent the statistical properties of the environment, but \textit{preempt} them before they arise.
However, whereas alternative neurcomputational theories suggest that forward models are generated via an \textit{auxiliary} mechanism separate to core mechanisms of perception of action \citep{Wolpert1997}, PC proposes that processes of prediction are \textit{integrated} into perception and action itself \citep[for a more detailed review of Auxiliary Forward Models and Integrative Forward Models, see Appendix~\ref{app2:motorControl};][]{Pickering2014}.  In addition, traditional auxiliary forward models lack an overarching theoretical principle (such as the FEP) that can explain competing drives ``exploitation'' of predictable sensory landscapes on the one hand, and ``exploration'' of novel sensory landscapes on the other \citep[also known as the dark room dilemma, explained in detail below in Section~\ref{sect:surprise}]{Clark2013}.




\myparagraph{Coordination smoothing}


\subsubsection{Affordances\label{sect:affordances}}
In this section I introduce the concept of affordances as a necessary compliment to the generative models and its relevance to the AIF.  In brief, affordances can be understood within the AIF as extra-neural resources that couple with generative models to produce loops of action and perception \citep{Ramstead2016,Clark2015}.  In other words, affordances are objects that facilitate active inference.  The concept of affordance is crucial to facilitate an understanding of how active inference unfolds in real-world settings. While affordances have traditionally been studied narrowly as localised resources for basic perception \citep[e.g.][]{Fajen2011}, researchers working with the AIF have proposed an extension of the the concept of affordances to include a spectrum of objects, ranging from content-limited ``natural affordances,'' through to ``conventional affordances'' mediated by highly contingent human cultural conventions and institutions  \citep[see][]{Roepstorff2010,Ramstead2016}.

The theory of affordances, originally proposed by psychologist \textcite{Gibson1979} states that the world is perceived not only in terms of object shapes and spatial relationships, but also in terms of object possibilities for action.  This definition accords neatly with the propositions of the AIF, which posits Bayesian inference machines reliant on cognitive resources distributed throughout brains, bodies, and physical features of the task-specific environment in order to anticipate sensory encounters.  Put simply, an affordance is the attribute of a hidden cause in the environment that induces (through PC architecture) predictions about perception and action \citep[908]{Pezzulo2013}.  Repeated coupling between generative models and their extra-neural correlations in the environment give rise to dense causal relations between particular affordances and particular predictions.

Working within the AIF, \textcite[7]{Ramstead2016} propose that the extra-neural cognitive resources to which generative models couple (i.e., affordances) can be understood as a spectrum, with ``natural affordances'' at one end, and ``conventional affordances'' at the other \citep{Ramstead2016}.  Natural affordances pertain mainly to basic correlations between the organism and the environment that enable sensorimotor control and regulation: the physical features of the environment; the ground beneath our feet.  Conventional affordances, by contrast, often rely on consensus from other agents and culturally derived regularities; the ground beneath our feet may be carpeted, we may have our shoes on or off depending on specific cultural convention that affords such behaviour.  As Ramstead and colleagues explain:

\begin{quote}
  Successfully learned human conventions that govern action are also best conceptualised as affordances. Such affordances depend on shared sets of expectations, reflected in the ability to engage immersively in patterned cultural practices, which reference, depend on, or enact folk ontologies, moralities and epistemologies. We might call these ``conventional'' affordances \citep[7]{Ramstead2016}
\end{quote}

As \textcite[906]{Pezzulo2014} points out, PC hierarchies extend well beyond hypotheses concerning the source and reliability of immediate sensory inputs. At higher levels of the PC hierarchy, more profound regularities can be represented, such as long-term beliefs that are increasingly more removed from sensorimotor events. Indeed, the human capacity for social learning is such that higher order beliefs may be acquired mainly through cultural learning (rather than purely ``from the ground up'' via natural affordances). But, these conventional beliefs may still remain ``grounded'' through the linkage with lower-level sensory events).  Indeed, higher-order beliefs may need to remain grounded in sensory experience in order to retain long-term viability \citep{Ramstead2016}.  Thus, it can be predicted that a higher order belief such as that of social identity, for example, would require at least a minimal degree of grounding in sensorimotor experience in order to sustain the viability of the generative model(s) that support the belief.  As I explain below, the experience of team click appears to provide a powerful sensorimotor grounding for higher order processes of social bonding.

\myparagraph{Typology of affordances}
\textcite{Bruineberg2014} propose that affordances should be conceived of using a three-step topology. For any given agent, there exists a ``landscape'' of all possible affordances, within which an agent only engages with a specific ``field'' of affordances prescribed by the patterned coupling between predictive models and natural and cultural features of the specific environment that the agent occupies. The affordances to which an individual brain is coupled at any one moment can be understood as ``solicitations.''  This three-level topology of affordances allows for a conception of the way in which repeated agent-environment coupling can give rise to particular regimes of attention, action, and perception within a vast landscape of all possible affordances.

Importantly, for the purposes of this dissertation, when a specific field affordances or set of solicitations are conventional and shared, repeated coupling can generate regimes of \textit{shared} attention \citep{Ramstead2016}.  Conventional (cultural) affordances can thus be understood as regularities in the environment that cue multi-modal predictions for action that is shared and contingent on the participation of others, i.e. joint action.  This understanding incorporates factors that are traditionally understood to be psychologically explicit or ``external,'' such as societal values or similar cultural dimensions, \citep{Hofstede1991,Schwartz1992}, social practices and artefacts  \citep{Nisbett2003a}, as well as and ``internal,'' such as dominant modes of self-construal or dispositional and linguistic tendencies \citep{Markus1991}.  The concept of affordances enables an extension of the AIF beyond immediate sensorimotor processes, and into the domain traditionally understood as ``culture'' \citep{Roepstorff2010}.  Within the AIF, culture can be understood as varying regimes of joint attention, which shape trajectories of action and perception.

In sum, the concept of affordances is an indispensable compliment to generative models of the AIF.  As discussed in the previous section (Section~\ref{sect:predictiveCoding}, generative models can be understood as containing a spectrum of statistical complexity, ranging from basic (seemingly content-free) to elaborate (and seemingly content-rich) correlations.  The affordances to which these models couple (and therefore on which such models depend) can likewise be conceived of as existing on a spectrum, spanning natural and conventional (cultural) variants.  Importantly, the concept of affordances reveals the fact that higher-order generative models are contingent on couplings with higher-order affordances, i.e., culturally shared affordances.  In other words, higher-order cognitive processes are fundamentally social cognitive processes. In this conception, capacities for belief, reason, and language appear are fundamentally contingent social and cultural processes, rather than purely innate capacities for which humans are naturally endowed  \citep{Sperber1997,Henrich2015}.  The concept of affordances also offers an explanation for observable cross-cultural variation in action, perception, and attention \citep[see][]{Nisbett2003}.  Selective engagement with specific fields of affordances give rise to regimes of shared attention and beliefs that direct trajectories for individual and collective activity.  Thus, the AIF predicts a human universal cognitive principle of free energy minimisation, it also predicts that free energy minimisation will be shaped by specific trajectories of affordances to which individuals and groups are subject, and to which generative models couple.





\myparagraph{Summary of dynamical cognitive theories of joint action}

  flexible deployment of spectrum of strategies




\subsubsection{Empirical evidence for minimal joint action architecture}



\section{Active inference applied to joint action \label{sect:activeInfJA}}
As explained above, traditional theoretical models of human cognition entail a conceptual partition between cognition and emotion, mental simulation and habitual response, flexibility and efficiency \citep{Clark2015}.  As such, these models have struggled to account for the embodied and enactive dimensions of cognition identifiable in real-world instances of human behaviour such as group exercise.  Team click is a powerful subjective experience in which many of these conceptual binaries appear to collapse and traditionally distinct processes appear to co-occur.  The AIF approach promises a testable theory of the dynamical, embodied, and social dimensions joint action.  In this section, I outline existing attempts to apply the AIF to joint action, and point to predictions relevant to team click and social bonding in joint action.

The AIF, when applied to joint action, posits two (or more) Bayesian predictive brains committed to preemptively modelling the world (including each other) and coupling with appropriate affordances in the world in order to minimise free energy \citep{Moutoussis2014,Friston2015,Friston2015a}.  To achieve free energy minimisation, the sensory stimuli produced by co-actors in joint action must be pre-emptively modelled, just like other feature of the sensorium (as explained above, see Section~\ref{sect:thermoCog}).  This necessitates a scenario in which brain A has a model of brain B, which includes the fact brain B is modelling brain A, and so on---\textit{ad infinitum}.  The recurrent predictions of both brains about one another threatens an infinite runaway regress that could preclude accurate modelling of either brain due to the computational intractability of the task \citep{Moutoussis2014,Friston2015}.  As Friston and Frith demonstrate formally (mathematically), this recursion (and its computational complexity) dissolves if the models of the two brains are formally similar \citep{Friston2015,Friston2015a}.
When grounded in computational similarity, each brain is able to generate predictions of the sensory outcomes caused by itself and the other in the same way, owing to the fact that the state that brain A is in imposes constraints on the states brain B can occupy \citep{Richardson2015}.

The AIF for joint action therefore predicts that joint action will involve strategies that enable functional equivalence of predictive models between co-actors, and in turn, the minimisation of free energy.  In this section I outline three areas of research in which such strategies become evident.  First, it appears that dynamic coupling of the movement degrees of freedom shared between co-actors in joint action serves to enable and sustain computationally viable joint action.  Second, research pertaining to the link between perceptual cues and action commands (action-perception links) suggests that co-actors in joint action develop commensurate neural pathways for processing actions of self and other.  Third, research suggests that individuals in joint action deploy type-based social representations of self, other, and group in joint action in order to reduce the computational uncertainty pertaining to the actions, goals, and intentions of agents.

Taken together, this evidence supports a more integrative alternative to existing models of joint action, which rely on mechanisms of sensory prediction (of self, other, and joint actions) that are individually-bounded and auxiliary auxiliary mechanisms to explain the cognition of joint action \citep[][; for an AFM model of coordination in joint action, see Appendix ~\ref{app2:AFMapproachJA}]{Pesquita2017}.  By contrast, the AIF predicts a scenario in which two or more brains are able to accurately predict each other's behaviour based on the flexible tuning of exteroceptive (sensory information from the actions of others and the task environment) and proprioceptive (pertaining one's own contribution to joint action) prediction errors.  As I demonstrate below, embodied and enactive dimensions to these processes in joint action can explain a link between joint action, team click, and social connection.



\myparagraph{Dynamic coupling}
An active inference approach to joint action is grounded in thermodynamic cognition and dynamical systems theory \citep{Friston2013}.  Friston and Frith point out that generalised synchronisation is an inevitable and emergent property of coupling two systems that are trying to predict each other ~\ref{Friston2015}.  Generalised synchrony refers to the synchronisation of chaotic dynamics \citep{Barreto2003}.  The basic principle is as follows.  If the universe comprised two biological (free energy minimising) systems---you and me---then your states and my states have to be restricted to an attracting set of states that is small relative to all possible states we could be in \citep{Friston2015}.  This attracting set of states enforces a generalised synchrony in the sense that the state you are in imposes constraints on states I occupy \citep{Richardson2015}.  It is in this sense that generalised synchronisation is a fundamental aspect of coupled dynamical systems that are free energy minimising.  If we are both trying to minimise the free energy of our attracting set (by reducing surprise or entropy), then synchronisation will be more manifest \citep{Friston2015a}.

In the case of the cognition of joint action, generalised synchronisation can be understood as the level of interdependence between levels of hierarchical models generated by two more (Bayesian) brains.  Friston and Frith call this the ``shared narrative'' that enables joint action.  Within general synchronisation, ``identical synchronisation'' refers a moment in which two or more models show identical equivalence between each hierarchical layer of prediction \citep{Friston2015}.   In contrast to behavioural synchrony, general synchronisation refers to alignment on a cognitive (information-theoretic) level, as opposed to merely a behavioural level.

Evidence that musicians maintain shared representation of desired unified sound of an ensemble \citep{Keller2008}, track deviations from desired joint state \citep{Loehr2013}, and allow shared goal predictions to guide future action \citep{Loehr2016}, confirms the hypothesis that co-actors share a goal---a shared narrative or common ground for joint action.  But the AIF approach to joint action specifically predicts that the foundation for this shared narrative will be dynamical, i.e., it will involve coupling of system component degrees of freedom of co-actors in joint action \citep{Turvey1978,Schmidt1990}. Dynamic coupling can be identified by two core properties:  1) dimensional compression (potentially independent DF are coupled so that the synergy possesses a lower dimensionality than the set of components from which it arises) and 2) reciprocal compensation \citep[the ability of one component of a synergy to react to changes in others][]{Riley2011}.

An accumulation of evidence on multiple levels of human behaviour, from brain function \citep{Yufik1998,Sengupta2013}, to interpersonal interactions \citep{Kelso2009,Riley2011,Fusaroli2014}, to large-scale human societal dynamics \citep{Nowak2017} supports the existence of dynamic coupling in human social activity.  In the case of dynamic multi-agent joint action in particular, individuals have been found to couple and reciprocally constrain their movements reducing the overall control needed to maintain effective cooperation \citep{Ramenzoni2011,Ramenzoni2012,Riley2011,Schmidt1990}.  Studies of real-world joint action scenarios such as dancing, martial arts, and moving objects like furniture have revealed evidence for dynamic coupling between co-actors.  In these studies, specific component degrees of freedom are modelled as coupled oscillators \citep[using the HKB model, which describes the change in the relative phase between two oscillatory components. See][]{Haken1985,Kelso1986}.  Dynamic coupling of movement has been measured beyond dyadic synchronisation, in the analysis of sub-phases of team sports  \citep{Passos2014,Duarte2012} and group dancing \citep{Chauvigne2017}.  For a more thorough explanation of dynamic coupling in human movement systems, including methods for measuring dynamical quantities, see Appendix~\ref{app2:dynamicCoupling}.

This evidence is supported by recent research focussed on the movement signatures of joint action that resemble instances of ``identical synchronisation,'' or the idea of being ``in the zone'' with a co-actor. Working within the common dyadic ``mirror game'' paradigm, \textcite{Noy2011,Noy2015,Hart2014} have identified ``co-confident motion'' (CC motion) as a canonical movement pattern of synchronised motion characterised by smooth and jitter-less motion, without the typical jitter resulting from reactive control in more commonly encountered leader-follower patterns.  In CC motion, different players appear to shift their basic motion signatures to a movement shape that is altogether different from their individually preferred shapes \citep{Hart2014}. Importantly, the pattern of CC motion shares the same sine wave shape as the optimal solution of the minimum jerk model, a well-known motor control model for rhythmic motion \citep{Hogan2007}.  This evidence accords with the proposal that joint action is underwritten by a ``shared narrative'' that transcends individual action tendencies of self and other and produces a  ``we-mode'' of social cognition (exemplified by CC motion) \citep{Gallotti2013}.  Taken together, this evidence for dynamical coupling in joint action supports an AIF (over and AFM approach) for joint action (see Section ~\ref{app2:AFMapproachJA}).  Specifically, evidence suggests that an individual's possibilities for movement will be constrained by the degrees of freedom of others in joint action \citep{Richardson2015}.


\myparagraph{Action-perception links}

The second area of evidence that supports the AIF for joint action pertains to the observable links between action and perception in joint action.  Parallel strands of research in psychology \citep{Prinz1990,Prinz1997,Prinz2013}, neurophysiology \citep{Rizzolatti2004,Rizzolatti2010}, and neurocognition \citep{Wolpert1998,Wolpert2000} suggest that interpersonal behavioural coordination in joint action is facilitated by the intrinsic links between action perception and action execution in the human brain.  In essence, action-perception links refers to the ostensive co-occurrence of a stimulus for action and its motor representation.  For example, for individuals who have mastered a certain sensorimotor task, the representation of a perceptual effect (say the sound of a middle-C on a piano) can trigger the movement necessary to produce the effect itself (motor instructions for playing the middle-C key on a piano) \citep{Novembre2014}.

Evidence suggests that skilled individuals not only develop generative models for self action, but also for the actions of others in joint action \citep{Novembre2012}. Action-perception links can be used for monitoring and integrating (e.g., timing or combined pitches) the actions of other ensemble members with self-generated actions \citep{Loehr2013}, and these effects appear to be stronger in individuals with high perspective taking skills \citep{Novembre2012,Loehr2013}.  The overlap between mechanisms for action production and action observation suggests that individuals may represent their own and others’ actions in a commensurable format.  Training-induced motoric representation of self and others' actions may facilitate various capacities important for joint action, such as prediction, adaptation, and entrainment (for a more detailed treatment of action-perception links and their relevance to joint action, see Appendix~\ref{app2:actionPerceptionLinks}).

From the perspective of the AIF for joint action, the development of action-perception links in skilled co-actors offers evidence for the development of functionally equivalent models for joint action.


\myparagraph{Social heuristics of ``you,'' ``me,'' and ``we''}























\section{Joint action in group exercise \label{sect:activeInfGE}}

Joint action in group exercise creates a unique environment for social cognition, defined by extreme levels of cognitive uncertainty.  Group exercise involves high cognitive load associated with complex joint action requirements involving many actors, many hierarchical layers of joint goals over various sensory modalities and spatiotemporal scales. The ``in the moment'' execution demands also constrain cognitive processing, and encourage a reliance on cognitive resources located in extra-neural and bio-external domains \citep{Bourbousson2016}.  In addition, strenuous physical exercise could also entail neurocognitive tradeoffs that further strain individual ability to reduce free energy in joint action \citep{Dietrich2004b}.

Evidence discussed above suggests that optimal solutions to joint action typical of group exercise may tend to favour the recruitment of more extra-neural resources as a way of minimising free energy, whereas less efficient solutions to joint action may rely on more computationally intensive procedures in order to reduce free energy (see Section ~\ref{sect:extraNeural}).  The social cognition of these processes in joint action have not yet been closely considered \citep[but see ][]{Marsh2009,Lumsden2012}.




    \subsection{Group exercise spikes uncertainty of joint action}


    \myparagraph{Group exercise involves multi-agent (and not just dyadic) joint action}
    Above and beyond normal day-to-day instances of communication and exchange, joint action in group exercise contexts such as sport place extreme cognitive load on participants. The active inference approach to joint action outlined above is based on preliminary models of dyadic joint action involving turn taking \citep[i.e., in bird song exchanges][]{Friston2015}.  Group exercise contexts, particularly modern sport contexts, often involve large numbers of co-participants, in either ``inter-active'' or ``co-active'' modes of coordination.
        \footnote{
        In co-active sports (e.g., bowling, archery), team members perform separately and the team outcome is a product of combined individual performances. In interactive sports (e.g., volleyball, soccer, rugby), goal accomplishment requires the establishment of complex patterns of interaction and coordination among team members \citep{Filho2014}.
        }
    Thus, achieving cognitive synchronisation in joint action of group exercise contexts may be much more difficult.  Evolutionary Anthropologist Robin Dunbar \textcite{Dunbar1992} proposes that the ratio of human neocortex size to total brain volume imposes an upper cognitive limit on real-time coordination of behaviour of approximately four to five individuals.  The sheer computational burden of modelling multiple agents in group exercise may place an unmanageable cognitive load on our normal healthy processes of active inference.  Indeed, at the very least, multi-agent joint action poses a challenge for the existing theoretical model for joint action (PJAM), which is formulated primarily based on dyadic interactions \citep{Pesquita2017}.

    \myparagraph{Joint action in group exercise is ``on-line'' and ``in-the-moment''}
    Particularly in the case of interactive team sports, interpersonal movement coordination is often executed ``on-line'' and ``in the moment,'' as opposed to step-by-step turn taking.  This fact poses a challenge to Friston and Frith's proposal that active inference in joint action comprises two modes (either actively attending to sensory stimuli, or else moving while in a state of sensory attenuation, see Section ~\ref{sect:activeInfJA} above).  Exactly what occurs when actors need to concurrently move and sense others moving at the same is poorly understood.  What happens to the precision weightings---the volume gauges---on proprioceptive, interoceptive, and exteroceptive prediction errors In instances of dynamic interactive joint action involving co-occurence of movement between agents in joint action?   What is the impact of on-line and in the moment join action on the experiences of agency in group exercise?  Empirical research is yet to provide answers to these questions.

    \myparagraph{Joint action in group exercise often involves competition}
    As if the cognitive load of multiple agents and on-line coordination of complex schemas for joint action was not enough for the humble human brain, interactional team sports also usually involves \textit{competition}.  While competition in sport is usually adorned with elaborate social meanings surrounding the ethics of winning and losing \citep{McNamee2008}, on a cognitive level, competition in joint action entails one individual or team of individuals actively attempting to foil or disrupt the predictive models of another individual or team of individuals \citep{Reimer2006}.  The competitive dimension of interactional team sports thus serves to further spike cognitive uncertainty between co-actors in joint action.  The uncertainty involved in competitive joint action scenarios could have further implications for the ability of second-order Bayesian inference concerning the reliability of sensory inputs \citep{Pezzulo2014}.

    \myparagraph{Group exercise involves metabolic tradeoffs in the brain}
    In addition to these heightened cognitive challenges associated with complex and dynamic joint action in group exercise, high levels of physiological exertion characteristic of group exercise could also serve to spike uncertainty.  Neuroscientist Arne Dietrich draws attention to the fact that physical exercise is at its core a stressor that places extreme energy demands the organism \citep{Dietrich2011}.
    Such a situation will necessitate an energy tradeoff in the brain, whereby energetically costly brain regions inessential to movement execution are temporarily downregulated \citep{Dietrich2004b}.  Dietrich suggests that experiences of flow and the ``runner's high'' in exercise could be the result of temporary downregulation of energetically costly brain regions inessential to movement execution  \citep{Dietrich2004b}.  Dietrich and colleagues propose candidate areas of the dorsolateral prefrontal cortex responsible for self-monitoring and proprioceptive sensory attenuation \citep[commonly known as the ``inner critic'' regions of the brain, see][]{Limb2008}.  It is currently not well understood precisely whether or how neurometabolic tradeoffs in the brain could impact upon joint action and the experience of team click.  However, it is conceivable that metabolic tradeoffs in the brain owing to prolonged physiological stress may have implications for the second-order inferential processes of precision weighting sensory inputs.


    \subsection{Free energy minimisation in group exercise demands greater reliance on extra-neural affordances \label{sect:extraNeural}}
    To summarise, the combination of cognitive demands associated with tracking and modelling multiple agents ``in-the-moment,'' the neurometabolic tradeoffs associated with (often extreme) levels of physiological exertion, and even the competitive dimension of some group exercise contexts (for example interactive team sports), could create an environment in which humans' usual cognitive capacities are strained and compromised.  The fact that experiences of team click are particularly prevalent in group exercise contexts (compared to more mundane or quotidian instances of joint action) suggests that amplification of uncertainty and stress in group exercise could be a critical factor in facilitating powerful psychosocial effects.

    The active inference approach would predict that individuals, when faced with the extreme cognitive uncertainty of joint action in group exercise, will tend to preference mechanisms that maximise uncertainty reduction (minimise uncertainty).  In the case of joint action in group exercise, this may entail reliance on predictive models that outsource the computational cost to affordances beyond the brain, or at least the metabolically expensive cortical areas of the brain \citep{Dietrich2004,Clark2015}.  In the case of highly skilled practitioners, whose predictive models for action have been finely tuned to the affordances of the task environment (including co-actors), it is plausible that extra-neural and even extra-personal resources (e.g., physical features of the task environment) could provide a more cognitively efficient and effective route to the performance of successful joint action and thus the minimisation of free energy.

    Various strands of evidence support these predictions.  Studies of highly skilled practitioners in joint action demonstrate that more technically competent practitioners generate more accurate predictive models for joint action than less technically competent practitioners \citep{Tomeo2012,Aglioti2008,Mulligan2016}.   In studies involving skilled versus non-skilled practitioners in dyadic interactions, it has been shown that more skilled practitioners create stronger dynamical coupling through flexibly modulating their actions with others \citep{Schmidt2011,Caron2017}. These findings are corroborated by other studies that find that professional footballers (versus novice controls) are able to more accurately predict the direction of a kick from another player's body kinematics (\cite{Tomeo2012}, see also \cite{Aglioti2008,Mulligan2016} for similar results with basketball and dart players).  When analysing co-regulation between members of basketball teams, \textcite{Bourbousson2015} showed that more expert teams made fewer mutual adjustments (at the level of the activity that was meaningful for co-actors), suggesting an enhanced capability of expert social systems to achieve and maintain an optimal level of awareness during the unfolding activity.

    A recent field study with expert rowers revealed that athletes predominantly utilised
    extra-personal (rather than inter-personal) regulation processes in order to facilitate and sustain joint action, and attention to interpersonal regulation occurred only during instances of expectation violation concerning performance in joint action \citep[; for a full explanation of this study, see Appendix ~\ref{app2:theory} Section ~\ref{sect:rowerStudy}]{RKiouak2016}. These results suggest that athletes used the affordances of the environment to mediate the arrangement of individual and joint activities \citep{Bourbousson2011,Bourbousson2012}.  Taken together, this evidence supports the prediction for a tendency for co-actors in joint action to utilise neurocomputationally conservative models and coupling with extra-neural affordances under circumstances of high levels of free energy (such as those common to group exercise).




















\subsection{Social resonance in dynamical coordination of physical movement}






















\section{Dynamical foundation for a novel theory  \label{sect:activeIn}}


    \subsection{Core tenets of active inference}
          \myparagraph{Thermodynamic cognition}
          \myparagraph{Predictive coding}
          \myparagraph{Affordances}

    \subsection{Application to joint action}
          \myparagraph{Dynamical coupling}
          \myparagraph{Action-perception links}
          \myparagraph{Type-based social representations}

    \subsection{Application to group exercise}






































\section{Team click mediates a relationship between joint action and social bonding}


    \subsection{Perceptions of performance in joint action predict team click}

      \myparagraph{Surprise}
      \myparagraph{Viscerality}
      \myparagraph{Agency}

    \subsection{Team click predicts social bonding}

    \myparagraph{Emotional Support}
    \myparagraph{Common goal}
    \myparagraph{Social Identity}










\section{Predictions of the theory}


    The overarching prediction of this thesis is that the psychological phenomenon of team click mediates a relationship between joint action and social bonding.

    Within this main hypothesis, I also formulate the following sub-hypotheses:
    \begin{enumerate}
      \item Athletes who perceive greater success in joint action will experience higher levels of felt ``team click.'' I predict that relevant perceptions of joint action success will relate to athlete perceptions of:
        \begin{enumerate}
          \item a combination of specific technical components; or
          \item an overall perception of team performance relative to prior expectations; or
          \item an interaction between these two dimensions of team performance.
        \end{enumerate}
      \item Athletes who experience higher levels of team click will report higher levels of social bonding.
      \item More positive perceptions of joint action success will predict higher levels of social bonding, driven by more positive:
      \begin{enumerate}
        \item perceptions of components of team performance; or
        \item violation of team performance expectations; or
        \item an interaction between these two predictors.
      \end{enumerate}
    \end{enumerate}

In addition to these core predictions, I also make the following predictions for the role variation to the theory based on cultural and individual variation:

\begin{enumerate}
  \item Individual variation in predispositions towards different movement coordination strategies will influence the relationship betweens joint action, team click, and social bonding.  In particular:
      \begin{enumerate}
        \item Athletes with more prosocial disposition (measured by personality type, e.g. extroversion) will experience higher levels of team click and social bonding in joint action.
        \item Athletes with higher levels of technical competence or experience will experience lower levels of team click and social bonding to the team, do to a lack of surprise associated with the experience.
      \end{enumerate}

  \item Informational affordances that are more dominant in an ecology will have a higher impact on shaping patterns of behaviour in joint action.

\end{enumerate}












%\subsubsection{Team click\label{sect:AIFteamClickJA}}
%As discussed above, the AIF for joint action, summarised in PJAM above, suggests that successful performance and social connection in joint action require not just exact behavioural synchrony, but generalised synchronisation of two or more free energy minimising systems \citep{Friston2015}.  Dynamical coupling of cognitive processes entails a continuum of generative models within brains tightly coupled to a spectrum of natural and conventional affordances---specifically the affordances of co-actors, but also other features of the neuro-external environment \citep{Clark2015}.

In this section I reduce the components of team click outlined above (Section~\ref{sect:dfProblem}) to three core dimensions—surprise, viscerality, and agency—--for which the mechanisms of the AIF offer predictions (see Table ~\ref{tab:teamClickMechanismsAIF}).  The mechanisms of the AIF relevant to team click are predominantly associated with processes of second-order Bayesian inference, specifically 1) affective reward mechanisms associated with ``exploitation-exploration'' dynamics in free energy minimisation \citep{Friston2012,Schwartenbeck2013,FitzGerald2014,Chetverikov2016},  2) reliability-based multi-modal sensory integration \citep{Ernst2004}, and 3) flexible attenuation \citep{Frith2007,Friston2015} of exteroceptive and proprioceptive inputs in dynamic joint action.

\newpage
\newgeometry{margin=0.5cm} % modify this if you need even more space
\begin{landscape}
% Please add the following required packages to your document preamble:
% \usepackage{booktabs}
% \usepackage[normalem]{ulem}
% \useunder{\uline}{\ul}{}

%\newpage
%\newgeometry{margin=0.5cm}
 % modify this if you need even more space
%\begin{landscape}


\begin{table}[]



  \begin{tabular}{@{}lllll@{}}
  \toprule
  \textbf{Component}                                                        & \textbf{Dimension} & \textbf{AIF mechanism}                                                                                                                         & \textbf{Predicted function}                                                                                                                                                                                                                                                                                                                                                                          & \textbf{Predictions for joint action}                                                                                                                                                                                                                                                         \\ \midrule
  Surprise                                                                  & \textbf{Affective} & \begin{tabular}[c]{@{}l@{}}Precision-weighting, \\ exploitation-exploration \\ dynamics \\ (Friston, 2013;\\ Chetkerov, 2016)\end{tabular} & \begin{tabular}[c]{@{}l@{}}Attention allocated to sensory\\ inputs as a function of prior \\ probability distribution; \\ Affect (surprise) allocated to \\ predictions as function of inverse \\ prior probability; i.e., higher \\ probability predictions receive \\ more attention but lower reward; \\ lower probability predictions\\ receive less attention but higher \\ reward\end{tabular} & \begin{tabular}[c]{@{}l@{}}Inherent uncertainty of real-world \\ joint action (the df problem, \\ Bernstein, 1967), makes \\ successful coordination improbable \\ but highly rewarding\end{tabular}                                                                                      \\
                                                                            &                    &                                                                                                                                                &                                                                                                                                                                                                                                                                                                                                                                                                      &                                                                                                                                                                                                                                                                                               \\
  Group flow                                                                & \textbf{Visceral}  & \begin{tabular}[c]{@{}l@{}}Multimodal sensory integration \\ (Pezzulo, 2014)\end{tabular}                                                      & \begin{tabular}[c]{@{}l@{}}More reliable sensory inputs will \\ receive greater reliability\end{tabular}                                                                                                                                                                                                                                                                                             & \begin{tabular}[c]{@{}l@{}}Unreliability of exteroceptive \\ predictions in joint action will generate \\ a decrease precision on exteroceptive \\ predictions, and increase  precision \\ of interoceptive predictions \\ (Pezzulo, 2014; Seth, 2013; \\ Feldman Barrett, 2015)\end{tabular} \\
                                                                            &                    &                                                                                                                                                &                                                                                                                                                                                                                                                                                                                                                                                                      &                                                                                                                                                                                                                                                                                               \\
  Tacit understanding                                                       &                    &                                                                                                                                                &                                                                                                                                                                                                                                                                                                                                                                                                      &                                                                                                                                                                                                                                                                                               \\
                                                                            &                    &                                                                                                                                                &                                                                                                                                                                                                                                                                                                                                                                                                      &                                                                                                                                                                                                                                                                                               \\
  Atmosphere or aura                                                        &                    &                                                                                                                                                &                                                                                                                                                                                                                                                                                                                                                                                                      &                                                                                                                                                                                                                                                                                               \\
                                                                            &                    &                                                                                                                                                &                                                                                                                                                                                                                                                                                                                                                                                                      &                                                                                                                                                                                                                                                                                               \\
  \begin{tabular}[c]{@{}l@{}}Blurred self and other \\ agency\end{tabular}  & \textbf{Agentic}   & \begin{tabular}[c]{@{}l@{}}Sensory attenuation \\ (Friston and Frith, 2015)\end{tabular}                                                       & \begin{tabular}[c]{@{}l@{}}Two heuristic modes of inference \\ in joint action: 1) action (attenuation \\ of proprioception); and 2) attention \\ to others\end{tabular}                                                                                                                                                                                                                             & \begin{tabular}[c]{@{}l@{}}Cognitive load of dynamic joint action \\ strains sensory attenuation;\\ blurs distinction between self and \\ other agency\end{tabular}                                                                                                                           \\
  \textit{}                                                                 &                    &                                                                                                                                                &                                                                                                                                                                                                                                                                                                                                                                                                      &                                                                                                                                                                                                                                                                                               \\
  \begin{tabular}[c]{@{}l@{}}Ability extended \\ by others\end{tabular}     &                    &                                                                                                                                                &                                                                                                                                                                                                                                                                                                                                                                                                      &                                                                                                                                                                                                                                                                                               \\
  \textit{}                                                                 &                    &                                                                                                                                                &                                                                                                                                                                                                                                                                                                                                                                                                      &                                                                                                                                                                                                                                                                                               \\
  \begin{tabular}[c]{@{}l@{}}Reliability of self \\ and others\end{tabular} &                    & Precision weighting                                                                                                                            & \begin{tabular}[c]{@{}l@{}}More reliable sensory inputs will \\ receive greater reliability\end{tabular}                                                                                                                                                                                                                                                                                             & \begin{tabular}[c]{@{}l@{}}Contributors to successful joint \\ action will receive greater precision \\ weighting\end{tabular}                                                                                                                                                                \\
                                                                            &                    &                                                                                                                                                &                                                                                                                                                                                                                                                                                                                                                                                                      &                                                                                                                                                                                                                                                                                               \\ \bottomrule
  \end{tabular}
\caption{Dimensions of team click and their relevant mechanisms as predicted by the AIF.}
\label{tab:teamClickMechanismsAIF}



\end{table}

%\end{landscape}
%\restoregeometry




\end{landscape}
\restoregeometry


%Considered from the perspective of the AIF, the experience of team click and social connection in joint action will derive not simply from exact coupling of sensorimotor processes (as in the case of behavioural synchrony), or else to more explicitly perceived alignment of shared expectations surrounding a joint task \citep[see][]{VanderWel2012}.  Rather, the click of joint action will theoretically entail an entire double helix of interlocking predictions and affordances spanning the most basic correlational models of sensorimotor coupling, through to the most densely semantic and proposition forms of explicit knowledge and communication.

In this section I reduce the components of team click outlined above to three core dimensions---surprise, viscerality, and agency---for which the mechanisms of the AIF offers predictions (see Table ~\ref{tab:teamClickMechanismsAIF}).  The relevant mechanisms of the AIF are predominantly associated with processes of second-order Bayesian inference, specifically 1) affective reward mechanisms associated with ``exploitation-exploration'' dynamics in free energy minimisation \citep{Friston2012,Schwartenbeck2013,FitzGerald2014,Chetverikov2016}, 2) reliability-based multi-modal sensory integration \citep{Ernst2004}, and 3) attenuation ~\citep{Frith2007,Friston2015} of exteroceptive and proprioceptive inputs in dynamic joint action.  In brief, the affective, visceral, and agentic dimensions of team click suggest that team click sets the foundation for social bonding.


\myparagraph{Surprise or positive violation of expectations in performance\label{sect:surprise}}
As explained in Section~\ref{sect:teamClickIntro}, team click in joint action is associated with positive affect: pleasurable, autotelic sensations of rush or flow.  While optimal performance in joint action is perhaps always the ultimate goal, particularly for elite practitioners such as athletes and music or dance performers, when team click occurs it also often appears to be unanticipated and surprising.  Thus, team click is experienced as positively valenced surprise, or positive violation of expectations around team performance.

Surprise is a term with various connotations on different levels of scientific analysis.  As an emotion, surprise can generally take one of two forms according to its affective valence.  Surprise can carry a positive valence (as in the case of team click, for example) or surprise can carry a negative valence \citep[in the case of shock or fright][]{Chetverikov2014}.  While the precise details of this link are yet to be fully understood \citep{Schwartenbeck2013}, the emotion of surprise thus appears to be linked with cognitive processes of prediction, specifically the conscious perception of discrepancy between a predicted states and an actual states \citep{Foster2015}.
Considered from the point of view of traditional (non-dynamical) predictive (forward) models of cognition (such as the AFM, see Appendix~\ref{app2:motorControl}), the function of negatively valenced surprise makes sense, at least intuitively.  If surprise is a signal of discrepancy between actual and predicted states, then surprise with a negative (aversive) valence function to motivate a reduction in prediction errors \citep{Egner2011}.  Without an overarching principle such as FEP, traditional AFM approaches lack a sufficient explanation of positively valenced surprise. If the goal of the predictive brain is to reduce prediction error, why would error-induced surprise be rewarded with such a positive sensation?

The puzzle of positive surprise relates to a more general paradox associated with traditional predictive forward models of cognition.  Mathematician and David Mumford evocatively described the problem as the ``dark room dilemma'':

      \begin{quote}
        How can a neural imperative to minimise prediction error by enslaving perception, action, and attention accommodate the obvious fact that animals don’t simply seek a nice dark room and stay in it? Surely staying still inside a darkened room would afford easy and nigh-perfect prediction of our own unfolding neural states? Doesn’t the story thus leave out much that really matters for adaptive success: things like boredom, curiosity, play, exploration, foraging, and the thrill of the hunt? \citep[243]{Mumford1992}
      \end{quote}

Although it is readily observable that animals live in a changing and challenging world and deploy quite complex strategies to survive and thrive within diverse and dynamic environments, original forward models lacked a sufficient theoretical justification for why the predictive drive did not simply lead to dark room stasis \citep{Clark2013}.  Encased within traditional linear and passive stimulus-response models of cognition, an account of co-existing and competing drives for familiarity (exploitation) and novelty (exploration) proves difficult \citep{Kelso2009}.  However, when considered from within the unifying AIF framework, the dark room dilemma dissolves, and the role of positive surprise becomes clearer.

The free energy principle overs an overarching explanation for why humans, far from being bound to seek out a dark room, are instead compelled to minimise free energy at multiple scales via co-existing strategies of ``exploitation'' (optimal free energy minimisation of a specific generative model), and ``exploration'' \citep[optimal free energy reduction of surprisal on broader scales, i.e., in the context of other generative models or the life of the organism more generally; see][]{Cohen2007}.

The AIF distinguishes between surprise as a conscious emotional experience, and surprise as an information-theoretic quantity roughly equivalent to free energy.  The FEP states, every system that maintains itself conforms to the imperative of minimising the surprise associated with the states it encounters \citep{Friston2012a}.  Surprise is defined here in an information-theoretic (rather than emotional) sense as an approximation of average free energy \citep[in fact, Tribus distinguished information-theoretic ``surprisal'' from surprise in an active attempt to separate the two concepts see][]{Tribus1961}.  Within the AIF, minimisation of surprisal exists on two scales: minimisation of long-term surprisal of the organism (i.e., entropy), as well as minimisation of surprisal in a given generative model \citep[i.e., free energy, see][2]{Schwartenbeck2013}. Given the complex and hierarchical structure of generative models, which exist across multiple timescales and sensory modalities,  minimising surprisal in the context of one model may generate surprise in the context of another model.

As Clark explains,

      \begin{quote}
        Change, motion, exploration, and search are themselves valuable for organisms living in worlds where resources are unevenly spread and new threats and opportunities continuously arise.  This means that change, motion, exploration, and search themselves become predicted and enacted accordingly \citep[193]{Clark2013}
      \end{quote}

Thus, while it is plausible that the emotion of surprise is somehow tethered to cognitive processes of prediction error minimisation, the AIF demonstrates that the relationship between perception of surprise and prediction error does not align in a one-to-one fashion. The emotional experience of surprise could conceivably span a full spectrum of positive and negative of emotional valence, depending on the location of surprisal in various generative models. In the case of team click, for example, positive surprise could arise form a scenario in which an individual is fixated on the exploitation (free energy minimisation) within a specific generative model (for example, focus on individual movement), and this model is violated (surprisal) by successful joint action of an alternative (and more rewarding) alternative generative model.

Researchers have hypothesised that exploitation-exploration dynamics are supported by a relationship between second-order Bayesian inference (precision-weighting) and affective reward  \citep{Friston2012,Chetverikov2016}.  First, precision (attention) will be assigned to error signals as an inverse function of the variance of their prior probability distribution.  In other words, if a prediction pertaining to a sensory input is highly likely (based on prior experiences), then the variance of its prior probability distribution will be low (positive kurtosis) and it will receive greater precision weighting.  Second, affective reward will be assigned to error signals as a direct (as opposed to inverse) function of the variance of their prior probability distribution. In essence, the less likely a prediction is to be correct, the more surprise (in an affective sense) it will entail.  Thus, while highly predictable inputs will receive a certain amount of affective reward, reward will diminish as unfolding action consumes the gradient of free energy, producing a spiky, less attractive reward.  Meanwhile, scenarios involving highly variable prior probability distributions, such as dynamic coordination of behaviour in joint action, promise powerful affective payoffs.

Neurobiological evidence generally supports this proposal, however it is yet to be thoroughly tested empirically.  Cortical processes of prediction error management appear to be mediated by the activity of the dopaminergic system \citep{Friston2012,Kakade2002,Schultz2016}, while subcortical neuromodulatory systems, such as those responsible for producing norepinephrine, acetylcholine, and endogenous opioids, appear to be involved in attuning cortical processing to signals from the body and environment that are important for survival \citep{Lewis2005}.

In sum, positive violation of expectation in joint action could pertain to the momentary transfer of attention away from more reliable inputs (such as the sensorium most proximate and controllable by an individual) towards sensory inputs deriving from dynamic general synchronisation.


\myparagraph{Viscerality}
The second dimension of team click is its viscerality, expressed in the components of group flow, tacit understanding, and team atmosphere.  The experience of team click is rooted in the body as something that is felt.  While not always able to articulate the sources of team click, athletes develop a fine-grained sensitivity for it; a gut feeling, or intuition.  What mechanisms are capable of explaining how an intricately complex cognitive phenomenon such as team click comes to be experienced primarily as a feeling rooted in physicality?

The AIF offers a mechanistic explanation of viscerality in joint action, based on the process of Bayesian multisensory integration \citep{Ernst2004}.  As explained in the wind vs thief example (see Section ~\ref{sect:windThief}), sensory evidence is weighted according to its reliability, judged probabilistically based on past experience.  It is intuitively plausible that for a Bayesian brain that wakes up in a dark bedroom in the middle of the night, the most immediately reliable sensory inputs will be those most accessible to the central nervous system;  exteroceptive inputs will be comparatively unreliable and will be precision-weighted as such \citep{Pezzulo2014}.  Exteroception in joint action entails an analogous reliability issue \citep{Sebanz2009}. Individuals in joint action face the challenge of modelling the various df of the task-specific movement system, which belong to autonomous agents capable of deviating at any moment---either accidentally or deliberately---from the dynamical common ground of joint action \citep{Keller2016}.  The prior probability distributions of exteroceptive predictions in joint action will therefore entail high variances, and accordingly attention will be assigned instead to interoceptive and proprioceptive inputs \citep{Seth2013}.  Consider too that active movement in joint action necessitates attenuation of proprioceptive inputs in order to enable smooth uninterrupted action ~\cite{Dietrich2004a,Friston2015}.  Thus, with both exteroception and proprioception dialled-down in dynamic joint action, the AIF predicts that interoception will function as the primary affordance for active inference.  As such, phenomena such as team click will be grounded in the body—as (gut) feelings and tacit understandings—owing to the reliability of these inputs in joint action relative to other inputs.

\myparagraph{Agency}
The agentic quality of team click is identifiable in components of experience that include: blurring of boundaries between self and other agency, the feeling that one's agency is extended by the contributions of others, and the perceived reliability of others and self to contribute effective execution of joint action (see Table ~\ref{tab:teamClickMechanismsAIF}).  When the team clicks, you and I become ``we'' (blurring), thereby you extend my ability beyond that which I could achieve alone (extension), and I experience certainty about mine and your capacity to contribute effectively to joint action execution (reliability).  What mechanisms enable blurring, extension, and reliability of agency in joint action?

Perception of agency in action has been formally defined as requiring the following conditions: 1) priority (whether the intention of action precedes the action), 2) consistency between the action and the original intention, and 3) exclusivity of explanations for the cause of the action \citep{Wegner1999}.  Generally, research of agency in (joint) action has focussed primarily on understanding circumstances in which people 1) experience agency over actions they do not produce themselves, or else 2) fail to experience agency over actions they do in fact cause \citep[see][]{VanderWel2012}.  Research using the theory of auxiliary forward models has established evidence for a strong correlation between attenuation of proprioception and experiences of self agency \citep{Wolpert2003,Sato2008}, as well as an inverse correlation between sensory attenuation and ascribing agency to sources external to the self \citep{Brown2013}.  As has been well documented in the case of schizophrenia, attribution of agency in social interaction may be modulated by individual variation in ``locus of control'' (the degree to which events are perceived to result from one’s own actions or not), and this may be related to improper function of the parietal cortex \citep{Frith2000}. In healthy adult populations of humans, meanwhile, \textcite{Sato2008} and colleagues suggest that discrepancy between prediction and sensory input can alter the experience of agency:  unpredicted sensory input can lead to ascribing agency for that input to an external source, for example, other participants in joint action or the external environment \citep{Sato2005,Frith2007}.

Empirical evidence addressing the question of how people experience agency in actions they intentionally produce in coordination with others is less abundant \citep[but hsee][]{VanderWel2012,VanderWel2013}. Current evidence suggests that perceptions of self agency in joint action may be most contingent on the condition of ``consistency'' (more so that priority and exclusivity) between predictions and prediction errors along a continuum of sensorimotor and higher order perceptual models \citep[see][]{VanderWel2012}.  van der Wel and colleagues, for example, show that perceptions of agency do not decrease when an individual transfers from performing a solo action to performing the same action with a partner (thus challenging the condition of exclusivity).  Individuals do however experience a boost in agency when they transfer from performing a joint action to the same action solo, suggesting that this transfer may induce enhanced consistency owing to the greater reliability of predictions pertaining to action that is individually controlled.

Together, this evidence accords with predictions from the AIF that agency in joint action will be most reliant on the condition of consistency, particularly when there is consistency between predictions pertaining to individual contributions to joint action and their (more reliable) prediction errors.  For the AIF, exclusivity should not be a necessary condition for perceptions of agency, as long as sensory routing between self and other(s) is effectively controlled by mechanisms of Bayesian precision-weighting \citep{Pesquita2017}.

It is possible that dynamical joint action will entail a strain on precision-weighting processes such that perceptions of agency will be altered.  The two heuristic modes of active inherence in joint action (active and sensory) described by Friston and Frith are based on models of dyadic joint action involving turn taking \citep[i.e., in bird song exchanges][]{Friston2015}.  Dynamic joint action, by contrast, requires continual on-line execution of movement, contemporaneous with the movement of others.  The AIF currently predicts that the volume on agency-correlated proprioceptive error signals will be turned down low while participants are executing movement, and otherwise will be held at locally optimal levels for attention to the movement of others.  The co-occurence between self and other movement could generate a conflict between processes of sensory attenuation, such that an individual is forced to sustain proprioception in dynamic joint action. This situation could  thus result in a blurring of the sense of self and other agency.  Indeed, it is plausible to predict that a tradeoff or middle ground between these two heuristic modes will be optimised in dynamic joint action scenarios. Thus, it can be predicted that such scenarios will be conducive to blurring of self and other agency (blurring). In turn, in scenarios in which the boundaries of agency are blurred, attention to the contribution (and reliability of contribution) to joint action by others will be increased, and iteratively strengthened with repeated instances of (successful) joint action execution.

To summarise, the AIF offers explanations for the three core dimensions of team click: surprise (positive expectation violation), viscerality, and agency.  The FEP dictates that human cognition is geared not only to exploit predictable free energy deposits, but also explore less probably but highly rewarding hypotheses, such as the prospect of optimal coordination in joint action.  Active inference in dynamical joint action will tend to place higher precision weighting on interoceptive inputs, and therefore generate inferences that are grounded in visceral sensations.  Third, dynamical joint action will place a strain on the modes of active and sensory attunement in joint action, and as such will lead to the modulation of experience of agency.   In the following section, I explain how the surprising, visceral, and agentic dimensions of team click set the foundation for social bonding.












\subsubsection{Social bonding\label{sect:AIFsocialBondingJA}}
In the previous section I use the AIF to formulate predictions concerning how joint action can be responsible for the surprising, visceral, and agentic dimensions of team click. In this section, I formulate testable explanations for how these dimensions of team click set the cognitive foundation for social bonding in joint action.  Social bonding in joint action includes experiences of emotional support and perceptions of common goal \citep[see][]{Dunbar2012,Wolf2015} as well as shared social identity between co-actors \citep{Whitehouse2014}.  I review each dimension of team click below and propose relevant mechanisms of the AIF.




\myparagraph{Dynamic coupling to conventional affordances}
The radical proposal of the AIF for joint action is that, unlike the AFM approach, the AIF formally theorises the way in which physical, cognitive, and social resources are shared between two or more brains, bodies, and the task-specific environment \citep{Clark2015}.  Therefore, social connection in joint action can be conceived as dynamic coupling between agents, which spans a continuum of basal sensorimotor processes, through to higher order beliefs facilitated by regimes of shared attention.  In this sense, social connection in joint action could arise in situations in which individuals perceive click between co-actors and other affordances contained within a certain field of activity.  In this dissertation, I develop the idea that in real world joint action scenarios such as interaction team sport, co-actors interact and rehearse their behaviours to produce a hierarchy of aligned representations, an implicit ``common ground'' \citep[see][]{Noy2017} on which joint action can unfold, and social connection may be found through team click.

%we are already connected!! (China, fusion)
%This ``shared narrative'' which supposedly transcends agency, is supported by the self-organising
%1. Generalised synchronisation through conventional affordances:patterned practice.

%leads to emotional support and perception of common goal

%rely on others to sanction the belief.

%vanderwel: haptic coupling learning result






\myparagraph{Team click demands higher order reflection}

While it may be near impossible to perform higher-order exegetical reflection on joint action ``in-the-moment'' of its execution, the attention dedicated to interoceptive and lower-level sensory motor prediction errors during joint action will ultimately need to be explained by higher levels of the generative models. Indeed, concepts such as ``tacit understanding,'' group flow, and team atmosphere are all examples of higher-order concepts that afford reconciliation between generative models and interoceptive prediction errors.  Team click does not exist as a concrete object that an athlete can see or touch, in the same sense that there is no material evidence for god or the bogeyman.  As Pezzulo suggests in relation to the bogeyman:

    \begin{quote}
      You quite literally recognise a bogeyman with your body, and with your fear in particular.  As a consequence, the bogeyman idea is a form of \textit{self-fulfilling prophecy}, because a terrified child can take his or her terror as evidence that the bogeyman exists (and is probably close), and the terror itself can increase due to the circular causality [of active inference] \citep[909]{Pezzulo2014}
    \end{quote}

In this sense, the bogeyman acts as a dummy hypothesis for beliefs that are grounded in interoceptive evidence.  Likewise, the AIF predicts that the interoceptive evidence pertaining to team click will demand affordances capable of reducing free energy of the generative model within which it resides.

This proposal resembles spontaneous exegetical reflection proposed by \textcite{Boyer2006}. The AIF provides a plausible unifying theory of how lower-order sensorimotor (Bayesian inferential) beliefs become associated with higher order (perceptual and semantic) beliefs.  The AIF proposes a process of coupling between an entire hierarchy of generative models with extra-neural affordances (to minimise free energy).  In the case of Hongwei, for example, while he was initially able to reproduce to me the explicit stereotypical discourses of group membership in his interview, he lacked an element of personal or implicit ``grasp'' \citep{Yufik2013}.  When he entered my room four months later, he had since embodied (internalised) belief. As such, it was completely transformative of his capacity to engage, relate, and interact. Here, the capacity of the AIF to account for the role of implicit and dynamical contributions to social cohesion and cultural transmission is rendered visible.








\section{Chapter overview}



\myparagraph{}








                                              \end{CJK}{UTF8}{gbsn}
