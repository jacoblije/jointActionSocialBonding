\chapter{\label{app1:intro}Appendix}



eCBs
A series of laboratory studies involving human and non-human subjects exercising on stationary exercise equipment (treadmills, watt bikes) show that sustained aerobic exercise at a moderate intensity ($\sim70-85\%$ of maximum heart rate)---but not low ($\sim45\%$) or high ($\sim90\%$) intensities---induces activity in the endocannabinoid (eCB) system \citep{Raichlen2013}, and similar results have been obtained in studies on the opioidergic system \citep{Boecker2008}.  Endocannabinoids appear to play an influential role in exercise-specific neurobiological reward, with studies showing activation eCB activation in moderately intense exercise and in cursorial mammals, such as humans and dogs \citep[but not non-cursorial mammals, e.g., ferrets;][]{Raichlen2012}.  In addition to direct peripheral (analgesic) and central (psychological well-being and alteration) effects, eCBs are also responsible for activating ``traditional'' neurotransmitters (opioids, dopamine, and serotonin) also responsible for rewarding and reinforcing behaviour \citep{Sparling2003}.  These findings lead Raichlen to propose eCBs as a key neurobiological substrate responsible for motivating habitual engagement in aerobic exercise, by generating ``appetitive'' and hedonic associations with exercise behaviour \citep{Raichlen2012}.







Grooming:


This ultimate evolutionary explanation for group exercise has its roots in studies of social grooming in non-human primates. Dunbar and colleagues propose a neuropharmacologically mediated affective mechanism linking dyadic grooming practices with group-size maintenance \citep{Machin2011}.  The capacity for social cohesion is thought to have arisen in primates as an adaptive response to the pressures of group living.  Aggregating in groups serves to reduce threat from predation.  At the same time, it can be individually costly due to stress arising from interactions at close proximity and conflict over resources among genetically unrelated individuals.  These pressures are hypothesised to have led to selection for social bonding (e.g., via dyadic grooming).  Resulting coalitional alliances among close partners allow for the maintenance of the group by buffering the stresses of group living.  Primate social grooming, for example, is associated with the release of endorphins, presumably leading to sustained rewarding and relaxing effects.  While other neurotransmitters such as dopamine, oxytocin, or vasopressin may also be important in facilitating social interaction, endorphins are argued to reinforce individuals (who are not related or mating) to interact with each other long enough to build ``cognitive relationships of trust and obligation'' \citep[1839]{Dunbar2012}.  It is thought that, as the homo genus evolved more complex collaborative capacities for survival in interdependent group contexts, grooming-like behaviours sustained social bonding in larger groups where dyadic grooming would cumulatively take too much time \citep{Dunbar2012}.

Experimental studies suggest that neurophysiological mechanisms activated by activities that involve physical exertion and coordinated movement, such as group laughter, dance and music-making, exercise, and group ritual can bring groups closer together, mediated by the psychological effects of endogenous opioid and endocannabinoid release \citep{Cohen2009,Fischer2014a,Fischer2014,Sullivan2014,Tarr2016,Tarr2015}.  Group exercise can in this sense be understood as a form of ``grooming at a distance.''

\subsubsection{The ``social high'' theory\label{sect:socialHigh}}

Existing evidence suggests that the link between group exercise and social cohesion could be explained in part by the way in which group exercise contexts foster psychophysiological environments conducive to forging social bonds---an important component to cohesive social groups \citep{Dunbar2010}.






\subsection{Knowledge gap in the science of sport and exercise}
It is in part due to the state of the science of sport and exercise that the project of cognitive and evolutionary anthropology of group exercise is yet to be formally defined and empirically substantiated. The anthropological science of sport and exercise---a field that should offer a rich source of theory and data---is currently marred by a bifurcation between sports psychology and social and cultural anthropology, with very little substantive theoretical predictions and empirical evidence existing in the space between.

Although social scientists of the past few centuries have occasionally included sport and group exercise as pseudo ritual-like activities responsible for producing social cohesion \citep{Durkheim1965,Mauss1935,Turner1977}, nonetheless, core biological, cognitive, and evolutionary questions surrounding sport and exercise remain largely unexplored by the behavioural sciences, including anthropology \citep{Blanchard1995,Downey2005a}.    While efforts to identify proximate cognitive and psychological mechanisms underlying the social function of religion and ritual began soon after the modern evolutionary synthesis \citep{Huxley1942} and cognitive revolution \citep[e.g.,][]{Turner1986,1987}—--the continuation of which has matured as a program of research in evolutionary anthropology and the cognitive science of religion \citep{Barrett2002,Lawson1993,Sperber1996,Whitehouse2004}---an equivalent program of research is yet to emerge around sport and exercise.

\myparagraph{Sport psychology}
Sport and exercise has been meanwhile been investigated within modes of scientific analysis concerned instrumentally with either the health benefits of exercise \citep{Fiuza-Luces2013,Morris1994}, or athletic performance \citep{Beedie2015a}, all the while neglecting to consider the cognitive or evolutionary role of exercise in human psychology or sociality \citep{Balish2013,Coulter2015}. A strict instrumental focus on athletic adherence and performance in sports psychology has restricted the psychology of exercise to an analysis of an ``athletic personality'' and the diverse motivations of the ``whole person,'' including the moral, ethical components of exercise, have been neglected \citep{Coulter2015,Laborde2014}.  As such, the cognitive and evolutionary dimensions of group exercise have escaped rigorous empirical analysis within the psychology of sport and exercise.

\myparagraph{Social and cultural anthropology of sport}
A broader spectrum of motivations and experiences in group exercise are reflected in social and cultural anthropological accounts. In these various monograph-length studies, authors focus less on the affective or sensorial experience of the athlete, and more on the personal and cultural commitments and meanings negotiated at the site of the athlete’s body in various cultural contexts.

Social anthropologists and sociologists have for some time emphasised the social function of exercise and sport in diverse cultural contexts, and various attempts have been made to analyse the phenomenological experience of exercise in terms of its sociological and psychological meaning \citep{Bourdieu1978}.  Social anthropologist Joseph Alter \textcite{Alter1993}, for example, argues that, for wrestlers in north India, the body functions as a nexus through which the symbolic and material structures of the state, family, and the individual coalesce.  In a similar vein, cultural anthropologist Susan Brownell \textcite{Bronwell1995}, in a seminal ethnography of sport in China, argues that sport functions as a crucial national symbolic practice for the Chinese nation-state in a project of ``rejoining the world,'' and that the ``micro-techniques'' (c.f. Foucault, 1977) of this project entail significant cost (and rich meaning) to the individual athlete.   Similarly, French sociologist Loic Wacquant \textcite{Wacquant2004}, in an ethnography of boxers in Chicago’s south side, describes a ``social logic'' of physical activity, claiming that ``the daily dedication and high technique that training demands; the regimented diet; the control, mutual respect, and tacit understandings necessary for actual fist-to-fist competition serve to create for the boxer an island of order and virtue'' \textcite[17]{Wacquant2004}.  In a recent extension of this line of ethnographic work, researchers have attempted to theorise the cognitive implications for different social-cultural frames of belief and understanding surrounding skill-acquisition and performance in sport and exercise \citep{Downey2005bDowney2007,Marchand2010}.

These samples of the social and cultural anthropology of sport represent an attempt to interpret an ethnographically observed connection between adherence to group exercise and the social and psychological meaning that appears to be tethered to this adherence.  While compelling and rich in ethnographic detail, these accounts do not explicitly engage in the project that is the central focus of this dissertation—--i.e., an \textit{explanatory} account of group exercise in human sociality.

In sum, existing scientific explanations of sport and exercise are split along two branches---one in which focus on causal physiological, cognitive, and social mechanisms of exercise is driven by priorities of athletic performance and public health outcomes, and another in which the study of group exercise forms part of a broader discipline in which social, ethical, and moral hermeneutics are more dominant than cognitive or evolutionary theory.  This bifurcation has created a scenario in which the evidence available for the cognitive and evolutionary anthropology of group exercise has been been derived either very close to the treadmill---from laboratory paradigms in which components and effects of group exercise (e.g., exact behavioural synchrony and direct or assay measures of neuropharmacological activity) are hyper-essentialised through experimental manipulation and operationalisation.  Alternatively, evidence has been derived very far from the treadmill---by ethnographic researchers whose priorities do not include identifying or testing causal processes. Cognitive and evolutionary approaches to the phenomenon of group exercise suffer from this bifurcation in anthropology, and would benefit from attempts to address the space between the treadmill and the field.












\subsection{Existing anthropology of group exercise}
A broader spectrum of motivations and experiences in group exercise are reflected in social and cultural anthropological accounts.  In these various monograph-length studies, authors focus less on the affective or sensorial experience of the athlete, and more on the personal and cultural commitments and meanings negotiated at the site of the athlete’s body in various cultural contexts.

Social anthropologists and sociologists have for some time emphasised the social function of exercise and sport in diverse cultural contexts, and various attempts have been made to analyse the phenomenological experience of exercise in terms of its sociological and psychological meaning \citep{Bourdieu1978}.  Social anthropologist Joseph Alter \textcite{Alter1993}, for example, argues that, for wrestlers in north India, the body functions as a nexus through which the symbolic and material structures of the state, family, and the individual coalesce.  In a similar vein, cultural anthropologist Susan Brownell \textcite{Bronwell1995}, in a seminal ethnography of sport in China, argues that sport functions as a crucial national symbolic practice for the Chinese nation-state in a project of ``rejoining the world,'' and that the ``micro-techniques'' (c.f. Foucault, 1975) of this project entail significant cost (and rich meaning) to the individual athlete.   Similarly, French sociologist Loic Wacquant \textcite{Wacquant2004}, in an ethnography of boxers in Chicago’s south side, describes a ``social logic'' of physical activity, claiming that ``the daily dedication and high technique that training demands; the regimented diet; the control, mutual respect, and tacit understandings necessary for actual fist-to-fist competition serve to create for the boxer an island of order and virtue'' \textcite[17]{Wacquant2004}.  In a recent extension of this line of ethnographic work, researchers have attempted to theorise the cognitive implications for different social-cultural frames of belief and understanding surrounding skill-acquisition and performance in sport and exercise \citep{Downey2005bDowney2007,Marchand2010}.

These samples of the social and cultural anthropology of sport represent an attempt to interpret an ethnographically observed connection between adherence to group exercise and the social and psychological meaning that appears to be tethered to this adherence.  While compelling and rich in ethnographic detail, these accounts do not explicitly engage in the project that is the central focus of this dissertation—--i.e., an \textit{explanatory} account of group exercise in human sociality.


\subsection{Bifurcation in the science of sport and exercise}

Noticeable gaps in cognitive and evolutionary accounts of group exercise can be in part explained by the specific history of the study of sport.  Although social scientists of the past few centuries have occasionally included sport and group exercise as pseudo ritual-like activities responsible for producing social cohesion \citep{Durkheim1965,Mauss1935,Turner1977}, nonetheless, core biological, cognitive, and evolutionary questions surrounding sport and exercise remain largely unexplored by the behavioural sciences, including anthropology \citep{Blanchard1995,Downey2005a}.

While efforts to identify proximate cognitive and psychological mechanisms underlying the social function of religion and ritual began soon after the modern evolutionary synthesis \citep{Huxley1942} and cognitive revolution \citep[e.g.,][]{Turner1986,1987}—---the continuation of which has matured as a program of research in evolutionary anthropology and the cognitive science of religion \citep{Barrett2002,Lawson1993,Sperber1996,Whitehouse2004}---an equivalent program of research is yet to emerge around sport and exercise \citep{Blanchard1995,Downey2005a}.  Sport and exercise has meanwhile been investigated within modes of scientific analysis concerned instrumentally with either the health benefits of exercise \citep{Fiuza-Luces2013,Morris1994}, or athletic performance \citep{Beedie2015a}, all the while neglecting to consider the cognitive or evolutionary role of exercise in human psychology or sociality \citep{Balish2013,Coulter2015}. A strict instrumental focus on athletic adherence and performance in sports psychology has restricted the psychology of exercise to an analysis of an ``athletic personality'' and the diverse motivations of the ``whole person,'' including the moral, ethical components of exercise, have been neglected \citep{Coulter2015,Laborde2014}.  As such, the cognitive and evolutionary dimensions of group exercise have escaped rigorous empirical analysis.

In sum, a review of existing literature in the anthropology of group exercise exposes a bifurcation of knowledge along two branches---one in which focus on causal physiological, cognitive, and social mechanisms of exercise is driven by priorities of athletic performance and public health outcomes, and another in which the study of group exercise forms part of a broader discipline in which social, ethical, and moral hermeneutics is more dominant than cognitive or evolutionary theory.  This bifurcation has created a scenario in which the evidence available for the cognitive and evolutionary anthropology of group exercise has been been derived either very close to the treadmill---from laboratory paradigms in which components and effects of group exercise (e.g., exact behavioural synchrony and direct or assay measures of neuropharmacological activity) are hyper-essentialised through experimental manipulation and operationalisation---or, alternatively, very far from the treadmill---by ethnographic researchers whose priorities do not include identifying or testing causal processes.  The existing social high theory of group exercise and social bonding suffers from this bifurcation in anthropology, and would benefit from attempts to address the space between the treadmill and the field.  In this dissertation, I actively address the knowledge gaps in the social high theory of group exercise and social bonding, using an inclusive human science integrating psychology, cognitive and neuroscience, and anthropology \citep{Whitehouse2012,Downey2014}.




\subsubsection{
  Anthropology of extreme ritual theorises cost and meaning
  \label{sect:extremeRitual}
              }
Evidence from the adjacent field of the cognitive and evolutionary anthropology of extreme ritual suggests a relationship between extreme physiological exertion and social bonding, pro-sociality, and cooperation \citep{Whitehouse2004,Xygalatas2013,Fischer2014a}. Researchers hypothesise that extreme ritual activates evolved neurocognitive processes of pain and reward \citep{Bastian2014,vanBunderen2014}, but it is understood that these mechanisms may not necessarily translate directly to feelings of positive affect, euphoria, or collective  effervescence (as the social high theory would suggest).  It has been suggested instead that the link between psychophysiological mechanisms of extreme ritual and social bonding may be mediated by cultural processes of group membership \citep{Whitehouse2004,Whitehouse2014a}.  For example, accumulating evidence suggests that extreme ritual practices increase levels of physiological arousal and personal agency, allowing for a process whereby individuals internalise group-level expectations as part of their own self-understandings to the extreme end of ``identity fusion''---a \textit{visceral} feeling of oneness with the group  \citep[hereafter simply ``fusion''][]{Swann2010,Swann2015}.

Whitehouse and colleagues propose that ritual practices responsible for generating higher levels of fusion have endured processes of cultural group selection \citep[or technically ``cultural attraction,'' cf.][]{Claidiere2007}, owing to the assemblage of adaptive benefits which they bestow to individuals and groups, among other contextual and cognitive factors of attraction, including memorability and transmission frequency in populations \citep{Atkinson2011}.  On balance, empirical data suggest that experiences of \textit{dysphoric} (more so than euphoric) shared experiences (of pain, torture, or warfare) strongly predict fusion (and in turn willingness to perform pro-group self-sacrifice,][]{Whitehouse2014,Whitehouse2014a}. These findings have been recently formalised in a mathematical model showing how conditioning cooperation on previous shared dysphoric experience can allow individually costly pro-group behaviour to evolve  \citep{Whitehouse2017}.

The anthropology of extreme ritual offers a pertinent example of how the social high theory could be developed to incorporate missing theoretical links, such as the link identified between extreme physiological cost and profound (social) meaning.  The relationship between extreme ritual practices and fusion appears contingent on behavioural affordances that are additional to those immediate physiological and cognitive mechanisms activated by the activity itself.  In this case, psycho-social processes of self construal \citep{Markus2003}, self-defining memory \citep[cf.][]{Moffitt1994}, and group membership \citep{Tajfel1971,Turner1987,Swann2009} all appear to play a supervening role in enabling social bonding, and---at an evolutionary scale---social cohesion.




\subsubsection{Group exercise resembles ``grooming at a distance''}
The ultimate evolutionary story of the social high theory has its roots in studies of social grooming in non-human primates. Dunbar and colleagues propose a neuropharmacologically mediated affective mechanism linking dyadic grooming practices with group-size maintenance \citep{Machin2011}.  The capacity for social cohesion is thought to have arisen in primates as an adaptive response to the pressures of group living.  Aggregating in groups serves to reduce threat from predation.  At the same time, it can be individually costly due to stress arising from interactions at close proximity and conflict over resources among genetically unrelated individuals.  These pressures are hypothesised to have led to selection for social bonding (e.g., via dyadic grooming).  Resulting coalitional alliances among close partners allow for the maintenance of the group by buffering the stresses of group living.  Primate social grooming, for example, is associated with the release of endorphins, presumably leading to sustained rewarding and relaxing effects.  While other neurotransmitters such as dopamine, oxytocin, or vasopressin may also be important in facilitating social interaction, endorphins are argued to reinforce individuals (who are not related or mating) to interact with each other long enough to build ``cognitive relationships of trust and obligation'' \citep[1839]{Dunbar2012}.  It is thought that, as the homo genus evolved more complex collaborative capacities for survival in interdependent group contexts, grooming-like behaviours sustained social bonding in larger groups where dyadic grooming would cumulatively take too much time \citep{Dunbar2012}.

Experimental studies suggest that neurophysiological mechanisms activated by activities that involve physical exertion and coordinated movement, such as group laughter, dance and music-making, exercise, and group ritual can bring groups closer together, mediated by the psychological effects of endogenous opioid and endocannabinoid release \citep{Cohen2009,Fischer2014a,Fischer2014,Sullivan2014,Tarr2016,Tarr2015}.  Group exercise can in this sense be understood as a form of ``grooming at a distance.''
