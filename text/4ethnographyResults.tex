\chapter{\label{ethnographyResults}Results of Ethnographic Research}

\textit{[Not included for CoS assessment]} \\
In this section I present the results of ethnographic research, the context and predictions for which were introduced in the previous chapter.

%\section{Preliminary Results}

%Evidence of a predominant relational mode of group membership
%I predict that professional Chinese rugby players understand group membership less as a function of categorical equivalence between abstract concepts of the self and the in-group, and more in terms of a relational mode of group membership, which entails varying levels of commitment to, and harmonisation of specific relationships that lead to self-enhancement.




%\subsubsection{Team as hierarchical family, not egalitarian assembly}

%As expected, ethnographic observations revealed a distinctly relational mode of self-construal and group formation.  Categorical modes of group membership were prominent in official team discourse: institute principals, coaches, and senior players often refer to the importance of selfless individual devotion to the team.  The activation of this categorical ``team'' consciousness, however, appears to be hopeful or aspirational, and forms part of an attempt to learn the foreign physical and social skills required for success in rugby.  In addition, attempts to promote this team consciousness (yishi) often appear to require the deployment of indigenous relational metaphors and logic: coaches and officials often refer to the team is ``a big family'', junior players refer to older players as elder brothers and so on,  almost to disguise the egalitarian pretense that these abstract categories demand.
%Indeed, when I probe team officials, coaches, and senior players in private conversations, about these discourses of categorical group membership, all, without fail, acknowledge the team’s and China’s general psychological deficiency in this, explaining that they are not behaviours produced by the Chinese system (tizhi).  Instead, it is claimed that team spirit, team-directed individual initiative, and intrinsic motivation are qualities cultivated by the educational and sporting systems of the West.

%In brief, while athletes appear able to switch frames to a categorical consciousness in order to process the technical and cultural demands of the team sport of rugby (J. Liu et al., 2009), a relational social identity still dictates the behaviours and dispositions of this process.  While athletes acknowledge and aspire to categorical modes of group membership (identification between categories of self and team) emblematic of modern Western sports such as rugby, adherence to the costly practices of rugby appear to be motivated by a more predominant commitment to strengthening and harmonising a network of hierarchically structured relationships for the ultimate purpose of individual-enhancement.

%I encountered many of the older athletes were keen to disparage to me, a retired Australian professional rugby player, that most Chinese rugby players didn't have the ability to work effectively as a team, and that their personal interests, motivations, and alegiences to other more \textit{familial} teams beyond the rugby pitch detracted from their ability to invest their ``spirit'' in the activity or cultivate full-on commitment to the sport.  In addition, the ``complexity'' of Chinese society was often implicated in this story - ``Chinese society is way to complicated...everyone is playing such a complex game, looking out for their own interests, its not as simple as it is for you overseas'' (not verbatim quote here).

%\subsubsection{Family first}
%During interviews, athletes ranked individual motivations for commitment to rugby (family (1st), education opportunities (2nd), earning respect of others (3rd)), before motivations based on the team (teammates (6th)), or enjoyment of the sport (8th).  In addition, there is distinct focus by senior members (coaches and players) on individual- (and not team-) centred commitment and discipline.  Individuals are subject to monetary fines for relatively minor discipline-related transgressions (for example, catching a common cold). In addition, I observe a pattern of technical instruction whereby coaches and senior players publicly single-out and scrutinise the technique of individuals for the benefit of the group, rather than for example making group-level generalisations about technical deficiencies.
%While an awareness of the importance of ``team cohesion'',(tuanjiue) team spirit (tuandui jingshen), and team-motivated individual initiative (zijue) is referred to aspirationally by coaches, senior players, and team officials at official meetings and team events, the same individuals admit to me in private conversations that the Chinese system of group relations does not produce innate motivation for self-less team contribution, and that Chinese athletes lack the essential characteristics of initiative and spirit that Western athletes appear to demonstrate.


%\subsubsection{Performance-related anxiety and agency over the team}
%In addition to interviews and participant observation, I conducted a number of surveys designed to understand athletes’ general and specific experiences group membership.  I conducted surveys following three training sessions: a session in which athletes (predominantly junior athletes) ran an aerobic fitness test involving straight-line running shuttle-running at and above the aerobic threshold (Beep Test), and two training sessions involving internal game-like scenarios.   Following a series of practice matches I administered a 9-item flow questionnaire (7 point Likert) to 10 junior athletes and eight senior athletes. The mean flow scores for senior (5.0) and junior (4.76) were equivalent. However, when asked if they 1) were not worried about their own performance, and 2) not worried how others would assess their performance (in the practice matches), senior players affirmed these questions (3.6 & 3.8) more confidently than junior players (1.5 & 2.4).
%In addition, I asked athletes about their general experiences of team membership agency over the team (weak-strong), role in the team (central-marginal), individual performance (weak-strong), team performance (weak-strong), training intensity (qiangdu)(light-heavy) at two three-month intervals (T1 and T2).  Selected results are collated in the table below (All responses are based on 7-point Likert).  I divide athletes into junior and senior categories based on training age (junior athletes 0-2 years; senior athletes 2-10yrs).

%Junior athletes report higher performance-related anxiety for the two game-like training scenarios (5.1 and 4.2) compared to the Beep Test (3.4), indicating that performance-related anxiety increases for junior athletes when training demands real-time performance of rugby-related skills.  This result is confirmed in interview responses, with junior athletes emphasising anxieties related to performance in game-like scenarios compared to fitness-only training sessions.  These observations supports the possibility that relationship between exercise-induced arousal and social bonding is moderated by technical competence in group-relevant behaviours.

%On average, junior athletes (5.1 and 4.2) appear to exhibit higher performance-related anxiety than senior athletes (3.2 and 3.25).  In line with the prediction that junior athletes would exhibit more pro-sociality than senior athletes, junior athletes (4 and 4.5) reported stronger overall team performance than senior athletes (3.6 and 3.9), whereas senior athletes (4.5 and 4.6) reported much higher feelings of agency over the team than junior athletes (2.3 and 1.7).  Similarly, senior athletes (4 and 3.9) reported a more central role in the team than junior athletes (2.8 and 2.9).  These findings are supported by interview responses, in which junior athletes pay more attention to the details team structures and relationships, whereas senior athletes devote more time to talking about themselves in relation to the team (figures to come).
%Interestingly, junior athletes reported higher training intensity (5.1 and 6.2) than senior athletes (4 and 4.8), perhaps indicating less habituation to the psychophysiological stress associated with full-time professional rugby training.

%Survey and interview data indicate that junior athletes exhibit higher performance-related anxiety and higher levels of pro-sociality but lower feelings of agency over the team; whereas more competent senior athletes exhibit lower performance-related anxiety, lower levels of pro-sociality, but higher levels of agency over the team.  In addition, observational and interview data suggest that athletes’s adherence to rugby is motivated primarily by individual and strategic goals, and that group membership is negotiated primarily using psychological tendencies ingrained by stable institutional and linguistic cultural environments.

%potentially supporting the prediction that less-competent junior players experience more “ethical dissonance” (however performance-related anxiety does not measure dissonance directly).



%\subsubsection{Exhilaration of skill acquisition and team click}

%One of the most striking results to emerge from the semi-structured interviews was athletes' testimonies surrounding the exhilaration surrounding notions of ``flow'' and team click experienced during high quality joint action.  Accounts of flow and team click were particularly focussed on the periods in which athletes first felt that they were developing a ``feel'' for the technical competencies of rugby, those first instances in which their expectations were positively violated.  Many terms were used to describe these feelings, including the incredible feeling of ``unspoken understanding'' between team members, that click derives from a strong team ``atmosphere''.
%I was able to corroborate these testimonies through longitudinal observation of junior athletes, who at certain points throughout the two-year period reported to me feelings of exhilaration about finally ``getting the feeling'' or ``sense'' of complex joint actions.  (Full analysis of interviews to follow Confirmation of Status)


%\section{Discussion of Preliminary Results}
%The performance anxiety and the exhilaration concerning team click was a sign of a smoking gun regarding the significance of coordination dynamics to social cognition of affiliation and bonding.  These phenomenological experiences emerge from beneath the surface of overt cultural discourses around group membership

%Discuss these results in terms of joint-action and social bonding - social cohesion
