

%%%%%%%%%%%%%%%%%%%%%%%%%%%%%%%%%%%%%%%%%%%%%%%%%%%%%%%%%%%%%%%%%%%%%%%%%%%%%%%%%%%%%%%%%%%%%%%%%%%%%%%%%%%%%%%%%%%%%%%%%%%%%%%%%%%%%%%%%%%%%%%%%%%%%%%%%%%%%%%%%%%%%%%%%%%%%%%%%%%
\section{Return to XNT Vignette}
The first associations between the Temple of Agriculture and sport began with a horse racing track located just outside its southern walls.  The kickball and polo games that featured in the Imperial life of the Song Dynasty (1960-1279) had all but vanished by the Ming Dynasty (1368–1644), and by the end of the Qing Dynasty (1644-1912), the only practices that closely resembled modern Western sports were horse races.  Horse racing functioned as an occasion for Manchu nobility to display their horses and finery, and also attracted a large contingent of spectators from the extra-jurisdictional international legations in Beijing.  The association between foreigners and horse racing in Beijing may explain why the grounds south of the Temple were known as the ``International'' racing grounds \citep{Brownell2008}. Indeed, this association may also account for why the site was one of the first targets of the Boxer Rebellion of (1900-1901), in which an army of anti-colonial Chinese subjects from rural areas of Shandong trained in martial arts and meditation surrounded and attacked the foreign legions in Beijing \citep{Brownell2008}.

In somewhat of a harsh twist of fate for the Boxer rebels, the influence of modern sport infiltrated the Temple of Agriculture even further in 1901, when the troops of the Eight-Nation Alliance (United Kingdom, The United States of America, Germany, Japan, Russia, Italy, and Austro-Hungary) arrived in Beijing to relieve the besieged international legions.  During this period, troops from the UK occupied the Temple of Heaven to the east of Yongding gate, while US troops occupied the Temple of Agriculture to the west. Compared to many other open public spaces in Beijing, the flat, open spaces of Beijing's temple grounds were conducive to the playing of Western sports common in garrison life such as field hockey and association football.

Throughout the final years of the Qing Dynasty, the sporting activities organised within the grounds of the Temple of Heaven and Temple of Agriculture attracted the participation of troops from other legions, as well as western missionary organisations such as the Young Men’s Christian Association (YMCA), and local Chinese elites \citep{Steel1985}.




Return to XNT Vignette:
When announcing to friends within the Chinese rugby community that I was preparing to conduct research with the Beijing rugby team, many warned me about becoming too involved, worried that I might suffer a similar fate to the group of coaches and athletes associated with the 2013 National Games controversy.  Adrian, for example, told me that the Institute was riddled with ghosts: ``There really are ghosts there, I'm telling you Lijie, you should keep your distance.'' (真的有鬼啊,我告诉你李杰,你最好离远点吧)

  Others hypothesised that the dramatic events of 2013 could be attributed in part to the blatant and irreversible disturbance of ``fengshui'' caused by the act of implanting a sports stadium on top of a sacred dynastic temple.

I naturally scoffed at these predominantly tongue-in-cheek warnings, although I was intrigued to learn more about the cosmological (as well as historical and political) principles that lay beneath the history of Beijing rugby and the Institute.

I had heard from members of the Beijing rugby community that the rugby program at the Institute had experienced a dramatic fall from grace after the drama of 2013.  All of the old guard of coaches, and almost all senior athletes from both the women's and men's squads left the Institute immediately after the Games in late 2013.  By all reports, the rugby program was all but deserted for at least 6 months after the games, before the Institute resurrected the men's program in 2014 by appointing a new pair of coaches (ZPH and SY) and enlisting the junior athletes from the previous National Games cycle to step in to the senior team.  The women's program was inactive for a full two years, and was only just starting to re-activate when I arrived in October 2015.  Having let the dust settle on the embarrassment of the women's program, the Institute had decided to continue with both the men's and women's team for the next 2017 National Games.


Meetings at the Institute:
I sat down in Jenny's office waiting for her to finish on a phone call. Jenny was a local Beijinger and a former National Champion athlete at the Institute in her chosen event of high jump. She spoke on the phone with all of the affectations that only a true Beijing local could produce---her word endings coloured with coarse yet elegant ``erhuayin'' suffixes, and the person to which she was speaking was addressed using the respectful version of the 2nd person pronoun ``nin3.''  Jenny possessed a rare and valuable combination of credentials that made her suitable for her position:  outstanding athletic achievement, and a Beijing residency permit.  China's infamously rigid ``Hukou''residency system means that it is still the case that only individuals with Beijing residency can hold permanent employment roles at government institutions such as the Institute.  Non-locals can become Beijing residents even if they were not born in the city, but the criteria for this process of naturalisation have become more and more stringent, and fewer and fewer applications are successfully processed, particularly in industries like sport.  Undoubtedly, Jenny's combination of outstanding athletic achievement and Beijing residency facilitated her post-athletic career as a coach and administrator at the Institute.  Listening to Jenny negotiate charismatically on the phone also reminded me of something that was clear during our first interactions in 2012, and that was that Jenny also appeared to posses a natural ability to manage the people and politics that came with the territory of her job.

Once she had finished with the call, Jenny welcomed me and tactfully explained to me that rugby at the Institute had indeed experienced a dramatic fall from grace. From its pre-2013 status as the Institute's flagship sport (including the bold performance targets of a gold medal for the women, and a top three finish for the men), rugby had dropped to the level of a sport that the Institute intended to continue to playing, but with no additional ambition beyond that goal. The men's team were performing at around the level of 3rd or 4th in the country, which was not terrible, but it was also against provinces who had yet to field their strongest teams before the 2017 National Games.  JXZ indicated that the head coach ZPH and his assistant SY really had their work cut out for them, and that my presence as observer and occasional coach would benefit the team.  She agreed to organise a room in the rugby program dormitory, as well as access to the Institute's canteen.


%Indeed, it seemed that the biggest performance goal for rugby was for nothing to publicly go amiss in the next round of the National Games.
%I didn't quite understand it at the time, but a large component of ZPH's was the lack of support he was receiving from the Institute leadership.  ``不出事就行''

%I later discovered that The original altar of the Temple of Agriculture had been preserved and restored, and was in the north-west corner of the campus.

After meeting with JXZ, I walked further north into the campus to where the rugby dormitory was located to meet with the head coach ZPH. My connection to ZPH went back to CAU in 2008, where he had been a coach at the time. From Shandong originally, ZPH was a graduate of of the Shanghai Sports University, another prominent rugby program at the time.  ZPH had been recruited from Shanghai to CAU by ZHJ to coach so that he could continue to play for the Chinese national team after he had completed his undergraduate studies.  After we had discussed my research and he had provisionally approved my plan to spend the next period with the team, I asked him about the current situation with the Beijing team.  ZPH explained that he was quite frustrated that the group of athletes he was coaching lacked experience and maturity. I asked him exactly what areas of the team's performance, and he indicated that all areas were not great, suggesting that not enough players had found that ``feeling'' for gameplay and very few were motivated to train hard.

The team were preparing for the final national Tournament of the season in Qingdao early the following week. The year's final national Tournament in Qingdao was planned to immediately follow the China leg of the Asia Sevens series also hosted in Qingdao. The After that the team would break for three weeks and then resume training for the following season after the China National Dat Holiday in early October.  We agreed that we would watch the Asian 7s and the National 7s in Qingdao this week and then reconvene in Beijing to discuss a strategy for the team's next chapter of training at the Institute.




%%%%%%%%%%%%%%%%%%%%%%%%%%%%%%%%%%%%%%%%%%%%%%%%%%%%%%%%%%%%%%%%%%%%%%%%%%%%%%%%%%%%%%%%%%%%%%%%%%%%%%%%%%%%%%%%%%%%%%%%%%%%%%%%%%%%%%%%%%%%%%%%%%%%%%%%%%%%%%%%%%%%%%%%%%%%%%%%%%%%
\section{Vignette: Qingdao}
Qingdao:
Kai's hometown was Qingdao, and so we travelled down together on the high speed train after Kai finished work on the Friday evening in time to watch the Asia Sevens Tournament on the Saturday.  The Asia Sevens is an annual series of three regional rugby sevens tournaments featuring men's and women's national sevens teams.  The men's series has been held regularly since 2009, and the women's series was established in 2013. The series usually consists of two three annual Tournaments, alternating between various locations including Hong Kong, South Korea, Sri Lanka, China, Japan, Malaysia, India, Singapore, and Thailand.  In 2015, The Asia Sevens Series Tournaments were held in Colombo, Bangkok, and Qingdao. Would be played on the Saturday and Sunday, and then the National Tournament would follow immediately after on the Monday and Tuesday.


Olympic Qualification:

When I arrived at the stadium in time for the beginning of the Asia Sevens Series, I met a number of coaches and athletes in the stands who I knew from my time in Beijing in 2008 and then coaching in 2013. I sensed from my various interactions that there was an air of nervousness around the Tournament, particularly on behalf of the Chinese women's team.   The Qingdao tournament was the final tournament before the Olympic Qualification Tournaments, to be held in Hong Kong and Tokyo in November 2015.  The top ranked team from those two legs would qualify for the Rio Olympics in 2016.  The Chinese men's team were not in serious contention for Olympic qualification, given the clear superiority of the more established men's rugby programs like Japan and Hong Kong. The Chinese Sports Commission did, however, expect the Chinese women's team to qualify for the Olympics.

Since the Chinese women's team was first established in 2002, China had made great strides in women's rugby, in both Asia and globally.  Not including occasional losses to closest rivals Kazakstan, between 2002 and 2012 the Chinese women's sevens team was the dominant women's team in Asia, easily outcompeting Japan and Hong Kong, and at times was competitive against the world's best including New Zealand and Australia.  The main reason for China's dominance in the women's game during this period was that other traditional rugby nations, despite having developed professional infrastructure for the men's game, lacked almost entirely an equivalent infrastructure for the women's game. China`s state sponsored sport system, on the other hand, was relatively agnostic towards gender in sport. According to the incentive structures of the Chinese sports system a gold medal is a gold medal, regardless of the gender of the recipient. (This is not to say that there are not distinct gender inequalities in relation to sport in China).
Indeed, beginning with the Chinese Women's Volleyball Team's gold medal victory at the LA Olympics in 1982, China has enjoyed a comparative advantage in women's sport due to the fact that the Chinese sport system was comparatively more supportive of women's sport.

Alarmingly for the Chinese women's rugby team's hopes of Olympic qualification, by 2014 it had become obvious that other more traditional rugby playing nations in Asia, namely Japan and Hong Kong, had begun to make up serious ground on China and Kazakstan in the women's game.  This naturally prompted nervousness among the GAS and CRFA.  In mid 2015 it was decided by the GAS and CRFA that they would enlist the services of a foreign coach.  According to sources close to CRFA, apparently the original plan was for the appointed foreign coach, BG, to act as a consultant for LXH and his existing group of coaches.  By the time I arrived in Qingdao in early September, however, the initial arrangement had since transformed into a situation in which BG was given 100\% control over the program as head coach, and LXH was more or less sidelined as coach.

The complication was that before being dethroned, LXH preferred to use his own athletes.  Most of the starting team at the LDN7s in June 2015 were indeed Shandong athletes, many of the women who had won gold at the National Games in 2013.  When BG took over the reigns as head coach, as directed by the GAS and CRFA, he set about reorganising the starting team and also scouting for new talent outside the squad, which was predominantly made up of Shandong athletes.

When I arrived in Qingdao it appeared that tensions between the old and new guard were at their peak.  LXH and many of his favoured athletes had been relegated to the bench, and some had been completely removed from the squad altogether.  There were 6 weeks to go before the all important first Olympic qualification tournament in Hong Kong.  I sat and watched the first few games of the women's Tournament, and I was indeed surprised to see that the Chinese women's side was missing some of its usual stalwarts, and indeed appeared in my eyes to lack the flow and familiarity that I had come to expect in LXH's clan.  In the stands I came across a group of Shandong women, one of which, QGG I knew quite well from when she toured to Australia with the Shandong team when I was still with the Australian rugby sevens team in early 2013.  I asked QGG why she wasn't playing for China, and it turned out that she was injured, and so wasn't eligible for selection.  But a few of the other women around her, who were all wearing Chinese national team uniforms, were all part of LXH's clan who had been effectively stood down by BG. ``How do you think they're playing?'' She asked me, after we had exchanged pleasantries. ``Not great'' I commented, hesitating, not knowing how much I should prime her, but also feeling obliged to be honest: ``it feels like they’re not coordinating together very well at the moment.  What do you think?'' (一般吧。感觉她们的配合不太好,目前。你觉得呢?) I asked.  ``They’re out there playing as individuals, not playing as a team! They can't get it together; there's no shared goal.'' `(她们都在打个人的,不打团体提的。打不到一块儿去啊,没有共同的目标.)  ``Hmm. Yes it does look like that.'' ``Hey, Lijie...'' she asked me quietly, ``...don’t you think they’re not even playing as well as our Shandong team could play?'' (嘿,李杰,是不是她们现在打的没有我们山东队打的好,是吧?)

As was common on my journeys through the world of rugby in China, I often did not quite grasp all the details and pieces of the puzzle that contextualised the interactions I was having until after the fact.

Later that day I bumped into LS, who had been assistant coach of China with LXH and now BG since 2014.  LS was a Qingdao local, a CAU graduate, and a member of the Beijing Men's team from 2010-2013.  I asked him about the current situation with the Chinese women's side and their prospects 6 weeks out from the first Olympic qualifier.  ``Chinese athletes need to see the (individual) benefits if they are going to put their bodies on the line and put in for each other'' he insisted, and he went on to explain why for these athletes, there were no obvious benefits available sufficient to motivate them.  There were indeed very few material benefits associated with representing China in rugby at the national level.  Athletes were payed a nominal USD100 per month on top of their provincial contracts when training and touring with the national program.  If athletes were injured while playing for China, CRFA at the time did not have access to sufficient health insurance to cover the costs of treatment, and athletes had no choice but to return to their provincial programs and seek treatment at the expense of the province.  The less tangible benefits of playing for China, for example, access to high quality coaching, or the pride of representing the country in a sport, or even the promise of a trip to the Olympics, were heavily outweighed by other less tangible costs: long stints of time away from family, the constant risk of falling out of favour with provincial programs...  In effect, the lack of incentives at the national level meant that athletes were by definition more committed to their provincial systems---the programs that provided athletes with the benefits that they were most interested in obtaining, such as tertiary education, future employment, modest but compared to CRFA, a reasonable salary (most national level players were paid 3-8k RMB/month). ``Its no wonder these athletes aren't performing well,'' LS exclaimed.

EXPLANATION: weakness of the institution?

Parallel to Beijing?


Stand talk about attributes and critical of technique etc.
Basketball:



National Women's teams: Li Sheng
Beijing Team: Huddle at the end






I arrived in Beijing late on a Friday evening at the end of August in 2015.  My close friend Kai---a former Chinese National rugby team representative, and now a lawyer working in Beijing---picked me up at the airport and drove me back to his home.  The plan was to stay with Kai until I was able to make solid arrangements with the Beijing Temple of Agriculture Sports Institute, the home of the Beijing Provincial Rugby Team.

After a day of acclimatising and running errands, on Saturday evening I was invited to a dinner hosted by Adrian, a respected elder within the Chinese rugby community of Beijing.  Adrian was the captain of the second ever class of rugby players to graduate from the Chinese Agricultural University in Beijing---the birthplace of rugby in China.
I first met Adrian through Kai in 2013,

 while coaching in China.  I was originally introduced to Adrian by Kai, a close friend of mine who I met earlier during another stint in Beijing in 2008.

 Kai was also at the dinner, as was Mr Shi, a sports television producer at Chinese Central Television (CCTV).  CCTV had just accepted the rights to the Rugby World Cup, which World Rugby---the international governing body of rugby---had made available to CCTV in an attempt to promote the development of the game in non-traditional rugby playing nations.


I arrived in Beijing late on a Friday evening at the end of August in 2015.  My close friend Kai---a former Chinese rugby player, graduate the Chinese Agricultural University (Chinese rugby's birthplace), and now a lawyer working in Beijing---met me at the airport and drove me back to his home.  When we got to his home, Kai turned on the television and we caught up while some footage from a rugby documentary played in the background.  As it so happened, the international rugby world was on the verge of another Rugby World Cup, which was being hosted by England in the coming months. World Rugby, the world governing body of rugby union, had made the television broadcast rights for the World Cup available to Chinese Central Television (CCTV), in an attempt to promote the game globally.  Having accepted the rights to the tournament, which is the 3rd largest sporting event in the world behind the Olympics and the Football World Cup, CCTV were in search of Chinese rugby experts to help produce the 48-match tournament.  As Kai quickly explained, CCTV's search had led them to the Chinese rugby community based in Beijing, who were almost all, like Kai, graduates of the Chinese Agricultural University---the birthplace of rugby in China and the base for the Chinese National Rugby team between 1996 and 2010.  In fact, CCTV's search led them first to Adrian, the captain of one of the first CAU rugby teams (1992), and currently working for a large international sport organisation in Beijing.  Adrian had then contacted his younger university brother (师弟) Kai, who like him was fluent in English and able to assist in sourcing and translating rugby materials relevant to the broadcast. The CCTV producer responsible for the broadcast, Mr Shi, had scheduled a dinner with Adrian and Kai on Saturday (tomorrow) night to thank them for their willingness to assist in the production.  I was also invited to the dinner. I wasn't scheduled to meet with the Principle and head coach of the Beijing Temple of Agriculture Sports Institute until the following Monday, so I agreed to accompany Kai.






First Week in Beijing:

The night after I arrived in Beijing in late August 2015, I was invited to a dinner hosted by Adrian, an elder of the Chinese rugby community in Beijing.  Adrian was the captain of the second graduating class of rugby players the Chinese Agricultural University (CAU), the home of China's first official rugby union program established in 1990.  I first met Adrian two years earlier in 2013, through a good friend Kai. Kai was  a more recent CAU graduate, a former Chinese National rugby team representative, and now a lawyer in Beijing.  Kai was also invited to the dinner, as was Mr Shi, a sports television producer from Chinese Central Television (CCTV).  The background to the dinner was that World Rugby, the international governing body of rugby, had recently given CCTV the broadcasting rights to the 2015 Rugby World Cup, to be held imminently in England during September and October 2015.  Mr Shi had been charged with the production of the 48-match tournament, which would be the first time international rugby was televised on Chinese national television.  Mr Shi was completely new to rugby, and so needed help making rugby accessible and understandable for Chinese audiences.  Shi reached out through his network in Beijing and soon tracked down Adrian, who was working in the digital media department for NBA China. Adrian in turn tracked down Kai. Both Adrian and Kai were well connected to the Chinese rugby community and fluent in English, and so were well placed to assist Mr Shi in the tasks of translating relevant rugby materials and organising expert commentators for the broadcast.  My arrival in Beijing was timely for this project, and Kai was quick to recruit me to join them at dinner.  I was eager to begin my fieldwork and so, despite my mild to moderate jet lag, I accepted the invitation and set off to the restaurant on the Saturday evening with my notepad and audio recorder (i.e., my mobile phone) in hand.

The Dinner:
Adrian, Kai, and I waited for Mr Shi to arrive in the upstairs area of the Korean BBQ restaurant in a quiet Peking willow lined street just inside Beijing's East 4th Ring Road.  Adrian, as host and elder, held the floor as we waited: he reminisced fondly about his time playing rugby at CAU as well as his time in Beijing after graduating, when he played with the Beijing Devils, a predominantly expat rugby club in Beijing:

  ``Rugby was so much fun in those days, not like today (in the professional era of Chinese rugby).  Everyone was just scraping together the money to go on tour, we all payed our own way, sometimes you'd get a bit of help from someone or whatever. It was for the love of the game, not for any other reason.''

When Mr Shi finally arrived, Adrian continued the nostalgic story telling mode but naturally shifted his target audience from Kai and me to Mr Shi.  When describing in rich detail the experience of participating in an overseas rugby tour with the Beijing Devils, he interrupted his own story with an explanatory aside directed at Mr Shi, accommodating for the fact that Mr Shi was relatively unacquainted with the sport: ``This sport, rugby union, it's actually very mysterious. If you haven't played it yourself you might not know this type of feeling,'' Adrian respectfully suggested to Mr Shi. ``Because rugby, you know, you're on the field playing together, there's body contact...'' he paused to find the right phrasing,  ``...its a very ``carnal'' type of feeling. Everyone is very close.'' His attempts to enrich his communication by gesticulating had led him to have both of his hands clenched as fists in front of him like they were cradling a rugby ball, a lit cigarette smouldering between the index and middle finger of his right hand.  Adrian concluded by reiterating: ``Its very mysterious,'' shaking his head as if baffled and releasing his clenched fists to dab the ash from his cigarette into the ashtray in front of him. He took another drag and  finally added,``So it means this rugby ``circle'' here in China is very tight...'' (Circle (quanzi) is a common way to refer to a social group or community of people) ``...but it doesn't mean that its not also a mess!'' The wisdom in this final note was confirmed with a knowing chuckle from all of us, including Mr Shi.

%英式橄榄球这个项目其实特别神秘,没玩过的话您可能不知道这种感觉,因为英式橄榄球么,大家在场上有身体接触,是一种``肉''的感觉,大家互相都特别亲,特别神秘。
%所以在橄榄球这个圈子特别亲, 但这不是说这个圈儿也不乱!

I was particularly struck by this snippet of Adrian's monologue, perhaps because it was so soon in to my fieldwork that I had happened upon such a rich description, in which the visceral sensation of playing rugby (rou) was linked conceptually to social closeness (qin) and the broader social cohesion of the rugby community (quanzi).  I did not fully realise it at the time, but Adrian's closing caveat would also be particularly relevant to the coming months of fieldwork with the Beijing provincial rugby team. I would come to experience first hand the complexity beneath Adrian's sarcastic assertion that the social closeness derived from the visceral experience of taking the field with teammates did not necessarily buffer against the (political) messiness of off-field social interactions between individuals within the imagined community of Chinese rugby players.  These interactions were, after all, structured by broader institutionalised incentives and cultural dispositions that existed well beyond and well before the existence of the rugby field in China.

Interestingly, however, I also came to understand the complexity of this statement from a different standpoint to Adrian.  When going looking for evidence of bonding and social closeness in rugby in the Beijing team, I was bombarded with testimonies and rationales for why social cohesion and closeness did not and could not exist.  (Chinese society is too complicated, athletes are not innately motivated to play rugby, etc).  Nonetheless, when I interrogated deeper in to the details of athlete experience of joint action in rugby, I was able to find evidence that the visceral dimensions of rugby were related to processes of social affiliation.

I left that first dinner motivated to investigate these relationships further.  My next stop was Xiannontan Sports Institute (XNT) on Monday morning, where I was scheduled to meet separately with the vice-principal of the school in charge of the rugby program, and the head coach of the Beijing rugby program, former CAU coach, ZPH.







%Hongwei Vignette

Many Chinese rugby players go through this difficult process of ``starting from scratch'' with the techniques of rugby, and

There was no ``flow'' in his actions, let alone any ``click'' with fellow teammates.  But more than just lack of fluidity of movement, Hongwei also appeared in my eyes lacked personality in his attempts to learn: there was no indication that he was bringing his own understanding to the actions, he remained rigid and mechanical.  After all, Hongwei had no prior understanding of what rugby and its repertoire ought to look or feel like, and was at the same time appeared desperate to ``fit in'' to this new social environment, presumably in order to access the benefits (prestigious tertiary eduction, career employment opportunities, etc) that membership to the Beijing rugby team could one day afford.
