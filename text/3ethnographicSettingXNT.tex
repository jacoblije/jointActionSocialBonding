First week in China (introduction vinnete):
- ``rou'': mysterious carnal sport --> social community
- XNT fall from grace:
- Qingdao:
    - basketball (actionScrutiny)
    - Asia 7s: (Qi gaige & Li sheng) (system, incentives)
    - National 7s: Beijing:WCY out of the huddle


XNT:
- Sport as a upward mobility strategy:
    - Many from athletics (IND) to rugby (TEAM)
- Fall from grace --> Lack of click?
- System: coach ()
- Zero to one: new comers, uni students, senior players, old motherlandStrength

In this system, how do they interaction

%%%%%%%%%%%%%%%%%%%%%%%%%%%%%%%%%%%%%%%%%%%%%%%%%%%%%%%%%%%%%%%%%%%%%%%%%%%%%%%%%%%%%%



First week in China (introduction vignette):

I arrived in Beijing late on a Friday evening at the end of August in 2015.  My close friend Kai---a former Chinese rugby player, graduate the Chinese Agricultural University (Chinese rugby's birthplace), and now a lawyer working in Beijing---met me at the airport and drove me back to his home.  When we got to his home, Kai turned on the television and we caught up while some footage from a rugby documentary played in the background.  As it so happened, the international rugby world was on the verge of another Rugby World Cup, which was being hosted by England in the coming months. World Rugby, the world governing body of rugby union, had made the television broadcast rights for the World Cup available to Chinese Central Television (CCTV), in an attempt to promote the game globally.  Having accepted the rights to the tournament, which is the 3rd largest sporting event in the world behind the Olympics and the Football World Cup, CCTV were in search of Chinese rugby experts to help produce the 48-match tournament.  As Kai quickly explained, CCTV's search had led them to the Chinese rugby community based in Beijing, who were almost all, like Kai, graduates of the Chinese Agricultural University---the birthplace of rugby in China and the base for the Chinese National Rugby team between 1996 and 2010.  In fact, CCTV's search led them first to Adrian, the captain of one of the first CAU rugby teams (1992), and currently working for a large international sport organisation in Beijing.  Adrian had then contacted his younger university brother (师弟) Kai, who like him was fluent in English and able to assist in sourcing and translating rugby materials relevant to the broadcast. The CCTV producer responsible for the broadcast, Mr Shi, had scheduled a dinner with Adrian and Kai on Saturday (tomorrow) night to thank them for their willingness to assist in the production.  I was also invited to the dinner. I wasn't scheduled to meet with the Principle and head coach of the Beijing Xiannongtan Sports Institute until the following Monday, so I agreed to accompany Kai.

It had been two years since I had last spent a long period of time in China, the last time being in 2013 when I spent eight months coaching the Chinese men's youth rugby 7s team in the lead up to the Nanjing Asian Youth Olympics.  Before that, I had spent one year studying sociology and social anthropology at Beijing University in 2008, and another year before that on an intensive Chinese language course at Liaoning University, Shenyang, in 2006.  Rugby featured heavily in both instances.  In 2006, an Australian classmate and friend Ed had caught wind of the fact that there was a rugby program down the road from Liaoning University at the Shenyang Sports College (SSC).  Despite diligently attending class and courageously using our elementary Chinese to order food at restaurants and befriend local taxi drivers, Ed and I were, nonetheless, three months into our intensive language exchange feeling that our Chinese skills were floundering, in large part due to the fact that we had met very few local Chinese friends.  So one afternoon we rode our bikes over to the Shenyang Sports College in time for the rugby team's afternoon training session.  Six months later, we were boarding a train to Shanghai with the team to compete in the annual Shanghai Rugby 7s Tournament.  We had become closely integrated into the community of rugby athletes at SSC, due in part to the common language of rugby that we all shared, and due to the overwhelming hospitality of the SSC rugby team.

Buoyed by this experience in 2006, I followed a similar template two years later when I arrived in Beijing on exchange from Sydney University to study sociology and social anthropology at Beijing University.  At that time, the only rugby program in Beijing was  based at the Chinese Agricultural University, a forty minute cycle from Beijing University. It was here during this time that I met and developed a strong friendship with Kai, who was at the time playing for CAU and China, while also finishing a Master's degree in Labour Law at CAU.  I also met and developed relationships with many of the Beijing rugby community, many of whom later dispersed to various professional provincial rugby programs set up in 2010 after the International Olympic Committee's announcement that Rugby Sevens would be included in the 2016 Rio De Janeiro Olympics.  For rugby in China, the immediate implication of rugby's Olympic inclusion was that the sport would become part of the state sport system, played by China's provinces at the quadrennial National Games, next held in 2013.

Between 2009 and 2013 I returned to Australia to finish my undergraduate degree, during which time my own rugby career also rapidly developed. After a successful season in the Sydney premiership competition in 2009, I was selected to play for the Australian Sevens team, from late 2009 through to late 2012.  In 2013, during the 9 month gap between my Australian rugby contract ending and the start of my graduate studies at Oxford University, I returned to China to coach the Chinese Youth Men's 7s program in their lead-up to the 2013 Nanjing Asian Youth Olympics. Along with a small team of Chinese coaches and management, I coached a group of roughly 25 athletes aged between 15 and 18 years old.  The program was temporarily based in Anhui province, and travelled to other provinces further afield to find suitable practice opportunities against provincial programs.

Soon after the completion of the Asian Youth Olympics in Nanjing, the Chinese National Games were held in Shenyang, and rugby appeared for the first time.



The dinner:
``ROU''

After spending Saturday doing odd jobs like organising a local Chinese mobile number, arrived


安哥’s explanation about the “肉” nature of rugby, and the bonds that are formed between rugby players who share the intimate, carnal, physical nature of rugby... Second, The reference to the “圈子“ - the imagination of membership to a group, a community, an identity.




XNT:
Fall from grace: ZPH, etc

Qingdao:
National Women's teams
Beijing Team.
